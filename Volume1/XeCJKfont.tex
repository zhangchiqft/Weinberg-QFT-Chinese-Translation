% !TeX root = Main.tex
% !TeX encoding = utf-8
% !TEX spellcheck = en-us
% !TeX program = xelatex


%%%%%%%%%% 中文字体与字号
\usepackage{fontspec,xunicode,xltxtra}
\usepackage[boldfont,slantfont,CJKchecksingle,
CJKglue = {\hskip .8pt plus 0.08\baselineskip}]{xeCJK}

\usepackage{xcolor}
\punctstyle{quanjiao}
\defaultfontfeatures{Mapping=tex-text,Ligatures = TeX}

\setCJKmainfont[BoldFont={FZHTK.TTF},ItalicFont={FZKTK.TTF}]{FZSSK.TTF}
\setCJKsansfont{FZXH1K.TTF}
\setCJKfamilyfont{fzsong}{FZSSK.TTF}
\setCJKfamilyfont{fzsongi}{FZSYK.TTF}
\setCJKfamilyfont{fzhei}{FZHTK.TTF}
\setCJKfamilyfont{fzkai}{FZKTK.TTF}
\setCJKfamilyfont{fzfs}{FZFSK.TTF}
\setCJKfamilyfont{fzli}{FZLSK.ttf}
\setCJKfamilyfont{fzyou}{FZY1K.TTF}
\setCJKfamilyfont{fzdbs}{FZDBSK.ttf}
\setCJKfamilyfont{fzxh1}{FZXH1K.TTF}
\setCJKfamilyfont{fzxbs}{FZXBSK.TTF}
\setCJKfamilyfont{fzzdx}{FZZDX.TTF}
\setCJKfamilyfont{fztimes}{cmunrm.otf}

\newcommand*{\SONG}{\CJKfamily{fzsong}}     % 宋体
\newcommand*{\SONGI}{\CJKfamily{fzsongi}}   % 宋一
\newcommand*{\HEI}{\CJKfamily{fzhei}}       % 黑体
\newcommand*{\KAI}{\CJKfamily{fzkai}}       % 楷书
\newcommand*{\FS}{\CJKfamily{fzfs}}         % 仿宋
\newcommand*{\LISHU}{\CJKfamily{fzli}}      % 隶书
\newcommand*{\YOU}{\CJKfamily{fzyou}}       % 细圆
\newcommand*{\DBS}{\CJKfamily{fzdbs}}       % 大标宋
\newcommand*{\XHI}{\CJKfamily{fzxh1}}       % 细黑
\newcommand*{\XBS}{\CJKfamily{fzxbs}}       % 小标宋
%\newcommand*{\ZDX}{\CJKfamily{fzzdx}}       % 中等线
\newcommand*{\FZTIMES}{\CJKfamily{fztimes}}

%%%%%%%%%%%%%%%%%%%%%%%%%%%%%%
\setCJKfamilyfont{caiyun}{STCaiyun}
\newcommand*{\caiyun}{\CJKfamily{caiyun}}  %华文彩云
\setCJKfamilyfont{hupo}{STHupo}
\newcommand*{\hupo}{\CJKfamily{hupo}}      %华文琥珀
\setCJKfamilyfont{you}{YouYuan}
\newcommand*{\you}{\CJKfamily{you}} %幼圆
\setCJKfamilyfont{ximing}{PMingLiU}
\newcommand*{\ximing}{\CJKfamily{ximing}}  %新細明
\setCJKfamilyfont{xinwei}{STXinwei}
\newcommand*{\xinwei}{\CJKfamily{xinwei}}  %华文新魏
\setCJKfamilyfont{xingkai}{STXingkai}
\newcommand*{\xingkai}{\CJKfamily{xingkai}}  %华文行楷
\setCJKfamilyfont{zhongsong}{STZhongsong}
\newcommand*{\zhongsong}{\CJKfamily{zhongsong}}  %华文中宋
\setCJKfamilyfont{song}{NSimSun}
\newcommand*{\song}{\CJKfamily{song}}  %宋体
\setCJKfamilyfont{hei}{SimHei}
\newcommand*{\hei}{\CJKfamily{hei}}  %黑体
\setCJKfamilyfont{li}{LiSu}
\newcommand*{\li}{\CJKfamily{li}}  %隶书
\setCJKfamilyfont{kai}{楷体}
\newcommand*{\kai}{\CJKfamily{kai}}  %楷体_GB2312
\setCJKfamilyfont{fs}{仿宋}
\newcommand*{\fs}{\CJKfamily{fs}}  %仿宋x_GB2312
\setCJKfamilyfont{liuti}{HAKUYOLiuTi3500}
\newcommand*{\liuti}{\CJKfamily{liuti}}  %柳体
\setCJKfamilyfont{outi}{HAKUYOOTi3500}
\newcommand*{\outi}{\CJKfamily{outi}}  %欧体
\setCJKfamilyfont{xingshu}{hakuyoxingshu7000}
\newcommand*{\xingshu}{\CJKfamily{outi}}  %博洋行书
%%%%%%%%%%%%%%%%%%%%%%%%%%%%%%%%%%%%%%%%%%%%%%%%
 \setCJKfamilyfont{zhuan}{FZzhuan}
 \newcommand*{\zhuan}{\CJKfamily{zhuan}}  %小篆
  \setCJKfamilyfont{jinglei}{FZjinglei}
 \newcommand*{\jinglei}{\CJKfamily{jinglei}}  %徐静蕾体
  \setCJKfamilyfont{yanti}{STFyanti}
 \newcommand*{\yanti}{\CJKfamily{yanti}}  %颜体
  \setCJKfamilyfont{tiejin}{FZtiejin}
 \newcommand*{\tiejin}{\CJKfamily{tiejin}}  %铁筋隶书
  \setCJKfamilyfont{shoujin}{FZshoujin}
 \newcommand*{\shoujin}{\CJKfamily{shoujin}}  %瘦金体
%%%%%%%%%%%%%%%%%%%%%%%%%%%%%%%%%%%%%%%%%%%%%%%%%
 \setCJKfamilyfont{HYzhuan}{HYZhuan}
 \newcommand*{\HYzhuan}{\CJKfamily{HYzhuan}}  %汉仪粗篆
 \setCJKfamilyfont{qigong}{QiGong}
 \newcommand*{\qigong}{\CJKfamily{qigong}}  %启功字体
 \setCJKfamilyfont{yueheixi}{YueHeiXi}
 \newcommand*{\yueheixi}{\CJKfamily{yueheixi}}  %悦黑细

%-------- 中文字号 (3个带*号的命名不统一)
\newcommand{\HZdatehao}{\fontsize{63bp}{\baselineskip}\selectfont}    %大特 = 五号的六倍
\newcommand{\HZzhongtehao}{\fontsize{56bp}{\baselineskip}\selectfont} %中特 = 四号的四倍
\newcommand{\HZtehao}{\fontsize{48bp}{\baselineskip}\selectfont}      %特号 = 小四的四倍*
\newcommand{\HZxiaotehao}{\fontsize{45bp}{\baselineskip}\selectfont}  %小特 = 小五的五倍*
\newcommand{\HZchuhao}{\fontsize{42bp}{\baselineskip}\selectfont}     %初号 = 五号的四倍
\newcommand{\HZxiaochuhao}{\fontsize{36bp}{\baselineskip}\selectfont} %小初 = 小五的四倍
\newcommand{\HZdayihao}{\fontsize{30bp}{\baselineskip}\selectfont}    %大一 = 七号的五倍*
\newcommand{\HZyihao}{\fontsize{28bp}{\baselineskip}\selectfont}      %一号 = 四号的二倍
\newcommand{\HZxiaoyihao}{\fontsize{24bp}{\baselineskip}\selectfont}  %小一 = 小四的二倍
\newcommand{\HZerhao}{\fontsize{21bp}{\baselineskip}\selectfont}      %二号 = 五号的二倍
\newcommand{\HZxiaoerhao}{\fontsize{18bp}{\baselineskip}\selectfont}  %小二 = 小五的二倍
\newcommand{\HZsanhao}{\fontsize{15.75bp}{\baselineskip}\selectfont}  %三号 = 六号的二倍
\newcommand{\HZsihao}{\fontsize{14bp}{\baselineskip}\selectfont}      %四号
\newcommand{\HZxiaosihao}{\fontsize{12bp}{\baselineskip}\selectfont}  %小四 = 七号的二倍
\newcommand{\HZwuhao}{\fontsize{10.5bp}{\baselineskip}\selectfont}    %五号
\newcommand{\HZzihao}{\fontsize{10bp}{\baselineskip}\selectfont}      %10磅
\newcommand{\HZxiaowuhao}{\fontsize{9bp}{\baselineskip}\selectfont}   %小五
\newcommand{\HZliuhao}{\fontsize{7.875bp}{\baselineskip}\selectfont}  %六号
\newcommand{\HZqihao}{\fontsize{6bp}{\baselineskip}\selectfont}       %七号



\XeTeXlinebreaklocale "zh"             %這兩行一定要加, 中文才能自動換行
\XeTeXlinebreakskip = 0pt plus 1pt     %這兩行一定要加, 中文才能自動換行
