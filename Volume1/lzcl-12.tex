\renewcommand{\theequation}{\arabic{chapter}.\arabic{section}.\arabic{equation}}   % 定义方程编号

\chapter{重正化的一般理论} \label{cha:12}
 \thispagestyle{empty} \marginpar[\flushright{\raisebox{17ex}[0pt]{{\small[499]\hspace*{5mm}}}}]{{\raisebox{17ex}[0pt]{\small\hspace*{5mm}[499]}}}
  \markboth{第12章\quad 重正化的一般理论}{第12章\quad 重正化的一般理论}

我们在上一章看到, 在含有单圈图的量子电动力学计算中, 出现了发散的动量空间发散积分,
但当我们将理论中的所有参量表示成``重正化''参量后, 例如实际测量的质量和电荷, 这些发散抵消了. 在\,1949\,年, Dyson\textsuperscript{\cite{1}}概述了一个证明, 表明这种抵消会在量子电动力学所有阶发生. 很快大家就明白了\,Dyson\,的讨论适用于更大的一类理论(我们将在\,\ref{sec:12.1}\,节和\,\ref{sec:12.2}\,节讲到), 这类理论具有有限个相对简单的相互作用, 即所谓的{\KAI{可重整}}理论, 而量子电动力学仅是这类理论的一个简单例子.

多年来, 大家普遍认为任何合理的物理理论都只能采用可重整量子场论的形式. 在发展弱作用、
电磁作用和强作用的现代``标准模型''中, 可重整的要求扮演了重要角色. 然而, 我们将在这里看到,
紫外发散的抵消并不真正依赖于可重整性; 只要我们将对称性所允许的无限多个相互作用一一包含在内,
所谓的不可重整理论实际上同可重整理论一样是可重整的.

现今, 大家普遍认为我们在当前可达到能区用来描述物理的真实理论是所谓的``有效场论''. 正如\,\ref{sec:12.3}\,节所要讨论的, 它们是某个更基础理论的低能近似, 而这个基础理论可能根本就不是场论. 任何有效场论必然包含无限个不可重整的相互作用. %
然而, 就像在\,\ref{sec:12.3}\,节和\,\ref{sec:12.4}\,节所讨论的, 在这种有效场论中, 我们预期所有的不可重整的相互作用在能量足够低时都被压低了. 因此, 尽管原因上与最初在这些理论中做出可重整假定的动机不同, 类似量子电动力学和标准模型这样的可重整理论保留了它们在物理中的特殊地位.

\newpage

\section{发散度} \label{sec:12.1}
\marginpar[\flushright
{\raisebox{5.5ex}[0pt]{{\small[500]\hspace*{5mm}}}}]{{\raisebox{5.5ex}[0pt]{\small\hspace*{5mm}[500]}}}

\setcounter{equation}{0}

我们来考察一类非常广泛的理论, 其中包含各种相互作用, 不同相互作用用$i$标记.
每种相互作用可以用每一类$f$型场的个数$n_{if}$, 以及作用在场上的导数个数$d_{i}$表征.

我们将从计算这种理论中任意一个连通单粒子不可约\,Feynman\,图的``表观发散度''$D$开始.
表观发散度$D$是被积函数分子中动量因子的数目减去分母中动量因子的数目,
对每个我们要做的独立$4$-动量空间积分还要再加上$4$. 表观发散是动量空间区域上的积分在所有内线动量都一起趋于无穷大时的实际发散度. 就是说, 如果$D>0$, 那么振幅中所有内动量以公共因子$\kappa$趋于无穷的部分, 其发散类似于
\begin{equation}
\int^{\infty} \kappa^{D-1}\:\dif \kappa \:.  \label{12.1.1}
\end{equation}%
在同样的意义上, 只要考察的动量空间区域没变, 发散度$D=0$的积分是对数发散的, 而$D<0$的积分是收敛的.
稍后我们将回到子积分的性质比在原区域积分更坏这个问题上.

为了计算$D$, 我们需要知道图的如下信息:%
\begin{align*}
I_{f} &\equiv f\,\text{类场内线的数目 ,} \\
E_{f} &\equiv f\,\text{类场外线的数目 ,} \\
N_{i} &\equiv i\,\text{类相互作用顶点的数目 .}
\end{align*}%
我们将$f$类场的传播子$\Delta _{f}(k)$的渐进行为写成如下形式
\begin{equation}
\Delta _{f}(k)\sim k^{-2+2s_{f}}\:.   \label{12.1.2}
\end{equation}%
回顾第6章, 我们看到, 对标量场, $s_{f}=0$, 对\,Dirac\,场, $s_{f}=\frac{1}{2}$, 对有质量矢量场, $s_{f}=1$. 更普遍地, 可以证明, 对$(A,B)$ 型\,Lorentz\,变换的有质量场, 我们有$s_{f}=A+B$. 不严格地说, %
我们可以称$s_{f}$ 为``自旋''. 然而, 扔掉那些由于规范不变性而对结果没有影响的项后, %
有效光子传播子$\eta_{\mu\nu}/k^{2}$有$s_{f}=0$. 对于有质量矢量场与守恒流的耦合, 只要这个流不依赖于该矢量场, 依然会有类似的结果. 也可以证明, 在相同意义下, 引力场$g_{\mu\nu}$的传播子也有$s_{f}=0$.

根据方程(\ref{12.1.2}), 传播子对$D$的总贡献等于\marginpar[\flushright
{\raisebox{-6ex}[0pt]{{\small[501]\hspace*{5mm}}}}]{{\raisebox{-6ex}[0pt]{\small\hspace*{5mm}[501]}}}
\begin{equation}
\sum_{f}I_{f}(2s_{f}-2)\:.   \label{12.1.3}
\end{equation}%
另外, 每个$i$类相互作用中的导数向积分中引入了$d_{i}$个动量因子, 对$D$的总贡献等于
\begin{equation}
\sum_{i}N_{i}\,d_{i}\:.   \label{12.1.4}
\end{equation}%
最后, 我们需要独立动量积分变量的总数. 每个内线可以用一个\,4\,-动量标记, 但是它们不全是独立的; 除了一个用以保证外动量守恒的$\updelta$-函数之外, 每个顶点所附带的$\updelta$-函数在这些内动量之间强加了一个线性关系. 因此, 动量空间积分体积元对$D$贡献了一项
\begin{equation}
4\left[ \sum_{f}I_{f}-\left( \sum_{i}N_{i}-1\right) \right] \:, \label{12.1.5}
\end{equation}%
显然, 这正是$4$乘上图中相互独立圈的数目. 加上贡献(\ref{12.1.3}), (\ref{12.1.4})和(\ref{12.1.5}), 我们发现
\begin{equation}
D=\sum_{f}I_{f}(2s_{f}+2)+\sum_{i}N_{i}(d_{i}-4)+4\:.   \label{12.1.6}
\end{equation}

方程(\ref{12.1.6})现在的形式并不方便, 因为它给出的$D$值似乎依赖于Feynman图的内部细节.
幸运地是, 可以利用拓扑恒等式
\begin{equation}
2I_{f}+E_{f}=\sum_{i}N_{i}\,n_{if}   \label{12.1.7}
\end{equation}%
简化它. (每个内线贡献两条与顶点相连的线, 而每个外线只贡献一条.) 利用方程(\ref{12.1.7})消掉$I_{f}$, 我们看到方程(\ref{12.1.6})变成%
\begin{equation}
D=4-\sum_{f}E_{f}(s_{f}+1)-\sum_{i}N_{i}\Delta_{i}\:,  \label{12.1.8}
\end{equation}%
其中$\Delta _{i}$是用来表征$i$类相互作用的一个参量:%
\begin{equation}
\Delta_{i}\equiv 4-d_{i}-\sum_{f}n_{if}(s_{f}+1)\:.   \label{12.1.9}
\end{equation}

无需考虑\,Feynman\,图的结构, 这个结果也可以通过简单的量纲分析得到. 场的传播子是一对自由场编时乘积的真空期望值的
四\marginpar[\flushright{\small[502]\hspace*{5mm}}]{{\small\hspace*{5mm}[502]}}维\,Fourier\,变换, 所以一个按照惯例归一化的场, 若它的量纲{}$^*$\footnote{$^*${}本章中, 在$\hbar =c=1$的单位值下, ``量纲\textquotedblright
总是指质量或动量幂次的量纲. 我们使用的场是按惯例归一化的,
也就是说自由场拉格朗日量中导数数目最多的那一项(其决定了传播子的渐进行为)的系数是无量纲的.}用动量幂次表示是$\mathscr{D}_{f}$, 那么这个场就有一个量纲为$-4+2\mathscr{D}_{f}$的传播子. 因此, 如果传播子在$k$远大于质量时的行为类似于$k^{-2+2s_{f}}$, 那么这个场的量纲必须是$-4+2\mathscr{D}_{f}=-2+2s_{f}$, 或者$\mathscr{D}_{f}=1+s_{f}$. 对于一个$i$类相互作用, 若它具有$n_{if}$个这样的场和$d_{i}$个导数, 那么它的量纲是$d_{i}+\sum_{f}n_{if}(1+s_{f})$. 但是%
作用量必须是无量纲的, 因此为了抵消$\dif^{4}x$的量纲$-4$, 拉格朗日密度中的每一项量纲必须是$+4$. 因此, 相互作用必须有一个量纲为$4-d_{i}-\sum_{f}n_{if}(1+s_{f})$的耦合常数, 这正好就是参数$\Delta _{i}$.
对于有$E_{f}$条$f$类外线的连通\,Feynman\,图, 它对应的动量空间振幅是总量纲为$\sum_{f}E_{f}(1+s_{f})$的场的编时%
乘积的真空期望值对$4\sum_{f}E_{f}$个坐标的\,Fourier\,变换,
所以它的量纲为$%
\,\sum_{f}E_{f}(-3+s_{f})$. 在这个量纲中, 有$-4$个来源于动量空间$\updelta$-函数, 而$\sum_{f}E_{f}(-2+2s_{f})$是外线传播子的量纲, 所以动量空间积分本身再加上所有耦合常数因子后的量纲是
\[
\sum_{f}E_{f}(-3+s_{f})-(-4)-\sum_{f}E_{f}(-2+2s_{f})=4-\sum_{f}E_{f}(s_{f}+1)\:.
\]%
一个给定\,Feynman\,图的耦合常数的总量纲是$\sum_{i}N_{i}\Delta_{i}$, 剩下的是量纲为$4-\sum_{f}E_{f}(s_{f}+1)-\sum_{i}N_{i}\Delta _{i}$的动量空间积分.
只要我们关心的积分区域中所有动量一起趋于无穷, 动量空间积分的发散度就是它的量纲, 这样就证明了方程(\ref{12.1.8}).

如果所有相互作用都有$\Delta _{i}\geq 0$, 那么方程(\ref{12.1.8})就给出了$D$的上界, 这个上界只依赖每类外线的数目,即对一个要计算振幅的物理过程
\begin{equation}
D\leq 4-\sum_{f}E_{f}(s_{f}+1)\:.   \label{12.1.10}
\end{equation}%
例如, 在上一章所研究的量子电动力学的简单版本中, 拉格朗日量中包含的项的类型如表12.1所示. 这里所有的相互作用都有$\Delta _{i}\geq 0$, 因而有$E_{\gamma }$个外光子线和$E_{e}$个外\,Dirac\,线的\,Feynman\,图的表观发散度满足方程(\ref{12.1.10})的限制:\marginpar[\flushright
{\raisebox{-5ex}[0pt]{{\small[503]\hspace*{5mm}}}}]{{\raisebox{-5ex}[0pt]{\small\hspace*{5mm}[503]}}}
\begin{equation}
D\leq 4-\frac{3}{2}E_{e}-E_{\gamma }\:.   \label{12.1.11}
\end{equation}%
仅有有限多组外线可以产生表观发散的积分; 我们将在\,\ref{sec:12.2}\,节将列举它们. 我们将要证明, %
对于所有相互作用的$\Delta _{i}\geq 0$的理论, 其中出现的有限多个发散通过对有限数目的物理常数的重新定义以及场的重正化被自动地消除了. 出于这个原因, 这种理论被称为{\KAI{可重整的}}. 在\,\ref{sec:12.3}\,节我们将罗列所有可重整理论, %
并讨论可重整性作为物理理论判据的意义.


``可重整''这个词也可用于单个相互作用. 可重整的相互作用是那些$\Delta _{i}\geq 0$的相互作用, 它们的耦合常数的量纲为正或为零. 有时会区分$\Delta _{i}=0$ 的相互作用和$\Delta _{i}>0$的相互作用, 前者称为可重整的, 后者称为{\KAI{超可重整的}}. 由于增加额外的场或导数总会降低$\Delta _{i}$, 包含任意给定类型场的可重整相互作用只能有有限个. 我们已经看到在量子电动力学的最简版本中, 所有相互作用都是可重整的, 而$\bar{\psi}\psi$项是超重整的.

\vspace{0.4cm}

\begin{small}

\noindent
{\bf 表~12.1}\quad 下表是量子电动力学拉格朗日密度中的项. 这里的$d_{i}$, $n_{i\gamma}$和$n_{ie}$分别是相互作用中导数, 光子场和电子场的个数, 而$\Delta_{i}$ 是相应系数的量纲. (回忆起$s_{\gamma}=0$, $s_{e}=\frac{1}{2}$.)
\end{small}
 \vspace{-0.4cm}

\begin{center}\begin{footnotesize}
\def\temptablewidth{\textwidth}
{\footnotesize {\rule{\temptablewidth}{1pt}}\\[-0.5mm]
\renewcommand{\arraystretch}{1.2}
\tabcolsep=18pt\begin{tabular}{ccccc}
  相互作用 & $d_{i}$ & $n_{i\gamma}$ & $n_{ie}$ &  \multicolumn{1}{c}{$\Delta_{i}$} \\  \hline
  $-\mi e\bar{\psi}\xxA\psi$ & 0 & 1 & 2 & $4-1-3=0$  \\
  $-\frac{1}{4}(Z_{3}-1)F_{\mu\nu}F^{\mu\nu}$ & 2 & 2 & 0 & $4-2-2=0$ \\
  $-(Z_{2}-1)\bar{\psi}\xxdd\psi$ & 1 & 0 & 2 & $4-1-3=0$ \\
  $[-(Z_{2}-1)m+Z_{2}\delta m]\bar{\psi}\psi$& 0 & 0 & 2 &   $4-3=1$ \\
\end{tabular}\\[-0.3mm]
\def\temptablewidth{\textwidth}{\rule{\temptablewidth}{1pt}}
}\vspace{-0.3cm}
\end{footnotesize}
\end{center}

另一方面, 如果相互作用的$\Delta _{i}<0$, 那么这样的顶点越多, 发散度(\ref{12.1.8})就越大. 不管我们把各种$E_{f}$取得多大, 最终, 足够多的$\Delta _{i}<0$ 的$i$类顶点都可以使方程(\ref{12.1.8})将变成正的(或零), 而积分会发散. 这种耦合常数量纲为负的相互作用被称为{\KAI{不可重整的}};{}$^*$\footnote{$^*${}在微扰统计力学中, 不可重整相互作用称为{\KAI{不相关的}},
因为它们在低能极限下变得不重要. 而可重整和超重整相互作用被分别称为{\KAI{临界的}}和{\KAI{相关的}}.} 含有任何不可重整相互作用的理论也被称为不可重整的. 但这并不意味着这样的理论就是没有希望的; 我\marginpar[\flushright{\small[504]\hspace*{5mm}}]{{\small\hspace*{5mm}[504]}}们将看到这些发散也可以被吸收进理论参量的重新定义中, 但这时我们需要无穷多个耦合.

需要铭记于心的是, 我们在这里计算的\,Feynman\,图发散度仅来源于所有内$4$-动量一起趋于无穷大的动量空间区域. 如果区域中仅属于某些子图的线的动量也趋于无穷大, 同样能产生发散. 例如, 在量子电动力学中, 对\,Compton\,散射(其中$E_{e}=2$, $E_{\gamma }=2$), 方程(\ref{12.1.11})给出$D\leq -1$, 并且事实上像图12.1(a)这样的图是收敛的, 但是像图12.1(b)或12.1(c)这样的图是对数发散的, 这是因为这些图包含$D\geq 0$ 的子图(由虚线框标出). 我们可以认为这些图的发散来自于两个独立内\,4\,-动量的八个分量在一个特定的\,4\,-维子空间上, 即, 唯一真正趋于无穷的\,4\,-动量是插入在内线或电子\lzx 光子顶点上的圈中环流的动量的子空间, 趋于无穷时造成的糟糕的反常渐进行为.

\begin{figure}[h!]
\centering
\includegraphics{1201.eps}\\
  \caption{一些\,Compton\,散射的两圈图. 这里直线是电子; 波浪线是光子. 图\,(a)\,的动量空间积分是收敛的, 而由于与虚线框中的子图相关的子积分是发散的, \,(b)\,和\,(c)\,的积分则是发散的.}
 \label{fig:12.1}
\end{figure}

已经被证明了\marginpar[\flushright{\small[505]\hspace*{5mm}}]{{\small\hspace*{5mm}[505]}},\textsuperscript{\cite{2}} 对应任意图的振幅, 使其真正收敛的要求是幂次计数不仅对整个积分的完整多重积分给出$D<0$, 对于通过保持任何一个或多个圈动量的线性组合不变所定义的任意子积分, 也要保证$D<0$. (图12.1(b)和12.1(c) 所示的图没有通过这个检验, 因为仅对那些虚线框中的圈的动量积分的子积分, $D\geq 0$.) 因为在较早的书\textsuperscript{\cite{3}}中有很好的处理, 并且这个证明方法无论在何种情况下都与我们的实际计算没多大关系, 我们不会在这里重复这个相当长的证明. 下一节将阐述这一要求是如何被满足的.

\section{发散的消除}  \label{sec:12.2}
\setcounter{equation}{0}

考虑一个\,Feynman\,图或\,Feynman\,图的一个部分, 它的表观发散度为正, 即$D\geq 0$. 那么, 所有内动量一起趋于无穷的那部分动量空间积分将会像$\int^{\infty }k^{D-1}\,\dif k$那样发散. 如果我们对任意外动量微分$D+1$次, 那么被积函数中的净动量因子的数目就会减少$D+1$个,{}$^*$\footnote{$^*${}例如, 如果一个内标量场线携带动量$k+p$, 其中$p$是外\,4\,-动量的线性组合而$k$是要积分的\,4\,-动量变量, 那么传播子$[(k+p)^{2}+m^{2}]^{-1}$对$p^{\mu }$ 的导数给出$-2(k_{\mu}+p_{\mu})[(k+p)^{2}+m^{2}]^{-2}$, 它在$k\to \infty$时趋于$k^{-3}$而不是$k^{-2}$.} 从而使这部分动量空间积分收敛. 这样做之后仍然可能存在源于子图的发散, 例如图12.1(b)和图12.1(c)中的子图; 我们暂且忽略这种可能性, 在本节后面再回到这种情况. 因为微分$D+1$次使积分有限, 因此这样的图或子图的贡献可以写成外动量的$D$次多项式, 多项式的系数发散但有一个有限的余项.

为了在没有无关复杂性的干扰下看到这一处理如何起作用, 我们考虑一个对数发散的一维积分
\[
\mathscr{I}(q)\equiv \int_{0}^{\infty}\frac{\dif k}{k+q}
\]%
其中$D=1-1=0$. 微分一次给出
\[
\mathscr{I}^{\prime }(q)\equiv -\int_{0}^{\infty }\frac{\dif k}{(k+q)^{2}}=-\frac{1}{q}\:,
\]%
所以
\[
\mathscr{I}(q)=-\ln q+c\:.
\]%
常数$c$显然是发散的\marginpar[\flushright{\small[506]\hspace*{5mm}}]{{\small\hspace*{5mm}[506]}}, 但是积分的其他部分完全是有限的. 以精确相同的方式, 我们可以算出$D=1$的积分
\[
\int_{0}^{\infty }\frac{k\,\dif k}{k+q}=a+bq+q\ln q
\]%
其中常数$a$和$b$是发散的.%

现在, 外动量的多项式项正是通过在拉格朗日量中添加合适的项来产生的: 如果一个图有$E_{f}$个$f$类外线, 发散度为$D\geq 0$, 那么它的紫外发散的多项式与通过添加各种具有$n_{if}=E_{f}$个$f$类场以及$d_{i}\leq D$个导数的$i$种相互作用的结果相同. 如果在拉格朗日量中已经有了这样的相互作用, 那么紫外发散就是给这些相互作用的耦合常数加上修正. 因此, 通过在这些耦合常数中引入合适的无限大项就可以抵消掉这些无穷大. 所有我们曾经测量到的耦合常数都是裸耦合常数与其中一个发散多项式相对应系数的和, 所以, 如果我们要求这个和等于(可能有限的)测量值, 那么裸耦合常数必须包含一个无穷大以抵消内动量的发散积分带来的无穷大. (一个条件: 当发散发生在只有两条外线的图或子图中, 它表现为对粒子传播子的辐射修正时, 我们不是要求某些有效耦合常数等于测量值, 而是要求全传播子与自由传播子在相同位置有留数相同的极点.) 以这种方式, 所有的无限大都被吸收进耦合常数, 质量以及场的重新定义中.

为了使这个重正化程序起作用, 最根本的是, 拉格朗日量中要包含{\KAI{所有}}与\,Feynman\,振幅紫外发散部分对应的相互作用. 当然, 拉格朗日量中的相互作用会被各种对称性原理限制, 例如\,Lorentz\,不变性、 规范不变性等等, 但是这也同样限制了紫外发散. (证明非阿贝尔规范对称性以限制相互作用的方式限制了无限大要费些功夫, 这个证明将在卷\,\textrm{II}\,中给出.) 在一般情况下, 紫外发散上没有其他限制, {\KAI{所以拉格朗日量必须包含所有符合对称性原理要求的项.}} (在超对称理论中, 这个规则存在例外.\textsuperscript{\cite{4}})

然而, 存在一类重要的理论, 它们只有有限多个相互作用, 但重正化程序依旧有效. 这就是所谓的可重整理论, 它们的相互作用都有$\Delta_{i}\geq 0$. 这样, 方程(\ref{12.1.8})就给出\marginpar[\flushright
{\raisebox{-4ex}[0pt]{{\small[507]\hspace*{5mm}}}}]{{\raisebox{-4ex}[0pt]{\small\hspace*{5mm}[507]}}}
\[
D\leq 4-\sum_{f}E_{f}(s_{f}+1)\:,
\]%
所以发散多项式只出现在有限多个\,Feynman\,图或子图中: 足够少的外线使得$D\geq 0$. 这种发散多项式与由以下方式产生的图相同: 将发散图或发散子图替换成一个单个顶点, 该顶点来自于拉格朗日量中具有$E_{f}$个$f$类场和$0,1,\cdots D$个导数的项. 但是, 与方程(\ref{12.1.9})相比, 我们看到{\KAI{它们与满足可重整性要求$\Delta_{i}\geq0$的相互作用是精确相同的}}, 或者换句话说,%
\[
0\leq d_{i}\leq 4-\sum_{f}n_{if}(s_{f}+1)\:.
\]

为了使所有的无限大在可重整理论中抵消掉, 通常需要对称性允许的{\KAI{所有}}相互作用都必须真正地出现在拉格朗日量中{}$^*$\footnote{$^*${}另外, 整体对称性不允许的相互作用和质量项, 只要它们是超重整的, 即$\Delta _{i}>0$, 就可以出现在拉格朗日量中. 这是因为超重整耦合的出现降低了发散度, 使得对称性破缺不影响那些被$\Delta _{i}=0$的严格可重整耦合抵消掉的发散. 要注意的是, 体现对称性的是{\KAI{裸的}}严格可重整耦合; 以质壳矩阵元定义的重正化耦合一般体现的是对称性破缺的影响.}. 例如, 如果有相互作用为$\bar{\psi}%
\psi \phi$ (或$\bar{\psi}\gamma _{5}\psi\phi$)的标量(或赝标量)场$\phi$和费米场$\psi$, 那么我们无法排除相互作用$\phi ^{4}$; 否则就没有抵消项来抵消与四个标量或赝标量线相连的费米圈所产生的对数发散.%

我们在量子电动力学的最简版本中来更细致地看一下无限大的抵消是如何运作的. 方程(\ref{12.1.11})表明只有如下的图或子图才有可能产%
生发散积分:

\noindent {\boldmath $\bE_{\be}=2,$\quad$\bE_{\bm{\gamma} }=1$}

\noindent 这是电子\lzx 光子顶点$\Gamma_{\mu}^{(\ell)}(p^{\prime},p)$. (上标$\ell$表明这里仅包含圈图的贡献.)
这种情况下$D=0$, 所以它的发散部分是不依赖动量的. 这样\,Lorentz\,不变性仅允许这个发散常数正比于$\gamma_{\mu}$, 所以
\begin{equation}
\Gamma_{\mu}^{(\ell)}=L\gamma_{\mu}+\Gamma_{\mu}^{(f)}  \label{12.2.1}
\end{equation}%
其中$L$是个对数发散的常数, 而$\Gamma _{\mu }^{(f)}$是有限的. 由于我们总可以把$\Gamma_{\mu}^{(f)}$中的有限项$\delta L\,\gamma_{\mu}$移到$L\gamma_{\mu }$中, 所以这并没有唯一地定义常数$L$.
为\marginpar[\flushright{\small[508]\hspace*{5mm}}]{{\small\hspace*{5mm}[508]}}了完成这个定义, 我们注意到, 就像\,\ref{sec:10.4}\,节中所证明的那样, $\Gamma_{\mu }(p,p)$, 继而$\Gamma _{\mu }^{(f)}(p,p)$, 在质壳\,Dirac\,旋量之间的在壳矩阵元正比于$\gamma^{\mu}$的同一个矩阵元, 所以我们可以通过对$p^{2}+m_{e}^{2}=0$规定
\begin{equation}
\bar{u}(\bp,\sigma ^{\prime })\Gamma _{\mu }^{(f)}(p,p)u(\bp,\sigma )=0  \label{12.2.2}
\end{equation}%
来定义$L$.

\newpage

\noindent{\boldmath $\bE_{\be}=2,$\quad$\bE_{\bm{\gamma}}=0$}

\noindent 这是电子自能插入$\Sigma^{\ast}(p)$. 这种情况下$D=1$, 所以它的发散部分对入费米子和出费米子携带的动量$p^{\mu }$是线性的. Lorentz\,不变性(包括宇称守恒)只允许它是$\xxp$的函数, 所以我们可以将圈贡献写成
\begin{equation}
\Sigma ^{(\ell )}(p)=A-(\mi\xxp+m)B+\Sigma ^{(f)}(\xxp)\:, \label{12.2.3}
\end{equation}%
其中$A$和$B$是发散常数, 而$\Sigma^{(f)}$是有限的. 同样地, 这并不能唯一地定义常数$A$和$B$, 因为我们总可以使$\Sigma ^{(f)}$偏移一个有限的$\xxp$ 的一阶多项式. 我们将通过规定
\begin{equation}
\Sigma^{(f)}=\frac{\partial \Sigma ^{(f)}}{\partial \xxp}=0\qquad \text{对}\:\mi\xxp=-m  \label{12.2.4}
\end{equation}%
来定义$A$和$B$. 实际上, $B$不是一个新的发散常数. 只要我们采用一个遵守流守恒的正规化方案, $\Gamma_{\mu}$和$\Sigma$就将通过\,Ward\,恒等式(\ref{10.4.27})相联系
\[
\Gamma^{\mu}(p,p) = \gamma^{\mu}+\mi\frac{\partial}{\partial p_{\mu}}\Sigma (p)
\]%
因而
\begin{equation}
L\gamma_{\mu}+\Gamma_{\mu}^{(f)}(p,p)=B\gamma_{\mu}+\mi\frac{\partial\Sigma^{(f)}(p)}{\partial p^{\mu}}\:. \label{12.2.5}
\end{equation}%
取该方程在$\bar{u}(\bp,\sigma^{\prime})$与$u(\bp,\sigma )$之间的矩阵元, 并利用方程(\ref{12.2.2})和(\ref{12.2.4}), 我们发现
\begin{equation}
L=B\:.   \label{12.2.6}
\end{equation}%

\noindent {\boldmath $\bE_{\bm{\gamma}}=2,$\quad$\bE_{\be}=0$}

\noindent 这是光子自能插入$\Pi_{\mu\nu}^{\ast}(q)$. 这种情况下$D=2$, 所以它的发散部分是$q$的二次多项式. Lorentz\, 不变性仅允许$\Pi _{\mu \nu }^{\ast }$的形式是$\eta_{\mu\nu}$和$q_{\mu}q_{\nu}$的线性组合, 而系数只能依赖于$q^{2}$, 所以圈贡献为如下形式\marginpar[\flushright
{\raisebox{-5ex}[0pt]{{\small[509]\hspace*{5mm}}}}]{{\raisebox{-5ex}[0pt]{\small\hspace*{5mm}[509]}}}
\[
\Pi_{\mu\nu}^{(\ell)}(q)=C_{1}\eta_{\mu\nu}+C_{2}\eta_{\mu\nu}q^{2}+C_{3}q_{\mu}q_{\nu}+\text{有限项 ,}
\]%
其中$C_{1},C_{2}$和$C_{3}$是发散常数. 只要采用遵守流守恒的正规化方案, 我们必然有
\[
q^{\mu}\Pi_{\mu\nu}^{(\ell)}(q)=0\:.
\]%
于是, 对发散项这些同样要成立, 所以$C_{1}q_{\nu}+(C_{2}+C_{3})q^{2}q_{\nu}$对所有的$q$都必须是有限的. 由此得出$C_{1}$和$C_{2}+C_{3}$都必须是有限的, 因而可以被纳入到$\Pi_{\mu\nu}^{(\ell)}(q)$的有限部分中. 因此
\begin{equation}
\Pi_{\mu\nu}^{(\ell)}(q)=(\eta_{\mu\nu}q^{2}-q_{\mu}q_{\nu})\left(C+\pi(q^{2})\right) \:,   \label{12.2.7}
\end{equation}
其中$\pi(q^{2})$是有限的而$C$是$\Pi_{\mu\nu}^{(\ell)}$中仅剩的发散. 为了确定$C$的定义, 我们可以将任意的有限常数$\pi(0)$移入$C$, 使得
\begin{equation}
\pi (0)=0\:.   \label{12.2.8}
\end{equation}%

\noindent {\boldmath $\bE_{\bm{\gamma}}=4,$\quad$\bE_{\be}=0$}

这是光被光散射的振幅$M_{\mu\nu\rho\sigma}$. 这种情况下$D=0$, 所以, 利用\,Lorentz\,不变性和\,Bose\,统计,
它可以写成(这里不存在非圈图贡献)%
\[
M_{\mu\nu\rho\sigma}=K(\eta_{\mu\nu}\eta_{\rho\sigma}+\eta_{\mu\rho}\eta_{\nu\sigma}
+\eta_{\mu\sigma}\eta_{\nu\rho})+\text{有限项}
\]%
其中$K$是潜在的发散常数. 然而, 流守恒给出
\[
q^{\mu}M_{\mu\nu\rho\sigma}=0
\]%
因而$K(q_{\nu}\eta_{\rho\sigma}+q_{\rho}\eta_{\nu\sigma}+q_{\sigma}\eta_{\nu\rho})$是有限的. 为了使其对$q\neq 0$依然成立, $K$本身必须是有限的. 这是对称性原理在重正化手序中所扮演角色的一个很好的例子; 如果$K$被证明是无限的, 那么无法通过相互作用$(A_{\mu}A^{\mu})^{2}$的耦合常数的重正化来移除它, 因为规范不变性根本就不允许这样的相互作用,
但是由于规范不变性所附加的流守恒条件, $K$是有限的.

\noindent {\boldmath $\bE_{\bm{\gamma}}=1,$\quad$\bE_{\be}=0$ \textbf{和} $\bE_{\be}=1$, \quad$\bE_{\bm{\gamma}}=0,1,2$}

\noindent 这时分别有$D=3$和$D=\frac{5}{2},\frac{3}{%
2}$和$\frac{1}{2}$, 但\,Lorentz\,不变性使所有这样的图为零.

\noindent {\boldmath $\bE_{\bm{\gamma}}=3,$\quad$\bE_{\be}=0$}

\noindent 这种情况下$D=1$, 但由于电荷共轭不变性为零.

读者也许已经注意到了\marginpar[\flushright{\small[510]\hspace*{5mm}}]{{\small\hspace*{5mm}[510]}}, 独立的发散常数$A,B,C$与量子电动力学拉格朗日量的抵消项部分(\ref{11.1.9})中的独立参量$Z_{2},Z_{3}$以及$\delta m$是一一对应的. 这些抵消项对$\Sigma^{\ast }(p)$有一个直接的贡献$Z_{2}\delta m-(Z_{2}-1)(\mi\xxp+m)$. 单粒子极点的位置和留数要与自由场传播子中的相同, 这个要求意味着我们必须选择$Z_{2}$和$\delta m$使总的$\Sigma ^{\ast }(p)$满足方程(\ref{12.2.4}), 即
\begin{align}
Z_{2}\,\delta m &=-A\:,   \label{12.2.9} \\
Z_{2}-1 &=-B\:,   \label{12.2.10}
\end{align}%
从而, 完全电子自能插入函数只是有限函数$\Sigma^{(f)}(p)$:%
\begin{equation}
\Sigma (p)=\Sigma ^{(f)}(p)\:.   \label{12.2.11}
\end{equation}%
另外, $\mathscr{L}_{2}$对$\Gamma_{\mu}$的直接贡献等于$(Z_{2}-1)\gamma_{\mu}$. 利用方程(\ref{12.2.6}), 我们看到全顶点是
\begin{equation}
\Gamma_{\mu}=\gamma_{\mu}+(Z_{2}-1)\gamma_{\mu}+\Gamma_{\mu}^{(\ell)}=\gamma_{\mu}+\Gamma_{\mu}^{(f)}\:.  \label{12.2.12}
\end{equation}%
这不仅是有限的, 而且满足条件
\begin{equation}
\bar{u}(\bp,\sigma^{\prime})\Gamma_{\mu}(p,p)u(\bp,\sigma)
=\bar{u}(\bp,\sigma^{\prime})\gamma_{\mu}u(\bp,\sigma) \:,   \label{12.2.13}
\end{equation}%
这也可以从方程(\ref{10.6.13})和(\ref{10.6.14})中看到. 最后, $\mathscr{L}_{2}$对$\Pi_{\mu \nu}^{\ast}(q)$的贡献是$-(Z_{3}-1)(q^{2}\eta_{\mu\nu}-q_{\mu}q_{\nu})$. 为了使光子传播子有一个留数与自由场相同的极点, 我们需要$q^{2}\eta_{\mu\nu}-q_{\mu}q_{\nu}$在总$\Pi_{\mu\nu}(q)$中的系数为零, 所以
\begin{equation}
Z_{3}=1+C  \label{12.2.14}
\end{equation}%
这样光子传播子就是有限的:%
\begin{equation}
\Pi_{\mu \nu}(q)=(\eta_{\mu \nu}q^{2}-q_{\mu}q_{\nu})\pi (q^{2})\:.  \label{12.2.15}
\end{equation}%

目前, 我们检验的发散来自于动量空间中所有内动量都很大(并且比例为一般值)的区域, 它们是外动量的多项式, 并可以被合适的抵消项抵消掉. 这样的图被称为{\KAI{表观收敛的}}. 在我们得出所有的紫外发散实际上都可以被重正化消除的结论之前, 我们还需要考察另一类发散, 它们出现在高阶图中, 来自动量空间积分变量的一些子集而非全体趋于无穷大. 例如, 在量子电动力学中,
子积分中的表观发散要么来源于光子自能部分$\Pi^{\ast}$, 要么来源于电子自能部分$\Sigma^{\ast}$, %
要么来源于电子\lzx 电子\lzx 光子顶点$\Gamma^{\mu}$. 这\marginpar[\flushright{\small[511]\hspace*{5mm}}]{{\small\hspace*{5mm}[511]}}些发散的问题在于它们无法通过对外动量微分去掉; 留给我们的项, 导数只能作用在{\KAI{不}}在发散子图中的内线上, 因而没有降低那些子图的发散度. 正如上一节所提到的, 对于一个图或几个图的和, 仅当它的以及它所有的子积分在动量幂次计数的意义下是表观收敛时, 它才是真正收敛的. 然而, 只要这样的发散子图出现, 一个无限大的抵消项都会伴随它出现. 在电动力学中, 它们是方程(\ref{11.1.9})中的项: 对每一个$\Pi_{\mu \nu}^{\ast}(q)$是$-(Z_{3}-1)(q^{2}\eta_{\mu\nu}-q_{\mu}q_{\nu})$, 对每一个$\Sigma^{\ast}(p)$是$Z_{2}\delta m-(Z_{2}-1)(\mi\xxp+m)$, 而对每一个$\Gamma^{\mu}$是$(Z_{2}-1)\gamma^{\mu}$. 把图作为一个整体, 这些抵消项抵消了来自发散子图的无限大.\textsuperscript{\cite{1}}%

不幸的是, 这个简单的论证中存在疏漏\ezx 没有考虑到交缠(overlap)发散的可能性. 就是说,
两个发散子图有可能共有一条内线, 使得我们无法将它们视为独立的发散积分. 在量子电动力学中,
仅当两个电子\lzx 电子\lzx 光子顶点在光子或电子自能插入图{}$^*$\footnote{$^*${}两个自能插入图或者一个自能插入图与一个顶点共享一条线会使得外线的数目不足以将这样的子图与图的剩余部分相连. 历史上, 为了绕过电子自能中的交缠发散问题, 曾用\,Ward\,恒等式(\ref{10.4.26})将电子自能表示成不会产生交缠发散的顶点函数. 因为这个方法不是必要的, 并且在任何情况下都不能解决光子或其他中性粒子的自能问题, 所以我们不在这里叙述这个方法.}中交缠时, 如图12.2和12.3所示, 才会遇到这样的发散.

\begin{figure}[h!]
\centering
\includegraphics{1202.eps}\\
  \caption{量子电动力学中一些包含交缠发散的\,4\,阶光子自能图. 带箭头的线是电子; 波浪线是光子. 十字代表抵消项的贡献.}
  \label{fig:12.2}
\end{figure}

\begin{figure}[h!]
\centering
\includegraphics{1203.eps}\\
  \caption{量子电动力学中一些包含交缠发散的\,4\,阶电子自能图. 波浪线是光子; 其他线是电子. 十字代表抵消项的贡献.}
  \label{fig:12.3}
\end{figure}


将交缠发散考虑\marginpar[\flushright{\small[512]\hspace*{5mm}}]{{\small\hspace*{5mm}[512]}}在内的完整重正化方案应该不仅要消除整体积分中的表观紫外发散, 也要消除所有子积分中的表观发散, 并且要证明这个处理可以被质量, 场以及耦合常数的重正化(至少形式上地)实现. 这样, 参考文献\,[2]\,的定理就确保了, %
对于所有重正化场的\,Green\,函数, 当它以重正化质量和耦合表示时是有限的. 场, 质量以及耦合的重正化使整体积分以及所有的子积分表观收敛的第一个证明由\,Salam(萨拉姆)给出.\textsuperscript{\cite{5}} 一个更具体的消除紫外发散的%
处理由\,Bogoliubov(玻戈留玻夫)和\,Parasiuk(泊拉奇克)\textsuperscript{\cite{6}}给出, 并由\,Hepp(海普)修正,\textsuperscript{\cite{7}} 而且他们证明了这等价于场, 质量以及耦合常数的重正化. 最后, Zimmerman\textsuperscript{\cite{8}} (齐默尔曼)证明了这一处理确实消除了整个积分以及所有子积分中的表观发散, 并利用参考文献\,[2]\, 的定理得到了这样的结论: 重正化的\,Feynman\,动量空间积分是收敛的.

简言之, 消除表观发散的\,``BPHZ''\,方案要求我们考虑用盒子围住整个图和(或)子图的所有可能方式(称为``树林''), 盒子可以彼此嵌套但不能重叠. (下面会给出一个例子.) 对于每一树林, 通过将盒子内(从最内部的盒子开始, 逐步向外扩展)表观发散度为$D$的所有子图的被积函数替换成它关于流入或流出该盒子的动量的\,Taylor\,级数展开的前$D+1$项, 我们可以定义一个减除项.%
{}$^*$\footnote{$^*${}正如这里所说的, 这个方案既适用于可重整理论也适用于不可重整理论. 在可重整理论中, 它意味着, 除非该盒子包含的图是与拉格朗日量中可重整项相对应的有限多个图中的一个, 否则就没有减除项.}
减\marginpar[\flushright{\small[513]\hspace*{5mm}}]{{\small\hspace*{5mm}[513]}}除后的\,Feynman\,图由原始图减去所有这些减除项给出, 这些减除项中包括由包围整个图的单个盒子组成的树林的减除项.

很容易看到, 以这种方式计算出的减除后的\,Feynman\,振幅与将原始拉格朗日量中的所有场, 耦合常数以及质量替换成它们相应的重正化后的结果得到的振幅是相同的. 这种方法与我们在第11章所采用的那类重正化之间的差异是: 用振幅定义的重正化场, 耦合常数和质量处在非常规的重正化点上, 这个点上所有的\,4\,-动量为零. (在这一方面, 本节开头讨论的一维发散积分提供了\,BPHZ\,方法分离发散项的一个基本例子.) 但重正化点没有什么特殊之处; 一旦用这些非常规的重正化量表示\,Feynman\, 振幅以使其收敛,
我们就能在不带来新的无限大的同时将其改写成用常规重正化场, 耦合以及质量表示的形式.

在实际问题中采用\,BPHZ\,减除方案是不必要的. 将场, 质量以及耦合替换成它们相应的(利用任何方便的重正化点定义的)重正化量就会自动给出抵消所有无限大的抵消项. 代替给出\,BPHZ\,减除方案确实会使所有积分收敛的证明, 我们就看一个例子, 在这个例子中展示重正化在即使有交缠发散时是如何起作用的.

考察图12.2所示的光子自能插入图$\Pi_{\mu \nu}^{\ast}(q)$的\,4\,阶贡献. (这里的树林由对$p$和$p^{\prime}$的整体积分, 单独对$p$ 的子积分以及单独对$p^{\prime}$的子积分构成.) 将顶点部分和光子场重正化的相应抵消项包含在内, 它的值是
\begin{eqnarray}
[\Pi_{\mu \nu}^{\ast}(q)]_{\text{overlap}} &=&-\frac{e^{4}}{(2\uppi)^{8}}
\int \dif^{4}p\int \dif^{4}p^{\prime}\frac{1}{(p-p^{\prime})^{2}-\mi\epsilon}  \nonumber \\
&&\quad\qquad\times \operatorname{Tr}\left\{ S(p^{\prime})\,\gamma_{\nu}\,S(p^{\prime}+q)\,\gamma^{\rho
}\,S(p+q)\,\gamma_{\mu}\,S(p)\,\gamma_{\rho}\right\}   \nonumber \cr
&&\quad-2(Z_{2}-1)_{2}\frac{\mi e^{2}}{(2\uppi)^{4}}\int \dif^{4}p\:
\operatorname{Tr}\left\{\gamma_{\nu}\,S(p+q)\,\gamma_{\mu}\,S(p)\right\}   \nonumber \\
&&\quad-(Z_{3}-1)_{\text{overlap}}\,(q^{2}\eta_{\mu \nu}-q_{\mu}q_{\nu}) \:,  \label{12.2.16}
\end{eqnarray}%
其中$S(p)\equiv[-\mi\xxp+m]/[p^{2}+m^{2}-\mi\epsilon]$; $(Z_{2}-1)_{2}$是$Z_{2}-1$中$e$的二阶项; 而$(Z_{3}-1)_{\text{overlap}}$ 是一个对数发散常数, 它是$e$的\,4\,阶项, %
用来抵消$[\Pi_{\mu\nu}^{\ast}(q)]_{\text{overlap}}$中$q^{\lambda}$的二阶项. 第二项中出现因子\,2\,是因为二阶光子自能中的两个顶点都有重正化抵消项$Z_{2}-1$. 然\marginpar[\flushright{\small[514]\hspace*{5mm}}]{{\small\hspace*{5mm}[514]}}而, 要注意的是, %
这里的第一项{\KAI{既}}可以被理解成由$p^{\prime}$-积分给出的顶点修正插入到由$p$-积分给出的光子自能上, {\KAI{也}}可以被理解成由$p$-积分给出的顶点修正插入到由$p^{\prime}$-积分给出的光子自能上, 但是, 因为只有一个光子传播子, 它{\KAI{不能}}理解成插入两个独立的顶点修正.

为了看到如何处理方程(\ref{12.2.16})中的无限大, 注意到
\begin{equation}
[(Z_{2}-1)_{2}+R_{2}]\gamma_{\mu} = \frac{\mi e^{2}}{(2\uppi)^{4}}
\int\frac{\dif^{4}p^{\prime}}{p^{\prime 2}-\mi\epsilon}\,\gamma_{\rho}\,S(p^{\prime})\,\gamma_{\mu}\,
S(p^{\prime})\gamma^{\rho}\:,   \label{12.2.17}
\end{equation}%
其中$R_{2}$是有限的余项. (Lorentz\,不变性告诉我们右边的积分正比于$\gamma_{\mu}$. %
这个积分与$(Z_{2}-1)_{2}\gamma_{\mu}$的差等于全重正化电子\lzx 电子\lzx 光子顶点在零电子动量和零光子动量处精确到$e$的二阶的值, 因而是有限的.) 这使我们可以将方程(\ref{12.2.16})重新表述成如下形式%
\begin{align}
&[\Pi_{\mu \nu}^{\ast}(q)]_{\text{overlap}} =-\frac{e^{4}}{(2\uppi)^{8}}
\int \dif^{4}p\int \dif^{4}p^{\prime}  \nonumber \\
&\qquad\times \biggl[ \frac{1}{(p-p^{\prime})^{2}-\mi\epsilon}\operatorname{Tr}%
\left\{S(p^{\prime})\,\gamma_{\nu}\,S(p^{\prime}+q)\,\gamma^{\rho}\,S(p+q)\,\gamma
_{\mu}\,S(p)\,\gamma_{\rho}\right\}  \nonumber \\
&\qquad\qquad-\frac{1}{p^{\prime 2}-\mi\epsilon}\operatorname{Tr}\left\{S(p^{\prime})\,\gamma
_{\nu}\,S(p^{\prime})\,\gamma^{\rho}\,S(p+q)\,\gamma_{\mu}\,S(p)\,\gamma_{\rho}\right\}  \nonumber \\
&\qquad\qquad-\frac{1}{p^{2}-\mi\epsilon}\operatorname{Tr}\left\{S(p^{\prime})\,\gamma_{\nu
}\,S(p^{\prime}+q)\,\gamma^{\rho}\,S(p)\,\gamma_{\mu}\,S(p)\,\gamma_{\rho}\right\}\biggr] \nonumber \\
&\qquad-2R_{2}\frac{\mi e^{2}}{(2\uppi )^{4}}\int \dif^{4}p\,\operatorname{Tr}\left\{\gamma
_{\nu}\,S(p+q)\,\gamma_{\mu}\,S(p)\right\}  \nonumber \\
&\qquad-(Z_{3}-1)_{\text{overlap}}\,(q^{2}\eta_{\mu \nu}-q_{\mu}q_{\nu})\:.%
  \label{12.2.18}
\end{align}%
首先考察只对$p^{\prime}$的积分. 前两项都是对数发散的, 但是它们的差是有限的. 第三项(加上一个规范不变的正规化子后)也是对数发散的, 但这一项中的发散(不像前两项)的形式是$q$的二次多项式, 而余项是有限的. 剩下的发散被$%
\,-(Z_{3}-1)(q^{2}\eta_{\mu \nu}-q_{\mu}q_{\nu})$这一项抵消, 这一项抵消了$\Pi_{\mu
\nu}^{\ast}(q)$中的所有二阶项.
所以$p^{\prime}$-子积分给出了有限的结果. 方程(\ref{12.2.16})的对称性表明$p$-积分也会以精确相同的方式给出一个有限的结果. 一般的对$p$和$p^{\prime}$ 的子积分, 即保持$ap+bp^{\prime}$(其中$a$和$b$为任意非零常数)不变的子积分, 明显是收敛的, 而抵消项$-(Z_{3}-1)_{\text{overlap}}(q^{2}\eta_{\mu \nu}-q_{\mu}q_{\nu})$使对$p$%
和$p^{\prime}$一起积分变得有限. 因此方程(\ref{12.2.18}){\KAI{和}}它的任意一个子积分都满足收敛的幂次计数要求,
因此根据上一节所引用的定理\textsuperscript{\cite{2}}, 总表达式实际上是收敛的.%

\subsection*{* * *}
\marginpar[\flushright
{\raisebox{3ex}[0pt]{{\small[515]\hspace*{5mm}}}}]{{\raisebox{3ex}[0pt]{\small\hspace*{5mm}[515]}}}

在电动力学中, 存在着重正化耦合以及重正化质量和场的一个自然定义. 但并非总是如此, 例如, 考察单个实标量场$\phi (x)$%
的理论, 其拉格朗日密度为\vspace{-1mm}
\begin{equation}
\mathscr{L}=-\tfrac{1}{2}\,\partial_{\lambda}\phi \partial^{\lambda}\phi -%
\tfrac{1}{2}\,m^{2}\phi^{2}-\tfrac{1}{24}\,g\,\phi^{4}\:. \vspace{-1mm}  \label{12.2.19}
\end{equation}%
到单圈阶, 标量\lzx 标量散射的$S$-矩阵由\,Feynman\,规则给出\vspace{-1mm}
\begin{equation}
S(q_{1}q_{2}\to q_{1}^{\prime}q_{2}^{\prime})=\frac{-\mi(2\uppi
)^{4}\updelta^{4}(q_{1}^{\prime}+q_{2}^{\prime}-q_{1}-q_{2})}{(2\uppi
)^{6}(16E_{1}^{\prime}E_{2}^{\prime}E_{1}E_{2})^{1/2}}F(q_{1}q_{2}%
\to q_{1}^{\prime}q_{2}^{\prime})\:,   \label{12.2.20}
\end{equation}%
其中\begin{align}
&{-}\mi(2\uppi)^{4}F(q_{1}q_{2} \to q_{1}^{\prime}q_{2}^{\prime}) = -\mi(2\uppi )^{4}g+\frac{1}{2}
\Bigl[-\mi(2\uppi)^{4}g\Bigr]^{2}\,\left[\frac{-\mi}{(2\uppi)^{4}}\right]^{2}  \nonumber \\
&\qquad\times \int \dif^{4}k\,\left[\frac{1}{\Bigl[ (q_{1}+k)^{2}+m^{2}-\mi\epsilon \Bigr] %
\Bigl[ (q_{2}-k)^{2}+m^{2}-\mi\epsilon \Bigr]} \right. \nonumber \\
&\qquad\quad\left.\phantom{\frac{1}{\Bigl[ (q_{1}+k)^{2}\Bigr]}}+(q_{2}\to-q_{1}^{\prime})+(q_{2}\to-q_{2}^{\prime})\right],
\label{12.2.21}
\end{align}%
而$q_{1},q_{2}$和$q_{1}^{\prime},q_{2}^{\prime}$是入\,4\,-动量和出\,4\,-动量. 像通常一样, 组合分母并%
旋转$k^{0}$-积分围道, 这给出
\begin{align}
F &=g-\frac{g^{2}}{16\uppi^{2}}\int_{0}^{\infty}k^{3}\,\dif k\int_{0}^{1}\dif x\,
\biggl\{\Bigl[ k^{2}+m^{2}-sx(1-x)\Bigr]^{-2}  \nonumber \\
&\quad+\Bigl[ k^{2}+m^{2}-tx(1-x)\Bigr]^{-2}+\Bigl[ k^{2}+m^{2}-ux(1-x)\Bigr]^{-2}\biggr\}\:,   \label{12.2.22}
\end{align}%
其中$s,t$和$u$是\,\textit{Mandelstam}({\KAI{曼德斯塔姆}}){\KAI{变量}}%
\begin{equation}
s=-(q_{1}+q_{2})^{2}\:, \qquad t=-(q_{1}-q_{1}^{\prime})^{2}\:, \qquad u=-(q_{1}-q_{1}^{\prime})^{2}\:,   \label{12.2.23}
\end{equation}%
满足$s+t+u=4m^{2}$; 另外, $x$是在组合分母时引入的\,Feynman\,参量. 加上一个在$k=\Lambda$处的紫外截断, 这给出结果(对于$\Lambda \gg m$)\marginpar[\flushright
{\raisebox{-12ex}[0pt]{{\small[516]\hspace*{5mm}}}}]{{\raisebox{-12ex}[0pt]{\small\hspace*{5mm}[516]}}}
\begin{align}
F &=g-\frac{g^{2}}{32\uppi^{2}}\int_{0}^{1}\dif x\,\biggl\{\ln \left(\frac{\Lambda^{2}}{m^{2}-sx(1-x)}\right) \nonumber \\
&\quad+\ln \left( \frac{\Lambda^{2}}{m^{2}-tx(1-x)}\right) +\ln \left(\frac{\Lambda^{2}}{m^{2}-ux(1-x)}\right) -3\biggr\}\:.   \label{12.2.24}
\end{align}%
假定我们处在$F$为实的区域, 我们可以将重正化耦合$g_{R}$定义为$F$在我们希望的任意$s,t,u$点的值.
例如, 假使为了保留标量间的对称性, 我们选择在离壳点{}$^*$\footnote{$^*${}回顾方程(\ref{12.2.25})的推导, 可以验证在这个推导中,
我们没有用到条件$q_{1}^{2}=q_{2}^{2}=q_{1}^{\prime 2}=q_{2}^{\prime 2}=-m^{2}$, 所以无论我们取外线动%
量是多少, 方程(\ref{12.2.24})总是适用的.}$q_{1}^{2}=q_{2}^{2}=q_{1}^{\prime
2}=q_{2}^{\prime 2}=\mu^{2}$, $s=t=u=-4\mu^{2}/3$处重正化. 定义重正化耦合$g_{R}$为$F$在该点的值, 我们有
\begin{equation}
g=g_{R}+\frac{3g^{2}}{32\uppi^{2}}\left[ \ln \left( \frac{\Lambda^{2}}{\mu
^{2}}\right) -1-\int_{0}^{1}\dif x\:\ln \left( \frac{4x(1-x)}{3}+\frac{m^{2}}{\mu
^{2}}\right) \right] +\cdots \:.   \label{12.2.25}
\end{equation}%
这样,  到$g_{R}^{2}$阶, 方程(\ref{12.2.24})对截断的依赖被抵消了, 留下了由$g_{R}$表示$F$的有限公式:%
\begin{align}
F &=g_{R}-\frac{g_{R}^{2}}{32\uppi^{2}}\int_{0}^{1}\dif x\:\biggl\{\ln \left( \frac{%
m^{2}+4x(1-x)\mu^{2}/3}{m^{2}-sx(1-x)}\right)   \nonumber \\
&\quad+\ln \left( \frac{m^{2}+4x(1-x)\mu^{2}/3}{m^{2}-tx(1-x)}\right) +\ln
\left( \frac{m^{2}+4x(1-x)\mu^{2}/3}{m^{2}-ux(1-x)}\right)\biggr\}+\cdots \:. \label{12.2.26}
\end{align}%
这里, $\mu^{2}$可以取任何大于$-3m^{2}$的值, 在这个范围内$g_{R}$是实的. 自然, 方程(\ref{12.2.26})中显式的$\mu$-依赖性被重正化耦合的$\mu$-依赖性抵消了. 在卷\,\textrm{II}\,中的重正化群方法中, 这种选择重正化方式(这当然也存在于电动力学和其他真实理论中)上的这种自由将变得非常重要.

\section{可重整性是必要的吗?} \label{sec:12.3}
\setcounter{equation}{0}

在上一节, 我们发现了一类特殊理论, 这类理论的拉格朗日量只包含有限个项, 然而重正化手续对于这类理论也是适用的.
在这些理论中, 所有相互作用都满足可重整条件\marginpar[\flushright
{\raisebox{-5ex}[0pt]{{\small[517]\hspace*{5mm}}}}]{{\raisebox{-5ex}[0pt]{\small\hspace*{5mm}[517]}}}
\[
\Delta_{i}\equiv 4-d_{i}-\sum_{f}n_{if}(s_{f}+1)\geq 0\:,
\]%
其中$d_{i}$和$n_{if}$分别是第$i$类相互作用中$f$类场和导数的数目, %
$s_{f}$是(附加一些限制后的)$f$类场的自旋. 为了使重正化在这类理论中起作用, 通常还要对称性原理所允许的所有可重整相互作用都应该真正地出现在拉格朗日量中.

很重要的是, 这样的相互作用只有有限多种. 如果有太多的场或导数, 或者场的自旋过高, $\Delta_{i}$都会变成负的. 除非有特殊的抵消, 可重整相互作用根本就不能包含$s_{f}\geq 1$的场, 这是因为对于拉格朗日量中的任何一项, 要包含这种场以及两个或多个其他场同时还有$\Delta_{i}\geq 0$, 那么它只能包含单个$s_{f}=1$ 的场连同两个标量场但不能有导数, %
而这不是\,Lorentz\,不变的. 我们将在卷\,\textrm{II}\,中看到, 在合适的规范下, 一般的无质量自旋\,1\,规范场等效于有%
$s_{f}=0$, 就像光子. 另外, 我们将在卷\,\textrm{II}\,中看到, 即便是有质量规范场, 取决于它们的质量实际上来自于哪里, 它们也可以等效地有$s_{f}=0$. 暂且将这些特殊情况搁置一旁, 对于只涉及标量($s=0$), 光子($s=0$)以及自旋$\frac{1}{2}$费米子($s=\frac{1}{2}$)的拉格朗日密度, 表\ref{tab:12.2}中给出了\,Lorentz\,不变性和规范不变性所允许的{\KAI{所有}}可重整项.


\vspace{0.4cm}
\begin{small}

\noindent
{\bf 表~12.2}\quad 包含标量$\phi$, Dirac\,场$\psi$和光子场$A^{\mu}$的拉格朗日密度中所允许的可重整项. %
  这里的$n_{if}$和$d_{i}$是$i$类相互作用中$f$类场和导数的数目, 而$\Delta_{i}$是相应系数的量纲.
\end{small}
 \vspace{-0.3cm}


\begin{center}\begin{footnotesize}
\def\temptablewidth{\textwidth}
{\footnotesize {\rule{\temptablewidth}{1pt}}\\[-0.5mm]
\renewcommand{\arraystretch}{1.15}
\tabcolsep=15pt\begin{tabular}{cccccc}
  & $\qquad n_{if}\quad $ &  & \quad $d_{i}$ &\quad  $\Delta_{i}$ &\quad   $\mathscr{H}_{i}$ \quad  \\
  \quad 标量\quad &\quad 光子\quad &\quad 自旋$\tfrac{1}{2}$ \quad & & & \\  \hline
  \quad 1 & \quad 0 & \quad 0 & \quad 0 &\quad  3 &\quad  $\phi$  \\
  \quad 2 & \quad 0 & \quad 0 & \quad 0 &\quad  2 & \quad $\phi^{2}$   \\
  \quad 2 & \quad 0 & \quad 0 & \quad 2 &\quad  0 & \quad $\partial_{\mu}\phi \partial^{\mu}\phi$   \\
  \quad 3 & \quad 0 & \quad 0 & \quad 0 &\quad  1 & \quad $\phi^{3}$   \\
  \quad 4 & \quad 0 & \quad 0 & \quad 0 &\quad  0 &\quad  $\phi^{4}$   \\
  \quad 2 & \quad 1 & \quad 0 & \quad 1 &\quad  0 & \quad $\phi\partial_{\mu}\phi A^{\mu}$   \\
  \quad 2 & \quad 2 & \quad 0 & \quad 0 &\quad  0 & \quad $\phi^{2}A_{\mu}A^{\mu}$   \\
  \quad 1 & \quad 0 & \quad 2 & \quad 0 &\quad  0 & \quad $\phi \bar{\psi} \psi$   \\
  \quad 0 & \quad 2 & \quad 0 & \quad 2 &\quad  0 & \quad $F_{\mu\nu}F^{\mu\nu}$   \\
  \quad 0 & \quad 0 & \quad 2 & \quad 0 &\quad  1 & \quad $\bar{\psi} \psi$   \\
  \quad 0 & \quad 0 & \quad 2 & \quad 1 &\quad  0 & \quad $\bar{\psi} \gamma^{\mu} \partial_{\mu} \psi$   \\
  \quad 0 & \quad 1 & \quad 2 & \quad 0 &\quad  0 & \quad $\bar{\phi}\gamma^{\mu} A_{\mu}\psi$   \\
\end{tabular}\\[-0.3mm]
\def\temptablewidth{\textwidth}{\rule{\temptablewidth}{1pt}}
}\vspace{-0.4cm}
\end{footnotesize}
\end{center}


我们看到, 可重整性的要求对我们能够考虑的物理理论的种类施加了苛刻的约束. 这些约束为物理理论结构提供了一条有价值的线索. 例如, Lorentz\,不变性和规范不变性本身允许在量子电动力学的拉格朗日量中引入正比于%
$\bar{\psi}[\gamma_{\mu},\gamma_{\nu}]\psi F^{\mu \nu}$的``Pauli''项, 而这一项使电子磁矩变成了一个可调的参量, 但因为它们不是可重整的, 我们排除了这些项. 量子电动力学的成功预言, 例如\,\ref{sec:11.3}\,节中概述的电子磁矩的计算, 可以看作可重整性原理起作用的结果. 同样的分析也适用于我们将在卷\,\textrm{II}\,讨论的弱作用, 电磁作用和强作用的标准模型; 可以在该理论中加入任意多个项, 例如夸克和轻子之间的\,4\,-费米子相互作用, 这个相互作用会使标准模型的所有预测失效, 而这些项仅仅因为是不可重整的被排除掉了.

我们非得相信拉格朗日量必须被限制成只能包含可重整的相互作用吗? 正如我们在上一节中所看到的, 如果我们在拉格朗日量中引入对称性所允许的{\KAI{所有}}相互作用, 这样的相互作用有无穷多个, 那么就会有可用来抵消每一个紫外发散的抵消项.
在\marginpar[\flushright{\small[518]\hspace*{5mm}}]{{\small\hspace*{5mm}[518]}}这个意义上, 只要我们像前面说过的那样在拉格朗日量中引入了所有可能的项, 不可重整理论就变得和可重整理论一样可重整.

近些年来, 可重整性不是一个基本的物理要求以及任何真实的量子场论实际上都将同时包含可重整项和不可重整项, 这些变得越来越显而易见. 这一观点的改变, 部分可以归因于在寻找可重整引力理论上接二连三地失败. 在由\,Einstein\,等效原理所支配的一类普遍的引力度规理论中, 可重整的相互作用根本就不存在\ezx 广义协变的相互作用必须从曲率张量和它的广义协变导数中构造, 因此, 即便是在引力子传播子形如$k^{-2}$的``规范''中,
这\marginpar[\flushright{\small[519]\hspace*{5mm}}]{{\small\hspace*{5mm}[519]}}些相互作用包含了过多的度规导数以至于丧失了可重整性. 特别地, 引力的耦合常数$8\uppi G_{N}=(2.43\times 10^{18}\,\mathrm{GeV})^{-2}$量纲为负, 我们可以从这一点很容易地看到广义相对论的不可重整性. 即使只有这些, 抵消虚引力子引起的发散也将要求拉格朗日量包含对称性所允许的所有相互作用\ezx 不仅是包含引力子的相互作用, 还有包含任意粒子的相互作用.

但是, 如果可重整性不是一个基本的物理原理, 那么我们又该如何解释量子电动力学和标准模型这类可重整理论的成功呢? 答案可以通过简单的量纲分析看到. 我们已经注意到$i$类相互作用的耦合常数具有量纲
\begin{equation}
[g_{i}]\sim [\text{质量}]^{\Delta_{i}}\:, \label{12.3.1}
\end{equation}%
其中$\Delta_{i}$是指标(\ref{12.1.9}). 不可重整相互作用就是那些耦合常数量纲为质量{\KAI{负}}幂次的相互作用. 现在从(\ref{12.3.1})出发可以不无道理地猜测: 耦合常数不仅有被$\Delta_{i}$确定的量纲, 而且量级上大体是%
\begin{equation}
g_{i}\approx M^{\Delta_{i}} \:,  \label{12.3.2}
\end{equation}%
其中$M$是某个通常的质量. (我们将会看到这实际上是下面要讨论的有效场论中的情况, 在卷\,\textrm{II}\,中会更加详细地讨论.) 在计算特征动量标度$k\ll M$ 的物理过程时, 将$\Delta_{i}<0$的$i$类不可重整相互作用包含在内将会引入因子%
$g_{i}\approx M^{\Delta_{i}}$, 由量纲分析可知, 这必将伴随一个因子$k^{-\Delta_{i}}$, 因而这种相互作用的效应在$k\ll M$时被因子$(k/M)^{-\Delta_{i}}\ll 1$压低{}$^*$\footnote{$^*${}在此有必要假定紫外发散已经被重正化移除了, 这样就不会有紫外截断因子$\Lambda$来干扰我们的量纲分析. 否则, 量纲分析会告诉我们, 对于每个额外的$\Delta_{i}<0$的不可重整耦合常数因子$g_{i}$, 它在$\Lambda \to \infty$时将伴随一个增长因子$\Lambda^{-\Delta_{i}}$. 这个对量纲的讨论促使\,Heisenberg\textsuperscript{\cite{9}}\,很早就依照耦合常数的量纲对相互作用进行了分类, %
并提出\textsuperscript{\cite{10}}新效应将发生在量级为$g_{i}^{1/\Delta_{i}}$的能量处, 例如能量%
$G_{F}^{-1/2}\approx300\,\mathrm{GeV}$处, 其中$G_{F}$是费米$\beta$-衰变理论的\,4\,-费米子耦合常数. 在重正化理论发展起来后,
Sakata(坂田)等人\textsuperscript{\cite{11}} 注意到了不可重整理论是那些耦合常数量纲为负的理论.}了. (在卷\,II\,中, 会用重正化群方法更加仔细地进行这个讨论.) 可重整的电弱理论和强相互作用理论的成功仅表明$M$远高于检验这些理论的能标.

例如\marginpar[\flushright{\small[520]\hspace*{5mm}}]{{\small\hspace*{5mm}[520]}}, 对电子或$\mu$子的传统电动力学的领头阶不可重整修正是那些量纲为\,5\,的相互作用, 它们仅被压低了一个$1/M$因子. 同时满足\,Lorentz\,不变性, 规范不变性和\,\textsf{CP}\,不变性的此类相互作用只有一个, 就是$(\mi e/2M)\bar{\psi}[\gamma_{\mu},\gamma_{\nu}]\psi F^{\mu \nu}$阶的\,Pauli\,项. 根据方程(\ref{10.6.24}), (\ref{10.6.17}) 和(\ref{10.6.19}), 这样的项会给电子或$\mu$子磁矩带来$4e/M$阶的贡献. 电子磁矩的计算结果与实验相符至$10^{-10}e/2m_{e}$ 阶, 所以$M$必须要大于约$8\times 10^{10}m_{e}=4\times 10^{7}\,\mathrm{GeV}$.

如果其他对称性约束了不可重整相互作用的形式, 这一限制会被削弱. 例如, 对于传统的量子电动力学, %
在手征变换$\psi \to \gamma_{5}\psi$下, 除了费米子质量项$-m\bar{\psi}\psi$有一个符号改变外, 拉格朗日量是不变的.
如果我们假定整个拉格朗日量在形式对称性$\psi \to \gamma_{5}\psi$, $m\to -m$下不变, 那么, %
为了使拉格朗日量中的\,Pauli\,项对电子磁矩的贡献仅是$4em/M^{2}$阶的, 它就必须伴随着一个额外因子$m/M$出现. 正是由于这个额外的因子$m$, 这里为$M$ 提供最有用限制的是$\mu$子而不是电子. $\mu$子磁矩的计算值与实验直至$10^{-8}e/2m_{\mu}$阶都是相符的, 所以$M$必须大于约$\sqrt{8\times 10^{8}}m_{\mu}=3\times 10^{3}\,\mathrm{GeV}$. 在任何情况下, 如果$M$非常接近$10^{18}\,\mathrm{GeV}$, 那么我们就可以合理地忽略掉可能出现在量子电动力学中的不可重整相互作用.

这些考察帮助我们解决了拉格朗日量中的高阶导数项所带来的一些谜题. 例如, 在实标量场$\phi$的一般理论中,
我们会期望在拉格朗日密度中找到形如$\phi \square^{n}\phi$的项. %
任何一个这样的项都会对标量自能函数$\Pi^{\ast}(q^{2})$有一个正比于$(q^{2})^{n}$的直接贡献. 如果我们计入了这样的贡献的所有阶, 但忽略所有其他不可重整相互作用的效应, 那么传播子$\Delta^{\prime
}(q^{2})=1/(q^{2}+m^{2}-\Pi^{\ast}(q^{2}))$就不会像\,\ref{sec:10.7}\,节的一般讨论中所预期的那样在$q^{2}$的负值处有单极点, 而是有$n$个这样的极点(其中一些会重合), 而且一般处在$q^{2}$的复值处. 但是, %
如果不可重整项$\phi\square^{n}\phi$有一个$M^{-2(n-1)}$阶的系数, 其中$M\gg m$, 那么多余的极点就在$M^{2}$%
阶的$q^{2}$处, 在这里忽略其他也必须出现在拉格朗日量中的无限多个不可重整相互作用就是不合理的. 因此, 在一般的不可重整拉格朗日量中出现高阶导数项与\,\ref{sec:10.7}\,节中使用的量子场论的一般底层原理并不矛盾. 然而, 正如反复提到的, 由于同样的原因, 我们也不能用高阶导数项来完全避免紫外发散. 拉格朗日密度中的\marginpar[\flushright{\small[521]\hspace*{5mm}}]{{\small\hspace*{5mm}[521]}}项$M^{-2(n-1)}\phi \square^{n}\phi$在动量$q^{2}\approx M^{2}$处提供了一个截断, 但是在这些动量处我们无法忽略所有其他必须要出现的不可重整相互作用.

尽管不可重整相互作用被高度压低了, 但如果它们有其他本要被禁止的效应, 不可重整相互作用也是可探测的. 例如, %
我们将在\,\ref{sec:12.5}\,节看到, 电荷共轭对称性和空间反演不变性是电磁相互作用的结构被要求了规范不变性, Lorentz\,不变性和可重整性后的自然结果, 然而我们可以很容易想出破坏这些对称性的不可重整项, 例如电子电偶极矩项$\bar{\psi}\gamma_{5}[\gamma_{\mu},\gamma_{\nu}]\psi F^{\mu \nu}$, 或者\,Fermi\,相互作用$\bar{\psi}\gamma
_{5}\gamma_{\mu}\psi \bar{\psi}\gamma^{\mu}\psi$. 现今被广泛接受的一个观点是, 重子数和轻子数守恒被高度压低的不可重整相互作用的微弱效应破坏了. 另一可探测不可重整相互作用的例子是引力给出的. 正如前面提到的, 引力子根本就没有可重整%
相互作用. 然而, 我们显然探测到了引力, 这是因为引力有这样的特殊性质: 一个宏观物体中的所有粒子的引力场可以相互叠加.

尽管不可重整理论包含无限多个自由参量, 它们依旧保持着可观的预测能力:\textsuperscript{\cite{12}} 它使我们可以计算\,Feynman\,振幅的不解析部分, 就像上一节开头一维例子中的$\ln q$和$q\ln q$项. 这种计算只是再次产生了$S$-矩阵理论的公理所要求的结果, %
即$S$-矩阵只有那些幺正性所要求的奇异性.

矛盾的是, 当对称性原理禁止了可重整相互作用时, 不可重整量子场论显示出了它最有用的地方. 在这种情况下, 通过做$k/M$的幂级数展开, 我们可以导出一个有用的微扰论. 这已经在低能$\pi$介子理论\textsuperscript{\cite{12,13}}以及低能引力子理论中做了详尽的推导,\textsuperscript{\cite{14}} 我们将在卷\,\textrm{II}\,中对$\pi$介子情况进行详细的讨论. 作为一个简单的例子, 考察一个实标量场的理论, 该场在平移
\[
\phi (x)\to \phi (x)+\epsilon
\]%
下满足不变性原理, 其中$\epsilon$是任意常数. 这个对称性禁止了任何可重整相互作用和标量质量, 但它允许无限多个不可重整的导数相互作用
\[
\mathscr{L}=-\frac{1}{2}\partial_{\mu}\phi \partial^{\mu}\phi -\frac{g}{4%
}(\partial_{\mu}\phi \partial^{\mu}\phi )^{2}-\cdots \:,
\]%
其中$g\approx M^{-4}$\marginpar[\flushright{\small[522]\hspace*{5mm}}]{{\small\hspace*{5mm}[522]}}, 而\,``$\cdots$''\,表示有更多导数或场的项. (简单起见, 在这里假定该理论在反演$\phi \to -\phi$下也具有对称性.) 根据上面的量纲分析, 所有能量和动量都是$k\ll M$阶的一般反应的图都被因子$(k/M)^{\nu}$压低了, 其中
\[
\nu =-\sum_{i}V_{i}\Delta_{i}=\sum_{i}V_{i}(d_{i}+n_{i}-4)\:,
\]%
而$n_{i}$和$d_{i}$是$i$类相互作用中标量场和导数的个数, $V_{i}$是这些相互作用顶点在我们图中的个数. %
当$k\ll M$时, 对任何过程的主要贡献都是那些$\nu$值最小的. 利用熟悉的连通图的拓扑恒等式:%
\[
\sum_{i}V_{i}=I-L+1\:,\qquad \sum_{i}V_{i}n_{i}=2I+E\:,
\]%
其中$I$, $E$和$L$是图中内线、 外线和圈的个数, $\nu$的公式可以用一个更有用的形式表示. 结合这些关系给出:%
\[
\nu =2E-4+4L+\sum_{i}V_{i}(d_{i}-n_{i})\:.
\]%
现在, 场变换对称性要求每个场必须伴随至少一个导数, 所以$d_{i}-n_{i}$以及$L$和$V_{i}$对于所有相互作用都是非负的. 因此, 对一个给定过程(即外线个数$E$ 固定), 主要的项是那些仅由{\KAI{树}}图(即$L=0$)和导数个数$d_{i}=n_{i}$最少的相互作用构成的项. 也就是说, 在领头阶中, 我们可以取拉格朗日密度仅依赖于场的{\KAI{一阶}}导数. 高阶修正可以包含圈和(或)那些包含某些场的更多导数的相互作用. 但到$k/M$的任意给定的$\nu$阶,
我们只需考虑有限个图, 即那些$L\leq (4-2E+\nu )/4$的图, 以及有限多种相互作用类型.%

例如, 标量\lzx 标量散射的领头阶是通过利用第一阶相互作用$-g(\partial_{\mu}\phi\partial^{\mu}\phi )^{2}$计算单顶点树图给出. 根据我们对$\nu$的公式, 在低能下被因子$(k/M)^{2}$压低的领头阶修正来源于另一单顶点树图,
它由形如{}$^*$\footnote{$^*${}依照\,\ref{sec:7.7}\,节的评述, 我们排除了包含$\square \phi$的相互作用, 因为这些相互作用可以用$\phi$ 的场方程以其他相互作用的形式表示出来.}$\partial_{\mu}\partial_{\nu}\partial^{\mu}\partial^{\nu
}\phi \partial_{\lambda}\phi \partial^{\lambda}\phi$这种包含两个额外导数的相互作用产生. 下一阶的%
修正, 在低能处又被两个额外的$k/M$因子压低了, 它们既来自图12.4中仅由相互作用$-g(\partial_{\mu}\phi
\partial^{\mu}\phi )^{2}$计\marginpar[\flushright{\small[523]\hspace*{5mm}}]{{\small\hspace*{5mm}[523]}}算的单圈图(包含外线间的置换), 还有一部分来自单顶点树图, 其中的单顶点来自有$8$个导数的四次相互作用, 它的耦合常数中含有用来抵消圈图发散{}$^{**}$\footnote{$^{**}${}如果我们采用维数正规化, 它们是我们在单圈图中唯一遇到的发散. 对于其他正规化方法, 也存在四次发散和二次发散, 它们被有\,4\,个或\,6\,个导数的\,4\,-标量相互作用中的抵消项抵消.}的无穷大部分. 圈图也在散射振幅中产生有限项, 这些有限项正比于类似$%
\,s^{4}\ln s+t^{4}\ln t+u^{4}\ln u$, $s^{2}t^{2}\ln u+t^{2}u^{2}\ln
s+u^{2}s^{2}\ln t$等等这样的项, 其中系数是可计算的且正比于$g^{2}$. 这些有限项代表为了保证$S$-矩阵幺正性所需的最低阶散射振幅修正, 然而到目前为止, 微扰量子场论仍是计算它们的最简单方法.%

\begin{figure}[h!]
\centering
\includegraphics{1204.eps}\\
  \caption{带有导数四线性相互作用的理论中的标量\lzx 标量散射的单圈图.}
  \label{fig:12.4}
\end{figure}

尽管不可重整理论可以提供一个有用的关于能量的幂级数展开, 但是当能量达到各种耦合共有的典型质量标度$M$处时, 它们在所难免地失去了所有预测能力. 如果我们打算逐级地做这些展开, $S$-矩阵元的结果将在$E\gg M$时破坏幺正性要求. 在这样的能量下究竟发生了什么? 似乎仅有两种可能. 一种是强度不断增长的不可重整耦合的效应以某种方式达到极点了, 避免了与幺正性矛盾.\textsuperscript{\cite{15}} 另一种可能是某种新物理在标度$M$处介入了. 在这种情况下, 描述能量$E\ll M$的自然规律的不可重整理论%
只是{\KAI{有效场论}}而不是真正基本的理论.%

有效场论最早的例子可能是\,Euler\,等人\textsuperscript{\cite{16}}在\,20\,世纪\,30\,年代给出的低能光子\lzx 光子相互作用理论. %
(参看\,\ref{sec:1.3}\,节.) 事实上, 他们计算了像图12.5中的那样\,Feynman\,图对光子\lzx 光子散射的贡献, 并发现在能量远小于$m_{e}$时, 光子被光子散射得结果与如下有效拉格朗日量计算出的结果相同
\begin{align*}
\mathscr{L}_{\text{eff}} &=\frac{2\alpha^{2}}{45m_{e}^{4}}
\Bigl[(\bE^{2}-\bB^{2})^{2}+7(\bE\cdot \bB)^{2}\Bigr]  \\
&\quad+\frac{eE}{m_{e}^{2}}\,\text{和}\,\frac{eB}{m_{e}^{2}}\text{的高阶项}\:.
\end{align*}%
为了计算光子相互作用矩阵元中的领头项\marginpar[\flushright{\small[524]\hspace*{5mm}}]{{\small\hspace*{5mm}[524]}}, Euler\,等人只在树图近似下使用这个有效拉格朗日量. 尽管这种拉格朗日量是不可重整的, 不久之后它们就被应用到树图近似以外了.\textsuperscript{\cite{12,17}}

\begin{figure}[h!]
\centering
\includegraphics{1205.eps}\\
  \caption{光子\lzx 光子散射图, 它的效应在低能时可以从\,Euler\,等人\textsuperscript{\cite{16}}的有效拉格朗日量中计算出来. 直线是电子; 波浪线是光子.}
 \label{fig:12.5}
\end{figure}

在现代术语中, 我们称, 电子在推导该拉格朗日量时被``积掉''了, 这是因为在单圈近似下我们有
\[
\exp \left( \mi\int \mathscr{L}_{\text{eff}}(\bE,\bB)\dif^{4}x\right)
=\int \left[ \prod_{x}\dif\psi_{e}(x)\right] \exp \left( \mi\int \mathscr{L}_{\textrm{QED}}(\psi_{e},\bA)\dif^{4}x\right) \:.
\]%
一个更加普遍的方法就是写出最普遍的不可重整有效拉格朗日量, 利用它计算出各种振幅的能量和动量的展开式, 然后通过比对振幅结果与从底层理论得到的结果来确定有效拉格朗日量中的常数.

我们还会遇到有效场论, 尤其是到卷\,\textrm{II}\,考察破缺对称性的时候. 我们将看到, 即便有效场论无法从一个底层理论中导出, 无法导出的原因要么是对底层理论并不清楚, 要么是相互作用过强以至于无法使用微扰论, 有效场论仍是十分有用的. 事实上, 即使我们对带电粒子的性质一无所知, 能量足够低的光子散射也只能由$(\bE^{2}-\bB%
^{2})^{2}$和$(\bE\cdot \bB)^{2}$构成的有效拉格朗日量来描述,
这是因为它们是唯一的\,Lorentz\,不变和规范不变且在$\bE$和$\bB$上没有导数作用的\,4\,次项. 有这种导数的项在低的光子能量$E$处被额外的因子$E/M$压低, 其中$M$是要被积掉的带电粒子的某个典型质量. 我们可以更进一步\marginpar[\flushright{\small[525]\hspace*{5mm}}]{{\small\hspace*{5mm}[525]}}: 我们将看到, 即便有效场论所描述的轻粒子根本没有出现在底层理论中, 只要重粒子组分被积掉了, 它仍是很有用的. 底层理论甚至可能根本不是场论\ezx 引力的不相容问题使得很多理论家相信这个底层理论, 事实上, 是弦论. 但是, 无论有效场论来自何处, 它都不可避免地是不可重整理论.

\section[浮动截断]{浮动截断{}$^*$\footnote{$^*${}本节内容或多或少在本书的发展主线之外, 可以在第一次阅读时略过.}} \label{sec:12.4}
\setcounter{equation}{0}

在结束本章之前, 值得讨论一下传统重正化理论与\,Wilson(威尔逊)所开创方法\textsuperscript{\cite{18}}之间的关系.
在\,Wilson\,方法中, 我们要在分量为$\Lambda$阶的动量上加入一个``浮动''的有限紫外截断(或尖锐或光滑), 并且取代要求$\Lambda \to \infty$, 我们将要求理论中的(那些出现在拉格朗日量中的)裸常数对$\Lambda$的依赖方式要使可观测量是%
$\Lambda$-无关的.

采用无量纲参量要方便一些. 如果一个裸耦合参量或裸质量参量$g_{i}(\Lambda )$的量纲是$[$质量$]^{\Delta_{i}}$, 我们定义相应的无量纲参量$\mathscr{G}_{i}$
\begin{equation}
\mathscr{G}_{i}(\Lambda )\equiv \Lambda^{-\Delta_{i}}g_{i}(\Lambda )\:. \label{12.4.1}
\end{equation}%
普通的量纲分析告诉我们, $\mathscr{G}_{i}$在某个截断$\Lambda^{\prime}$处的值可以表示成$\mathscr{G}_{j}$在另一截断$\Lambda$处的值和无量纲比值$\Lambda^{\prime}/\Lambda$的函数:%
\begin{equation}
\mathscr{G}_{i}(\Lambda^{\prime})=F_{i}\Bigl( \mathscr{G}(\Lambda),\Lambda^{\prime}/\Lambda \Bigr) \:.   \label{12.4.2}
\end{equation}%
除$\Lambda^{\prime}$和$\Lambda$外, $F$中不可能再有其他带量纲的参量, 这是因为没有任何紫外或红外发散可以%
进来; 常数$\Lambda$与$\Lambda^{\prime}$之差来自于内线动量被限制在$\Lambda$和$\Lambda^{\prime}$之间的那些图. 将方程(\ref{12.4.2})对$\Lambda^{\prime}$微分, 然后令$\Lambda^{\prime}$等于$\Lambda$, 这产生了一个$\mathscr{G}_{i}$的微分方程:%
\begin{equation}
\Lambda \frac{\dif}{\dif\Lambda}\mathscr{G}_{i}(\Lambda )=\beta_{i}(\mathscr{G}(\Lambda ))\:,   \label{12.4.3}
\end{equation}%
其中$\beta_{i}(\mathscr{G})\equiv[\partial /\partial z\:F_{i}(\mathscr{G},z)]_{z=1}$. 对于弱耦合, 函数$\beta_{i}(\mathscr{G})$可以在微扰论中计算出来. 这是重正化群方程的\,Wilson\,版本,
我们将在卷\,\textrm{II}\,中以稍许不同的方式来讨论它.

对于任意有限的截断值\marginpar[\flushright{\small[526]\hspace*{5mm}}]{{\small\hspace*{5mm}[526]}}, 拉格朗日量定义了一个有效场论, 这个有效场论不是(或不只是)积掉了``重''粒子,
例如\,Euler\,等人工作中的电子, 而是积掉了{\KAI{所有}}动量大于$\Lambda$的粒子. 即便我们出发的理论在某个截断$\Lambda_{0}$处只有有限个耦合参量$\mathscr{G}_{i}^{0}$, 在任何其他的截断值处, 微分方程(\ref{12.4.3})一般会产生对称性原理允许的所有耦合的非零值.{}$^*$\footnote{$^*${}这一规则唯一已知例外出现在基于超对称的理论中.\textsuperscript{\cite{4}}}

我们现在对可重整耦合和不可重整耦合做一区分, 分别标记为$\mathscr{G}_{a}$和$\mathscr{G}_{n}$, 其中$a$取遍有限多个$\Delta_{a}\geq 0$的耦合(包括质量), 这些耦合的数目记为$N$, 而$n$取遍量纲$\Delta_{n}<0$的无限个耦合.
我们希望证明, 如果在某个初始截断值$\Lambda_{0}$处$\mathscr{G}_{a}(\Lambda_{0})$和$\mathscr{G}_{n}(\Lambda_{0})$处在一个一般的$N$-维初始曲面$\mathscr{S}_{0}$上, 那么(附加一些条件)当$\Lambda\ll \Lambda_{0}$时, 它们会到达一个与$\Lambda_{0}$和初始曲面无关的固定曲面$\mathscr{G}$.{}$^{**}$\footnote{$^{**}${}这一定理源于\,Polchinski.\textsuperscript{\cite{19}} 这里给出的是一个不太严格的简化版本. (在Polchinski的证明中, 初始曲面被取成所%
有不可重整耦合都为零. 我们将在这里看到, 对于一般的初始曲面, 耦合会趋于同一固定曲面.)} 这个固定曲面是稳定的, 也就是说, 从曲面上的任意一点出发, 由方程(\ref{12.4.3})生成的轨道都在曲面上. 这样一个稳定曲面定义了一组参量个数有限的理论, 这组理论的物理内容都与截断无关, 而正如上一节所论述的, 这正是可重整理论必需的性质. 更进一步, 这一构造表明, 在截断$\Lambda_{0}$处定义的一般理论在$\Lambda \ll \Lambda_{0}$时看起来将像个可重整理论.{}$^\dag$\footnote{$^\dag${}当然, 有些理论的对称性和场根本就不允许存在任何可重整相互作用. 只含费米场或只含引力场的理论就是这种情况.这样的理论在$\Lambda \ll \Lambda_{0}$时看起来像自由场理论.}

为了证明这些结果, 考察在满足方程(\ref{12.4.3})的$\mathscr{G}_{i}(\Lambda )$值上的一个小微扰$%
\,\delta \mathscr{G}_{i}(\Lambda )$. 它将满足微分方程
\begin{equation}
\Lambda \frac{\dif}{\dif\Lambda}\delta \mathscr{G}_{i}(\Lambda)
=\sum_{j}M_{ij}(\mathscr{G}(\Lambda))\,\delta \mathscr{G}_{j}(\Lambda)\:, \label{12.4.4}
\end{equation}%
其中
\begin{equation}
M_{ij}(\mathscr{G})\equiv \frac{\partial}{\partial \mathscr{G}_{j}}\beta_{i}(\mathscr{G})\:.   \label{12.4.5}
\end{equation}%
这一方程将可重整耦合与不可重整耦合耦合在一起, 使我们很难看清它们行为之间的差异. 为了让它们退耦, 我们引入线性变换
\begin{equation}
\xi_{n}\equiv \delta \mathscr{G}_{n}-\sum_{ab}\frac{\partial \mathscr{G}_{n}}{\partial \mathscr{G}_{a}^{0}}
\left( \frac{\partial \mathscr{G}}{\partial\mathscr{G}^{0}}\right)_{ab}^{-1}\delta \mathscr{G}_{b}\:,
\label{12.4.6}
\end{equation}%
其中$\mathscr{G}_{n}^{0}$是\marginpar[\flushright{\small[527]\hspace*{5mm}}]{{\small\hspace*{5mm}[527]}}可重整耦合在截断$\Lambda_{0}$处的值, 我们将用它作为初始曲面的坐标, 而%
$\mathscr{G}_{n}$是从微分方程(\ref{12.4.3})中导出的不可重整耦合在截断$\Lambda$处的值, 它在$
\,\Lambda_{0}$的初值是初始表面上坐标为$\mathscr{G}_{a}^{0}$的点. 为了计算$\xi_{n}$对$\Lambda$%
的导数, 我们注意到导数$\partial \mathscr{G}_{i}/\partial \mathscr{G}_{a}^{0}$与$\delta \mathscr{G}_{i}$满足同一个微分方程(\ref{12.4.4}). 这样, 证明下式就是一个初等练习
\begin{equation}
\Lambda \frac{\dif}{\dif\Lambda}\xi_{n}=\sum_{m}N_{nm}\,\xi_{m}\:, \label{12.4.7}
\end{equation}%
其中
\begin{equation}
N_{nm}\equiv M_{nm}-\sum_{ab}\frac{\partial \mathscr{G}_{n}}{\partial\mathscr{G}_{a}^{0}}\left( \frac{\partial \mathscr{G}}{\partial \mathscr{G}^{0}}\right)_{ab}^{-1}M_{bm}\:.   \label{12.4.8}
\end{equation}%
现在我们必须估计$N_{nm}$的元. 自由场不需要截断, 所以对于非常弱的耦合, 所有裸参量$g_{i}(\Lambda )$都变成$\Lambda$-无关的. 因此, 对于弱耦合, 无量纲参量$\mathscr{G}_{i}$的标度就是$\Lambda^{-\Delta_{i}}$, %
而矩阵$M_{ij}$是
\begin{equation}
M_{ij}\approx -\Delta_{i}\updelta_{ij}\:.   \label{12.4.9}
\end{equation}%
由此可知矩阵$N_{nm}$近似为$-\Delta_{n}\updelta_{nm}$. 不可重整耦合的标志性特征是$\Delta_{n}<0$, 所以方程(\ref{12.4.7})告诉我们, 至少对于某些有限范围内的耦合, 其中$N_{nm}$是正定的, $\xi_{n}$在$\Lambda
\ll \Lambda_{0}$时像$\Lambda /\Lambda_{0}$的正幂次那样衰减. 于是在这一极限下, 微扰之间的关系是%
\begin{equation}
\delta \mathscr{G}_{n}=\sum_{ab}\frac{\partial \mathscr{G}_{n}}{\partial %
\mathscr{G}_{a}^{0}}\left( \frac{\partial \mathscr{G}}{\partial \mathscr{G}%
^{0}}\right)_{ab}^{-1}\updelta \mathscr{G}_{b}\:.   \label{12.4.10}
\end{equation}%
特别地, 如果我们在初始曲面$\mathscr{S}_{0}$上和(或)该曲面的起始点上和(或)初始截断$\Lambda_{0}$上做一点小变动, 使得可重整耦合的微扰$\updelta \mathscr{G}_{a}$在某个截断$\Lambda \ll \Lambda_{0}$处为零, 那么所有其他耦合的微扰$\updelta \mathscr{G}_{n}$在截断$\Lambda$处也为零. 因此$\Lambda \ll \Lambda_{0}$处的不可重整耦合$\mathscr{G}_{n}(\Lambda )$只能依赖于可重整耦合$\mathscr{G}_{a}(\Lambda )$, 而不另外依赖初始曲面或曲面上的起始点或初始截断$\Lambda_{0}$. 因此, 在截断$\Lambda \ll \Lambda_{0}$处, 所有耦合趋于一个$N$-维曲面$\mathscr{S}$, 坐标为$\mathscr{G}_{\alpha}(\Lambda )$, 这个曲面既不依赖初始曲面也不依赖$\Lambda_{0}$. 要注意的是, 不可重整耦合$\mathscr{G}_{n}$在$\mathscr{S}$ 上一般不是小量; 重要的是它们变成了可重整耦合的函数. 保持$\Lambda$远小于$\Lambda_{0}$的$\Lambda$变分将改变耦合, 但耦合将依旧贴近于$\mathscr{S}$(至少只要耦合不要变得过大以至于$N_{nm}$ 不再是正定矩阵). 因此, 正如所要证明的, $\mathscr{S}$是稳定曲面.

我们已经知道所有\marginpar[\flushright{\small[528]\hspace*{5mm}}]{{\small\hspace*{5mm}[528]}}的物理量都可以用$\Lambda$和$\mathscr{G}_{n}(\Lambda )$表示, 并且是$\Lambda$-无关的. 特别是对那$N$个传统重正化耦合和质量, 例如量子电动力学中的$e$和$m_{e}$, 这是正确的. 然而, 我们可以反过来用这个关系, 用传统参量和$\Lambda$来表示$\mathscr{G}_{n}(\Lambda )$. 以这种方式, 我们可以证明通常的重正化程序的合理性:
所有的物理量以一种不依赖截断的方式表示成了传统重正化耦合和质量.

Wilson\,方法在是实际应用中有一些优点. 那就是无需担心子积分和交缠发散; 这个动量截断适用于所有内线.
另外, 超对称理论的一些仅对截断相关的裸耦合成立的不可重整定理告诉我们, 某些耦合是不受辐射修正影响的.\textsuperscript{\cite{20}}

另一方面, Wilson\,方法也有一些缺点. 一个是, 在处理类似量子电动力学这样可重整理论时, 使用\,Wilson\,方法必须放弃方法上的一些特殊的简单性; 一旦开始积掉动量高于某个标度$\Lambda$的粒子, 所得到的有限场论将包含所有的\,Lorentz\,不变且规范不变的相互作用, 但这些相互作用的耦合是依赖$\Lambda$的. (虽然如此, 在能量$E\ll \Lambda$处的物理过程中, 起主导作用的耦合仍将是那些可重整的.) 另外, 截断一般会破坏{\KAI{明显的}}规范不变性, 亦或是明显的\,Lorentz\, 不变性或幺正性. 在建立\,Wilson\,方法的原始背景\ezx 凝聚态物理中, 这些都不是问题, 因为没有人会期望一个真实的凝聚态理论是严格可重整的, 而且不存在一定会被截断破坏的基本物理原理. 实际上, 在晶体中, 光子动量上就{\KAI{存在}}截断, 这一截断由倒格子空间给出.

实际上, 传统方法与\,Wilson\,方法之间的差异是数学上的便利性而非物理解释. 确实, 传统重正化已经提供了一类可调截断; 当我们将我们的答案表示成耦合常数, 而这些耦合常数又被定义成物理振幅在某个$\mu$阶动量处的值(就上节所讨论的标量场论而言), 使积分收敛的抵消就开始在$\mu$阶虚动量上开始起作用了. 相反, Wilson\,方法中$\Lambda$-相关的耦合常数最终必须用可观察质量和电荷表示, 而一旦做到了这点, 所得结果显然与传统方法所得结果相同.

\section[偶然对称性]{偶然对称性{}$^*$\footnote{$^*${}本节或多或少处在本书发展主线之外, 可以在第一次阅读时跳过.}} \label{sec:12.5}
\setcounter{equation}{0}
\marginpar[\flushright
{\raisebox{5.5ex}[0pt]{{\small[529]\hspace*{5mm}}}}]{{\raisebox{5.5ex}[0pt]{\small\hspace*{5mm}[529]}}}

在\,\ref{sec:12.3}\,节中我们看到, 在能量足够低时, 将可重整场论用作自然的近似描述是有一些好处的. 经常遇到的情况是:
可重整条件如此之强以至于有效拉格朗日量自动满足一个或多个对称性, 而这些对称性不是底层理论中的对称性, 因而可以被有效拉格朗日量中被压低的不可重整项破坏. 实际上, 大多数实验上发现的基本粒子物理对称性都是这类``偶然对称性''.

带电轻子的电动力学中的反演与味守恒提供了一个经典例子. 光子和自旋$\frac{1}{2}$, 电荷$-e$的场$\psi_{i}$的规范不变且\,Lorentz\,不变的最一般的可重整拉格朗日密度为
\begin{align}
\mathscr{L} &=-\tfrac{1}{4}Z_{3}F_{\mu \nu}F^{\mu \nu}  \nonumber \\
&\quad-\sum_{ij}Z_{Lij}\bar{\psi}_{Li}[\xxdd+\mi e\xxA]\psi_{Lj}
-\sum_{ij}Z_{Rij}\bar{\psi}_{Ri}[\xxdd+\mi e\xxA]\psi_{Rj}  \nonumber \\
&\quad-\sum_{ij}M_{ij}\bar{\psi}_{Li}\psi_{Rj}-\sum_{ij}M_{ij}^{\dag}\bar{\psi}_{Ri}\psi_{Lj}\:,   \label{12.5.1}
\end{align}%
其中$i,j$对三个轻子味($e,\mu$和$\tau $)求和, $\psi_{iL}$和$\psi_{iR}$是场$\psi_{i}$的左手部分和右手部分, 定义成
\begin{equation}
\psi_{Li}=\tfrac{1}{2}(1+\gamma_{5})\psi_{i}\:, \qquad
\psi_{Ri}=\tfrac{1}{2}(1-\gamma_{5})\psi_{i}\:, \label{12.5.2}
\end{equation}%
而$Z_{L},Z_{R}$和$M$是数值矩阵. 我们没有对轻子味守恒做任何假定, 所以矩阵$Z_{Lij},Z_{Rij}$和$M_{ij}$不需要是对角的. 另外, 我们也没有对$\mathsf{P,C}$ 和$\mathsf{T}$不变性做任何假定, 所以$Z_{L}$和$Z_{R}$之间或$M$和$M^{\dag}$之间没有必然联系. 对这些矩阵的唯一约束来自于拉格朗日密度是实的以及正则反对易关系, 前者要求$Z_{Lij}$和$Z_{Rij}$是厄米的, 后者要求$Z_{Lij}$和$Z_{Rij}$是正定的.

现在假定我们用新的场$\psi_{L}^{\prime}$和$\psi_{R}^{\prime}$来代替轻子场$\psi_{L}$和$\psi_{R}$, 这些%
新的场定义为
\begin{equation}
\psi_{L}=S_{L}\psi_{L}^{\prime}\:,\qquad \psi_{R}=S_{R}\psi_{R}^{\prime}\:,   \label{12.5.3}
\end{equation}%
其中$S_{L,R}$是我们任选的非奇异矩阵. 当用这些新场表示拉格朗日密度时, 拉格朗日密度的形式与方程(\ref{12.5.1})相同,
只不过矩阵被替换为新矩阵\marginpar[\flushright
{\raisebox{-5ex}[0pt]{{\small[530]\hspace*{5mm}}}}]{{\raisebox{-5ex}[0pt]{\small\hspace*{5mm}[530]}}}
\begin{equation}
Z_{L}^{\prime}=S_{L}^{\dag}Z_{L}S_{L}\:, \qquad
Z_{R}^{\prime}=S_{R}^{\dag}Z_{R}S_{R}\:, \qquad
M^{\prime}=S_{L}^{\dag}MS_{R}\:.   \label{12.5.4}
\end{equation}%
我们可以选择$S_{L}$和$S_{R}$以使$Z_{L}^{\prime}=Z_{R}^{\prime}=1$. (取$S_{L,R}=U_{L,R}D_{L,R}$, %
其中$U_{L,R}$是使得正定厄米矩阵$Z_{L,R}$对角化的幺正矩阵, 而$D_{L,R}$是对角矩阵, 矩阵元是$Z_{L,R}$%
的本征值平方根的倒数.)%

现在做另一变换, 变换到轻子场$\psi_{i}^{\prime \prime}$, 定义为
\begin{equation}
\psi_{L}^{\prime}=S_{L}^{\prime}\psi_{L}^{\prime \prime}\:, \qquad
\psi_{R}^{\prime}=S_{R}^{\prime}\psi_{R}^{\prime \prime}\:.
\label{12.5.5}
\end{equation}%
再一次地, 用这些新场和如下矩阵表示的拉格朗日密度取同一形式, 只不过矩阵变成
\begin{equation}
Z_{L}^{\prime \prime}=S_{L}^{\prime \dag}S_{L}^{\prime}\:,\qquad
Z_{R}^{\prime \prime}=S_{R}^{\prime \dag}S_{R}^{\prime}\:,\qquad
M^{\prime \prime}=S_{L}^{\prime \dag}M^{\prime}S_{R}^{\prime}\:. \label{12.5.6}
\end{equation}%
这一次我们让$S_{L,R}^{\prime}$幺正, 从而再次使$Z_{L}^{\prime \prime}=Z_{R}^{\prime \prime}=1$. 我们选择这些幺正矩阵使$M^{\prime \prime}$是实的且对角. (通过极分解原理, 同所有方阵一样, $M^{\prime}$可以表示成$M^{\prime}=VH$,
其中$V$幺正而$H$厄米. 取$S_{L}^{\prime}=S_{R}^{\prime \dag}V^{\dag}$并选$S_{R}^{\prime}$为使$H$%
对角化的幺正矩阵.) 扔掉撇号, 拉格朗日密度现在的形式为
\begin{align}
\mathscr{L} &=-\tfrac{1}{4}Z_{3}F_{\mu\nu}F^{\mu\nu}
-\sum_{i}\bar{\psi}_{Li}[\xxdd+\mi e\xxA]\psi_{Li}
-\sum_{i}\bar{\psi}_{Ri}[\xxdd+\mi e\xxA]\psi_{Ri}  \nonumber \\
&\quad-\sum_{i}m_{i}\bar{\psi}_{Li}\psi_{Ri}-\sum_{i}m_{i}\bar{\psi}_{Ri}\psi_{Li}\:,   \label{12.5.7}
\end{align}%
其中$m_{i}$是实数, 是厄米矩阵$H$的本征值. 最后, 它可以写成更常见的形式
\begin{equation}
\mathscr{L}=-\tfrac{1}{4}Z_{3}F_{\mu\nu}F^{\mu\nu}-\sum_{i}\bar{\psi}_{i}[\xxdd+\mi e\xxA]\psi_{i}
-\sum_{i}m_{i}\bar{\psi}_{i}\psi_{i}\:.  \label{12.5.8}
\end{equation}%
当拉格朗日量取这一形式时, 轻子电动力学的任何可重整拉格朗日量明显将自动保证$\mathsf{P},%
\mathsf{C}$和$\mathsf{T}$守恒, 并且使每个味的轻子数(减去反轻子数)守恒: 电子, $\mu$子和$\tau$子的轻子数守恒.{}$^*$\footnote{$^*${}这是由\,Feinberg(范伯格), Kabir(卡比尔)和我\textsuperscript{\cite{21}}首先证明的. Feinberg\textsuperscript{\cite{22}}更早注意到了, 在只有一种中微子的理论中, 弱作用效应将使过程$\mu \to e+\gamma$有一个可观测的发生率, 这个问题直到发现了第二种中微子才得以解决.} 特别地, 尽管有方程(\ref{12.5.1}), 这一理论并不允许诸如$\mu
\to e+\gamma$这样的过程. 读者或许会担心将方程(\ref{12.5.8})中的$\psi_{i}$(前面写作$\psi
_{i}^{\prime \prime}$)而非方程(\ref{12.5.1})中的$\psi_{i}$作为轻子场是否正确, 前者显然使轻子数守恒, 而后者似乎允许类似$\mu \to e+\gamma$这样的过程. 这样的担心可以暂且搁置; 正如\,\ref{sec:10.3}\,节所强调的, %
不存在{\KAI{仅}}作为电子场或$\mu$子场的场. 事实上, 尽管方\marginpar[\flushright{\small[531]\hspace*{5mm}}]{{\small\hspace*{5mm}[531]}}程(\ref{12.5.1}) 给出了轻子\,1\,辐射衰变到轻子\,2\,的非零矩阵元, 但这个矩阵元{\KAI{不在}}轻子质量壳上, 通过令轻子在质量壳上, 即使采用方程(\ref{12.5.1})进行计算, 我们也会发现所有这种过程的$S$-矩阵元为零.

推导这些结果中的关键是, 在方程(\ref{12.5.1})中, 电荷对于轻子场的左手部分和右手部分是相同的, 换句话说,
轻子场的左手部分和右手部分在电磁规范变换下的变换形式是相同的. 我们将在卷\,\textrm{II}\,中看到, 由于类似的原因, 被称为量子色动力学的强相互作用的现代可重整理论自动满足$\mathsf{C}$守恒、 (除了某些的非微扰效应)$\mathsf{P}$守恒和$\mathsf{T}$守恒, 以及每一夸克味的夸克数(减去反夸克数)守恒. 在卷\,\textrm{II\,}中, 我们还将看到, 出于与这里所描述的电动力学相类似的原因, 弱作用和电磁作用的可重整标准模型的最简单版本将自动满足轻子味守恒(尽管$\mathsf{C}$和$\mathsf{P}$是不守恒的). 来自更高质量标度的不可重整相互作用或许会破坏所有这些守恒律的可能性依然存在.




\subsection*{\bf 习\qquad 题}

 \addcontentsline{toc}{section}{习题}


\begin{KAI}

1. 对于\,2, 3\,和\,6\,维时空中的单个标量场, 列出它的拉格朗日量中所有可重整(或超可重整)的\,Lorentz\,不变项.


2. 说明电子自能的交缠发散在量子电动力学中是如何抵消的.


3. 考察标量场$\phi$和旋量场$\psi$的相互作用哈密顿量为$g\phi\bar{\psi}\psi$的理论. 将标量自能函数$\Pi^{\ast}(q)$的单圈部分写成$p^{\mu}$ 的发散多项式加上一个明显收敛的积分.


4. 假如电子和光子的量子电动力学其实是一个有效场论, 是通过积掉某个质量$M\gg m_{e}$的未知粒子得到的. 假定这个理论满足规范不变性和\,Lorentz\, 不变性, 但没有$\mathsf{C}$, $\mathsf{P}$和$\mathsf{T}$不变性. 拉格朗日量中$1/M$的领头阶不可重整项是什么? 次领头阶又是什么?

\newpage
\ \vspace{-5mm}
\marginpar[\flushright
{\raisebox{-5.5ex}[0pt]{{\small[532]\hspace*{5mm}}}}]{{\raisebox{-5.5ex}[0pt]{\small\hspace*{5mm}[532]}}}
 \end{KAI}

\begin{thebibliography}{99}                                                                                               %


\bibitem {1}F. J. Dyson, {\textit{Phys. Rev.}} {\bf{75}}, 486, 1736 (1949). L. M. Brown\,编辑的\,{\textit{Renormalization}} (Springer-Verlag, New York, 1993)\,中从历史角度给了一个评论. 一个全面的现代处理, %
    参看\,J. Collins, {\textit{Renormalization}} (Cambridge University Press, Cambridge, 1984).
     \addcontentsline{toc}{section}{参考文献}
\bibitem {2}S. Weinberg, {\textit{Phys. Rev.}} {\bf{118}}, 838 (1959). 这个证明仅依赖于\,Feynman\,图在欧几里得动量空间中的被积函数的一般渐进性质, 这种被积函数通过\,Wick\,旋转所有积分围道得到. 通过利用被积函数更详尽的性质, 这个证明被\,Y. Hahn\,和\,W. Zimmerman\,简化了, {\textit{Commun. Math. Phys.}} {\bf{10}}, 330 (1968), 然后被\,W. Zimmerman\,拓展至闵可夫斯基动量空间, {\textit{Commun. Math. Phys.}} {\bf{11}}, 1 (1968).
\bibitem {3}J. D. Bjorken and S. D. Drell, {\textit{Relativistic Quantum Fields}} (McGraw-Hill, New York, 1965): Section 19.10 and 19.11.
\bibitem {4}可参看\,J. Wess and J. Bagger, {\textit{Supersymmetry and Supergravity}} (Princeton University Press, Princeton, 1983), 以及那里引用的原始文献.
\bibitem {5}A. Salam, {\textit{Phys. Rev.}} {\bf{82}}, 217 (1951); \textit{Phys. Rev.} {\bf{84}}, 426 (1951); P. T. Matthews and A. Salam, {\textit{Phys. Rev.}} {\bf{94}}, 185 (1954).
\bibitem {6}N. N. Bogoliubov and O. Parasiuk, {\textit{Acta Math}}. {\bf{97}}, 227 (1957).
\bibitem {7}K. Hepp, {\textit{Comm. Math. Phys.}} {\bf{2}}, 301 (1966). Hepp\,评论道``很难找到两个理论家, 他们对\,[Bogoliubov\,和\,Parasiuk\,]证明中关键步骤的理解是相同的,'' 而\,Hepp\,的论文本身就很难读.
\bibitem {8}W. Zimmerman, {\textit{Comm. Math. Phys.}} {\bf{15}}, 208 (1969). 另见\,W. Zimmerman, 收录于 {\textit{Lecture on Elementry Particles and Quantum Field Theory \yzx  Brandeis University Summer Institute in Theoretical Physics}} (M.I.T. Press, Cambridge, 1970).
\bibitem {9}W. Heisenberg, {\textit{Z. Physik}} {\bf{110}}, 251 (1938).
\bibitem {10}W. Heisenberg, 文献\,[6]\,和\,{\textit{Z. Physik}} {\bf{113}}, 61 (1939).
\bibitem {11}S. Sakata, H.Umezawa, and S. Kamefuchi, {\textit{Prog. Theor. Phys.}} {\bf{7}}, 327 (1952).
\bibitem {12}S. Weinberg, {\textit{Physica}} {\bf{96A}}, 327 (1979).
\bibitem {13}J. Gasser and H. Leutwyler\marginpar[\flushright{\small[533]\hspace*{5mm}}]{{\small\hspace*{5mm}[533]}}, {\textit{Ann. Phys.}} (NY) {\bf{158}}, 142 (1984); {\textit{Nucl. Phys.}} {\bf{B250}}, 465 (1985).
\bibitem {14}J. F. Donoghue, {\textit{Phys. Rev.}} {\bf{D 50}}, 3874 (1994).
\bibitem {15}这一现象能够发生的一种可能方式是通过``渐进安全''现象; 参看\,S. Weinberg, {\textit{General Relativity \yzx  An Einstein Centenary Survey}}, S. W. Hawking\,和\,W. Israel\,编辑(Cambridge University Press, Cambridge, 1979): Section 16.3.
\bibitem {16}H. Euler and B. Kockel, {\textit{Naturwiss}}. {\bf{23}}, 246 (1935); W. Heisenberg and H. Euler, {\textit{Z. Physik}} {\bf{98}}, 714 (1936).
\bibitem {17}J. Halter\,在单圈计算中采用了\,Euler\,等人的有效拉格朗日量, {\textit{Phys. Lett.}} {\bf{B 316}}, 155 (1993).
\bibitem {18}K. G. Wilson, {\textit{Phys. Rev.}} {\bf{B4}}, 3174, 3184 (1971); {\textit{Rev. Mod. Phys.}} {\bf{47}}, 773 (1975).
\bibitem {19}J. Polchinski, {\textit{Nucl. Phys.}} {\bf{B231}}, 269 (1984); {\textit{Recent Directions in Particle Theory \yzx  Proceedings of the 1992 TASI Conference}}\,中的讲稿, J. Harvey\,和\,J. Polchinski\,编辑(World Scientific, Singapore, 1993): p. 235.
\bibitem {20}V. Novikov, M. A. Shifman, A. I. Vainshtein, and V. I. Zakharov, {\textit{Nucl. Phys.}} {\bf{B229}}, 381 (1983); M. A. Shifman and A. I. Vainshtein, {\textit{Nucl. Phys.}} {\bf{B229}}, 456 (1986); 以及那里所引的文献. 另见, M. A. Shifman and A. I. Vainshtein, {\textit{Nucl. Phys.}} {\bf{B359}}, 571 (1991).
\bibitem {21}G. Feinberg, P. Kabir, and S. Weinberg, {\textit{Phys. Rev. Lett.}} {\bf{3}}, 527 (1959).
\bibitem {22}G. Feinberg, {\textit{Phys. Rev.}} {\bf{110}}, 1482 (1958).
\end{thebibliography}
