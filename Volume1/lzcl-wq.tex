\thispagestyle{empty}\pagenumbering{roman}

\


\vspace{85mm}

\begin{center}
{\xinwei\Huge 献给\quad 路易丝}
\end{center}

\newpage

\

\thispagestyle{empty}


\newpage

\thispagestyle{empty}

\setcounter{page}{1}      %页码手动计数


\chapter*{第一卷序言}
\thispagestyle{empty}

 \addcontentsline{toc}{chapter}{第一卷序言}
 \markboth{第一卷序言}{第一卷序言}      %%前双后单书眉

%\setlength{\baselineskip}{16pt}

{\FS 为什么要为量子场论再写一本书? 现在供学生选择的优秀量子场论教科书不下二十本, 其中几本还非常现代. %
一本书只有在内容和观点上有新颖之处才值得写出来.

在内容上, 尽管这本书包含了大量新材料, 但我认为本书最具特色之处是它的普遍性; %
我自始至终都试图在一个尽可能普遍的框架下进行讨论. 这一方面是因为量子场论在它过去极为成功的量子电动力学之外
又发现了新的用武之地, 而更重要的是我认为这种普遍性会有助于防止关键思想被淹没在个别理论的技术细节中. %
当然, 我们会经常用特殊的例子来说明普遍结论,
这些例子是从当代粒子物理或核物理以及量子电动力学中挑选出来的.

然而, 促使我写这本书的主要动机不是其内容而是观点. 我希望这样来展现量子场论: %
让读者最大限度地理解{\KAI 为什么}这个理论采取它现有的形式, %
以及为什么它在现有的形式下能够非常漂亮地描述这个世界.

自~Heisenberg 和~Pauli 关于一般量子场论的第一篇论文以来, 基于我们在电磁学中的经验, %
传统方法把场视作理所当然的存在并量子化它们\ezx
也就是在各种简单场理论中应用正则量子化或路径积分法则.
在第~1~章的历史引言中会看到这些传统方法. 这当然是快速进入这个课题的一种方法, 但对我来说, %
这似乎给那些深入思考的读者留下了太多没有回答的问题. 为什么我们要相信正则量子化或路径积分法则? %
为什么我们要采用文献中的那些简单的场方程和拉格朗日量? 说到这, 为什么要有场呢? 对我来说, %
仅仅诉诸于经验是无法让人完全满意的; 归根究底, 理论物理的目标不仅是描述我们所发现的世界, %
还要去解释\ezx 通过几个简单的基本原理\ezx
为什么世界是这个样子?

本书的观点是: 世界是量子场论的, 这是因为, (除了弦论那种有无穷多种粒子类型的理论外), %
它是调和量子力学原理~(包括集团分解性质) 和狭义相对论原理的唯一途径. 这是我多年所持的观点, %
这一观点最近也越来越为人接受. 近些年, 我们已经学会将我们成功的量子场论, 包括量子电动力学, 视为~``有效场论'', %
一个更底层理论的低能近似, 这个更底层的理论甚至可能都不是个场论而是别的东西, 例如弦论. %
基于此, 在所能到达的能量下刻画物理世界的理论之所以是量子场论的原因是, %
{\KAI 任何}相对论性的量子理论在能量足够低时看起来都像是一个量子场论. %
因此, 通过相对论和量子力学的原理去理解量子场论是非常重要的. %
另外, 现在我们对量子场论中的一些问题, 例如不可重正性和~``平庸性'', 有了不同看法, %
在过去我们把这些理论看作真正的基本理论时, 这些问题曾使我们烦恼不已, 本书的讨论将反映出这些改变. %
本书打算成为有效场论时代的一本量子场论教科书.

相对论和量子力学最直接和确切的结果是粒子态的性质, 所以这里首先考虑的是粒子\ezx
它们在第~2~章中作为非齐次~Lorentz~群在量子力学~Hilbert~空间中表示的元素被引入. %
第~3~章为处理基本动力学问题提供了一个框架: 在遥远的过去给定一个看起来像是自由粒子的某个组合的态, %
那么它在未来看起来像什么? 知道了时间平移生成元, 哈密顿量, %
我们可以通过被称为~$S$-矩阵的跃迁振幅阵列的微扰展开来回答这个问题. %
在第~4~章,
我们将用集团分解原理描述如何用产生和湮没算符来构造时间平移的生成元,
哈密顿量.
然后在第~5~章, 我们回到~Lorentz~不变性, 并阐释~Lorentz~不变性将要求这些产生和湮没算符在因果量子场论中要组合起来. %
作为一个副产品, 我们推导了~$\mathsf{CPT}$ 定理以及自旋与统计间的关系. %
这套体系将在第~6~章被用来推导计算~$S$-矩阵的~Feynman~规则.

直到第~7~章, 我们才开始讨论拉格朗日量和正则体系. 这里引入它们的理由不是因为它们在
物理的其他地方被证明是有用的~(这肯定不是一个让人满意的解释), %
而是因为这套体系使选择能够让~$S$-矩阵满足各种所要求的对称性的相互作用哈密顿量变得很容易. %
特别地, 拉格朗日密度的~Lorentz~不变性确保了满足~Poincar\'{e}~代数的十个算符是存在的, %
并且正如我们在第~3~章证明的, 这是证明~$S$-矩阵的~Lorentz~不变性所需的关键条件. %
最后, 量子电动力学出现在第~8~章. 第~9~章引入路径积分, %
并用它来证明在第~8~章考虑量子电动力学~Feynman~规则时略去的步骤. 相比现今时兴的做法, %
这时引入路径积分显得有些晚, 但是对我来说, 尽管迄今为止路径积分是从一个给定的拉格朗日量快速推导~Feynman~规则的最好方法, %
但它遮掩了植根于这些计算底层的量子力学动机.

卷~I~以第~10~至~14~章为结尾, 这几章是一般场论中辐射修正的计算,
包括圈图的一个简引.
这里的安排同样也有些不同以往; 我们以非微扰方法的一章作为出发点, 这部分是因为, %
这样得到的结果能帮助我们理解场和质量重正化的必要性而不涉及理论是否包含无限大. 第~11~章展示了量子电动力学的经典单圈计算, %
这样既解释有用的计算技巧~(Feynman~参数化, Wick~旋转, 维数和~Pauli-Villars~正规化), %
同时又为如何操作重正化提供了一个具体的例子. 第~12~章中, 在第~11~章获得的经验被推广至所有阶和更一般的理论, %
这一章同时还描述了用于有效场论的不可重正性的现代观点. 第~13~章离开主线讨论了低能或动量平行的无质量粒子引起的特殊问题. %
我们直到第~14~章才第一次看到一个电子在一个外电磁场中的~Dirac~方程, 而在历史上, 这几乎是相对论量子力学中束缚态问题的起点, 我们直到这里才讨论它是因为, 这个方程不应该~(像~Dirac~那样)
被看作~Schr\"{o}dinger~方程的相对论版, %
而是应该视为真正相对论性量子理论\ezx 光子和电子的量子场论\ezx 的一个近似. 这一章最后介绍了对~Lamb~位移的处理, %
给出了理论与实验最新结果间的印证.

读者可能会觉得这里的一些专题放在核物理或者基本粒子物理的教科书中会更合适些, 尤其是第~3~章中的部分内容. %
这样他们就可能用特定的动力学模型而不是对称性和量子力学的一般原理处理这些问题, %
但就我所知, 这些专题通常不是没有被涉及就是涉及得很少. 我遇见过从没有听说过时间反演不变性与末态相移关系的弦论学家, %
也遇到过不理解共振为什么服从~Breit-Wigner~公式的核物理学家. %
所以我把这些内容放在了开始的几章中.

卷~II~将讨论近年来重振量子场论的进展: 非阿贝尔规范理论, 重正化群,
破缺对称性, 反常, 瞬子, 等等.

文献方面, 我尽力兼顾量子场论的经典文献和本书涉及但未详细讨论的专题的文献. 我并不总是知道本书涉及的材料是谁提出的, %
而且, 没有标注引用的内容并不是说这个内容是本书原创的. 不过其中一些确实是原创的. 在一些地方, %
我希望我对原始文献或标准教科书中的处理有所改进, 例如对称性算符不是幺正就是反幺正的证明; 超选择定则的讨论; %
与反演的非常规表示相关的粒子简并的分析; 集团分解原理的应用;
约化公式的推导; 外场近似的推导; 甚至~Lamb~位移的计算.

我为除第一章以外的其他各章准备了习题. 其中的一些习题旨在为本章讲到的技术提供一个练习; %
其他的则旨在将本章的结果推广到更普遍的理论.

在讲授量子场论的过程中, 我发现本书的前两卷各自为一年的研究生课程提供了足够的材料. %
我希望这本书对熟悉非相对论力学和经典电动力学的学生是适合的. 我假定读者具有复分析和矩阵代数的基本知识, %
而涉及群论和拓扑的概念时会随时进行解释.

这本书不是针对那些希望立刻就能计算电弱和强相互作用的标准模型的~Feynman~图的学生. %
这本书也不是针对那些试图探寻更高层次数学严谨性的人. 诚然, %
书中部分内容对缺乏数学严谨性的缺失会让那些对此有较高要求的读者难过.
但相反, 对那些希望弄清{\KAI 为什么}量子场论是它现在的样子的物理学家和学生, %
从而使他们对物理学上任何超出我们理解的新的进展做好准备,
我希望这本书能够适合他们的需要.


\subsection*{* * *}

这本书的很多材料是这些年来从与许多物理学家的交流中学到的,
无法一一列举了. 然而我必须特别致谢~Sidney Coleman,
以及我在德克萨斯大学的同事: Arno Bohm, Luis Boya, Phil Candelas,
Bryce DeWitt, Cecile DeWitt-Morette, Jacques Distler, Willy
Fischler, Josh Feinberg, Joaquim Gomis, Vadim Kaplunovsky, Joe
Polchinski, 以及~Paul Shapiro. 我感谢~Gerry Holton, Arthur
Miller~和~Sam Schweber~帮我准备历史简介. 还要感谢~Alyce
Wilson~帮我准备了插图并帮我录入~LATEX~文稿, 直到我学会使用它,
感谢~Terry Riley~帮我找了无数的书和文章.
我感谢许多学生和同事指出了本卷第一次印刷中的各种错误, 特别是~Stephen
Adler, Hideaki Aoyama, Kevin Cahill, Amir Kashani-Poor, Achim
Kupferoth, Michio Masujima, Herbert Neuberger, Fabio Siringo,
以及~San Fu Tuan. 感谢剑桥出版社的~Maureen Storey~和~Alison
Woollatt~帮助准备此书的出版, 特别是我的编辑~Rufus
Neal~提出的中肯建议.

\

\hspace*{\fill} 斯蒂芬~$\cdot$ 温伯格

\hspace*{\fill} 德克萨斯州, 奥斯汀

\hspace*{\fill} 1994 年~10 月

}


\chapter*{符号约定}

\thispagestyle{empty}
\setcounter{page}{1}

 \addcontentsline{toc}{chapter}{符号约定}
 \markboth{符号约定}{符号约定}      %%前双后单书眉


拉丁指标$i,j,k$等一般取遍三维空间坐标指标, 通常取$1,2,3.$
%\vspace{2mm}

希腊指标$\mu,\nu$等一般取遍四维时空坐标指标$1,2,3,0$, 其中$x^{0}$是时间坐标.
%\vspace{2mm}

重复指标一般表示求和, 除非另有说明.
%\vspace{2mm}

时空度规$\eta_{\mu\nu}$是对角的, 其对角元为$\eta_{11}=\eta_{22}=\eta
_{33}=1,\eta_{00}=-1$.
%\vspace{2mm}

达朗贝尔算符定义为$\square\equiv\eta^{_{\mu\nu}}\partial^{2}/\partial
x^{\mu}\partial x^{\nu}=\nabla^{2}-\partial^{2}/\partial
t^{2}$, 其中$\nabla^{2}$是拉普拉斯算符$\partial^{2}/\partial
x^{i}\partial x^{i}$.
%\vspace{2mm}

莱维\bzx 齐维塔张量$\epsilon^{\mu\nu\rho\sigma}$定义为全反对称量, 并有%
$\epsilon^{0123}=+1$.
%\vspace{2mm}

空间三矢由粗体字母标记.
%\vspace{2mm}

任意矢量上的``帽子''代表相应的单位矢量: 因%
此$\hat{\bv}\equiv \bv/\lvert \bv\rvert$.
%\vspace{2mm}

任意量上加点代表该量对时间的导数.
%\vspace{2mm}

狄拉克矩阵$\gamma_{\mu}$的定义满足$\gamma_{\mu}\gamma_{\nu}+\gamma_{\nu}%
\gamma_{\mu}=2\eta_{\mu\nu}$. 并且$\gamma_{5}=\mi\gamma_{0}\gamma_{1}%
\gamma_{2}\gamma_{3}$, $\beta=\mi\gamma^{0}$.%\vspace{2mm}

阶跃函数$\theta(s)$: 当$s>0$时为$1$, $s<0$时为$0$.
%\vspace{2mm}

矩阵或矢量$A$的复共轭、转置、厄米共轭分别记为$A^{\ast
}$, $A^{\mathrm{T}}$以及$A^{\dag}=A^{\ast \mathrm{T}}$. 在强调一个算符的矩阵或矢量是非转置时, 我们用星号, 除此之外, 算符$O$的厄米共轭记为$O^{\dag}$. 在方程末尾的\,+H.c.\,或\,c.c.\,表示前面几项的厄米共轭或复共轭. Dirac\,旋量$u$上加横线定义为$\bar{u}=u^{\dag}\beta$.
%\vspace{2mm}

除了第一章, 我们取$\hbar$和$c$为单位\,1. 这样$-e$是电子的有理化电荷, 精细结构常数%
是$\alpha=e^{2}/4\uppi\simeq1/137$.%\vspace{2mm}

引用数据末尾括号中的数字给出了引用数据末尾数字的不确定度, 在没有额外指明的情况下, 实验数据取自`Review of Particle Properties,' {\textit{Phys. Rev.}} {\bf{D50}}, 1173 (1994).

\newpage
\
\thispagestyle{empty}