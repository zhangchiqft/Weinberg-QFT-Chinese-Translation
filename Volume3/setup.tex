

%==========================各种格式的设置=====================================

%% 手动添加公式和文本的间隔
\xeCJKsetup{CJKecglue={}}

%%%%%%%%%%%%%%%%%%%%%%%%%%%%%%%%重定义字体、字号命令 %%%%%%%%%%%%%%%%%%%%%%%%%
\newcommand{\song}{\CJKfamily{zhsong}}    % 宋体   (Windows自带simsun.ttf)
\newcommand{\fs}{\CJKfamily{zhfs}}        % 仿宋体 (Windows自带simfs.ttf)
\newcommand{\kai}{\CJKfamily{zhkai}}      % 楷体   (Windows自带simkai.ttf)
\newcommand{\hei}{\CJKfamily{zhhei}}      % 黑体   (Windows自带simhei.ttf)
\newcommand{\li}{\CJKfamily{zhli}}        % 隶书   (Windows自带simli.ttf)
\newcommand{\you}{\CJKfamily{zhyou}}      % 幼圆   (Windows自带simyou.ttf)
\newcommand{\chuhao}{\fontsize{42pt}{\baselineskip}\selectfont}     % 字号设置
\newcommand{\xiaochuhao}{\fontsize{36pt}{\baselineskip}\selectfont} % 字号设置
\newcommand{\yichu}{\fontsize{32pt}{\baselineskip}\selectfont}      % 字号设置
\newcommand{\yihao}{\fontsize{28pt}{\baselineskip}\selectfont}      % 字号设置
\newcommand{\erhao}{\fontsize{21pt}{\baselineskip}\selectfont}      % 字号设置
\newcommand{\xiaoerhao}{\fontsize{18pt}{\baselineskip}\selectfont}  % 字号设置
\newcommand{\sanhao}{\fontsize{15.75pt}{\baselineskip}\selectfont}  % 字号设置
\newcommand{\xiaosanhao}{\fontsize{15pt}{\baselineskip}\selectfont} % 字号设置
\newcommand{\sihao}{\fontsize{14pt}{\baselineskip}\selectfont}      % 字号设置
\newcommand{\xiaosihao}{\fontsize{12pt}{\baselineskip}\selectfont}  % 字号设置
\newcommand{\wuhao}{\fontsize{10.5pt}{\baselineskip}\selectfont}    % 字号设置
\newcommand{\xiaowuhao}{\fontsize{9pt}{\baselineskip}\selectfont}   % 字号设置
\newcommand{\liuhao}{\fontsize{7.875pt}{\baselineskip}\selectfont}  % 字号设置
\newcommand{\qihao}{\fontsize{5.25pt}{\baselineskip}\selectfont}    % 字号设置
%%%%%%%%%  END %%%%%%%%%%%%%%%%%%%%%%%%%%%%%%%%%%%%%%%%%%%%%%%%%%%%%%%%%%%%%%%




%-------------------- 用于中文段落缩进和正文版式 ------------------%

\setlength{\hoffset}{0cm}
\setlength{\voffset}{0cm}
\setlength{\parindent}{2em}                 % 首行两个汉字的缩进量
\setlength{\parskip}{3pt plus1pt minus1pt}  % 段落之间的竖直距离
\renewcommand{\baselinestretch}{1.2}        % 定义行距
\setlength{\abovecaptionskip}{5pt}
\setlength{\belowcaptionskip}{5pt}
\setlength{\abovedisplayskip}{8.5pt plus 3pt minus 4pt}
\setlength{\belowdisplayskip}{8.5pt plus 3pt minus 4pt}
\setlength{\abovedisplayshortskip}{20pt plus 2pt}
\setlength{\belowdisplayshortskip}{4pt plus 2pt minus 2pt}
%------------------------------------------------------------------%

%===============================数学环境的设置======================%

\renewcommand{\proof}{\kai{证明}}%




%===========================章节的标题格式====================================%
\titleformat{\chapter}{\centering\hei\huge}{\chaptertitlename}{1em}{}
\titlespacing{\chapter}{0pt}{3.5ex plus .1ex minus .2ex}{10\wordsep}
\titleformat{\section}{\centering\hei\Large}{\thesection}{1em}{}
\titlespacing{\section}{2em}{3.5ex plus .1ex minus .2ex}{1.5\wordsep}
\titleformat{\subsection}{\centering\kai\large}{}{0em}{}
\titlespacing{\subsection}{2em}{1.5ex plus .1ex minus .2ex}{\wordsep}


\renewcommand{\captionfont}{\small}
%===============================================================================%

%+++++++++++++++++++++++++罗列环境的设置++++++++++++++++++++++++++++++++++++++++%

\setenumerate{fullwidth,itemindent=\parindent,listparindent=\parindent,itemsep=0ex,partopsep=0pt,parsep=0ex}

%+++++++++++++++++++++++++++++++++++++++++++++++++++++++++++++++++++++++++++++++%

%+++++++++++++++++++++++数学字体的设置++++++++++++++++++++++++++++++++++++++++%
\newcommand{\me}{\mathrm{e}}  % for math e
\newcommand{\mi}{\mathrm{i}} % for math i
\newcommand{\dif}{\mathrm{d}} %for differential operator d
\newcommand{\cvec}[1]{\!\vec{\,#1}}
\DeclareSymbolFont{lettersA}{U}{txmia}{m}{it}
 \DeclareMathSymbol{\piup}{\mathord}{lettersA}{25}
 \DeclareMathSymbol{\muup}{\mathord}{lettersA}{22}
 \DeclareMathSymbol{\deltaup}{\mathord}{lettersA}{14}
 \newcommand{\uppi}{\piup}
\newcommand{\rmQ}{\text{\fontfamily{ptm}\selectfont{Q}}}


%\newcommand{\mpi}{\uppi} % for math pi
%\newcommand\uppi{\text{\gr p}}
%========================修改脚注的样式==========================================
\numberwithin{footnote}{section}
\makeatletter
\renewcommand{\thefootnote}{\fnsymbol{footnote}}
\renewcommand{\@fnsymbol}[1]{\ifcase#1\or *\or **\or \dag\or \dag\dag\or
 \ddag \or \ddag\ddag \or  $\natural$ \or  $\natural\natural$\or \|\| \or \ding{172} \else\@ctrerr\fi\relax}
\makeatletter
%+++++++++++++++++++++++++++++++++++++++++++++++++++++++++++++++++++++++++++++++%





%===========================修改参看文献格式=================================
\makeatletter
\def\@cite#1#2{\textsuperscript{[{#1\if@tempswa , #2\fi}]}}
\makeatother
%++++++++++++++++++++++++++++++++++++++++++++++++++++++ 