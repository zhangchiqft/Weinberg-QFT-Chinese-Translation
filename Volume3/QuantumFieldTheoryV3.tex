
% Authors:   张驰       天津大学理学院物理系2011级本科
%           李文都  lwdallen@126.com  天津大学理学院物理系理论物理13级博士

\documentclass[11pt,a4paper,UTF8]{book}%文档类型为书籍类
\usepackage{fontspec}
    \setmainfont{Alegreya}
    \newfontfamily\AlegreyaSC{Alegreya SC}
    \renewcommand{\textsc}[1]{{\AlegreyaSC#1}}




%=====================�����package��===============================

\usepackage{amsmath}                      % AMSLaTeX����������ų�����Ư���Ĺ�ʽ
\usepackage{amsfonts}                     % AMS�ṩ����ѧ���ŵ��ֿ�
\usepackage{amssymb}                      % ��ѧ������������
\usepackage{amsthm}                       % ��ѧ��������
%\usepackage{extarrows}                    %����ͷ��ʹ��
\usepackage[c]{esvect}                       % ʸ����ͷ�����¶���
\usepackage{CJK}                          % ������
\usepackage{CJKnumb}                      % ������������ת�����������ֵ�����
\usepackage{wasysym}                      % ֧��ֱ���Ļ��ֺ�
\usepackage{indentfirst}                  % �����������
\usepackage{fancyhdr}                     % fancyhdr���֧���Զ���ҳü��ҳ��
%\usepackage{lmodern}
%\usepackage[Symbol]{upgreek}             % ֱ��ϣ����ĸ
\usepackage{bm}                           % ������ѧ��ʽ�еĺ�б��
\usepackage{titlesec}                     % ���Ʊ���ĺ��
\usepackage{titletoc}                     % ����Ŀ¼�ĺ��
\usepackage{color}                        % ֧�ֲ�ɫ
\usepackage{float}                        % ֧��ǿ��ִ�и������λ��
\usepackage{enumitem}                     % ���л����޸ı�ǩ��ʽ�ĺ��
%\usepackage{enumerate}
 \usepackage{hypbmsec}                     % ����������ǩ�б�����ʾ����
\usepackage[CJKbookmarks=true,
            unicode,
            hyperfootnotes=true,
            bookmarks=true,
            colorlinks,
            citecolor=blue]{hyperref}     % ������ǩ

\usepackage{graphicx}                     % ֧�ֲ�ͼ����
\usepackage{psfrag}                       % �ı�ת��tex
%\usepackage{CJKutf8}
\usepackage{epstopdf}                     % epsͼƬת����pdf
%\usepackage[centerlast]{subfigure}        % ֧����ͼ %centerlast �������һ���Ƿ����
\usepackage{tabularx}                     % �Զ����ñ�����п�
\usepackage{caption}                     % ���������ĸ�ʽ����
\DeclareCaptionLabelSeparator{twospace}{\ ~}
\captionsetup{labelsep=twospace}
\usepackage{ccmap}                        % PDFLaTeX�������ɵ�PDF�ļ����и���/ճ�������������
\usepackage{amsthm}                       % ֧���Զ��嶨�������ĺ��
\usepackage[top=1.35in,bottom=1.15in,left=1.00in,right=0.9in]{geometry}
                                          % ֧�ְ���ߴ�����
\usepackage[sectionbib,rootbib]{chapterbib}
\usepackage{cite}     %�޸IJο�����������ʽ
%\usepackage{footmisc}     % ��ע��ʽ
\usepackage{slashed}      %Dirac �
\usepackage{mathrsfs}   %֧�ֻ�дӢ��
\DeclareMathAlphabet{\mathscrbf}{OMS}{mdugm}{b}{n}
\usepackage{exscale}
\usepackage{relsize}
\usepackage{verbatim}   % ����ע��
\usepackage{subfig}                    % ֧����ͼ
\usepackage{multirow}
\usepackage{array}
\usepackage{tabularx}
\usepackage{pifont}    %��Ȧ����
\usepackage[multiple]{footmisc}
\usepackage{manyfoot,perpage}
\usepackage{booktabs}   %�������������������ݵļ��
\DeclareNewFootnote{B}[alph]
%\usepackage{floatrow}


 %调入package



%==========================各种格式的设置=====================================

%% 手动添加公式和文本的间隔
\xeCJKsetup{CJKecglue={}}

%%%%%%%%%%%%%%%%%%%%%%%%%%%%%%%%重定义字体、字号命令 %%%%%%%%%%%%%%%%%%%%%%%%%
\newcommand{\song}{\CJKfamily{zhsong}}    % 宋体   (Windows自带simsun.ttf)
\newcommand{\fs}{\CJKfamily{zhfs}}        % 仿宋体 (Windows自带simfs.ttf)
\newcommand{\kai}{\CJKfamily{zhkai}}      % 楷体   (Windows自带simkai.ttf)
\newcommand{\hei}{\CJKfamily{zhhei}}      % 黑体   (Windows自带simhei.ttf)
\newcommand{\li}{\CJKfamily{zhli}}        % 隶书   (Windows自带simli.ttf)
\newcommand{\you}{\CJKfamily{zhyou}}      % 幼圆   (Windows自带simyou.ttf)
\newcommand{\chuhao}{\fontsize{42pt}{\baselineskip}\selectfont}     % 字号设置
\newcommand{\xiaochuhao}{\fontsize{36pt}{\baselineskip}\selectfont} % 字号设置
\newcommand{\yichu}{\fontsize{32pt}{\baselineskip}\selectfont}      % 字号设置
\newcommand{\yihao}{\fontsize{28pt}{\baselineskip}\selectfont}      % 字号设置
\newcommand{\erhao}{\fontsize{21pt}{\baselineskip}\selectfont}      % 字号设置
\newcommand{\xiaoerhao}{\fontsize{18pt}{\baselineskip}\selectfont}  % 字号设置
\newcommand{\sanhao}{\fontsize{15.75pt}{\baselineskip}\selectfont}  % 字号设置
\newcommand{\xiaosanhao}{\fontsize{15pt}{\baselineskip}\selectfont} % 字号设置
\newcommand{\sihao}{\fontsize{14pt}{\baselineskip}\selectfont}      % 字号设置
\newcommand{\xiaosihao}{\fontsize{12pt}{\baselineskip}\selectfont}  % 字号设置
\newcommand{\wuhao}{\fontsize{10.5pt}{\baselineskip}\selectfont}    % 字号设置
\newcommand{\xiaowuhao}{\fontsize{9pt}{\baselineskip}\selectfont}   % 字号设置
\newcommand{\liuhao}{\fontsize{7.875pt}{\baselineskip}\selectfont}  % 字号设置
\newcommand{\qihao}{\fontsize{5.25pt}{\baselineskip}\selectfont}    % 字号设置
%%%%%%%%%  END %%%%%%%%%%%%%%%%%%%%%%%%%%%%%%%%%%%%%%%%%%%%%%%%%%%%%%%%%%%%%%%




%-------------------- 用于中文段落缩进和正文版式 ------------------%

\setlength{\hoffset}{0cm}
\setlength{\voffset}{0cm}
\setlength{\parindent}{2em}                 % 首行两个汉字的缩进量
\setlength{\parskip}{3pt plus1pt minus1pt}  % 段落之间的竖直距离
\renewcommand{\baselinestretch}{1.2}        % 定义行距
\setlength{\abovecaptionskip}{5pt}
\setlength{\belowcaptionskip}{5pt}
\setlength{\abovedisplayskip}{8.5pt plus 3pt minus 4pt}
\setlength{\belowdisplayskip}{8.5pt plus 3pt minus 4pt}
\setlength{\abovedisplayshortskip}{20pt plus 2pt}
\setlength{\belowdisplayshortskip}{4pt plus 2pt minus 2pt}
%------------------------------------------------------------------%

%===============================数学环境的设置======================%

\renewcommand{\proof}{\kai{证明}}%




%===========================章节的标题格式====================================%
\titleformat{\chapter}{\centering\hei\huge}{\chaptertitlename}{1em}{}
\titlespacing{\chapter}{0pt}{3.5ex plus .1ex minus .2ex}{10\wordsep}
\titleformat{\section}{\centering\hei\Large}{\thesection}{1em}{}
\titlespacing{\section}{2em}{3.5ex plus .1ex minus .2ex}{1.5\wordsep}
\titleformat{\subsection}{\centering\kai\large}{}{0em}{}
\titlespacing{\subsection}{2em}{1.5ex plus .1ex minus .2ex}{\wordsep}


\renewcommand{\captionfont}{\small}
%===============================================================================%

%+++++++++++++++++++++++++罗列环境的设置++++++++++++++++++++++++++++++++++++++++%

\setenumerate{fullwidth,itemindent=\parindent,listparindent=\parindent,itemsep=0ex,partopsep=0pt,parsep=0ex}

%+++++++++++++++++++++++++++++++++++++++++++++++++++++++++++++++++++++++++++++++%

%+++++++++++++++++++++++数学字体的设置++++++++++++++++++++++++++++++++++++++++%
\newcommand{\me}{\mathrm{e}}  % for math e
\newcommand{\mi}{\mathrm{i}} % for math i
\newcommand{\dif}{\mathrm{d}} %for differential operator d
\newcommand{\cvec}[1]{\!\vec{\,#1}}
\DeclareSymbolFont{lettersA}{U}{txmia}{m}{it}
 \DeclareMathSymbol{\piup}{\mathord}{lettersA}{25}
 \DeclareMathSymbol{\muup}{\mathord}{lettersA}{22}
 \DeclareMathSymbol{\deltaup}{\mathord}{lettersA}{14}
 \newcommand{\uppi}{\piup}
\newcommand{\rmQ}{\text{\fontfamily{ptm}\selectfont{Q}}}


%\newcommand{\mpi}{\uppi} % for math pi
%\newcommand\uppi{\text{\gr p}}
%========================修改脚注的样式==========================================
\numberwithin{footnote}{section}
\makeatletter
\renewcommand{\thefootnote}{\fnsymbol{footnote}}
\renewcommand{\@fnsymbol}[1]{\ifcase#1\or *\or **\or \dag\or \dag\dag\or
 \ddag \or \ddag\ddag \or  $\natural$ \or  $\natural\natural$\or \|\| \or \ding{172} \else\@ctrerr\fi\relax}
\makeatletter
%+++++++++++++++++++++++++++++++++++++++++++++++++++++++++++++++++++++++++++++++%





%===========================修改参看文献格式=================================
\makeatletter
\def\@cite#1#2{\textsuperscript{[{#1\if@tempswa , #2\fi}]}}
\makeatother
%++++++++++++++++++++++++++++++++++++++++++++++++++++++    %调入全局设置

\graphicspath{{figures/}}  % 定义所有的 .eps 文件在 figures 子目录下


%======================= 文档开始============================%


\begin{document}
	%\hypersetup{CJKbookmarks=true}
% \begin{CJK*}{GBK}{song}%开始CJK
% \CJKindent



%!!!!!!!!!!!!!!!!!!标题页!!!!!!!!!!!!!!!!!!!!!!!!!!!!!!!!%

\title{\Huge\bfseries The Quantum Theory of Fields \\ \Large \bfseries Volume III Supersymmetry}

\author{Steven Weinberg(著)\\University of Texas at Austin \\
\\  张驰(译)\thanks{此版仅为译稿, 未经校对, 如有任何翻译录入错误, 请联系zhangchi\_110@163.com} }
\date{}



\maketitle
%!!!!!!!!!!!!!!!!!!!!!!!!!!!!!!!!!!!!!!!!!!!!!!!!!!!!!!!!%
\setcounter{tocdepth}{1}
\clearpage{\pagestyle{empty}\cleardoublepage} \setcounter{page}{1}
\pagenumbering{roman}

\renewcommand{\contentsname}{目\qquad 录}
%\pdfbookmark[0]{目~~录}{mulu}
\titlecontents{chapter} [6em]{\sihao\song}
                             {\contentslabel{3em}} {}{\titlerule*[1em]{$\cdot$}\contentspage}%目录中章的显示格式
\titlecontents{section} [6em]{\xiaosihao\song}
                             {\contentslabel{3em}}{} {\titlerule*[1em]{$\cdot$}\contentspage}

\tableofcontents

\setcounter{tocdepth}{2}



\newpage\clearpage{\pagestyle{empty}\cleardoublepage}








% %======================= 自定义页眉与页脚的格式============================%
% \fancyhf{}
% \fancyhead[CE]{\leftmark}
% \fancyhead[CO]{\rightmark}
% \fancyhead[RO,LE]{$\cdot$\ \thepage\ $\cdot$}
% \renewcommand{\headrulewidth}{0.8pt}
% \pagestyle{fancy}
% \renewcommand{\chaptermark}[1]{\markboth{第\,\thechapter\,章\ #1}{}}
% \renewcommand{\sectionmark}[1]{\markright{\thesection\ #1}{}}
% %==========================================================================%


%===============================数学环境的设置======================%

\renewcommand{\proof}{\kai{证明:}}%
\newtheoremstyle{mystyle}{3pt}{3pt}{\kai}{0cm}{\hei}{}{1em}{}  \theoremstyle{mystyle}
\theoremstyle{mystyle}
\newtheorem{assumption}{假设}




%======================= 自定义页眉与页脚的格式============================%
\fancyhf{}
\fancyhead[CE]{\leftmark}
\fancyhead[CO]{\rightmark}
\fancyhead[RO,LE]{$\cdot$\ \thepage\ $\cdot$}
\renewcommand{\headrulewidth}{0.8pt}
\pagestyle{fancy}
\renewcommand{\chaptermark}[1]{\markboth{第\,\thechapter\,章\ #1}{}}
\renewcommand{\sectionmark}[1]{\markright{\thesection\ #1}{}}
%==========================================================================%

%=======================定义各种中文标题===================================%
\renewcommand{\chaptername}{第~\arabic{chapter} 章}
%\renewcommand{\thesubsection}{\chinese{\value{subsection}}、}
\renewcommand\figurename{图}
\renewcommand\tablename{\hei 表}
%\renewcommand{\captionlabeldelim}{\ }%去掉图标签后面的冒号


%@@@@@@@@@@@@@@@@@@@@@@@更改计数器值及各计数器的从属关系设置@@@@@@@@@@@@@@@@%

%\setcounter{page}{155}%更改page的计数器值
\setcounter{chapter}{23}%更改chapter的计数器值
\numberwithin{equation}{section}%公式计数器从属于节计数器
\numberwithin{figure}{chapter}%图计数器从属于节计数器
\numberwithin{table}{chapter}%表计数器从属于章计数器

%@@@@@@@@@@@@@@@@@@@@@@@@@@@@@@@@@@@@@@@@@@@@@@@@@@@@@@@@@@@@@@@@@@@@@@@@@@@%
\phantomsection
\markboth{记~号}{记~号}
%\pdfbookmark[0]{记号}{记号}
\addcontentsline{toc}{chapter}{记~~号}

\chapter*{符号约定}

对于超对称方面的书籍, 在选择符号约定上有一个大问题: 是选择旋量的2分量表示还是4分量表示. 超对称的标准教科书选用的是2分量Weyl表示. 而在本书中, 除了在最开始构建超对称代数和超对称多重态时, 我将使用4分量Dirac表示, 这样做的原因是, 这会使得那些在粒子物理唯像学和模型构建上进行工作的物理学家更容易接受本书. 看到超对称专家独立团体的成长是一件很遗憾的事, 超对称专家们彼此之间可以很好地交流, 但是却由于符号与粒子理论家的大团体割裂开了.

将4分量的表示转换成2分量表示没有什么太大困难. 贯穿本书始终, 在我们所用的Dirac矩阵表示中, $\gamma_{5}$是对角矩阵, 主对角元是$+1$, $+1$, $-1$和$-1$, 任何4分量Majorana旋量$\psi_{\alpha}$(例如超对称生成元$Q_{\alpha}$, 超空间坐标$\theta_{\alpha}$, 或者超导数$\mathscr{D}_{\alpha}$)都可以写成两分量旋量$\chi_{a}$
\begin{equation*}
\psi =\left(
\begin{array}{c}
e\chi^{\ast} \\ \chi
\end{array}\right) \:,
\end{equation*}
其中$e$是$2\times 2$反对称矩阵, 并有$e_{12}=+1$. 在其它教科书中, 两分量旋量$\chi_{a}$通常记做$\psi_{a}^{\ast}=\bar{\psi}_{\dot{a}}$, 而$(e\chi^{\ast})_{a}$则记做$\psi_{a}$. 在第26章的附录中, 我们将给出4分量Majorana旋量有用性质的总结.
\\
\\
下面是本书所用的其它符号约定:

拉丁指标$i,j,k$等一般取遍三维空间坐标指标, 通常取做$1,2,3.$ 在有特殊说明的情况下, 它们取遍值1,2,3,4, 其中$x^{4}=\mi t$.
\\

希腊指标$\mu,\nu$等, 从希腊字母表的中间开始, 一般取遍四维时空坐标指标$1,2,3,0$, 其中$x^{0}$是时间坐标. 当需要在一般坐标系的时空指标和定域惯系的时空指标之间做出区分时, 我们会用$\mu,\nu$等来标记前者, 而用$a,b$等来标记后者.
\\

希腊指标$\alpha,\beta$等, 从希腊字母表的开头开始, (除了第24章)一般取遍4分量旋量的所有分量. 为了避免混淆, 我在这里采用与卷II不同的符号约定, 用大写字母$A,B$等来标记对称性代数的生成元. 2分量旋量的分量用指标$a,b$等进行标记. 特别地, 4分量超对称生成元记做$Q_{\alpha}$, 而2分量生成元($Q_{\alpha}$的后两个分量)记做$\,\mathcal{Q}_{a}$.
\\

重复指标一般表示求和, 除非另有说明.
\\

时空度规$\eta_{\mu\nu}$是对角的, 其元素为$\eta_{11}=\eta_{22}=\eta
_{33}=1,\:\eta_{00}=-1$.
\\

达朗贝尔算符定义为$\square\equiv\eta^{_{\mu\nu}}\partial^{2}/\partial
x^{\mu}\partial x^{\nu}=\mathbf{\nabla}^{2}-\partial^{2}/\partial
t^{2} $, 其中$\mathbf{\nabla}^{2}$是拉普拉斯算符$\partial^{2}/\partial
x^{i}\partial x^{i}$.
\\

列维-奇维塔张量$\epsilon^{\mu\nu\rho\sigma}$定义为全反对称量, 并有%
$\epsilon^{0123}=+1$.
\\

Dirac矩阵$\gamma_{\mu}$的定义使得$\gamma_{\mu}\gamma_{\nu}+\gamma_{\nu}\gamma_{\mu}=2\eta_{\mu\nu}$. 另外, $\gamma_{5}=\mi\gamma_{0}\gamma_{1}\gamma_{2}\gamma_{3}$, 且$\beta=\mi\gamma^{0}=\gamma_{4}$. 当需要显式矩阵时, 它们由分块矩阵给出
\begin{equation*}
\gamma^{0} =-\mi\left[
\begin{array}{cc}
\bm{0} & \bm{1} \\ \bm{1} & \bm{0}
\end{array}\right] \:, \qquad\qquad
\bm{\gamma} =-\mi\left[
\begin{array}{cc}
\bm{0} & \bm{\sigma} \\ -\bm{\sigma} & \bm{0}
\end{array}\right] \:,
\end{equation*}
其中$\bm{1}$是$2\times 2$单位矩阵, $\bm{0}$是$2\times 2$零矩阵, $\bm{\sigma}$的3个分量是通常的Pauli矩阵
\begin{equation*}
\sigma^{1} = \left(
\begin{array}{cc}
0 & 1 \\ 1 & 0
\end{array}\right) \:, \qquad
\sigma_{2} = \left(
\begin{array}{cc}
0 & -\mi \\ \mi & 0
\end{array}\right) \:, \qquad
\sigma_{3} = \left(
\begin{array}{cc}
1 & 0 \\ 0 & -1
\end{array}\right) \:.
\end{equation*}
我们还会经常使用$4\times 4$分块矩阵
\begin{equation*}
\gamma_{5} =\left[
\begin{array}{cc}
\bm{1} & \bm{0} \\ \bm{0} & -\bm{1}
\end{array}\right] \:, \qquad\qquad
\epsilon =-\mi\left[
\begin{array}{cc}
e & \bm{0} \\ \bm{0} & e
\end{array}\right] \:,
\end{equation*}
其中的$e$依旧是反对称$2\times 2$矩阵$\mi\sigma_{2}$. 例如, 我们对4分量Majorana旋量$s$的相位约定可以表示成$s^{\ast}=-\beta\,\gamma_{5}\,\epsilon\,s$.
\\

阶跃函数$\theta(s)$: 当$s>0$时为$1,\:s<0$时为$0$.
\\

矩阵或矢量$A$的复共轭、转置、厄米伴随分别记为$A^{\ast
}$、$A^{T}$以及$A^{\dag}=A^{\ast T}$. 算符$O$的厄米%
共轭记为$O^{\dag}$. 有时会用剑号$\dag$来标记矩阵的转置是通过算符的厄米伴随或数的复共轭构建的, 这时会用星号$\ast$来表示算符的厄米伴随或一个数的复共轭. 在方程末%
尾的+H.c.或c.c.表示前面几项的厄米共轭或复共轭. 4分量旋量$u$上加横线定义%
为$\bar{u}=u^{\dag}\beta$.
\\

使用的单位制通常取$\hbar$和$c$为1. 自始至终$-e$是电子的有理化电荷, 使得精细结构常数%
是$\alpha=e^{2}/4\uppi\simeq1/137$. 温度处在能量单位制下, 而Boltzmann常数取成1.
\\

引用数据末尾括号中的数字给出了引用数据末尾数字的不确定度, 在没有额外指明的情况下, 实验数据取自`Review of Particle Properties,' The Particle Data Group,
{\textit{European Physics Journal C}} {\bf{3}}, 1 (1998).



\newpage\clearpage{\pagestyle{empty}\cleardoublepage}
\setcounter{page}{1}%单独从 1 开始编页码
\pagenumbering{arabic}


%第二十四章 历史介绍
\chapter{历史介绍} \label{cha:24}


如同科学史中的任何事物一样, 超对称自身的历史是非常独特的. 自\,20\,世纪\,70\,年代早期一提出, 超对称就被整合进一个优美的数学体系, 这个数学体系将不同自旋的粒子统一成对称性多重态并对基础物理有深远的影响. 迄今为止, 只有少数几个间接证据表明超对称与现实世界无关, 而直接证据则是一个都没有. 如果超对称最后确实与自然相关(也是我所期望的), 那么它将代表纯理论洞察的一个惊人的成功.

第\,25\,章将开始从第一原理出发来构建超对称理论. 在本章, 我们将沿着历史而非逻辑的顺序来介绍超对称.

\section{非传统的对称性与``止步''定理} \label{sec:24.1}

在\,20\,世纪\,60\,年代早期, (在\,19.7\,节讨论过的)Gell-Mann\,和\,Ne'eman\,的$\,SU(3)\,$对称性成果解释了各种强相互作用粒子之间的关系, 这些粒子有着不同的荷和奇异数, 但是它们的自旋相同. 这一想法然后发展为$\,SU(3)\,$也许是某个更大对称性的一部分, 这个更大对称性有非传统的效应, 它将不同自旋的$\,SU(3)\,$多重态统一在一起.\cite{1} 在非相对论性的夸克模型中有这样的近似对称性, 类比\,Wigner\,于\,1937\,年在核物理中引入的早期$\,SU(4)\,$对称性,\cite{2} 在夸克自旋和味道的$\,SU(6)\,$变换下有一近似对称性. 这一对称性将在本章附录\,A\,进行详细讨论, 它将赝标量介子八重态$\,\pi,K,\bar{K},\eta$, 矢量介子八重态$\,\rho,K^{\ast},\bar{K}^{\ast},\omega$, 以及矢量介子单态$\,\phi\,$统一进一个$\,\mathbf{35}\,$多重态, 并将自旋$\,1/2\,$重子八重态$\,N,\Sigma,\Lambda,\Theta\,$和自旋$\,3/2\,$的重子十重态%
$\,\Delta,\Sigma(1385),\Theta(1530),\Omega\,$统一进一个$\,\mathbf{56}\,$多重态. $SU(6)\,$对称性取得了数个成功, 但是它实际上只是夸克模型中的力与自旋和味道无关的结果; $SU(6)\,$对称性比自旋与味道无关性的假定还要稍微弱一些, 但是, 正如附录\,A\,所表明的, 与假定完全的自旋和味道无关性所给出的预测相比, 没有什么证据表明$\,SU(6)\,$对称性的预测要更精确.

然而, 曾有数次努力试图将非相对论夸克模型的$\,SU(6)\,$对称性推广到完全相对论性的量子理论.\cite{3} 这些尝试都失败了, 但是数个学者证明了, 在各种限制性假设下, 这实际上是不可能的.\cite{4} 这一类定理中最深远的那一个是\,Coleman\,和\,Mandula\,在\,1967\,年证明的一个定理.\cite{5} 他们采取了数个合理的假定, 包括在任意给定质量下只有有限多个粒子种类, 在几乎所有能标上都能发生散射, 以及$\,S\,$-矩阵的解析性, 然后使用这些假定证明了, 对于那些与$\,S\,$-矩阵对易, 那些将单粒子态变到单粒子态, 以及在多粒子态上的作用等价于在单粒子态上作用的直和, 对于这些对称性算符, 它们的最一般\,Lie\,代数的构成元素只有: Poincar\'{e}\,群的生成元$\,P_{\mu}\,$和$\,J_{\mu\nu}$, 再加上作用在单粒子态的内部对称性的生成元, 其中内部对称性生成元的矩阵与自旋和动量无关且关于它们是对角的. 在第\,25\,章分析四维时空中所有可能的超对称代数, 以及在第\,32\,章分析高维时空中所有可能的超对称代数时, 这个定理将是我们要使用的一个重要元素. 在\,32.3\,节, 我们将会在包含扩展物体的理论中考察超对称代数, 那里\,Coleman-Mandula\,定理是不成立的.

Coleman\,和\,Mandula\,的证明非常巧妙且比较复杂. 本章的附录\,B\,会给出这个证明的一个版本. 在本节, 我们仅对这个定理的一部分给出一个非常简单的纯运动学证明, 尽管要证明的只是\,Coleman-Mandula\,定理的一部分, 但是这部分已经可以非常清楚地说明为什么$\,SU(6)\,$这种非传统的对称性可以出现在非相对论性的理论%
却不能出现在相对论性的理论中. 我们将使用\,Lorentz\,不变性证明: 对于所有与动量生成元$\,P_{\mu}\,$的对称性算符$\,B_{\alpha}$, 如果它们的\,Lie\,代数的生成元只有$\,P_{\mu}\,$本身, 再加上某个只有有限个参量的半单紧致\footnote{关于半单紧致\,Lie\,代数的定义, 参看\,15.2\,节中的脚注.}%
\,Lie\,子代数$\,\mathscr{A}\,$的厄米生成元$\,B_{A}$, 那么$\,B_{A}\,$必须只能是某个内部对称性的生成元, 也就是说, 它们在单粒子态上的作用是一个关于动量和自旋对角且与动量和自旋无关的矩阵. 在这个定理的证明中没有使用$\,S\,$-矩阵的性质和粒子频谱的有限性, 对对称性算符是如何作用在物理态上也没有做假定. $SU(6)\,$的\,Lie\,代数当然是半单且紧致的, 所以这个定理表明, 在相对论性的理论中不能使用这样的对称性去推导不同自旋的粒子之间的关系.

下面是证明. 设所有与\,4\,-动量$\,P_{\mu}\,$对易的对称性生成元构成了一个\,Lie\,代数, 且这个\,Lie\,代数由生成元$\,B_{\alpha}\,$张开. 考察固有\,Lorentz\,变换$\,x^{\mu}\to \Lambda\indices{^\mu_\nu}\,$在这些生成元上的效应, 它在\,Hilbert\,空间中由幺正算符$\,U(\Lambda)\,$表示. 很容易看到算符$\,U(\Lambda)B_{\alpha}U^{-1}(\Lambda)\,$是与$\,\Lambda\indices{_\mu^\nu}P_{\nu}\,$%
对易的厄米对称性算符, 所以, 由于$\,\Lambda\indices{_\mu^\nu}\,$是非奇异的, 这个算符也必须与$\,P_{\mu}\,$对易, 因此必须是$\,B_{\alpha}\,$的线性组合:
\begin{equation}
U(\Lambda)B_{\alpha}U^{-1}(\Lambda)
=\sum_{\beta}D\indices{^{\beta}_{\alpha}}(\Lambda)B_{\beta} \:,  \label{24.1.1}
\end{equation}%
$D\indices{^{\beta}_{\alpha}}\,$是一组实系数, 它们构成了齐次\,Lorentz\,群的一个表示
\begin{equation}
D(\Lambda_{1})D(\Lambda_{2})=D(\Lambda_{1}\Lambda_{2}) \:. \label{24.1.2}
\end{equation}%
更进一步, $U(\Lambda)B_{\alpha}U^{-1}\Lambda\,$满足的对易关系与$\,B_{\alpha}\,$相同, 所以这个\,Lie\,代数的结构常数$\,C_{\alpha\beta}^{\gamma}\,$是不变张量, 也就是说
\begin{equation}
C_{\alpha \beta }^{\gamma} = \sum_{\alpha^{\prime}\beta^{\prime}\gamma^{\prime}}
D\indices{^{\alpha^{\prime}}_{\alpha}}(\Lambda)\, D\indices{^{\beta^{\prime}}_{\beta}}(\Lambda)\,
D\indices{^{\gamma}_{\gamma^{\prime}}}(\Lambda^{-1})\,
C_{\alpha^{\prime}\beta^{\prime}}^{\gamma^{\prime}}\:.  \label{24.1.3}
\end{equation}%
用$\,C_{\gamma\delta}^{\alpha}\,$的相应方程进行收缩, 我们发现
\begin{equation}
g_{\beta\delta} = \sum_{\beta^{\prime}\delta^{\prime}}
D\indices{^{\beta^{\prime}}_{\beta}}(\Lambda)\,
D\indices{^{\delta^{\prime}}_{\delta}}(\Lambda)\,g_{\beta^{\prime}\delta^{\prime}} \:,  \label{24.1.4}
\end{equation}%
其中$\,g_{\beta\delta}\,$是\,Lie\,代数度规
\begin{equation}
g_{\beta\delta} \equiv \sum_{\alpha\gamma}C_{\alpha\beta}^{\gamma}\,C_{\gamma\delta}^{\alpha}\:. \label{24.1.5}
\end{equation}%
由于所有这些生成元与$\,P_{\mu}\,$对应, 我们有$\,C_{\mu\beta}^{\alpha}=-C^{\alpha}_{\beta\mu}=0$, 所以$\,g_{\mu\alpha}=g_{\alpha\mu}=0$.

取代$\,\alpha,\beta,\cdots$, 我们用下标$\,A,B,\cdots\,$来标记$\,P_{\mu}\,$以外的对称性生成元. 在方程(\ref{24.1.5})中使用$\,C^{A}_{\mu B}=-C_{B\mu}^{A}\,$为零就给出了$\,g_{AB}=\sum_{CD}C_{AC}^{D}\,C_{BD}^{C}$. 我们假定了生成元$\,B_{A}\,$张成了紧致半单\,Lie\,代数, 所以矩阵$\,g_{AB}\,$是正定的. 方程(\ref{24.1.4})和(\ref{24.1.2})就表明矩阵$\,g^{1/2}D(\Lambda)g^{-1/2}\,$构成了齐次\,Lorentz\,群的有%
限维实正交表示, 也即有限维幺正表示. 但是, 因为\,Lorentz\,群是不紧的, {\kai{这样的表示只能是平庸表示}}, 对这样的表示$\,D(\Lambda)=1$. (这就是为什么相对论性产生差异的所有原因; 伽利略群的半单部分是{\kai{紧}}群$\,SU(2)$, 它显然有无限多个有限维幺正表示.) 有了$\,D(\Lambda)=1$, 对于所有\,Lorentz\,变换$\,\Lambda\indices{^{\mu}_{\nu}}$, 生成元$\,B_{A}\,$与$\,U(\Lambda)\,$对易.

对于动量为$\,p^{\mu}\,$的单粒子稳定态$\,\lvert p,n \rangle$, 其中离散指标$\,n\,$标记粒子的自旋和种类, 当$\,B_{A}\,$这种与$\,P_{\mu}\,$对易的算符作用在它上面时, 能够产生的只能是这种态的线性组合
\begin{equation}
B_{A}\lvert p,n\rangle =\sum_{m}\Bigl( b_{A}(p)\Bigr) _{mn}\lvert p,m\rangle \:. \label{24.1.6}
\end{equation}
$B_{A}\,$与我们在\,2.5\,节中所谓的``增速''对易, 这一性质暗示了$\,b_{A}(p)\,$与动量无关, 而$\,B_{A}\,$与旋转对易这一性质暗示了$\,b_{A}(p)\,$以单位矩阵的方式作用在自旋指标上, 正如我们所要证明的, 所以$\,B_{A}\,$是普通内部对称性的生成元.

\section{超对称的诞生} \label{sec:24.2}

如果理论物理的发展是遵循逻辑的, 那么在\,Coleman-Mandula\,定理证明之后, 我们应该试图去寻找这个定理的例外, 然后注意到这个定理只涉及玻色子到玻色子和费米子到费米子的变换, 因而它只由满足对易关系而非反对易关系的算符生成. 这就会给出一个问题: 一个相对论性的理论是否能够有一个作用在粒子自旋上的非平庸对称性, 这个对称性将玻色子和费米子变换到彼此. 沿着下一章描述的路线探索这种超代数的可能结构, 超对称性将作为唯一的可能性浮现出来.

历史上却并非如此. 相反, 超对称独立出现在弦论的一系列文章和两篇少为人知的文章中, 关于后者我们会在后面进一步讨论, 但是它们中的任何一篇都没有迹象表明作者关心过\,Coleman-Mandula\,定理.

自\,20\,世纪\,60\,年代后期, 对强相互作用过程就开始试图构建满足各种理论要求的$\,S\,$-矩阵元, 这种努力导致了各种强子的一个新图景, 即各种强子是弦振动的不同模式.\cite{6} 对于弦上由参量$\,\sigma\,$标记的那一点, 它在某个固定时钟的时刻$\,\tau\,$会有时空坐标$\,X^{\mu}(\sigma,\tau)$, 所以弦在$\,d\,$维时空中运动的理论可以认为是有$\,d\,$个玻色场的二维场论, 它的作用量是
\begin{align}
I[X] &= \frac{T}{2} \int \dif\sigma \int \dif\tau \:\eta_{\mu\nu}
\left[ \frac{\partial X^{\mu}}{\partial \tau}\,\frac{\partial X^{\nu}}{\partial \tau}
-\frac{\partial X^{\mu}}{\partial \sigma}\,\frac{\partial X^{\nu}}{\partial\sigma}\right]   \nonumber \\
&=T\int \dif\sigma^{+} \int \dif\sigma^{-}\:\eta_{\mu\nu}\,
\frac{\partial X^{\mu}}{\partial \sigma^{+}}\,\frac{\partial X^{\nu}}{\partial \sigma^{-}}\:, \label{24.2.1}
\end{align}%
其中$\,T\,$是被称为弦张力的常数; $\mu=0,1,\cdots,d-1$; $\sigma^{\pm}\,$是二维``光锥''坐标$\,\sigma^{\pm}\equiv \tau\pm\sigma$. 这个作用量可以从下面这个更一般的版本中推导出来,
\begin{equation}
I[X]= -\frac{T}{2} \int \dif^{2}\sigma \: \eta_{\mu \nu}\sqrt{\operatorname{Det}g}\,g^{kl}\,
\frac{\partial X^{\mu}}{\partial \sigma^{k}}\frac{\partial X^{\nu}}{\partial \sigma^{l}} \:, \label{24.2.2}
\end{equation}%
它在两个``世界面坐标''$\,\sigma_{k}\,$的变换下拥有完全的不变性,\footnote{对于玻色弦, 除非处在$\,d=26\,$维时空中, 或者在引入费米子后处在$\,d=10\,$维时空中, 同第\,22\,章中讨论的那些量子反常一样, 这个对称性会被量子反常破坏.} 通过回到一个特殊的坐标系, 使得世界面度规$\,g_{kl}\,$满足
\begin{equation}
\sqrt{\operatorname{Det}g}\,g^{kl}=
\begin{pmatrix}
1 & 0 \\
0 & -1%
\end{pmatrix} \:,   \label{24.2.3}
\end{equation}%
我们就能回到原来的作用量. 在电动力学中, 作用量对于类时光子是负号, 它带来的问题被理论的规范不变性消除了, 这里也是一样的, 方程(\ref{24.2.1})和(\ref{24.2.2})中的$\,\eta_{\mu\nu}\,$在$\,\mu=\nu=0\,$也是负号, 它带来的问题被作用量(\ref{24.2.2})(对于合适的边界条件)在广义世界面坐标变换下的不变性消除了. 在作用量取(\ref{24.2.1})的特殊坐标系下, 在广义世界面坐标变换下还残存着一个重要的不变性: 在定域{\kai{共形变换}}\footnoteB{原书误植为整体共形变换, 在二维情况下, 整体共形变换是分式线性变换$z\to (a+bz){\big/}(c+dz)$.\qquad ------译者注}下的不变性:
\begin{equation}
\sigma^{\pm} \to f^{\pm}(\sigma^{\pm}) \:,  \label{24.2.4}
\end{equation}%
其中$\,f^{\pm}\,$是两个独立的任意函数.

弦论描述的粒子并没有对上现实世界中的粒子. 在\,1971\,年, Ramond\cite{7}以及\,Neveu\,和\,Schwarz\cite{8}, 分别以引入半整数自旋的粒子和带有$\,\pi\,$子量子数的粒子为目的, 他们提出再加上$\,d\,$个费米场二元组$\,\psi_{1}^{\mu}(\sigma,\tau)\,$和$\,\psi_{2}^{\mu}(\sigma,\tau)$. 不久之后, Gervais\,和\,Sakita\cite{9}就为这个理论引入了一个作用量:
\begin{equation}
I[X,\psi] = \int \dif\sigma^{+} \int \dif\sigma^{-}
\left[ T\frac{\partial X^{\mu}}{\partial \sigma^{+}}\,\frac{\partial X_{\mu}}{\partial \sigma^{-}}%
+\mi\psi_{2}^{\mu}\frac{\partial}{\partial \sigma^{+}}\psi_{2\mu}
+\mi\psi_{1}^{\mu}\frac{\partial}{\partial \sigma^{-}}\psi_{1\mu}\right] \:,  \label{24.2.5}
\end{equation}%
注意到, 通过扩展共形变换(\ref{24.2.4}), 使其以如下的方式也作用在费米场上, 共性不变依旧是存在的,
\begin{equation}
\psi_{1}^{\mu} \to \left( \frac{\dif f^{+}}{\dif\sigma^{+}}\right)^{-1/2}\psi_{1}^{\mu} \:,\qquad\qquad
\psi_{2}^{\mu} \to \left( \frac{\dif f^{-}}{\dif\sigma^{-}}\right)^{-1/2}\psi_{2}^{\mu} \:. \label{24.2.6}
\end{equation}%
Gervais\,和\,Sakita\,指出, 除了二维共性不变性和$\,d\,$-维\,Lorentz\,不变性, 对于合适的边界条件, 这个理论在交换玻色场$\,X^{\mu}\,$和费米场$\,\psi_{r}^{\mu}\,$的无限小变换下有一个对称性
\begin{align}
\delta \psi_{2}^{\mu}(\sigma^{+},\sigma^{-}) &=\mi T\,\alpha_{2}(\sigma^{-})\,
\frac{\partial}{\partial \sigma^{-}}X^{\mu}(\sigma^{+},\sigma^{-}) \:,  \nonumber \\
\delta \psi_{1}^{\mu}(\sigma^{+},\sigma^{-}) &=\mi T\,\alpha_{1}(\sigma^{+})\,
\frac{\partial}{\partial \sigma^{+}}X^{\mu}(\sigma^{+},\sigma^{-}) \:,  \label{24.2.7} \\
\delta X^{\mu}(\sigma^{+},\sigma^{-}) &=\alpha_{2}(\sigma^{-})\,\psi_{2}^{\mu}(\sigma^{+},\sigma^{-})
+\alpha_{1}(\sigma^{+})\,\psi_{1}^{\mu}(\sigma^{+},\sigma^{-}) \:,  \nonumber
\end{align}%
其中$\,\alpha_{1}\,$和$\,\alpha_{2}\,$分别是$\,\sigma^{+}\,$和$\,\sigma^{-}\,$的无限小费米函数, 类似于\,9.5\,节引入的格拉斯曼变量. 这个对称性是后面被称为超对称性的一个例子, 这个对称性将费米子和玻色子联系在了一起, 但现在只是二维场论的一个对称性, 还不是四维时空中的物理理论的对称性.

 几年之后, Wess\,和\,Zumino\,回顾了参考文献[7-9]中提供的超对称性的例子, 并做了评述: 很自然地, 应该尝试将超对称的概念推广到{\kai{四}}维时空中的量子场论. 他们构建了几个超对称模型. 最简单的那一个包含一个\,Majorana\,(自荷共轭\,Dirac\,)场$\,\psi$, 实的标量和赝标量玻色场$\,A\,$和$\,B$, 实的标量和赝标量辅助玻色场$\,F\,$和$\,G$, 它在如下的无限小变换下是不变的\footnote{这里对\,Dirac\,矩阵使用的符号是前言和\,5.4\,节解释过的那些符号. 这里的$\,\gamma_{5}\,$(满足$\,\gamma_{5}^{2}=1$\,)是\,Wess\,和\,Zumino\,使用的$\,\gamma_{5}\,$乘以$\,\mi$, 对于任何旋量$\,\psi\,$的协变共轭$\,\bar{\psi}$, 这里的定义是\,Wess\,和\,Zumino\,的定义乘以$\,\mi$. 由于这个原因, 方程(\ref{24.2.8})---(\ref{24.2.10})中的一些相位与参考文献[10]中的那些相位不同.}
\begin{align}
\delta A &=\Bigl(\bar{\alpha}\,\psi\Bigr) \:, \qquad
\delta B=-\mi\Bigl(\bar{\alpha}\,\gamma_{5}\,\psi \Bigr) \:,  \nonumber \\
\delta \psi  &= \partial_{\mu}(A+\mi\gamma_{5}B)\gamma^{\mu}\alpha
+(F-\mi\gamma_{5}G)\alpha \:,  \label{24.2.8} \\
\delta F &=\Bigl(\bar{\alpha}\,\gamma^{\mu}\,\partial_{\mu}\psi \Bigr) \:,\qquad
\delta G=-\mi\Bigl( \bar{\alpha}\,\gamma_{5}\gamma^{\mu}\,\partial_{\mu}\psi \Bigr) \:,  \nonumber
\end{align}
其中$\,\alpha\,$是任意的无限小\,Majorana\,费米\,c\,-数常参量. 如果我么要求作用量在这些变换下不变, 那么用这些元素构造出来的最一般的Lorentz\,不变且宇称守恒的可重整实拉格朗日密度是
\begin{align}
\mathscr{L} &= -\tfrac{1}{2}\partial_{\mu} A\partial^{\mu }A
-\tfrac{1}{2}\partial_{\mu}B\partial^{\mu}B-\frac{1}{2}\bar{\psi}\gamma^{\mu}\partial_{\mu}\psi \nonumber \\
&\quad +\tfrac{1}{2}(F^{2}+G^{2}) + m[FA+GB-\tfrac{1}{2}\bar{\psi}\psi] \nonumber \\
&\quad +g\Bigl[ F(A^{2}+B^{2})+2GAB-\bar{\psi}(A+\mi\gamma_{5}B)\psi \Bigr] \:.  \label{24.2.9}
\end{align}%
由于辅助场$\,F\,$和$\,G\,$是以二次型的方式进入到理论中, 通过令它们等于场方程给出的值
\begin{equation}
F=-mA-g(A^{2}+B^{2})\:, \qquad G=-mB-2gAB  \: .  \label{24.2.10}
\end{equation}%
我们就能导出等价的拉格朗日量. 这样拉格朗日量就变成
\begin{align}
\mathscr{L} &= -\tfrac{1}{2}\partial_{\mu}A\partial^{\mu}A
-\tfrac{1}{2}\partial_{\mu}B\partial^{\mu}B-\tfrac{1}{2}\bar{\psi}\gamma^{\mu}\partial_{\mu}\psi \nonumber \\
&\quad-\tfrac{1}{2}m^{2}[A^{2}+B^{2}]-\tfrac{1}{2}m\bar{\psi}\psi   \nonumber \\
&\quad -gmA(A^{2}+B^{2}) - \tfrac{1}{2}g^{2}(A^{2}+B^{2})^{2}-g\bar{\psi}(A+\mi\gamma_{5}B)\psi \:. \label{24.2.11}
\end{align}
这个拉格朗日密度不仅呈现出标量质量和费米子质量相关, 并且呈现出\,Yukawa\,相互作用和标量自耦合相关, 这些正是超对称理论的特征. Wess\,和\,Zumino\,还描述了包含一个矢量场的超多重态的超对称变换, 并给出了拉格朗日量. (我们会在第\,26\,章更细致的讨论这些.) 最后, 在第二篇文章中, Wess\,和\,Zumino\cite{11}回顾了\,Coleman-Mandula\,定理, 并追溯出了对这一定理的明显破坏是由于这里的对称性生成元满足反对易关系而不是对易关系. 在几年之后, Gliozzi, Scherk\,和\,Olive\cite{11a}证明了, 通过给\,Ramond-Neveu-Schwarz\,模型的场附加合适的边界条件, 可以构造出一个既有时空超对称性又有世界面超对称性的超弦理论.

Wess\,和\,Zumino\,不知道的是, 在他们第一篇关于四维时空超对称的文章之前, 这个对称性就已经出现在发表在苏联的两篇文章中了. 在\,1971\,年, Gol'fand\,和\,Likhtman\,就已经将\,2.4\,节讨论的\,Poincar\'{e}\,群拓展至超代数, 并使用在这个超代数下不变的要求构建了四维时空中的超对称场论. 他们的文章虽然很有预见性, 但只给了少量的细节, 并在之后很长的一段时间内被普遍忽视了. Volkov\,和\,Akulov\cite{13}在\,1973\,年独立地发现了现在所谓的自发破缺超对称性, 但是他们用他们的体系将与超对称破缺相联系的\,Goldstone\,费米子同中微子等同起来, 这是一个注定失败的想法. 对于大多数理论物理学家, 尤其是苏联以外的理论物理学家, 超对称性能够作为四维时空中真实世界的一种可能的对称性, 是从\,Wess\,和\,Zumino\,他们\,1974\,年的文章开始的.

%+++++++++++++++++++++++附录A++++++
\titleformat{\chapter}{\centering\CJKfamily{zhhei}\huge}{\chaptertitlename}{1em}{}
\titlespacing{\chapter}{0pt}{3.5ex plus .1ex minus .2ex}{10\wordsep}
\titleformat{\section}{\centering\CJKfamily{zhhei}\Large}{附 录\thesection}{1em}{}
\titlespacing{\section}{2em}{3.5ex plus .1ex minus .2ex}{1.5\wordsep}
\titleformat{\subsection}{\centering\CJKfamily{zhhei}\large}{\thesubsection}{0em}{}
\titlespacing{\subsection}{2em}{1.5ex plus .1ex minus .2ex}{\wordsep}
\renewcommand{\captionfont}{\small}
\newcounter{app}[chapter]
\setcounter{app}{1}
\renewcommand\thesection{\Alph{app}}
\renewcommand\theequation{\arabic{chapter}.\Alph{app}.\arabic{equation}}
\fancyhf{}
\fancyhead[CE]{\leftmark}
\fancyhead[CO]{\rightmark}
\fancyhead[RO,LE]{$\cdot$\ \thepage\ $\cdot$}
\renewcommand{\headrulewidth}{0.8pt}
\pagestyle{fancy}
\renewcommand{\chaptermark}[1]{\markboth{第\,\thechapter\,章\ #1}{}}
\renewcommand{\sectionmark}[1]{\markright{附录 \thesection\quad\ #1}{}}

\section{非相对论夸克模型的$SU(6)$对称性}

在这个附录中, 我们将描述$\,SU(6)\,$对称性是如何将非相对论夸克模型中各种自旋的粒子关联起来的. 这与超对称性没有直接关系, 但是它为\,Coleman-Mandula\,定理提供了一个历史背景, 而这对于\,25.1\,节和\,31.1\,节中一般超对称代数的构造是一个重要的输入信息.

一般而言, 非相对论夸克模型的哈密顿量不仅依赖动量和位置, 而且依赖自旋算符和味算符 $\,\sigma_{i}^{(n)}\,$和$\,\lambda_{A}^{(n)}$, 其中$\,\sigma_{i}^{(n)}\,$(\,$i=1,2,3$\,)就作为方程(\textcolor{foo}{5.4.18})中定义%
的\,Pauli\,矩阵$\,\sigma_{i}\,$作用在第$\,n\,$个夸克的自旋指标上, 而$\,\lambda_{A}^{(n)}\,$(\,$A=1,2,\cdots,8$\,)就作为方程(\textcolor{foo}{19.7.2})中定义的\,Gell-Manna\,%
$SU(3)\,$矩阵$\,\lambda_{A}\,$作用在第$\,n\,$个夸克的味指标上. (当$\,n\,$指代反夸克时, $\sigma_{i}^{(n)}\,$和$\,\lambda_{A}^{(n)}\,$就作为逆步表示的矩阵$\,-\sigma_{i}^{\mathrm{T}}\,$%
和$\,-\lambda_{A}^{\mathrm{T}}$ 进行作用.) 如果我们假定没有自旋-轨道耦合, 使得总轨道角动量$\,L_{i}\,$分别守恒, 那么我们可以得出: 哈密顿量与总自旋和总幺旋(\,unitary spin\,)
\begin{equation}
S_{i}\equiv \tfrac{1}{2}\sum_{n}\sigma_{i}^{(n)}\:,\qquad\qquad
T_{A}\equiv \tfrac{1}{2}\sum_{n}\lambda_{A}^{(n)} \:,  \label{24.A.1}
\end{equation}%
以及$\,L_{i}\,$对易. 另一方面, 如果我们假定哈密顿量只依赖于夸克的动量和位置, 但是与自旋和夸克味完全无关, 那么这样的哈密顿量不仅与总轨道角动量$\,\mathbf{L}\,$对易, 还与算符$\,\sigma_{i}^{(n)}\,$和$\,\lambda_{A}^{(n)}\,$中的{\kai{每一}}个都分别对易. 在这两种极端情况之间有一种有趣的可能性,  除了与$\,L_{i}$, $S_{i}\,$和$\,T_{A}\,$对易外, 哈密顿量还与如下算符对易
\begin{equation}
R_{iA}\equiv \tfrac{1}{2}\sum_{n}\pm \sigma_{i}^{(n)}\lambda_{A}^{(n)} \:,  \label{24.A.2}
\end{equation}%
其中符号对于夸克和反夸克分别是\,$+$\,号和$\,-\,$号.\footnote{对于反夸克产生负号是因为: 对于反夸克, $R_{iA}\,$中的项必须作为矩阵$\,-(\sigma_{i}\lambda_{A})^{\mathrm{T}}=
-(-\sigma_{i}^{\mathrm{T}})(-\lambda_{A}^{\mathrm{T}})\,$作用在自旋指标和味指标上} $S_{i}$, $T_{A}\,$和$\,R_{iA}\,$构成了$\,SU(6)\,$群的\,Lie\,代数, 对易关系是
\begin{align}
&[S_{i},S_{j}] =\mi\sum_{k}\epsilon_{ijk}S_{k} \: ,\qquad%
[T_{A},T_{B}]=\mi\sum_{C}f_{ABC}T_{C}\:,\qquad [S_{i},T_{A}]=0\:, \nonumber \\
&[S_{i},R_{jA}] =\mi\sum_{k}\epsilon _{ijk}R_{kA} \:,\qquad%
[T_{A},R_{iB}]=\mi\sum_{C}f_{ABC}R_{iC} \:, \label{24.A.3} \\
&[R_{Ai},R_{Bj}] =\mi\delta _{ij}\sum_{C}f_{ABC}T_{C}+\tfrac{2}{3}%
\mi\delta _{AB}\sum_{k}\epsilon _{ijk}S_{k}+\mi\sum_{kC}\epsilon_{ijk}d_{ABC}R_{kC} \:,  \nonumber
\end{align}%
这里的$\,f_{ABC}\,$和$\,d_{ABC}\,$分别是全反对称和完全对称的数值系数,\cite{14} 其中独立且不为零的值是
\begin{equation}
\begin{split}
f_{123} &=1\:, \qquad f_{458}=f_{678}=\sqrt{3}/2 \:,  \\
f_{147} &=f_{165}=f_{246}=f_{257}=f_{345}=f_{376}=1/2  \end{split}  \label{24.A.4}
\end{equation}
和
\begin{align}
d_{146} &=d_{157}=-d_{247}=d_{256}=d_{344}=d_{355}=-d_{366}=-d_{377}=1/2 \:,  \nonumber \\
d_{118} &=d_{228}=d_{338}=-d_{888}=1/\sqrt{3} \:,  \label{24.A.5} \\
d_{448} &=d_{558}=d_{668}=d_{778}=-1/(2\sqrt{3}) \:.  \nonumber
\end{align}%
如果我们在哈密顿量中引入的是与$\,R_{iA}$, $S_{i}\,$和$\,T_{A}\,$相对易的两体相互作用, 这一对称性会保留下来. 这样的相互作用是存在的, 由如下形式的两体算符的线性组合给出
\begin{equation}
H^{(nm)}\propto \left[ 1\pm \sum_{i}\sigma_{i}^{(n)}\sigma_{i}^{(m)}\right]
\left[ \frac{2}{3}\pm \sum_{A}\lambda_{A}^{(n)}\lambda_{A}^{(m)}\right] \: ,  \label{24.A.6}
\end{equation}%
其中, 如果粒子$\,n,m\,$中一个是夸克而另一个是反夸克, 符号$\,\pm\,$取负号, 如果都是夸克或都是反夸克则取正号.

当然, 哪怕是在非相对论夸克模型中, $SU(6)\,$对称性最多也只是个近似对称性. 它被自旋-轨道耦合, 自旋-自旋力以及$\,s\,$夸克的质量破缺掉了, 而后者将$\,SU(3)\,$味对称性降成同位旋守恒和超荷(hypercharge)\footnoteB{在本卷的翻译过程中, 有这样一个问题, 粒子物理中的守恒荷\,hypercharge\,和超对称守恒荷\,supercharge\,译成中文均是超荷, 为了区分, 译者在这两种荷后面附上英文以便区分.\qquad ------译者注}守恒的$\,SU(2)\,$和$\,U(1)$. 如果我们限制在那些只由$\,u\,$夸克, $d\,$夸克和它们的反夸克这种轻夸克构成的强子, 这样就能夸克的质量差带来的影响, 这时不为零的$\,\lambda_{A}\,$矩阵是$\,a=1,2,3\,$的$\,\lambda_{a}\,$%
(这些矩阵由\,Pauli\,矩阵(\textcolor{foo}{5.4.18})给出, 在这一情形下通常记做$\,\tau_{a}$)和$\,\lambda_{8}$(对于$\,u\,$夸克和$\,d\,$夸克只是数字$\,1/\sqrt{3}$, 对于反夸克$\,\bar{u}\,$和$\,\bar{d}\,$是$\,-1/\sqrt{3}$). 相互作用(\ref{24.A.6})因此变成
\begin{equation}
H^{(nm)}\propto \left[ 1\pm \sum_{i}\sigma_{i}^{(n)}\sigma_{i}^{(m)}\right]
\left[ 1\pm \sum_{A}\tau _{A}^{(n)}\tau _{A}^{(m)}\right] \:.
\label{24.A.7}
\end{equation}%
除了夸克味守恒外, 剩下的对称性是$\,SU(4)$, 生成元是与(\ref{24.A.7})对易的$\,S_{i}$, $T_{a}\,$和$\,R_{ia}$. 在\,1937\,年, Wigner\,提出它可能是核力的一种对称性, 当然, 那时认为这是质子和中子的对称性, 而非$\,u\,$夸克和$\,d\,$夸克. 在核力理论中, 不依赖于自旋或同位旋的相互作用被称为\,{\textit{Wigner}\,\kai{势}}, 正比于(\ref{24.A.7})自旋部分的相互作用被称为\,{\textit{Bartlett}\,\kai{势}}, 只正比于(\ref{24.A.7})同位旋部分的相互作用被称为\,{\textit{Heisenberg}\,\kai{势}}, 与这三者相区分, 相互作用(\ref{24.A.7})被称为\,\textit{Majorana}{\kai{势}}.

有趣的是, 尽管像$\,SU(6)\,$这种既作用在自旋上又作用在粒子种类上的对称性在非相对论理论中没有任何理论障碍, {\kai{但是, 没有任何证据表明, 相较于完全独立于自旋和味的假定, 非相对论夸克模型更好地满足这种$\,SU(6)\,$对称性.}} 这些假定是不同的; 如果\,$N$\,个非相对论夸克和(或)反夸克组成的系统的哈密顿量完全独立于自旋和味, 那么它的对称性不是$\,SU(6)\,$而是$\,SU(6)^{N}$. 例如像(\ref{24.A.6})的两体相互作用以及各种其它多粒子相互作用会把$\,SU(6)^{N}\,$破缺至$\,SU(6)$. 当然, 所有这些对称性都只是个近似. 问题是$\,SU(6)\,$被破坏的程度是不是要弱于$\,SU(6)^{N}$?

要想回答这个问题不能去研究包含重子八重态的多重态, 即由核子和重核子$\,\Lambda$, $\Sigma\,$和$\,\Xi\,$构成的多重态. 在非相对论夸克模型中, 这些粒子被解释成三个夸克的轨道角动量为零的束缚态. 由于这些态是色中性的, 波函数关于没写出的色指标是全反对称的, 所以它在自旋和味的组合交换下是完全对称的. 因此, 重子八重态必须被放进$\,SU(6)\,$的对称三阶张量表示$\,\mathbf{56}\,$中, 而这个表示除了重子八重态外还包含一个自旋\,3/2\,十重态, 这个十重态可以被认为是由著名的``3-3''共振$\,\Delta$以及粒子$\,\Sigma(1385)$, $\Xi(1530)\,$和$\,\Omega\,$构成的那个十重态. (括号中的数字是以$\,\mathrm{Mev}\,$为单位的粒子质量, 写出它们是为了与那些同位旋和奇异数相同但质量更低的粒子进行区分.) $SU(6)\,$对称性对重子磁矩给出了一个很好的预测: 夸克荷算符是$\,q=e(\lambda_{3}/2+\lambda_{8}/2\sqrt{3})$, 而有这样的荷且质量为$\,m_{N}/3\,$的\,Dirac\,粒子的磁矩是$\,3q/2m_{N}$, 如果夸克拥有磁矩$\,3q/2m_{N}$, 那么磁矩算符是
\[
\mu _{i}=3\mu_{N}\left[ \frac{1}{2}R_{i3}+\frac{1}{2\sqrt{3}}R_{i8}\right] \: ,
\]
其中$\,\mu_{N}\equiv e/2m_{N}\,$是核磁矩, $R_{iA}\,$由方程(\ref{24.A.2})定义. 计算这个对称性生成元在$\,\mathbf{56}\,$多重态的成员之间的矩阵元是直接的, 以$\,\mu_{N}\,$为单位, 对于$\,p$, $n$, $\Lambda$, $\Sigma^{+}$, $\Sigma^{-}$, $\Xi^{-}\,$和$\,\Xi^{0}$, 磁矩的计算结果分别是$\,+3$, $-2$, $-1$, $+3$, $-1$, $-1\,$和$\,-2$, 可以与相应的实验值$\,+2.79$, $-1.91$, $-0.61$, $+2.46$, $-1.16$, $-0.65\,$和$\,-1.25\,$进行比较. 与实验的误差是可接受的, 如果我们取夸克的磁矩比$\,3\mu_{N}\,$稍小一些, 结果(除了$\,\Sigma^{-}\,$)还要更好些. 由于\,3\,-夸克波函数的对称性, 如果假定哈密顿量完全独立于自旋和味, 这里并没有新东西; 角动量为零的态仍然会落进由$\,6\times 7\times 8/6!=56\,$个成员构成的多重态中. 特别地, 对于两个夸克的任意态, 只要它在自旋和味的同时交换下是对称的------无论在自旋的交换和味的交换下均为对称还是反对称, 算符(\ref{24.A.6})都有相同的值$\,4$.

为了判定$\,SU(6)^{N}\,$是否比$\,SU(6)\,$要好些, 研究介子要更有用些, 在非相对论夸克模型中, 它们被解释成夸克和反夸克的束缚态. 如果这些态的哈密顿量完全独立于自旋和味, 那么它的对称性是$\,SU(6)^{2}$, 介子态就落入它的\,36\,维表示$\,(6,\bar{6})\,$中, 然而对于$\,SU(6)\,$对称性, 我们只能说介子属于$\,\mathbf{6}\times \bar{\mathbf{6}}\,$中包含的两个$\,SU(6)\,$表示中的一个: 伴随表示$\,\mathbf{35}\,$或单态表示. 更具体些, 组成$\,\mathbf{35}\,$的是一个自旋$\,S=1\,$的$\,SU(3)\,$单态, 一个自旋$\,S=0\,$的$\,SU(3)\,$八重态, 以及一个自旋$\,S=1\,$的$\,SU(3)\,$八重态, 对应于$\,SU(6)\,$生成元$\,S_{i}$, $T_{A}\,$和$\,R_{iA}$, 它通过相互作用(\ref{24.A.6})与$\,S=0\,$的$\,SU(3)\,$单态分离开来. 既然所有这些假定都只是个近似, 判定$\,SU(6)\,$对称性是否要比完全独立于自旋和味更加精确这个问题就变成: $S=0\,$的$\,SU(3)\,$单态与其它轨道角动量相同的$\,35\,$个态的分离程度是否比%
$\,\mathbf{35}\,$超多重态内部的分离程度大.

当轨道角动量$\,L=0\,$时, 夸克-反夸克态的宇称$\,\mathsf{P}\,$为负, 再根据总自旋$\,S\,$是零还是一, 它的荷共轭量子数$\,\mathsf{C}\,$(对于自荷共轭态)分别为正和负. (关于这一点的解释, 参看\,5.5\,节.) 因此组成$\,\mathbf{35}\,$的是一个$\,J^{\mathsf{PC}}=1^{--}\,$的单态, 一个$\,0^{-+}\,$八重态, 以及一个$\,1^{--}\,$八重态, 它们可以被认为是: $\phi(1020)$; 赝标量八重态$\,\pi$, $\eta$, $K\,$和$\,\bar{K}$; 以及矢量八重态$\,\rho$, $\omega$, $K^{\ast}\,$和$\,\bar{K}^{\ast}$. 在$\,958\,\mathrm{MeV}\,$处还存在一个$\,0^{-+}\,$的$\,SU(3)\,$单态$\,\eta^{\prime}$, 可以认为它是$\,SU(6)\,$单态. 这个单态与$\,\mathbf{35}\,$多重态中的粒子的分离程度并没有比$\,\mathbf{35}\,$多重态内部的分离大多少.

有人可能会说$\,L=0\,$的介子并不能给非相对论模型的对称性提供一个很好的检验, 它们包含\,Goldstone\,玻色子$\,\pi$, $\eta$, $K\,$ 和$\,\bar{K}$, 而这些粒子在$\,u\,$夸克和$\,d\,$夸克质量为零的情况下变成无质量粒子, 因而不能被这一模型很好的描述. 那么我们来考虑$\,L=1\,$的夸克-反夸克态. 这些态的$\,\mathsf{P}\,$为正, $\mathsf{C}\,$根据$\,S=1\,$还是$\,S=0\,$分别为正或负, 所以组成\,p\,-波$\,\mathbf{35}\,$的是: $S=1\,$且$\,J^{\mathsf{PC}}=0^{++},\,1^{++},\,2^{++}$的\\$\,SU(3)\,$单态, 它们可以被认为是$\,f_{0}(1370)$, $f_{1}(1285)\,$和$\,f_{2}(1270)$; $S=0\,$的$\,1^{+-}\,$八重态, 它们可以被认为是$\,h_{1}(1170)$, $b_{1}(1235)$, $K_{1}(1400)\,$和$\,\bar{K}_{1}(1400)$; 以及$\,S=1\,$的八重态: 由$\,f_{0}(980)$, $a_{0}(980)$, $K_{0}^{+}(1950)\,$和$\,\bar{K}_{0}^{+}(1950)\,$构成的$\,0^{++}\,$八重态; $f_{1}(1420)$, $a_{1}(1260)$, $K_{1}^{+}(1650)\,$ 和$\,\bar{K}_{1}^{+}(1650)\,$构成的$\,1^{++}\,$\\八重态; 以及$f_{2}(1430)$, $a_{2}(1320)$, $K_{2}^{+}(1980)\,$和$\,\bar{K}_{2}^{+}(1980)\,$构成的$\,2^{++}\,$八重态. 除了这$\,35\times 3\,$个态外, 还有另外一个粒子拥有成为\,p\,-波$\,SU(6)\,$单态的正确量子数: $1^{+-}\,$的同位旋标量$\,h_{1}(1380)$. 当然, 我们可以交换$\,h_{1}(1170)\,$和$\,h_{1}(1380)\,$的身份, 或者认为$\,SU(3)\,$单态和八重态同位旋标量$\,1^{+-}\,$态是%
$\,h_{1}(1170)\,$和$\,h_{1}(1380)\,$的两个相互正交的线性组合. 问题的重点在于: 有{\kai{两个}}$\,1^{+-}\,$同位旋标量, 我们无法指出其中那一个属于$\,SU(6)\,$单态, 就更谈不上说$\,SU(6)\,$单态中的粒子与$\,\mathbf{35}\,$中的粒子的分离程度要强于$\,\mathbf{35}\,$内部的分离程度. 那么, 这里依旧没有什么证据表明$\,SU(6)\,$对称性要比完全独立于自旋和味这个假定更加精确.

\setcounter{app}{2}

\section{Coleman-Mandula定理}

这个附录将为著名的\,Coleman-Mandula\,定理\cite{5}给出一个证明, 即, 对称性生成元的唯一可能\\ Lie\,代数(与超代数相反)的组成部分是: 平移和齐次\,Lorentz\,变换的生成元$\,P_{\mu}\,$和$\,J_{\mu\nu}$, 以及可能的内部对称性的生成元, 其中内部对称性的生成元与$\,P_{\mu}\,$和$\,J_{\mu\nu}\,$对易, 并且它们作用在物理态上的方式是用一个与自旋和动量无关的厄米矩阵乘这些态.\footnote{我们将会看到, 在只有无质量粒子的理论中, 除了生成元$\,P_{\mu}\,$和$\,J_{\mu\nu}\,$外还可能存在额外的生成元$\,D\,$和$\,K_{\mu}$, 它们填充了共形群的\,Lie\,代数.\cite{15}} 这里的``对称性生成元''是指任何满足如下条件的厄米算符: 与$\,S\,$-矩阵对易; 对易子也是对称性生成元; 将单粒子态变到另一个单粒子态; 在多粒子态上的作用是它们在单粒子态上的作用的直和(就像方程(\ref{24.B.1})中那样). 一个更加技巧性的要求会在后面需要的时候加上来. 除了第\,2\,章和第\,3\,章所描述的相对论量子力学的一般原理外, 这个证明还需要的其它假设只有:
\begin{assumption}
对于任意$\,M$, 只有有限种粒子的质量小于$\,M$.
\end{assumption}
\begin{assumption}
任何二粒子态在几乎所有能量(即, 除了一个离散集外的所有能量)处都能发生某个反应. \label{assum:2}
\end{assumption}
\begin{assumption}
弹性两体散射的振幅在几乎所有能量和角度处都是散射角的解析函数.\footnote{\song{严格来讲, 在有红外发散的理论中, 例如量子电动力学, 这个假定是不被满足, 而在电动力学中, 我们在\,13.3\,节看到, 除了向前弹性散射外, 对于任何一个包含带电粒子的散射过程, 它的$\,S\,$-矩阵元实际上为零. 在电动力学这样的阿贝尔规范理论中, 通过对带有虚拟规范玻色子质量的理论使用\,Coleman-Mandula\,定理, 并且只处理那些``红外安全''的量, 例如质量以及一些合适的积分截面------在规范玻色子质量为零的极限下保持有限的截面, 这个问题可以被规避掉. 在量子色动力学这种所有无质量粒子都是陷俘的非阿贝尔规范理论中是没有问题的------%
对称性如果没有破缺的话将只能决定规范中性束缚态的$\,S\,$-矩阵元, 例如量子色动力学中的介子和重子. 就我所知, 在有非陷俘无质量粒子的非阿贝尔规范理论中, 例如带有多个夸克味的量子色动力学, Coleman-Mandula\,定理还没有被证明.}}
\end{assumption}
\noindent 不需要假定$\,S\,$-矩阵由一个定域量子场论决定. 这里给出的证明在一定程度上重新安排并进行了梳理, 并且补充了一些\,Coleman\,和\,Mandula\,留给读者的步骤.

证明这个定理的一个比较方便的出发点是:
从那些与\,4\,-动量算符$\,P_{\mu}\,$对易的对称性生成元$\,B_{\alpha}$\\ 组成的子代数着手处理. (定理的这个部分本身是很有意思的; 它在相对论理论中排除了作用方式类似于非相对论夸克模型的$\,SU(6)\,$对称性这样的对称性.)  这种对称性生成元在多粒子态上的作用是
\begin{align}
&B_{\alpha }\,\lvert p\,m,q\,n,\cdots \rangle  =\sum_{m^{\prime }}\Bigl( b_{\alpha
}(p)\Bigr) _{m^{\prime }m}\lvert p\,m^{\prime },q\,n,\cdots \rangle   \nonumber \\
&\qquad\qquad+\sum_{n^{\prime }}\Bigl( b_{\alpha }(q)\Bigr) _{n^{\prime
}n}\lvert p\,m,q\,n^{\prime },\cdots \rangle +\cdots \:,  \label{24.B.1}
\end{align}%
其中$\,m,n\,$等是标记拥有确定质量$\,\sqrt{-p_{\mu}p^{\mu}}\,$的粒子的自旋$\,z\,$-分量和粒子种类的离散指标, 而$\,b_{\alpha}(p)$ 是有限厄米矩阵, 它定义了$\,B_{\alpha}\,$在单粒子态上的作用.

现在, 从方程(\ref{24.B.1})我们可以看到, 对于固定的$\,p$, 将$\,B_{\alpha}\,$映到$\,b_{\alpha}(p)\,$的映射是个同态, 也就是说, $B_{\alpha}\,$满足的对易关系
\begin{equation}
[B_{\alpha},B_{\beta }]=\mi\sum_{\gamma}C_{\alpha\beta}^{\gamma}B_{\gamma}  \label{24.B.2}
\end{equation}%
也是厄米矩阵$\,b_{\alpha}(p)\,$满足的对易关系:
\begin{equation}
[b_{\alpha}(p),b_{\beta}(q)]=\mi\sum_{\gamma}C_{\alpha \beta}^{\gamma}\,b_{\gamma }(p) \:.  \label{24.B.3}
\end{equation}%
15.2\,节证明的一个著名定理告诉我们, 像$\,b_{\alpha}(p)\,$这样的有限厄米矩阵, 它的任何\,Lie\,代数必须是紧致半单\,Lie\,代数和$\,U(1)\,$代数的直和. 然而, 由于算符$\,B_{\alpha}\,$和矩阵$\,b_{\alpha}(p)\,$之间的同态不一定是个同构, 我们并不能直接应用这个定理. 为了使它成为一个同构, 我们还得要求, 只要对某些系数$\,c^{\alpha}\,$和某个动量$\,p\,$有$\,\sum_{\alpha}c^{\alpha}b_{\alpha}(p)=0$, 那么对所有动量$\,k\,$都有$\,\sum_{\alpha}c^{\alpha}b_{\alpha}(k)=0$, 这等价于条件$\,\,\sum_{\alpha}c^{\alpha}B_{\alpha}$ $=0$.

取代考察将$\,B_{\alpha}\,$映射到单粒子矩阵$\,b_{\alpha}(p)\,$的同态, Coleman\,和\,Mandula\,考察的同态是将$\,B_{\alpha}\,$映射到定义了$\,B_{\alpha}\,$在{\kai{二粒子态}}上的作用的矩阵, 若这个二粒子态有固定的\,4\,-动量$\,p\,$和$\,q$, 那么这个矩阵是:
\begin{equation}
\Bigl(b_{\alpha}(p,q)\Bigr)_{m^{\prime}n^{\prime},mn}
=\Bigl( b_{\alpha}(p)\Bigr)_{m^{\prime}m}\delta_{n^{\prime}n}
+\Bigl( b_{\alpha}(q)\Bigr)_{n^{\prime}n}\delta_{m^{\prime}m} \:.  \label{24.B.4}
\end{equation}%
对于两个\,4\,-动量为$\,p\,$和$\,q\,$的粒子到两个\,4\,动量为$\,p^{\prime}\,$和$\,q^{\prime}\,$%
的粒子的弹性或准弹性散射, 其中这两个粒子有质量$\,\sqrt{-p_{\mu}^{\prime}p^{\prime\mu}}=\sqrt{-p_{\mu}p^{\mu}}\,$%
和$\,\sqrt{-q_{\mu}^{\prime}q^{\prime\mu}}=\sqrt{-q_{\mu}q^{\mu}}$, $S\,$-矩阵的不变性给出条件
\begin{equation}
b_{\alpha}(p^{\prime},q^{\prime})\,S(p^{\prime},q^{\prime};p,q)
=S(p^{\prime},q^{\prime};p,q)\,b_{\alpha}(p,q) \: . \label{24.B.5}
\end{equation}%
这里的$\,S(p^{\prime},q^{\prime};p,q)\,$是一个与$\,b(p,q)\,$和$\,b(p^{\prime},q^{\prime})\,$量纲相同的矩阵, 以连通\,$S$\,-矩阵元$\,S(p\,m,q\,n\to p^{\prime}\,m^{\prime},q^{\prime}\,n^{\prime})\,$的形式, 它被定义成
\begin{equation}
S(p\,m,q\,n\rightarrow p^{\prime }\,m^{\prime },q^{\prime }n^{\prime })\equiv
\delta ^{4}(p^{\prime }+q^{\prime }-p-q)\,\Bigl( S(p^{\prime },q^{\prime
};p,q)\Bigr)_{m^{\prime }n^{\prime },mn} \:.  \label{24.B.6}
\end{equation}%
根据假设\,2\,和光学定理(见\,3.6\,节), 对于几乎任何$\,p\,$和$\,q\,$的选择, 弹性散射振幅在向前方向上是不为零的, 那么根据假设\,3, 对于几乎所有满足守恒条件以及在壳的$\,p^{\prime}\,$和$\,q^{\prime}$, 矩阵$\,S(p^{\prime},q^{\prime};p,q)\,$是非奇异的, 所以对于几乎所有这样的\,4\,-动量, 方程(\ref{24.B.5})是个{\kai{相似变换}}.

由此可以得出, 如果对于几乎任意的固定\,4\,-动量$\,p\,$和$\,q\,$有$\,\sum_{\alpha}c^{\alpha}b_{\alpha}(p,q)=0$, 那么对于几乎任何处在同一质壳且满足$\,p^{\prime}+q^{\prime}=p+q\,$的$\,p^{\prime}\,$和$\,q^{\prime}\,$有%
$\,\sum_{\alpha}c^{\alpha}b_{\alpha}(p^{\prime},q^{\prime})=0$. 不幸的是, 这并不能告诉我们$\,\sum_{\alpha}c^{\alpha}b_{\alpha}(p^{\prime})\,$和%
$\,\sum_{\alpha}c^{\alpha}b_{\alpha}(q^{\prime})\,$为零, 只是告诉我们这两个矩阵正比于单位矩阵(比例系数相反). 为了做的更好些, 我们要考察的既不是$\,b_{\alpha}(p)\,$也不是$\,b_{\alpha}(p,q)$, 而是它们的无迹部分.

方程(\ref{24.B.5})的一个直接结果是
\begin{equation}
\operatorname{Tr}b_{\alpha }(p^{\prime},q^{\prime})=\operatorname{Tr}b_{\alpha}(p,q)\:.\label{24.B.7}
\end{equation}%
再加上方程(\ref{24.B.4}), 这告诉我们
\begin{align}
&N(\sqrt{-q_{\mu}q^{\mu}})\operatorname{tr}b_{\alpha}(p^{\prime})
+N(\sqrt{-p_{\mu}p^{\mu}})\operatorname{tr}b_{\alpha}(q^{\prime})  \nonumber \\
&\quad=N(\sqrt{-q_{\mu}q^{\mu}})\operatorname{tr}b_{\alpha }(p)
+N(\sqrt{-p_{\mu}p^{\mu}})\operatorname{tr}b_{\alpha}(q) \:,  \label{24.B.8}
\end{align}%
其中$\,N(m)\,$是质量为$\,m\,$的粒子种类的多重数,\footnote{Coleman\,和\,Mandula\,没有显式地写出这些多重数因子. 他们不加说明地定义了核无迹的对称性生成元$\,B_{\alpha}^{\sharp}$, 而要说明这一步是合理的就需要这些因子.} 而``$\operatorname{tr}$''中小写的$\,\mathrm{t}\,$是指对单粒子指标而不是二粒子指标求和. 为了使上式对几乎所有满足$\,p^{\prime}+q^{\prime}=p+q\,$的在壳\,4\,-动量都满足, 函数$\operatorname{tr}b_{\alpha}(p)/N(\sqrt{-p_{\mu}p^{\mu}})$ 对$\,p\,$必是线性%
\footnote{很容易看到常数项从方程(\ref{24.B.9})中被排除了, 这是因为存在粒子数不守恒的过程, 而这样的过程在满足集团分解原理的相对论量子理论中是不可避免的. 即使我们仅考虑二粒子过程并且不使用这个讨论, 方程(\ref{24.B.9})中的常数项最后也只相当于内部对称性在粒子态上的作用中有一个变化.}的:
\begin{equation}
\frac{\operatorname{tr}b_{\alpha}(p)}{N(\sqrt{-p_{\mu }p^{\mu}})}
=a_{\alpha}^{\mu}\,p_{\mu} \:,  \label{24.B.9}
\end{equation}
其中$\,a_{\alpha}^{\mu}\,$独立于$\,p\,$(以及已写出指标外的任何东西.) 通过减除掉动量算符的一个线性项, 我们可以定义新的对称性算符:
\begin{equation}
B_{\alpha}^{\sharp}\equiv B_{\alpha}-a_{\alpha}^{\mu}P_{\mu}\:,\label{24.B.10}
\end{equation}
根据方程(\ref{24.B.9}), 它在单粒子态上被表示成无迹矩阵
\begin{equation}
    \Bigl(b_{\alpha}^{\sharp}(p)\Bigr)_{n^{\prime}n}=\Bigl(b_{\alpha}(p)\Bigr)_{n^{\prime}n}
    -\frac{\operatorname{tr}b_{\alpha}(p)}{N(\sqrt{-p_{\mu}p^{\mu}})}\delta_{n^{\prime}n}\:.\label{24.B.11}
\end{equation}
因为$\,P_{\mu}\,$与$\,B_{\alpha}\,$对易, 单位矩阵与所有一切都对易, $B_{\alpha}^{\sharp}\,$的对易子与$\,B_{\alpha}\,$的对易子相同, 并且$\,b_{\alpha}^{\sharp}(p)\,$的对易子与$\,b_{\alpha}(p)\,$的对易子相同:
\begin{align}
    &[B_{\alpha}^{\sharp},B_{\beta}^{\sharp}]=\mi\sum_{\gamma}C_{\alpha\beta}^{\gamma}B_{\gamma}
    =\mi\sum_{\gamma}C_{\alpha\beta}^{\gamma}\,\Bigl[B_{\gamma}^{\sharp}+a_{\gamma}^{\mu}P_{\mu}\Bigr]\:,
    \label{24.B.12} \\
    &[b_{\alpha}^{\sharp}(p),b_{\beta}^{\sharp}(p)]=\mi\sum_{\gamma}C_{\alpha\beta}^{\gamma}b_{\gamma}(p)
    =\mi\sum_{\gamma}C_{\alpha\beta}^{\gamma}\,\Bigl[b_{\gamma}^{\sharp}(p)+a_{\gamma}^{\mu}p_{\mu}\Bigr]\:.
    \label{24.B.13}
\end{align}
另外, 根据有限矩阵$\,b_{\alpha}^{\sharp}\,$的对易子的迹为零\footnote{这是我们用到假设\,1\,的一个地方, 没有这个假设, 对易子的迹不一定为零. 另外, 这里关键的地方是我们处理的是对易关系而不是反对易关系, 对于反对易关系, 首先单位矩阵并不与其它矩阵反对易, 其次有限矩阵的反对易子的迹不一定为零.}这个性质, 方程(\ref{24.B.13})给出了$\,\sum_{\gamma}C^{\gamma}_{\alpha\beta}\,a_{\gamma}^{\mu}=0$, 在方程(\ref{24.B.12})中使用这个结果就表明$\,B_{\alpha}^{\sharp}\,$满足的对易关系与$\,B_{\alpha}\,$相同:
\begin{equation}
[B_{\alpha}^{\sharp},B_{\beta}^{\sharp}] =\mi\sum_{\gamma}C_{\alpha\beta}^{\gamma}B_{\gamma}^{\sharp}\:.  \label{24.B.14}
\end{equation}%
因为$\,B_{\alpha}^{\sharp}\,$是对称性算符, 散射振幅满足
\begin{equation}
b_{\alpha}^{\sharp}(p^{\prime},q^{\prime})\,S(p^{\prime},q^{\prime};p,q)
=S(p^{\prime},q^{\prime};p,q)\,b_{\alpha}^{\sharp}(p,q) \label{24.B.15}
\end{equation}%
其中$\,b_{\alpha}^{\sharp}\,$是$\,B_{\alpha}^{\sharp}\,$在二粒子态上的表示矩阵
\begin{equation}
\Bigl(b_{\alpha}^{\sharp}(p,q)\Bigr)_{m^{\prime}n^{\prime},mn}
=\Bigl(b_{\alpha}^{\sharp}(p)\Bigr)_{m^{\prime}m}\delta_{n^{\prime}n}
+\Bigl(b_{\alpha}^{\sharp}(q)\Bigr)_{n^{\prime}n}\delta_{m^{\prime}m} \label{24.B.16}
\end{equation}%
并且它满足的对易关系与$\,B_{\alpha}^{\sharp}\,$相同:
\begin{equation}
[b_{\alpha}^{\sharp}(p,q),b_{\beta}^{\sharp}(p,q)]=\mi\sum_{\gamma}C_{\alpha\beta}^{\gamma}
\,b_{\gamma}^{\sharp}(p,q)\:. \label{24.B.17}
\end{equation}%
用这些二粒子矩阵进行处理的优点是, 由于$\,S(p^{\prime},q^{\prime};p,q)\,$是非奇异矩阵, 由此可以得出, 如果对于某两个固定的$\,4\,$-动量$\,p\,$和$\,q\,$有$\,\sum_{\alpha}c^{\alpha}b_{\alpha}^{\sharp}(p,q)=0$, 那么对于几乎所有处在同一质壳上且满足$\,p^{\prime}+q^{\prime}=p+q\,$的$\,p^{\prime}\,$和$\,q^{\prime}\,$都有%
$\,\sum_{\alpha}c^{\alpha}b_{\alpha}^{\sharp}(p^{\prime},q^{\prime})=0$. 由于我们现在处理的是无迹矩阵, 这告诉我们
\begin{equation}
 \sum_{\alpha}c^{\alpha}\,b_{\alpha}^{\sharp}(p^{\prime})
=\sum_{\alpha}c^{\alpha}\,b_{\alpha}^{\sharp}(q^{\prime})=0 \:.  \label{24.B.18}
\end{equation}%
我们本希望从此得出, 对于{\kai{所有}}在壳\,4\,-动量$\,k\,$都有$\sum_{\alpha}c^{\alpha}b_{\alpha}^{\sharp}(k)=0$, 但到现在为止我们仅证明了, 对于几乎所有那些使得$\,q^{\prime}=p+q-p^{\prime}\,$和$\,p^{\prime}\,$在壳的$\,p^{\prime}\,$有%
$\sum_{\alpha}c^{\alpha}b_{\alpha}^{\sharp}(p^{\prime})=0$, (对$\,q^{\prime}\,$类似.) 为了跳出这个限制, 我们可以使用\,Coleman\,和\,Mandula\,的一个技巧, 注意到, 如果$\,\sum_{\alpha}c^{\alpha}b_{\alpha}^{\sharp}(p,q)=0$, 那么方程(\ref{24.B.18})加上方程(\ref{24.B.16})给出 \[
\sum_{\alpha}c^{\alpha}\,b_{\alpha}^{\sharp}(p,q^{\prime})=0 \:,
\]%
根据方程(\ref{24.B.15}), 这使得
\[
\sum_{\alpha}c^{\alpha}\,b_{\alpha}^{\sharp}(k,p+q^{\prime}-k)=0 \:,
\]%
因此, 对于几乎所有使得$\,p+q^{\prime}-k\,$和$\,k\,$都在壳的\,4\,-动量$\,k$, 我们有
\begin{equation}
\sum_{\alpha}c^{\alpha}\,b_{\alpha}^{\sharp}(k)=0  \:.  \label{24.B.19}
\end{equation}%
现在, $q^{\prime}\,$和$\,p+q-q^{\prime}\,$都在质壳上这个条件使得$\,q^{\prime}\,$中自由参量的个数还有两个, 这使得我们在$\,q^{\prime}\,$的选择上还有足够的自由度以至于$\,p+q^{\prime}-k\,$在壳这个条件仍然可以让%
我们自由地选择$\,\mathbf{k}$, 或者至少可以在动量空间的某个有限体积内可以随便选择$\,\mathbf{k}$. 通过将$\,\mathbf{p}\,$和$\,\mathbf{q}\,$取得充分大, 这个空间可以被调成我们希望的大小, 所以, 如果对某两个固定的在壳\,4\,-动量$\,p\,$和$\,q\,$有$\,\sum_{\alpha}c^{\alpha}b_{\alpha}^{\sharp}(k)=0$, 那么对于几乎所有在壳\,4\,-动量$\,k\,$都有$\,\sum_{\alpha}c^{\alpha}b_{\alpha}^{\sharp}(k)=0$. 这样一来, 如果对某个特定的\,4\,-动量$\,k_{0}\,$有$\,\sum_{\alpha}c^{\alpha}b_{\alpha}^{\sharp}(k_{0})\neq 0$, 那么对于\,4\,-动量为$\,k_{0}\,$和$\,k\,$的两个粒子散射散射成\,4\,-动量为$\,k^{\prime}\,$和%
$\,k^{\prime\prime}\,$这样的散射过程, 对于几乎所有$\,k$, $k^{\prime}\,$和$\,k^{\prime\prime}$, 它都会被$\,\sum_{\alpha}c^{\alpha}B_{\alpha}^{\sharp}\,$生成的对称性禁止, 而这与我们对散射振幅的解析性所做的假设矛盾. 我们由此得出, 如果对某两个固定的在壳\,4\,-动量$\,p\,$和$\,q\,$有$\,\sum_{\alpha}c^{\alpha}b_{\alpha}^{\sharp}(k)=0$, 那么对于{\kai{所有}}$\,k\,$都有$\,\sum_{\alpha}c^{\alpha}b_{\alpha}^{\sharp}(k)=0$, 这使得将$\,B_{\alpha}\,$变成$\,b_{\alpha}^{\sharp}\,$的映射是个同构.

由此马上可以得到一个推论, 由于独立矩阵$\,b_{\alpha}^{\sharp}(p,q)\,$的个数不能超过%
$\,N(\sqrt{-p_{\mu}p^{\mu}})N(\sqrt{-q_{\mu}q^{\mu}})$, 所以独立的对称性生成元$\,B_{\alpha}\,$最多只有有限个. 这也是\,Coleman\,和\,Mandula\,强调的, 在证明他们的定理时, 没有必要就对称性代数是有限维这一点再单独做一假设.

15.2\,节的定理进一步告诉我们, 对于有限厄米矩阵, 例如$\,p,q\,$固定的$\,b_{\alpha}^{\sharp}(p,q)$, 它的\,Lie\,代数至多是一个半单紧致\,Lie\,代数与数个$\,U(1)\,$Lie\,代数的直和. 我们已经看到这个\,Lie\,代数同构于对称性生成元为$\,B_{\alpha}^{\sharp}\,$的那个\,Lie\,代数, 所以$\,B_{\alpha}^{\sharp}\,$张开的至多也只是一个半单紧致\,Lie\,代数与数个$\,U(1)\,$Lie\,代数的直和.

我们先来分解\,$U(1)$\,Lie\,代数. 对于任意一对在壳动量$\,p\,$和$\,q$, 我们能找到一个\,Lorentz\,生成元$\,J$ 保持它们不变. (如果$\,p\,$和$\,q\,$类光且平行, 那么选择$\,J\,$使其生成围绕$\,\mathbf{p}\,$和$\,\mathbf{q}\,$的共同方向的旋转即可. 若不是这种情况, $p+q\,$将是类时的, 那么在$\,\mathbf{p}=-\mathbf{q}\,$的质心系中围绕$\,\mathbf{p}\,$和$\,\mathbf{q}\,$%
的共同方向的旋转保持$\,p,q\,$不变, 选择$\,J\,$使其生成这个旋转即可.) 我们可以选择二粒子态的基去对角化\,$J$, 使得
\begin{equation}
J\lvert pm,qn\rangle =\sigma (m,n)\lvert pm,qn\rangle \: .  \label{24.B.20}
\end{equation}%
现在, $P_{\mu}\,$与所有$\,B_{\alpha}^{\sharp}\,$对易, 而$\,[J,P_{\mu}]\,$又是$\,P_{\mu}\,$的线性组合, 所以$\,P_{\mu}\,$与所有$\,[J,B_{\alpha}^{\sharp}]\,$对易, 因而对称性生成元$\,[J,B_{\alpha}^{\sharp}]\,$必是$\,B_{\beta}\,$的线性组合, 根据定义, 这构成了与$\,P_{\mu}\,$对易的对称性生成元的一组完备集. 更确切些, 由于表示对易子的矩阵必然是无迹的, $[J,B_{\alpha}^{\sharp}]\,$必然是$\,B_{\beta}^{\sharp}\,$的线性组合. 然而, 在$\,B_{\beta}^{\sharp}\,$的代数中, 任何$\,U(1)\,$生成元$\,B_{i}^{\sharp}\,$(取成厄米的)必与所有$\,B_{\beta}^{\sharp}\,$对易, 因此也必须与$\,[J,B_{i}^{\sharp}]\,$对易:
\[
[B_{i}^{\sharp},[J,B_{i}^{\sharp}]]=0 \:.
\]%
在$\,J\,$是对角化的二粒子基上取这个二重对易子的期望值, 那么对于任意的$\,m\,$和$\,n$, 我们有
\begin{equation}
0=\sum_{m^{\prime },n^{\prime}}\Bigl(\sigma(m^{\prime},n^{\prime})-\sigma(m,n)\Bigr)
\biggl\lvert\Bigl(b_{\alpha}^{\sharp}(p,q)\Bigr)_{m^{\prime}n^{\prime},mn}\biggr\rvert^{2} \:.\label{24.B.21}
\end{equation}%
指标的取值范围是有限的, 所以如果存在任何$\,\sigma\,$使得存在$\,\sigma(m,n)=\sigma\,$的 $m\,$和$\,n$, $\sigma(m^{\prime},n^{\prime})\neq\sigma\,$的$\,m^{\prime}\,$和$\,n^{\prime}$, 并有$(b_{i}^{\sharp}(p,q))_{m^{\prime}n^{\prime},mn}\neq0$, 那么这样的$\,\sigma\,$必须会有最小的一个, 取这个最小的$\,\sigma$, 那么方程(\ref{24.B.21})的右边对于这个$\,m\,$和$\,n\,$必然是正定的, 与方程(\ref{24.B.21})矛盾. 我们由此得出, 对于所有使得$\,\sigma(m^{\prime},n^{\prime})\neq\sigma(m,n)\,$的$\,m,n,m^{\prime},n^{\prime}$, $(b_{i}^{\sharp}(p,q))_{m^{\prime}n^{\prime},mn}\,$必须为零. 由于$\,b_{i}^{\sharp}(p,q)\,$的代数同构于$\,B_{i}^{\sharp}\,$的代数, 这意味每个\,$U(1)\,$生成元$\,B_{i}^{\sharp}\,$与$\,J\,$对易. 由于我们可以将$\,p+q\,$选在任何类时方向上, 由此得出每个\,$U(1)$\,生成元$\,B_{i}^{\sharp}\,$与齐次\,Lorentz\,群的所有生成元$\,J^{\mu\nu}\,$对易.
它们与我们在\,2.5\,节称作``增速''的变换对易这一性质暗示了$\,(b_{i}^{\sharp}(p))_{n^{\prime}n}\,$与%
\,3\,-动量独立, 而它们与旋转对易这一性质暗示了$\,(b_{i}^{\sharp}(p))_{n^{\prime}n}\,$作为%
单位矩阵作用在自旋指标上, 所以这些生成元是普通内部对称性的生成元.

剩下的$\,B_{\alpha}^{\sharp}\,$生成了半单紧致\,Lie\,代数. 24.1\,节的讨论(比\,Coleman\,和\,Mandula\,给出的推导稍微明显一些)告诉我们, Lie\,代数的半单紧致部分的生成元与\,Lorentz\,变换对易, 而正如对\,$U(1)\,$生成元所证明的, 这意味着它们也是内部对称性的生成元. 因此我们证明了, 与$\,P_{\mu}\,$对易的生成元$\,B_{\alpha}\,$不是内部对称性的生成元就是$\,P_{\mu}\,$自身的线性组合.

接下来, 我们必须要考虑对称性生成元与动量算符不对易的可能性. 一般对称性的生成元$\,A_{\alpha}$ 在$\,4\,$-动量为$\,p\,$的单粒子态$\,\lvert p\,n\rangle\,$上的作用是
\begin{equation}
A_{\alpha }\,\lvert p,n\rangle = \sum_{n^{\prime}}\int \dif^{4}p^{\prime }\:
\Bigl(\mathscr{A}_{\alpha}(p^{\prime},p)\Bigr)_{n^{\prime}n}\,\lvert p^{\prime},n^{\prime}\rangle \:,  \label{24.B.22}
\end{equation}%
其中$\,n\,$和$\,n^{\prime}\,$依旧是标记自旋\,$z$\,-分量和粒子种类的离散指标. 当然, 除非$\,p\,$和$\,p^{\prime}\,$都在质壳上, 否则核$\,\mathscr{A}_{\alpha}(p^{\prime},p)\,$必须为零. 我们会先证明任何$\,p^{\prime}\neq p\,$的$\,\mathscr{A}_{\alpha}(p^{\prime},p)\,$为零.

对于这个目的, 注意到, 如果$\,A_{\alpha}\,$是对称性生成元, 那么
\begin{equation}
A_{\alpha}^{f}\equiv \int \dif^{4}x\,\exp(\mi \,P\cdot x)\,A_{\alpha }\,\exp(-\mi\,P\cdot x)\,f(x)  \label{24.B.23}
\end{equation}%
也是对称性生成元, 其中$\,P_{\mu}(x)\,$是\,4\,-动量算符, $f(x)\,$是一个我们可以任意选择的函数. 这个生成元作用在单粒子态上给出
\begin{equation}
A_{\alpha }^{f}\,\lvert p,n\rangle =\sum_{n^{\prime }}\int \dif^{4}p^{\prime }\, \tilde{f%
}(p^{\prime }-p)\Bigl( \mathscr{A}_{\alpha }(p^{\prime },p)\Bigr) _{n^{\prime
}n}\lvert p^{\prime },n^{\prime }\rangle \:,  \label{24.B.24}
\end{equation}%
其中$\,\tilde{f}\,$是\,Fourier\,变换
\begin{equation}
\tilde{f}(k)\equiv \int \dif^{4}x\,\exp (\mi x\cdot k)\,f(x) \:.  \label{24.B.25}
\end{equation}%
假定存在一对在壳\,4\,-动量$\,p\,$和$\,p+\Delta$, 其中$\,\Delta\neq 0$, 这对\,4\,-动量使得$\,\mathscr{A}(p+\Delta,p)\neq0$. 对于满足$\,p^{\prime}+q^{\prime}=p+q\,$的在壳\,4\,-动量$\,q$, $p^{\prime}\,$和$\,q^{\prime}$, $q+\Delta\,$或者$\,p^{\prime}+\Delta\,$或者$\,q^{\prime}+\Delta\,$一般不会在壳. 如果我们取$\,\tilde{f}(k)\,$仅在$\,\Delta\,$附近一块充分小的区域内不为零, 那么$\,A_{\alpha}^{f}\,$将会湮灭所有\,4\,-动量为$\,q,p^{\prime},q^{\prime}\,$的单粒子态, 但是它不会湮灭\,4\,-动量为$\,p\,$的单粒子态, 所以, 对于任何散射过程, 只要其中有粒子的动量从$\,p\,$和$\,q\,$变成$\,p^{\prime}\,$和$\,q^{\prime}$, 这样的散射过程就会被这个对称性禁止, 与假设\,2\,和\,3\,的``在几乎所有能标和所有角度处都存在散射''的结果相矛盾.

因为核$\,\mathscr{A}_{\alpha}(p^{\prime},p)\,$可能会包含正比于$\,\delta^{4}(p^{\prime}-p)\,$的导数的项%
以及正比于$\,\delta^{4}(p^{\prime}-p)\,$自身的项, 这一结果并不意味着所有对称性生成元$\,A_{\alpha}\,$都要与$\,P_{\mu}\,$对易. 为了处理这种情况, Coleman\,和\,Mandula 做了``丑陋的技术性假定'': 核$\,\mathscr{A}_{\alpha}(p^{\prime},p)\,$是{\kai{分布}}, 这意味着核中包含的$\,\delta^{4}(p^{\prime}-p)\,$的导数至多是有限的$\,D_{\alpha}\,$阶导数. 换句话说, 就是假定每个对称性生成元$\,A_{\alpha}\,$在单粒子态上的作用是求导运算$\,\partial/\partial p_{\mu}$ 的$\,D_{\alpha}\,$次多项式, 其中在这一点的矩阵系数允许依赖于动量和自旋. 为了使用上面对称性生成元与动量算符对易的结果, Coleman\,和\,Mandula\,考虑了动量算符对$\,A_{\alpha}\,$的$\,D_{\alpha}\,$重对易子
\begin{equation}
B_{\alpha }^{\mu _{1}\cdots \mu _{D_{\alpha }}}\equiv \lbrack P^{\mu_{1}},[P^{\mu _{2}},\cdots [P^{\mu_{D_{\alpha}}},A_{\alpha}]]\cdots] \: .  \label{24.B.26}
\end{equation}%
在\,4\,动量为$\,p^{\prime}\,$和$\,p\,$的态上, $B_{\alpha}^{\mu_{1}\cdots\mu_{D_{\alpha}}}\,$与$\,P^{\mu}\,$的对易子的矩阵元正比于\,$p^{\prime}-p\,$的%
$\,D_{\alpha}+1\,$次方, 再乘以动量导数的$\,D_{\alpha}\,$次多项式作用在$\,\delta^{4}(p^{\prime}-p)\,$上的结果, 因此为零. 由于生成元$\,B_{\alpha}^{\mu_{1}\cdots\mu_{D_{\alpha}}}\,$与动量算符对易, 根据上面获得的结果, 它们在单粒子态上的作用是如下形式的矩阵
\begin{equation}
b_{\alpha }^{\mu _{1}\cdots \mu _{D_{\alpha }}}(p)=b_{\alpha}^{\sharp\mu_{1}\cdots\mu_{D_{\alpha}}}
+a_{\alpha}^{\mu\mu_{1}\cdots\mu_{D_{\alpha}}}p_{\mu}\,1\:,\label{24.B.27}
\end{equation}%
其中$\,b_{\alpha}^{\sharp\mu_{1}\cdots\mu_{D_{\alpha}}}\,$是与动量无关的无迹厄米矩阵, 它们生成了通常的内部对称性代数, $a_{\alpha}^{\mu\mu_{1}\cdots\mu_{D_{\alpha}}}\,$是与动量无关的数值常数, $b_{\alpha}^{\sharp\mu_{1}\cdots\mu_{D_{\alpha}}}\,$和$\,a_{\alpha}^{\mu\mu_{1}\cdots\mu_{D_{\alpha}}}\,$%
关于指标$\,\mu_{1},\cdots\mu_{D_{\alpha}}\,$均是对称的. 另外, 尽管$\,A_{\alpha}\,$不一定与$\,P_{\mu}\,$对易, 但是它们不能使单粒子态离壳, 而假设\,1\,又要求指标质量平方算符$\,-P_{\mu}P^{\mu}\,$仅有离散本征值, 所以$\,A_{\alpha}\,$必须与$\,-P_{\mu}P^{\mu}\,$对易. 特别地, 对于$\,D\geq1\,$可以得出
\[
0=[P^{\mu _{1}}P_{\mu _{1}},[P^{\mu _{2}},\cdots \lbrack P^{\mu _{D_{\alpha
}}},A_{\alpha }]]\cdots ]=2P_{\mu _{1}}B_{\alpha }^{\mu _{1}\cdots\mu_{D_{\alpha}}} \:,
\]%
这使得
\begin{equation}
0=p_{\mu _{1}}b_{\alpha }^{\mu _{1}\cdots \mu _{D_{\alpha }}}(p)\:. \label{24.B.28}
\end{equation}%
只要理论包含有质量粒子, 对于处在任何类时方向的$\,p$, 这都必须满足, 所以对于$\,D_{\alpha}\geq1$
\begin{equation}
b_{\alpha }^{\sharp \mu _{1}\cdots \mu _{D_{\alpha }}}=0 \:,  \label{24.B.29}
\end{equation}%
且
\begin{equation}
a_{\alpha}^{\mu\mu_{1}\cdots\mu_{D_{\alpha}}}=-a_{\alpha}^{\mu_{1}\mu\cdots\mu_{D_{\alpha}}} \:.  \label{24.B.30}
\end{equation}%
但是当$\,D_{\alpha}\geq2\,$时, 加上$\,a_{\alpha}^{\mu\mu_{1}\cdots\mu_{D_{\alpha}}}\,$关于指标$\,\mu_{1},\cdots,\mu_{D_{\alpha}}\,$的对称性, 方程(\ref{24.B.30})将要求$\,a_{\alpha}^{\mu\mu_{1}\cdots\mu_{D_{\alpha}}}\\=0$. (这时$\,a_{\alpha}^{\mu\mu_{1}\mu_{2}\cdots\mu_{D_{\alpha}}}=
a_{\alpha}^{\mu\mu_{2}\mu_{1}\cdots\mu_{D_{\alpha}}}=-a_{\alpha}^{\mu_{2}\mu\mu_{1}\cdots\mu_{D_{\alpha}}}
=-a_{\alpha}^{\mu_{2}\mu_{1}\mu\cdots\mu_{D_{\alpha}}}=a_{\alpha}^{\mu_{1}\mu_{2}\mu\cdots\mu_{D_{\alpha}}}
=a_{\alpha}^{\mu_{1}\mu\mu_{2}\cdots\mu_{D_{\alpha}}}\\=-a_{\alpha}^{\mu\mu_{1}\mu_{2}\cdots\mu_{D_{\alpha}}}$.)
因此我们最多还有两种非零的对称性生成元: $\,D_{\alpha}=0\,$的, 这时生成元$\,A_{\alpha}\,$与 $P_{\mu}\,$对易, 因此它不是内部对称性的生成元就是$\,P_{\mu}\,$的某个线性组合; 以及$\,D_{\alpha}=1\,$的, 在这种情况下,
\begin{equation}
[P^{\nu },A_{\alpha}]=a_{\alpha}^{\mu\nu}P_{\mu}\:, \label{24.B.31}
\end{equation}%
其中$\,a_{\alpha}^{\mu\nu}\,$是一些关于$\,\mu\,$和$\,\nu\,$反对称的数值常数. 方程(\ref{24.B.31})要求
\begin{equation}
A_{\alpha }=-\tfrac{1}{2}\mi a_{\alpha}^{\mu\nu}J_{\mu\nu}+B_{\alpha}\:,  \label{24.B.32}
\end{equation}%
其中$\,J_{\mu\nu}\,$是固有\,Lorentz\,变换的生成元, 根据方程(\textcolor{foo}{2.4.13})它满足$\,[P^{\nu},J^{\rho\sigma}]
=-\mi\eta^{\nu\rho}P^{\sigma}+\mi\eta^{\nu\sigma}P^{\rho}$, $B_{\alpha}\,$与$\,P_{\mu}\,$对易. 由于$\,A_{\alpha}\,$和$\,J_{\mu\nu}\,$是对称性生成元, 所以$\,B_{\alpha}\,$也是, 因此它必须是内部对称性生成元和(或)$\,P_{\mu}\,$分量的线性组合. 方程(\ref{24.B.32})因此补完了\,Coleman-Mandula\,定理的证明.

\subsection*{* * *}

在只有无质量粒子的理论中, 方程(\ref{24.B.28})不一定只能给出方程(\ref{24.B.30}); 由于$\,p_{\mu}p^{\mu}=0$, 我们还可以有
\begin{equation}
a_{\alpha}^{\mu \mu_{1}\cdots \mu_{D_{\alpha}}}+a_{\alpha}^{\mu_{1}\mu
\cdots \mu_{D_{\alpha}}}\propto\eta^{\mu\mu_{1}} \:. \label{24.B.33}
\end{equation}%
在这一情况下, 对称性代数由内部对称性和共形群的代数构成, 其中共形群由生成元$\,K_{\mu}\,$和$\,D\,$以及\,Poincar\'{e}\,群的生成元$\,J^{\mu\nu}\,$和$\,P^{\mu}\,$张开. 对易关系是
\begin{align}
&[P^{\mu},D] =\mi P^{\mu} \:, \qquad [K^{\mu},D]=-\mi K^{\mu} \:,  \nonumber \\
&[P^{\mu},K^{\nu}] =2\mi\eta^{\mu\nu}D + 2\mi J^{\mu\nu} \:,
 \qquad [K^{\mu},K^{\nu}]=0 \:, \label{24.B.34} \\
&[J^{\rho\sigma},K^{\mu}] =\mi\eta^{\mu\rho}K^{\sigma}-\mi\eta^{\mu\sigma}K^{\rho} \:, \qquad
[J^{\rho\sigma},D]=0 \:,  \nonumber
\end{align}%
和\,Poincar\'{e}\,代数的对易关系(\textcolor{foo}{2.4.12})---(\textcolor{foo}{2.4.14})
\begin{align}
\mi[J^{\mu\nu},J^{\rho\sigma}] &=\eta^{\nu \rho }J^{\mu \sigma }-\eta
^{\mu \rho }J^{\nu \sigma }-\eta ^{\sigma \mu }J^{\rho \nu }+\eta ^{\sigma
\nu }J^{\rho \mu } \:,  \nonumber \\
\mi[P^{\mu },J^{\rho \sigma }] &=\eta ^{\mu \rho }P^{\sigma }-\eta ^{\mu
\sigma }P^{\rho }\:,  \label{24.B.35} \\
[P^{\mu },P^{\rho }]& =0 \:.  \nonumber
\end{align}%
无穷小群元
\begin{equation}
U(1+\omega ,\epsilon ,\lambda ,\rho )=1+(\mi/2)J_{\mu \nu }\omega ^{\mu \nu
}+\mi P_{\mu }\epsilon ^{\mu }+\mi\lambda D+\mi K_{\mu }\rho ^{\mu }
\label{24.B.36}
\end{equation}%
诱导出了无穷小时空变换
\begin{equation}
x^{\mu }\to x^{\mu }+\omega ^{\mu \nu }x_{\nu }+\epsilon ^{\mu
}+\lambda x^{\mu }+\rho ^{\mu }x^{\nu }x_{\nu }-2x^{\mu }\rho ^{\nu }x_{\nu }\:.  \label{24.B.37}
\end{equation}
这些是最一般的保持光锥不变的无穷小时空变换.



\section*{习题}
\noindent 1. 在所有粒子都是无质量的情况下,对于\,Coleman-Mandula\,定理的假设所允许的最一般对称性代数, 证明组成这个代数的不是内部对称性生成元加上\,Poincar\'{e}\,代数%
就是内部对称性生成元加上共形代数(\ref{24.B.34}), (\ref{24.B.35}).\\

\noindent 2. 证明\,Gervais-Sakita\,作用量(\ref{24.2.5})在世界面超对称变换(\ref{24.2.7})下不变.\\

\noindent 3. 计算\,Wess-Zumino\,拉格朗日密度(\ref{24.2.9})在时空超对称变换(\ref{24.2.8})下的变化.

%++++++++++++++++++参考文献+++++++++
\renewcommand{\sectionmark}[1]{\markright{ #1}{}}
\renewcommand{\bibname}{参考文献}

\begin{thebibliography}{99}
\bibitem{1} B. Sakita, \textit{Phys. Rev.} {\bf{136}}, B 1756 (1964); F. Gursey and L. A. Radicati, \textit{Phys. Rev. Lett.} {\bf{13}}, 173 (1964); A. Pais, \textit{Phys. Rev. Lett.} {\bf{13}}, 175 (1964); F. Gursey, A. Pais and L. A. Radicati, {\textit{Phys. Rev. Lett.}} {\bf{13}}, 299 (1964). 这些文章翻印于\,\textit{Symmetry Groups in Nuclear and Particle Physics}, F. J. Dyson, ed. (W. A. Benjamin, New York, 1966), 附带还有\,Dyson\,关于这一主题的一套课堂笔记.
\bibitem{2} E. P. Wigner, {\textit{Phys. Rev.}} {\bf{51}}, 106 (1937). 翻印于\,\textit{Symmetry Groups in Nuclear and Particle Physics}, 参考文献[1].
\bibitem{3} A. Salam, R. Delbourgo, and J. Strathdee, {\textit{Proc. Roy. Soc. (London)}} {\bf{A 284}}, 146 (1965); M. A. Beg and A. Pais, {\textit{Phys. Rev. Lett.}} {\bf{14}}, 267 (1965); B. Sakita and K. C. Wali, {\textit{Phys. Rev.}} {\bf{139}}, B 1355 (1965). 这些文章翻印于\,\textit{Symmetry Groups in Nuclear and Particle Physics}, 参考文献[1].
\bibitem{4} W. D. McGlinn, {\textit{Phys. Rev. Lett.}} {\bf{12}}, 467 (1964); O. W. Greenberg, {\textit{Phys. Rev.}} {\bf{135}}, B 1447 (1964); L. Michel, {\textit{Phys. Rev.}} {\bf{137}}, B 405 (1964); L. Michel and B. Sakita, {\textit{Ann. Inst. Henri-Poincar\'{e}}} {\bf{2}}, 167 (1965); M. A. B. Beg and A. Pais, {\textit{Phys. Rev. Lett.}} {\bf{14}}, 509, 577 (1965); S. Coleman, {\textit{Phys. Rev.}} {\bf{138}}, B 1262 (1965); S. Weinberg, {\textit{Phys. Rev.}} {\bf{139}}, B 597 (1965); L. O'Raifeartaigh, {\textit{Phys. Rev.}} {\bf{139}}, B 1052 (1065). 这些文章翻印于\,\textit{Symmetry Groups in Nuclear and Particle Physics}, 参考文献[1].
\bibitem{5} S. Coleman and J. Mandula, {\textit{Phys. Rev.}} {\bf{159}}, 1251 (1967).
\bibitem{6} 关于原始文献的简介和引用, 参看\,M. B. Green, J. H. Schwarz, and E. Witten, {\textit{Superstring Theory}} (Cambridge University Press, Cambridge, 1987); J. Polchinski, {\textit{String Theory}} (Cambridge University Press, Cambridge, 1998).
\bibitem{7} P. Ramond, {\textit{Phys. Rev.}} {\bf{D3}}, 2415 (1971). 这篇文章翻印于\,{\textit{Superstrings --- The First 15 Years of Superstring Theory,}} J. H. Schwarz, ed. (World Scientific, Singapore, 1985).
\bibitem{8} A. Neveu and J. H. Schwarz, {\textit{Nucl. Phys.}} {\bf{B31}}, 86 (1971); {\textit{Phys. Rev.}} {\bf{D4}}, 1109 (1971). 这些文章翻印于\,{\textit{Superstrings --- The First 15 Years of Superstring Theory,}} 参考文献[7]. 另见\,Y. Aharonov, A. Casher, and L. Susskind, {\textit{Phys. Rev.}} {\bf{D5}}, 988 (1972).
\bibitem{9} J.-L. Gervais and B. Sakita, {\textit{Nucl. Phys.}} {\bf{B34}}, 632 (1971). 这篇文章翻印于\,{\textit{Superstrings --- The First 15 Years of Superstring Theory,}} 参考文献[7].
\bibitem{10} J. Wess and B. Zumino, {\textit{Nucl. Phys.}} {\bf{B70}}, 39 (1974). 这篇文章翻印于\,{\textit{Supersymmetry}}, S. Ferrara, ed. (North Holland/World Scientific, Amsterdam/Singapore, 1987).
\bibitem{11} J. Wess and B. Zumino, {\textit{Phys. Lett.}} {\bf{49B}}, 52 (1974). 这篇文章翻印于\,{\textit{Supersymmetry}}, 参考文献[10].
\bibitem[11a]{11a} F. Gliozzi, J. Scherk, and D. Olive, {\textit{Nucl. Phys.}} {\bf{B122}}, 253 (1977).
\bibitem{12} Yu. A. Gol'fand and E. P. Likhtman, {\textit{JETP Letters}} {\bf{13}}, 323 (1971). 这篇文章翻印于\,{\textit{Supersymmetry}}, 参考文献[10].
\bibitem{13} D. V. Volkov and V. P. Akulov, {\textit{Phys. Lett.}} {\bf{46B}}, 109 (1973). 这篇文章翻印于\,{\textit{Supersymmetry}}, 参考文献[10].
\bibitem{14} M. Gell-Mann, Cal. Tech. Synchotron Laboratory Report CTSL-20 (1961), 未发表. \,Gell-Mann\,和\,Y. Ne'eman\,在一些关于$\,SU(3)\,$对称性的文章中加入了这篇文章, {\textit{The Eightfold Way}} (Benjamin, New York, 1964).
\bibitem{15} R. Haag, J. T. Lopuszanski, and M. Sohnius, {\textit{Nucl. Phys.}} {\bf{B88}}, 257 (1975). 这篇文章翻印于\,{\textit{Supersymmetry}}, 参考文献[10].
\end{thebibliography}


\renewcommand\thesection{\thechapter.\arabic{section}}
%===============================================================================%
\fancyhf{}
\fancyhead[CE]{\leftmark}
\fancyhead[CO]{\rightmark}
\fancyhead[RO,LE]{$\cdot$\ \thepage\ $\cdot$}
\renewcommand{\headrulewidth}{0.8pt}
\renewcommand\theequation{\arabic{chapter}.\arabic{section}.\arabic{equation}}
\pagestyle{fancy}
\renewcommand{\chaptermark}[1]{\markboth{第\,\thechapter\,章\ #1}{}}
\renewcommand{\sectionmark}[1]{\markright{\thesection\ #1}{}}
%===========================章节的标题格式====================================%
\titleformat{\chapter}{\centering\CJKfamily{zhhei}\huge}{\chaptertitlename}{1em}{}
\titlespacing{\chapter}{0pt}{3.5ex plus .1ex minus .2ex}{10\wordsep}
\titleformat{\section}{\centering\CJKfamily{zhhei}\Large}{\thesection}{1em}{}
\titlespacing{\section}{2em}{3.5ex plus .1ex minus .2ex}{1.5\wordsep}
\titleformat{\subsection}{\centering\CJKfamily{zhkai}\large}{}{0em}{}
\titlespacing{\subsection}{2em}{1.5ex plus .1ex minus .2ex}{\wordsep}
\renewcommand{\captionfont}{\small}

%第二十五章 超对称代数
\chapter{超对称代数} \label{cha:25}

这一章将会从第一原理出发, 沿用\,Haag, Lopuszanski\,和\,Sohnius\,的处理,\cite{1} 来发展超对称代数的形式. 我们将会看到, 在\,Coleman-Mandula\,定理成立的条件下, 这一结构几乎被\,Lorentz\,不变的要求完全确定了. 在这之后, 超多重态的结构就可以直接从超对称代数中推出.


\section{阶化\,Lie\,代数和阶化参量} \label{sec:25.1}

我们在\,2.2\,节看到了如何将任意连续对称变换写成\,Lie\,代数的形式, 这个\,Lie\,代数由线性独立的对称性生成元$\,t_{a}\,$生成, 并且生成元满足对易关系$\,[t_{a},t_{b}]=\mi\sum_{c}C_{ab}^{c}t_{c}$. 以非常类似的方式, 用来表示超对称的生成元$\,t_{a}\,$构成了{\kai{阶化}}Lie代数,\cite{2} 这样的\,Lie\,代数表现为如下形式的对易关系{\kai{和}}反对易关系
\begin{equation}
t_{a}t_{b}-(-1)^{\eta_{a}\eta_{b}}t_{b}t_{a}=\mi\sum_{c}C_{ab}^{c}t_{c}\:.\label{25.1.1}
\end{equation}
(本节不使用求和决定.) 对于每个$\,a$, $\eta_{a}\,$是$\,+1\,$或$\,0$, 它是生成元$\,t_{a}\,$的{\kai{阶数}}, 而$\,C_{ab}^{c}\,$是一组数值的结构常数. $\eta_{a}=1\,$的生成元$\,t_{a}\,$被称为{\kai{费米的}}; 其它那些$\,\eta_{a}=0\,$的生成元则被称为{\kai{玻色的}}. 对于玻色算符和玻色算符以及玻色算符和费米算符, 方程(\ref{25.1.1})提供了对易关系, 而对于费米算符和费米算符, 它则提供了反对易关系. 我们暂且先来看一下它对结构常数产生的影响, 在此之后再回到提出方程(\ref{25.1.1})的动机.

根据方程(\ref{25.1.1}), 结构常数必须满足条件
\begin{equation}
C_{ab}^{c}=-(-1)^{\eta_{a}\eta_{b}}C_{ab}^{c}\:.\label{25.1.2}
\end{equation}
对于任何由场算符的泛函构成的算符, 两个玻色算符的乘积或者两个费米算符的乘积是玻色的, 而一个费米算符与一个玻色算符的乘积是费米的, 这使得
\begin{equation}
C_{ab}^{c}=0\quad\text{除非}\quad \eta^{c}=\eta^{a}+\eta^{b}\:(\operatorname{mod} 2)\:. \label{25.1.3}
\end{equation}
另外, 对于任何以这种方式构建的算符, 玻色算符和费米算符的厄米伴分别是玻色的和费米的. 如果$\,t_{a}\,$是厄米算符, 那么结构常数满足实条件
\begin{equation}
{C_{ab}^{c}}^{\ast}=-C_{ba}^{c}\:.\label{25.1.4}
\end{equation}

结构常数同时还满足一个非线性约束, 这个约束来自于超\,Jacobi\,恒等式
\begin{equation}
(-1)^{\eta_{c}\eta_{a}}[[t_{a},t_{b}\},t_{c}\}+(-1)^{\eta_{a}\eta_{b}}[[t_{b},t_{c}\},t_{a}\}
+(-1)^{\eta_{b}\eta_{c}}[[t_{c},t_{a}\},t_{b}\}=0\:.   \label{25.1.5}
\end{equation}
这里的``$[\cdots\}$''类似于方程(\ref{25.1.1})左边出现的对易子/反对易子, 但在这里推广至任意的阶化算符 $O,O^{\prime}\,\cdots$
\begin{equation}
[O,O^{\prime}\}\equiv OO^{\prime}-(-1)^{\eta(O)\eta(O^{\prime})}O^{\prime}O=-(-1)^{\eta(O)\eta(O^{\prime})}[O^{\prime},O\}\:, \label{25.1.6}
\end{equation}
现在它被理解成, 生成元的任意乘积$\,O=t_{a}t_{b}t_{c}\cdots\,$被赋予阶数$\,\eta(O)\equiv \eta_{a}+\eta_{b}+\eta_{c}+\cdots(\operatorname{mod}2)$. (为证明方程(\ref{25.1.5}), 只需证明$\,t_{a}t_{b}t_{c}\,$和$\,t_{a}t_{c}t_{b}\,$的系数为零即可, 至于生成元的其它乘积, 方程(\ref{25.1.5}) 左边在轮换$\,abc\to bca\to cab\,$下的对称性会确保它们的系数为零. $t_{a}t_{b}t_{c}\,$在方程(\ref{25.1.5})中的系数是
\[
(-1)^{\eta_{c}\eta_{a}}-(-1)^{\eta_{a}\eta_{b}}(-1)^{\eta_{a}(\eta_{b}+\eta_{c})}=0\:,
\]
而$\,t_{a}t_{c}t_{b}\,$的系数是
\[
(-1)^{\eta_{a}\eta_{b}}(-1)^{\eta_{b}\eta_{c}}(-1)^{\eta_{a}(\eta_{b}+\eta_{c})}
-(-1)^{\eta_{b}\eta_{c}}(-1)^{\eta_{c}\eta_{a}}=0\:,
\]
证毕.) 将方程(\ref{25.1.1})代入方程(\ref{25.1.5}), 我们发现约束
\begin{equation}
\sum_{d}(-1)^{\eta_{c}\eta_{a}}C_{ab}^{d}C_{dc}^{e}+\sum_{d}(-1)^{\eta_{a}\eta_{b}}C_{bc}^{d}C_{da}^{e}
+\sum_{d}(-1)^{\eta_{b}\eta_{c}}C_{ca}^{d}C_{db}^{e}=0 \:.\label{25.1.7}
\end{equation}
当然, 在所有生成元都是玻色生成元的情况下, 方程(\ref{25.1.5})就是通常的\,Jacobi\,恒等式, 而方程 (\ref{25.1.7}) 就是结构常数之间通常的非线性约束(\textcolor{foo}{2.2.22}).

方程(\ref{25.1.1})可以取作我们的出发点, 但是就像在\,2.2\,节我们对普通\,Lie\,代数所做的那样, 我们可以赋予它一个动机, 这样它就不是出发点而是有限连续对称变换的一个必要特征. 与\,2.2\,节不同的是, 现在这些变换依赖于连续的{\kai{阶}}参量. 一组阶化\,c\,-数参量可以视为``数'', 这些数既包含格拉斯曼参量(参看\,9.5\,节)也包含普通数, 它们满是算术的结合律和分配率, 但是不再满足简单的交换律, 而是满足关系
\begin{equation}
\alpha^{a}\beta^{b}=(-1)^{\eta_{a}\eta_{b}}\beta^{b}\alpha^{a}\:, \label{25.1.8}
\end{equation}
其中$\,\alpha^{a},\,\beta^{a},\cdots\,$用来区分第$\,a\,$个参量的不同值, 以矢量代数中的方法, 我们可以用$\,v^{a}\,$和$\,u^{a}\,$来标记两个不同实矢量的$\,a\,$-分量. 和以前一样, 第$\,a\,$个阶化参量被赋予阶数$\,\eta_{a}\,$, 当$\,\alpha^{a}\,$分别是费米参量和玻色参量时, $\eta_{a}\,$分别等于$\,+1\,$和$\,0$. 即, 如果这些参量中有一个是玻色的, 那么它们就是对易的, 如果两个参量都是费米的, 那么它们就是反对易的. 阶化参量的乘积$\,\alpha^{a}\beta^{b}\gamma^{c}\cdots\,$被赋予阶数$\,\eta_{a}+\eta_{b}+\eta_{c}+\cdots\:%
(\operatorname{mod}2)$; 即, 如果这个乘积中包含奇数个费米参量, 那么它就是费米的, 否则就是玻色的. 有了这个阶数, 很容易看到阶化参量的乘积满足的对易规则或反对易规则类似于方程(\ref{25.1.8}).

考察这样的连续变换$\,T_{\alpha}$, 在形式上它由阶化参量$\,\alpha^{a}\,$的幂级数给出:
\begin{equation}
T(\alpha)=1+\sum_{a}\alpha^{a}t_{a}+\sum_{ab}\alpha^{a}\alpha^{b}t_{ab}+\cdots \:, \label{25.1.9}
\end{equation}
其中$\,t_{a}\,$, $\,t_{ab}\,$等是一组与$\,\alpha\,$无关的算符系数, 这时我们还没有假定它们要满足任何像方程(\ref{25.1.1})这样的代数关系. 由于参量$\,\alpha^{a}\,$满足方程(\ref{25.1.8}), 系数$\,t_{ab\cdots}\,$必须要满足对称/反对称条件, 例如
\begin{equation}
t_{ab}=(-1)^{\eta_{a}\eta_{b}}t_{ba}\:.\label{25.1.10}
\end{equation}
同时假定变换$\,T(\beta)\,$与任意阶化参量的任意值$\,\alpha^{a}\,$对易, 这将会方便我们的讨论, 在这一情况下, (\ref{25.1.9})中的算符系数满足条件
\begin{equation}
\alpha^{a}t_{b}=(-1)^{\eta_{a}\eta_{b}}t_{b}\alpha^{a}\:,\label{25.1.11}
\end{equation}
\begin{equation}
\alpha^{a}t_{bc}=(-1)^{\eta_{a}(\eta_{b}+\eta_{c})}t_{bc}\alpha^{a}\:.\label{25.1.12}
\end{equation}
即, 在$\,t_{a}\,$和$\,t_{bc}\,$与阶化参量满足的对易关系和反对易关系中, 它们自身分别就像是阶数分别为$\,\eta_{b}\,$和 $\eta_{b}+\eta_{c}\,(\operatorname{mod}2)\,$的阶化参量.

算符上的其它约束来自于$\,T(\alpha)\,$构成半群的要求; 即, 对于阶化参量取不同值$\,\alpha\,$和$\,\beta\,$时的$\,T\,$算符, 它们的乘积也是一个$\,T\,$算符
\begin{equation}
T(\alpha)T(\beta)=T(f(\alpha,\beta))\:,\label{25.1.13}
\end{equation}
其中$\,f^{c}(\alpha,\beta)\,$本身是阶化参量的形式幂级数. 由于$\,T(0)T(\beta)=T(\beta)\,$以及$\,T(\alpha)T(0)=T(\alpha)$, 我们必须有
\begin{equation}
f^{c}(0,\beta)=\beta^{c}\:,\qquad\qquad f^{c}(\alpha,0)=\alpha^{c}\:,\label{25.1.14}
\end{equation}
因此$\,f(\alpha,\beta)\,$的幂级数展开必须采取如下的形式
\begin{equation}
f^{c}(\alpha,\beta)=\alpha^{c}+\beta^{c}+\sum_{ab}f_{ab}^{c}\,\alpha^{a}\,\beta^{b}+\cdots\:,\label{25.1.15}
\end{equation}
其中$\,f_{ab}^{c}\,$是一组普通常数(即, 玻色常数), 而``$\cdots$''代表阶化参量的三阶项或者更高阶项. 为了使 $f^{c}(\alpha,\beta)\,$是阶化参量, 方程(\ref{25.1.15})的每一项必须要有相同的阶数, 这意味着
\begin{equation}
f_{ab}^{c}=0\quad{\text{除非}}\quad \eta^{c}=\eta^{a}+\eta^{b}\:(\operatorname{mod}2)\:.\label{25.1.16}
\end{equation}
将幂级数(\ref{25.1.9})和(\ref{25.1.15})代入乘积规则(\ref{25.1.13}), 这给出
\begin{align*}
&\Bigl[1+\sum_{a}\alpha^{a}t_{a}+\sum_{ab}\alpha^{a}\alpha^{b}t_{ab}+\cdots\Bigr]
\Bigl[1+\sum_{a}\beta^{a}t_{a}+\sum_{ab}\beta^{a}\beta^{b}t_{ab}+\cdots\Bigr]  \\
&\quad=1+\sum_{c}\Bigl(\alpha^{c}+\beta^{c}+\sum_{ab}f_{ab}^{c}\alpha^{a}\beta^{b}+\cdots \Bigr)t_{c}\\
&\qquad +\sum_{cd}\Bigl(\alpha^{c}+\beta^{c}+\cdots\Bigr)\Bigl(\alpha^{d}+\beta^{d}+\cdots\Bigr)t_{cd}+\cdots\:.
\end{align*}
$1,\,\alpha^{a},\,\beta^{a},\,\alpha^{a}\alpha^{b}\,$和$\,\beta^{a}\beta^{b}\,$的系数在方程两边自动匹配, 而要求$\,\alpha^{a}\beta^{b}\,$的系数相等这个条件给出了不平庸的关系
\begin{equation}
(-1)^{\eta_{a}\eta_{b}}t_{a}t_{b}=\sum_{c}f_{ab}^{c}t_{c}+t_{ab}+(-1)^{\eta_{a}\eta_{b}}t_{ba}
=\sum_{c}f_{ab}^{c}t_{c}+2t_{ab}\:.\label{25.1.17}
\end{equation}
(左边的符号因子来自于$\,t_{a}\,$和$\,\beta^{b}\,$的交换.) 加上同一类的高阶关系, 如果我们知道生成元$\,t_{a}\,$和群组合函数$\,f^{a}(\alpha,\beta)$, 这将使得我们
可以计算出整个函数(\ref{25.1.9}). 但为了使这个计算是可能的, $t_{a}\,$必须要满足一个约束. 利用方程(\ref{25.1.10}), 方程(\ref{25.1.17})与交换$\,a,b\,$后的同一方程的差或和给出\,Lie\,超代数关系(\ref{25.1.1}), 而结构常数给定为
\begin{equation}
\mi\,C_{ab}^{c}=(-1)^{\eta_{a}\eta_{b}}f_{ab}^{c}-f_{ba}^{c}\:.\label{25.1.18}
\end{equation}
另外, 从方程(\ref{25.1.16})和(\ref{25.1.18})就立即得出了方程(\ref{25.1.3}).

对于反对易\,c\,-数$\,\alpha\,$的复共轭$\,\alpha^{\ast}$, 它的定义要使得$\,\alpha\,$与任意算符$\,\mathcal{O}\,$的乘积的厄米共轭是
\begin{equation}
(\alpha\mathcal{O})^{\ast}=\mathcal{O}^{\ast}\alpha^{\ast}\:.\label{25.1.19}
\end{equation}
由此得出\,c\,-数在复共轭下的行为与算符在厄米共轭下的行为相同:
\begin{equation}
(\alpha\beta)^{\ast}=\beta^{\ast}\alpha^{\ast}\:, \label{25.1.20}
\end{equation}
并且$\,\alpha^{\ast}\,$与$\,\alpha\,$的阶数相同.

阶化\,Lie\,代数对物理的意义被时空对称性严格限制了. 我们现在转向对这些约束的考察.

\section{超对称代数} \label{sec:25.2}

考察一个对称性生成元与$S$-矩阵对易的一般阶化\,Lie\,代数. 如果$\,Q\,$是其中的一个费米对称性生成元, 那么$\,U^{-1}(\Lambda)\,Q\,U(\Lambda)\,$也是, 其中$\,U(\Lambda)\,$是任意齐次\,Lorentz\,变换$\,\Lambda\indices{^\mu_{\nu}}\,$对应的量子力学算符. 因此$\,U^{-1}(\Lambda)\,Q\,U(\Lambda)\,$是费米对称性生成元完备集的线性组合, 由此得出, 这组生成元必须构成齐次\,Lorentz\,群的一个表示. 这样, 根据各个生成元所属的齐次\,Lorentz\,群不可约表示就可以对它们进行分类.

正如\,5.6\,节所描述的, 对于任何一组构成齐次\,Lorentz\,群表示的算符, 我们可以通过给出这组算符与生成元$\,\mathbf{A}\,$和$\,\mathbf{B}\,$的对易关系来指定这个表示, 其中$\,\mathbf{A}\,$和$\,\mathbf{B}\,$定义成
\begin{equation}
\mathbf{A}\equiv\tfrac{1}{2}\Bigl(\mathbf{J}+\mi\mathbf{K}\Bigr)\:,\qquad\qquad
\mathbf{B}\equiv\tfrac{1}{2}\Bigl(\mathbf{J}-\mi\mathbf{K}\Bigr)\:, \label{25.2.1}
\end{equation}
其中$\,\mathbf{J}\,$和$\,\mathbf{K}\,$分别是旋转和增速(boost)的厄米生成元. $\mathbf{A}\,$和$\,\mathbf{B}\,$满足对易关系
\begin{equation}
[A_{i},A_{j}]=\sum_{k}\epsilon_{ijk}A_{k}\:,\qquad  [B_{i},B_{j}]=\sum_{k}\epsilon_{ijk}B_{k}\:,\qquad
[A_{i},B_{j}]=0\:, \label{25.2.2}
\end{equation}
其中$\,i,j,k\,$取遍值$\,1,2,3$, $\epsilon_{ijk}$全反对称, $\epsilon_{123}=+1$. 因此, 就像带有两个独立自旋的态一样 齐次\,Lorentz\,群的表示由一对整数或半整数$\,A\,$和$\,B\,$标记, 而表示中的元素由一对指标$\,a,b\,$标记, 它们以\,1\,为步长分别从$\,-A\,$取到$\,A\,$以及从$\,-B\,$取到$\,B$. 更确切一些, 一组总数为$\,(2A+1)(2B+1)\,$的算符$\,Q_{ab}^{AB}\,$构成了齐次\,Lorentz\,群的一个$\,(A,B)\,$表示, 它们满足对易关系
\begin{equation}
[\mathbf{A},Q_{ab}^{AB}]=-\sum_{a^{\prime}}\mathbf{J}_{aa'}^{(A)}Q_{a'b}^{AB}\:,\qquad \qquad
[\mathbf{B},Q_{ab}^{AB}]=-\sum_{b^{\prime}}\mathbf{J}_{bb'}^{(B)}Q_{ab'}^{AB}\:, \label{25.2.3}
\end{equation}
其中$\,\mathbf{J}^{(j)}\,$是角动量为$\,j\,$的自旋\,3\,-矢矩阵:
\begin{equation}
\Bigl(J_{1}^{(j)}\pm\mi J_{2}^{(j)}\Bigr)_{\sigma'\sigma}=\delta_{\sigma',\sigma\pm1}\sqrt{(j\mp\sigma)(j\pm\sigma+1)}\:, \qquad
\Bigl(J_{3}^{(j)}\Bigr)_{\sigma'\sigma}=\delta_{\sigma'\sigma}\sigma\:.\label{25.2.4}
\end{equation}
从方程(\ref{25.2.4})得出\footnote{我们用星号来表示算符的厄米共轭或者数的复共轭. 对于那些由算符的厄米共轭或者数的复共轭构成的矩阵, 我们用剑号$\,\dag\,$来标记它的转置.}
\begin{equation}
-\Bigl(\mathbf{J}^{(j)}\Bigr)^{\ast}_{\sigma',\sigma}
=(-1)^{\sigma'-\sigma}\Bigl(\mathbf{J}^{(j)}\Bigr)_{-\sigma',-\sigma}\:.\label{25.2.5}
\end{equation}
因此, 如果$\,Q_{\sigma}^{j}\,$是一组按照旋转群的自旋$\,j\,$表示进行变换的算符, 那么$\,(-1)^{j-\sigma}Q_{-\sigma}^{j\ast}\,$也是. 另外, 方程(\ref{25.2.1})表明$\,\mathbf{A}^{\ast}=\mathbf{B}$. 通过对方程(\ref{25.2.3})取厄米共轭, 我们看到, 按照齐次Lorentz群的$\,(A,B)\,$表示进行变换的算符的厄米共轭$\,Q_{ab}^{AB\ast}\,$与按照$\,(B,A)\,$表示进行变换的算符%
$\,\bar{Q}_{ba}^{BA}$, 它们通过一个相似变换彼此关联:
\begin{equation}
Q_{ab}^{AB\ast}=(-1)^{A-a}(-1)^{B-b}\bar{Q}^{BA}_{-b,-a}\:.\label{25.2.6}
\end{equation}

Haag-Lopuszanski-Sohnius\,定理\cite{1}的部分表述是, 费米对称性生成元只能属于$\,(0,1/2)\,$表示和 $(1/2,0)\,$表示. 我们已经看到\,$(0,1/2)\,$算符或$\,(1/2,0)\,$算符的厄米共轭分别是$\,(1/2,0)\,$算符或$\,(0,1/2)$ 算符的线性组合, 因此, 费米对称性算符的完备集可以分成$\,(0,1/2)\,$生成元$\,\mathcal{Q}_{ar}\,$(省略了下标$\,0\tfrac{1}{2}\,$)%
和它们的$\,(1/2,0)\,$厄米共轭$\,\mathcal{Q}^{\ast}_{ar}$, 其中$\,a\,$是取值$\,\pm1/2\,$的旋量指标, $r\,$用来区分\,Lorentz\,变换性质相同的不同\,2\,分量生成元.\footnote{为了与本节后面将要引入的%
\,4\,分量\,Dirac\,旋量区分, 取代斜体字母, 我们用手写体$\,\mathcal{Q}_{ar}\,$(原书用的是罗马体字母, 因为与斜体字母比较像, 译本中改用手写体.------译者注)来标记\,Weyl\,旋量. 还有一种\,van der Waerden\,提倡的符号约定, 根据这个符号约定, 在写$\,\mathcal{Q}\,$这样的$\,(0,1/2)\,$算符时指标要加点, 例如$\,\mathcal{Q}_{\dot{a}}$, 而$\,(1/2,0)\,$算符的指标不加点. 在这里我们不使用这个符号约定, 相反我们将清楚地指明那些\,2\,分量旋量按照齐次\,Lorentz\,群的$\,(0,1/2)\,$表示变换, 那些按照$\,(1/2,0)\,$表示变换.} 这个定理进一步表述了, 可以定义费米生成元使它们满足反对易关系
\begin{align}
\{\mathcal{Q}_{ar},\mathcal{Q}_{bs}^{\ast}\}&=2\delta_{rs}\,\sigma_{ab}^{\mu}\,P_{\mu}\:,\label{25.2.7} \\
\{\mathcal{Q}_{ar},\mathcal{Q}_{bs}\}&=e_{ab}\,Z_{rs}\:,\label{25.2.8}
\end{align}
其中$\,P_{\mu}\,$是4-动量算符, $Z_{rs}=-Z_{sr}\,$是玻色对称性生成元, $\sigma_{\mu}\,$和$\,e\,$是$\,2\times2\,$矩阵(行与列用$\,+1/2,-1/2$ 标记):
\begin{equation}
\begin{split}
\sigma_{1}&= \begin{pmatrix}
0 & 1 \\ 1 & 0
\end{pmatrix} \:, \qquad
\sigma_{2}= \begin{pmatrix}
0 & -\mi \\ \mi & 0
\end{pmatrix}\:, \qquad
\sigma_{3}= \begin{pmatrix}
1 & 0 \\ 0 & -1
\end{pmatrix} \:, \\
\sigma_{0}&=\begin{pmatrix}
1 & 0 \\ 0 & 1
\end{pmatrix}\:, \qquad
e=\begin{pmatrix}
0 & 1 \\ -1 & 0
\end{pmatrix}\:.
\end{split} \label{25.2.9}
\end{equation}
最后, 费米对称性生成元与能量和动量对易:
\begin{equation}
[P_{\mu},\mathcal{Q}_{ar}]=[P_{\mu},\mathcal{Q}_{ar}^{\ast}]=0\:,\label{25.2.10}
\end{equation}
而$\,Z_{rs}\,$和$\,Z_{rs}^{\ast}\,$是这个代数的一组中心荷, 也就是说
\begin{align}
0&=[Z_{rs},\mathcal{Q}_{at}]=[Z_{rs},\mathcal{Q}^{\ast}_{at}]=[Z_{rs},Z_{tu}]=[Z_{rs},Z_{tu}^{\ast}]\nonumber\\
&=[Z_{rs}^{\ast},\mathcal{Q}_{at}]=[Z^{\ast}_{rs},\mathcal{Q}^{\ast}_{at}]=[Z^{\ast}_{rs},Z^{\ast}_{tu}]\:.
\label{25.2.11}
\end{align}


为了证明这些结果, 我们先来考察非零费米对称性生成元, 并要求它属于齐次\,Lorentz\,群的某个$\,(A,B)\,$不可约表示的, 这样它就可以记做$\,Q_{ab}^{AB}$, 其中$\,a\,$和$\,b\,$以\,1\,为步长分别从$\,-A\,$取到$\,+A\,$以及从$\,-B\,$取到$\,B$. 正如前面提到的, 厄米共轭通过方程(\ref{25.2.6})与$\,(B,A)\,$表示下的算符相关联, 所以这些算符的反对易子必须采取如下的形式
\begin{align}
\{Q_{ab}^{AB},Q_{a'b'}^{AB\ast}\}&=(-1)^{A-a'}(-1)^{B-b'}\sum_{C=\lvert A-B\rvert}^{A+B}
\sum_{D=\lvert A-B\rvert}^{A+B}\sum_{c=-C}^{C}\sum_{d=-D}^{D}\nonumber\\
&\quad\times C_{AB}(Cc;a,-b')\,C_{AB}(Dd;-a'b)\,X_{cd}^{CD}\:, \label{25.2.12}
\end{align}
其中$\,C_{AB}(j\sigma;ab)\,$是通常的\,Clebsch-Gordan\,系数, 它耦合了自旋$\,A\,$和自旋$\,B\,$以形成自旋$\,j$, $X_{cd}^{CD}\,$ 是按照齐次\,Lorentz\,群的$\,(C,D)\,$表示变换的算符的\,$(c,d)\,$-分量. 利用\,Clebsch-Gordan\,系数从所周知的幺正性, 我们可以将算符$\,X_{cd}^{CD}\,$表示成这些反对易子:
\begin{align}
X_{cd}^{CD}&=\sum_{a=-A}^{A}\sum_{b=-B}^{B}\sum_{a'=-A}^{A}\sum_{b'=-B}^{B}(-1)^{A-a'}(-1)^{B-b'}\nonumber\\
&\quad\times C_{AB}(Cc;a,-b')\,C_{AB}(Dd;-a'b)\,\{Q_{ab}^{AB},Q_{a'b'}^{AB\ast}\}\:. \label{25.2.13}
\end{align}
这些算符不一定都非零. 但是, 当$\,j=\sigma=A+B\,$和$\,j=-\sigma=A+B\,$时, 不为零的\,Clebsch-Gordan\,系数$\,C_{AB}(j\sigma,ab)\,$分别只有那些$\,a=A$, $b=B\,$的和$\,a=-A$, $b=-B\,$的, 而这些系数的值均为\,1, 所以通过在方程中取$\,C=D=c=-d=A+B$, 我们发现
\begin{equation}
X_{A+B,-A-B}^{A+B,A+B}=(-1)^{2B}\,\{Q_{A,-B}^{AB},Q_{A,-B}^{AB\ast}\}\:. \label{25.2.14}
\end{equation}
除非$\,Q_{A,-B}^{AB}=0$, 否则这不可能为零, 而$\,Q_{A,-B}^{AB}=0\,$(通过取$\,Q_{A,-B}^{AB}\,$与``下降''算符$\,A_{1}-\mi A_{2}$%
和``上升''算符$\,B_{1}+\mi B_{2}\,$的对易子)又暗示了所有$\,Q_{ab}^{AB}\,$为零. 因此, 如果存在任何不为零的$\,(A,B)\,$费米生成元, 那么它们与共轭的反对易子至少必须要包含属于表示$\,(A+B,A+B)\,$的非零玻色对称性生成元.

现在, Coleman-Mandula\,定理告诉我们, 组成玻色对称性生成元的是平移的$\,(1/2,1/2)\,$生成元$\,P_{\mu}$, 固有\,Lorentz\,变换的$\,(1,0)+(0,1)\,$生成元$\,J_{\mu\nu}$, 以及可有可无的内部对称性的$\,(0,0)\,$生成元$\,T_{A}$. (回顾一下, $N\,$阶对称无迹张量按照表示$\,(N/2,N/2)\,$变换, 2\,阶反对称张量按照表示$\,(1,0)+(0,1)\,$变换, 而\,Dirac\,场按照表示$\,(1/2,0)+(0,1/2)\,$变换.) 因此费米对称性生成元只能属于$\,A+B\leq 1/2\,$的 $(A,B)\,$表示. 这些算符将玻色子变成费米子并将费米子变成玻色子, 所以它们不能是标量, 这样只剩下了$\,(1/2,0)\,$表示和$\,(0,1/2)\,$表示, 正是所要证明的. 用$\,\mathcal{Q}_{ar}\,$标记线性独立的$\,(0,1/2)\,$费米生成元, 反对易子$\,\{\mathcal{Q}_{ar},\mathcal{Q}_{bs}^{\ast}\}\,$属于表示$\,(0,1/2)\times(1/2,0)=(1/2,1/2)$, 因此它只能正比于$\,(1/2,1/2)\,$玻色对称性生成元, 即动量\,4\,-矢$\,P_{\mu}$. Lorentz\,不变性表明这个关系的形式必须是
\begin{equation}
\{\mathcal{Q}_{ar},\mathcal{Q}^{\ast}_{bs}\} = 2N_{rs}\,\sigma_{ab}^{\mu}\,P_{\mu} \:, \label{25.2.15}
\end{equation}
其中$\,N_{rs}\,$是数值矩阵.

为了看到这点, 我们使用\,2.7\,节讨论的\,Lorentz\,群(或者更准确些, 它的覆盖群)与二维幺模复矩阵群$\,SL(2,C)\,$之间的同构. Lorentz\,变换$\,\Lambda\indices{^\mu_\nu}\,$在$\,(0,1/2)\,$费米生成元上的作用效果是
\begin{equation}
U^{-1}(\Lambda)\,\mathcal{Q}_{ar}\,U(\Lambda) = \sum_{b}\lambda_{ab}\,\mathcal{Q}_{br} \:, \label{25.2.16}
\end{equation}
其中$\,\Lambda\,$是
\begin{equation}
\lambda\, \sigma_{\mu} \,\lambda^{\dag} =\Lambda\indices{_\nu^\mu}\sigma_{\nu}  \label{25.2.17}
\end{equation}
定义的\,Lorentz\,变换. 我们可以验证方程(\ref{25.2.16})对$\,(0,1/2)\,$算符是成立的, 方法是, 注意到对于无限小\,Lorentz\,变换$\,\Lambda\indices{^\mu_\nu}=\delta\indices{^\mu_\nu}+\omega\indices{^\mu_\nu}$, 其中$\,\omega_{\mu\nu}=\omega_{\nu\mu}$, 方程(\ref{25.2.17})对
\[
\lambda=1+\tfrac{1}{2}\Bigl[\tfrac{1}{2}\,\mi\,\epsilon_{ijk}\omega_{ij}+\omega_{k0}\Bigr]\sigma_{k}
\]
是满足的, 而此时\footnote{这里的$\,K_{i}\,$定义成$\,J_{i0}$. 在卷\,I\,的前两次印刷中有一个错误: $K_{i}\,$在\,2.4\,节, 3.3\, 节和\,3.5\,节定义成了$\,J^{i0}$, 而在\,5.6\,节和\,5.9\,节定义成了$\,J_{i0}$, 而$\,\mathbf{A}\,$和$\,\mathbf{B}\,$自始至终由方程(\ref{25.2.1})给定.}
\[
U(\Lambda)=1+\tfrac{1}{2}\,\mi\,\omega_{\mu\nu}J^{\mu\nu}=1 + \tfrac{1}{2}\,\mi\,\epsilon_{ijk}\omega_{ij}J_{k}-\mi\,\omega_{i0}K_{i} \:.
\]
(重复拉丁指标$\,i,j,k\,$对值$\,1,2,3\,$求和.) 在这一情况下, 通过$\,\omega_{ij}\,$和$\,\omega_{i0}\,$在方程(\ref{25.2.16})两边的系数相等, 我们发现
\[
[\mathbf{J},\mathcal{Q}_{a}]=-\tfrac{1}{2}\sum_{b}\bm{\sigma}_{ab}\,\mathcal{Q}_{b} \:, \qquad
[\mathbf{K},\mathcal{Q}_{a}]=-\tfrac{1}{2}\,\mi\sum_{b} \bm{\sigma}_{ab}\,\mathcal{Q}_{b} \:,
\]
或者等价的
\[
[\mathbf{B},\mathcal{Q}_{a}]=-\tfrac{1}{2}\sum_{b}\bm{\sigma}_{ab}\,\mathcal{Q}_{b} \:, \qquad
[\mathbf{A},\mathcal{Q}_{a}]=0 \:,
\]
这表明满足方程(\ref{25.2.16})的算符属于$\,(0,1/2)\,$表示. 现在, $\sigma_{\mu}\,$构成了$\,2\times2\,$矩阵的一个完备集, 所以我们可以将反对易子$\,\{\mathcal{Q}_{ar},\mathcal{Q}_{bs}^{\ast}\}\,$%
写成$\,N_{rs}^{\mu}\,(\sigma_{\mu})_{ab}\,$的形式, 其中$\,N^{\mu}\,$是算符的某个矩阵. 方程(\ref{25.2.16}) 和(\ref{25.2.17})表明这些算符是\,4\,-矢, 也就是说$\,U^{-1}(\Lambda)N^{\mu}U(\Lambda)=\Lambda\indices{^\mu_\nu}\,N^{\nu}$, 那么根据 Coleman-Mandula 定理它们只能正比于玻色对称性算符中的唯一\,4\,-矢, $P^{\mu}$. 令$\,N^{\mu}_{rs}=2P^{\mu}N_{rs}\,$就给出了方程(\ref{25.2.15}).

现在我们要对$\,\mathcal{Q}_{ar}\,$做一个线性变换使得它们的反对易子是(\ref{25.2.7})的形式. 为了这个目的, 我们需要构建厄米且正定的矩阵$\,N_{rs}$. 通过取方程(\ref{25.2.15})的厄米共轭我们可以立刻得出\,$N_{rs}\,$是厄米的. 为了看到它是正定的, 回忆\,$\mathcal{Q}_{ar}\,$是被取成线性独立的, 所以对于任何非零的线性组合$\,\mathcal{Q}\equiv\sum_{r}d_{a}\,c_{r}\,\mathcal{Q}_{ar}$, 必存在某个$\,\mathcal{Q}\,$湮灭不了的态$\,\lvert\Psi\rangle$. 取方程(\ref{25.2.15})在这个态上的期望值, 这给出
\[
2\langle\Psi\vert\sum_{ab}\sigma_{ab}^{\mu}P_{\mu}d_{a}d_{b}^{\ast}\vert\Psi\rangle\sum_{rs}c_{r}c_{s}^{\ast}N_{rs}
=\langle\Psi\{\mathcal{Q},\mathcal{Q}^{\ast}\}\vert\Phi\rangle >0 \:.
\]
由此可以立刻得出, 对于任何不全为零的$\,c_{r}$, $\sum_{rs}\,c_{r}c_{s}^{\ast}N_{rs}\,$必不为零, 所以$\,N_{rs}\,$不是正定的就是负定的. 在$\,-P^{\mu}P_{\mu}\geq0\,$且$\,P^{0}>0\,$的物理态的空间上, 算符$\,\sum_{ab}(\sigma_{\mu})_{ab}P^{\mu}d_{a}d_{b}^{\ast}\,$是正定的, 所以矩阵$\,N^{rs}\,$也必须是正定的.\footnote{这个论证也可以反过来. 假定一个$\,N_{rs}\,$正定的超对称性, 就像方程(\ref{25.2.7})中那样, 我们可以推出所有态的$\,P^{0}>0$.\cite{2} 然而, 当引力也考虑在内时, 这个结论是不成立的, 一个例外情况是, 引力被当成了所有态中能量的一个偏移, 并且这个偏移都一样以至于没有任何物理效应.}

现在我们可以定义新的费米生成元
\[
\mathcal{Q}_{ar}^{\prime}\equiv \sum_{s} N_{rs}^{-1/2}\mathcal{Q}_{as} \:,
\]
使得反对易子取如下的形式
\[
\{\mathcal{Q}_{ar}^{\prime},\mathcal{Q}_{bs}^{\prime\ast}\}=2\delta_{rs}\,\sigma_{ab}^{\mu}\,P_{\mu}\:.
\]
从现在起, 我们将假定所有费米生成元都以这种方式定义并去掉撇号, 使得方程(\ref{25.2.7})成立.


接下来我们必须要证明\,$\mathcal{Q}_{ar}\,$与动量\,4\,-矢$\,P_{\mu}\,$对易. $P_{\mu}\,$这样的$\,(1/2,1/2)$ 算符与$\,\mathcal{Q}\,$这样的$\,(0,1/2)\,$算符, 它们的对易子只能是$\,(1/2,0)\,$算符或$\,(1/2,1)\,$算符, 但是我们看到不存在$\,(1/2,1)\,$对称性生成元, 所以$\,P_{\mu}\,$与$\,\mathcal{Q}\,$的对易子只能正比于$\,(1/2,0)\,$对称性生成元$\,\mathcal{Q}^{\ast}$. Lorentz\,不变性要求这个关系取如下的形式
\begin{equation}
    [\mathscr{M}_{ab},\mathcal{Q}_{cr}]=\sum_{s}e_{ac}\,K_{rs}\,\mathcal{Q}_{bs}^{\ast}\:,\label{25.2.18}
\end{equation}
其中$\,K\,$是一数值矩阵, $\mathscr{M}\,$是算符矩阵
\begin{equation}
    \mathscr{M}\equiv\sigma_{\mu}P^{\mu}\:. \label{25.2.19}
\end{equation}
(矩阵$\,e_{ac}\,$是将两个自旋$\,1/2\,$耦合成零自旋的\,Clebsch-Gordan\,系数.) 由此可以直接得出
\begin{equation}
    [\mathscr{M}_{{-}\frac{1}{2}{-}\frac{1}{2}},[\mathscr{M}_{{-}\frac{1}{2}{-}\frac{1}{2}},
    \{\mathcal{Q}_{\frac{1}{2}r},\mathcal{Q}_{\frac{1}{2}s}^{\ast}\}]]
    =-4(\mathscr{M})_{{-}\frac{1}{2}{-}\frac{1}{2}}(KK^{\dag})_{rs}\:.\label{25.2.20}
\end{equation}
利用方程(\ref{25.2.7}), 左边是多重对易子$\,[P_{\mu},[P_{\nu},P_{\lambda}]]\,$的线性组合, 所有这样的对易子都为零, 而 $\mathscr{M}_{{-}1/2{-}1/2}$ 对于一般的动量不为零, 所以$\,K^{\dag}K=0$, 因此$\,K=0$, 加上方程(\ref{25.2.18}), 这表明\,$[P_{\mu},\mathcal{Q}_{ar}]=0$. 复共轭给出$\,[P_{\mu},\mathcal{Q}_{ar}^{\ast}]=0$.

现在我们可以着手处理两个$\,\mathcal{Q}\,$的反对易子. 两个$\,(0,1/2)\,$对称性算符的反对易子必须是$\,(0,1)$ 对称性生成元和$\,(0,0)\,$对称性生成元的线性组合. Coleman-Mandula\,定理告诉我们唯一的$\,(0,1)\,$对称性生成元是固有齐次\,Lorentz\,变换的%
生成元$\,J_{\nu\lambda}\,$的线性组合, 但由于$\,\mathcal{Q}\,$与$\,P_{\mu}$对易, 继而它们的反对易子也与$\,P_{\mu}\,$对易, 而方程(\textcolor{foo}{2.4.13})告诉我们$\,J_{\nu\lambda}\,$的线性组合与$\,P_{\mu}\,$不对易. 这样就只剩下了$\,(0,0)$ 算符, 它既与$\,P_{\mu}\,$对易又与$\,J_{\nu\lambda}\,$对易. 这样, Lorentz\,不变性就会要求$\,\mathcal{Q}\,$之间的反对易子必须采取方程(\ref{25.2.8})的形式. 内部对称性生成元$\,Z_{rs}\,$关于$\,r\,$和$\,s\,$是反对称的, 这是因为整个表达式必须在$\,r\,$与 $s\,$和$\,a\,$与$\,b\,$的交换下是对称的, 而矩阵$\,e_{ab}\,$关于$\,a\,$和$\,b\,$是反对称的.

剩下来要证明的是$\,Z\,$是中心荷. 从方程(\ref{25.2.8})和(\ref{25.2.10})立即可以得出
\begin{equation}
    [P_{\mu},Z_{rs}]=0 \:. \label{25.2.21}
\end{equation}
接下来考察包含两个$\,\mathcal{Q}\,$和一个$\,\mathcal{Q}^{\ast}\,$的推广\,Jacobi\,恒等式(\ref{25.1.5}):
\[
0=[\{\mathcal{Q}_{ar},\mathcal{Q}_{bs}\},\mathcal{Q}_{ct}^{\ast}]
+[\{\mathcal{Q}_{bs},\mathcal{Q}_{ct}^{\ast}\},\mathcal{Q}_{ar}]
+[\{\mathcal{Q}_{ct}^{\ast},\mathcal{Q}_{ar}\},\mathcal{Q}_{bs}] \:.
\]
方程(\ref{25.2.7})和(\ref{25.2.10})表明第二项和第三项为零, 所以
\begin{equation}
    [Z_{rs},\mathcal{Q}_{ct}^{\ast}]=0\:. \label{25.2.22}
\end{equation}
最后, 考察一个$\,Z$, 一个$\,\mathcal{Q}\,$和一个$\,\mathcal{Q}^{\ast}\,$的推广\,Jacobi\,恒等式:
\[
0=-[Z_{rs},\{\mathcal{Q}_{at},\mathcal{Q}_{bu}^{\ast}\}]
+\{\mathcal{Q}_{bu}^{\ast},[Z_{rs},\mathcal{Q}_{at}]\}
-\{\mathcal{Q}_{at},[\mathcal{Q}_{bu}^{\ast},Z_{rs}]\} \:.
\]
第一项和第三项分别因为方程(\ref{25.2.21})和(\ref{25.2.22})为零, 所以我们只剩下了第二项
\begin{equation}
    \{\mathcal{Q}_{bu}^{\ast},[Z_{rs},\mathcal{Q}_{at}]\} =0 \:. \label{25.2.23}
\end{equation}
现在, $[Z_{rs},\mathcal{Q}_{at}]\,$是$\,(0,1/2)\,$对称性生成元, 所以它必须是$\,\mathcal{Q}\,$的线性组合:
\begin{equation}
    [Z_{rs},\mathcal{Q}_{at}]=\sum_{u}M_{rstu}\,\mathcal{Q}_{au} \:. \label{25.2.24}
\end{equation}
那么对于所有$a$, $b$, $r$, $s$, $t\,$和$\,u$, 方程(\ref{25.2.23})就变成
\[
\sigma_{ab}^{\mu}P_{\mu}M_{rstu}=0\:.
\]
由于算符$\,\sigma_{ab}^{\mu}P_{\mu}\,$不为零, 我们得出$\,M_{rstu}=0$, 这使得
\begin{equation}
    [Z_{rs},\mathcal{Q}_{at}]=0 \:. \label{25.2.25}
\end{equation}
利用反对易关系(\ref{25.2.8})和它的共轭, 再加上对易关系(\ref{25.2.22})和(\ref{25.2.25})与它们的的共轭, 这给出
\begin{equation}
    [Z_{rs},Z_{tu}]=[Z_{rs},Z_{tu}^{\ast}] = [Z_{rs}^{\ast},Z_{tu}^{\ast}] =0\:, \label{25.2.26}
\end{equation}
这完成了方程(\ref{25.2.11})的证明, 有了这个也就证明了\,Hagg-Lopuszanski-Sohnius\,定理.

当然, $Z_{rs}\,$是超对称代数的中心荷这一点并不会排除还存在{\kai{其它}}阿贝尔或非阿贝尔内部对称性的可能性. 设$\,T_{A}\,$张开了玻色内部对称性的整个\,Lie\,代数. 那么$\,[T_{A},\mathcal{Q}_{ar}]\,$就是$\,(0,1/2)\,$算符, 所以它必须是$\,\mathcal{Q}\,$的线性组合:
\begin{equation}
    [T_{A},\mathcal{Q}_{ar}]=-\sum_{s}(t_{A})_{rs}\mathcal{Q}_{as} \label{25.2.27}
\end{equation}
从两个$\,T\,$和一个$\,\mathcal{Q}\,$的\,Jacobi\,恒等式, 我们可以得知$\,t_{A}\,$矩阵构成了内部对对称性代数的一个表示
\begin{equation}
    [t_{A},t_{B}] = \mi \sum_{C}C_{AB}^{C}\,t_{C} \:, \label{25.2.28}
\end{equation}
其中系数$\,C_{AB}^{C}\,$是内部对称性代数的结构常数
\begin{equation}
    [T_{A},T_{B}] = \mi \sum_{C}C_{AB}^{C}\,T_{C} \:. \label{25.2.29}
\end{equation}
这样, $Z_{rs}\,$不仅是$\,\mathcal{Q}$, $\mathcal{Q}^{\ast}$, $P_{\mu}$, $Z\,$和$\,Z^{\ast}\,$构成的超代数的中心荷, 同时还是包含所有$\,T_{A}\,$的更大的超代数的中心荷. 为了看到这点, 从方程(\ref{25.2.27})和(\ref{25.2.8})中注意到
\[
    [T_{A},Z_{rs}] = -\sum_{r^{\prime}}(t_{A})_{rr^{\prime}} Z_{r^{\prime}s}
    -\sum_{s^{\prime}}(t_{A})_{ss^{\prime}}Z_{rs^{\prime}} \:,
\]
所以$\,Z_{rs}\,$构成了整个玻色对称性代数的一个{\kai{不变}}阿贝尔子代数. 但是回顾\,Coleman-Mandula\,定理的证明, 我们发现内部玻色对称性的整个\,Lie\,代数, 在这里就是$\,T_{A}\,$张开的\,Lie\,代数, 它同构于一个紧致半单\,Lie\,代数和几个\,$U(1)$\,代数的直和. 这种\,Lie\,代数的不变阿贝尔子代数只有那些$\,U(1)\,$生成元张开的, 所以$\,Z_{rs}\,$必须是$\,U(1)\,$生成元, 因而与所有$\,T_{A}\,$对易.

即使$\,Z\,$与所有对称性算符都对易, 它们也不只是个数; 它们是量子算符, 它们的值可能随着态的变化而变化. 事实上, 对于超对称真空态, 由于它被所有超对称性生成元湮灭, $Z\,$显然必须在这个态上取零值, 但是一般而言它们不需要为零. 在\,27.9\,节, 我们将看到如何在有扩充超对称性的规范理论中计算$\,Z$.

在没有中心荷的情况下, 超对称代数(\ref{25.2.7}), (\ref{25.2.8})在内部对称群$\,U(N)\,$下不变
\begin{equation}
    \mathcal{Q}_{ar}\to \sum_{s}V_{rs}\mathcal{Q}_{as} \:, \label{25.2.30}
\end{equation}
其中$\,V_{rs}\,$是$\,N\times N\,$幺正(不一定幺模)矩阵. 这被称为$\,R\,$-{\kai{对称性}}. 这个对称性可能是也可能不是一个好对称性, 如果它是, 那么它可能被反常破坏也可能自发破缺, 或者它就是自然的一个好对称性.

$r,s\,$等指标的取值$\,N>1\,$的超对称代数被称为$\,N\,$-{\kai{扩充超对称性}}. 当只有一个$\,\mathcal{Q}\,$时, $Z_{rs}=-Z_{sr}\,$的条件告诉我们$\,Z\,$为零, 这给出了反对易关系的一个更加简单的形式
\begin{align}
    \{\mathcal{Q}_{a},\mathcal{Q}_{b}^{\ast}\} &= 2\sigma_{ab}^{\mu}P_{\mu}\:,\label{25.2.31} \\
    \{\mathcal{Q}_{a},\mathcal{Q}_{b}\} &= 0 \:. \label{25.2.32}
\end{align}
这种情况被称为{\kai{简单超对称}}, 或者$\,N=1\,$超对称. 在这一情况下, $R\,$-对称变换是\,$U(1)\,$相位变换
\begin{equation}
    \mathcal{Q}_{a}\to \exp(\mi\varphi)\, \mathcal{Q}_{a} \:, \label{25.2.33}
\end{equation}
其中$\,\varphi\,$是一个实相位.

为了多个目的, 将$\,(0,1/2)\,$算符$\,\mathcal{Q}_{ar}\,$与$\,(1,2)\,$算符相结合写成\,4\,分量\,Majorana\,旋量%
生成元$\,Q_{ar}$ 将是方便的, 其中$\,(1,2)\,$算符根据方程(\ref{25.2.6})可以取成$\,e_{ab}\mathcal{Q}_{br}^{\ast}$, 那么$\,Q\,$定义成
\begin{equation}
    Q_{r}\equiv
    \begin{pmatrix}
    e\mathcal{Q}_{r}^{\ast} \\ \mathcal{Q}_{r}
    \end{pmatrix} \:, \label{25.2.34}
\end{equation}
或者更明显些
\[
Q_{1r}=\mathcal{Q}^{\ast}_{-\frac{1}{2}\,r} \:, \quad
Q_{2r}=-\mathcal{Q}^{\ast}_{\frac{1}{2}\,r} \:, \quad
Q_{3r}=\mathcal{Q}_{\frac{1}{2}\,r} \:, \quad
Q_{4r}=\mathcal{Q}_{-\frac{1}{2}\,r} \:. \quad
\]
这是一个\,Majorana\,旋量, 也就是说
\[
Q_{r}=\beta \epsilon \gamma_{5} Q_{r}^{\ast} \:,
\]
其中$\,\beta$, $\epsilon\,$和$\,\gamma_{5}\,$是$\,4\times4\,$矩阵, 它们可以写成\,$2\times2$分块矩阵:
\[
\beta = \begin{pmatrix}
0 & 1 \\ 1 & 0
\end{pmatrix} \qquad
\epsilon = \begin{pmatrix}
e & 0 \\ 0 & e
\end{pmatrix} \qquad
\gamma_{5} = \begin{pmatrix}
1 & 0 \\ 0 & -1
\end{pmatrix} \:.
\]
(第\,26\,章的附录将会回顾\,Majorana\,旋量的性质.) 选择(\ref{25.2.34})的形式是为了与齐次\,Lorentz\,群通常的\,4\,分量\,Dirac\,表示的记法一致, 根据方程(\textcolor{foo}{5.4.4}), 在这个表示中旋转和增速生成元按照方程(\textcolor{foo}{5.4.19})和(\textcolor{foo}{5.4.20})被表示成
\begin{equation}
    \mathscr{J}_{i} = \frac{1}{2}
    \begin{bmatrix}
    \sigma_{i} & 0 \\ 0 & \sigma_{i}
    \end{bmatrix} \:,  \qquad
     \mathscr{K}_{i} = -\frac{\mi}{2}
    \begin{bmatrix}
    \sigma_{i} & 0 \\ 0 & -\sigma_{i}
    \end{bmatrix} \:.
\end{equation}
再加上方程(\ref{25.2.1}), 这表明$\,\mathbf{A}\,$和$\,\mathbf{B}\,$分别只作用在\,Dirac\,旋量的前两个分量和后两个分量上, 这就是为什么我们将$\,(0,1/2)\,$算符$\,\mathcal{Q}_{ar}\,$用作方程(\ref{25.2.34})的下分量而不是上分量

在这个\,4\,分量记法中, 简单超对称的基础反对易关系(\ref{25.2.31})和(\ref{25.2.32})写成
\begin{equation}
    \{Q,\overline{Q}\} = 2
    \begin{pmatrix}
    0 & -e\,(\sigma_{\mu}P^{\mu})^{\mathrm{T}}\,e \\
    \sigma_{\mu} P^{\mu} & 0
    \end{pmatrix}
    =-2\mi\,P_{\mu}\gamma^{\mu} \:. \label{25.2.36}
\end{equation}
本卷的前言部分回顾了我们使用了\,Dirac\,矩阵的约定; 这里我们仅需要记起
\begin{equation}
    \gamma^{0} = -\mi\beta = -\mi
    \begin{pmatrix}
    0 & \sigma_{0} \\ \sigma_{0} & 0
    \end{pmatrix} \:, \qquad
    \bm{\gamma} = -\mi
    \begin{pmatrix}
    0 & \bm{\sigma} \\ -\bm{\sigma} & 0
    \end{pmatrix}  \:, \label{25.2.37}
\end{equation}
以及$\,e\,\bm{\sigma}^{\mathrm{T}}\,e=\bm{\sigma}$, $e\sigma_{0}e=-\sigma_{0}$, 和通常的$\,\overline{Q}\equiv Q^{\dag}\beta$. 在扩充超对称的情况下, 中心荷的出现会改变这个公式; 取代方程(\ref{25.2.36}), 我们有
\begin{equation}
    \{Q_{r},\overline{Q}_{s}\} = -2\mi\,P_{\mu}\gamma^{\mu}\delta_{rs}
    +\biggl(\frac{1+\gamma_{5}}{2}\biggr) Z_{sr}^{\ast} + \biggl(\frac{1-\gamma_{5}}{2}\biggr) Z_{rs} \:.
    \label{25.2.38}
\end{equation}

这里给出的分析针对的是时空维数为\,4\,的情况, 在第\,32\,章, 我们将对一般的时空维数以一种不太显式的形式重复这个分析. 在那里我们将会看到, 在高维时空中, 即使理论中有扩充的量使得可以构造出\,Coleman-Mandula\,定理允许范围以外的玻色对称性生成元, 超对称性生成元也总是属于高维\,Lorentz\,群的基础旋量表示.


\subsection{* * *}

在无质量的理论中, 对于那些在共形对称性代数(\ref{24.B.34})---(\ref{24.B.35})下不变的理论, 存在两个额外的能够出现在超对称反对易关系右边的对称性生成元, $D\,$和$\,K_{\mu}$. 这些新生成元分别拥有标量和矢量的\,Lorentz\,变换性质, 就像$\,Z_{rs}\,$和$\,P_{\mu}$, 所以和前面一样, 费米生成元必须属于\,Lorentz\,代数的基础$\,(1/2,0)\,$旋量表示, 而它的共轭必须属于$\,(0,1/2)\,$表示. 同时, 根据所有这些元与伸缩生成元$\,D\,$的对易关系对它们进行分类是方便的; 如果一个算符$\,X\,$有
\begin{equation}
    [X,D] = \mi a X\:,\label{25.2.39}
\end{equation}
那就称它有量纲$\,a$. 对方程(\ref{24.B.34})的观察表明玻色对称性生成元$\,J^{\mu\nu}$, $P^{\mu}$, $K^{\mu}\,$和$\,D\,$分别拥有量纲$\,0$, $+1$, $-1\,$和$\,0$. 另外, 对于任何内部对称性的\,Lie\,群, 它的生成元的量纲为零. 量纲为$\,a\,$的费米生成元与它的共轭的反对易子是量纲为$\,2a\,$的正定玻色算符, 又因为正定玻色对称性生成元只能是$\,P_{\mu}\,$分量和$\,K_{\mu}\,$分量的线性组合, 费米对称性算符只能有量纲$\,+1/2\,$和$\,-1/2$. 量纲$\,1/2\,$的$\,(0,1/2)$ 费米对称性算符和它们的共轭可以再次被装配成\,Majorana\,旋量$\,Q_{r\alpha}$, 并满足
\begin{align}
    &\{Q_{r\alpha},\overline{Q}_{s\beta}\} = -2\mi P_{\mu}(\gamma^{\mu})_{\alpha\beta}\delta_{rs} \:, \label{25.2.40} \\
    &[P_{\mu},Q_{r\alpha}] = 0 \:, \label{25.2.41} \\
    &[D,Q_{r\alpha}] = -\tfrac{1}{2}\mi Q_{r\alpha} \:. \label{25.2.42}
\end{align}
(注意, 因为中心荷的量纲是$\,0\,$而不是$\,+1$, 这里是不允许存在中心荷的.) $K_{\mu}\,$与$\,Q_{r\alpha}\,$的对易子是 Majorana\,费米对称性生成元$\,Q_{r\alpha}^{\sharp}\,$的线性组合, Lorentz\,不变性使得我们可以将它写成如下形式
\begin{equation}
    [K^{\mu},Q_{r\alpha}] = \mi\,(\gamma^{\mu})_{\alpha\beta}Q_{r\beta}^{\sharp} \:. \label{25.2.43}
\end{equation}
(右边的一个任意因子已经被吸收进$\,Q_{r\beta}^{\sharp}\,$的定义中. 对右边的相位已经进行了选择, 使得$\,Q_{r\beta}^{\sharp}\,$满足\,Majorana\,旋量的标准实条件(\ref{26.A.2}).) $\,Q_{r\beta}^{\sharp}\,$的量纲是$\,+1/2-1=-1/2$, 所以
\begin{equation}
    [D,Q_{r\alpha}^{\sharp}] = + \tfrac{1}{2}\mi\,Q_{r\alpha}^{\sharp} \:. \label{25.2.44}
\end{equation}
取方程(\ref{25.2.43})与$\,P^{\nu}\,$的对易子并使用方程(\ref{24.B.34})给出的$\,K^{\mu}\,$和$\,P^{\nu}\,$的对易关系, 这给出
\begin{equation}
    [P^{\nu},Q_{r\alpha}^{\sharp}] = -\mi(\gamma^{\nu})_{\alpha\beta}Q_{r\beta} \:. \label{25.2.45}
\end{equation}
我们看到$\,Q\,$与$\,Q^{\sharp}\,$是相联系的. 通过取反对易关系(\ref{25.2.40})与$\,K_{\mu}\,$的对易子, 我们发现了$\,Q^{\sharp}\,$与$\,Q\,$的反对易子:
\begin{equation}
    \{Q_{r\alpha}^{\sharp},\overline{Q}_{s\beta}\} = 2\mi D\delta_{rs}\delta_{\alpha\beta}
    + 2J_{\mu\nu}\delta_{rs}\mathscr{J}_{\alpha\beta}^{\mu\nu}
    + O_{rs}\delta_{\alpha\beta} + O_{rs}^{\prime}(\gamma_{5})_{\alpha\beta} \:, \label{25.2.46}
\end{equation}
其中$\,\mathscr{J}^{\mu\nu}=-\mi[\gamma^{\mu},\gamma^{\nu}]/4$, $O_{rs}\,$和$\,O_{rs}^{\prime}\,$是量纲为零的\,Lorentz\,不变算符, 并满足
\begin{equation}
    O_{rs}=-O_{sr}\:, \qquad O_{rs}^{\prime} = + Q_{sr}^{\prime} \:. \label{25.2.47}
\end{equation}
取方程(\ref{25.2.43})与$\,K_{\nu}\,$的对易子并使用$\,[K_{\nu},K_{\mu}]=0$, 我们发现$\,(\gamma^{\mu})_{\alpha\beta}[K^{\nu},Q_{r\beta}]\,$关于$\,\mu\,$和$\,\nu\,$是对称的, 在经过一些计算, 这告诉我们
\begin{equation}
    [K^{\nu},Q_{r\beta}] = 0 \:. \label{25.2.48}
\end{equation}
另外, 取方程(\ref{25.2.46})与$\,K_{\nu}\,$的对易子给出
\begin{equation}
    \{Q_{r\alpha}^{\sharp},\overline{Q^{\sharp}}_{s\beta}\}
    = +2\mi K_{\mu}(\gamma^{\mu})_{\alpha\beta}\delta_{rs} \:. \label{25.2.49}
\end{equation}
最后, 取方程(\ref{25.2.46})与$\,Q_{t\gamma}\,$的对易子, 这会表明$\,O_{rs}\,$和$\,O_{rs}^{\prime}\,$的作用就像$\,R\,$-对称群$\,U(N)\,$的生成元, 而$\,Q_{r\alpha}\,$的左边和右边分别按照表示$\,\mathbf{N}\,$和$\,\bar{\mathbf{N}}\,$变换, $P_{\mu}$, $K_{\mu}\,$和$\,D\,$都是$\,U(N)\,$-不变量. 这些生成元彼此之间以及它们与其它生成元的$\,U(N)\,$对易关系, 再加上$\,J_{\mu\nu}\,$和$\,D\,$与各种生成元的对易子, 这些合起来构成了{\kai{超共形代数}}. 这个代数与普通的简单超对称或者$\,N\,$-扩充超对称之间的一个重要差异是, $U(N)\,$对称性不再只是超对称代数的一个外自同构, 那时它可以是也可以不是作用量的一个对称性------现在它是超共形代数的一部分, 因此它必须是任何共形不变的超对称理论的一个对称性.

\section{超对称性生成元的空间反演性质} \label{sec:25.3}

在遵循宇称守恒的理论中, 用宇称算符$\,\mathsf{P}\,$作用费米对称性算符$\,\mathcal{Q}_{ar}\,$得到的%
$\,\mathsf{P}^{-1}\mathcal{Q}_{ar}\mathsf{P}\,$也必须是个%
费米对称性算符. 由于$\,J_{i}\,$和$\,K_{i}\,$在空间反演分别为偶和奇, 方程(\ref{25.2.1})表明用宇称算符作用$\,A_{i}$ 的结果是
\begin{equation}
    \mathsf{P}^{-1}A_{i}\mathsf{P}=B_{i} \:. \label{25.3.1}
\end{equation}
根据方程(\ref{25.2.3}), 将$\mathcal{Q}_{ar}\,$定义为$\,(0,1/2)\,$算符意味着
\begin{equation}
    [B_{i},\mathcal{\mathcal{Q}}_{ar}]=-\tfrac{1}{2}\sum_{b}\bigl(\sigma_{i}\bigr)_{ab}\mathcal{Q}_{br}\:, \qquad
    [A_{i},\mathcal{Q}_{ar}]=0 \:. \label{25.3.2}
\end{equation}
使用宇称算符给出
\begin{equation}
    [A_{i},\mathsf{P}^{-1}\mathcal{Q}_{ar}\mathsf{P}]
    =-\tfrac{1}{2}\sum_{b}\bigl(\sigma_{i}\bigr)_{ab}\mathsf{P}^{-1}\mathcal{Q}_{br}\mathsf{P} \:, \qquad
    [B_{i},\mathsf{P}^{-1}\mathcal{Q}_{ar}\mathsf{P}] =0 \:, \label{25.3.3}
\end{equation}
所以$\,\mathsf{P}^{-1}\mathcal{Q}_{ar}\mathsf{P}\,$是$\,(1/2,0)\,$对称性算符, 因此必须是$\,\mathcal{Q}_{ar}^{\ast}\,$的线性组合. 根据方程(\ref{25.2.6}), Lorentz\,不变性指明这个关系的形式是
\begin{equation}
    \mathsf{P}^{-1}\mathcal{Q}_{ar}\mathsf{P}=\sum_{bs}\mathscr{P}_{rs}\,e_{ab}\mathcal{Q}_{bs}^{\ast} \:, \label{25.3.4}
\end{equation}
其中$\,\mathscr{P}\,$是数值矩阵, 而矩阵$\,e\,$由方程(\ref{25.2.9})给出.

通过要求方程(\ref{25.3.4})与基础反对易关系(\ref{25.2.7})自洽, 我们可以知道矩阵$\,\mathscr{P}\,$的一些性质. 方程(\ref{25.3.4})和它的共轭给出
\[
\mathsf{P}^{-1}\{\mathcal{Q}_{ar},\mathcal{Q}_{bs}^{\ast}\} \mathsf{P}
=\sum_{cdtu}\mathscr{P}_{rt}\,e_{ac}\,\mathscr{P}_{su}^{\ast}\,e_{bd}\,\{\mathcal{Q}_{ct}^{\ast},\mathcal{Q}_{du}\} \:.
\]
代入方程(\ref{25.2.7}), 这变成
\[
\delta_{rs}\sigma_{ab}^{\mu}\,\mathsf{P}^{-1}P_{\mu}\mathsf{P}
=\sum_{cdtu}\mathscr{P}_{rt}\,e_{ac}\,\mathscr{P}_{su}^{\ast}\,e_{bd}\,\delta_{tu}\,\sigma_{dc}^{\mu}\,P_{\mu}\:.
\]
但是$\,e\sigma_{i}^{\mathrm{T}}e^{-1}=-\sigma_{i}\,$且$\,e\sigma_{0}^{\mathrm{T}}e^{-1}=+\sigma_{0}$, 而$\,\mathsf{P}^{-1}P_{i}\mathsf{P}=-P_{i}\,$以及$\,\mathsf{P}^{-1}P_{0}\mathsf{P}=-P_{0}$, 所以这退化成表述$\,\mathscr{P}\,$是幺正的
\begin{equation}
    \mathscr{P}\,\mathscr{P}^{\dag}=1\:. \label{25.3.5}
\end{equation}

矩阵$\,\mathscr{P}\,$在某种意义上是任意的, 这是因为, 对于任何一组满足方程(\ref{25.3.2})和(\ref{25.2.7})的费米生成元$\,\mathcal{Q}_{ar}$, 通过幺正变换
\begin{equation}
    \mathcal{Q}_{ar}^{\prime}=\sum_{s}\mathscr{U}_{rs}\,\mathcal{Q}_{as} \:, \qquad
    \mathscr{U}^{\dag}=\mathscr{U}^{-1}\:, \label{25.3.6}
\end{equation}
我们可以构建另一组也满足方程(\ref{25.3.2})和(\ref{25.2.7})的$\,\mathcal{Q}_{ar}^{\prime}$, 使得宇称变换规则(\ref{25.3.4})变成
\begin{equation}
    \mathsf{P}^{-1}\mathcal{Q}_{ar}^{\prime}\mathsf{P}
    =\sum_{bs}\mathscr{P}_{rs}^{\prime}e_{ab}\mathcal{Q}_{bs}^{\prime\ast}\:,\label{25.3.7}
\end{equation}
其中
\begin{equation}
    \mathscr{P}^{\prime} =\mathscr{U}\mathscr{P}\mathscr{U}^{-1\ast}
    =\mathscr{U}\mathscr{P}\mathscr{U}^{\mathrm{T}} \:. \label{25.3.8}
\end{equation}

对于简单超对称, $\mathscr{P}\,$就是$\,1\times1\,$相因子, 而方程(\ref{25.3.4})变成
\begin{equation}
    \mathsf{P}^{-1}\mathcal{Q}_{a}\mathsf{P}=\mathscr{P}\sum_{b}e_{ab}\mathcal{Q}_{b}^{\ast} \:. \label{25.3.9}
\end{equation}
结合它的共轭给出
\begin{equation}
    \mathsf{P}^{-2}\mathcal{Q}_{a}\mathsf{P}^{2} = -\mathcal{Q}_{a}\:, \label{25.3.10}
\end{equation}
这独立于我们对相位因子$\,\mathscr{P}\,$选的值. 这有一个显著的结果, 如果处在粒子超重态的一个玻色子拥有使得内禀宇称, 那么通过用$\,\mathcal{Q}_{a}\,$作用这个玻色态获得的费米子就会有{\kai{纯虚}}的内禀宇称.

因为对于简单超对称$\,\mathscr{U}\,$和$\,\mathscr{P}\,$就是相因子, 从方程(\ref{25.3.8})显然可以得出, 通过合适地选择$\,\mathscr{U}$, 相因子$\,\mathscr{P}^{\prime}\,$可以变成任何我们想要的东西. 选择$\,\mathscr{P}^{\prime}=+\mi\,$将是方便的, 这使得方程(\ref{25.3.7})取如下的简单形式(现在扔掉撇号)
\begin{equation}
    \mathsf{P}^{-1}\mathcal{Q}_{a}\mathsf{P} = \mi\sum_{b}e_{ab}\mathcal{Q}_{b}^{\ast} \:. \label{25.3.11}
\end{equation}
就像我们对旋量场算符做的那样, 如果我们将$\,(0,1/2)\,$算符$\,\mathcal{Q}_{a}\,$和$\,(1/2,0)\,$算符$\,\sum_{b}e_{ab}\mathcal{Q}_{b}^{\ast}\,$%
并入方程 (\ref{25.2.34})定义的\,Dirac\,旋量生成元$\,Q_{\alpha}$, 空间反演的表示会更加简单. 写成这些项, 方程(\ref{25.3.11})和它的共轭变成
\begin{equation}
    \mathsf{P}^{-1}Q\mathsf{P}=\mi\beta\,Q\:. \label{25.3.12}
\end{equation}
(我们对\,Dirac\,矩阵使用的约定是\,5.4\,节和本卷前言给出的那个约定, 这个约定中
\[
\beta = \begin{pmatrix}
0 & 1 \\ 1 & 0
\end{pmatrix} \:,
\]
其中$\,1\,$和$\,0\,$被理解成$\,2\times2\,$子矩阵.)

对于扩充超对称, 选择$\,\mathscr{U}\,$使得$\,\mathscr{P}^{\prime}\,$对角不总是可行的. 然而, 矩阵代数中的(第\,2\,章附录\,C\,证明的)一个定理表明, 选择$\,\mathscr{U}\,$使得$\,\mathscr{P}^{\prime}\,$分块对角是可行的, 其中, 一般而言, 主对角上$\,1\times1\,$子矩阵可以选成$\,\mi\,$(或者任何其它我们想要的相因子), 而其它子矩阵是$\,2\times2\,$矩阵, 可以对其进行选择使其拥有如下形式
\[
\begin{pmatrix}
0 & \exp(\mi\phi) \\ \exp(-\mi\phi) & 0
\end{pmatrix} \:,
\]
其中$\,\phi\,$是各种相位. 相应地, 在$\,\mathscr{U}\,$的这个选择下, (现在扔掉撇号) 有两种\,2\,分量$\,\mathcal{Q}$. 第一种$\,\mathcal{Q}\,$也满足方程(\ref{25.3.11}):
\begin{equation}
    \mathsf{P}^{-1}\mathcal{Q}_{ar}\mathsf{P}=\mi\sum_{b}e_{ab}\mathcal{Q}_{br}^{\ast} \:. \label{25.3.13}
\end{equation}
第二种的\,2\,分量$\,\mathcal{Q}\,$成对出现, 我们记做$\,\mathcal{Q}_{a\,s1}\,$和$\,\mathcal{Q}_{a\,s2}$, 其中第\,$s$\,对的宇称变换规则是
\begin{equation}
    \mathsf{P}^{-1}\mathcal{Q}_{a\,s1}\mathsf{P} = \me^{\mi \phi_{s}}\sum_{b}e_{ab}\mathcal{Q}_{b\,s2}^{\ast} \:,
    \qquad
    \mathsf{P}^{-1}\mathcal{Q}_{a\,s2}\mathsf{P} = \me^{\mi \phi_{s}}\sum_{b}e_{ab}\mathcal{Q}_{b\,s1}^{\ast} \:,
    \label{25.3.14}
\end{equation}
特别地, 我们现在有
\begin{equation}
    \mathsf{P}^{-2}\mathcal{Q}_{a\,s1}\mathsf{P}^{2} = -\me^{2\mi\phi_{s}}\mathcal{Q}_{a\,s1}\:, \qquad
    \mathsf{P}^{-2}\mathcal{Q}_{a\,s2}\mathsf{P}^{2} = -\me^{-2\mi\phi_{s}}\mathcal{Q}_{a\,s2}\:.\label{25.3.15}
\end{equation}
这表明, 除非$\,\phi_{s}=0\:(\operatorname{mod}\uppi)$, 否则不可能用第二种扩充超对称性生成元的线性组合来构建第一种超对称性生成元.

写成\,4\,分量旋量(\ref{25.2.34})的形式, 宇称算符在第一种扩充超对称性生成元上的效果是
\begin{equation}
    \mathsf{P}^{-1}Q_{r}\mathsf{P}=\mi\beta\,Q_{r} \:, \label{25.3.16}
\end{equation}
对于第二种生成元则是
\begin{equation}
    \mathsf{P}^{-1} Q_{s1}\mathsf{P}=\beta\,\gamma_{5}\,\exp(\mi\gamma_{5}\phi_{s})Q_{s2}\:, \qquad
    \mathsf{P}^{-1} Q_{s2}\mathsf{P}=\beta\,\gamma_{5}\,\exp(-\mi\gamma_{5}\phi_{s})Q_{s1}\:. \label{25.3.17}
\end{equation}

\section{无质量粒子的超多重态} \label{sec:25.4}

超对称要求已知的粒子在超对称代数的不可约表示下要伴随``超粒子''(sparticles): 伴随夸克和轻子的是玻色``标量夸克''(squarks)和``标量轻子''(sleptons), 伴随规范玻色子的是费米``规范微子''(gauginos). 所有这些粒子还都没有观测到, 所以超对称性肯定破缺, 而超粒子的质量几乎肯定要比电弱$\,SU(2)\times U(1)\,$规范群自发破缺产生的夸克质量, 轻子质量和规范玻色子质量大得多, 因此和超对称多重态内部分裂的大小处于同一量级. 因此, 在能标足够高时很有可能我们可以忽略掉超对称破缺和这些质量分裂, 此时我们也可以将已知的夸克, 轻子, 规范玻色子和它们的超对称伴视作是无质量的. 因此, 我们对无质量粒子的超对称多重态很感兴趣.

考察这样一个态, 它只包含一个无质量粒子且这个粒子属于某个超对重态. 通过用算符$\,\mathcal{Q}_{ar}$ 和(或)$\,\mathcal{Q}_{ar}^{\ast}\,$作用这个态, 我们可以获得同一超多重态中的其它态. 由于$\,\mathcal{Q}_{ar}\,$和$\,\mathcal{Q}_{ar}^{\ast}\,$与$\,P_{\mu}\,$对易, 所有这些态有相同的$\,4\,$-动量值. 我们将在这些态的\,4\,-动量是$\,p^{1}=p^{2}=0\,$且$\,p^{3}=p^{0}=E\,$的\,Lorentz\,参考系下进行处理. 在\,4\,-动量的这一选择下, 我们有
\begin{equation}
    \sigma_{\mu}p^{\mu} = E(\sigma_{0}+\sigma_{3})= 2E
    \begin{pmatrix}
    1 & 0 \\ 0 & 0
    \end{pmatrix} \:, \label{25.4.1}
\end{equation}
除去因子$\,2E\,$来看, 这是到螺旋度为$\,+1/2\,$的子空间上的投射矩阵. 因此, 反对易关系(\ref{25.2.7})表明, 对于有这种\,4\,-动量的超多重态, $\{\mathcal{Q}_{(-1/2)\,r}, \mathcal{Q}^{\ast}_{(-1/2)\,r} \}\,$作用在这个超多重态中的任何态上都给出零, 同理, $\mathcal{Q}_{(-1/2)\,r}\,$和$\,\mathcal{Q}^{\ast}_{(-1/2)\,r}\,$也是如此. 因此我们只能通过用$\,\mathcal{Q}_{(1/2)\,r}\,$和$\,\mathcal{Q}^{\ast}_{(1/2)\,r}\,$进行作用来构建%
这个超多重态中的态. 更进一步, 我们可以用$\,\mathcal{Q}\,$的$\,J_{3}\,$值来标记它们, 也就是说
\begin{equation}
    [J_{3},\mathcal{Q}_{ar}] =-a\,\mathcal{Q}_{ar}\:, \label{25.4.2}
\end{equation}
所以$\,\mathcal{Q}_{(1/2)\,r}\,$和$\,\mathcal{Q}^{\ast}_{(1/2)\,r}\,$分别将螺旋度减小和提高$\,1/2$.

我们首先来考察简单超对称的情况. 考察一个最大螺旋度为$\,\lambda_{\mathrm{max}}\,$的超多重态, 用$\,\lvert \lambda_{\mathrm{max}}\rangle\,$标记任何螺旋度为$\,\lambda_{\mathrm{max}}\,$的单粒子态, 并设它的\,4\,-动量是$\,p^{\mu}$. 那么
\begin{equation}
    \mathcal{Q}_{\frac{1}{2}}^{\ast}\lvert \lambda_{\mathrm{max}}\rangle = 0 \:, \label{25.4.3}
\end{equation}
用$\,\mathcal{Q}_{1/2}\,$作用这个态则给出螺旋度为$\,\lambda_{\mathrm{max}}-1/2\,$的态%
$\,\lvert\lambda_{\mathrm{max}}-1/2\rangle$. 我们会将这个态定义成
\begin{equation}
    \lvert \lambda_{\mathrm{max}} - 1/2\rangle \equiv (4E)^{-1/2}\,\mathcal{Q}_{\frac{1}{2}}\,\lvert \lambda_{\mathrm{max}}\rangle \:. \label{25.4.4}
\end{equation}
加上方程(\ref{25.4.1})和(\ref{25.4.3}), 基础反对易关系(\ref{25.2.7})表明这个态的归一化方式与$\,\lvert\lambda_{\mathrm{max}}\rangle\,$相同
\begin{equation}
    \langle \lambda_{\mathrm{max}}-1/2\vert \lambda_{\mathrm{max}}-1/2\rangle
    = \langle \lambda_{\mathrm{max}}\vert \lambda_{\mathrm{max}} \rangle \:,\label{25.4.5}
\end{equation}
特别地, 这个态不能为零. 方程(\ref{25.2.32})表明$\,\mathcal{Q}_{1/2}^{2}=0$, 所以用$\,\mathcal{Q}_{1/2}\,$作用$\,\lvert\lambda_{\mathrm{max}}-1/2\rangle\,$给出零:
\begin{equation}
    \mathcal{Q}_{\frac{1}{2}}\lvert \lambda_{\mathrm{max}}-1/2\rangle =
    (4E)^{-1/2}\mathcal{Q}_{\frac{1}{2}}^{2}\lvert \lambda_{\mathrm{max}}\rangle = 0 \:.\label{25.4.6}
\end{equation}
另一方面, 用$\,\mathcal{Q}_{1/2}^{\ast}\,$作用这个态给出的是我们作为出发点的态. 即,
\[
 \mathcal{Q}_{\frac{1}{2}}^{\ast}\lvert \lambda_{\mathrm{max}}-1/2\rangle =
    (4E)^{-1/2}\mathcal{Q}_{\frac{1}{2}}^{\ast}\mathcal{Q}_{\frac{1}{2}}\lvert \lambda_{\mathrm{max}}\rangle
    = (4E)^{-1/2}\{\mathcal{Q}_{\frac{1}{2}}^{\ast},\mathcal{Q}_{\frac{1}{2}}\}\lvert \lambda_{\mathrm{max}}\rangle \:,
\]
这使得方程(\ref{25.4.1})和反对易关系(\ref{25.2.31})产生
\begin{equation}
    \mathcal{Q}_{\frac{1}{2}}^{\ast}\lvert \lambda_{\mathrm{max}}-1/2\rangle =
    (4E)^{1/2} \lvert \lambda_{\mathrm{max}}\rangle \:. \label{25.4.7}
\end{equation}
因此超多重态仅由两个态构成, 螺旋度分别是$\,\lambda_{\mathrm{max}}\,$和$\,\lambda_{\mathrm{max}}-1/2$. 在这两个态提供的基下, 算符$\,\mathcal{Q}_{1/2}\,$和$\,\mathcal{Q}_{1/2}^{\ast}\,$被表示成矩阵
\begin{equation}
    q_{\frac{1}{2}} = \sqrt{4E} \begin{pmatrix}
    0 & 0 \\ 1 & 0
    \end{pmatrix} \:, \qquad
    q_{\frac{1}{2}}^{\dag} = \sqrt{4E} \begin{pmatrix}
    0 & 1 \\ 0 & 0
    \end{pmatrix} \:, \label{25.4.8}
\end{equation}
而算符算符$\,\mathcal{Q}_{-1/2}\,$和$\,\mathcal{Q}_{-1/2}^{\ast}\,$被表示成零.

值得强调的是, 在有简单超对称性的理论中, 这是{\kai{唯一}}一种无质量超多重态. 不存在没有超对称伴的无质量粒子, 也不存在有多个超对称伴的无质量粒子. 当然, $\mathsf{CPT}\,$不变性暗示了, 对于每个螺旋度为$\,\lambda\,$和$\,\lambda-1/2\,$的无质量粒子超多重态, 必存在一个螺旋度为$\,-\lambda+1/2\,$和$\,-\lambda\,$的反多重态. 特别地, 螺旋度为$\,+1/2\,$和$\,-1/2\,$的无质量粒子和反粒子不是伴随着螺旋度为$\,+1\,$和$\,-1\,$的无质量粒子和反粒子, 就是伴随着螺旋度均为零的无质量粒子和反粒子.

那么已知的夸克, 轻子和规范玻色子该如何填进这个图景中呢? 我们将假定超对称性生成元与$\,SU(3)\times SU(2)\times U(1)\,$规范群的生成元对易.\footnote{在简单超对称性中, 由于$\,SU(3)\times SU(2)\,$这样的半单\,Lie\,代数没有不平庸的一维表示, 所以生成元$\,Q_{\alpha}\,$在任何情况下都必须与$\,SU(3)\times SU(2)\,$生成元对易.} 夸克和轻子所属的规范群表示与规范玻色子所属的规范群表示不同, 所以它们不可能处在同一个超多重态中. 由此我们不得不得到如下的结论: 在$\,SU(2)\times U(1)\,$对称性破缺可以被忽略的高能极限下, 每种色和味的无质量夸克和轻子都伴随着色和味相同但螺旋度为零的无质量标量夸克和标量轻子{\kai{成对}}出现在超多重态中, 而与无质量规范玻色子伴随的螺旋度为$\,\pm1/2\,$的无质量规范微子构成了$\,SU(3)\times SU(2)\times U(1)\,$的一个伴随表示.

由于存在引力, 我们知道除了标准模型的粒子以外还必须存在螺旋度为$\,\pm2\,$的粒子, {\kai{引力子}}. 对于螺旋度为$\,\lambda\,$的无质量粒子, 如果$\,\lvert\lambda\rvert >1/2$, 那么在低动量时它必须与守恒量耦合.\footnote{我们在\,13.1\,节讨论过螺旋度是整数的情况. 关于半整数螺旋度的讨论是由\,Grisaur\,和\,Pendleton\,给出的.\cite{3}} 螺旋度为$\,\pm1\,$的软无质量粒子可以与各种内部对称性的生成元耦合, 螺旋度为$\,\pm3/2\,$的软无质量粒子可以与超对称生成元$\,\mathcal{Q}_{a}\,$耦合, 而螺旋度为$\,\pm2\,$的软粒子可以与一个守恒量耦合, 动量\,4\,-矢$\,P_{\mu}$, 但是没有什么守恒量可以与$\,\lvert\lambda\rvert>2\,$的软粒子耦合. 由此我们得出, 引力子所处的超多重态不能含有螺旋度为$\,\pm5/2\,$的粒子, 所以它所处的超多重态必须含有螺旋度为$\,\pm3/2\,$的粒子, 这个粒子被称为{\kai{引力微子}}(gravitino), 它与超对称性生成元自身耦合. 这个超多重态的场论被称为{\kai{超引力}}, 我们将在第\,31\,章讨论.

现在我们来考察有$\,N\,$个超对称生成元的扩充超对称性情况. 我们首先注意到, $\mathcal{Q}_{(-1/2)\,r}\,$作用在超多重态中的态(包含那些通过$\,\mathcal{Q}_{(1/2)\,s}$\,作用这个多重态的任何其它态获得的态)上时都给出零, 所以$\,Z_{rs}\,$也湮灭这个多重态的任何态.
在中心荷不在这个图景的前提下, 当超对称生成元$\,\mathcal{Q}_{(1/2)\,r}$ 作用在无质量粒子超多重态上时, 它们反对易, 所以用$\,n\,$个这样的生成元作用拥有最大螺旋度 $\lambda_{\mathrm{max}}$ 且动量为$\,p^{\mu}\,$的单粒子态时, 我们会得到$\,N!/n!(N-n)!\,$个螺旋度为$\,\lambda_{\mathrm{max}}-n/2\,$且动量相同的单粒子态, 这些单粒子态构成$\,SU(N)\,R$-对称性\footnote{$U(N)\,R$-对称性的$\,U(1)\,$部分通常会被量子力学反常破坏掉.}%
(\ref{25.2.30})的一个$\,n\,$阶反对称表示. 能够给出一个非零态的最大$\,n\,$值是$\,n=N$, 所以一个超多重态中的最小螺旋度是
\begin{equation}
    \lambda_{\mathrm{min}} =\lambda_{\mathrm{max}}-N/2 \:. \label{25.4.9}
\end{equation}
如果我们希望排除掉那些螺旋度$\,\lambda\,$满足$\,\lvert\lambda\rvert>2\,$的无质量粒子, 那么$\,\lambda_{\mathrm{max}}-\lambda_{\mathrm{min}}\leq 4$, 所以只有那些$\,N\leq 8\,$的扩充超对称性才是被允许的.

当$\,N=8\,$且$\,\lvert\lambda\rvert>2\,$的螺旋度被排除时, 只有一种可能的超多重态, 组成它的是: 螺旋度为$\,\pm2\,$的引力子各\,1\,个; 螺旋度为$\,\pm3/2\,$的引力微子各$\,8\,$个; 螺旋度为$\,\pm1\,$的规范玻色子各$\,28\,$个; 螺旋度为$\,\pm1/2\,$的费米子各$\,56\,$个; 以及$\,70\,$个螺旋度为零的玻色子.

对比$\,N=7\,$的情况, 依旧排除掉$\,\lvert\lambda\rvert>2\,$的螺旋度. 这里存在两个超多重态. 一个超多重态包含: 1\,个螺旋度为$\,+2\,$的引力子; 7\,个螺旋度为$\,+3/2\,$的引力微子; 21\,个螺旋度为$\,+1\,$的规范玻色子; 35\,个螺旋度为$\,+1/2\,$的费米子; 35\,个螺旋度为零的玻色子; 21\,个螺旋度为$\,-1/2\,$的费米子; 7\,个螺旋度为$\,-1\,$的规范玻色子; 以及\,1\,个螺旋度为$\,-3/2\,$的引力微子. 另一个是\,$\mathsf{CPT}\,$-共轭的超多重态, 它的所有螺旋度反号. 将这两个超多重态中的粒子数加一下, 我们有螺旋度为$\,\pm2\,$的引力子各\,1\,个; 螺旋度为$\,\pm3/2\,$的引力微子各$\,7+1=8\,$个; 螺旋度为$\,\pm1\,$的规范玻色子各$\,21+7=28\,$个; 螺旋度为$\,\pm1/2\,$的费米子各$\,35+21=56\,$个; 以及$\,35+35=70\,$个螺旋度为零的玻色子. 因此$\,N=8\,$和$\,N=7\,$的扩充超引力理论有精确相同的粒子内容, 它们实质上是等价的.

另一方面, 在$\,N\leq6\,$的扩充超引力理论中, 螺旋度为$\,\pm3/2\,$的引力微子分别只有$\,N\,$个, 因此它们都是不同的.

当$\,N\leq 4\,$时, 还存在{\kai{整体}}超对称理论的可能性, 在这样的理论中, 超多重态没有引力子和引力微子. 对于整体$\,N=4\,$超对称性, 仅存在一个超多重态, 组成它的是: 螺旋度为$\,\pm1\,$的规范玻色子各\,1\,个; 螺旋度为$\,\pm 1/2\,$的费米子各\,4\,个; 以及\,6\,个螺旋度为零的玻色子. 这等价于$\,N=3\,$的整体超对称理论, 它有两个超多重态: 一个超多重态有\,1\,个螺旋度为$\,+1\,$的规范玻色子; 3\,个螺旋度为$\,+1/2\,$的费米子; 3\,个螺旋度为零的玻色子; 以及\,1\,个螺旋度为$\,-1/2\,$的费米子; 而另一个\,$\mathsf{CPT}$\,共轭的超多重态有相反的螺旋度. 将这两个$\,N=3\,$超多重态中螺旋度相同的粒子数加起来就给出了同$\,N=4\,$整体超对称性相同的粒子内容. 拥有$\,N=4\,$超对称性的规范场论拥有显著的性质, 这将在\,27.9\,节进行讨论.

对于$\,N=2\,$的扩充超对称, 除了那些通过$\,\mathsf{CPT}\,$关联的超多重态外, 存在两种不同类型的超多重态. 一种是{\kai{规范}}超多重态, 它们包含一个螺旋度为$\,+1\,$的规范玻色子, 两个螺旋度为$\,+1/2\,$的费米子, 并且这两个费米子构成了$\,SU(2)\,R$-对称性下的一个双重态, 以及一个螺旋度为零的玻色子, 再加上螺旋度都反号的$\,\mathsf{CPT}\,$-共轭超多重态. 每个规范超多重态和和它的反多重态合起来包含了: 螺旋度为$\,\pm1\,$的规范玻色子各\,1\,个, 螺旋度为$\,\pm1/2\,$的费米子双重态各一个, 以及两个螺旋度为零的$\,SU(2)\,$单态玻色子. 另一种是{\kai{极多重态}}(hypermultiplet), 它们包含螺旋度$\,\pm1/2\,$的费米子各一个, 以及螺旋度为零的玻色子构成的$\,SU(2)\,$双重态, 再加上这种超多重态的$\,\mathsf{CPT}$-共轭. (在量子场论中, 极多重态不能是自身的反多重态, 若非如此, 螺旋度为零的粒子将会被两个{\kai{实}}标量场描述, 而它们是无法形成一个$\,SU(2)\,$多重态的.) 当然, 在真实世界中还必须存在引力子超多重态, 它包含一个螺旋度为$\,+2\,$的引力子, 一个螺旋度为$\,+3/2\,$的引力微子的$\,SU(2)\,$双重态, 以及一个螺旋度为$\,+1\,$的规范玻色子, 再加上它们螺旋度相反的$\,\mathsf{CPT}\,$-共轭. 我们将在\,27.9\,节构造$\,N=2\,$超对称规范理论, 并在\,29.5\,节以非微扰的方式探索它的性质.

在可达到的能标试图将扩充超对称性融入进粒子的真实理论时, 这些超多重态的粒子内容反映出了一个困难. 除了一种情况外, 在任何其它情况中, 螺旋度$\,+1/2\,$的费米子和螺旋度$\,+1\,$的规范玻色子同属一个超多重态. 规范玻色子属于规范群的伴随表示, 所以如果超对称生成在规范群下不变, 那么螺旋度$\,+1/2\,$的费米子也必须属于伴随表示, 而这是一个实表示. 而已知的夸克和轻子所属的$\,SU(3)\times SU(2)\times U(1)\,$表示是{\kai{手征的}}------即, 对于这样的表示, 螺旋度$\,+1/2\,$的费米子属于一个复表示, 那么它们的$\,\mathsf{CPT}\,$共轭, 螺旋度$\,-1/2\,$的费米子构建的表示肯定与这个表示不同, 二者是矛盾的. 唯一的例外是上面讨论的$\,N=2\,$极多重态, 在这种情况中, 螺旋度$\,+1/2\,$的费米子不在规范玻色子所处的超多重态中. 但在这一情况中, 螺旋度为$\,+1/2\,$和$\,-1/2\,$的粒子处在同一超多重态中, 因此在任何使得超对称生成元不变的规范变换下, 它们的变换必须相同. 它们可能属于这个规范群的一个复表示, 那么这个极多重态的$\,\mathsf{CPT}\,$-共轭就属于复共轭表示, 这样一来, 各个螺旋度的费米子属于两个表示的和, 这是实的, 依旧与已知夸克和轻子的手征性相矛盾.

与之相反, 对于简单超对称性存在只包含螺旋度$\,+1/2\,$和零的超多重态, 它们可能处在规范群的一个复表示中并与$\,\mathsf{CPT}\,$-共轭超多重态构建的表示不同. 这里不存在与手征性相悖的矛盾. 由于这个原因, 将超对称在可实现的能标视为没有破缺的对称性的讨论大多集中在简单超对称性而非扩充超对称性上.


\section{有质量粒子的超多重态} \label{sec:25.5}

尽管已知的夸克, 轻子和规范玻色子和它们的超对称伴在超对称破缺可以被忽视的能标处可以被视为是无质量的, 但对于其它粒子, 包括统一强相互作用和电弱相互作用的理论要求的质量很大的额外规范玻色子, 这不一定是成立的. 另外, 自\,Wess-Zumino\,模型起, 对于研究超对称理论, 有质量粒子的理论就已经是个很有用的测试情况. 因此对我们来说, 简要地考察未破缺超对称对有质量粒子的意义将是值得的.

就像在上一节, 通过用算符$\,\mathcal{Q}_{ar}\,$和$\,\mathcal{Q}^{\ast}_{ar}$作用超多重态中的任何一个单粒子态, 我们获得了超多重态中的各种单粒子态, 并且所有这些态有相同的\,4\,-动量. 不同于零质量的情况, 当质量$\,M>0\,$时, 我们现在可以取静止粒子的\,4\,-动量, 其中$\,i=1,2,3\,$的$\,p^{i}=0\,$且$\,p^{0}=M$. 在这个参考系下, 我们有
\begin{equation}
    \sigma_{\mu}p^{\mu} = M\sigma_{0} = M
    \begin{pmatrix}
    1 & 0 \\ 0 & 1
    \end{pmatrix} \:. \label{25.5.1}
\end{equation}
因此, 作用在有这一\,4\,-动量的超多重态中的任何态$\,\lvert\:\rangle\,$上, 反对易关系(\ref{25.2.7})给出
\begin{equation}
    \{\mathcal{Q}_{ar},\mathcal{Q}_{bs}^{\ast}\}\lvert\:\rangle = 2M\,\delta_{ab}\,\delta_{rs}\,\lvert\:\rangle \:.\label{25.5.2}
\end{equation}
与零质量的情况相反, 这里没有哪个$\,\mathcal{Q}_{ar}\,$或$\,\mathcal{Q}_{ar}^{\ast}\,$的分量可以在整个多重态上为零, 所以我们有两组上升和下降算符: $\mathcal{Q}_{(1/2)\,r}\,$和$\,\mathcal{Q}^{\ast}_{(-1/2)\,r}\,$均将自旋\,3\,-分量降低\,1/2, 而$\,\mathcal{Q}_{(-1/2)\,r}\,$和$\,\mathcal{Q}^{\ast}_{(1/2)\,r}\,$均将自旋\,3\,-分量提高\,1/2. 然而, 我们将会看到, 对于扩充超对称性, $Q\,$和$\,Q^{\ast}\,$的特定线性组合有可能为零.

我们将首先考察简单超对称的情况. 通过使用超对称代数(\ref{25.2.31})和(\ref{25.2.32}), 我们将证明一般的有质量超多重态由一个自旋\,$j+1/2\,$的粒子, 一{\kai{对}}自旋$\,j\,$的粒子和一个自旋$\,j-1/2\,$的粒子构成. 当宇称守恒时, 自旋为$\,j\pm1/2\,$的两个粒子拥有相同的内禀宇称, 由某个相位$\,\eta\,$给定, 而两个自旋$\,j\,$的粒子分别有宇称$\,+\mi\eta\,$和$\,-\mi\eta$. 这里的$\,j\,$是大于零的整数或半整数. 同时还存在坍缩超多重态, 它由两个自旋零的粒子一个自旋$\,1/2\,$的粒子构成. 当宇称守恒时, 自旋零的粒子有宇称$\,\mi\eta\,$和$\,-\mi\eta$, 其中$\,\eta\,$是自旋$\,1/2\,$粒子的宇称.

下面是证明. 我们首先证明任何超多重态将包含至少一个自旋多重态$\,\lvert j,\sigma\rangle$, 其中自旋\,3\,-分量$\,\sigma\,$以一为步长从$\,-j\,$取到$\,+j$, 它有特殊性质, 对于所有这样的$\,\sigma\,$和$\,a=\pm1/2$,
\begin{equation}
    \mathcal{Q}_{a}\,\lvert j,\sigma \rangle =0 \:. \label{25.5.3}
\end{equation}
从这个超多重态中的任何非零态$\,\lvert\psi\rangle\,$出发, 我们可以定义非零态
\[
\lvert \psi^{\prime}\rangle\equiv
\begin{cases}
(2M)^{-1/2}\mathcal{Q}_{1/2}\lvert\psi \rangle & \qquad \mathcal{Q}_{1/2}\lvert\psi\rangle \neq 0 \\
\lvert \psi \rangle & \qquad \mathcal{Q}_{1/2}\lvert\psi\rangle =0
\end{cases} \:,
\]
和
\[
\lvert \psi^{\prime\prime}\rangle\equiv
\begin{cases}
(2M)^{-1/2}\mathcal{Q}_{-1/2}\lvert\psi^{\prime} \rangle & \qquad \mathcal{Q}_{-1/2}\lvert\psi^{\prime}\rangle \neq 0 \\
\lvert \psi^{\prime} \rangle & \qquad \mathcal{Q}_{-1/2}\lvert\psi^{\prime}\rangle =0
\end{cases} \:.
\]
由于$\,\mathcal{Q}_{a}\,$反对易, $\mathcal{Q}_{1/2}\lvert\psi^{\prime}\rangle=0$, 因此对于$\,a=\pm1/2\,$有$\,\mathcal{Q}_{a}\lvert\psi^{\prime\prime}\rangle=0$. 如果任何态$\,\lvert\psi^{\prime\prime}\rangle\,$满足条件$\,\mathcal{Q}_{a}\lvert\psi^{\prime\prime}=0$, 那么对于表示任意空间旋转的幺正表示$\,U(R)$, $U(R)\lvert\psi^{\prime\prime}\rangle\,$也满足这个条件. 由此得出满足这个条件态可以被分解进完整的自旋多重态$\,\lvert j,\sigma\rangle$, 它满足条件(\ref{25.5.3})

现在集中在任何一个满足方程(\ref{25.5.3})的自旋多重态上, 对它进行归一化使得
\begin{equation}
    \langle j,\sigma^{\prime}\vert j,\sigma \rangle = \delta_{\sigma^{\prime}\sigma} \:. \label{25.5.4}
\end{equation}
当$\,j>0\,$时, 通过用自旋$\,1/2\,$算符\footnote{既然$\,\mathcal{Q}_{a}\,$在旋转下的变换就像湮灭一个自旋为$\,1/2\,$且\,3\,-分量%
为$\,\sigma\,$的粒子的场, 那么$\,\mathcal{Q}_{a}^{\ast}\,$的变换就像产生这种粒子的场, 因此它的变换就像粒子本身. 形式上, 因为$[J_{i},\mathcal{Q}_{a}]=-\sum_{b}\tfrac{1}{2}(\sigma_{i})_{ab}\mathcal{Q}_{b}$, 所以$[J_{i},\mathcal{Q}^{\ast}_{a}]=-\sum_{b}\tfrac{1}{2}(\sigma_{i})_{ba}\mathcal{Q}_{b}^{\ast}$, 它可以与自旋$\,1/2\,$粒子的变换性质, $J_{i}\lvert a\rangle=\sum_{b}\tfrac{1}{2}(\sigma_{i})_{ba}\lvert b\rangle$, 相比较.}$\,\mathcal{Q}_{a}^{\ast}\,$作用这些态, 我们可以构造出自旋$\,j\pm1/2\,$的态:
\begin{equation}
    \lvert j\pm 1/2,\sigma \rangle = \frac{1}{\sqrt{2M}}
    \sum_{a}C_{\frac{1}{2}\,j}\Bigl(j\pm 1/2 ,\sigma\, ;a, \sigma-a\Bigr)\, \mathcal{Q}_{a}^{\ast} \lvert j,\sigma-a\rangle\:,
    \label{25.5.5}
\end{equation}
其中$\,C_{jj^{\prime}}(j^{\prime\prime},\sigma^{\prime\prime};\sigma,\sigma^{\prime})\,$是%
传统的\,Clebsch-Gordan\,系数, 它将\,3\,-分量为$\,\sigma\,$和$\,\sigma^{\prime}\,$的自旋$\,j\,$和$\,j^{\prime\prime}\,$耦合成 3\,-分量为%
$\,\sigma^{\prime\prime}\,$的自旋$\,j^{\prime\prime}$. 利用方程(\ref{25.5.2})---(\ref{25.5.5})和%
\,Clebsch-Gordan\,系数的正交性, 我们可以证明这些态的归一化是正确的:
\begin{equation}
    \langle j\pm 1/2,\sigma \vert j\pm 1/2,\sigma^{\prime}\rangle =\delta_{\sigma\sigma^{\prime}}\:,\qquad
    \langle j\pm 1/2,\sigma \vert j\mp 1/2,\sigma^{\prime}\rangle =0 \:, \label{25.5.6}
\end{equation}
所以态$\,\lvert j\pm 1/2,\sigma\rangle\,$中的任何一个都不能为零. 唯一的例外是$\,j=0$, 这时显然是由于不存在态$\,\lvert j-1/2,\sigma\rangle$. 我们以可以通过作用{\kai{两}}个$\,\mathcal{Q}^{\ast}\,$%
在$\,\lvert j,\sigma\rangle\,$上获得其它态. 由于每个$\,\mathcal{Q}_{a}^{\ast}\,$与它自身反对易, 唯一这样的非零态是通过作用算符$\,\mathcal{Q}_{1/2}^{\ast}\mathcal{Q}_{-1/2}^{\ast}
=-\mathcal{Q}_{-1/2}^{\ast}\mathcal{Q}_{1/2}^{\ast}\,$形成的. 这个算符可以写成$\,\frac{1}{2}e_{ab}\mathcal{Q}_{a}^{\ast}\mathcal{Q}_{b}^{\ast}$, 这表明它是一个旋转不变量, 所以这给出了第二个自旋为$\,j\,$的自旋多重态:
\begin{equation}
    \lvert j,\sigma\rangle^{\flat} = \frac{1}{2M} \, \mathcal{Q}_{1/2}^{\ast}\, \mathcal{Q}_{-1/2}^{\ast} \lvert j,\sigma \rangle\:,
    \label{25.5.7}
\end{equation}
它与$\,\lvert j,\sigma\rangle\,$不同是因为, 取代方程(\ref{25.5.3}), 我们有
\begin{equation}
    \mathcal{Q}_{a}^{\ast} \lvert j,\sigma\rangle^{\flat} = 0  \:. \label{25.5.8}
\end{equation}
再次使用方程(\ref{25.5.2})---(\ref{25.5.4}), 我们发现它们也是归一化态:
\begin{equation}
    {}^{\flat}\langle j,\sigma^{\prime}\vert j,\sigma\rangle^{\flat}=\delta_{\sigma^{\prime}\sigma}\:, \qquad
    \langle j,\sigma^{\prime}\vert j,\sigma\rangle ^{\flat} = 0\:. \label{25.5.9}
\end{equation}
那么很容易证明迄今为止构造的态构成了超对称代数的一个完整表示. Clebsch-Gordan\,系数的正交性使得我们可以将方程(\ref{25.5.5})重写成
\begin{equation}
    \mathcal{Q}_{a}^{\ast}\lvert j,\sigma\rangle = \sqrt{2M}\sum_{\pm}
    C_{\frac{1}{2}\,j}\Bigl(j\pm1/2,\sigma+a\,;a,\sigma\Bigr)\,\lvert j\pm1/2,\sigma+a\rangle \:. \label{25.5.10}
\end{equation}
另外, 方程(\ref{25.5.2})表明, 对于超对重态中的任何态$\,\lvert \:\rangle$,
\begin{equation}
    \Bigl[\mathcal{Q}_{a}, \mathcal{Q}_{\frac{1}{2}}^{\ast}\mathcal{Q}_{-\frac{1}{2}}^{\ast}\Bigr]\,\lvert \:\rangle
    =2M \sum_{b}e_{ab}\,\mathcal{Q}_{b}^{\ast}\,\lvert\:\rangle \:, \label{25.5.11}
\end{equation}
所以方程(\ref{25.5.7})和(\ref{25.5.3})给出
\begin{align}
    & \mathcal{Q}_{a}\,\lvert j,\sigma\rangle^{\flat} = \sum_{b} e_{ab}\,\mathcal{Q}_{b}^{\ast}\,\lvert j,\sigma\rangle \nonumber \\
    & \quad = \sqrt{2M}\sum_{b}e_{ab}\sum_{\pm}C_{\frac{1}{2}\,j}\Bigl(j\pm 1/2,\sigma+b\,;b,\sigma\Bigr)\,
    \lvert j\pm 1/2,\sigma+b\rangle \:. \label{25.5.12}
\end{align}
从方程(\ref{25.5.2}), (\ref{25.5.3})和(\ref{25.5.5})中我们得出
\begin{equation}
    \mathcal{Q}_{a}\,\lvert j\pm1/2,\sigma\rangle = \sqrt{2M}C_{\frac{1}{2}\,j}\Bigl(j\pm1/2,\sigma\,;a,\sigma-a\Bigr)\,
    \lvert j,\sigma-a\rangle \:, \label{25.5.13}
\end{equation}
而方程(\ref{25.5.5}), (\ref{25.2.31})和(\ref{25.5.7})给出
\begin{equation}
    \mathcal{Q}_{a}^{\ast}\,\lvert j\pm 1/2,\sigma\rangle = \sqrt{2M}\sum_{b}e_{ab}\,
    C_{\frac{1}{2}\,j}\Bigl(j\pm 1/2,\sigma\,;b,\sigma-b\Bigr)\,\lvert j,\sigma-b\rangle^{\flat}\:. \label{25.5.14}
\end{equation}
方程(\ref{25.5.3}), (\ref{25.5.8}), (\ref{25.5.10})和(\ref{25.5.12})---(\ref{25.5.14})给出了$\,\mathcal{Q}\,$和$\,\mathcal{Q}^{\ast}\,$在这个
超多重态中的所有态上的作用.

对于$\,j=0\,$我们有坍缩超多重态: 方程(\ref{25.5.3}), (\ref{25.5.8}), (\ref{25.5.10})和(\ref{25.5.12})---(\ref{25.5.14})变成
\begin{align}
    &\mathcal{Q}_{a}\,\lvert0,0\rangle =0 \:, && \mathcal{Q}_{a}^{\ast}\,\lvert0,0\rangle^{\flat} =0\:, \nonumber \\
    &\mathcal{Q}_{a}^{\ast}\,\lvert0,0\rangle =\sqrt{2M}\,\lvert 1/2\rangle\:,
    &&\mathcal{Q}_{a}\,\lvert0,0\rangle^{\flat}=\sqrt{2M}\,{\textstyle\sum}_{b}\,e_{ab}\,\lvert 1/2,b\rangle\:, \nonumber\\
    &\mathcal{Q}_{a}\,\lvert 1/2,b\rangle =\sqrt{2M}\,\delta_{ab}\,\lvert0,0\rangle \:,
    && \mathcal{Q}_{a}^{\ast}\,\lvert1/2,b\rangle=\sqrt{2M}\,e_{ab}\lvert 0,0\rangle^{\flat}\:. \label{25.5.15}
\end{align}


现在假定宇称是守恒的. 回忆, 我们可以选择超对称算符的相位使得宇称算符在这些生成元山的作用由方程(\ref{25.3.13})给定. 那么$\,\mathcal{Q}_{a}^{\ast}\,$作用在$\,\mathsf{P}\lvert j,\sigma\rangle\,$是态$\,\mathsf{P}\mathcal{Q}_{a}\lvert j,\sigma\rangle\,$的线性组合, 它们为零, 又由于$\,\mathsf{P}\lvert j,\sigma\rangle\,$和$\,\lvert j,\sigma\rangle\,$有着相同的选择性质, 它必须正比于它
\begin{equation}
    \mathsf{P}\lvert j,\sigma\rangle =-\eta \lvert j,\sigma \rangle^{\flat} \:. \label{25.5.16}
\end{equation}
由于$\,\mathsf{P}\,$是幺正的, $\eta\,$是满足$\,\lvert\eta\rvert =1\,$的相位因子. 对应的讨论表明$\,\mathsf{P}\lvert j,\sigma\rangle^{\flat}\,$正比于$\,\lvert j,\sigma$. 为了找到比例系数, 我们注意到
\begin{align*}
    \mathsf{P}\lvert j,\sigma\rangle &= (2M)^{-1}\mathsf{P}\,\mathcal{Q}_{\frac{1}{2}}^{\ast}\,\mathcal{Q}_{-\frac{1}{2}}^{\ast}\,
    \lvert j,\sigma \rangle = -\eta (2M)^{-1}\,\mathcal{Q}_{-\frac{1}{2}}\,\mathcal{Q}_{\frac{1}{2}}\,\lvert j,\sigma \rangle^{\flat}\\
    &=-\eta (2M)^{-2}\,\mathcal{Q}_{-\frac{1}{2}}\,\mathcal{Q}_{\frac{1}{2}}\,\mathcal{Q}_{\frac{1}{2}}^{\ast}\,\mathcal{Q}_{-\frac{1}{2}}^{\ast}\,
    \lvert j,\sigma\rangle = -\eta \lvert j,\sigma\rangle\:.
\end{align*}
这样我们就可以定义自旋$\,j\,$的态
\begin{equation}
    \lvert j,\sigma \rangle^{\pm} \equiv \frac{1}{\sqrt{2}}\Bigl(\lvert j,\sigma \rangle \pm
    \mi\,\lvert j,\sigma \rangle^{\flat}\Bigr) \:, \label{25.5.17}
\end{equation}
它们有确定的宇称
\begin{equation}
    \mathsf{P}\lvert j,\sigma \rangle^{\pm} =\pm \mi\eta \,\lvert j,\sigma \rangle^{\pm}\:.\label{25.5.18}
\end{equation}
最后, 用宇称算符作用方程(\ref{25.5.5})并使用方程(\ref{25.3.13})和(\ref{25.5.16}), 给出
\[
    \mathsf{P}\,\lvert j\pm 1/2,\sigma \rangle = -\frac{\eta}{\sqrt{2M}}
    \sum_{a}C_{\frac{1}{2}\,j}\Bigl(j\pm 1/2,\sigma\,; a,\sigma-a \Bigr)\,
    \sum_{b}e_{ab}\mathcal{Q}_{b}\,\lvert j,\sigma-a\rangle^{\flat} \:.
\]
那么方程(\ref{25.5.12})和\,Clebsch-Gordan\,系数的正交性给出
\begin{equation}
    \mathsf{P}\,\lvert j\pm 1/2,\sigma \rangle =\eta \,\lvert j\pm 1/2 ,\sigma \rangle\:, \label{25.5.19}
\end{equation}
这正是所要证明的.

我们现在简单地提一下有$\,N\,$个超对称生成元的扩充超对称性的情况. 正如上一节提及的, 对于任何中心荷, 不可能存在有非零本征值的无质量粒子. 我们可以更进一步并证明中心荷算符的的本征值为任何超多重态的质量提供了一个下界. 由于中心荷$\,Z_{rs}\,$和$\,Z_{rs}^{\ast}\,$彼此对易且与$\,P_{\mu}\,$对易, 单粒子态可以选择成所有中心荷和$\,P_{\mu}\,$的共同本征态, 又因为中心荷与$\,\mathcal{Q}_{ar}\,$和$\,\mathcal{Q}_{ar}^{\ast}\,$对易, 超多重态中的所有态拥有相同的本征值.

为了推导出将超多重态的质量$\,M\,$与中心荷在这个多重态上的本征值关联起来的不等式, 我们使用反对易关系(\ref{25.2.7})和(\ref{25.2.8})写下
\begin{align}
    &\sum_{ar} \Bigl\{\Bigl(\mathcal{Q}_{ar}-\sum_{bs}e_{ab}U_{rs}\mathcal{Q}_{bs}^{\ast}\Bigr)\:,
    \Bigl(\mathcal{Q}_{ar}^{\ast} - \sum_{ct} e_{ac} U_{rt}^{\ast}\mathcal{Q}_{ct}\Bigr) \Bigr\} \nonumber \\
    &\phantom{\sum_{ar} \Bigl\{\Bigl(\mathcal{Q}_{ar}}= 8NP^{0}
    - 2\operatorname{Tr}\Bigl(ZU^{\dag}+UZ^{\dag}\Bigr)\:, \label{25.5.20}
\end{align}
其中$\,U_{rs}\,$是一任意的$\,N\times N\,$幺正矩阵. 左边是正定算符, 通过让它作用在静止的超多重态上, 我们发现
\begin{equation}
    M\geq \frac{1}{4N}\operatorname{Tr}\Bigl(ZU^{\dag}+UZ^{\dag}\Bigr)\:, \label{25.5.21}
\end{equation}
其中$\,Z_{rs}\,$现在是指质量为$\,M\,$的超多重态的中心荷值. 极分解定理告诉我们任何方阵$\,Z\,$可以写成 $H\,V$, 其中$\,H\,$是正定厄米矩阵而$\,V\,$是幺正的. 通过令$\,U=V$, 我们可以获得一个有用的不等式(事实上是最理想的), 在这一情况下, 方程(\ref{25.5.21})变成
\begin{equation}
    M\geq \frac{1}{2N}\operatorname{Tr}H=\frac{1}{2N}\operatorname{Tr}\sqrt{Z^{\dag}Z}\:.\label{25.5.22}
\end{equation}
类比\,23.3\,节中讨论的\,Bogomol'nyi-Prasad-Sommerfeld\,磁单极构形, 在那里质量等于一般单极子质量下界的态被称为\,\emph{BPS}\,态, $M\,$等于这个不等式所允许的最小值的态被称为\,\textit{BPS}\,态. 事实上, 这不只是个类比; 我们将会在\,27.9\,节看到, 在有扩充超对称性的理论中, 单极子质量的下界是下界(\ref{25.5.22})的一个特殊情况.

从方程(\ref{25.5.22})的推导中可以看到, 对于\,BPS\,超多重态, 当算符$\,\mathcal{Q}_{ar}-\sum_{bs}e_{ab}U_{rs}\mathcal{Q}_{bs}^{\ast}\,$作用在这个超多重态中的任何态上时, 它给出零, 所以只有$\,N\,$个独立的螺旋度下降算符$\,\mathcal{Q}_{(1/2)\,r}\,$和$\,N\,$个独立的螺旋度上升%
算符$\,\mathcal{Q}_{(-1/2)\,r}$, 和无质量超多重态的情况相同. 这给出的超多重态要比一般情况下超多重态要小.

例如, 对于$\,N=2\,$超对称性, 中心荷由一个复数给定\footnote{在一些关于$\,N=2\,$超对称性的文章中, 那里的中心荷$\,Z\,$是我们的$\,Z/2\sqrt{2}$.}
\begin{equation}
    Z= \begin{pmatrix}
    0 & Z_{12} \\ -Z_{12} & 0
    \end{pmatrix} \:. \label{25.5.23}
\end{equation}
不等式(\ref{25.5.22})在这里是
\begin{equation}
    M\geq \lvert Z_{12} \rvert /2 \:. \label{25.5.24}
\end{equation}
当$\,M=\lvert Z_{12}\rvert /2\,$时, 有质量粒子超多重态的螺旋度列表与无质量的相同: 有规范超多重态, 它们由一个自旋$\,1\,$的粒子, 一个自旋$\,1/2\,$的$\,SU(2)\,R\,$-对称性双重态以及一个自旋$\,0\,$的粒子构成%
(另一个螺旋度为零的态属于自旋\,1\,粒子), 还有极多重态, 它们由一个自旋$\,1/2\,$的粒子和一个自旋\,0\,的$\,SU(2)$ $R\,$-对称性二重态构成. 为了与在$\,M>\lvert Z_{12}\rvert /2\,$时遇到的较长的超多重态相区分, 它们有时被称为``短''超多重态.


\section*{习题}
\noindent 1. 找到一组$\,2\times2\,$矩阵, 使得它们构成既包含费米生成元又包含玻色生成元的阶化\,Lie\,代数. \\

\noindent 2. 沿用\,Haag, Lopuszanski\,和\,Sohnius\,的方法, 推导出$\,2+1\,$维时空中最一般对称超代数的形式. (提示: 在将\,$2+1\,$维时空中的\,Lorentz\,群生成元标记成$\,A_{1}=-\mi J_{10}$, $A_{2}=-\mi J_{20}\,$和$\,A_{3}=J_{12}\,$后, Poincar\'{e}\, 代数的对易关系是$\,[A_{i},A_{j}]=\mi\sum_{k}\epsilon_{ijk}A_{k}$, 所以$\,2+1\,$维时空中的齐次\,Lorentz\,群的表示只用{\kai{一}}个整数或半整数指标$\,A\,$标记.) 在这里假定\,Coleman-Mandula\,定理成立的条件是满足的. \\

\noindent 3. 假定没有螺旋度大于$\,+3/2\,$或小于$\,-3/2\,$的无质量粒子. 找到$\,N=6\,$扩充超对称性和(使用$\,\mathsf{CPT}$ 对称性)$\,N=5\,$扩充超对称性的最一般的无质量粒子超多重态. 对这两个扩充超对称性, 你发现的这两个超多重态之间的差异说明了什么? \\

\noindent 4. 对于扩充$\,N=2\,$超对称性短超多重态中的粒子, 它们可能的宇称是什么?



%++++++++++++++++++参考文献+++++++++
\renewcommand{\sectionmark}[1]{\markright{ #1}{}}
\renewcommand{\bibname}{参考文献}

\begin{thebibliography}{99}
    \bibitem{1} R. Haag, J. T. Lopuszanski, and M. Sohnius, {\textit{Nucl. Phys.}} {\bf{B88}}, 257 (1975). 这篇文章重印于{\textit{Supersymmetry}}, S. Ferrar编辑(North Holland/World Scientific, Amsterdam/Singapore, 1987).
    \bibitem{2} B. Zumino, {\textit{Nucl. Phys.}} {\bf{B89}}, 535 (1975). 这篇文章重印于{\textit{Supersymmetry}}, 参考文献[1].
    \bibitem{3} M. T. Grisaru and H. N. Pendleton, {\textit{Phys. Lett.}} {\bf{67B}}, 323 (1977).
\end{thebibliography}


%第二十六章 超对称场论


\chapter{超对称场论} \label{cha:26}

现在我们知道了最一般超对称代数的结构, 并且, 我们看到了如何解出这一对称性对粒子谱的意义. 为了知道超对称性对粒子相互作用的含义, 我们需要知道如何构建超对称场论.

最初, 场超多重态的构造是通过重复使用\,Jacobi\,恒等式直接完成的, 很像\,25.4\,节和\,25.5\,节中单粒子态的超多重态构造. 26.1\,节给出了这一方法的一个例子, 在那里用来构造只包含标量场和\,Dirac\,场的超多重态. 幸运的是, 还有一个\,Salam\,和\,Strathdee\,发明的更加简单的技术,\cite{1} 在这种技术里, 场超多重态被并入``超场'', 这种场不仅依赖通常的\,4\,维时空坐标还依赖费米坐标. 我们将在\,26.2\,节引入超场并用它来构建超对称场论, 然后在\,26.3---26.8\,节研究它们的一些结果. 本章将只考虑$\,N=1\,$的超对称, 这是超场形式理论的主要用武之地. 在下一章的末尾, 我们将通过给$\,N=1\,$超场理论附加$\,U(N)\,R\,$-对称性来构造有$\,N\,$-扩充超对称性的理论.


\section{场超多重态的直接构造} \label{sec:26.1}

为了阐明场多重态的直接构造, 对于我们将要考察的场, 它湮灭掉的粒子属于\,25.5\,节讨论过的任意质量的最简超多重态: 两个无自旋的粒子和一个自旋$\,1/2\,$的粒子. 我们在方程(\ref{25.5.15})中看到, 湮灭掉零自旋单粒子态$\,\lvert 0,0\rangle\,$的是$\,\mathcal{Q}_{a}\,$而不是$\,\mathcal{Q}_{a}^{\ast}$, 所以我们预期从真空(它假设成被所有超对称生成元湮灭)中创造这一粒子的标量场%
$\,\phi(x)\,$与$\,\mathcal{Q}_{a}\,$对易而不是$\,\mathcal{Q}_{a}^{\ast}$. 即,
\begin{equation}
    [\mathcal{Q}_{a},\phi(x)] = 0 \:, \label{26.1.1}
\end{equation}
\begin{equation}
    {-}\mi\sum_{b}e_{ab}[\mathcal{Q}_{b}^{\ast},\phi(x)] \equiv \zeta_{a}(x) \neq 0 \:. \label{26.1.2}
\end{equation}
在这里引入$\,2\times2\,$的反对称矩阵$\,e_{ab}\,$(其中$\,e_{1/2,-1/2}\equiv +1$)是因为, 在齐次\,Lorentz\,群下按照$\,(1/2,0)$ 表示变换的是$\,\sum_{b}e_{ab}\mathcal{Q}_{b}^{\ast}$. 由此得出$\,\zeta_{a}(x)\,$这个二分量旋量场也属于齐次\,Lorentz\,群的$\,(1/2,0)\,$表示.\footnote{到目前为止, 我们并没有对这些场湮灭的粒子的质量或相互作用做任何假设, 但能够注意到, 就像\,5.9\,节中解释的那样, 一个$\,(1/2,0)\,$自由场能够产生的无质量粒子只能是螺旋度$\,+1/2\,$的, 这与(\ref{25.5.15})的结果一致, 即被$\,\mathcal{Q}_{a}\,$湮灭掉的无质量零自旋单粒子态$\,\lvert 0,0\rangle\,$伴随着一个螺旋度$\,+1/2\,$的态处在超多重态中.}

从方程(\ref{26.1.1})---(\ref{26.1.2})和反对易关系(\ref{25.2.31})中, 我们发现
\[
\{\mathcal{Q}_{b},\zeta_{a}\} = -\mi \sum_{c} e_{ac}[\{\mathcal{Q}_{b},\mathcal{Q}_{c}^{\ast}\},\phi(x)]
=2\mi(\sigma^{\mu}e)_{ba}\,[P_{\mu},\phi] \:,
\]
因而
\begin{equation}
    \{\mathcal{Q}_{b},\zeta_{a}(x)\} = -2(\sigma^{\mu}e)_{ba}\partial_{\mu}\phi(x) \:. \label{26.1.3}
\end{equation}
另一方面, 方程(\ref{26.1.2})和反对易关系(\ref{25.2.32})给出
\[
{-}\mi\sum_{c}e_{ac}\{\mathcal{Q}_{b}^{\ast},\zeta_{c}\} = \{\mathcal{Q}_{b}^{\ast},[\mathcal{Q}_{a}^{\ast},\phi]\}
=-\{\mathcal{Q}_{a}^{\ast},[\mathcal{Q}_{b}^{\ast},\phi]\} = \mi\sum_{c}e_{bc}\{\mathcal{Q}_{a}^{\ast},\zeta_{c}\} \:,
\]
所以$\,\sum_{c}e_{ac}\{\mathcal{Q}_{b}^{\ast},\zeta_{c}\}\,$是反对称的, 因而正比于$\,2\times2\,$反对称矩阵$\,e_{ab}\,$:
\begin{equation}
    \mi\{\mathcal{Q}_{b}^{\ast},\zeta_{a}(x)\} = 2\delta_{ab}\,\mathscr{F}(x) \:. \label{26.1.4}
\end{equation}
Lorentz\,不变性要求系数$\,\mathscr{F}(x)\,$是个标量场.

我们现在必须更进一步, 计算超对称生成元与$\,\mathscr{F}(x)\,$的对易子. 利用方程(\ref{26.1.4}), (\ref{26.1.2})和\\(\ref{25.2.32}), 我们有
\[
\delta_{ab}\,[\mathcal{Q}_{c}^{\ast},\mathscr{F}]=\tfrac{1}{2}\mi[\mathcal{Q}_{c}^{\ast},\{\mathcal{Q}_{b}^{\ast},
\zeta_{a}\}]
=\tfrac{1}{2}\mi[\{\mathcal{Q}_{c}^{\ast},\zeta_{a}\},\mathcal{Q}_{b}^{\ast}] = -\delta_{ac}[\mathcal{Q}_{b}^{\ast},\mathscr{F}]\:.
\]
取$\,a=b\neq c$, 我们发现这个对易子为零:
\begin{equation}
    [\mathcal{Q}_{c}^{\ast},\mathscr{F}(x)]=0 \:. \label{26.1.5}
\end{equation}
最后, 利用方程(\ref{26.1.4}), (\ref{25.2.31})和(\ref{26.1.3}), 我们有
\begin{align*}
    \delta_{ab}\,[\mathcal{Q}_{c},\mathscr{F}] &= \tfrac{1}{2}\mi [\mathcal{Q}_{c},\{\mathcal{Q}_{b}^{\ast},\zeta_{a}\}]
    =\tfrac{1}{2}\mi [\{\mathcal{Q}_{c},\mathcal{Q}_{b}^{\ast}\},\zeta_{a}]-\tfrac{1}{2}\mi [\mathcal{Q}_{b}^{\ast},\{\mathcal{Q}_{c},\zeta_{a}\}] \\
    &= -\sigma_{cb}^{\mu}\,\partial_{\mu}\zeta_{a} + \mi(\sigma^{\mu}e)_{ca}\,[\mathcal{Q}_{b}^{\ast},\partial_{\mu}\phi] \\
    &= -\sigma_{cb}^{\mu}\,\partial_{\mu}\zeta_{a} + \sum_{d}e_{bd}\,(\sigma^{\mu}e)_{ca}\,\partial_{\mu}\zeta_{d} \:.
\end{align*}
与$\,\delta_{ab}\,$收缩, 这变成
\begin{equation}
    [\mathcal{Q}_{c},\mathscr{F}(x)] = -\sum_{a}\sigma_{ca}^{\mu}\,\partial_{\mu}\zeta_{a}(x) \:. \label{26.1.6}
\end{equation}
方程(\ref{26.1.1})---(\ref{26.1.6})表明场$\,\phi(x)$, $\zeta_{a}(x)\,$和$\,\mathscr{F}(x)\,$构成了超对称代数的一个完整表示. 这些场不是厄米的, 所以它们的复共轭构成了另一个超多重态:
\begin{align}
    &[\mathcal{Q}_{a}^{\ast},\phi^{\ast}(x)]=0 \:, \label{26.1.7} \\
    &{-}\mi\sum_{b}e_{ab}[\mathcal{Q}_{b},\phi^{\ast}(x)]=\zeta_{a}^{\ast}(x) \:, \label{26.1.8} \\
    &\{\mathcal{Q}_{b}^{\ast},\zeta_{a}^{\ast}(x)\} =2(e\sigma^{\mu})_{ab}\partial_{\mu}\phi^{\ast}(x)\:.\label{26.1.9} \\
    &{-}\mi\{\mathcal{Q}_{b},\zeta_{a}^{\ast}(x)\} =2\delta_{ab}\mathscr{F}^{\ast}(x)\:, \label{26.1.10} \\
    &[\mathcal{Q}_{c},\mathscr{F}^{\ast}(x)] = 0 \:, \label{26.1.11} \\
    &[\mathcal{Q}_{c}^{\ast},\mathscr{F}^{\ast}(x)] = \sum_{a}\sigma_{ac}^{\mu}\,\partial_{\mu}\zeta_{a}^{\ast}(x)\:.\label{26.1.12}
\end{align}

我们可以将这些对易关系和反对易关系表示成在一个超对称变换下的变换规则, 这个超对称变换使得任何玻色或费米场算符$\,\mathcal{O}(x)\,$偏移一个无限小量
\begin{equation}
    \delta \mathcal{O}(x)\equiv \left[\sum_{a}\bigl(\epsilon_{a}^{\ast}\mathcal{Q}_{a}+\epsilon_{a}\mathcal{Q}_{a}^{\ast}\bigr), \mathcal{O}(x)\right] \:, \label{26.1.13}
\end{equation}
其中$\,\epsilon_{a}\,$是无限小的费米\,c\,-数旋量. (因为$\,\epsilon_{a}\,$和$\,\epsilon_{a}^{\ast}\,$与$\,\mathcal{Q}_{a}\,$和$\,\mathcal{Q}_{a}^{\ast}\,$反对易, $\,\epsilon_{a}^{\ast}\mathcal{Q}_{a}+\epsilon_{a}\mathcal{Q}_{a}^{\ast}\,$是{\kai{反厄米}}的, 所以方程(\ref{26.1.13})给出$\,(\delta\mathcal{Q})^{\ast}=\delta\mathcal{Q}^{\ast}$.) 对易和反对易规则(\ref{26.1.1})---(\ref{26.1.6})等价于变换规则
\begin{align}
    &\delta\phi(x) = -\mi\sum_{ab}\epsilon_{a}\,e_{ab}\,\zeta_{b}(x) \:, \label{26.1.14} \\
    &\delta\zeta_{a}(x) = -2 \sum_{b}\epsilon_{b}^{\ast}\,(\sigma^{\mu}e)_{ba}\,\partial_{\mu}\phi(x)
    -2\mi\epsilon_{a}\mathscr{F}(x) \:,  \label{26.1.15} \\
    &\delta \mathscr{F}(x) = - \sum_{ab} \epsilon_{b}^{\ast}\,\sigma_{ba}^{\mu}\,\partial_{\mu}\zeta_{a}(x) \:. \label{26.1.16}
\end{align}

通过引入一个无限小的\,Majorana\footnote{在我们将要使用的相位约定下, Majorana\,4\,-分量旋量由\,2\,-分量$\,(1/2,0)\,$旋量$\,u_{a}\,$以
\[
\begin{pmatrix}
u \\ -eu^{\ast}
\end{pmatrix}
\]
的方式构成. 方程(\ref{26.1.17})符合这一定义, 这里$\,u=-\mi\epsilon$. 等价地, Majorana\,旋量可以由\,2\,-分量$\,(0,1/2)\,$旋量$\,v_{a}\,$以
\[
\begin{pmatrix}
ev^{\ast} \\ v
\end{pmatrix}
\]
的方式构成. 方程(\ref{25.2.34})提供了另外一个例子. 本章附录会细致地考察\,Majorana\,旋量的性质.
}\,4-分量旋量变换参量
\begin{equation}
    \alpha \equiv -\mi
    \begin{pmatrix}
    \epsilon_{a} \\ \sum_{b}e_{ab}\epsilon_{b}^{\ast}
    \end{pmatrix}  \:, \label{26.1.17}
\end{equation}
这可以写成\,Dirac\,4\,-分量的形式, 使得方程(\ref{26.1.13})变成
\begin{equation}
    \mi\,\delta\mathcal{O}(x) \equiv [\bar{\alpha}Q, \mathcal{O}(x)] \:. \label{26.1.18}
\end{equation}

通过引入一组实玻色场$\,A$, $B$, $F\,$和$\,G$, 以及一个\,4\,-分量\,Majorana\,旋量$\,\psi$, 变换规则(\ref{26.1.14})---(\ref{26.1.16})和它们的复共轭可以写成一个方便的协变形式, 这些实玻色场的定义是
\begin{equation}
    \frac{A+\mi B}{\sqrt{2}} \equiv \phi \:, \qquad \quad
    \frac{F-\mi G}{\sqrt{2}} \equiv \mathscr{F} \:, \label{26.1.19}
\end{equation}
$\psi\,$的定义是
\begin{equation}
    \psi\equiv \frac{1}{\sqrt{2}}
    \begin{pmatrix}
    \zeta_{a} \\ -\sum_{b} e_{ab}\zeta_{b}^{\ast}
    \end{pmatrix} \:. \label{26.1.20}
\end{equation}
我们同时回忆起$\,4\times4\,$Dirac\,矩阵和$\,2\times2\,$矩阵$\,\sigma_{\mu}\,$的关系是:
\[
\gamma_{\mu} = \begin{pmatrix}
0 & -\mi\,e\sigma_{\mu}^{\mathrm{T}}e \\
\mi\,\sigma_{\mu} & 0
\end{pmatrix} \:.
\]
变换规则现在采取如下的形式
\begin{align}
    &\delta A =\bar{\alpha}\,\psi \:, \qquad \delta B =-\mi\,\bar{\alpha}\,\gamma_{5}\,\psi \:, \nonumber \\
    &\delta \psi =\partial_{\mu}(A+\mi\gamma_{5}B)\gamma^{\mu}\alpha + (F-\mi\gamma_{5}G)\alpha \:, \label{26.1.21} \\
    &\delta F =\bar{\alpha}\,\gamma^{\mu}\,\partial_{\mu}\psi \:, \qquad
    \delta G= -\mi\bar{\alpha}\,\gamma_{5}\gamma^{\mu}\,\partial_{\mu}\psi \:. \nonumber
\end{align}
一个繁琐但直接的计算表明, 这个变换保持作用量
\begin{align}
    I&= \int \dif^{4}x \: \Bigl\{ -\tfrac{1}{2}\partial_{\mu}A\,\partial^{\mu}A - \tfrac{1}{2}\partial_{\mu}B\,\partial^{\mu}B -\tfrac{1}{2}\bar{\psi}\gamma^{\mu}\partial_{\mu}\psi \nonumber \\
    &\quad \tfrac{1}{2}(F^{2}+G^{2}) + m\,[FA+GB-\tfrac{1}{2}\bar{\psi}\psi]\nonumber \\
    &\quad g\Bigl[F(A^{2}+B^{2})+2GAB -\bar{\psi}(A+\mi\gamma_{5}B)\psi\Bigr]\Bigr\}  \label{26.1.22}
\end{align}
不变. 方程(\ref{26.1.21})和(\ref{26.1.22})与变换规则(\ref{24.2.8})以及\,Wess\,和\,Zumino\,的原始工作中发现的拉格朗%
日密度(\ref{24.2.9})一致. 在接下来的三节, 我们将会探索检验方程(\ref{26.1.22})的超对称性和导出更一般的超对称理论的一个方便技巧.

当费米场$\,\psi(x)\,$满足自由场\,Dirac\,方程$\,(\gamma^{\mu}\partial_{\mu}+m)\psi=0\,$时, 这些变换规则表明$\,F+mA\,$和$\,G+mB\,$是不变的, 因此与$\,\mathcal{Q}_{a}\,$和$\,\mathcal{Q}_{a}^{\ast}\,$对易, 随之也与$\,P_{\mu}\,$对易. 这并不能证明$\,F=-mA\,$和$\,G=-mB$, 但在不改变对易和反对易规则(\ref{26.1.1})---(\ref{26.1.6})或变换规则(\ref{26.1.21})的前提下, 我们可以通过分别减除掉常数$\,F+mA\,$和$\,B+mG\,$重新定义场$\,F\,$和$\,G$, 使得新的场$\,F\,$和$\,G\,$由$\,F=-mA\,$和$\,G=-mB\,$给定, 因而就有$\,\mathscr{F}=-m\phi^{\ast}$. 在有相互作用时, 这是不成立的, 但即使是在有相互作用的情况下, $\mathscr{F}(x)$, $F(x)\,$和$\,G(x)\,$一般是辅助场, 就像作用量(\ref{26.1.22})的情况, 它们可以被超多重态的其它场表示.

\section{一般超场} \label{sec:26.2}

通过上一节阐述的直接技巧, 构造场超多重态是直接的, 但是为了构造超对称作用量, 我们还需要知道如何将场超多重态乘起来给出其它的超多重态. 使用\,Salam\,和\,Strathdee\,发明的一套形式理论可以省下大量的功夫, 在那个形式理论中, 任何超多重态中的场被整合进单个超场中.

就像\,4\,-动量算符\,$P_{\mu}\,$定义成普通时空坐标$\,x^{\mu}\,$的平移生成元, 4\,个超对称生成元$\,\mathcal{Q}_{a}\,$和$\,\mathcal{Q}_{a}^{\ast}\,$也可以视作\,4\,个费米\,c\,-数超空间坐标的%
平移生成元, 这些坐标彼此反对易且与费米场反对易, 但与$\,x^{\mu}$ 和所有玻色场对易. 我们的目的是构建\,Lorentz\,不变的拉格朗日密度, 所以采取\,25.2\,描述的\,4\,-分量 Dirac\,形式体系将是方便的. 超对称生成元被合并进一个\,4\,-分量\,Majorana\,旋量$\,Q_{\alpha}$, 相应地, 超空间坐标被合并进另一个\,4\,-分量\,Majorana\,旋量$\,\theta_{\alpha}$. (本章附录会概述\,Majorana\,旋量的各种性质.) 超对称生成元有不为零的反对易子, 所以我们不能简单地将它们取成正比于超坐标平移算符$\,\partial/\partial\theta_{\alpha}$. 相反, Salam\,和\,Strathdee\,发现, 如果我们设超对称生成元$\,Q\,$与任何玻色或费米超场$\,S(x,\theta)\,$的对易子或反对易子是
\begin{equation}
[Q,S\} = \mi\mathscr{Q}S\:, \label{26.2.1}
\end{equation}
其中$\,\mathscr{Q}\,$是超空间微分算符
\begin{equation}
\mathscr{Q}\equiv -\frac{\partial}{\partial \bar{\theta}}
+ \gamma^{\mu}\,\theta\,\frac{\partial}{\partial x^{\mu}} \:, \label{26.2.2}
\end{equation}
那么超空间代数就是被满足的. (像往常一样, $\bar{\theta}\equiv \theta^{\dag}\beta$. 所有对费米\,c\,-数变量的导数都应被理解成{\kai{左}}导数, 计算时在对它微分前要将这个变量移至任何表达式的左边.) 对于\,Majorana\,旋量$\,\bar{\theta}=\theta^{\mathrm{T}}\gamma_{5}\epsilon$, 其中$\,4\times4\,$矩阵$\,\epsilon\,$由方程(\ref{26.A.3})给定, 所以方程(\ref{26.2.1})可以写成更加明显的
\begin{equation}
\mathscr{Q}_{\alpha}= \sum_{\gamma}(\gamma_{5}\epsilon)_{\alpha\gamma}\,\frac{\partial}{\partial \theta_{\gamma}}
+\sum_{\gamma} \gamma_{\alpha\gamma}^{\mu}\theta_{\gamma}\,\frac{\partial}{\partial x^{\mu}}\:. \label{26.2.3}
\end{equation}
同理,
\begin{equation}
\overline{\mathscr{Q}}_{\beta} = \sum_{\gamma}\mathscr{Q}_{\gamma}\, (\gamma_{5}\,\epsilon)_{\gamma\beta}
= \frac{\partial}{\partial \theta_{\beta}}
- \sum_{\gamma}(\gamma_{5}\,\epsilon\,\gamma^{\mu})_{\beta\gamma}\,\theta_{\gamma}\,\frac{\partial}{\partial x^{\mu}} \:. \label{26.2.4}
\end{equation}
直接计算可以给出
\begin{equation}
\Bigl\{\mathscr{Q}_{\alpha},\overline{\mathscr{Q}}_{\beta}\Bigr\}
= (\gamma_{5}\,\epsilon\,\gamma^{\mu}\,\gamma_{5}\,\epsilon)_{\beta\alpha}\,\frac{\partial}{\partial x^{\mu}}
+\gamma_{\alpha\beta}^{\mu}\,\frac{\partial}{\partial x^{\mu}} \:. \label{26.2.5}
\end{equation}
但是方程(\textcolor{foo}{5.4.35})表明$\,\gamma_{\mu}^{\mathrm{T}}=-\mathscr{C}\,\gamma_{\mu}\,\mathscr{C}^{-1}$, 其中$\,\mathscr{C}\,$是矩阵$\,\mathscr{C}=-\gamma_{5}\epsilon$, 所以方程(\ref{26.2.5})右边的两项是相等的, 因此
\begin{equation}
\Bigl\{\mathscr{Q}_{\alpha},\overline{\mathscr{Q}}_{\beta}\Bigr\}
=2\gamma_{\alpha\beta}^{\mu}\,\frac{\partial}{\partial x^{\mu}} \:. \label{26.2.6}
\end{equation}
方程(\ref{26.2.6})和(\ref{26.2.1})再加上广义\,Jacobi\,恒等式(\ref{25.1.5})表明
\begin{equation}
[\{Q_{\alpha},\overline{Q}_{\beta}\},S] = \{\mathscr{Q}_{\alpha},\overline{\mathscr{Q}}_{\beta}\}S
=2\gamma_{\alpha\beta}^{\mu}\partial_{\mu}S = -2\mi\gamma_{\alpha\beta}^{\mu}[P_{\mu},S]\:, \label{26.2.7}
\end{equation}
与反对易关系(\ref{25.2.36})一致.

将对易和反对易关系(\ref{26.2.1})表示成在无限小超对称变换下的变换规则通常会更加方便. 结合方程(\ref{26.1.18}), (\ref{26.2.1}) 和(\ref{26.2.2})表明无限小\,Majorana\,旋量参量为$\,\alpha\,$的超对称变换对超场$\,S(x,\theta)$ 的变换是
\begin{equation}
\delta S = (\bar{\alpha}\,\mathscr{Q})\,S = -\left(\bar{\alpha}\,\frac{\partial S}{\partial \bar{\theta}}\right)
+(\bar{\alpha}\,\gamma^{\mu}\,\theta)\,\frac{\partial S}{\partial x^{\mu}} \:. \label{26.2.8}
\end{equation}
注意这里的$\,\partial/\partial\bar{\theta}\,$作用在任何表达式的左边. 特别的, 当$\,M\,$是矩阵$\,1$, $\gamma_{5}\gamma_{\mu}\,$和$\,\gamma_{5}\,$的线性组合且使得$\,\bar{\theta}M\theta\,$不为零, 我们有$\bar{\theta}^{\prime}M\theta^{\prime\prime}=\bar{\theta}^{\prime\prime}M\theta^{\prime}$, 所以
\begin{equation}
\frac{\partial }{\partial \theta} (\bar{\theta}M\theta) =2M\theta \:. \label{26.2.9}
\end{equation}

$\theta\,$的分量反对易, 所以对于它们的分量的任意乘积, 如果其中有两个相等, 这个乘积为零. 但是$\,\theta\,$只有\,4\,个分量, 所以对于$\,\theta\,$的任何函数, 它的幂级数展开至于四次项. 更进一步, 本章附录将会证明, 两个$\,\theta\,$的乘积正比于$\,(\bar{\theta}\theta)$, $(\bar{\theta}\gamma_{\mu}\gamma_{5}\theta)\,$和$\,(\bar{\theta}\gamma_{5}\theta)\,$的线性组合; 三个$\,\theta\,$的乘积正比于$\,(\bar{\theta}\gamma_{5}\theta)\theta$; 四个$\,\theta\,$的乘积正比于$\,(\bar{\theta}\gamma_{5}\theta)^{2}$. 因此$\,x^{\mu}\,$和$\,\theta\,$的最一般函数可以表示成
\begin{align}
S(x,\theta) &= C(x) -\mi\Bigl(\bar{\theta}\,\gamma_{5}\,\omega(x)\Bigr) - \frac{\mi}{2}\Bigl(\bar{\theta}\,\gamma_{5}\,\theta\Bigr)M(x)
- \frac{1}{2}\Bigl(\bar{\theta}\,\theta\Bigr)N(x) \nonumber \\
&\quad + \frac{\mi}{2}\Bigl(\bar{\theta}\,\gamma_{5}\,\gamma_{\mu}\,\theta\Bigr) V^{\mu}(x)
-\mi \Bigl(\bar{\theta}\,\gamma_{5}\,\theta\Bigr)\biggl(\bar{\theta}
\biggl[\lambda(x)+\frac{1}{2}\slashed{\partial}\omega(x)\biggr]\biggr) \nonumber \\
&\quad -\frac{1}{4} \Bigl(\bar{\theta}\,\gamma_{5}\,\theta\Bigr)^{2}
\biggl(D(x)+\frac{1}{2}\square C(x)\biggr) \:. \label{26.2.10}
\end{align}
(为了后面的方便, 我们分别从$\,\lambda(x)\,$和$\,D(x)\,$中分别分离出了$\,\tfrac{1}{2}\slashed{\partial}\omega\,$%
和$\,\tfrac{1}{2}\square C(x)$.) 如果$\,S(x,\theta)\,$是标量, 那么$\,C(x)$, $M(x)$, $N(x)\,$和$\,D(x)\,$是标量(或赝标量)场; $\omega(x)\,$和$\,\lambda(x)\,$是\,4\,-分量旋量场; $V^{\mu}(x)\,$是矢量场. 另外, 通过使用本章附录给出的\,Majorana\,场双线性积的实性质, 我们可以看到, 如果 $S(x,\theta)\,$是实的, 那么$\,C(x)$, $M(x)$, $N(x)$, $V^{\mu}(x)\,$和$\,D(x)\,$都是实的, 而$\,\omega(x)\,$和$\,\lambda(x)\,$是满足相位约定$\,s^{\ast}=-\beta\epsilon\gamma_{5}s\,$的\,Majorana\,旋量.

现在我们必须解出方程(\ref{26.2.10})中的分量场的超对称变换性质. 对展开(\ref{26.2.10})应用(\ref{26.2.8}) 和(\ref{26.2.9}), 这给出
\begin{align*}
\delta S &= (\bar{\alpha}\gamma^{\mu}\theta)\,\frac{\partial C}{\partial x^{\mu}}  \\
&\quad +\mi\,(\bar{\alpha}\gamma_{5}\omega) - \mi\,(\bar{\alpha}\gamma^{\mu}\theta)\,\biggl(\bar{\theta}\gamma_{5}\frac{\partial \omega}{\partial x^{\mu}}\biggr)\\
&\quad +\mi\,(\bar{\alpha}\gamma_{5}\theta)\,M - \frac{\mi}{2}\,(\bar{\alpha}\gamma^{\mu}\theta)
\Bigl(\bar{\theta}\gamma_{5}\theta\Bigr) \frac{\partial M}{\partial x^{\mu}} \\
&\quad +(\bar{\alpha}\theta)\,N - \frac{1}{2}\,(\bar{\alpha}\gamma^{\mu}\theta)\,\Bigl(\bar{\theta}\theta\Bigr)
\frac{\partial N}{\partial x^{\mu}} \\
&\quad -\mi\,(\bar{\alpha}\gamma_{5}\gamma_{\nu}\theta)\,V^{\nu} + \frac{\mi}{2}\,
(\bar{\alpha}\gamma^{\mu}\theta)\,\Bigl(\bar{\theta}\gamma_{5}\gamma_{\nu}\theta\Bigr)
\,\frac{\partial V^{\nu}}{\partial x^{\mu}} \\
&\quad +2\mi\,(\bar{\alpha}\gamma_{5}\theta)\,\Bigl(\bar{\theta}[\lambda
+\tfrac{1}{2}\,\slashed{\partial}\omega]\Bigr) + \mi\,\Bigl(\bar{\theta}\gamma_{5}\theta\Bigr)\,
\bigl(\bar{\alpha}[\lambda+\tfrac{1}{2}\,\slashed{\partial}\omega]\bigr)  \\
&\quad -\mi\,(\bar{\alpha}\gamma^{\mu}\theta)\,\Bigl(\bar{\theta}\gamma_{5}\theta\Bigr)\,
\Bigl(\bar{\theta}\partial_{\mu}[\lambda+\tfrac{1}{2}\,\slashed{\partial}\omega]\Bigr)
+\Bigl(\bar{\theta}\gamma_{5}\theta\Bigr)\,(\bar{\alpha}\gamma_{5}\theta)\,[D+\tfrac{1}{2}\square C]\:.
\end{align*}
我们需要将每一项变成方程(\ref{26.2.10})的标准形式. 对于这个目的, 我们注意到: 方程(\ref{26.A.9})给出
\[
(\bar{\alpha}\gamma^{\mu}\theta)(\bar{\theta}\gamma_{5}\partial_{\mu}\omega)
=-\tfrac{1}{4}(\bar{\theta}\theta)(\bar{\alpha}\,\slashed{\partial}\gamma_{5}\omega)
-\tfrac{1}{4}(\bar{\theta}\gamma_{5}\gamma^{\nu}\theta)(\bar{\alpha}\,\slashed{\partial}\gamma_{\nu}\omega)
-\tfrac{1}{4}(\bar{\theta}\gamma_{5}\theta)(\bar{\alpha}\,\slashed{\partial}\omega)\:;
\]
方程(\ref{26.A.16})给出
\[
(\bar{\alpha}\gamma^{\mu}\theta)(\bar{\theta}\theta)=
-(\bar{\alpha}\gamma^{\mu}\gamma_{5}\theta)(\bar{\theta}\gamma_{5}\theta) \:;
\]
方程(\ref{26.A.17})给出
\[
(\bar{\alpha}\gamma^{\mu}\theta)(\bar{\theta}\gamma_{5}\gamma_{\nu}\theta)
=-(\bar{\alpha}\gamma^{\mu}\gamma_{\nu}\theta)(\bar{\theta}\gamma_{5}\theta) \:;
\]
方程(\ref{26.A.9})给出
\begin{align*}
(\bar{\alpha}\gamma_{5}\theta)(\bar{\theta}[\lambda+\tfrac{1}{2}\slashed{\partial}\omega])
&= -\tfrac{1}{4} (\bar{\theta}\theta)(\bar{\alpha}\gamma_{5}[\lambda+\tfrac{1}{2}\slashed{\partial}\omega])
+\tfrac{1}{4} (\bar{\theta}\gamma_{5}\gamma^{\mu}\theta)(\bar{\alpha}\gamma_{\mu}
[\lambda+\tfrac{1}{2}\slashed{\partial}\omega]) \\
&\quad -\tfrac{1}{4}(\bar{\theta}\gamma_{5}\theta)(\bar{\alpha}[\lambda+\tfrac{1}{2}\slashed{\partial}\omega]) \:;
\end{align*}
方程(\ref{26.A.19})给出
\[
(\bar{\alpha}\gamma^{\mu}\theta)(\bar{\theta}\gamma_{5}\theta)
(\bar{\theta}\partial_{\mu}[\lambda+\tfrac{1}{2}\slashed{\partial}\omega])
=-\tfrac{1}{4}(\bar{\alpha}\slashed{\partial}\gamma_{5}[\lambda+\tfrac{1}{2}\slashed{\partial}\omega])
(\bar{\theta}\gamma_{5}\theta)^{2} \:.
\]
利用这些关系并按照$\,\theta\,$因子数目递增的顺序重排这些项, 我们有
\begin{align*}
\delta S &= \mi\,(\bar{\alpha}\gamma_{5}\omega)+
(\bar{\alpha}[\slashed{\partial}C+\mi\gamma_{5}M+N-\mi\gamma_{5}\slashed{V}]\theta) \\
&\quad -\tfrac{1}{2}\mi\,\Bigl(\bar{\theta}\theta\Bigr)\,(\bar{\alpha}\gamma_{5}[\lambda+\slashed{\partial}\omega])
+\tfrac{1}{2}\mi\,\Bigl(\bar{\theta}\gamma_{5}\theta\Bigr)\,(\bar{\alpha}[\lambda+\slashed{\partial}\omega]) \\
&\quad +\tfrac{1}{2}\mi\,\Bigl(\bar{\theta}\gamma_{5}\gamma^{\mu}\theta\Bigr)\,(\bar{\alpha}\gamma_{\mu}\lambda)
+\tfrac{1}{2}\mi\,\Bigl(\bar{\theta}\gamma_{5}\gamma^{\nu}\theta\Bigr)\,(\bar{\alpha}\partial_{\nu}\omega) \\
&\quad +\tfrac{1}{2}\,\Bigl(\bar{\theta}\gamma_{5}\theta\Bigr)\,
\bigl(\bar{\alpha}[-\mi\slashed{\partial}M-\gamma_{5}\slashed{\partial}N-\mi\slashed{\partial}\slashed{V}
+\gamma_{5}(D+\tfrac{1}{2}\square C)]\theta\bigr) \\
&\quad -\tfrac{1}{4}\mi\,\Bigl(\bar{\theta}\gamma_{5}\theta\Bigr)^{2}
\bigl(\bar{\alpha}\gamma_{5}[\slashed{\partial}\lambda+\tfrac{1}{2}\square\omega]\bigr) \:,
\end{align*}
或者, 利用对称性(\ref{26.A.7})
\begin{align*}
\delta S &= \mi\,(\bar{\alpha}\gamma_{5}\omega)+
\Bigl(\bar{\theta}[-\slashed{\partial}C+\mi\gamma_{5}M+N-\mi\gamma_{5}\slashed{V}]\alpha\Bigr)  \\
&\quad -\tfrac{1}{2}\mi\,\Bigl(\bar{\theta}\theta\Bigr)\,(\bar{\alpha}\gamma_{5}[\lambda+\slashed{\partial}\omega])
+\tfrac{1}{2}\mi\,\Bigl(\bar{\theta}\gamma_{5}\theta\Bigr)\,(\bar{\alpha}[\lambda+\slashed{\partial}\omega]) \\
&\quad +\tfrac{1}{2}\mi\,\Bigl(\bar{\theta}\gamma_{5}\gamma^{\mu}\theta\Bigr)\,(\bar{\alpha}\gamma_{\mu}\lambda)
+\tfrac{1}{2}\mi\,\Bigl(\bar{\theta}\gamma_{5}\gamma^{\nu}\theta\Bigr)\,(\bar{\alpha}\partial_{\nu}\omega) \\
&\quad +\tfrac{1}{2}\,\Bigl(\bar{\theta}\gamma_{5}\theta\Bigr)\,
\Bigl(\bar{\theta}[\mi\slashed{\partial}M-\gamma_{5}\slashed{\partial}N-\mi\partial_{\mu}\slashed{V}\gamma^{\mu}
+\gamma_{5}(D+\tfrac{1}{2}\square C)]\alpha\Bigr) \\
&\quad -\tfrac{1}{4}\mi\,\Bigl(\bar{\theta}\gamma_{5}\theta\Bigr)^{2}\,
\bigl(\bar{\alpha}\gamma_{5}[\slashed{\partial}\lambda+\tfrac{1}{2}\square\omega]\bigr) \:.
\end{align*}
如果我们比较上式与展开(\ref{26.2.10})中$\,\theta\,$的零阶, 一阶和二阶项, 我们发现变换规则:
\begin{align}
\delta C &= \mi\,\Bigl(\bar{\alpha}\gamma_{5}\omega\Bigr) \:, \label{26.2.11} \\
\delta \omega &= \Bigl(-\mi\gamma_{5}\slashed{\partial}C-M+\mi\gamma_{5}N+\slashed{V}\Bigr)\,\alpha\;, \label{26.2.12} \\
\delta M &= -\Bigl(\bar{\alpha}[\lambda + \slashed{\partial}\omega]\Bigr) \:,\label{26.2.13} \\
\delta N &= \mi\,\Bigl(\bar{\alpha}\gamma_{5}[\lambda+\slashed{\partial}\omega]\Bigr) \:, \label{26.2.14} \\
\delta V_{\mu} &= \Bigl(\bar{\alpha}\gamma_{\mu}\lambda\Bigr) + (\bar{\alpha}\partial_{\mu} \omega) \:. \label{26.2.15}
\end{align}
$\theta\,$的三阶项和四阶项给出
\begin{align*}
\delta[\lambda+\tfrac{1}{2}\slashed{\partial}\omega] &=
\tfrac{1}{2}\Bigl[-\slashed{\partial}M-\mi\gamma_{5}\slashed{\partial}N+\partial_{\mu}\slashed{V}\gamma^{\mu}
+\mi\gamma_{5}\Bigl(D+\tfrac{1}{2}\square C\Bigr)\Bigr]\alpha \\
\delta[D+\tfrac{1}{2}\square C] &= \mi\Bigl(\bar{\alpha}\gamma_{5}[\slashed{\partial}\lambda
+\tfrac{1}{2}\square \omega]\Bigr) \:.
\end{align*}
结合后两个变换规则与$\,C\,$和$\,\omega\,$的变换规则(\ref{26.2.11})和(\ref{26.2.12}), 这给出$\,\lambda\,$和$\,D\,$更加简单的变换规则:
\begin{align}
\delta \lambda &= \biggl(\tfrac{1}{2}
\Bigl[\partial_{\mu}\slashed{V},\gamma^{\mu}\Bigr]+\mi\gamma_{5}D\biggr)\alpha\:, \label{26.2.16} \\
\delta D &= \mi\Bigl(\bar{\alpha}\,\gamma_{5}\,\slashed{\partial}\lambda\Bigr) \:. \label{26.2.17}
\end{align}
在展开(\ref{26.2.10})中从$\,\lambda\,$和$\,D\,$中分离出$\,\frac{1}{2}\slashed{\partial}\omega\,$和%
$\,\tfrac{1}{2}\square C\,$正是为了实现这个简化.


超场形式体系的关键就是简化了用超多重态制造其它超多重态这个任务. 给定两个都满足变换规则(\ref{26.2.8})的超场$\,S_{1}\,$和$\,S_{2}$, 它们的乘积$\,S\equiv S_{1}S_{2}\,$满足
\begin{align}
\delta S &\equiv [(\bar{\alpha}Q),S_{1}S_{2}] = (\delta S_{1})S_{2} + S_{1}(\delta S_{2}) \nonumber\\
&=\Bigl((\bar{\alpha}\mathscr{Q})S_{1}\Bigr)S_{2} + S_{1}\Bigl((\bar{\alpha}\mathscr{Q})\Bigr) S_{2}
=(\bar{\alpha}\mathscr{Q})S \:, \label{26.2.18}
\end{align}
因此也是个超场. 利用方程(\ref{26.A.7}), (\ref{26.A.16}), (\ref{26.A.18})和(\ref{26.A.19})直接进行计算就给出了它的分量场
\begin{align}
&C=C_{1}C_{2}  \:, \label{26.2.19} \\
&\omega = C_{1}\omega_{2} + C_{2}\omega_{1} \:, \label{26.2.20}  \\
&M = C_{1}M_{2} + C_{2}M_{1} + \tfrac{1}{2}\mi\Bigl(\overline{\omega_{1}}\,\gamma_{5}\,\omega_{2}\Bigr)\:,\label{26.2.21} \\
&N= C_{1}N_{2}+ C_{2}N_{1} - \tfrac{1}{2}\Bigl(\overline{\omega_{1}}\,\omega_{2}\Bigr) \:,\label{26.2.22} \\
&V^{\mu} = C_{1}V_{2}^{\mu}+ C_{2}V_{1}^{\mu} - \tfrac{1}{2}\mi\Bigl(\overline{\omega_{1}}\,\gamma_{5}\gamma^{\mu}\,\omega_{2}\Bigr) \:, \label{26.2.23} \\
&\lambda = C_{1}\lambda_{2}+C_{2}\lambda_{1} - \tfrac{1}{2}\gamma^{\mu}\omega_{1}\partial_{\mu}C_{2}
-\tfrac{1}{2}\gamma^{\mu}\omega_{2}\partial_{\mu}C_{1} +\tfrac{1}{2}\mi\,\slashed{V}_{1}\gamma_{5}\,\omega_{2}
+ \tfrac{1}{2}\mi\,\slashed{V}_{2}\gamma_{5}\,\omega_{1} \nonumber \\
&\phantom{\lambda=} + \tfrac{1}{2}(N_{1}-\mi\gamma_{5}M_{1})\omega_{2}
+\tfrac{1}{2}(N_{2}-\mi\gamma_{5}M_{2})\omega_{1} \:, \label{26.2.24} \\
&D= -\partial_{\mu}C_{1}\,\partial^{\mu}C_{2} + C_{1}D_{2} + C_{2}D_{1} + M_{1}M_{2} +N_{1}N_{2} \nonumber\\
&\phantom{D=}-\Bigl(\overline{\omega_{1}}[\lambda_{2}+\tfrac{1}{2}\slashed{\partial}\omega_{2}]\Bigr)
-\Bigl(\overline{\omega_{2}}[\lambda_{1}+\tfrac{1}{2}\slashed{\partial}\omega_{1}]\Bigr)
-V_{1\mu}V_{2}^{\mu} \:. \label{26.2.25}
\end{align}
超场的线性组合平庸地是超场, 在同等意义下, 超场的时空导数和复共轭也是超场. 但是给超场乘以$\,\theta\,$的某个函数或者做相对于$\,\theta\,$的微分一般不会给出超场. (例如, $\theta\,$自身显然不是超场, 这是因为$\,\theta\,$是费米\,c\,-数因而与$\,\bar{\alpha}Q\,$对易, 而$\,\mathscr{Q}\theta\neq0$.) 然而, 有一种方式可以组合超场对$\,\theta\,$的导数和它与因子$\,\theta\,$的乘积, 使得确实产生另一个超场.

考察超空间微分算符$\,\mathscr{D}_{\alpha}$, 它定义成
\begin{equation}
\mathscr{D}\equiv -\frac{\partial}{\partial \bar{\theta}}
- \gamma^{\mu}\,\theta\,\frac{\partial}{\partial x^{\mu}} \:, \label{26.2.26}
\end{equation}
或者更加清楚些
\begin{equation}
\mathscr{D}_{\alpha} = \sum_{\gamma} (\gamma_{5}\epsilon)_{\alpha\gamma}\,\frac{\partial}{\partial \theta_{\gamma}}
-\sum_{\gamma} \gamma_{\alpha\gamma}^{\mu}\theta_{\gamma}\,\frac{\partial}{\partial x^{\mu}}\:. \label{26.2.27}
\end{equation}
$\mathscr{D}\,$和$\,\mathscr{Q}\,$的定义的唯一差异是包含时空导数那一项的符号变了. 这个符号变化的结果是, 在$\,\mathscr{D}_{\beta}\,$和 $\mathscr{Q}_{\alpha}\,$的反对易子中, 取代像方程(\ref{26.2.6})中那样获得相同的两项, 我们现在得到的项有相反的符号, 这使得它们抵消:
\begin{equation}
\{\mathscr{D}_{\beta},\mathscr{Q}_{\alpha}\} = 0 \:. \label{26.2.28}
\end{equation}
由于$\,\alpha\,$是费米的, 可以得出$\,(\alpha\mathscr{Q})\,$与$\,\mathscr{D}_{\beta}\,$对易, 所以, 如果$\,S(x,\theta)\,$是超场, 那么
\begin{equation}
\delta \mathscr{D}_{\beta}S \equiv -\mi[\bar{\alpha}Q,\mathscr{D}_{\beta}S]
=-\mi\mathscr{D}_{\beta}[(\bar{\alpha}Q),S] =\mathscr{D}_{\beta}(\bar{\alpha}\mathscr{Q})S
=(\bar{\alpha}\mathscr{Q})\mathscr{D}_{\beta}S \:, \label{26.2.29}
\end{equation}
这使得$\,\mathscr{D}_{\beta}S\,$也是个超场. {\kai{因此, 超场$\,S\,$和它们的超导数$\,\mathscr{D}_{\beta}S$, $\mathscr{D}_{\beta}\mathscr{D}_{\gamma}S\,$等的任意多项式函数也是超场.}}

虽然不是必须的, 但在这里提一下, 在用超场构建超场是可以纳入它们的时空导数. 这是因为这些时空导数可以通过二阶超导数获得. 由于$\,\mathscr{D}_{\beta}\,$和$\,\mathscr{Q}_{\beta}\,$的唯一差异是包含$\,\partial_{\mu}\,$那一项的符号, 除了一个符号变化外, $\mathscr{D}\,$的反对易子与$\,\mathscr{Q}\,$的反对易子相同:
\begin{equation}
\Bigl\{\mathscr{D}_{\alpha},\overline{\mathscr{D}}_{\beta}\Bigr\}
=-2\gamma_{\alpha\beta}^{\mu}\,\frac{\partial}{\partial x^{\mu}} \:. \label{26.2.30}
\end{equation}

现在, 我们来考察如何用超场构建超对称作用量. 首先不存在超对称拉格朗日密度这种东西, 这是因为, 反对易关系(\ref{26.2.6})表明, 如果$\,\delta\mathscr{L}=0$, 那么$\,\mathscr{L}\,$必须是个常数. 即使拉格朗日密度不是超对称的, 如果$\,\delta\mathscr{L}(x)\,$是导数项, 它就不会贡献$\,\delta\int\mathscr{L}\,\dif^{4}x$, 那么作用量将仍然是超对称的. 一般而言, 拉格朗日密度可以写成一些项的和, 其中每一项是用基本超场和它们的超导数构建的超场的某个分量. 观察各个分量的变换规则(\ref{26.2.11})---(\ref{26.2.17})表明, 如果在一般超场上没有特殊条件, 这种超场在变分下是个导数的唯一分量是$\,D\,$-分量. 另外, 为了是任何超场的$\,D\,$-分量是个标量, 超场本身也必须是个标量. 因此, 除非在构建拉格朗日密度的各个超场上有特殊条件, 否则超对称作用量只能是对标量超场$\,\Lambda\,$的\,$D$\,-项的积分:
\begin{equation}
I =\int \dif^{4}x \: [\Lambda]_{D} \:.\label{26.2.31}
\end{equation}
然而, 事实上, 如果在构建作用量的超场上没有特殊条件, 那么这类作用量没有一个在物理上是令人满意的. 对于一般超场$\,S(x,\theta)$, 唯一一种既是$\,S\,$和$\,S^{\ast}\,$的双线性, 对分量场的导数又不超过二阶的超对称动能作用量$\,I_{0}\,$是如下的形式
\begin{equation}
I_{0} \propto \int \dif^{4} x\: \Bigl[S^{\ast}\,S\Bigr]_{D} \:. \label{26.2.32}
\end{equation}
我们从方程(\ref{26.2.25})中看到$\,S^{\ast}\,S\,$有$\,D\,$-分量
\begin{align}
\Bigl[S^{\ast}\,S\Bigr]_{D} &= -\partial_{\mu}C^{\ast}\,\partial^{\mu}C
-\tfrac{1}{2}\Bigl(\bar{\omega}\,\gamma^{\mu}\partial_{\mu}\omega\Bigr)
+\tfrac{1}{2}\Bigl((\partial_{\mu}\bar{\omega})\gamma^{\mu}\omega\Bigr) \nonumber \\
&\quad +C^{\ast}D+D^{\ast}C - \Bigl(\bar{\omega}\,\lambda\Bigr) + \Bigl(\bar{\lambda}\,\omega\Bigr) \nonumber \\
&\quad +M^{\ast}M + N^{\ast}N - V_{\mu}^{\ast}V^{\mu} \:. \label{26.2.33}
\end{align}
$C\,$和$\,\omega\,$的二次项和期望中的一样看起来是自旋零和$\,1/2\,$的无质量场的动能拉格朗日量; 最后三项是无害的; 但是包含$\,D\,$和$\,\lambda\,$的项在路径积分中有灾难性的结果, 它们会约束$\,C\,$和$\,\omega\,$为零. 幸运的是, 就像我们将在下一节看到的, 存在一些受约束的超场, 使得我们{\kai{可以}}用它们构造出物理上合理的作用量. 引入这些受约束的超场同时也打开在作用量中构建超对称项的方法, 它们可以不是超场函数的$\,D\,$-分量.

如果宇称是守恒的, 那么超场的分量场的空间反演性质将通过超对称关联起来. 为了解出这个关系, 我们对对易(反对易)关系(\ref{26.2.1})作用宇称算符$\,\mathsf{P}\,$并使用超对称生成元的变换性质(\ref{25.3.16}), 这给出
\begin{equation}
\mi\beta \Bigl [ Q, \mathsf{P}^{-1}S(x,\theta)\mathsf{P}\Bigr\}
=\mathscr{Q}\,\mathsf{P}^{-1}S(x,\theta)\mathsf{P} \:. \label{26.2.34}
\end{equation}
对于标量超场, 方程(\ref{26.2.34})的解是如下形式
\begin{equation}
\mathsf{P}^{-1}S(x,\theta)\mathsf{P} = \eta\, S(\Lambda_{P\,}x,-\mi\beta\theta) \:, \label{26.2.35}
\end{equation}
其中$\,\eta\,$是某个相位(超场的内禀宇称), $\Lambda_{P\,}x\equiv(-\mathbf{x},+x^{0})$. (为了验证方程(\ref{26.2.35})满足(\ref{26.2.34}), 注意到方程(\ref{26.2.35})给出的方程(\ref{26.2.34})的左边是
\[
\mi\eta\beta\,\biggl(-\frac{\partial}{\partial\overline{(-\mi\beta\theta)}}
+\gamma^{\mu}(-\mi\beta\theta)\frac{\partial}{\partial(\Lambda_{P\,}x)^{\mu}} \biggr)\,S(\Lambda_{P\,}x,\theta)
=\eta\,\mathscr{Q}\,S(\Lambda_{P}\,x,-\mi\beta\theta) \:,
\]
与方程(\ref{26.2.35})给出的方程(\ref{26.2.34})的右边一致.) 在方程(\ref{26.2.35})中使用展开(\ref{26.2.10})就给出了分量场的空间反演性质:
\begin{align}
& \mathsf{P}^{-1} C(x)\mathsf{P} = \eta\,C(\Lambda_{P\,}x) \:, \nonumber  \\
& \mathsf{P}^{-1} \omega(x)\mathsf{P} = -\mi\eta\,\beta\,\omega(\Lambda_{P\,}x) \:,  \nonumber  \\
& \mathsf{P}^{-1} M(x)\mathsf{P} = -\eta\,M(\Lambda_{P\,}x) \:, \nonumber  \\
& \mathsf{P}^{-1} N(x)\mathsf{P} = \eta\,N (\Lambda_{P\,}x) \:, \label{26.2.36} \\
& \mathsf{P}^{-1} V^{\mu}(x)\mathsf{P} =
 -\eta\,(\Lambda_{P})\indices{^\mu_\nu}V^{\nu}(\Lambda_{P\,}x) \:, \nonumber  \\
& \mathsf{P}^{-1} \lambda(x)\mathsf{P} = \mi\eta\,\beta\,\lambda(\Lambda_{P\,}x) \:, \nonumber  \\
& \mathsf{P}^{-1} D(x)\mathsf{P} = \eta\, D(\Lambda_{P\,}x) \:. \nonumber
\end{align}

\subsection*{* * *}


一般实超场$\,S\,$包含四个无自旋实场$\,C$, $M$, $N\,$和$\,D$, 加上一个实\,4\,-矢场$\,V_{\mu}$, 总共有八个独立的玻色场分量. 相较而言, 存在两个\,4\,-分量\,Majorana\,旋量场$\,\omega\,$和$\,\lambda$, 独立场分量的总个数也是八个. 一般而言, 独立玻色场分量的个数和独立费米场分量的个数相等不仅对于本节研究的不约束一般超场是成立的, 对于通过对一般超场附加超对称约束获得的所有超场也是成立的, 下一节讨论的手征超场和其它受约束超场就是这样的例子.

为了在一般情况下看到这点, 假定我们有由$\,N_{B}\,$个线性独立实玻色场算符$\,b_{n}(x)\,$和$\,N_{F}\,$个线性独立费米场算符$\,f_{k}(x)\,$%
提供的超对称代数的表示. 我们将假定这些场仅满足不平庸的场方程, 这使得$\,b_{n}\,$或$\,f_{k}\,$的系数非零的线性组合不可能满足齐次线性场方程. 考察一个实超对称生成元$\,Q(u)$, 定义为
\begin{equation}
Q(u) \equiv \Bigl(\bar{u}\,Q\Bigr) = \Bigl(\bar{Q}\,u\Bigr) \:, \label{26.2.37}
\end{equation}
其中$\,u\,$是某个普通的数值\,Majorana\,旋量({\kai{不是}}反对易\,c\,-数). (对于扩充超对称, 取代$\,Q_{\alpha}$, 我们将使用$\,Q_{r\alpha}\,$中的任何一个, 例如$\,Q_{1\,\alpha}$.) 为了使$\,b_{n}\,$和$\,f_{k}\,$构成这个超对称代数的一个表示, 我们必须有
\begin{align}
[Q(u),b_{n}] &= \mi\sum_{k}q_{nk}(\partial)\,f_{k} \:, \label{26.2.38} \\
\{Q(u),f_{k}\} &= \sum_{n} p_{kn}(\partial)\,b_{n} \:, \label{26.2.39}
\end{align}
其中$\,q(\partial)\,$和$\,p(\partial)\,$是一些矩阵微分算符. 取方程(\ref{26.2.38})和$\,Q(u)\,$的对易子以及方程(\ref{26.2.39})和 $Q(u)$ 的反对易子, 这给出
\begin{align}
[Q^{2}(u),b_{n}] &= \mi\sum_{m}\Bigl(q(\partial)\,p(\partial)\Bigr)_{nm}\,b_{m} \:, \label{26.2.40} \\
[Q^{2}(u),f_{k}] &= \mi\sum_{\ell} \Bigl(p(\partial)\,q(\partial)\Bigr)_{k\ell}\,f_{\ell} \:. \label{26.2.41}
\end{align}
反对易关系(\ref{25.2.36})或(\ref{25.2.38})给出$\,Q(u)\,$的平方是$\,Q^{2}(u)=-\mi P_{\mu}\Bigl(\bar{u}\gamma^{\mu}u\Bigr)$. 这样方阵$\,p(\partial)q(\partial)\,$和 $q(\partial)p(\partial)\,$必须非奇异, 原因是, 如果存在非零系数$\,c_{n}(\partial)\,$或$\,d_{k}(\partial)\,$使得%
$\,\sum_{n}c_{n}(\partial)(q(\partial)p(\partial))_{nm}=0$ 或者%
$\,\sum_{k}d_{k}(\partial)(p(\partial)q(\partial))_{k\ell}=0$, 那么$\,b_{n}\,$或$\,f_{k}\,$将满足齐次线性场方程
\[
\Bigl(\bar{u}\gamma^{\mu}u\Bigr)\partial_{\mu}\sum_{n}c_{n}(\partial)b_{n} = 0 \qquad \text{或} \qquad
\Bigl(\bar{u}\gamma^{\mu}u\Bigr)\partial_{\mu}\sum_{k}d_{k}(\partial)f_{k} = 0 \:,
\]
而我们前面假定了这些场不满足这样的场方程, 二者矛盾. 为了使$\,qp\,$不奇异, 我们必须有$\,N_{F}\geq N_{B}$, 而为了使$\,pq\,$不奇异, 我们必须有$\,N_{B}\geq N_{F}$, 所以我们可以得出$\,N_{B}=N_{F}$. 另外, 方程$\,q\,$和$\,p\,$必须都是不奇异的, 所以方程(\ref{26.2.38})的复共轭告诉我们$\,f^{\ast}=q^{\ast-1}qf$, 这使得独立费米场的个数是$\,N_{F}\,$而不是$\,2N_{F}$, 因而等于独立玻色场的个数$\,N_{B}$, 这正是所要证明的.


\section{手征超场和线性超场} \label{sec:26.3}

我们在上一节发现, 一般超场中出现了$\,D\,$和$\,\lambda\,$这一点阻止了我们在满足物理要求的拉格朗日密度中使用这种超场. 那么假定我们考察这样的超场, 它满足
\begin{equation}
\lambda = D= 0 \:. \label{26.3.1}
\end{equation}
这些条件是否受到超对称变换的保护? 根据方程(\ref{26.2.17})和(\ref{26.2.16}), 如果$\,\lambda=0$, 那么$\,D=0\,$的条件不变的, 但是要想$\,\lambda=0\,$的条件在超对称变换下不变则需要我们附加$\,\partial_{\mu}V_{\nu}-\partial_{\nu}V_{\mu}=0\,$的条件, 这要求$\,V_{\mu}\,$是纯规范:
\begin{equation}
V_{\mu}(x) = \partial_{\mu}Z(x) \:. \label{26.3.2}
\end{equation}
方程(\ref{26.2.15})表明, 有了$\,\lambda=0$, 这个条件是被超对称变换保护的. 因此我们得到了一个退化的超场, 它满足约束(\ref{26.3.1})和(\ref{26.3.2}), 它的分量场有如下的变换性质
\begin{align}
\delta C &= \mi\,\Bigl(\bar{\alpha}\,\gamma_{5}\,\omega\Bigr) \:, \label{26.3.3} \\
\delta \omega &= \Bigl(-\mi\gamma_{5}\,\slashed{\partial}C- M + \mi\gamma_{5}N+\slashed{\partial}Z\Bigr)\,\alpha\:,
\label{26.3.4}  \\
\delta M &= -\Bigl(\bar{\alpha}\,\slashed{\partial}\omega\Bigr) \:, \label{26.3.5} \\
\delta N &= \mi\,\Bigl(\bar{\alpha}\gamma_{5}\,\slashed{\partial}\omega\Bigr) \:, \label{26.3.6} \\
\delta Z &= \Bigl(\bar{\alpha}\,\omega\Bigr) \:, \label{26.3.7}
\end{align}
与方程(\ref{26.1.21})比较, 我们看到这与\,26.1\,节通过直接方法构建的超多重态是相同的, 它们之间的对应是
\begin{equation}
C=A\:,\qquad \omega=-\mi\gamma_{5}\psi \:, \qquad M=G \:, \qquad N=-F\:, \qquad Z=B \:. \label{26.3.8}
\end{equation}
满足条件(\ref{26.3.1})和(\ref{26.3.2})的超场被称作是{\kai{手征}}的.\footnote{一些学者习惯用``手征''这个词来%
描述这种超场的一个特殊情况, 在这里则被称作左手征或右手征, 我们会在后面进行介绍. 我们在这里使用``手征''这个词第一眼看上去好像有些奇怪, 因为这里并没有\,Dirac\,旋量的对应物. 任何\,Dirac\,旋量是左手征\,Dirac\,旋量和右手征\,Dirac\,旋量的和, 也就是说它们分别正比于$\,1+\gamma_{5}\,$和$\,1-\gamma_{5}$, 所以对于\,Dirac\,旋量的这种和, 我们不需要特殊的项. 相反, 只有满足方程(\ref{26.3.1})和(\ref{26.3.2})的超场才能表示为左手征超场和右手征超场的和.}

为了将手征超场$\,X(x,\theta)\,$与上一节的一般超场$\,S(x,\theta)\,$区分开来, 取代$\,C$, $M$, $N$, $Z\,$和$\,\omega$, 我们用$\,A$, $B$, $F$, $G\,$和$\,\psi\,$表示它的分量. 通过在方程(\ref{26.2.10})中使用方程(\ref{26.3.1}), (\ref{26.3.2})和(\ref{26.3.8}), 我们发现一般手征超场的形式是
\begin{align}
    X(x,\theta) &= A(x) - \Bigl(\bar{\theta}\,\psi(x)\Bigr) + \frac{1}{2}\Bigl(\bar{\theta}\,\theta\Bigr)F(x)
    -\frac{\mi}{2}\Bigl(\bar{\theta}\,\gamma_{5}\,\theta\Bigr)G(x) \nonumber \\
    &\quad +\frac{\mi}{2}\Bigl(\bar{\theta}\,\gamma_{5}\,\gamma_{\mu}\,\theta\Bigr)\partial^{\mu}B(x)
    +\frac{1}{2}\Bigl(\bar{\theta}\,\gamma_{5}\,\theta\Bigr)
    \Bigl(\bar{\theta}\gamma_{5}\,\slashed{\partial}\psi(x)\Bigr) \nonumber \\
    &\quad -\frac{1}{8}\Bigl(\bar{\theta}\,\gamma_{5}\,\theta\Bigr)^{2}\square A(x)\:. \label{26.3.9}
\end{align}
(我们本也可以取$\,C=-B$, $\omega=\psi$, $M=-F$, $N=-G\,$和$\,Z=A$. 我们做(\ref{26.3.8})这种对应是因为, 正如我们这里所看到的, 对于标量超场, 它们与$\,A\,$和$\,F\,$是标量而$\,B\,$和$\,G\,$是赝标量这个常见的约定是一致的.)

手征超场(\ref{26.3.9})可以进一步分解成
\begin{equation}
    X(x,\theta) = \frac{1}{\sqrt{2}}\Bigl[\Phi(x,\theta)+\tilde{\Phi}(x,\theta)\Bigr]\:, \label{26.3.10}
\end{equation}
其中
\begin{align}
    \Phi(x,\theta) &=\phi(x) - \sqrt{2}\Bigl(\bar{\theta}\psi_{L}(x)\Bigr)
    +\mathscr{F}(x)\biggl(\bar{\theta}\biggl(\frac{1+\gamma_{5}}{2}\biggr)\theta\biggr)
    +\frac{1}{2}\Bigl(\bar{\theta}\gamma_{5}\gamma_{\mu}\theta\Bigr)\partial^{\mu}\phi(x) \nonumber \\
    &\quad -\frac{1}{\sqrt{2}}\Bigl(\bar{\theta}\gamma_{5}\theta\Bigr)
    \Bigl(\bar{\theta}\,\slashed{\partial}\psi_{L}(x)\Bigr)
    -\frac{1}{8}\Bigl(\bar{\theta}\gamma_{5}\theta\Bigr)^{2}\square\phi(x) \:, \label{26.3.11} \\
    \tilde{\Phi}(x,\theta) &=\tilde{\phi}(x) - \sqrt{2}\Bigl(\bar{\theta}\psi_{R}(x)\Bigr)
    +\tilde{\mathscr{F}}(x)\biggl(\bar{\theta}\biggl(\frac{1-\gamma_{5}}{2}\biggr)\theta\biggr)
    -\frac{1}{2}\Bigl(\bar{\theta}\gamma_{5}\gamma_{\mu}\theta\Bigr)\partial^{\mu}\tilde{\phi}(x) \nonumber \\
    &\quad +\frac{1}{\sqrt{2}}\Bigl(\bar{\theta}\gamma_{5}\theta\Bigr)
    \Bigl(\bar{\theta}\,\slashed{\partial}\psi_{R}(x)\Bigr)
    -\frac{1}{8}\Bigl(\bar{\theta}\gamma_{5}\theta\Bigr)^{2}\square\tilde{\phi}(x) \:, \label{26.3.12}
\end{align}
它们的分量场定义成
\begin{equation}
    \phi \equiv \frac{A+\mi B}{\sqrt{2}} \:, \qquad
    \psi_{L} \equiv \biggl(\frac{1+\gamma_{5}}{2}\biggr) \psi \:, \qquad
    \mathscr{F} \equiv \frac{F-\mi G}{\sqrt{2}} \:, \label{26.3.13}
\end{equation}
\begin{equation}
    \tilde{\phi} \equiv \frac{A-\mi B}{\sqrt{2}} \:, \qquad
    \psi_{R} \equiv \biggl(\frac{1-\gamma_{5}}{2}\biggr) \psi \:, \qquad
    \tilde{\mathscr{F}} \equiv \frac{F+\mi G}{\sqrt{2}} \:, \label{26.3.14}
\end{equation}
无论是$\,\Phi\,$还在$\,\tilde{\Phi}$, 它们的分量场都构成了超对称代数的完整表示:
\begin{align}
    & \delta \psi_{L} = \sqrt{2}\partial_{\mu}\phi\,\gamma^{\mu}\,\alpha_{R}
    +\sqrt{2}\mathscr{F}\,\alpha_{L}\:, \label{26.3.15}  \\
    & \delta \mathscr{F} = \sqrt{2}\Bigl(\overline{\alpha_{L}}\,\slashed{\partial}\psi_{L}\Bigr)\:,\label{26.3.16} \\
    & \delta \phi = \sqrt{2}\Bigl(\overline{\alpha_{R}}\psi_{L}\Bigr)\:, \label{26.3.17}
\end{align}
\begin{align}
    &\delta \psi_{R} =\sqrt{2}\partial_{\mu}\tilde{\phi}\gamma^{\mu}\,\alpha_{L}
    +\sqrt{2}\tilde{\mathscr{F}}\,\alpha_{R} \:, \label{26.3.18} \\
    & \delta \tilde{\mathscr{F}} = \sqrt{2}\Bigl(\overline{\alpha_{R}}\slashed{\partial}\psi_{R}\Bigr)\:,\label{26.3.19}\\
    & \delta \tilde{\phi} =\sqrt{2} \Bigl(\overline{\alpha_{L}}\psi_{R}\Bigr) \:, \label{26.3.20}
\end{align}
其中, 像往常一样,
\[
\alpha_{L}=\biggl(\frac{1+\gamma_{5}}{2}\biggr)\alpha \:, \qquad \quad
\alpha_{R}=\biggl(\frac{1-\gamma_{5}}{2}\biggr)\alpha \:,
\]
对$\,\theta\,$类似. 形如(\ref{26.3.11})和(\ref{26.3.12})的超场分别被称为是{\kai{左手征}}和{\kai{右手征}}的. 在手征超场$\,X(x,\theta)$ 是{\kai{实}}场的特殊情况下, 它的左手征和右手征部分$\,\Phi\,$和$\,\tilde{\Phi}$互为复共轭, 这使得$\,\tilde{\phi}=\phi^{\ast}$, $\,\tilde{\mathscr{F}}=\mathscr{F}^{\ast}$, 以及$\,\psi\,$是\,Majorana\,场. 然而, 如果我们不要求$\,X(x,\theta)\,$是实的, 那么$\,\Phi\,$和$\,\tilde{\Phi}\,$之间一般没有关系; 它们中的一个甚至有可能为零.

超场$\,\Phi\,$的分量中含有两个复的玻色分量$\,\phi\,$和$\,\mathscr{F}$, 或者说\,4\,个独立的实玻色分量, 以及一个 Majorana 费米场$\,\psi$, 它有\,4\,个独立的费米分量. 这是上一节末尾推导的一般结果的又一例子, 即任何构成超对称代数表示的一组场必有相同数目的独立玻色分量和独立费米分量.

我们可以用方程(\ref{26.A.5}), (\ref{26.A.17})和(\ref{26.A.18})重写方程(\ref{26.3.11})和(\ref{26.3.12})以阐明这些超场对%
$\,\theta_{L}\,$和$\,\theta_{R}\,$的依赖方式:
\begin{align}
    \Phi(x,\theta) &= \phi(x_{+}) - \sqrt{2}\Bigl(\theta_{L}^{\mathrm{T}}\,\epsilon\,\psi_{L}(x_{+})\Bigr)
    + \mathscr{F}(x_{+})\Bigl(\theta_{L}^{\mathrm{T}}\,\epsilon\,\theta_{L}\Bigr) \:, \label{26.3.21} \\
    \tilde{\Phi}(x,\theta) &= \tilde{\phi}(x_{-}) + \sqrt{2}\Bigl(\theta_{R}^{\mathrm{T}}\,\epsilon\,\psi_{R}(x_{-})\Bigr)
    - \tilde{\mathscr{F}}(x_{-})\Bigl(\theta_{R}^{\mathrm{T}}\,\epsilon\,\theta_{R}\Bigr) \:, \label{26.3.22}
\end{align}
其中
\begin{equation}
    x_{\pm}^{\mu} \equiv x^{\mu} \pm \tfrac{1}{2}\Bigl(\bar{\theta}\gamma_{5}\gamma^{\mu}\theta\Bigr)
    =x^{\mu} \pm \Bigl(\theta_{R}^{\mathrm{T}}\epsilon\gamma^{\mu}\theta_{L}\Bigr)\:. \label{26.3.23}
\end{equation}
$\phi(x_{+})\,$和$\,\tilde{\phi}(x_{-})\,$对$\,x^{\mu}-x_{\pm}^{\pm}\,$的幂级数展开至于四次项, $\psi_{L,R}(x_{\pm})\,$的展开则止于线性项, 而$\,\mathscr{F}(x_{+})\,$和\\$\,\tilde{\mathscr{F}}(x_{-})\,$的展开至于零阶项, 这是因为所有高阶项在方程(\ref{26.3.21})和(\ref{26.3.22})中的贡献都会包含三个或%
多个$\,\theta_{L}\,$或$\,\theta_{R}\,$因子, 因此为零. 由于相同的原因, 很容易看到, 任何只依赖于$\,\theta_{L}\,$和$\,x_{+}^{\mu}\,$但不额外依赖于$\,\theta_{R}\,$的超场必须取(\ref{26.3.21})的形式, 任何只依赖于$\,\theta_{R}\,$和$\,x_{-}^{\mu}\,$但不额外依赖于$\,\theta_{L}\,$的超场必须取(\ref{26.3.22})的形式.

我们已经看到, 一个超场是左手征还是右手征完全由这个超场允许依赖的量决定. 由此理解得出, {\kai{左手征超场(右手征超场)的任何函数, 而不是它们的复共轭或时空导数, 将是左(右)手征超场.}} 这也可以通过一个更加形式的方法证明. 因为$\,\Phi(x,\theta)\,$仅是通过$\,x_{+}\,$依赖于$\,\theta_{R}$, 而$\,\tilde{\Phi}(x,\theta)\,$仅是通过$\,x_{-}\,$依赖于$\,\theta_{L}$, 它们满足条件
\begin{equation}
    \mathscr{D}_{R\alpha}\Phi = \mathscr{D}_{L\alpha}\tilde{\Phi} = 0 \:, \label{26.3.24}
\end{equation}
其中$\,\mathscr{D}_{R}\,$和$\,\mathscr{D}_{L}\,$分别是超导数(\ref{26.2.26})的左手征部分和右手征部分:
\begin{align}
    \mathscr{D}_{R\alpha}& \equiv \biggl[\biggl(\frac{1-\gamma_{5}}{2}\biggr)\mathscr{D}\biggr]_{\alpha}
    = -\sum_{\beta}\epsilon_{\alpha\beta}\frac{\partial}{\partial \theta_{R\beta}}
    -(\gamma^{\mu}\theta_{L})_{\alpha} \frac{\partial}{\partial x^{\mu}} \:, \label{26.3.25} \\
     \mathscr{D}_{L\alpha}& \equiv \biggl[\biggl(\frac{1+\gamma_{5}}{2}\biggr)\mathscr{D}\biggr]_{\alpha}
    = +\sum_{\beta}\epsilon_{\alpha\beta}\frac{\partial}{\partial \theta_{L\beta}}
    -(\gamma^{\mu}\theta_{R})_{\alpha} \frac{\partial}{\partial x^{\mu}} \:, \label{26.3.26}
\end{align}
它们满足
\[
\mathscr{D}_{R\alpha}x_{+}^{\mu} = \mathscr{D}_{L\alpha}x_{-}^{\mu}=0 \:.
\]
反之, 如果超场$\,\Phi\,$满足$\,\mathscr{D}_{R}\Phi=0$, 那么它是左手征的, 如果它满足$\,\mathscr{D}_{L}\Phi=0$, 那么它是右手征的. 对于一组超场$\,\Phi_{n}$, 如果它们都满足$\,\mathscr{D}_{R}\Phi_{n}=0\,$或$\,\mathscr{D}_{L}\Phi_{n}=0$, 它们的任意函数$\,f(\Phi)\,$也将满足$\,\mathscr{D}_{R}f(\Phi)=0\,$或$\,\mathscr{D}_{L}f(\Phi)=0$, 因此也分别是左手征和右手征的. 但是, 左手征超场{\kai{和}}右手征超场的函数在一般情况下根本不是手征的.

左手征超场的表示(\ref{26.3.21})使得解出它们的乘积性质变得很容易. 例如, 如果$\,\Phi_{1}\,$和$\,\Phi_{2}\,$是两个左手征超场, 那么它们的乘积$\,\Phi=\Phi_{1}\Phi_{2}\,$是左手征超场, 分量是
\begin{align}
    \phi &= \phi_{1}\phi_{2} \:, \label{26.3.27} \\
    \psi_{L} &= \phi_{1}\psi_{2L} + \phi_{2}\psi_{1L} \:, \label{26.3.28} \\
    \mathscr{F} &= \phi_{1}\mathscr{F}_{2}+\phi_{2}\mathscr{F}_{1} - \Bigl(\psi_{1L}^{\mathrm{T}}\,\epsilon\,\psi_{2L}\Bigr) \:. \label{26.3.29}
\end{align}

理论中出现了手征超场打开了构造超对称作用量的另一种可能性. 对变换规则(\ref{26.3.16})的观察表明, 左手征超场$\,\Phi\,$的$\,\mathscr{F}\,$-项在超对称变换下的变化是一个导数项, 这使得对任何左手征超场的$\,\mathscr{F}\,$-项的积分是超对称的. 因此我们可以将超对称作用量构建为
\begin{equation}
    I = \int \dif^{4}x\: \Bigl[f \Bigr]_{\mathscr{F}} + \int \dif^{4}x\: \Bigl[f\Bigr]^{\ast}_{\mathscr{F}}
    +\frac{1}{2} \int \dif^{4}x\: \Bigl[K\Bigr]_{D} \:, \label{26.3.30}
\end{equation}
其中$\,f\,$和$\,K\,$分别是左手征超场和一般实超场且是用基本超场构建的.

$f\,$和$\,K\,$能够依赖于什么? 如果函数$\,f\,$只依赖于左手征基本超场$\,\Phi_{n}\,$而不依赖与它们的右手征复共轭, 那么$\,f\,$将是左手征的. 另一方面, 手征超场的超导数不是手征的, 所以我们不能随意地在$\,f\,$中引入$\,\Phi_{n}\,$的超导数. 正确的是, 对于不是左手征的超场$\,S\,$(例如包含左手征超场的复共轭的超场), 一对右超导数作用在上面会给出左手征超场, 原因是独立的右超导数只有两个且互相反对易:
\[
\mathscr{D}_{R\alpha}(\mathscr{D}_{R\beta}\mathscr{D}_{R\gamma}S) = 0 \: .
\]
然而, 对于任何以这种方式构建的函数$\,f$, 它的$\,\mathscr{F}\,$-项对作用量的贡献与其它某个复合超场的$\,D\,$项对作用量的贡献相同. 由于$\,\mathscr{D}\,$反对易, 通过用两个$\,\mathscr{D}_{R}\,$作用在一般超场$\,S\,$上的得到最一般左手征超场可以表示成%
$\,(\mathscr{D}_{R}^{\mathrm{T}}\epsilon\mathscr{D}_{R})S$. 如果超势中的一个左手征超场是这种形式, 由于每个$\,\mathscr{D}_{R}\,$湮灭超势中所有其它超场, 我们可以将整个超势写成$\,f=(\mathscr{D}_{R}^{\mathrm{T}}\epsilon\mathscr{D}_{R})h$, 其中$\,h\,$是另外一个超场. 现在
\[
\Bigl(\mathscr{D}_{R}^{\mathrm{T}}\,\epsilon\,\mathscr{D}_{R}\Bigr)\,
\Bigl(\theta_{R}^{\mathrm{T}}\,\epsilon\,\theta_{R}\Bigr) =-4 \:,
\]
所以, 除了对作用量没有贡献的时空导数外, $(\mathscr{D}_{R}^{\mathrm{T}}\epsilon\mathscr{D}_{R})h\,$是%
$\,-(\theta_{R}^{\mathrm{T}}\epsilon\theta_{R})/4\,$在$\,h\,$中的系数. 但是, 再一次地, 除时空导数外, $[f]_{\mathscr{F}}\,$ 是$\,(\theta_{L}^{\mathrm{T}}\epsilon\theta_{L})\,$在$\,f\,$中的系数, 所以$\,[(\mathscr{D}_{R}^{\mathrm{T}}\epsilon\mathscr{D}_{R})h]_{\mathscr{F}}\,$等于%
$\,-(\theta_{L}^{\mathrm{T}}\epsilon\theta_{L})(\theta_{R}^{\mathrm{T}}\epsilon%
\theta_{R})/4$\\$=-(\bar{\theta}\gamma_{5}\theta)^{2}/4\,$在$\,h\,$中的系数, 因此
\begin{equation}
    \int \dif^{4}x\: [(\mathscr{D}_{R}^{\mathrm{T}}\,\epsilon\,\mathscr{D}_{R})h]_{\mathscr{F}}
    =2\int \dif^{4}x\: [h]_{D} \:. \label{26.3.31}
\end{equation}
因此, 对于那些依赖于形式为$\,\mathscr{D}_{R\beta}\mathscr{D}_{R\gamma}S\,$的左手征超场的项, 我们不需要把这些项的贡献计入$\,f\,$------任何这样的项将会被纳入所有可能的$\,D\,$-项中. 当$\,f\,$被表示成仅是基本手征超场而非它们的超导数或时空导数的函数时, 它被称为{\kai{超势}}.

与之相反, 实标量函数$\,K\,$一般既是左手征手征超场$\,\Phi_{n}\,$和它们的复共轭$\,\Phi_{n}^{\ast}\,$的函数, 也是它们的超导数和时空导数的函数, 它被称为\,\emph{K\"{a}hler}\,{\kai{势}}. (任何右手征超场都是某个左手征超场的复共轭, 所以这里假定$\,K\,$只依赖于左手征超场和它们的复共轭是不失一般性的.) 然而, 以这种方式获得的$\,K\,$并不都给出不同的作用量. 例如, 手征超场没有$\,D\,$-项, 所以如果两个$\,K\,$只相差一个手征超场, 那么它们对作用量的贡献是相同的.

通过在超空间中部分积分, 我们也可以在不改变作用量的情况下改变$\,K\,$的形式. 对于任意超场的超导数$\,\mathscr{D}_{\alpha}S$, 因为
\begin{equation}
    \int \dif^{4}x\: [\mathscr{D}_{\alpha}S]_{D} = 0 \:, \label{26.3.32}
\end{equation}
所以它的$\,D\,$-项对作用量无贡献. 为了看到这点, 回忆起
\[
\mathscr{D}_{\alpha}S = \sum_{\beta}\Bigl(\gamma_{5}\epsilon\Bigr)_{\alpha\beta}
 \frac{\partial S}{\partial \theta_{\beta}} - (\gamma^{\mu}\theta)_{\alpha}\frac{\partial S}{\partial x^{\mu}}\:.
\]
由于$\,S\,$最多是$\,\theta\,$的四次多项式, $\mathscr{D}_{\alpha}S\,$中的第一项最多是$\,\theta\,$的三次多项式, 因此它的$\,D\,$-项只要非零就必是一个导数, 而第二项也是一个时空导数, 所以它的$\,D\,$-项也是时空导数, 因此$\,\mathscr{D}_{\alpha}S\,$中的第一项和第二项对方程(\ref{26.3.32})中的积分都没有贡献. 另外, 超导数的作用满足分配率, 所以从方程(\ref{26.3.32})可以得出, 我们可以在超空间做分部积分: 对于任何两个玻色超场$\,S_{1}\,$和$\,S_{2}$,
\begin{equation}
    \int \dif^{4}x\: [S_{1}\mathscr{D}_{\alpha}S_{2}]_{D}
    =-\int \dif^{4}x\: [S_{2}\mathscr{D}_{\alpha}S_{1}]_{D} \:. \label{26.3.33}
\end{equation}
在\,26.4\,节和\,26.8\,节, 我们将会细致考察$\,f\,$和$\,K\,$只依赖基本超场但不依赖它们的超导数或普通导数的情况.

我们在上一节看到, 在宇称守恒的理论中, 空间反演算符在一般标量超场上的效应是对它的变换做变换%
$\,x^{\mu}\to(\Lambda_{P})\indices{^\mu_\nu}x^{\nu}\,$和$\,\theta\to-\mi\beta\theta$, 然后再乘上可能的相位$\,\eta$. 在这些变换下, 方程(\ref{26.3.21}) 和(\ref{26.3.22})中的变量$\,x^{\mu}_{\pm}\,$的变化是
\begin{equation}
    x_{\pm}^{\mu}\to (\Lambda_{P\,}x)^{\mu} \pm \tfrac{1}{2}\Bigl(\bar{\theta}\beta\gamma_{5}\gamma^{\mu}\beta\theta\Bigr)=
    (\Lambda_{P\,}x_{\mp})^{\mu} \:, \label{26.3.34}
\end{equation}
以及$\,\theta_{L}\to -\mi\beta\theta_{R}\,$和$\,\theta_{R}\to-\mi\beta\theta_{L}$. 因此空间反演将左手征超场变到右手征超场, 并将右手征超场变到左手征超场. 分量场中包含相同粒子的产生湮灭算符且这些粒子也被左手征超场$\,\Phi\,$产生湮灭的唯一右手征超场是%
$\,\tilde{\Phi}\propto\Phi^{\ast}$, 所以$\,\mathsf{P}^{-1}\Phi\mathsf{P}\,$必须正比于$\,\Phi^{\ast}$. 通过对$\,\Phi\,$的相位做合适的选择, 我么可以重新整理这一变换规则使其变成
\begin{equation}
    \mathsf{P}^{-1}\Phi(x,\theta)\mathsf{P} = \Phi^{\ast}(\Lambda_{P\,}x,-\mi\beta\theta)\:. \label{26.3.35}
\end{equation}
以分量场的形式, 这个变换是
\begin{align}
    \mathsf{P}^{-1}\phi(x)\mathsf{P} &= \phi^{\ast}(\Lambda_{P\,}x) \:, \nonumber  \\
    \mathsf{P}^{-1}\psi_{L}(x)\mathsf{P}
    &= -\mi\epsilon\gamma_{5}\beta\psi_{L}^{\ast}(\Lambda_{P\,}x)\:,\label{26.3.36} \\
    \mathsf{P}^{-1}\mathscr{F}(x)\mathsf{P} &= \mathscr{F}^{\ast}(\Lambda_{P\,}x) \:. \nonumber
\end{align}

还有另一种可能的对称性类型, 称为\,\textit{R}\,-{\kai{对称性}}, 它在\,26.5\,节将要讨论的一些超对称自发破缺模型中十分重要, 也会在\,27.6\,节被用来证明不可重整定理. 正如在\,25.2\,节中所提及的, 在简单\,$N=1$\,超对称理论中, $R\,$-对称性就是在$\,U(1)\,$变换下的不变性, 在这个变换下, 生成元的左手分量(在\,25.2\,节记做$\,\mathcal{Q}_{a}\,$)携带不为零的量子数, 例如$\,-1$, 而它们的共轭, 超对称生成元的右手分量携带相反的量子数$\,+1$. 对方程(\ref{26.2.2})的观察表明, $\theta\,$超空间坐标在$\,R\,$-变换下的性质是不平庸的: $\theta_{L}\,$携带$\,R\,$量子数$\,+1$, 而正比于$\,\theta_{L}^{\ast}\,$的$\,\theta_{R}\,$携带$\,R\,$量子数$\,-1$. 另外, 整个超场可以被赋予一个$\,R\,$-量子数. 如果我们给左手征超场$\,\Phi\,$赋予$\,R\,$量子数$\,R_{\Phi}$, 那么它的标量分量$\,\phi\,$会有相同的$\,R\,$量子数, 左手征旋量分量$\,\psi_{L}\,$有$\,R_{\psi}=R_{\Phi}-1$, 辅助场$\,\mathscr{F}\,$有$\,R_{\mathscr{F}}=R_{\Phi}-2$. 特别地, 为了使超势项$\,\int\dif^{4}x\,[f]_{\mathscr{F}}\,$是 $R\,$守恒的, 超势本身必须有$\,R_{f}=+2$, 所以如果$\,f\,$依赖单个左手征超场$\,\Phi$, 那么它必须正比于$\,\Phi^{2/R_{\Phi}}$. 将其写成另一种形式, 如果$\,f(\Phi)\,$是正比于$\,\Phi^{2}\,$的纯质量项, 那么我们必须选$\,R_{\Phi}=+1$, 而如果$\,f(\Phi)$ 是正比于$\,\Phi^{3}\,$的纯相互作用项, 那么我们必须选$\,R_{\Phi}=2/3$. 另一方面, 对方程(\ref{26.2.10})的观察表明, 超场的$\,D\,$-项和超场的$\,R\,$值相同, 所以为了使作用量中的$\,\int\dif^{4}x\,[K]_{D}\,$项\footnoteB{此处原书误植为$\,\int\dif^{4}x\,[g]_{D}$.\qquad ------译者注}是$\,R\,$守恒的, 唯一需要的是$\,K\,$有$\,R=0$, 而无论我们给$\,\Phi\,$赋予什么样的$\,R\,$值, 只要$\,K\,$中的每一项拥有个数相同的$\,\Phi\,$因子和$\,\Phi^{\ast}\,$因子, $K\,$就满足要求. 当然, 为什么作用量{\kai{应该}}遵循$\,R\,$-对称性, 亦或$\,R\,$-对称性为什么没有自发破缺, 这些现象并没有普遍的原因.


\subsection*{* * *}

还存在其它约束超场使其产生其他类型的场超多重态的方式. 其中较常见的是{\kai{线性}}超场. 为了掌握这类超场的定义条件, 我们注意到, 如果$\,S\,$是一般超场, 那么我们可以构建手征超场
\begin{equation}
    S^{\prime}\equiv \frac{1}{4}\Bigl(\bar{\mathscr{D}}\mathscr{D}\Bigr)\,S \:. \label{26.3.37}
\end{equation}
这是手征超场是因为它可以写成左手征超场$\,\frac{1}{4}(\bar{\mathscr{D}}_{L}\mathscr{D}_{L})S\,$和左手征超场%
$\,\frac{1}{4}(\bar{\mathscr{D}}_{R}\mathscr{D}_{R})S\,$的和. 用$\,S\,$的分量, 它的分量可以写成
\begin{align}
    C^{\prime} &= N \:, \label{26.3.38} \\
    \omega^{\prime} &= \lambda + \slashed{\partial}\omega \:, \label{26.3.39} \\
    M^{\prime} &= -\partial_{\mu}V^{\mu} \:, \label{26.3.40}  \\
    N^{\prime} &= D +\square C \:, \label{26.3.41} \\
    V_{\mu}^{\prime} &= -\partial_{\mu}M \:, \label{26.3.42} \\
    \lambda^{\prime} &= D^{\prime} = 0 \:. \label{26.3.43}
\end{align}
如果以这种方式定义的超场$\,S^{\prime}\,$为零
\begin{equation}
    \Bigl(\bar{\mathscr{D}}\mathscr{D}\Bigr)\,S=0 \:, \label{26.3.44}
\end{equation}
或者用它的分量表示
\begin{equation}
    N=M=\partial_{\mu}V^{\mu} =0 \:, \qquad \lambda = -\slashed{\partial}\omega \:, \qquad
    D=-\square C \:, \label{26.3.45}
\end{equation}
那么多重态$\,S\,$就被称作是线性的. 这种构造留下了四个独立的玻色场------$\,C\,$和使得条件$\,\partial_{\mu}V^{\mu}=0\,$得以满足的$\,V^{\mu}\,$的三个分量, 以及四个独立的费米场------\,Majorana\,4\,-旋量$\,\omega\,$的四个分量. 我们将会在\,26.6\,节看到, 那里定义的流超场的$\,V_{\mu}\,$-项是与对称变换相联系的守恒流, 这个超场是线性超场.

\section{手征超场的可重整理论} \label{sec:26.4}

我们现在将给出标量超场的一般可重整理论的细节. 这会为超对称的应用提供一些启发, 并且我们获得的理论将是第\,28\,章讨论的超对称标准模型的一部分.

正如在\,12.2\,节中所讨论的, 可重整理论的拉格朗日密度只能包含量纲(以动量或能量为单位, 且有$\,\hbar=c=1$)小于或等于\,4\,的算符. 方程(\ref{26.2.6})表明$\,\mathscr{Q}_{\alpha}\,$和随之的$\,\partial/\partial\theta_{\alpha}\,$拥有量纲$\,1/2$, 所以$\,\mathscr{D}_{\alpha}\,$有量纲$\,+1/2$, 而$\,\theta_{\alpha}\,$有量纲$\,-1/2$. 超场$\,S\,$的$\,\mathscr{F}\,$-项和$\,D\,$-项分别是两个$\,\theta\,$因子和四个$\,\theta\,$因子的系数, 所以如果超场的量纲是$\,d(S)$, 那么它的$\,\mathscr{F}\,$-项和$\,D\,$-项分别有量纲$\,d(\mathscr{F}^{S})=d(S)+1\,$和 $d(D^{S})=d(S)+2$. 因此在可重整理论中, 方程(\ref{26.3.30})中的函数$\,f\,$和$\,K\,$分别由量纲最多为\,3\,和\,2\,的算符构成.

基本标量超场$\,\Phi_{n}\,$的量纲是基本标量场的量纲, 或者说$\,+1$, 所以, 为了使函数$\,f\,$中每一项的量纲小于等于\,3, 它能包含的$\,\Phi_{n}\,$因子个数和(或)导数$\,\partial/\partial x^{\mu}\,$的个数和(或)成对旋量超导数$\,\mathscr{D}_{\alpha}\,$的个数不超过\,3. 我们在上一节讨论过, $f\,$中任何包含超导数的左手征项都可以被$\,K\,$中的项替换, 所以可以忽略$\,f\,$中的超导数. 方程(\ref{26.2.30})表明时空导数可以表示成超导数, 所以它们也可以被忽略掉. (在任何情况下, Lorentz\,不变性可以排除掉只有一个时空导数的项, 而在可重整理论中, 有两个导数的项只能包含一个$\,\Phi_{n}\,$因子, 这些导数必须要作用在这个因子上, 所以这样的项对作用量无贡献.) 我们得出: $f(\Phi)\,$最多是$\,\Phi_{n}\,$的三次多项式并且不含时空导数和超导数.

同样的量纲分析表明, 可重整理论中的$\,K\,$最多是$\,\Phi_{n}\,$和$\,\Phi_{n}^{\ast}\,$的四次函数并且没有导数. 然而, $K(\Phi,\Phi^{\ast})\,$中任何只含$\,\Phi_{n}\,$或$\,\Phi_{n}^{\ast}\,$的项将是手征超场, 而手征超场从定义上就没有$\,D\,$-项, 所以 $K(\Phi,\Phi^{\ast})$ 中对$\,[K(\Phi,\Phi^{\ast})]_{D}\,$有贡献的项只能是那些%
{\kai{既}}包含$\,\Phi_{n}\,${\kai{又}}包含$\,\Phi_{n}^{\ast}\,$的项. 因此$\,K(\Phi,\Phi^{\ast})\,$必须是如下的形式
\begin{equation}
    K(\Phi,\Phi^{\ast}) = \sum_{mn}g_{nm}\,\Phi_{n}^{\ast}\Phi_{m} \:, \label{26.4.1}
\end{equation}
它的常系数$\,g_{nm}\,$构成厄米矩阵.

我们现在必须要分别计算$\,f(\Phi)\,$的$\,\mathscr{F}\,$-分量和$\,K(\Phi,\Phi^{\ast})\,$的$\,D\,$-分量. 为了计算$\,K(\Phi,\Phi^{\ast})\,$的$\,D\,$-分量, 我们注意到$\,\Phi_{n}^{\ast}\Phi_{m}\,$中$\,\theta\,$的四阶项是
\begin{align*}
\Bigl[\Phi_{n}^{\ast}\Phi_{m}\Bigr]_{\theta^{4}} &= -\frac{1}{8}\Bigl(\bar{\theta}\gamma_{5}\theta\Bigr)^{2}
\Bigl[\phi_{n}^{\ast}\square\phi_{m}+\Bigl(\square\phi_{m}^{\ast}\Bigr)\phi_{n}\Bigr] \\
&\quad + \Bigl(\bar{\theta}\gamma_{5}\theta\Bigr)\Bigl[\Bigl(\overline{\psi_{n}}\,\theta\Bigr)
\Bigl(\bar{\theta}\gamma^{\mu}\partial_{\mu}\psi_{m}\Bigr)+
\Bigl((\partial_{\mu}\overline{\psi_{n}})\gamma^{\mu}\theta\Bigr)\Bigl(\bar{\theta}\,\psi_{m}\Bigr)\Bigr] \\
& \quad + \frac{1}{4}\mathscr{F}_{n}^{\ast}\mathscr{F}_{m}\,\Bigl(\bar{\theta}(1-\gamma_{5})\theta\Bigr)
\Bigl(\bar{\theta}(1+\gamma_{5})\theta\Bigr) \\
&\quad -\frac{1}{4}\partial^{\mu}\phi_{n}^{\ast}\partial^{\nu}\phi_{m}\,
\Bigl(\bar{\theta}\gamma_{5}\gamma_{\mu}\theta\Bigr)\Bigl(\bar{\theta}\gamma_{5}\gamma_{\nu}\theta\Bigr) \:.
\end{align*}
(\ref{26.A.18})和(\ref{26.A.19})可以让我们把这一表达式对$\,\theta\,$的依赖关系转换成一个总因子%
$\,(\bar{\theta}\gamma_{5}\theta)^{2}$:
\begin{align*}
   \Bigl[\Phi_{n}^{\ast}\Phi_{m}\Bigr]_{\theta^{4}} &= -\frac{1}{4}\Bigl(\bar{\theta}\gamma_{5}\theta\Bigr)^{2}
   \biggl[ \frac{1}{2}\phi_{n}^{\ast}\square\phi_{m}+ \frac{1}{2} \Bigl(\square\phi_{m}^{\ast}\Bigr)\phi_{n}
   -\Bigl(\overline{\psi_{n}}\,\gamma^{\mu}\partial_{\mu}\,\psi_{m}\Bigr) \\
   &\quad +\Bigl((\partial_{\mu}\overline{\psi_{n}})\,\gamma^{\mu}\,\psi_{m}\Bigr)
   +2\mathscr{F}_{n}^{\ast}\mathscr{F}_{m} - \partial^{\mu}\phi_{n}^{\ast}\partial_{\mu}\phi_{m}\biggr] \:.
\end{align*}
一个超场的$\,D\,$-项是$\,-\frac{1}{4}(\bar{\theta}\gamma_{5}\theta)^{2}\,$的系数减去$\,\frac{1}{2}\square\,$作用%
在与$\,\theta\,$独立的项上, 后者对于$\,\Phi_{n}^{\ast}\Phi_{m}\,$就是 $\phi_{n}^{\ast}\phi_{m}$, 所以
\begin{align}
    \frac{1}{2}\Bigl[K(\Phi,\Phi^{\ast})\Bigr]_{D} &= \sum_{nm}g_{nm}\,\biggl[ -\partial_{\mu}\phi_{n}^{\ast}\partial^{\mu}\phi_{m}+ \mathscr{F}_{n}^{\ast}\mathscr{F}_{m} \nonumber \\
    &\quad -\frac{1}{2}\Bigl(\overline{\psi_{nL}}\,\gamma^{\mu}\partial_{\mu}\,\psi_{mL}\Bigr)
    +\frac{1}{2}\Bigl((\partial_{\mu}\overline{\psi_{nL}})\,\gamma^{\mu}\,\psi_{mL}\Bigr)
    \biggr] \:. \label{26.4.2}
\end{align}
如果我们将$\,\Phi_{n}\,$写成新超场$\,\Phi_{m}^{\prime}\,$的线性组合$\,\sum_{m}N_{nm}\Phi_{m}^{\prime}$, 那么表示成新的超场, $K(\Phi,\Phi^{\ast})\,$与原来的在形式上的差别只是$\,g_{nm}\,$被换成了%
$\,g_{nm}^{\prime}=(N^{\dag}gN)_{nm}$. 为了使标量场和旋量场的动能项在符号上和量子对易关系以及反对易关系一致, 正如\,12.5\,节所证明的, 厄米矩阵$\,g_{nm}\,$必须是正定的, 这意味着我们可以选择$\,N\,$使得$\,g_{nm}^{\prime}=\delta_{nm}$. 扔掉撇号, (\ref{26.4.2})现在是
\begin{align}
    \frac{1}{2}\Bigl[K(\Phi,\Phi^{\ast})\Bigr]_{D} &= \sum_{n}\,\biggl[ -\partial_{\mu}\phi_{n}^{\ast}\partial^{\mu}\phi_{n}+ \mathscr{F}_{n}^{\ast}\mathscr{F}_{n} \nonumber \\
    &\quad -\frac{1}{2}\Bigl(\overline{\psi_{nL}}\,\gamma^{\mu}\partial_{\mu}\,\psi_{nL}\Bigr)
    +\frac{1}{2}\Bigl((\partial_{\mu}\overline{\psi_{nL}})\,\gamma^{\mu}\,\psi_{nL}\Bigr)
    \biggr] \:. \label{26.4.3}
\end{align}
我们仍然可以用一个幺正变换重新定义超场而不改变方程(\ref{26.4.3})的形式, 我们不久之后就需要使用这个自由度.


方程(\ref{26.4.3})中包含$\,\phi_{n}\,$和$\,\psi_{nL}\,$的项是传统归一化复标量场和\,Majorana\,旋量场的正确拉格朗日量. 在我们可以考虑质量项后, 我们会将费米子项写成更加熟悉的形式.

在计算$\,f(\Phi)\,$的$\,\mathscr{F}\,$-项时, 最方便的做法是使用超场表示(\ref{26.3.21}), 然后挑出$\,\theta_{L}\,$的二阶项:
\begin{align*}
    \Bigl[f\Bigl(\Phi(x,\theta)\Bigr)\Bigr]_{\theta_{L}^{2}} &=
    \sum_{nm}\Bigl(\theta_{L}^{\mathrm{T}}\,\epsilon\,\psi_{nL}(x)\Bigr)\,
    \Bigl(\theta_{L}^{\mathrm{T}}\,\epsilon\,\psi_{mL}(x)\Bigr)\,
    \frac{\partial^{2}f\Bigl(\phi(x)\Bigr)}{\partial\phi_{n}(x)\,\partial\phi_{m}(x)} \\
    &\quad +\sum_{n}\mathscr{F}_{n}(x)\,\frac{\partial f\Bigl(\phi(x)\Bigr)}{\partial \phi_{n}(x)}\,
    \Bigl(\theta_{L}^{\mathrm{T}}\,\epsilon\,\theta_{L}\Bigr) \:.
\end{align*}
(我们已经这里的$\,x_{+}\,$替换成了$\,x$, 这是因为当乘以一个有两个$\,\theta_{L}\,$因子的式子后, 方程(\ref{26.3.21})中的$\,(\theta_{R}^{\mathrm{T}}\epsilon\gamma^{\mu}\theta_{L})\,$项为零.) 通过使用方程(\ref{26.A.11}), 右边第一项对$\,\theta\,$的依赖关系可以写成标准形式\footnote{注意, $\bar{\psi}_{nL}\,$是$\,\bar{\psi}_{n}\,$的左手分量, 不是$,\overline{\psi_{nL}}$.}
\begin{align*}
    \Bigl(\theta_{L}^{\mathrm{T}}\,\epsilon\,\psi_{nL}\Bigr)\Bigl(\theta_{L}^{\mathrm{T}}\,\epsilon\,\psi_{mL}\Bigr)
&=\biggl(\psi_{nL}^{\mathrm{T}}\,\epsilon\,\biggl(\frac{1+\gamma_{5}}{2}\biggr)\theta\biggr)\,
  \biggl(\theta^{\mathrm{T}}\,\epsilon\,\biggl(\frac{1+\gamma_{5}}{2}\biggr)\psi_{mL}\biggr) \\
&=-\frac{1}{2}\Bigl(\bar{\psi}_{nL}\,\psi_{mL}\Bigr)\,\Bigl(\theta_{L}^{\mathrm{T}}\,\epsilon\,\theta_{L}\Bigr)\:.
\end{align*}
任何左手征超场的$\,\mathscr{F}\,$-项是$\,\mathscr{F}\,$-项是$\,(\theta_{L}^{\mathrm{T}}\epsilon\theta_{L})\,$的系数, 所以这里有
\begin{equation}
    \Bigl[f(\Phi)\Bigr]_{\mathscr{F}} = -\frac{1}{2}\sum_{nm}
    \frac{\partial^{2}f(\phi_{n})}{\partial\phi_{n}\,\partial\phi_{m}}\Bigl(\bar{\psi}_{nL}\,\psi_{mL}\Bigr)
    +\sum_{n}\mathscr{F}_{n}\,\frac{\partial f(\phi)}{\partial \phi_{n}} \:. \label{26.4.4}
\end{equation}
完整的拉格朗日密度是(\ref{26.4.3}), (\ref{26.4.4})和(\ref{26.4.4})的复共轭这三项的和:
\begin{align}
    \mathscr{L} &= \sum_{n}\biggl[-\partial_{\mu}\phi_{n}^{\ast}\partial^{\mu}\phi_{n}
    +\mathscr{F}_{n}^{\ast}\mathscr{F}_{n} \nonumber \\
    &\quad -\frac{1}{2}\Bigl(\overline{\psi_{nL}}\,\gamma^{\mu}\partial_{\mu}\,\psi_{nL}\Bigr)
    +\frac{1}{2}\Bigl((\partial_{\mu}\overline{\psi_{nL}})\,\gamma^{\mu}\,\psi_{nL}\Bigr)  \Biggr]\nonumber \\
    &\quad -\frac{1}{2}\sum_{nm}\frac{\partial^{2}f(\phi)}{\partial\phi_{n}\partial\phi_{m}}
    \,\Bigl(\bar{\psi}_{nL}\,\psi_{mL}\Bigr)-\frac{1}{2}\sum_{nm}
    \Biggl(\frac{\partial^{2}f(\phi)}{\partial\phi_{n}\partial\phi_{m}}\Biggr)^{\ast}
    \,\Bigl(\bar{\psi}_{nL}\,\psi_{mL}\Bigr)^{\ast}  \nonumber \\
    &\quad+\sum_{n}\mathscr{F}_{n}\,\frac{\partial f(\phi)}{\partial \phi_{n}}
    +\sum_{n}\mathscr{F}_{n}^{\ast}\biggl(\frac{\partial f(\phi)}{\partial \phi_{n}}\biggr)^{\ast}\:.\label{26.4.5}
\end{align}

辅助场$\,\mathscr{F}_{n}\,$在作用量中是二次型的形式并且二次项的系数是常数, 所以通过令$\,\mathscr{F}_{n}\,$等于%
拉格朗日密度(\ref{26.4.5})相对$\,\mathscr{F}_{n}\,$和$\,\mathscr{F}_{n}^{\ast}\,$是驻定的值:
\begin{equation}
    \mathscr{F}_{n} = -\biggl(\frac{\partial f(\phi)}{\partial \phi_{n}}\biggr)^{\ast} \:, \label{26.4.6}
\end{equation}
我们就可以消掉它们. 将上式代入方程(\ref{26.4.5})给出
\begin{align}
    \mathscr{L}&= \sum_{n}\biggl[-\partial_{\mu}\phi_{n}^{\ast}\partial^{\mu}\phi_{n}
    -\frac{1}{2}\Bigl(\overline{\psi_{nL}}\,\gamma^{\mu}\partial_{\mu}\,\psi_{nL}\Bigr)
    +\frac{1}{2}\Bigl((\partial_{\mu}\overline{\psi_{nL}})\,\gamma^{\mu}\,\psi_{nL}\Bigr)\biggr]\nonumber \\
    &\quad -\frac{1}{2}\sum_{nm}\frac{\partial^{2}f(\phi)}{\partial\phi_{n}\partial\phi_{m}}\,
    \Bigl(\bar{\psi}_{nL}\,\psi_{mL}\Bigr) - \frac{1}{2} \biggl(\frac{\partial^{2}f(\phi)}{\partial\phi_{n}\partial\phi_{m}}\biggr)^{\ast}
    \,\Bigl(\bar{\psi}_{nL}\,\psi_{mL}\Bigr)^{\ast}  \nonumber \\
    &\quad -\sum_{n}\biggl(\frac{\partial f(\phi)}{\partial \phi_{n}}\biggr)^{\ast}
    \frac{\partial f(\phi)}{\partial \phi_{n}} \:. \label{26.4.7}
\end{align}
因此标量场的势是$\,V(\phi)=\sum_{n}\lvert \partial f(\phi)/\partial\phi_{n}\rvert^{2}$.

当辅助场以这种方式被消除后, 在剩下的场$\,\psi_{nL}\,$和$\,\phi_{n}\,$的超对称变换(\ref{26.3.15})和(\ref{26.3.17})下
\[
\delta\psi_{nL} = \sqrt{2}\partial_{\mu}\phi_{n}\gamma^{\mu}\alpha_{R}
-\sqrt{2}\biggl(\frac{\partial f(\phi)}{\partial \phi_{n}}\biggr)^{\ast}\alpha_{L} \:,\qquad
\delta\phi_{n}=\sqrt{2}\Bigl(\overline{\alpha_{R}}\psi_{nL}\Bigr) \:,
\]
作用量不再是不变的. 原因是: 表达式(\ref{26.4.6})不在服从(\ref{26.3.16})给出的$\,\mathscr{F}_{n}\,$的变换规则%
$\,\delta\mathscr{F}_{n}=\sqrt{2}(\overline{\alpha_{L}}\,\slashed{\partial}\psi_{nL})$, 而是
\[
\delta \biggl(-\frac{\partial f(\phi)}{\partial \phi_{n}}\biggr)^{\ast}
=-\sum_{m} \Biggl(\frac{\partial^{2}f(\phi)}{\partial\phi_{n}\partial\phi_{m}}\Biggr)^{\ast}\delta \phi_{m}^{\ast}
=-\sqrt{2}\sum_{m}\Biggl(\frac{\partial^{2}f(\phi)}{\partial\phi_{n}\partial\phi_{m}}\Biggr)^{\ast}
\Bigl(\overline{\alpha_{L}}\psi_{mR}\Bigr) \:.
\]
由于相同的原因, 在消掉辅助场后, $\phi_{n}\,$和$\,\psi_{nL}\,$的超对称变换的对易子不再由超对称反对易关系给定, 事实上, 它们并没有形成封闭的\,Lie\,超代数. 但这并不与存在满足超对称反对易关系的量子力学算符$\,Q_{\alpha}\,$相矛盾. 这些算符生成了超对称变换, 也就是说, $-\mi(\bar{\alpha}Q)\,$与任何\,Heisenberg\,绘景量子场$\,\phi_{n}\,$或$\,\psi_{nL}\,$的对易子等于这个场%
在无限小参量为$\,\alpha\,$的超对称变换下的变化. 当$\,\mathscr{F}_{n}\,$由方程 (\ref{26.4.6}) 给定后, $-\mi(\bar{\alpha}Q)\,$与$\,\mathscr{F}_{n}\,$的对易子{\kai{由}}%
$\,\delta\mathscr{F}_{n}=\sqrt{2}(\overline{\alpha_{L}}\,\slashed{\partial}\psi_{nL})\,$给定, 这是因为, 在\,Heisenberg\,绘景下, 量子场$\,\psi_{nL}\,$满足从拉格朗日量(\ref{26.4.7})导出的场方程:
\[
\slashed{\partial}\psi_{nL} = -\sum_{m}
\Biggl(\frac{\partial^{2}f(\phi)}{\partial\phi_{n}\partial\phi_{m}}\Biggr)^{\ast}\psi_{mR} \:.
\]
同样的, 当把场方程考虑在内后, 量子场$\,\phi_{n}\,$和$\,\psi_{nL}\,$的超对称变换确实构成封闭的\,Lie\,超代数. 这样的代数通常被称作是{\kai{在壳的}}.

标量场$\,\phi_{n}\,$的零阶期望值$\,\phi_{n0}\,$必须处在方程(\ref{26.4.7})最后一项的极大值点处. 由于这一项不是负的就是零, 时空独立的场值$\,\phi_{n0}\,$使得最大值为零, 这使得
\begin{equation}
\frac{\partial f(\phi)}\partial \phi_{n}\rvert_{\phi=\phi_{0}} =0 \:, \label{26.4.8}
\end{equation}
当然前提是假定这个方程存在解. 方程(\ref{26.4.8})不仅使得方程(\ref{26.4.7})的最后一项取最大值------它也是超对称性不破缺的条件. 真空在超对称变换下不变这个要求使得任何场在超对称变换下的变分的真空期望值应该为零. 玻色场的变分是费米场, 它的真空期望值显然总是为零, 当方程 (\ref{26.3.15})表明$\,\delta\psi_{nL}\,$的真空期望值正比于%
辅助场$\,\mathscr{F}_{n}\,$的真空期望值, 如果超对称性不破缺, 它因此必须为零. 根据方程(\ref{26.4.6}), 在零阶微扰论中, 这个条件要求方程(\ref{26.4.8})必须被满足. 在\,27.6\,节我们会看到, 如果方程(\ref{26.4.8})是被满足的, 那么超对称性直到微扰论的所有阶都是不破缺的.

对于一个左手征标量场$\,\Phi$, 代数基本定理告诉我们多项式$\,\partial f(\phi)/\partial\phi\,$总是在复平面上的某处至少有一个解. 当超场的个数不止一个时, 这是不一定的. 如果我们{\kai{假定}}方程(\ref{26.4.8})存在一个解$\,\phi_{n0}$, 通过令
\begin{equation}
\phi_{n}=\phi_{n0}+\varphi_{n} \:, \label{26.4.9}
\end{equation}
并对$\,\varphi_{n}\,$做幂级数展开, 我们可以计算出这个理论的物理自由度. 通过观察$\,\phi\,$和$\,\psi\,$的二阶项, 我们可以计算出这个理论中的粒子质量:
\begin{align}
\mathscr{L}_{0} &= \sum_{n}\Biggl[
-\partial_{\mu}\varphi_{n}^{\ast}\partial^{\mu}\varphi_{n}-\frac{1}{2}
\Bigl(\overline{\psi_{nL}}\,\gamma^{\mu}\partial_{\mu}\,\psi_{nL}\Bigr)
+\frac{1}{2}\Bigl(\partial_{\mu}(\overline{\psi_{nL}})\,\gamma^{\mu}\,\psi_{nL}\Bigr)\Biggr] \nonumber \\
&\quad -\frac{1}{2}\sum_{nm}\mathscr{M}_{nm}\Bigl(\bar{\psi}_{nL}\,\psi_{mL}\Bigr)
-\frac{1}{2}\sum_{nm}\mathscr{M}_{nm}^{\ast}\Bigl(\bar{\psi}_{nL}\,\psi_{mL}\Bigr)^{\ast} \nonumber \\
&\quad -\sum_{nm}\Bigl(\mathscr{M}^{\dag}\mathscr{M}\Bigr)_{mn}\varphi_{m}^{\ast}\varphi_{n}\:,
\label{26.4.10}
\end{align}
其中$\,\mathscr{M}\,$是对称复矩阵
\begin{equation}
\mathscr{M}_{mn} \equiv \Biggl(\frac{\partial^{2}f(\phi)}{\partial \phi_{n}\partial \phi_{m}}\Biggr)_{\phi=\phi_{0}} \:. \label{26.4.11}
\end{equation}
现在, 如果我们通过一个幺正变换重新定义这些场
\begin{equation}
\varphi_{n}=\sum_{m}\mathscr{U}_{nm}\varphi_{m}^{\prime} \:, \qquad
\psi_{nL}=\sum_{m}\mathscr{U}_{nm}\psi_{mL}^{\prime} \:, \label{26.4.12}
\end{equation}
自由场拉格朗日量(\ref{26.4.10})的形式不会因此改变, 但是要把$\,\mathscr{M}\,$换成$\,\mathscr{M}^{\prime}$,
其中
\begin{equation}
\mathscr{M}^{\prime}= \mathscr{U}^{\mathrm{T}}\mathscr{M}\mathscr{U} \:. \label{26.4.13}
\end{equation}
根据矩阵代数的一个定理, 对于任何复对称矩阵$\,\mathscr{M}$, 总能找到一个幺正矩阵$\,\mathscr{U}\,$使得%
方程(\ref{26.4.13}) 定义的矩阵$\,\mathscr{M}^{\prime}\,$是对角矩阵且矩阵元为正实数$\,m_{n}$. (为了将来的使用, 我们注意到 $\mathscr{M}^{\prime\dag}\mathscr{M}^{\prime}=\mathscr{U}^{\dag}\mathscr{M}^{\dag}\mathscr{M}%
\mathscr{U}$, 所以$\,m_{n}^{2}\,$就是正定厄米矩阵$\,\mathscr{M}^{\dag}\mathscr{M}\,$的本征值.) 以这种方式重新定义场并扔掉撇号, 拉格朗日量的二次部分现在是
\begin{align}
\mathscr{L}_{0} &= \sum_{n}\Biggl[
-\partial_{\mu}\varphi_{n}^{\ast}\partial^{\mu}\varphi_{n}-\frac{1}{2}
\Bigl(\overline{\psi_{nL}}\,\gamma^{\mu}\partial_{\mu}\,\psi_{nL}\Bigr)
+\frac{1}{2}\Bigl(\partial_{\mu}(\overline{\psi_{nL}})\,\gamma^{\mu}\,\psi_{nL}\Bigr)\Biggr] \nonumber \\
&\quad -\frac{1}{2}\sum_{n}m_{n}\Bigl(\bar{\psi}_{nL}\,\psi_{nL}\Bigr)
-\frac{1}{2}\sum_{n}m_{n}^{\ast}\Bigl(\bar{\psi}_{nL}\,\psi_{nL}\Bigr)^{\ast} \nonumber \\
&\quad -\sum_{n}m_{n}^{2}\varphi_{n}^{\ast}\varphi_{n}\:.
\label{26.4.14}
\end{align}
为了将费米子质量项变成更加熟悉的形式, 我们引入作为{\,\textit{Majorana}\,}场定义的场$\,\psi_{n}(x)$, 它的左手分量是$\,\psi_{nL}(x)$. 那么利用\,Majorana\,双线性型的对称性质(\ref{26.A.7}):
\begin{align*}
&-\frac{1}{2}\Bigl(\overline{\psi_{nL}}\,\gamma^{\mu}\partial_{\mu}\,\psi_{nL}\Bigr)
+\frac{1}{2}\Bigl(\partial_{\mu}(\overline{\psi_{nL}})\,\gamma^{\mu}\,\psi_{nL}\Bigr) \\
&\qquad =-\frac{1}{2}\biggl(\overline{\psi_{n}}\gamma^{\mu}\biggl(\frac{1+\gamma_{5}}{2}\biggr)
\partial_{\mu}\psi_{n}\biggr)
+\frac{1}{2}\biggl(\partial_{\mu}(\overline{\psi_{n}})\gamma^{\mu}\biggl(\frac{1+\gamma_{5}}{2}\biggr)
\psi_{n}\biggr) \\
&\qquad =-\frac{1}{2}\biggl(\overline{\psi_{n}}\gamma^{\mu}\biggl(\frac{1+\gamma_{5}}{2}\biggr)
\partial_{\mu}\psi_{n}\biggr)
-\frac{1}{2}\biggl(\overline{\psi_{n}}\gamma^{\mu}\biggl(\frac{1-\gamma_{5}}{2}\biggr)
\partial_{\mu}\psi_{n}\biggr) \\
&\qquad =-\frac{1}{2}\Bigl(\overline{\psi_{n}}\gamma^{\mu}\partial_{\mu}\psi_{n}\Bigr) \:,
\end{align*}
而实性质(\ref{26.A.21})给出
\[
\Bigl(\bar{\psi}_{nL}\,\psi_{nL}\Bigr) + \Bigl(\bar{\psi}_{nL}\,\psi_{nL}\Bigr)^{\ast}
=2\operatorname{Re}\biggl(\overline{\psi_{n}}\,\biggl(\frac{1+\gamma_{5}}{2}\biggr)\,\psi_{n}\biggr)
=\Bigl(\overline{\psi_{n}}\,\psi_{n}\Bigr) \:.
\]
这样, 完整的二次拉格朗日量就是
\begin{align}
\mathscr{L}_{0}&=\sum_{n}\Biggl[ -\partial_{\mu}\varphi_{n}^{\ast}\partial^{\mu}\varphi_{n}
-\sum_{n}m_{n}^{2}\,\varphi_{n}^{\ast}\varphi_{n} \nonumber \\
&\quad -\frac{1}{2}\Bigl(\overline{\psi_{n}}\,\gamma^{\mu}\partial_{\mu}\,\psi_{n}\Bigr)
-\frac{m_{n}}{2}\Bigl(\overline{\psi_{n}}\,\psi_{n}\Bigr) \Biggr]\:. \label{26.4.15}
\end{align}
费米子项前面有因子$\,1/2\,$是因为它们是\,Majorana\,费米场, 而标量项前面没有因子$\,1/2\,$是因为它们是复标量. 我们看到无自旋粒子和自旋\,1/2\,粒子有相等的质量$\,m_{n}$, 这正是该理论的未破缺超对称性所要求的.

拉格朗日密度中的相互作用部分$\,\mathscr{L}^{\prime}\,$由方程(\ref{26.4.7})中比
$\,\varphi_{n}\,$和$\,\psi_{n}\,$的二阶项还要高阶的项给定. 由于超势$\,f(\phi_{0}+\varphi)\,$被假定是三次多项式且在$\,\varphi_{n}=0\,$处驻定, 而$\,\varphi\,$的定义又使得二阶项是 $\tfrac{1}{2}\sum_{n}m_{n}\varphi_{n}^{2}$, 我们可以将超势(除了一个不重要的常数项以外的部分)写成
\begin{equation}
    f(\phi_{0}+\varphi) = \frac{1}{2}\sum_{n}m_{n}\,\varphi_{n}^{2}
    +\frac{1}{6}\sum_{nm\ell}f_{nm\ell}\,\varphi_{n}\,\varphi_{m}\,\varphi_{\ell} \:. \label{26.4.16}
\end{equation}
在方程(\ref{26.4.7})中使用上式就给出了相互作用
\begin{align}
    \mathscr{L}^{\prime} &= -\frac{1}{2}\sum_{nm\ell} f_{nm\ell}\,\varphi_{n}\,
    \biggl(\overline{\psi_{m}}\,\biggl(\frac{1+\gamma_{5}}{2}\biggr)\,\psi_{\ell}\biggr)\nonumber \\
    &\quad -\frac{1}{2}\sum_{nm\ell} f^{\ast}_{nm\ell}\,\varphi^{\ast}_{n}\,
    \biggl(\overline{\psi_{m}}\,\biggl(\frac{1-\gamma_{5}}{2}\biggr)\,\psi_{\ell}\biggr)\nonumber \\
    &\quad -\frac{1}{2}\sum_{nm\ell}m_{n}\,f_{nm\ell}\varphi_{n}^{\ast}\varphi_{m}\varphi_{\ell}
    -\frac{1}{2}\sum_{nm\ell}m_{n}\,f_{nm\ell}^{\ast}\varphi_{n}\varphi_{m}^{\ast}\varphi_{\ell}^{\ast} \nonumber\\
    &\quad-\frac{1}{4}\sum_{nm\ell m^{\prime}\ell^{\prime}} f_{nm\ell}f^{\ast}_{nm^{\prime}\ell^{\prime}}
    \varphi_{m}\varphi_{\ell}\varphi_{m^{\prime}}^{\ast}\varphi_{\ell^{\prime}}^{\ast} \:. \label{26.4.17}
\end{align}
我们看到, 知道了质量$\,m_{n}\,$以及标量和费米子的\,``Yukawa''\,耦合$\,f_{nml}\,$就足以%
定出无自旋场的所有三次项和四次自对偶耦合.

作为一个例子, 考察只有一个左手征超场的情况. 为了与前面的结果进行对比, 我们将方程(\ref{26.4.16})中的单个系数$\,f\,$写成
\begin{equation}
    f\equiv 2\sqrt{2}\,\me^{\mi\alpha}\,\lambda \:, \label{26.4.18}
\end{equation}
其中$\,\lambda\,$是实的而$\,\alpha\,$是某个实相位. 我们同时还会引入一对无自旋实场$\,A(x)\,$和$\,B(x)$, 方法是将这里的单个复标量写成
\begin{equation}
    \varphi \equiv \me^{-\mi\alpha}\biggl(\frac{A+\mi B}{\sqrt{2}}\biggr) \:. \label{26.4.19}
\end{equation}
这样, 方程(\ref{26.4.15})和(\ref{26.4.17})就给出了整个拉格朗日密度
\begin{align}
    \mathscr{L}&= -\tfrac{1}{2}\partial_{\mu}A\partial^{\mu}A -\tfrac{1}{2}\partial_{\mu}B\partial^{\mu}B
    -\tfrac{1}{2}m^{2}\,(A^{2}+B^{2}) \nonumber \\
    &\quad -\tfrac{1}{2}\Bigl(\bar{\psi}\gamma^{\mu}\partial_{\mu}\psi\Bigr)
    -\tfrac{1}{2}\,\Bigl(\bar{\psi}\psi\Bigr) \nonumber \\
    &\quad -\lambda\,A\,\Bigl(\bar{\psi}\psi\Bigr) -\mi\lambda\,B\,\Bigl(\bar{\psi}\gamma_{5}\psi\Bigr)\nonumber \\
    &\quad -m\,\lambda\,A\,(A^{2}+B^{2})-\tfrac{1}{2}\lambda^{2}\,(A^{2}+B^{2})^{2} \:. \label{25.4.20}
\end{align}
它与\,Wess\,和\,Zumino\,最初发现的拉格朗日密度(\ref{24.2.9})是相同的.\cite{2} 在这个简单情况中值得注意的是, 即使我们不假定宇称守恒再进行推导, 拉格朗日量现在在如下的空间反演变换下是不变的:
\begin{equation}
    A(x)\to A(\Lambda_{P\,}x)\:,\qquad B(x)\to -B(\Lambda_{P\,}x)\:,\qquad
    \psi(x)\to \mi\beta\psi(\Lambda_{P\,}x) \:. \label{26.4.21}
\end{equation}
宇称守恒作为``偶然''对称性出现是各种可重整规范理论熟悉的特征(参看\,12.5\,节和\,18.7\,节), 但是它不是涉及无自旋场的理论的特征, 所以这是超对称性在单个标量超场的可重整理论中的特殊结果.

\section{树级近似下的自发超对称性破缺} \label{sec:26.5}

我们在上一节看到, 在手征超场的可重整理论中, 如果方程(\ref{26.4.8})有解, 即, 如果存在场值$\,\phi_{0}$ 使得超势是驻定的:
\begin{equation}
    \frac{\partial f(\phi)}{\partial\phi_{n}}\bigg\rvert_{\phi=\phi_{0}}=0\:,\label{26.5.1}
\end{equation}
那么超对称性(至少在树级近似下)是不破缺的. 这里独立变量的个数和方程的个数相等, 所以我们一般会期待方程(\ref{26.5.1})有解. 为了使超对称性在这些理论中自发破缺, 我们有必要给超势的形式附加一些限制.

为了看到该如何选择超势进而使得超对称性可以自发破缺, 我们将会考察\,O'Raifeartaigh\,给出的一类模型的推广.\cite{3} 假定超势是一组左手征超场$\,Y_{i}\,$的线性组合, 且它的系数由第二组左手征超场$\,X_{n}\,$的函数$\,f_{i}(X)\,$给定:
\begin{equation}
    f(X,Y) = \sum_{i} Y_{i}f_{i}(x) \:. \label{26.5.2}
\end{equation}
超对称性不被这些超场的标量分量值$\,x_{n}\,$和$\,y_{i}\,$破缺的条件是
\begin{align}
    0&=\frac{\partial f(x,y)}{\partial y_{i}} = f_{i}(x) \:, \label{26.5.3} \\
    0&=\frac{\partial f(x,y)}{\partial x_{n}} = \sum_{i}y_{i}\frac{\partial f_{i}(x)}{\partial x_{n}}\:.\label{26.5.4}
\end{align}
方程(\ref{26.5.4})总可以通过取$\,y_{i}=0\,$被解掉, 并且它对解方程(\ref{26.5.3})没有影响. 另一方面, 如果超场$\,X_{n}\,$的个数小于超场$\,Y_{i}\,$的个数, 那么方程(\ref{26.5.3})在$\,x_{n}\,$上附加的条件要多于变量, 所以如果没有精细调节, 找到解是不可能的, 这时超对称是破缺的.

初始假定(\ref{26.5.2})本身看上去似乎代表了精细调节的一个基本形式, 但是这一形式可以通过假定合适的$\,R\,$-对称性附加到超势上. 就像在\,26.3\,节讨论的, 在有$\,N=1\,$超对称性的理论中, $\,R\,$-对称性是使得$\,\theta\,$超空间坐标有不平庸变换性质的$\,U(1)\,$对称性. 如果我们假定一个$\,R\,$-对称性使得$\,\theta_{L}\,$携带量子数$\,+1$, 那么任何超势的$\,\mathscr{F}\,$-项的量子数等于这个超势本身的量子数减\,2, 所以$\,R\,$不变性要求超势本身有$\,R=2$. 因此, 在超场$\,Y_{i}\,$和$\,X_{n}\,$分别有$\,R\,$量子数$\,+2\,$和$\,0\,$时, 我们可以通过要求$\,R\,$不变性来附加结构(\ref{26.5.2}).

在这类模型中, 标量场有势
\begin{equation}
    V(x,y) = \sum_{i} \lvert f_{i}(x)\rvert^{2} +
    \sum_{n} \Biggl\lvert \sum_{i}y_{i}\frac{\partial f_{i}(x)}{\partial x_{n}}\Biggr\rvert^{2}\:. \label{26.5.5}
\end{equation}
通过选择$\,x_{n}\,$使得第一项最小, 我们总可以到达这个势的最小值; 无论这要求$\,x_{n}\,$取什么值, 第二项总可以通过取$\,y_{i}=0\,$到达最小值. 无论超对称性是否自发破缺, 这些模型有一个独特的特征: 在场的空间中总有一个方向使得势的极小值点是平坦的. 无论$\,x_{n}\,$使得方程(\ref{26.5.5})第一项最小的值$\,x_{n0}\,$是什么, 第二项不仅对于$\,y_{i}=0\,$是零, 而且对于任何矢量$\,y_{i}$, 只要它所处的方向与矢量 $(v^{n})_{i}=(\partial f/\partial x_{n})_{x=x_{0}}\,$均垂直, 那么第二项也是零. 如果存在$\,N_{X}\,$个超场$\,X_{n}\,$和$\,N_{Y}\,$个超场$\,Y_{i}\,$且 $N_{Y}>N_{X}$, 那么$\,v^{n}\,$无法展开$\,y\,$的空间, 这样只有有$\,N_{Y}-N_{X}\,$个平坦反向. 对于任何沿着这些平坦方向的非零$\,y_{i}=y_{0i}$, 拉格朗日密度的$\,R\,$-对称性是自发破缺的, 与这个整体对称性破缺相联系的\,Goldstone\,玻色场$\,\phi\,$对应于与$\,y_{i}\,$中的$\,y_{0i}\,$正比的项.

这类模型中最简单的一个例子是只有一个$\,X\,$超场和两个$\,Y\,$超场的情况. 可重整性要求系数函数$\,f_{i}(X)\,$是$\,X\,$的二次函数, 通过取合适的$\,Y_{i}\,$的线性组合并对$\,X\,$做偏移和重标度, 我们可以选择这些函数使得
\begin{equation}
    f_{1}(X) = X-a\:, \qquad \quad f_{2}(X) =X^{2} \:, \label{26.5.6}
\end{equation}
其中$\,a\,$是任意常数. 除非超势被精细调节使得$\,a=0$, 否则两个方程(\ref{26.5.1})显然不可能同时有解. 势(\ref{26.5.5})在这里是
\begin{equation}
    V(x,y) = \lvert x\rvert^{4} + \lvert x-a\rvert^{2} + \lvert y_{1}+2xy_{2}\rvert^{2} \:. \label{26.5.7}
\end{equation}
前两项的和有唯一的整体最小值点$\,x_{0}$. 这里的平坦方向是使得$\,y_{1}+2x_{0}y_{2}=0\,$的方向. 当$\,a=0\,$时, 我们有$\,x_{0}=0$, 势能的最小值就是沿着$\,y_{1}=0\,$而$\,y_{2}\,$任意的那条线.

无论自发破缺的原因是什么, 这个现象总需要存在一个无质量的自旋\,1/2\,粒子, {\kai{戈德斯通微子}}(\textit{goldstino}), 它是与普通整体对称性自发破缺相联系的\,Goldstone\,玻色子的类似物. (在超引力理论中有一个例外, 我们会在\,31.3\,节讨论, 那里的超对称性是定域对称性, 戈德斯通微子是作为有质量自旋\,3/2\,粒子的$\,\pm1/2\,$螺旋度态出现的, 即引力微子的$\,\pm1/2\,$螺旋度态.) 在手征超场的可重整理论中, 标量场的树级近似期望值$\,\phi_{n0}\,$必须处在方程(\ref{26.4.7})中势能%
$\,\sum_{n}\lvert\partial f(\phi)/\partial\phi_{n}\rvert^{2}\,$的极小值处, 所以
\begin{equation}
    \sum_{m}\mathscr{M}_{nm}
    \Biggl(\frac{\partial f(\phi)}{\partial\phi_{m}}\bigg\rvert_{\phi=\phi_{0}}\Biggr)^{\ast}=0\:,\label{26.5.8}
\end{equation}
其中
\begin{equation}
    \mathscr{M}_{nm}\equiv \frac{\partial^{2}f(\phi)}{\partial\phi_{n}\partial \phi_{m}}\Bigg\rvert_{\phi=\phi_{0}}\:.
    \label{26.5.9}
\end{equation}
如果方程(\ref{26.5.1})没有被满足, 那么方程(\ref{26.5.8})告诉我们矩阵$\,\mathscr{M}_{nm}\,$%
至少有一个本征值为零的本征矢, 所以, 根据方程(\ref{26.4.10}), 至少存在一组由$\,\psi_{n}\,$描述的自旋$\,1/2\,$粒子的线性组合使得它的质量为零. 例如, 对于方程(\ref{26.5.2})和(\ref{26.5.6})定义的模型, 矩阵$\,\mathscr{M}\,$不为零的分量是
\begin{equation}
    \mathscr{M}_{xy_{1}}=\mathscr{M}_{y_{1}x}=1\:, \qquad
    \mathscr{M}_{xy_{2}}=\mathscr{M}_{y_{2}x}=2x_{0} \:, \label{26.5.10}
\end{equation}
所以这个矩阵有本征值$\,\pm2x_{0}\,$和$\,0$, 其中最后一个本征值对应戈德斯通微子模. 在第\,29\,章, 我们将在不使用微扰论的情况下证明超对称性自发破缺要求存在戈德斯通微子, 并在那里探索它们的性质.



\section{超空间积分, 场方程和流超场} \label{sec:26.6}

我们用来构建拉格朗日密度的``$\,\mathscr{F}\,$-项''和``$\,D\,$-项可以表示为在超空间坐标$\,\theta_{\alpha}\,$%
上的积分. 最初由\,Berezin\,\cite{4}给出的费米参量积分规则已经在\,9.5\,节推导过了. 简言之, 由于任何费米参量的平方为零, $N\,$ 个费米参量$\,\xi_{n}\,$的任何函数可以表示成
\begin{equation}
f(\xi) = \biggl(\prod_{n=1}^{N}\xi_{n}\biggr)c + \xi\,\text{因子较少的项}  \:, \label{26.6.1}
\end{equation}
而它对$\,\xi\,$的积分就定义成
\begin{equation}
    \int \dif^{N}\xi \: f(\xi) \equiv c \:. \label{26.6.2}
\end{equation}
系数$\,c\,$本身可以依赖其它未积分的\,c\,-数变量, 这些变量与我们要进行积分的$\,\xi\,$反对易, 在这种情况下, 固定$\,c\,$的定义就十分重要, 做法和方程(\ref{26.6.1})一样, 在积分之前把所有$\,\xi\,$移至$\,c\,$的左边. 在这个定义下, 对费米变量的积分是线性算符. 由于将变量$\,\xi_{n}\,$偏移一个常数$\,a_{n}\,$的%
$\,\xi_{n}\to\xi_{n}+a_{n}\,$对乘积的影响只是那些$\,\xi\,$因子较少的项, 它不影响积分的值
\begin{equation}
    \int \dif^{N}\xi \: f(\xi+a) =\int \dif^{N}\xi \: f(\xi) \:, \label{26.6.3}
\end{equation}
在这种意义下它类似于对实变量的积分. 另外, 作为方程(\ref{26.6.2})的特殊情况, 如果$\,N\,$个费米参量的多项式的阶数$\,<N$, 那么对它的积分为零. 在变量代换对积分的影响上, 对费米参量的积分和对玻色参量的积分非常不同: 对于玻色参量$\,x_{n}$, 我们有$\,\dif^{N}x^{\prime}=\operatorname{Det}(\partial x^{\prime}/\partial x)\,\dif^{N}x$, 而对于费米参量
\begin{equation}
    \dif^{N}\xi^{\prime} = [\operatorname{Det}(\partial \xi^{\prime}/\partial \xi)]^{-1}\dif^{N}\xi\:.\label{26.6.4}
\end{equation}
特别地, $\dif\xi\,$的量纲与$\,\xi\,$的量纲{\kai{相反}}.

根据方程(\ref{26.2.10}), 一般超场$\,S(x,\theta)\,$(可能是基本的也可能是复合的)的$\,D\,$-项在相差一%
个导数的意义下等于$\,-(\bar{\theta}\gamma_{5}\theta)^{2}/4=-(\theta^{\mathrm{T}}\epsilon\theta)^{2}/4\,$的系数. 四个$\,\theta\,$中的任何一个都可能是$\,\theta_{1}$, 而每种可能性给出相等的贡献, 所以我们可以假定$\,\theta_{1}\,$是最左边的, 这样就挑出了一个因子$\,4$. 这样$\,\theta_{2}\,$必须是下一个最左边的. 剩下两个$\,\theta\,$中任何一个都可能是$\,\theta_{3}$, 每种可能性给出相同的贡献, 所以我们可以假定$,\theta_{3},$是左边第三个并挑出了因子$\,2$, 这样$\,\theta_{4}\,$必须在最右边. 即,
\[
-\tfrac{1}{4}(\bar{\theta}\gamma_{5}\theta)^{2} = -\tfrac{1}{4}\times 4\times 2 \times \theta_{1}\theta_{2}\theta_{3}\theta_{4} \:,
\]
所以$\,\theta\,$的这个函数的系数是$\,-1/2\,$乘以对$\,\dif^{4}\theta\,$的积分. 因为这在相差一个导数的意义下是$\,D\,$-项, 我们就有
\begin{equation}
    \int \dif^{4}x\:[S]_{D} = -\frac{1}{2}\int \dif^{4}x\int \dif^{4}\theta \: S(x,\theta) \:.\label{26.6.5}
\end{equation}
以同样的方式, 利用方程(\ref{26.3.11}), 我们发现对一般左手征超场$\,\Phi\,$(和前面一样, 可以是基本的也可以是复合的)的$\,\mathscr{F}\,$-项的时空积分可以表示为
\begin{equation}
    \int \dif^{4}x\: [\Phi]_{\mathscr{F}}
    = \frac{1}{2}\int \dif^{4}x\int \dif^{2}\theta_{L}\:\Phi(x,\theta)\:. \label{26.6.6}
\end{equation}

既然我们现在要对$\,\theta\,$积分, 引入$\,\delta\,$函数是方便的, 它像往常一样被如下的条件定义: 对于任意函数$\,f(\theta)$,
\begin{equation}
    \int \dif^{4}\theta^{\prime}\:\delta^{4}(\theta^{\prime}-\theta)\,f(\theta^{\prime})=f(\theta)\:.\label{26.6.7}
\end{equation}
根据方程(\textcolor{foo}{9.5.40}),
\begin{align}
    &\delta^{4}(\theta^{\prime}-\theta)=(\theta_{1}^{\prime}-\theta_{1})(\theta_{2}^{\prime}-\theta_{2})
    (\theta_{3}^{\prime}-\theta_{3})(\theta_{4}^{\prime}-\theta_{4}) \nonumber \\
    &\quad = \frac{1}{4}\Bigl[\Bigl(\theta_{L}-\theta_{L}^{\prime}\Bigr)^{\mathrm{T}}\,\epsilon\,
    \Bigl(\theta_{L}-\theta_{L}^{\prime}\Bigr)\Bigr]\:\Bigl[\Bigl(\theta_{R}-\theta_{R}^{\prime}\Bigr)^{\mathrm{T}}
    \,\epsilon\,\Bigl(\theta_{R}-\theta_{R}^{\prime}\Bigr)\Bigr]  \label{26.6.8}
\end{align}
是满足这个条件的.

将作用量表示成超空间上的积分使得推导超场形式的场方程变得容易. 例如, 考察一组左手征超场$\,\Phi_{n}\,$的作用量%
(左手征超场$\,\Phi_{n}\,$的一般可重整理论是它的一个特殊情况):
\begin{equation}
    I=\frac{1}{2}\int \dif^{4}x\: \Bigl[K(\Phi,\Phi^{\ast})\Bigr]_{D}
    +2\operatorname{Re}\int\dif^{4}x\:[f(\Phi)]_{\mathscr{F}} \:, \label{26.6.9}
\end{equation}
其中$\,K\,$是$\,\Phi_{n}\,$和$\,\Phi_{n}^{\ast}\,$的不带导数的任意函数, $f\,$是$\,\Phi_{n}\,$的任意函数同时也不含导数. (写下这种形式的作用量并它表示成分量场的动机将在\,26.8\,节进行阐述.) 仅通过要求作用量对$\,\Phi\,$的任意变分均是驻定, 我们无法导出正确的场方程, 这是因为$\,\Phi_{n}\,$是被左手征超场的要求$\,\mathscr{D}_{R}\Phi_{n}=0\,$约束的. 为了确保任意变分不会破坏这个条件, 我们要用到一个小技巧, 这个技巧在第\,30\,章推导超空间 Feynman\,规则时也将是有用的. 我们将$\,\Phi_{n}\,$写成{\kai{势超场}}$\,S_{n}(x,\theta)$
\begin{equation}
    \Phi_{n} = \mathscr{D}_{R}^{2}\,S_{n} \:, \label{26.6.10}
\end{equation}
(利用方程(\ref{26.A.21}))由此可以得出
\begin{equation}
    \Phi^{\ast}_{n} = -\mathscr{D}_{L}^{2}\,S_{n}^{\ast} \:, \label{26.6.11}
\end{equation}
其中$\,\mathscr{D}_{R}^{2}\,$和$\,\mathscr{D}_{L}^{2}\,$分别是%
$\,(\mathscr{D}_{R}^{\mathrm{T}}\epsilon\mathscr{D}_{R})=-(\bar{\mathscr{D}}_{R}\mathscr{D}_{R})\,$和%
$\,(\mathscr{D}_{L}^{\mathrm{T}}\epsilon\mathscr{D}_{L})=(\bar{\mathscr{D}}_{L}\mathscr{D}_{L})\,$的简写. 为了看到总能找到满足方程(\ref{26.6.10})的解(不一定定域), 注意到, 对于任何左手征超场$\,\Phi_{n}$,
\begin{equation}
   \mathscr{D}_{R}^{2}\mathscr{D}_{L}^{2}\Phi_{n} = -16\square \Phi_{n} \:, \label{26.6.12}
\end{equation}
这使得
\begin{equation}
    {-}16\square S_{n} =\mathscr{D}_{L}^{2}\Phi_{n} \:. \label{26.2.13}
\end{equation}
的解满足方程(\ref{26.6.10}).

对于任何$\,S$, $\mathscr{D}_{R}^{2}S\,$是左手征的, 所以作用量相对$\,S_{n}\,$的任意变分必须是驻定的. 利用方程 (\ref{26.6.5}), 写成$\,S_{n}\,$和$\,S_{n}^{\ast}$, 作用量可以表示成
\begin{equation}
    I=-\frac{1}{4}\int\dif^{4}x\int\dif^{4}\theta\:K(-\mathscr{D}_{L}^{2}S^{\ast},\mathscr{D}_{R}^{2}S)
    + 2\operatorname{Re}\int \dif^{4}x\:\Bigl[f(\mathscr{D}_{R}^{2}S)\Bigr]_{\mathscr{F}} \:. \label{26.6.14}
\end{equation}
第一项在$\,S_{n}\,$(而不是$\,S_{n}^{\ast}$)的无限小变化$\,\delta S_{n}\,$下的变分可以通过超空间中的分部积分简单地计算出来:
\begin{align*}
    &{-}\delta\frac{1}{4}\int\dif^{4}x\int\dif^{4}\theta\:K(-\mathscr{D}_{L}^{2}S^{\ast},\mathscr{D}_{R}^{2}S) \\
    &\qquad = -\sum_{n}\int \dif^{4}\theta\:\delta S_{n}\,\mathscr{D}_{R}^{2}\,
    \frac{\delta K(-\mathscr{D}_{L}^{2}S^{\ast},\mathscr{D}_{R}^{2}S)}{\delta \mathscr{D}_{R}^{2}S_{n}} \:.
\end{align*}
对于超势项中的积分在$\,S_{n}\,$的无限小变化$\,\delta S_{n}\,$下的变分, 方程(\ref{26.3.31})和(\ref{26.6.5})使得我们可以将其表示成
\begin{align*}
    \delta \int \dif^{4}x\:\Bigl[f(\mathscr{D}_{R}^{2}S)\Bigr]_{\mathscr{F}}
    &= \sum_{n}\int \dif^{4}x\:\Biggl[\frac{\partial f(\Phi)}{\partial \Phi_{n}}\bigg\rvert_{\Phi=\mathscr{D}_{R}^{2}S}\mathscr{D}_{R}^{2}\delta S_{n}\Biggr]_{\mathscr{F}} \\
    &=\sum_{n}\int \dif^{4}x\:\Biggl[\mathscr{D}_{R}^{2} \Biggl(\frac{\partial f(\Phi)}{\partial \Phi_{n}}\bigg\rvert_{\Phi=\mathscr{D}_{R}^{2}S}\delta S_{n}\Biggr)\Biggr]_{\mathscr{F}} \\
    &= 2\sum_{n}\int \dif^{4}x\:\Biggl[\frac{\partial f(\Phi)}{\partial \Phi_{n}}\bigg\rvert_{\Phi=\mathscr{D}_{R}^{2}S}\delta S_{n}\Biggr]_{D} \\
    &= -\sum_{n}\int \dif^{4}x\int\dif^{4}\theta\:\frac{\partial f(\Phi)}{\partial \Phi_{n}}\bigg\rvert_{\Phi=\mathscr{D}_{R}^{2}S}\delta S_{n} \:.
\end{align*}
这样, 方程(\ref{26.6.14})对$\,S_{n}\,$的任意变分是驻定的这一条件就是
\[
\mathscr{D}_{R}^{2}\,\frac{\delta K(-\mathscr{D}_{L}^{2}S^{\ast},\mathscr{D}_{R}^{2}S)}{\delta \mathscr{D}_{R}^{2}S_{n}}
=-4 \,\frac{\partial f(\Phi)}{\partial \Phi_{n}}\bigg\rvert_{\Phi=\mathscr{D}_{R}^{2}S} \:,
\]
或者表示成手征超场
\begin{equation}
    \mathscr{D}_{R}^{2}\,\frac{\delta K(\Phi,\Phi^{\ast})}{\delta \Phi_{n}}
    =-4 \,\frac{\partial f(\Phi)}{\partial \Phi_{n}} \:. \label{26.6.15}
\end{equation}
复共轭给出
\begin{equation}
    \mathscr{D}_{L}^{2}\,\frac{\delta K(\Phi,\Phi^{\ast})}{\delta \Phi_{n}^{\ast}}
    =4 \,\biggl(\frac{\partial f(\Phi)}{\partial \Phi_{n}} \biggr)^{\ast} \:. \label{26.6.16}
\end{equation}
可以很容易地验证这些方程的分量给出$\,\Phi_{n}^{\ast}\,$和$\,\Phi_{n}\,$分量的场方程. 例如, 回忆起$\,\mathscr{D}_{R}^{2}(\theta_{R}^{\mathrm{T}}\epsilon\theta_{R})=-4$, $\mathscr{D}_{R}^{2}\Phi_{n}^{\ast}\,$中与$\,\theta\,$无关的部分是$\,4\mathscr{F}_{n}^{\ast}$, 而$\,\partial f(\Phi)/\partial\Phi_{n}\,$中与$\,\theta\,$无关的部分是$\,\partial f(\phi)/\partial \phi_{n}$, 所以对于$\,K=\sum_{n}\Phi_{n}^{\ast}\Phi_{n}$, 方程(\ref{26.6.15})中与$\,\theta\,$无关的部分给出关系%
$\,\mathscr{F}_{n}^{\ast}=-\partial f(\phi)/\partial\phi_{n}$, 这与方程(\ref{26.4.6})一致.

作为如何使用这一形式体系的一个例子, 我们来考察守恒流所属的那个超场. 假定作用量中的超势和\,K\"{a}hler\,势在如下的整体变换下不变
\begin{equation}
    \delta \Phi_{n} =\mi\epsilon\sum_{m}\mathscr{T}_{nm}\Phi_{m} \:, \qquad \quad
    \delta \Phi_{n}^{\ast} =-\mi\epsilon\sum_{m}\mathscr{T}_{mn}\Phi_{m}^{\ast} \:, \label{26.6.17}
\end{equation}
其中$\,\epsilon\,$是实的无限小参量, $\mathscr{T}_{nm}\,$是厄米矩阵, 它可能是相似变换矩阵的部分\,Lie\,代数. 由于超势只依赖$\,\Phi_{n}$, 它自动在如下的扩充变换下不变
\begin{equation}
    \delta \Phi_{n} =\mi\epsilon\Lambda\sum_{m}\mathscr{T}_{nm}\Phi_{m} \:, \qquad \quad
    \delta \Phi_{n}^{\ast} =-\mi\epsilon\Lambda^{\ast}\sum_{m}\mathscr{T}_{mn}\Phi_{m}^{\ast} \:, \label{26.6.18}
\end{equation}
其中$\,\Lambda(x,\theta)\,$是超场, 为了使$\,\delta\Phi_{n}\,$是左手征的, 它必须也取成左手征的. 另一方面, 因为$\,\Lambda\neq\Lambda^{\ast}$, 诸如\,K\"{a}hler\,势这样的其它项在这些变换下一般不是不变的. 因此, 对于一般的场, 作用量的变换必须取如下的形式
\begin{equation}
    \delta I =\mi\epsilon \int \dif^{4}x\int \dif^{4}\theta\:[\Lambda-\Lambda^{\ast}]\mathscr{J}\:,\label{26.6.19}
\end{equation}
其中$\,\mathscr{J}(x,\theta)\,$是某个实超场, 称为{\kai{流超场}}. 但是, 如果场方程是成立的, 那么作用量在超场的{\kai{任何}}变分下都是驻定的, 所以积分(\ref{26.6.19})对于任何左手征超场$\,\Lambda(x,\theta)\,$都必须为零. 任何这样的$\,\Lambda\,$都可以写成$\,\Lambda=\mathscr{D}_{R}^{2}S$, 所以这意味着流超场必须满足
\begin{equation}
    \mathscr{D}_{R}^{2}\mathscr{J} = \mathscr{D}_{L}^{2}\mathscr{J} =0 \:. \label{26.6.20}
\end{equation}
即, $\mathscr{J}\,$是{\kai{线性}}超场. 正如我们在\,26.3\,节看到的, 这意味着它的分量满足
\begin{equation}
    N^{\mathscr{J}}=M^{\mathscr{J}}=\partial^{\mu}V_{\mu}^{\mathscr{J}}=0 \:, \qquad
    \lambda^{\mathscr{J}} = -\slashed{\partial}\omega^{\mathscr{J}} \:, \qquad
    D^{\mathscr{J}} = -\square C^{\mathscr{J}} \:. \label{26.6.21}
\end{equation}
这使得我们可以将$\,V\,$-分量$\,V_{\mu}^{\mathscr{J}}\,$等同为与这个对称性相联系的守恒流.

对于特殊的作用量(\ref{26.6.9}), 流超场采取如下的形式
\begin{equation}
    \mathscr{J}=\sum_{nm}\frac{\partial K(\Phi,\Phi^{\ast})}{\partial \Phi_{n}}\mathscr{T}_{nm}\Phi_{m}
    =\sum_{nm}\frac{\partial K(\Phi,\Phi^{\ast})}{\partial \Phi_{n}^{\ast}}\mathscr{T}_{mn}\Phi_{m}^{\ast}\:.\label{26.6.22}
\end{equation}
这两个式子相等就是在变换(\ref{26.6.17})下的对称性的结果. 那么, 利用场方程(\ref{26.6.15})
\begin{equation}
     \mathscr{D}_{R}^{2}\mathscr{J}=
     \sum_{nm}\biggl[\mathscr{D}_{R}^{2}\frac{\partial K(\Phi,\Phi^{\ast})}{\partial \Phi_{n}}\biggr]\,\mathscr{T}_{nm}\Phi_{m}
    =-4\sum_{nm}\frac{\partial f(\Phi)}{\partial\Phi_{n}}\mathscr{T}_{nm}\Phi_{m}\:.\label{26.6.23}
\end{equation}
由于假定超势在变换(\ref{26.6.17})下不变, 这为零. 以同样的方式, 利用$\,\mathscr{J}\,$的第二个表达式以及场方程(\ref{26.6.16}), 我们发现$\,\mathscr{D}_{L}^{2}\mathscr{J}=0$, 因此证实了守恒条件(\ref{26.6.20})

\section{超流}  \label{sec:26.7}

同任何其它整体连续对称性一样, 超对称性会导致一个守恒流.\cite{5} 超对称流的守恒和对易性质是一些算符方程, 即使在超对称自发破缺的情况下, 这些算符方程也是成立的, 因此在第\,29\,章我们在非微扰的意义下考察超对称自发破缺的理论时是有用的. 另外, 超对称流与称为{\kai{超流}}的超场的分量相关,\cite{6} 这一点在第\,31\,章我们处理超引力时具有基础的重要性.

就像我们在\,7.3\,节看到的, 拉格朗日密度在无限小变换$\chi^{\ell}\to\chi^{\ell}+\epsilon\mathscr{F}^{\ell}\,$下%
(其中$\,\chi^{\ell}\,$是一般的正则或辅助玻色场或费米场, $\mathscr{F}^{\ell}\,$是正则场和辅助场的函数.)有一个普通的整体对称性会导致一个守恒流
\[
J^{\mu}(x) \propto \sum_{\ell}\frac{\partial \mathscr{L}(x)}{\partial(\partial\chi^{\ell}(x)/\partial x^{\mu})}
\mathscr{F}^{\ell}(x)\:,
\]
它对于满足场方程的场是守恒的并生成了对称性, 也就是说正则对易关系给出
\[
\biggl[\int\dif^{3}x\,J^{0}(x)\:,\chi^{\ell}(x)\biggr] \propto \mathscr{F}^{\ell}(y) \:.
\]
由于两个原因, 超对称流的处理要稍微复杂一些. 一个原因是, 超对称性是作用量的对称性而不是拉格朗日密度或者拉格朗日量的对称性. 取而代之, 拉格朗日密度在无限小超对称变换下的变分是时空导数, 我们可以将其写成如下形式
\begin{equation}
    \delta \mathscr{L} = \sum_{\ell} \Bigl(\bar{\alpha}\,\partial_{\mu}K^{\mu}\Bigr) \:, \label{26.7.1}
\end{equation}
其中$\,K^{\mu}\,$是\,Majorana\,旋量的\,4\,-矢. 结果是, 超对称流不是通常的\,Noether\,流. Noether\,流是
\begin{equation}
    \sum_{\ell}\frac{\partial_{R}\mathscr{L}}{\partial(\partial_{\mu}\chi^{\ell})}\delta \chi^{\ell}
    \equiv -\Bigl(\bar{\alpha}\,N^{\mu}\Bigr)  \label{26.7.2}
\end{equation}
定义的\,Majorana\,旋量的\,4\,-矢\,$N^{\mu}$, 它的散度由\,Euler-Lagrange\,方程给出
\begin{align}
    \Bigl(\bar{\alpha}\,\partial_{\mu} N^{\mu}\Bigr) &= -\sum_{\ell}\frac{\partial_{R}\mathscr{L}}{\partial\chi^{\ell}}
    \delta\chi^{\ell}-\sum_{\ell}\frac{\partial_{R}\mathscr{L}}{\partial(\partial_{\mu}\chi^{\ell})}
    \partial_{\mu}\delta \chi^{\ell} \nonumber \\
    &= -\delta \mathscr{L} \:. \label{26.7.3}
\end{align}
(这里的$\,\partial_{R}\,$是右导数, 它定义成在微分之前将要微分的费米变量移至右边的.) 相反, 我们必须要将超对称流定义成
\begin{equation}
    S^{\mu} \equiv N^{\mu}+K^{\mu} \:, \label{26.7.4}
\end{equation}
方程(\ref{26.7.1})和(\ref{26.7.3})告诉我们它{\kai{是}}守恒的:
\begin{equation}
    \partial_{\mu}S^{\mu} = 0 \:. \label{26.7.5}
\end{equation}

第二个复杂性是, 正则场$\,\chi^{\ell}\,$在超对称变换下的变化$\,\delta\chi^{\ell}\,$不仅是正则场的函数, 还是它们的正则共轭的函数. 例如, 方程(\ref{26.3.15})表明手征标量超场的$\,\psi\,$-分量的变化包含$\,\phi\,$-分量%
的时间导数. 结果是, Noether\,荷$\,\int\dif^{3}x\,N^{0}\,$与一般正则场的对易子并不给出那个场的超对称变换. 幸运的是, 这一复杂性被第一个复杂性抵消了: 当$\,\int\dif^{3}x\,K^{0}\,$和$\,\int\dif^{3}x\,N^{0}\,$与场的对易子被考虑在内时, 算符 $\int\dif^{3}x\,S^{0}\,$确实生成了超对称变换,\footnote{对于以这种方式构建的流, 这是一般结果. 例如, 考察依赖一组正则变量$\,q^{n}\,$和它们时间导数$\,\dot{q}^{n}\,$的拉格朗日量$\,L\,$(不是拉格朗日密度), 其中没有任何一类约束. 在量子场论中, 指标$\,n\,$包含空间坐标以及离散自旋指标和种类指标, 且有$\,L=\int\dif^{3}x\,\mathscr{L}$. 我们这里假定了拉格朗日密度在某个无限小变换$\,\delta\,$下在相差一个时空导数的意义下不变, 这意味着$\,\delta L\,$是某个泛函$\,F\,$的时间导数. 即,
\[
\sum_{n}\frac{\partial L}{\partial q^{n}}\delta q^{n} +
\sum_{n}\frac{\partial L}{\partial\dot{q}^{n}}\delta\dot{q}^{n}=\frac{\dif}{\dif t}F\:.
\]
利用正则运动方程, 这可以写成守恒律$\,\dot{Q}=0$, 其中守恒荷是
\[
Q=-\sum_{n}\frac{\partial L}{\partial\dot{q}^{n}}\delta q^{n} + F \:.
\]
在我们这里的情况中, $Q=\int\dif^{3}x\,[N^{0}+K^{0}]$. 我们假定通常的未约束对易关系
\[
\biggl[\frac{\partial L}{\partial\dot{q}^{n}}, q^{m}\biggr]=-\mi\,\delta_{n}^{m}\:,\qquad
\Bigl[q^{n},q^{m}\Bigr]=0 \:,
\]
并找到对易子
\[
\Bigl[Q,q^{m}\Bigr]=\mi\,\delta q^{m}-\sum_{nl}\frac{\partial L}{\partial\dot{q}^{l}}
\frac{\partial\delta q^{l}}{\partial \dot{q}^{n}}\,\Bigl[\dot{q}^{n},q^{m}\Bigr]
+\sum_{n}\frac{\partial F}{\partial \dot{q}^{n}}\,\Bigl[\dot{q}^{n},q^{m}\Bigr] \:.
\]
为了计算第二项和第三项, 我们注意到二阶时间导数$\,\ddot{q}^{n}\,$以线性的方式出现在不变性条件中, 所以它们的系数必须互相匹配: 即使没使用运动方程, 我们也有
\[
\sum_{l}\frac{\partial L}{\partial\dot{q}^{l}}
\frac{\partial\delta q^{l}}{\partial \dot{q}^{n}}=\frac{\partial F}{\partial \dot{q}^{n}} \:.
\]
对易子中的第二项和第三项因此抵消了, 留给我们想要的结果
\[
\Bigl[Q,q^{m}\Bigr]=\mi\,\delta q^{m} \:.
\]
取时间导数也给出
\[
\Bigl[Q,\dot{q}^{m}\Bigr]=\mi\,\delta\dot{q}^{m} \:.
\]
这个结果已经被扩展到有约束的理论中了.\cite{7}
} 也就是说
\begin{equation}
    \bigg[\int\dif^{3}x\,\Bigl(\bar{\alpha}\,S^{0}\Bigr)\:,\chi^{\ell}\biggr]=\mi\,\delta\chi^{\ell} \:,\label{26.7.6}
\end{equation}
与方程(\ref{26.2.1})和(\ref{26.2.8})一致.

例如, 我们可以在左手征超场$\,\Phi_{n}\,$的一般可重整理论中导出超对称流的显式公式, 这个公式可以用来检验它确实生成了超对称变换, 即方程(\ref{26.7.6}). 这个理论的拉格朗日量(\ref{26.4.7})可以写成如下形式
\begin{align}
    \mathscr{L}&=\sum_{n}\Biggl[-\partial_{\mu}\phi_{n}^{\ast}\partial^{\mu}\phi_{n}
    -\frac{1}{2}\Bigl(\overline{\psi_{nL}}\,\gamma^{\mu}\partial_{\mu}\,\psi_{nL}\Bigr)
    -\frac{1}{2}\Bigl(\overline{\psi_{nR}}\,\gamma^{\mu}\partial_{\mu}\,\psi_{nR}\Bigr)\Biggr] \nonumber \\
    &\quad + \text{非导数项}\:.     \label{26.7.7}
\end{align}
利用变换规则(\ref{26.3.15}), (\ref{26.3.17}), (\ref{26.3.18})和(\ref{26.3.20})(其中$\,\tilde{\phi}=\phi^{\ast}\,$), 方程(\ref{26.7.2})定义的\,Noether\,\\流是
\begin{align}
    N^{\mu} &= \frac{1}{\sqrt{2}}\sum_{n}\Bigl[ 2\,(\partial^{\mu}\phi_{n}^{\ast})\,\psi_{nL}
    +2\,(\partial^{\mu}\phi_{n})\,\psi_{nR} + (\slashed{\partial}\phi_{n})\,\gamma^{\mu}\psi_{nR}
    +(\slashed{\partial}\phi_{n}^{\ast})\,\gamma^{\mu}\psi_{nL} \nonumber \\
    &\quad -\mathscr{F}_{n}\,\gamma^{\mu}\psi_{nR}-\mathscr{F}_{n}^{\ast}\,\gamma^{\mu}\psi_{nL}\Bigr]\:.\label{26.7.8}
\end{align}
我们可以直接计算拉格朗日密度的变化, 另一种更简单的方法是, 注意到$\,D\,$-项和$\,\mathscr{F}\,$-项在超对称变换下分别由方程(\ref{26.2.17})和(\ref{26.3.16})给出. 无论以哪种方法, 我们发现方程(\ref{26.7.1})中的流$\,K^{\mu}\,$是
\begin{align}
    K^{\mu} &= \frac{1}{\sqrt{2}}\sum_{n}\gamma^{\mu}\Biggl[-(\slashed{\partial}\phi_{n})\psi_{nR}
    -(\slashed{\partial}\phi_{n}^{\ast})\psi_{nL}+\mathscr{F}_{n}^{\ast}\psi_{nL}
    +\mathscr{F}_{n}\psi_{nR} \nonumber \\
    &\quad+2\biggl(\frac{\partial f(\phi)}{\partial \phi_{n}}\biggr) \,\psi_{nL}
    +2\biggl(\frac{\partial f(\phi)}{\partial \phi_{n}}\biggr)^{\ast} \,\psi_{nR}\Biggr] \:.\label{26.7.9}
\end{align}
将(\ref{26.7.8})和(\ref{26.7.9})加起来就给出了这类理论的超对称流
\begin{equation}
    S^{\mu} = \sqrt{2}\sum_{n}\Biggl[(\slashed{\partial}\phi_{n})\gamma^{\mu}\psi_{nR}
    +(\slashed{\partial}\phi_{n}^{\ast})\gamma^{\mu}\psi_{nL}
    +\biggl(\frac{\partial f(\phi)}{\partial \phi_{n}}\biggr) \,\gamma^{\mu}\psi_{nL}
    +\biggl(\frac{\partial f(\phi)}{\partial \phi_{n}}\biggr)^{\ast} \,\gamma^{\mu}\psi_{nR}\Biggr]\:.\label{26.7.10}
\end{equation}
这样一来, 用正则对易关系和反对易关系证实$\,\int\dif^{3}x\,S^{0}\,$满足对易关系(\ref{26.7.6})就是直接的.

对称流有另外一种定义, 即按照物质作用量对定域对称变换的响应来定义, 当相应的对称性被``规范化''后, 这个定义特别有用, 而当我们在第\,31\,章转向超引力理论时, 超对称就会变成这样的情况. 在没有超引力场的情况下, 作用量在定域超对称变换下不是不变的. 如果我们做这样一个带有时空相关参量$\,\alpha(x)\,$的变换, 为了使得作用量的变化在$\,\alpha(x)\,$是常数时为零, 它必须(即使场方程没有被满足)采取如下的形式
\begin{equation}
    \delta I = -\int \dif^{4}x\:\Big((\partial_{\mu}\bar{\alpha}(x))\,S^{\mu}(x)\Bigr) \:, \label{26.7.11}
\end{equation}
其中$\,S^{\mu}(x)\,$是\,Majorana\,旋量算符系数的\,4\,-矢. 由于在我们将整体超对称变换推广至定域变换后, 一般情况下, 对于场$\,\chi\,$在定域对称变换的变化$\,\delta\chi$, 我们可以让它以任意的方式依赖于$\,\alpha(x)\,$的导数, 所以这并不唯一地定义$\,S^{\mu}(x)$. 然而, 有一种定义定域对称变换的方式保证了方程(\ref{26.7.11})中的系数$\,S^{\mu}(x)\,$与方程(\ref{26.7.4})定义的流相同, 而正如我们看到的, 后者生成了对称变换, 即方程(\ref{26.7.6}). 方法是指定{\kai{正则场或辅助场$\,\chi^{\ell}\,$的超对称变换中不出现$\,\alpha(x)\,$的导数.}} 例如, 对于左手征超场的分量, 变换规则(\ref{26.3.15})---(\ref{26.3.17})的定域版本是
\begin{align}
    &\delta\psi_{L}(x) = \sqrt{2}\partial_{\mu}\phi(x)\,\gamma^{\mu}\,\alpha_{R}(x)\,\phi(x)
    +\sqrt{2}\mathscr{F}(x)\,\alpha_{L}(x) \:, \label{26.7.12}  \\
    &\delta \mathscr{F}(x) =\sqrt{2}\Bigl(\overline{\alpha_{L}}(x)\,\slashed{\partial}\psi_{L}(x)\Bigr)\:,
    \label{26.7.13} \\
    &\delta \phi(x) = \sqrt{2}\Bigl(\overline{\alpha_{R}}(x)\,\psi_{L}(x)\Bigr)\:. \label{26.7.14}
\end{align}
方程(\ref{26.3.21})表明超场可以表示成它在$\,x_{+}^{\mu}\,$没有导数的分量场, 所以这个超场的变换规则可以表示成
\begin{equation}
    \delta \Phi(x,\theta) = \Bigl(\bar{\alpha}(x_{+})\,\mathscr{Q}\Bigr) \Phi(x,\theta) \:, \label{26.7.15}
\end{equation}
其中$\,\mathscr{Q}\,$是算符(\ref{26.2.2}).

当定域对称变换以这种方式定义后, 它们诱导出的作用量的变化由两项组成. 首先, 尽管正则场在超对称变换下的变分不包含$\,\alpha(x)\,$的导数, 但是正则场导数的变分确实包含$\,\alpha(x)\,$的导数. 除了要将$\,\bar{\alpha}\,$换成$\,\partial_{\mu}\bar{\alpha}\,$, 它产生的拉格朗日密度的变化与方程(\ref{26.7.2})相同:
\[
    \delta_{1}I = -\int \dif^{4}x\:\Bigl([\partial_{\mu}\bar{\alpha}(x)]\,N^{\mu}(x)\Bigr)\:.
\]
作用量变化中的第二项源于如下的事实: 即使在不含$\,\alpha(x)\,$导数的那部分超对称变换下, 拉格朗日密度也不是不变的. 根据方程(\ref{26.7.1}), 它产生的作用量的变化是
\[
\delta_{2}I = \int \dif^{4}x\:\Bigl(\bar{\alpha}(x)\,\partial_{\mu}K^{\mu}(x)\Bigr)
=-\int\dif^{4}x\:\Bigl((\partial_{\mu}\bar{\alpha}(x))\,K^{\mu}(x)\Bigr) \:.
\]
将$\,\delta_{1}I\,$和$\,\delta_{2}I\,$加起来就给出作用量形式为(\ref{26.7.11})的总变化, 其中$\,S^{\mu}(x)\,$由方程(\ref{26.7.4})给出, 这正是所要证明的.

即使在这样指定分量场的变换性质后, 超对称流$\,S^{\mu}(x)\,$也没有被方程(\ref{26.7.11})唯一地指定, 这是因为我们总可以引入修正流
\begin{equation}
    S^{\mu}_{\text{new}}= S^{\mu}+\partial_{\nu}A^{\mu\nu} \:, \label{26.7.16}
\end{equation}
其中$\,A^{\mu\nu}=-A^{\nu\mu}\,$是\,Majorana\,旋量的任意反对称张量. 无论场方程是否被满足, $\partial_{\nu}A^{\mu\nu}\,$这一项总是守恒的, 并且它的时间分量是空间导数, 所以$\,\int\dif^{3}x\,S^{0}_{\text{new}}=\int\dif^{3}x\,S^{0}$, 这使得方程(\ref{26.7.6})保持不变.

事实上$\,A^{\mu\nu}\,$有一个特殊的选择使得$\,\gamma_{\mu}S^{\mu}_{\text{new}}\,$有这样的方便特征: 它衡量了理论对标度不变性的破坏程度. 通过使用从拉格朗日密度(\ref{26.4.7})导出的\,Dirac\,方程:
\begin{equation}
    \slashed{\partial}\psi_{mL}=-\sum_{n}\Biggl(\frac{\partial^{2}f(\phi)}{\partial \phi_{m}\partial\phi_{n}}\Biggr)^{\ast} \psi_{nR} \:, \qquad
    \slashed{\partial}\psi_{mR}=-\sum_{n}\Biggl(\frac{\partial^{2}f(\phi)}{\partial \phi_{m}\partial\phi_{n}}\Biggr) \psi_{nL} \:, \label{26.7.17}
\end{equation}
直接计算给出
\begin{align*}
    \gamma_{\mu}S^{\mu} &= -2\sqrt{2}\sum_{n}\Biggl\{
    \slashed{\partial}\Bigl(\phi_{n}\psi_{nR}+\phi_{n}^{\ast}\psi_{nL}\Bigr) \\
    &\quad + \Biggl(\sum_{m}\phi_{m}\frac{\partial^{2}f(\phi)}{\partial\phi_{n}\partial\phi_{m}}
    -2\frac{\partial f(\phi)}{\partial\phi_{n}}\Biggr) \psi_{nL} \\
    &\quad + \Biggl(\sum_{m}\phi_{m}\frac{\partial^{2}f(\phi)}{\partial\phi_{n}\partial\phi_{m}}
    -2\frac{\partial f(\phi)}{\partial\phi_{n}}\Biggr)^{\ast} \psi_{nR} \Biggr\}
\end{align*}
通过引入方程(\ref{26.7.16})那样一般类型的修正超对称流, 我们可以消除掉第一项:
\begin{equation}
    S^{\mu}_{\text{new}} = S^{\mu} + \frac{\sqrt{2}}{3}[\gamma^{\mu},\gamma^{\nu}]
    \sum_{n}\partial_{\nu}\Bigl(\phi_{n}\psi_{nR}+\phi_{n}^{\ast}\psi_{nL}\Bigr) \:, \label{26.7.18}
\end{equation}
使得
\begin{align}
    \gamma_{\mu}S^{\mu}_{\text{new}} &= -2\sqrt{2} \sum_{n} \Biggl\{\Biggl(
    \sum_{m}\phi_{m}\frac{\partial^{2}f(\phi)}{\partial\phi_{n}\partial\phi_{m}}
    -2\frac{\partial f(\phi)}{\partial\phi_{n}}\Biggr)\,\psi_{nL} \nonumber \\
    &\quad+\Biggl(\sum_{m}\phi_{m}\frac{\partial^{2}f(\phi)}{\partial\phi_{n}\partial\phi_{m}}
    -2\frac{\partial f(\phi)}{\partial\phi_{n}}\Biggr)^{\ast}\,\psi_{nR}\Biggr\} \:. \label{26.7.19}
\end{align}
对于标度不变的拉格朗日密度, 即$\,f(\Phi)\,$是$\,\Phi_{n}\,$的三阶齐次多项式的拉格朗日密度, 右边为零.

我们现在转向超对称流的超对称变换性质. 可以直接验证方程(\ref{26.7.18})和(\ref{26.7.10})给出的流与一个非手征实超场$\,\Theta_{\mu}\,$%
的$\,\omega\,$-分量$\,\omega_{\mu}^{\Theta}\,$有如下的关系\footnote{这里我们引入一个将在第\,31\,章广泛使用%
的符号约定; 延续方程(\ref{26.2.10}), 任意超场$\,S(x,\theta)\,$的分量$\,C^{S}$, $\omega^{S}$, $M^{S}$, $N^{S}$, $V_{\nu}^{S}$, $\lambda^{S}\,$和$\,D^{S}\,$通过如下展开定义:
\begin{align*}
S(x,\theta) &= C^{S}(x) -\mi\Bigl(\bar{\theta}\,\gamma_{5}\,\omega^{S}(x)\Bigr)
-\frac{\mi}{2}\Bigl(\bar{\theta}\,\gamma_{5}\,\theta\Bigr)M^{S}(x)
-\frac{1}{2}\Bigl(\bar{\theta}\,\theta\Bigr)N^{S}(x) \\
&\quad +\frac{\mi}{2}\Bigl(\bar{\theta}\,\gamma_{5}\,\gamma^{\nu}\,\theta\Bigr)V_{\nu}^{S}(x)
-\mi\Bigl(\bar{\theta}\,\gamma_{5}\,\theta\Bigr)\Biggl(\bar{\theta}\Bigl[
\lambda^{S}(x)+\frac{1}{2}\,\slashed{\partial}\omega^{S}(x)\Bigr]\Biggr) \\
&\quad -\frac{1}{4}\Bigl(\bar{\theta}\,\gamma_{5}\,\theta\Bigr)^{2}\Biggl[
D^{S}(x)+\frac{1}{2}\square C^{S}(x)\Biggr] \:.
\end{align*}}
\begin{equation}
    S^{\mu}_{\text{new}} = -2\omega^{\Theta\,\mu} +2\gamma^{\mu}\gamma^{\nu}\omega^{\Theta}_{\nu}\:, \label{26.7.20}
\end{equation}
其中
\begin{equation}
    \Theta_{\mu} = \frac{\mi}{12}\sum_{n}
    \Biggl[4\Phi_{n}^{\ast}\partial_{\mu}\Phi_{n} - 4\Phi_{n}\partial_{\mu}\Phi_{n}^{\ast}
    +\Bigl((\bar{\mathscr{D}}\Phi_{n}^{\ast})\gamma_{\mu}(\mathscr{D}\Phi_{n})\Bigr)\Biggr] \:. \label{26.7.21}
\end{equation}
超场$\,\Theta^{\mu}\,$被称为{\kai{超流}}.

超流服从的守恒律包含了超对称流的守恒以及其它很多守恒律. 为了推导它, 我们可以使用反对易关系(\ref{26.2.30})写下\footnote{要注意, $\bar{\mathscr{D}}_{L}\,$和$\,\bar{\mathscr{D}}_{R}\,$是协变伴随$\,\bar{\mathscr{D}}\,$的左手分量和右手分量, 而不是$\,\mathscr{D}_{L}\,$和$\,\mathscr{D}_{R}\,$的协变伴随$\,\overline{\mathscr{D}_{L}}\,$和%
$\,\overline{\mathscr{D}_{R}}$.}
\[
[\mathscr{D}_{R},(\bar{\mathscr{D}}_{L}\mathscr{D}_{L})] = -4\,\slashed{\partial}\mathscr{D}_{L}\:.
\]
加上手征条件$\,\mathscr{D}_{R}\Phi_{n}=\mathscr{D}_{L}\Phi_{n}^{\ast}=0$, 这给出
\[
\gamma^{\mu}\mathscr{D}_{L}\sum_{n}\Bigl[\Phi_{n}^{\ast}\partial_{\mu}\Phi_{n}
-\Phi_{n}\partial_{\mu}\Phi_{n}^{\ast}\Bigr] = -\tfrac{1}{4}\sum_{n}\Phi_{n}^{\ast}
\mathscr{D}_{R}\Bigl(\bar{\mathscr{D}}_{L}\mathscr{D}_{L}\Bigr)\Phi_{n}
-\sum_{n}(\slashed{\partial}\Phi_{n}^{\ast})\,\mathscr{D}_{L}\Phi_{n}
\]
和
\[
\gamma^{\mu}\mathscr{D}_{L}\sum_{n}\Bigl((\bar{\mathscr{D}}\Phi_{n}^{\ast})\gamma_{\mu}(\mathscr{D}\Phi_{n})\Bigr)
=4\sum_{n}(\slashed{\partial}\Phi_{n}^{\ast})\mathscr{D}\Phi_{n}
+2\sum_{n}\mathscr{D}\Phi_{n}^{\ast}\Bigl(\bar{\mathscr{D}}_{L}\mathscr{D}_{L}\Bigr)\Phi_{n} \:,
\]
使得超场(\ref{26.7.21})满足
\begin{equation}
    \gamma_{\mu}\mathscr{D}_{L}\Theta^{\mu} = \tfrac{1}{6}\,\mi\sum_{n}(\mathscr{D}_{R}\Phi_{n}^{\ast})
    \Bigl(\bar{\mathscr{D}}_{L}\mathscr{D}_{L}\Bigr)\Phi_{n}
    -\tfrac{1}{12}\,\mi\sum_{n}\Phi_{n}^{\ast}\mathscr{D}_{R}\Bigl(\bar{\mathscr{D}}_{L}\mathscr{D}_{L}\Bigr)\Phi_{n}\:.
    \label{26.7.22}
\end{equation}
我们在\,26.6\,节看到, 拉格朗日密度(\ref{26.4.7})的场方程可以表示成如下形式
\begin{equation}
    \Bigl(\bar{\mathscr{D}}_{L}\mathscr{D}_{L}\Bigr)\Phi_{n} = -4\biggl(\frac{\partial f(\Phi)}{\partial\Phi_{n}}\biggr)^{\ast} \:. \label{26.7.23}
\end{equation}
在方程(\ref{26.7.22})中使用上式最后给出
\begin{align}
    \gamma^{\mu}\mathscr{D}_{L}\Theta_{\mu} &= -\frac{2}{3}\mi\sum_{n}(\mathscr{D}_{R}\Phi_{n}^{\ast})\,
    \biggl(\frac{\partial f}{\partial\Phi_{n}}\biggr)^{\ast} + \frac{1}{3}\mi\sum_{n}\Phi_{n}^{\ast}
    \mathscr{D}_{R}\biggl(\frac{\partial f}{\partial\Phi_{n}}\biggr)^{\ast} \nonumber \\
    &= \frac{1}{3}\,\mi\,\mathscr{D}_{R}\Biggl[\sum_{n}\Phi_{n}\frac{\partial f(\Phi)}{\partial\Phi_{n}}
    -3\,f(\Phi)\Biggr]^{\ast} \:. \label{26.7.24}
\end{align}
方程(\ref{26.7.24})的厄米共轭是
\begin{equation}
    \gamma^{\mu}\mathscr{D}_{R}\Theta_{\mu}=-\frac{1}{3}\,\mi\,\mathscr{D}_{L}
    \Biggl[\sum_{n}\Phi_{n}\frac{\partial f(\Phi)}{\partial\Phi_{n}}-3\,f(\Phi)\Biggr] \:. \label{26.7.25}
\end{equation}
这样, 它与方程(\ref{26.7.24})给出守恒流
\begin{equation}
    \gamma^{\mu}\mathscr{D}\Theta_{\mu} = \mathscr{D}X \:, \label{26.7.26}
\end{equation}
其中$\,X\,$是一个实手征超场, 它在这类理论中(在相差一个额外的常数的意义下)给定为
\begin{equation}
    X=\frac{2}{3}\operatorname{Im}\Biggl[\sum_{n}\Phi_{n}\frac{\partial f(\Phi)}{\partial\Phi_{n}}-3\,f(\Phi)\Biggr]\:.
    \label{26.7.27}
\end{equation}

尽管这里仅对手征超场的可重整理论做了推导, 我们可以预期守恒律(\ref{26.7.27})在更一般的情况下也会成立, 当然, 由于它还包含其它守恒律, $X\,$不一定由方程(\ref{26.7.27})给出. (31.4\,节将会给出$\,X\,$的一个推广公式.) 为了推导这些关系, 我们必须用方程(\ref{26.2.10})将$\,\Theta_{\mu}\,$表示成$\,C_{\mu}^{\Theta}$, $\omega_{\mu}^{\Theta}\,$等分量, 并用方程(\ref{26.3.9})将手征超场$\,X\,$表示成$\,A^{X}$, $\psi^{X}\,$等分量. 在方程(\ref{26.A.9}), (\ref{26.A.16}) (\ref{26.A.17})和 Dirac\,矩阵恒等式的帮助下
\begin{equation}
    [\gamma^{\rho},\gamma^{\sigma}] = -\tfrac{1}{2}\mi\,\epsilon^{\rho\sigma\mu\nu}\,\gamma_{5}\,
    [\gamma_{\mu},\gamma_{\nu}] \:, \label{26.7.28}
\end{equation}
\begin{equation}
    \gamma^{\mu}\gamma^{\rho}\gamma^{\sigma}=\eta^{\mu\rho}\gamma^{\nu}-\eta^{\mu\nu}\gamma^{\rho}
    +\eta^{\nu\rho}\gamma^{\mu} + \mi\,\gamma_{5}\,\epsilon^{\mu\nu\rho\sigma}\gamma_{\sigma}\:, \label{26.7.29}
\end{equation}
我们可以将方程(\ref{26.7.26})两边展到如下各项上
\begin{align*}
    &1\:,\:\: \theta\:,\:\: \gamma_{5}\theta\:,\:\: \gamma^{\nu}\theta\:,\:\:\gamma_{5}\gamma^{\nu}\theta\:,\:\:
    \gamma_{5}[\gamma^{\mu},\gamma^{\nu}]\theta \:, \\
    &\Bigl(\bar{\theta}\theta\Bigr)\:,\:\:\Bigl(\bar{\theta}\gamma_{5}\theta\Bigr)\:,\:\:
    \Bigl(\bar{\theta}\gamma_{5}\gamma_{\nu}\theta\Bigr)\:, \\
    &\theta\,\Bigl(\bar{\theta}\gamma_{5}\theta\Bigr)\:,\:\:
    \gamma_{5}\theta\,\Bigl(\bar{\theta}\gamma_{5}\theta\Bigr)\:,\:\:
    \gamma^{\nu}\theta\,\Bigl(\bar{\theta}\gamma_{5}\theta\Bigr)\:,  \\
    &\gamma^{\nu}\gamma_{5}\theta\,\Bigl(\bar{\theta}\gamma_{5}\theta\Bigr)\:,\:\:
    [\gamma^{\rho},\gamma^{\sigma}]\theta\,\Bigl(\bar{\theta}\gamma_{5}\theta\Bigr)\:,\:\:
    \Bigl(\bar{\theta}\gamma_{5}\theta\Bigr)^{2}\:.
\end{align*}
分别比对$\,1$, $\theta$, $\gamma_{5}\theta$, $\gamma^{\nu}\theta$, $\gamma_{5}\gamma^{\nu}\theta$, $\gamma_{5}[\gamma^{\mu},\gamma^{\nu}]\theta\,$的系数, 这给出结果\footnote{注意, $V_{\mu\nu}^{\Theta}\,$是$\,\Theta_{\mu}\,$的$\,V_{\nu}\,$-分量, 不是$\,\Theta_{\nu}\,$的$\,V_{\mu}\,$-分量.}
\begin{align}
    \psi^{X} &= -\mi\gamma_{5}\gamma^{\mu}\omega_{\mu}^{\Theta} \:, \label{26.7.30} \\
    F^{X} &= \partial^{\mu}C_{\mu}^{\Theta} \:, \label{26.7.31} \\
    G^{X} &= (V^{\Theta})\indices{^\mu_\mu} \:, \label{26.7.32} \\
    \partial_{\mu}A^{X} &= -N_{\mu}^{\Theta} \:, \label{26.7.33} \\
    \partial_{\mu}B^{X} &= M_{\mu}^{\Theta} \:, \label{26.7.34}  \\
    0 &= V^{\Theta}_{\mu\nu}-V^{\Theta}_{\nu\mu}+
    \epsilon_{\mu\nu\rho\sigma}\partial^{\sigma}C^{\Theta\,\rho}\:.\label{26.7.35}
\end{align}
比对$\,(\bar{\theta}\theta)\,$或$\,(\bar{\theta}\gamma_{5}\theta)\,$的系数给出同一个结果:
\begin{equation}
    0=\gamma^{\mu}\,\lambda_{\mu}^{\Theta} \:, \label{26.7.36}
\end{equation}
比对$\,(\bar{\theta}\gamma_{5}\gamma^{\nu}\theta)\,$的系数给出结果
\begin{equation}
    {-}\mi\,\gamma_{5}\,[\gamma^{\nu},\slashed{\partial}]\,\psi^{X}
    =2\gamma^{\mu}\gamma^{\nu}\lambda_{\mu}^{\Theta}+
    \gamma^{\mu} \,[\gamma^{\nu},\slashed{\partial}]\,\omega_{\mu}^{\Theta} \:. \label{26.7.37}
\end{equation}
我们从方程(\ref{26.7.30}), (\ref{26.7.36})和(\ref{26.7.37})获得了超对称流(\ref{26.7.20})的守恒:
\begin{equation}
    0=\partial_{\mu}S^{\mu}_{\text{new}} = -2\partial^{\mu}\omega_{\mu}^{\Theta}
    +2\,\slashed{\partial}\gamma^{\mu}\omega_{\mu}^{\Theta} \:, \label{26.7.38}
\end{equation}
以及$\,\lambda_{\mu}^{\Theta}\,$和$\,\omega_{\mu}^{\Theta}\,$之间的关系:
\begin{equation}
    \lambda_{\nu}^{\Theta} = -\,\slashed{\partial}\omega_{\nu}^{\Theta}
    +\partial_{\nu}\gamma^{\mu}\omega_{\mu}^{\Theta} \:. \label{26.7.39}
\end{equation}
比对$\,\theta(\bar{\theta}\gamma_{5}\theta)\,$和$\,\gamma_{5}\theta(\bar{\theta}\gamma_{5}\theta)\,$的系数%
所给出的关系可以分别通过对方程(\ref{26.7.34})和(\ref{26.7.33})取散度获得. 比对$\,\gamma^{\rho}\theta(\bar{\theta}\gamma_{5}\theta)\,$的系数给出
\begin{equation}
    \partial_{\rho}G^{X} = \partial^{\mu}V_{\mu\rho}^{\Theta}+\partial^{\mu}V_{\rho\mu}^{\Theta}
    -\partial_{\rho}V^{\Theta}{}\indices{^\lambda_\lambda} \:, \label{26.7.40}
\end{equation}
结合方程(\ref{26.7.32}), 这给出守恒律
\begin{equation}
    \partial_{\mu}T^{\mu\nu} = 0 \:, \label{26.7.41}
\end{equation}
其中$\,T^{\mu\nu}\,$是对称张量
\begin{equation}
    T_{\mu\nu}\equiv -\tfrac{1}{2}V_{\mu\nu}^{\Theta} - \tfrac{1}{2}V_{\nu\mu}^{\Theta}
    +\eta_{\mu\nu}V^{\Theta}{}\indices{^\lambda_\lambda} \:. \label{26.7.42}
\end{equation}
比对$\,\gamma^{\rho}\gamma_{5}\theta(\bar{\theta}\gamma_{5}\theta)\,$的系数给出
\begin{equation}
    \partial_{\mu}F^{X} = 2D_{\mu}^{\Theta} + \square C_{\mu}^{\Theta}
    +\epsilon_{\rho\nu\sigma\mu}\partial^{\nu}V^{\Theta\,\rho\sigma} \:, \label{26.7.43}
\end{equation}
结合方程(\ref{26.7.31})和(\ref{26.7.35}), 这给出了$\,D_{\mu}^{\Theta}\,$和$\,C_{\mu}^{\Theta}\,$之间的一个关系:
\begin{equation}
    D_{\mu}^{\Theta} = -\square C_{\mu}^{\Theta} + \partial_{\mu}\partial^{\nu}C_{\nu}^{\Theta} \:. \label{26.7.44}
\end{equation}
比对$\,[\gamma^{\rho},\gamma^{\sigma}]\theta(\bar{\theta}\gamma_{5}\theta)\,$和%
$\,(\bar{\theta}\gamma_{5}\theta)^{2}\,$给出的结果可以分别从方程(\ref{26.7.34})以及%
方程(\ref{26.7.38})和(\ref{26.7.39}) 得出.

守恒的对称张量$\,T^{\mu\nu}\,$可以被视为该系统的能动量张量. 为了验证这点, 我们用方程(\ref{26.1.18})和\\(\ref{26.2.12})将$\,\omega_{\mu}^{\Theta}(x)\,$在无限小参量为$\,\alpha\,$的%
超对称变换下的变化写成
\begin{align*}
    \delta\omega_{\mu}^{\Theta} &= -\mi\Bigl[(\bar{Q}\alpha),\omega_{\mu}^{\Theta}\Bigr]
    = +\mi\Bigl[\omega_{\mu}^{\Theta},(\bar{Q}\alpha)\Bigr] \\
    &= \Bigl(-\mi\gamma_{5}\,\slashed{\partial}C_{\mu}^{\Theta}-M_{\mu}^{\Theta}
    +\mi\gamma_{5}N_{\mu}^{\Theta} +\gamma^{\nu}\,V_{\mu\nu}^{\Theta}\Bigr)\alpha \:.
\end{align*}
方程(\ref{26.7.33})---(\ref{26.7.35})使得我们可以将其变成如下形式
\[
\mi\Bigl\{\omega_{\mu}^{\Theta},\bar{Q}\Bigr\} = \tfrac{1}{2}\gamma^{\nu}(V_{\mu\nu}^{\Theta}+V_{\nu\mu}^{\Theta})
-\partial_{\mu}(B^{X}+\gamma_{5}A^{X}) - \mi\gamma_{5}\,\slashed{\partial}C_{\mu}^{\Theta}
+\tfrac{1}{2}\epsilon_{\mu\nu\kappa\sigma}\gamma^{\nu}\partial^{\kappa}C^{\Theta\,\sigma}\:.
\]
以流(\ref{26.7.20})和(\ref{26.7.42})的形式, 这是
\begin{align}
    \mi\,\{S_{\text{new}}^{\mu},\bar{Q}\} &= 2\gamma_{\nu}T^{\mu\nu}
    +2(\partial^{\mu}-\gamma^{\mu}\,\slashed{\partial})(B^{X}+\gamma_{5}A^{X})
    -\epsilon^{\mu\nu\kappa\sigma}\gamma_{\nu}\partial_{\kappa}C_{\sigma}^{\Theta} \nonumber \\
    &\quad + 2\mi\gamma_{5}\Bigl(\slashed{\partial}C^{\Theta\,\mu}
    -\gamma^{\mu}\gamma^{\lambda}\,\slashed{\partial}C_{\lambda}^{\Theta}
    -\tfrac{1}{2}\gamma^{\mu}[\slashed{\partial},\gamma^{\sigma}]C_{\sigma}^{\Theta}\Bigr)\:. \label{26.7.45}
\end{align}
当$\,\mu=0\,$时, 右边除了第一项以外的所有项都是空间导数, 所以它们在我们对空间做积分后为零, 留下
\begin{equation}
    \mi \,\biggl\{\int\dif^{3}x\,S^{0}_{\text{new}}\:,\bar{Q}\biggr\}
    =2\gamma_{\nu}\int\dif^{3}x\:T^{0\nu} \:. \label{26.7.46}
\end{equation}
我们已经定义了超对称流$\,S^{\mu}_{\text{new}}\,$以给出$\,\int\dif^{3}x\,S_{\text{new}}^{0}=Q$, 所以基本反对易关系(\ref{25.2.36})告诉我们
\begin{equation}
    \int \dif^{3}x\: T^{0\nu} = P^{\nu} \:, \label{26.7.47}
\end{equation}
加上守恒条件(\ref{26.7.41}), 这使得我们可以将$\,T^{\mu\nu}\,$等同为能动量张量.

需要注意的是我们以这种方式构建的能动量张量是{\kai{哪一个}}. 无论是直接从方程(\ref{26.7.21})出发, 还是通过考察流(\ref{26.7.18})的超对称变换, 对于手征超场的可重整理论, 我们可以计算出能动量张量$\,T^{\mu\nu}\,$是
\begin{align}
    T^{\mu\nu} &= \sum_{n}\Bigl[\partial^{\mu}\phi_{n}^{\ast}\partial^{\nu}\phi_{n}
    +\partial^{\nu}\phi_{n}^{\ast}\partial^{\mu}\phi_{n}\Bigr]
    -\eta^{\mu\nu}\sum_{n}\Biggl[\partial^{\lambda}\phi_{n}^{\ast}\partial_{\lambda}\phi_{n}
    +\biggl\lvert\frac{\partial f(\phi)}{\partial\phi_{n}}\biggr\rvert^{2}\Biggr] \nonumber \\
    &\quad + \tfrac{1}{3}(\eta^{\mu\nu}\square-\partial^{\mu}\partial^{\nu})\sum_{n}\lvert\phi_{n}\rvert^{2}
    +\cdots \:, \label{26.7.48}
\end{align}
其中省略号代表那些包含费米子的项, 我们在这里不考虑它们. 最后一项通过超对称变换与方程(\ref{26.7.18})的修正项相关, 对于没有超势的无质量自由场理论, 在这种情况下$\,\square\phi_{n}=0$, 我们看到它的效果是使得能动量张量无迹. 一个简单的计算表明, 更一般地, 对于$\,f(\phi)\,$是$\,\phi_{n}\,$的三阶齐次多项式的标度不变理论, $T^{\mu\nu}\,$也是无迹的.


超对称性也在标度不变性的破坏和\,$R$\,守恒之间附加了一个有趣的关系. 方程 (\ref{26.7.30}) --- (\ref{26.7.32})表明$\,\gamma_{\mu}S^{\mu}_{\text{new}}=6\gamma_{\mu}\omega^{\Theta\,\mu}$, $\partial^{\mu}C_{\mu}^{\Theta}\,$和$\,T\indices{^\lambda_\lambda}=2V^{\Theta}{}\indices{^\mu_\mu}\,$%
(它衡量了标度不变性的破坏)正比于手征超场$\,X\,$的分量, 所以, 如果其中一个作为算符方程(即, 不只是某个特殊的场构形)为零, 那么它们全部为零. 在这一情况下, 我们可以证明$\,C^{\Theta\,\rho}\,$正比于$\,R\,$量子数的流. 为了看到这点, 注意到方程(\ref{26.2.11})给出
\[
\delta C_{\sigma}^{\Theta} = \mi\,\Bigl[C_{\sigma}^{\Theta}\,,(\bar{\alpha}\,Q)\Bigr]
=\mi\,\Bigl(\bar{\alpha}\,\gamma_{5}\,\omega_{\sigma}^{\Theta}\Bigr) \:,
\]
这使得一般有
\begin{equation}
    \Bigl[C_{\sigma}^{\Theta}\,, Q\Bigr] =\gamma_{5}\omega_{\sigma}^{\Theta} \:. \label{26.7.49}
\end{equation}
我们已经看到, 如果$\,C_{\sigma}^{\Theta}\,$守恒, 那么$\,\gamma_{\mu}S^{\mu}=0$, 这使得方程(\ref{26.7.20})给出$\,S_{\sigma}=-2\omega_{\sigma}^{\Theta}$. 这样, 在方程(\ref{26.7.49})中令$\,\sigma=0\,$并对$\,\mathbf{x}\,$积分就给出
\begin{equation}
    \bigg[\int\dif^{3}x\,C^{\Theta\,0}\:, Q\biggr] = -\tfrac{1}{2}\gamma_{5}Q \:. \label{26.7.50}
\end{equation}
因此我们可以引入流
\begin{equation}
    \mathscr{R}^{\mu} \equiv 2\,C^{\Theta\,\mu} \:, \label{26.7.51}
\end{equation}
这个流{\kai{如果}}守恒, 那么它是量子数$\,\mathscr{R}\equiv\int\dif^{3}x\,\mathscr{R}^{0}\,$的流, 而$\,Q_{L}\,$和$\,Q_{R}\,$在这个量子数上分别湮灭值$\,+1\,$和$\,-1$. 由于$\,Q_{L}\,$和标量超场$\,\Phi\,$的对易子包含$\,\partial\Phi/\partial\theta_{L}\,$项, 这意味着$\,\theta_{L}\,$携带$\,\mathscr{R}\,$值$\,+1$, 与通常的定义一致. 如果一个理论中的超场$\,X\,$为零, 或者等价地, $T\indices{^\mu_\mu}$, $\gamma_{\mu}S^{\mu}\,$和$\,\partial_{\mu}\mathscr{R}^{\mu}\,$全部为零, 那么这个理论就在一组扩展的超对称变换下不变, 即\,25.2\,节末尾描述的超共形代数生成的变换.

在标度不变的理论中, 各种超场携带的$\,\mathscr{R}\,$量子数值被拉格朗日量的结构决定. 例如, 在手征标量超场的标度不变理论中, 超势必须是超场的三阶齐次多项式. 超势的$\,\mathscr{F}\,$-项正比于$\,\theta_{L}^{2}\,$的系数, 它有$\,\mathscr{R}\,$量子数$\,+2$, 所以超势$\,\mathscr{F}\,$-项的$\,\mathscr{R}\,$量子数是超势本身的$\,\mathscr{R}\,$量子数减二. 这样, $\mathscr{R}\,$不变性就要求我们赋予标量超场$\,\mathscr{R}\,$量子数$\,+2/3$, 这使得超场的$\,\mathscr{R}\,$量子数是$\,+2$, 而它的$\,\mathscr{F}\,$-项的$\,\mathscr{R}\,$量子数是零. 即, 标量分量$\,\phi_{n}\,$有$\,\mathscr{R}=2/3\,$而旋量分量$\,\psi_{nL}\,$(正比于$\,\theta_{L}\,$在超场中的系数)%
有$\,\mathscr{R}=-1/3$. 这可以通过从这类理论的超流(\ref{26.7.21})的\,$C$\,-项计算出流$\,\mathscr{R}^{\mu}\,$证实:
\begin{equation}
    \mathscr{R}_{\mu} = \tfrac{2}{3}\,\mi\,[\phi^{\ast}\partial_{\mu}\phi-\phi\partial_{\mu}\phi^{\ast}]
    -\tfrac{1}{6}\,\mi\,\Bigl(\bar{\psi}\gamma_{\mu}\gamma_{5}\psi\Bigr)\:. \label{26.7.52}
\end{equation}
(因为$\,\psi\,$是\,Majorana\,旋量, 第二项包含一个额外的因子$\,1/2$.)

量子修正会引入对$\,\mathscr{R}\,$不变性的破坏(通过\,Adler-Bell-Jackiw\,反常)和标度不变性的破坏%
(通过耦合常数的重整化群跑动), 但即使这些对称性被这些修正破坏了, 超对称性仍然会在这些对称性之间附加一个关系.\cite{7a} 我们会在\,29.3\,节看到这样的一个例子.

\subsection*{* * *}

守恒条件(\ref{26.7.26})并不唯一地决定超流$\,\Theta^{\mu}\,$或相应的手征超场$\,X$. 特别地, 我们可以给$\,\Theta^{\mu}\,$加上一项
\begin{equation}
    \Delta\Theta^{\mu} = \partial^{\mu}Y \:, \label{26.7.53}
\end{equation}
其中$\,Y\,$是任意的的手征超场. 那么方程(\ref{26.7.26})的左边就会有如下的变化
\[
    \gamma_{\mu}\mathscr{D}\Delta\Theta^{\mu}=\slashed{\partial}\mathscr{D}Y \:.
\]
对于左手征超场$\,Y_{L}$, 手征条件$\,\mathscr{D}_{R}Y_{L}=0\,$和反对易关系(\ref{26.2.30})给出
\begin{align*}
    \slashed{\partial}\mathscr{D}_{\alpha}Y &= -\tfrac{1}{2}
    \Bigl[\{\mathscr{D}_{L},\bar{\mathscr{D}}_{R}\}\mathscr{D}_{R}\Bigr]_{\alpha}\,Y_{L} \nonumber \\
    &=-\tfrac{1}{2}\Biggl[\mathscr{D}_{L\alpha}\Bigl(\bar{\mathscr{D}}_{R}\mathscr{D}_{R}\Bigr)Y_{L}
    +\sum_{\beta}\bar{\mathscr{D}}_{L\beta}\mathscr{D}_{R\alpha}\mathscr{D}_{L\beta}Y_{L}\Biggr] \:.
\end{align*}
上面展开右边第二项的$\,\bar{\mathscr{D}}_{L\beta}\,$中的矩阵$\,\epsilon\gamma_{5}\,$可以移至最后一个算符%
$\,\mathscr{D}_{L\beta}$, 这使得守恒条件和反对易关系给出
\[
\sum_{\beta}\bar{\mathscr{D}}_{L\beta}\mathscr{D}_{R\alpha}\mathscr{D}_{L\beta}Y_{L}
=-\sum_{\beta}\mathscr{D}_{L\beta}\mathscr{D}_{R\alpha}\bar{\mathscr{D}}_{L\beta}Y_{L}
=2(\slashed{\partial}\mathscr{D})_{\alpha}Y_{L} \:,
\]
以及随之的
\[
\slashed{\partial}\mathscr{D}Y_{L}=-\tfrac{1}{2}\Bigl[\mathscr{D}\Bigl(\bar{\mathscr{D}}\mathscr{D}\Bigr)Y_{L}
+2\slashed{\partial}\mathscr{D}Y_{L}\Bigr]
= -\tfrac{1}{4}\mathscr{D}\Bigl(\bar{\mathscr{D}}\mathscr{D}\Bigr)Y_{L} \:.
\]
对于任何右手征超场可以用相同的方法推出相同的结果, 因此它对于左手征超场和右手征超场的任意和$\,Y\,$也是成立的
\begin{equation}
    \gamma_{\mu}\mathscr{D}\Delta\Theta^{\mu} = \slashed{\partial}\mathscr{D}Y
    =-\tfrac{1}{4}\mathscr{D}\Bigl(\bar{\mathscr{D}}\mathscr{D}\Bigr) Y \:. \label{26.7.54}
\end{equation}
这与守恒条件(\ref{26.7.26})是同一形式, 而相应的手征超场$\,X\,$多出了如下的手征超场
\begin{equation}
    \Delta X = -\tfrac{1}{4}\bigl(\bar{\mathscr{D}}\mathscr{D}\bigr)Y \:. \label{26.7.55}
\end{equation}
很容易验证给$\,\Theta^{\mu}\,$加上$\,\Delta\Theta^{\mu}\,$对$\,T^{\mu0}\,$和$\,S^{0}_{\text{new}}\,$的改变%
仅是空间导数, 因此并不会改变能动量\,4\,-矢$\,P^{\mu}$ 和超荷$\,Q$.

我们在\,26.6\,节看到, 任何手征超场$\,X\,$都可以表示成$\,X=(\bar{\mathscr{D}}\mathscr{D})S\,$的形式, 因此, 通过给$\,\Theta^{\mu}\,$加上形如(\ref{26.7.55})且其中$\,Y=4S\,$的一项, $X\,$可以被消掉. 但一般而言, $S\,$和以这种方式构建的新$\,\Theta^{\mu}$ 将不是定域的. 从我们在第\,22\,章对三角反常的经验而言, 这一情况并不陌生------我们在里那里看到, 尽管总能构造出加在拉格朗日密度上的项使得反常被抵消, 但这些项一般不是定域的, 因此必须从拉格朗日密度中排除出去. 存在可以表示成$\,(\bar{\mathscr{D}}\mathscr{D})S\,$且$\,S\,$定域的手征超场, 因此, 如果它出现连带的手征超场$\,X\,$中, 它可以通过给$\,\Theta^{\mu}\,$加上形如(\ref{26.7.53})的定域项被抵消掉. 例如, 因为方程(\ref{26.6.15})和(\ref{26.6.16})表明$\,(\bar{\mathscr{D}}\mathscr{D})%
\operatorname{Re}(k^{ast}\Phi)=4\operatorname{Re}(k\partial f(\Phi)/\partial\Phi)$, 这样的项包括像 $\operatorname{Re}(k\partial f(\Phi)/\partial\Phi)$ 这样的项, 其中$\,k\,$是任意的复常数. 但是一般而言, 这种方法对$\,X\,$造成的变化是相当有限的.



\section[一般\,K\"{a}hler\,势]{一般\,K\"{a}hler\,势\footnote{本节有些脱离本书的发展主线, 可以在第一次阅读时跳过.}} \label{sec:26.8}


在几种情况下我们必须要考察有一般形式(\ref{26.3.30})的不可重整拉格朗日密度
\begin{equation}
\mathscr{L} = 2\operatorname{Re}\Bigl[f(\Phi)\Bigr]_{\mathscr{F}}
+\tfrac{1}{2}\Bigl[K(\Phi ,\Phi^{\ast})\Bigr]_{D} \:,  \label{26.8.1}
\end{equation}%
其中超势$\,f\,$是左手征超场$\,\Phi_{n}\,$的函数但不是它们导数的函数, 而\,K\"{a}hler\,势$\,K\,$是$\,\Phi_{n}\,$和$\,\Phi_{n}^{\ast}\,$的函数但不是它们导数的函数.

在对称性排除掉任何可重整相互作用或者可重整相互作用碰巧都很小的有效场论中, 这样的情况会出现. 对于导数个数, 费米场个数和任何小耦合常数个数的某个组合, 利用值最小的拉格朗日量通常可以从树图算出低能的散射振幅. 在\,19.5\,节, 我们检验过这种没有可重整耦合的有效场论, 它包含核子和软\,$\pi$\,子. 21.4\,节讨论的动力学破缺的规范场论提供了可重整耦合都很小的那类理论的例子. 在对称性不允许有超势或者超势由于某个原因非常小的超对称理论中, 这一情况也会出现. 当我们在\,29.5\,节考察阿贝尔规范超场和规范中性手征标量超场的$\,N=2\,$扩充超对称理论时, 我们会遇到这样的例子. 我们将会那里证明, 通过使用形如(\ref{26.8.1})的拉格朗日密度, 其中$\,f=0\,$且$\,K\,$仅是$\,\Phi_{n}\,$和$\,\Phi_{n}^{\ast}\,$的函数而不是它们导数的函数, 树图生成了这类理论中的低能散射振幅. 在有效场论中, 如果一些标量场与潜在理论的基本能标处在同一量级, 即使所有其它场的值和所有能量得要小的多, 这是引入对$\,\Phi_{n}\,$和$\,\Phi_{n}^{\ast}\,$依赖方式任意但不依赖它们导数的\,K\"{a}hler\,势是特别重要的. 例如, 非常有趣的, 它会与引力传递超对称破缺的理论相关联, 这将在\,31.6\,节进行讨论.

我们来考察如何将拉格朗日密度(\ref{26.8.1})表示成分量场. 在推导方程(\ref{26.4.4})时, 我们没有使用$\,f(\Phi)\,$是{\kai{三次}}多项式的假定, 所以它依旧给出任意超势给拉格朗日量贡献的$\,\mathscr{F}\,$-项. 为了推导$\,D\,$-项, 我们注意到\,K\"{a}hler\,势中$\,\theta\,$的四阶项是
\begin{align}
K(\Phi,\Phi^{\ast})_{\theta^{4}} &= -\frac{1}{8} \Bigl( \bar{\theta}\gamma_{5}\theta\Bigr)^{2}
\sum_{n}\biggl[\frac{\partial K(\phi,\phi^{\ast})}{\partial \phi_{n}}\square\phi_{n}
+\frac{\partial K(\phi,\phi^{\ast})}{\partial \phi_{n}^{\ast}}\square\phi_{n}^{\ast}\biggr]
\nonumber \\
&\quad +\sum_{nm}\frac{\partial^{2}K(\phi,\phi^{\ast})}{\partial\phi_{n}\partial\phi_{m}^{\ast}}
\Bigl(\bar{\theta}\gamma_{5}\theta\Bigr) \Biggl[\Bigl(\bar{\theta}\psi_{mR}\Bigr)\,
\Bigl(\bar{\theta}\,\slashed{\partial}\psi_{nL}\Bigr)   \nonumber \\
&\quad\qquad -\Bigl(\bar{\theta}\psi_{nL}\Bigr)\,
\Bigl(\bar{\theta}\,\slashed{\partial}\psi_{mR}\Bigr)\Biggr]  \nonumber \\
&\quad +2\operatorname{Re}\sum_{nml}
\frac{\partial^{3}K(\phi,\phi^{\ast})}{\partial\phi_{n}\partial\phi_{m}\partial\phi_{l}^{\ast}}
\Bigl(\theta_{L}^{\mathrm{T}}\epsilon\psi_{nL}\Bigr) \Bigl(\theta_{L}^{\mathrm{T}}\epsilon\psi_{mL}\Bigr)
\Bigl(\theta_{L}^{\mathrm{T}}\epsilon\theta_{L}\Bigr)^{\ast} \mathscr{F}_{l}^{\ast}  \nonumber \\
&\quad +2\operatorname{Re}\sum_{nml}
\frac{\partial^{3}K(\phi,\phi^{\ast})}{\partial\phi_{n}\partial\phi_{m}\partial\phi_{l}^{\ast}}
\,\Bigl(\bar{\theta}\psi_{mL}\Bigr)\Bigl(\bar{\theta}\psi_{lR}\Bigr)
\Bigl(\bar{\theta}\gamma_{5}\gamma_{\mu}\theta\Bigr) \partial^{\mu}\phi_{n}  \nonumber \\
&\quad+\sum_{nmlk}
\frac{\partial^{4}K(\phi,\phi^{\ast})}{\partial\phi_{n}\partial\phi_{m}\partial\phi_{k}^{\ast}\partial\phi_{l}^{\ast}}
\,\Bigl(\bar{\theta}\psi_{nL}\Bigr)\,\Bigl(\bar{\theta}\psi_{mL}\Bigr)\,
\Bigl(\bar{\theta}\psi _{lR}\Bigr)\, \Bigl(\bar{\theta}\psi_{kR}\Bigr)  \nonumber \\
&\quad-\frac{1}{4}\sum_{nm}\frac{\partial^{2}K(\phi,\phi^{\ast})}{\partial\phi_{n}\partial\phi_{m}^{\ast}}
\mathscr{F}_{n}\mathscr{F}_{m}^{\ast}\,\Bigl(\bar{\theta}(1+\gamma_{5})\theta\Bigr)
\Bigl(\bar{\theta}(1-\gamma_{5})\theta\Bigr)   \nonumber \\
&\quad+\frac{1}{4}\Bigl(\bar{\theta}\gamma_{5}\gamma^{\mu}\theta\Bigr)
\Bigl(\bar{\theta}\gamma_{5}\gamma^{\nu}\theta\Bigr) \sum_{mn}\Biggl[
-\frac{\partial^{2}K(\phi,\phi^{\ast})}{\partial\phi_{n}\partial\phi_{m}^{\ast}}
\partial_{\mu}\phi_{n}\partial_{\nu}\phi_{m}^{\ast}  \nonumber \\
&\quad+\frac{1}{2}\frac{\partial^{2}K(\phi,\phi^{\ast})}{\partial\phi_{n}\partial\phi_{m}}
\partial_{\mu}\phi_{n}\partial_{\nu}\phi_{m}
+\frac{1}{2}\frac{\partial^{2}K(\phi,\phi^{\ast})}{\partial\phi_{n}^{\ast}\partial\phi_{m}^{\ast}}
\partial_{\mu}\phi_{n}^{\ast}\partial_{\nu}\phi_{m}^{\ast}\Biggr]\:.  \label{26.8.2}
\end{align}%
我们可以再次使用方程(\ref{26.A.18}), (\ref{26.A.19})和方程(\ref{26.A.9})将这一展开中对$\,\theta\,$的依赖写成一个总因子%
$\,(\bar{\theta}\gamma_{5}\theta)^{2}$, 并发现
\begin{align}
K(\Phi,\Phi^{\ast})_{\theta^{4}} &=\frac{1}{4}\Bigl(\bar{\theta}\gamma_{5}\theta\Bigr)^{2}
\Biggl\{-\frac{1}{2}\sum_{n}\frac{\partial K(\phi,\phi^{\ast})}{\partial\phi_{n}}\square\phi_{n}
-\frac{1}{2}\sum_{n}\frac{\partial K(\phi,\phi^{\ast})}{\partial \phi_{n}^{\ast}}\square\phi_{n}^{\ast}\nonumber \\
&\quad+\sum_{nm}\frac{\partial^{2}K(\phi,\phi^{\ast})}{\partial \phi_{n}\partial \phi_{m}^{\ast}}
\Bigl[ \Bigl(\overline{\psi_{m}}\,\slashed{\partial}\psi_{nL}\Bigr)
+\Bigl(\overline{\psi_{n}}\,\slashed{\partial}\psi_{mR}\Bigr)
-2\mathscr{F}_{n}\mathscr{F}_{m}^{\ast}\Bigr]   \nonumber \\
&\quad +2\operatorname{Re} \sum_{nml}
\frac{\partial^{3}K(\phi,\phi^{\ast})}{\partial\phi_{n}\partial\phi_{m}\partial\phi_{l}^{\ast}}
\Bigl(\overline{\psi_{n}}\psi_{mL}\Bigr)\mathscr{F}_{l}^{\ast}  \nonumber \\
&\quad-2\operatorname{Re}\sum_{nml}
\frac{\partial^{3}K(\phi,\phi^{\ast})}{\partial\phi_{n}\partial\phi_{m}\partial\phi_{l}^{\ast}}
\Bigl(\overline{\psi_{m}}\gamma^{\mu}\psi_{lR}\Bigr)\partial_{\mu}\phi_{n}  \nonumber \\
&\quad-\frac{1}{2}\sum_{nmlk}
\frac{\partial^{4}K(\phi,\phi^{\ast})}{\partial\phi_{n}\partial\phi_{m}\partial\phi_{k}^{\ast}\partial\phi_{l}^{\ast}}
\Bigl(\overline{\psi_{n}}\psi_{mL}\Bigr) \,\Bigl(\overline{\psi_{k}}\psi_{lR}\Bigr)   \nonumber \\
&\quad+\sum_{nm}\frac{\partial^{2}K(\phi,\phi^{\ast})}{\partial\phi_{n}\partial\phi_{m}^{\ast}}
\partial_{\mu}\phi_{n}\partial^{\mu}\phi_{m}^{\ast}
-\frac{1}{2}\sum_{nm}\frac{\partial^{2}K(\phi,\phi^{\ast})}{\partial\phi_{n}\partial\phi_{m}}
\partial_{\mu}\phi_{n}\partial^{\mu}\phi_{m}  \nonumber \\
&\quad-\frac{1}{2}\sum_{nm}\frac{\partial^{2}K(\phi,\phi^{\ast})}{\partial\phi_{n}^{\ast}\partial\phi_{m}^{\ast}}
\partial_{\mu}\phi_{n}^{\ast}\partial^{\mu}\phi_{m}^{\ast}\Biggr\} \:.  \label{26.8.3}
\end{align}
为了使费米子动能项的实性质是显然的, 我们可以使用方程(\ref{26.A.21})写下
\[
\Bigl(\overline{\psi_{n}}\,\slashed{\partial}\psi_{mR}\Bigr) =\Bigl(\overline{\psi_{n}}\,\slashed{\partial}\psi_{mL}\Bigr)^{\ast}\:.
\]
$K(\Phi,\Phi^{\ast})\,$的$\,D\,$-项是$\,-(\bar{\theta}\gamma_{5}\theta)^{2}/4\,$的系数减去%
达朗贝尔算符作用在$\,K(\Phi,\Phi^{\ast})\,$中$\,\theta\,$无关项上的结果的一半, 这一项就是$\,K(\phi,\phi^{\ast})$, 所以
\begin{align}
    \frac{1}{2}\Bigl[K(\Phi,\Phi^{\ast})\Bigr]_{D} &= \operatorname{Re}\sum_{nm}\mathscr{G}_{nm}
    \Biggl[-\frac{1}{2}\Bigl(\overline{\psi_{m}}\,\slashed{\partial}(1+\gamma_{5})\psi_{n}\Bigr) \nonumber \\
    &\quad \phantom{\operatorname{Re}\sum_{nm}\mathscr{G}_{nm}\Biggl[-\frac{1}{2}}
    +\mathscr{F}_{n}\mathscr{F}_{m}^{\ast}-\partial_{\mu}\phi_{n}\partial^{\mu}\phi_{m}^{\ast}\Biggr] \nonumber \\
    &\quad -\operatorname{Re}\sum_{nml} \frac{\partial^{3}K(\phi,\phi^{\ast})}{\partial\phi_{n}\partial\phi_{m}\partial\phi_{l}^{\ast}}
    \Bigl(\overline{\psi_{n}}\psi_{mL}\Bigr)\mathscr{F}_{l}^{\ast} \nonumber \\
    &\quad +\operatorname{Re} \sum_{nml}
    \frac{\partial^{3}K(\phi,\phi^{\ast})}{\partial\phi_{n}\partial\phi_{m}\partial\phi_{l}^{\ast}}
    \Bigl(\overline{\psi_{m}}\gamma^{\mu}\psi_{lR}\Bigr)\partial_{\mu}\phi_{n} \nonumber \\
    &\quad +\frac{1}{4} \sum_{nmlk} \frac{\partial^{4}K(\phi,\phi^{\ast})}{\partial\phi_{n}\partial\phi_{m}
    \partial\phi_{l}^{\ast}\partial\phi_{k}^{\ast}}\Bigl(\overline{\psi_{n}}\psi_{mL}\Bigr)
    \Bigl(\overline{\psi_{k}}\psi_{lR}\Bigr) \:, \label{26.8.4}
\end{align}
其中$\,\mathscr{G}(\phi,\phi^{\ast})$\,是\emph{\,K\"{a}hler}\,{\kai{度规}}
\begin{equation}
\mathscr{G}_{nm}(\phi,\phi^{\ast})\equiv \frac{\partial^{2}K(\phi,\phi^{\ast})}{\partial\phi_{n}\partial\phi_{m}^{\ast}} \:. \label{26.8.5}
\end{equation}
注意到方程(\ref{26.4.2})中的常矩阵$\,g_{nm}\,$在这里被换成了\,K\"{a}hler\,度规%
$\,\mathscr{G}_{nm}(\phi,\phi^{\ast})$. 因为\,K\"{a}hler\,度规与场相关, 所以我们一般无法通过重新定义场使得它等于一个单位矩阵, 所以总的拉格朗日量必须是如下的形式
\begin{align}
\mathscr{L} &= \operatorname{Re}\sum_{nm}\mathscr{G}_{nm}\Biggl[-\frac{1}{2}
\Bigl(\overline{\psi_{m}}\,\slashed{\partial}(1+\gamma_{5})\psi_{n}\Bigr)+\mathscr{F}_{n}\mathscr{F}_{m}^{\ast}
-\partial_{\mu}\phi_{n}\partial^{\mu}\phi_{m}^{\ast} \Biggr] \nonumber \\
&\quad -\operatorname{Re}\sum_{nml}
\frac{\partial^{3}K(\phi,\phi^{\ast})}{\partial\phi_{n}\partial\phi_{m}\partial\phi_{l}^{\ast}}
\Bigl(\overline{\psi_{n}}\psi_{mL}\Bigr)\mathscr{F}_{l}^{\ast} \nonumber \\
&\quad +\operatorname{Re}\sum_{nml}
\frac{\partial^{3}K(\phi,\phi^{\ast})}{\partial\phi_{n}\partial\phi_{m}\partial\phi_{l}^{\ast}}
\Bigl(\overline{\psi_{m}}\gamma^{\mu}\psi_{lR}\Bigr)\partial_{\mu}\phi_{n} \nonumber \\
&\quad +\frac{1}{4} \sum_{nml}\frac{\partial^{4}K(\phi,\phi^{\ast})}{\partial\phi_{n}\partial\phi_{m}
    \partial\phi_{l}^{\ast}\partial\phi_{k}^{\ast}}\Bigl(\overline{\psi_{n}}\psi_{mL}\Bigr)
    \Bigl(\overline{\psi_{k}}\psi_{lR}\Bigr) \nonumber \\
&\quad -\operatorname{Re}\sum_{nm}\frac{\partial^{2}f(\phi)}{\partial\phi_{n}\partial\phi_{m}}
\Bigl(\overline{\psi_{n}}\psi_{mL}\Bigr)
+2\operatorname{Re}\sum_{n}\mathscr{F}_{n}\frac{\partial f(\phi)}{\partial \phi_{n}}\:. \label{26.8.6}
\end{align}
双线性型$\,(\overline{\psi_{m}}\,\slashed{\partial}\gamma_{5}\psi_{n})\,$是全导数, 因此如果$\,\mathscr{G}_{nm}\,$是常数, 它可以被扔掉, 但对于一般的\,K\"{a}hler\,势则必须保留. 在\,27.4\,节的末尾, 这一结果会被推广以纳入规范超场.


\subsection*{* * *}


就像在\,19.6\,节讨论的那样, 整体对称群$\,G\,$到子群$\,H\,$的自发破缺蕴含了一组无质量\,Goldstone\,实玻色子, 其玻色场为$\,\pi_{k}$, 对于这组玻色子, 拉格朗日量中导数最小的项取如下的形式
\begin{equation}
\mathscr{L}_{G/H}=-\sum_{k\ell}G_{k\ell}(\pi)\partial_{\mu}\pi_{k}\partial^{\mu}\pi_{\ell} \:, \label{26.8.7}
\end{equation}
其中$\,G_{k\ell}(\pi)\,$是陪集空间$\,G/H\,$的度规.
(拉格朗日密度属于这种一般形式的理论被称为{\kai{非线性$\,\sigma\,$-模型}}.) 通过将复标量场$\,\phi_{n}\,$写成它们的实部和虚部, 拉格朗日密度(\ref{26.8.6})中的$\,-\sum_{nm}\mathscr{G}_{nm}(\phi,\phi^{\ast})$\\$\partial_{\mu}%
\phi_{n}\partial^{\mu}\phi_{m}^{\ast}$可以被写成(\ref{26.8.7})的形式, 但反过来一般不成立: 像\,Goldstone\,玻色场$\,\pi_{k}\,$这样的一组实坐标可以被解释成场$\,\phi_{n}\,$这样的一组复坐标的实部和虚部, 并且这些坐标的度规定域地由方程 (\ref{26.8.5})给定, 这个条件定义了所谓的\,\emph{K\"{a}hler}\,{\kai{流形}}.\footnote{K\"{a}hler\,流形在这个背景下的重要意义%
是\,Zumino\,的一个早期文章指出的.\cite{8} 注意, 度规在整个流形上表示成(\ref{26.8.5})的形式时不需要只来自{\kai{一个}}K\"{a}hler\,势$\,K(\phi,\phi^{\ast})$;
唯一需要的是这个流形可以被有限多个互相重合的补片(patch)覆盖, 在每个补片上这是成立的且\,K\"{a}hler\,势不相同.
K\"{a}hler\,流形最简单的例子是复平面, 它的\,K\"{a}hler\,势是$\,\lvert z\rvert^{2}$. 就陪集空间$\,G/H\,$是\,K\"{a}hler\,流形的例子, Zumino\,给出了$\,G=GL(p,\mathds{C})\times GL(p+q,\mathds{C})\,$而$\,H=GL(p,\mathds{C})\,$的情况, 其中$\,p\,$和$\,q\,$是任意的正整数, $GL(N,\mathds{C})\,$是非奇异\,$N\times N\,$复矩阵的群. 这里陪集空间$\,G/H\,$的复坐标$\,\phi_{n}\,$可以被取成$\,p\times(p+q)\,$复矩阵$\,A\,$的分量, 它在$\,G\,$和$\,H\,$下分别进行变换$\,A\to BAC\,$和$\,A\to BA$, 其中$\,B\,$和$\,C\,$分别是维度为$\,p\,$和$\,p+q\,$的非奇异复方阵. 这一情况下的\,K\"{a}hler\,势就是$\,K\propto \ln\operatorname{Det}AA^{\dag}$.} 在通常的$\,G/H\,${\kai{不}}是\,K\"{a}hler\,流形的情况下, 不应该认为$\,G\,$在自发破缺到$\,H\,$的同时无法保持超对称性不破缺. 在这些情况中发生的是出现了额外的无质量玻色子, 它们和\,Goldstone\,玻色子合在一起确实形成了\,K\"{a}hler\,流形.

这是因为超势$\,f(\phi)\,$依赖于$\,\phi\,$但不依赖$\,\phi^{\ast}$, 所以, 如果整个拉格朗日量在整体对称群$\,G\,$下不变, 那么超势自动在$\,G\,$的复化群$\,G_{\mathds{C}}\,$不变: 如果$\,G\,$由变换$\,\exp(\mi\sum_{A}\theta_{A}t_{A})\,$构成, 其中$\,t_{A}\,$是生成元而$\,\theta_{A}\,$是任意实参量, 那么$\,G_{\mathds{C}}\,$有生成元相同但参量为任意复数$\,z_{A}\,$的变换$\,\exp(\mi\sum_{A}z_{A}t_{A})\,$构成. (例如, 如果$\,G\,$是$\,U(n)$, 那么$\,G_{\mathds{C}}\,$就是$\,GL(n,\mathds{C})$, 即所有非奇异复矩阵的群, 如果$\,G\,$是$\,SU(n)$, 那么$\,G_{\mathds{C}}\,$ 是$\,SL(n,\mathds{C})$, 即所有行列式为\,1\,的复矩阵的群.) 同理, 如果$\,f(\phi)\,$的某个驻点$\,\phi^{(0)}\,$在$\,G\,$的某个子群$\,H\,$下保持不变, 那么它在$\,G_{\mathds{C}}\,$的子群$\,H_{\mathds{C}}$------$\,H\,$的复化群------下保持不变. 无论$\,G/H\,$是不是 K\"{a}hler 流形, 复化陪集空间$\,G_{\mathds{C}}/H_{\mathds{C}}\,$总是\,K\"{a}hler\,流形.  能得出这点的原因是, $G_{\mathds{C}}/H_{\mathds{C}}\,$是$\,\phi_{n}\,$的平坦复空间的复子流形, 前者是\,K\"{a}hler\,流形, 而有一个定理保证了\,K\"{a}hler\,流形的复子流形也是\,K\"{a}hler 流形.\cite{9} 如果$\,G_{\mathds{C}}/H_{\mathds{C}}\,$被$\,\phi_{n}(z)=[\exp(\mi\sum_{A}z_{A}t_{A})\phi^{(0)}]_{n}\,$的值%
参数化, 那么通过将它嵌入到$\,\phi_{n}\,$的平坦复空间中就能获得度规, 通常通过线元$\,\sum_{n}\dif\phi_{n}\dif\phi_{n}^{\ast}$获得.

$G_{\mathds{C}}\,$确实不是整个拉格朗日量的对称性, 但是$\,G_{\mathds{C}}\,$破缺到$\,H_{\mathds{C}}\,$带出的\,Goldstone\,玻色子却是严格无质量的. 这是被\,27.6\,节的不可重整定理, 或者更简单地, 25.4\,节的结果所保证的, 即无质量零自旋粒子必须与通过超对称变换相联系的粒子成对出现, 因此对于任何与超对称对易的整体对称群$\,G$, 它们在这个群下有相同的变换.



%+++++++++++++++++++++++附录A++++++
\titleformat{\chapter}{\centering\CJKfamily{hei}\huge}{\chaptertitlename}{1em}{}
\titlespacing{\chapter}{0pt}{3.5ex plus .1ex minus .2ex}{10\wordsep}
\titleformat{\section}{\centering\CJKfamily{hei}\Large}{附 录\thesection}{1em}{}
\titlespacing{\section}{2em}{3.5ex plus .1ex minus .2ex}{1.5\wordsep}
\titleformat{\subsection}{\centering\CJKfamily{hei}\large}{\thesubsection}{0em}{}
\titlespacing{\subsection}{2em}{1.5ex plus .1ex minus .2ex}{\wordsep}
\renewcommand{\captionfont}{\small}
\newcounter{app26}[chapter]
\setcounter{app26}{1}
\renewcommand\thesection{}
\renewcommand\theequation{\arabic{chapter}.\Alph{app26}.\arabic{equation}}
\fancyhf{}
\fancyhead[CE]{\leftmark}
\fancyhead[CO]{\rightmark}
\fancyhead[RO,LE]{$\cdot$\ \thepage\ $\cdot$}
\renewcommand{\headrulewidth}{0.8pt}
\pagestyle{fancy}
\renewcommand{\chaptermark}[1]{\markboth{第\,\thechapter\,章\ #1}{}}
\renewcommand{\sectionmark}[1]{\markright{附录 {}\quad\ #1}{}}

\section{Majorana\,旋量}

这个附录总结了一些处理超场时需要的\,Majorana\,旋量的代数性质.

考察像$\,Q\,$或$\,\theta\,$这样的\,4\,分量费米\,Majorana\,旋量$\,s$, 它可以表示成
\begin{equation}
s=\begin{pmatrix}
e\,\varsigma^{\ast} \\ \varsigma
\end{pmatrix} \:, \label{26.A.1}
\end{equation}
其中$\,\varsigma\,$是某个\,2\,分量旋量而$\,e\,$是$\,2\times2\,$矩阵
\[
e\equiv \begin{pmatrix}
0 & 1 \\ -1 & 0
\end{pmatrix} = \mi\sigma_{2} \:.
\]
这样的旋量与它的复共轭的关系是
\begin{align}
    s^{\ast} = \begin{pmatrix}
    0 & e \\ -e & 0
    \end{pmatrix} s = -\beta\,\gamma_{5}\,\epsilon\,s \:, \label{26.A.2}
\end{align}
其中$\,\epsilon\,$是$\,4\times 4\,$矩阵
\begin{equation}
    \epsilon \equiv \begin{pmatrix}
    e & 0 \\ 0 & e
    \end{pmatrix}  \label{26.A.3}
\end{equation}
而$\,\gamma_{5}\,$和$\,\beta\,$同往常一样是$\,4\times4\,$矩阵
\[
\gamma_{5}=\begin{pmatrix}
1 & 0 \\ 0 & -1
\end{pmatrix} \:, \qquad \quad
\beta = \begin{pmatrix}
0 & 1\\ 1& 0
\end{pmatrix} \:,
\]
其中$\,1\,$和$\,0\,$在这里被理解成$\,2\times2\,$子矩阵. 取方程(\ref{26.A.2})的转置然后从右边乘上$\,\beta\,$就给出了等价公式
\begin{equation}
\bar{s} \equiv s^{\dag}\beta = s^{\mathrm{T}}\,\epsilon\,\gamma_{5}\:. \label{26.A.4}
\end{equation}


旋量分量的反对易系限制了能够从\,Majorana\,旋量中构造出的协变量的种类. 为了看到这点, 首先考察双线性协变量的对称性质将是方便的, 而它们自身也是有趣的. 对于一对\,Majorana\,旋量$\,s_{1}\,$和$\,s_{2}\,$以及任意$\,4\times4\,$数值矩阵$\,M$, 方程(\ref{26.A.4})给出
\begin{align*}
\overline{s_{1}}\,M\,s_{2} &= \sum_{\alpha\beta}s_{1\alpha}\,s_{2\beta}\,(\epsilon\,\gamma_{5}\,M)_{\alpha\beta}
=-\sum_{\alpha\beta}s_{2\alpha}\,s_{1\beta}\,(\epsilon\,\gamma_{5}\,M)_{\beta\alpha} \\
&=+\sum_{\alpha\beta}s_{2\alpha}\,s_{1\beta}\,(M^{\mathrm{T}}\,\epsilon\,\gamma_{5})_{\alpha\beta}
=\overline{s_{2}}\,(\epsilon\gamma_{5})^{-1}\,M^{\mathrm{T}}\,\epsilon\gamma_{5}\,s_{1}\:,
\end{align*}
其中第二个等号后面的负号是因为这些旋量的费米性. 我们在\,5.4\,节发现, 从\,Dirac\,矩阵构造出来的\,16\,个协变矩阵满足
\begin{equation}
M^{\mathrm{T}}= \begin{cases}
+\mathscr{C}M\mathscr{C}^{-1} &\qquad M=1,\:\:\gamma_{5}\gamma_{\mu},\:\:\gamma_{5} \\
-\mathscr{C}M\mathscr{C}^{-1} &\qquad M=\gamma_{\mu},\:\:[\gamma_{\mu},\gamma_{\nu}]
\end{cases} \:, \label{26.A.5}
\end{equation}
其中$\,\mathscr{C}\,$是矩阵
\begin{equation}
\mathscr{C}=\gamma_{2}\beta=-\epsilon\gamma_{5}=
\begin{pmatrix}
-e & 0 \\ 0 & e
\end{pmatrix} \:. \label{26.A.6}
\end{equation}
从它得出
\begin{equation}
(\overline{s_{1}}\,Ms_{2})= \begin{cases}
+(\overline{s_{2}}\,Ms_{1}) &\qquad M=1,\:\:\gamma_{5}\gamma_{\mu},\:\:\gamma_{5} \\
-(\overline{s_{2}}\,Ms_{1}) &\qquad M=\gamma_{\mu},\:\:[\gamma_{\mu},\gamma_{\nu}]
\end{cases} \:. \label{26.A.7}
\end{equation}
特别地, 令$\,s_{1}=s_{2}=s$, 我们发现
\begin{equation}
\bar{s}\,\gamma_{\mu}\,s=\bar{s}\,[\gamma_{\mu},\gamma_{\nu}]\,s =0 \:, \label{26.A.8}
\end{equation}
所以从单个\,Majorana\,旋量$\,s\,$构造出来的双线性协变量只有$\,\bar{s}\,s$, $\bar{s}\,s\gamma_{5}\gamma_{\mu}\,s\,$和$\,\bar{s}\,\gamma_{5}\,s$.

在考察最一般超场的形式时, 我们需要两个或多个\,Majorana\,旋量乘积的表达式. 对于两个旋量, 我们回忆起任何$\,4\times4\,$矩阵都可以表示成\,16\,个协变矩阵$\,1$, $\gamma_{\mu}$, $[\gamma_{\mu},\gamma_{\nu}]$, $\gamma_{5}\gamma_{\mu}$, $\gamma_{5}\,$的和. Lorentz\,不变性告诉我们, 对于矩阵$\,s_{\alpha}\,\bar{s}_{\beta}$, 这个表达式必须采取如下的形式
\begin{align*}
s\,\bar{s} &= k_{S}\,(\bar{s}\,s) + k_{V}\,\gamma_{\mu}\,(\bar{s}\,\gamma^{\mu}\,s)
+k_{T}\,[\gamma_{\mu},\gamma_{\nu}]\,(\bar{s}\,[\gamma^{\mu},\gamma^{\nu}]\,s) \\
&\quad +k_{A}\,\gamma_{5}\gamma_{\mu}\,(\bar{s}\,\gamma_{5}\gamma^{\mu}\,s)
+k_{P}\,\gamma_{5}\,(\bar{s}\,\gamma_{5}\,s) \:,
\end{align*}
其中这些$\,k\,$是需要决定的常数. 方程(\ref{26.A.8})表明我们可以取$\,k_{V}=k_{T}=0$. 通过从右边乘上$\,1$, $\gamma_{5}\gamma^{\mu}\,$和$\,\gamma_{5}\,$然后在取迹, 我们可以计算出剩下的系数, 这个方法给出$\,k_{S}=-1/4$, $k_{A}=+1/4$\, 和$\,k_{P}=-1/4$. 以这种方法, 我们发现
\begin{equation}
s\,\bar{s} = -\tfrac{1}{4}(\bar{s}\,s) + \tfrac{1}{4}\gamma_{5}\gamma_{\mu}\,(\bar{s}\,\gamma_{5}\gamma^{\mu}\,s)
-\tfrac{1}{4}\gamma_{5}\,(\bar{s}\,\gamma_{5}\,s) \:. \label{26.A.9}
\end{equation}
通过给右边乘上$\,-\epsilon\gamma_{5}\,$并使用方程(\ref{26.A.4}), 我们可以将其变成如下形式
\begin{equation}
s_{\alpha}\,s_{\beta}= \tfrac{1}{4}(\epsilon\gamma_{5})_{\alpha\beta}\,(\bar{s}\,s)
+\tfrac{1}{4}(\gamma_{\mu}\epsilon)_{\alpha\beta}\,(\bar{s}\,\gamma_{5}\gamma^{\mu}\,s)
+\tfrac{1}{4}\epsilon_{\alpha\beta}\,(\bar{s}\,\gamma_{5}\,s) \:, \label{26.A.10}
\end{equation}
或者, 等价地,
\begin{equation}
s_{\alpha}\,s_{\beta}= \tfrac{1}{4}(\epsilon\gamma_{5})_{\alpha\beta}\,(s^{\mathrm{T}}\,\epsilon\,\gamma_{5}\,s)
+\tfrac{1}{4}(\gamma_{\mu}\epsilon)_{\alpha\beta}\,(s^{\mathrm{T}}\,\epsilon\,\gamma^{\mu}\,s)
+\tfrac{1}{4}\epsilon_{\alpha\beta}\,(s^{\mathrm{T}}\,\epsilon\,s)\:. \label{26.A.11}
\end{equation}

现在, 考察\,Majorana\,旋量$\,s\,$的\,3\,个分量的乘积$\,s_{\alpha}s_{\beta}s_{\gamma}$. 我们可以把$\,s\,$分成左手部分和右手部分
\begin{equation}
s=s_{L}+s_{R}\:, \qquad s_{L} = \tfrac{1}{2}(1+\gamma_{5})s\:, \qquad
s_{R}=\tfrac{1}{2}(1-\gamma_{5})s\:. \label{26.A.12}
\end{equation}
$s_{L}\,$和$\,s_{R}\,$都只有两个独立分量, 又因为任何费米\,c\,-数的平方为零, 所以对于所有$\,\alpha$, $\beta\,$和$\,\gamma$, 我们有$\,s_{L\alpha}s_{L\beta}s_{L\gamma}=0\,$和$\,s_{R\alpha}s_{R\beta}s_{R\gamma}=0$, 因此
\[
s_{\alpha}s_{\beta}s_{\gamma} = s_{L\alpha}s_{L\beta}s_{L\gamma}+s_{L\alpha}s_{R\beta}s_{L\gamma}
+s_{R\alpha}s_{L\beta}s_{L\gamma} + L\leftrightarrow R\:,
\]
其中``$ \,L{\leftrightarrow }R\,$''表示对前面的项交换$\,L\,$和$\,R\,$指标后的和. 为了计算这个表达式, 我们给方程(\ref{26.A.11}) 乘上合适的因子$\,(1+\gamma_{5})/2$, 并发现
\[
s_{L\alpha}\,s_{L\beta}= \tfrac{1}{4}[\epsilon(1+\gamma_{5})]_{\alpha\beta}\,(s_{L}^{\mathrm{T}}\epsilon s_{L})\:.
\]
如果我们现在给它乘上$\,s_{R\gamma}$, 由于$\,(s_{R}^{\mathrm{T}}\epsilon s_{R})\,s_{R\gamma}=0$, 我们可以扔掉双线性型$\,(s_{L}^{\mathrm{T}}\epsilon s_{L})\,$中旋量上的指标$\,L$:
\[
s_{L\alpha}\,s_{L\beta}\,s_{R\gamma}=\tfrac{1}{4}[\epsilon(1+\gamma_{5})]_{\alpha\beta}
\,(s^{\mathrm{T}}\,\epsilon\, s)\,s_{R\gamma} \:.
\]
相同的讨论也给出
\[
s_{R\alpha}\,s_{L\beta}\,s_{L\gamma}=\tfrac{1}{4}[\epsilon(1-\gamma_{5})]_{\alpha\beta}
\,(s^{\mathrm{T}}\,\epsilon\, s)\,s_{L\gamma} \:.
\]
对这个两个表达式求和, 再将结果中的$\,\gamma\,$换成$\,\alpha\,$或$\,\beta$, 把所有这些加起来最后给出
\begin{align}
    s_{\alpha}s_{\beta}s_{\gamma} &= \tfrac{1}{4}\Bigl(s^{\mathrm{T}}\epsilon s\Bigr)
    \Bigl[ \epsilon_{\alpha\beta}\,s_{\gamma}- (\epsilon\gamma_{5})_{\alpha\beta}\,(\gamma_{5}s)_{\gamma}
    -\epsilon_{\alpha\gamma}\,s_{\beta} \nonumber \\
    &\quad +(\epsilon\gamma_{5})_{\alpha\gamma}\,(\gamma_{5}s)_{\beta} +\epsilon_{\beta\gamma}\,s_{\alpha}
    -(\epsilon\gamma_{5})_{\beta\gamma}\,(\gamma_{5}s)_{\alpha}\Bigr]\:. \label{26.A.13}
\end{align}


为了计算\,4\,个\,Majorana\,旋量分量的乘积, 我们注意到$\,(s^{\mathrm{T}}\epsilon s)\,$只包含两个$\,s_{L}\,$的项或两个$\,s_{R}\,$的项, 所以
\[
(s^{\mathrm{T}}\epsilon s)s_{\gamma}s_{\delta} = (s^{\mathrm{T}}\epsilon s)
[s_{R\gamma}s_{R\delta}+s_{L\gamma}s_{L\delta}] \:.
\]
利用方程(\ref{26.A.11})计算方括号中的和, 并注意到
\[
(s^{\mathrm{T}}\epsilon s)(s^{\mathrm{T}}\epsilon\gamma_{5} s)
=(s_{L}^{\mathrm{T}}\epsilon s_{L})(s_{R}^{\mathrm{T}}\epsilon s_{R})
-(s_{R}^{\mathrm{T}}\epsilon s_{R})(s_{L}^{\mathrm{T}}\epsilon s_{L})=0 \:,
\]
我们发现
\begin{equation}
    (s^{\mathrm{T}}\epsilon s)s_{\gamma}s_{\delta}
    = \tfrac{1}{4}\,\epsilon_{\gamma\delta}(s^{\mathrm{T}}\epsilon s)^{2} \:. \label{26.A.14}
\end{equation}
因此给方程(\ref{26.A.13})乘上$\,s_{\delta}\,$就给出结果
\begin{align}
    s_{\alpha}s_{\beta}s_{\gamma}s_{\delta} &= \tfrac{1}{16}\Bigl(s^{\mathrm{T}}\epsilon s\Bigr)^{2}
    \Bigl[\epsilon_{\alpha\beta}\,\epsilon_{\gamma\delta}-
    (\epsilon\gamma_{5})_{\alpha\beta}\,(\epsilon\gamma_{5})_{\gamma\delta}-
    \epsilon_{\alpha\gamma}\,\epsilon_{\beta\delta} \nonumber \\
    &\quad +(\epsilon\gamma_{5})_{\alpha\gamma}\,(\epsilon\gamma_{5})_{\beta\delta}
    + \epsilon_{\beta\gamma}\,\epsilon_{\alpha\delta}-
    (\epsilon\gamma_{5})_{\beta\gamma}\,(\epsilon\gamma_{5})_{\alpha\delta}\Bigr]\:. \label{26.A.15}
\end{align}
五个\,$s\,$分量的任意乘积都为零, 所以这样就列完了\,Majorana\,旋量分量的乘积公式.

我们可以用这些公式推导一些在处理超场时有用的加法关系. 通过用$\,(\epsilon\gamma_{5})_{\beta\gamma}\,$和$\,(\epsilon\gamma_{\mu})_{\beta\gamma}\,$收%
缩方程(\ref{26.A.13}), 我们发现
\begin{equation}
    s_{\alpha}\,\Bigl(\bar{s}\,s\Bigr) =-(\gamma_{5}\,s)_{\alpha}\Bigl(\bar{s}\,\gamma_{5}\,s\Bigr)\label{26.A.16}
\end{equation}
和
\begin{equation}
    s_{\alpha}\,\Bigl(\bar{s}\,\gamma_{5}\gamma_{\mu}\,s\Bigr)= -(\gamma_{\mu}\,s)_{\alpha}
    \Bigl(\bar{s}\,\gamma_{5}\,s\Bigr) \:. \label{26.A.17}
\end{equation}
我们可以从方程(\ref{26.A.16})和(\ref{26.A.17})导出``Fierz''恒等式
\begin{equation}
    \Bigl(\bar{s}\,s\Bigr)^{2}=-\Bigl(\bar{s}\,\gamma_{5}\,s\Bigr)^{2}\:, \qquad
    \Bigl(\bar{s}\,\gamma_{5}\gamma_{\mu}\,s\Bigr)\Bigl(\bar{s}\,\gamma_{5}\gamma_{\nu}\,s\Bigr)
    =-\eta_{\mu\nu}\Bigl(\bar{s}\,\gamma_{5}\,s\Bigr)^{2} \:. \label{26.A.18}
\end{equation}
另外, 方程(\ref{26.A.14})可以写成协变形式
\begin{equation}
    (\bar{s}\,\gamma_{5}\,s)^{2}\,s\,\bar{s} = -\tfrac{1}{4}\gamma_{5}\,(\bar{s}\,\gamma_{5}\,s)^{2}\:.\label{26.A.19}
\end{equation}

标明\,Majorana\,旋量双线性积的实性质也将是有用的. 对于任何一对满足相位约定(\ref{26.A.1})的\\\,Majorana\,旋量$\,s_{1}\,$和$\,s_{2}$, 方程(\ref{26.A.2})和(\ref{26.A.4})给出
\[
(\overline{s_{1}}\,M\,s_{2})^{\ast} = -(s_{1}^{\dag}\epsilon\gamma_{5}\,M^{\ast}\,s_{2}^{\ast})
=(\overline{s_{1}}\,\beta\,\epsilon\,\gamma_{5}\,M^{\ast}\,\beta\,\epsilon\,\gamma_{5}\,s_{2})\:.
\]
(中间表达式的负号来自于我们撤销了$\,s_{1}\,$和$\,s_{2}\,$的交换, 这会在我们取复共轭时发生.) 但是方程 (\textcolor{foo}{5.4.40})和(\ref{26.A.6})给出$\,\beta\epsilon\gamma_{5}\gamma_{\mu}^{\ast}\beta%
\epsilon\gamma_{5}=\gamma_{\mu}$, 所以
\begin{equation}
    \beta\,\epsilon\,\gamma_{5}\,M^{\ast}\,\beta\,\epsilon\,\gamma_{5}=
    \begin{cases}
    +M &\qquad M=1,\:\:\gamma_{\mu},\:\:[\gamma_{\mu},\gamma_{\nu}] \\
    -M &\qquad M=\gamma_{\mu}\gamma_{5},\:\:\gamma_{5}
    \end{cases} \:, \label{26.A.20}
\end{equation}
因此
\begin{equation}
(\overline{s_{1}}\,M\,s_{2})^{\ast}= \begin{cases}
+(\overline{s_{1}}\,M\,s_{2}) &\qquad M=1,\:\:\gamma_{\mu},\:\:[\gamma_{\mu},\gamma_{\nu}]  \\
-(\overline{s_{1}}\,M\,s_{2}) &\qquad M=\gamma_{\mu}\gamma_{5},\:\:\gamma_{5}
\end{cases} \:. \label{26.A.21}
\end{equation}

最后我们提一下, 任何旋量$\,u\,$都可以写成一对\,Majorana\,旋量$\,s_{\pm}$
\begin{equation}
u=s_{+}+\mi\,s_{-} \:, \label{26.A.22}
\end{equation}
其中
\begin{equation}
s_{+} \equiv \frac{1}{2}\Bigl(u-\beta\epsilon\gamma_{5}u^{\ast}\Bigr) \:, \qquad \quad
s_{-} \equiv \frac{1}{2\mi}\Bigl(u+\beta\epsilon\gamma_{5}u^{\ast}\Bigr) \:.\label{26.A.23}
\end{equation}
为了验证$\,s_{\pm}\,$是满足方程(\ref{26.A.2})的\,Majorana\,旋量, 只需回忆起$\,\beta\epsilon\gamma_{5}\,$是实的, 以及$\,(\beta\epsilon\gamma_{5})^{2}=1$.



\section*{习题}
\noindent 1. 在$\,N=2\,$超对称的情况下, 利用\,26.1\,节的直接技巧, 找到只有一个\,Majorana\,旋量场和两个复标量场的有质量场超多重态的超对称变换规则. \\

\noindent 2. 计算时间反演超场
\[
\mathsf{T}^{-1}S(x,\theta)\mathsf{T}
\]
的分量场, 将它们写成超场$\,S(x,\theta)\,$的分量场的形式. 对于左手征超场的时间反演, 我们得到了哪类超场? 对于线性超场又是什么?
\\

\noindent 3. 考察单个左手征超场$\,\Phi\,$的$\,N=1\,$超对称理论. 在超场的符号约定下, 列出所有包含$\,\Phi$\,和(或)$\,\Phi^{\ast}\,$且量纲为\,5\,的可以加到拉格朗日密度上的项. \\

\noindent 4. 考察三个左手征超场$\,\Phi_{1}$, $\Phi_{2}\,$和$\,\Phi_{3}\,$的理论, 它有通常的动能项, 以及超势
\[
f(\Phi_{1},\Phi_{2},\Phi_{3})=\Phi_{1}\Phi_{3}^{2}+\Phi_{2}\Bigl(\Phi_{3}^{2}+a\Bigr) \:,
\]
其中$\,a\,$是一个非零实常数. 证明这是一个超对称自发破缺的理论. 找到势能的最小值. 将戈德斯通微子的场表示成$\,\Phi_{1}$, $\Phi_{2}\,$和$\,\Phi_{3}\,$的费米分量. \\

\noindent 5. 对于作用量(\ref{26.6.9}), 找到流超场的所有分量, 将它们写成左手征超场$\,\Phi\,$的分量, 超势\,$f\,$的导数以及\,K\"{a}hler\,势$\,K\,$的形式. \\

\noindent 6. 验证方程(\ref{26.7.18})和(\ref{26.7.10})给出的超对称流与超场(\ref{26.7.21})的$\,\omega\,$-分量%
通过方程(\ref{26.7.20})相关联.






%++++++++++++++++++参考文献+++++++++
\renewcommand{\sectionmark}[1]{\markright{ #1}{}}
\renewcommand{\bibname}{参考文献}

\begin{thebibliography}{99}
    \bibitem{1} A. Salam and J. Strathdee, {\textit{Nucl. Phys.}} {\bf{B76}}, 477 (1974). 这篇文章重印于{\textit{Supersymmetry}}, S. Ferrar\,编辑(North Holland/World Scientific, Amsterdam/Singapore, 1987).
    \bibitem{2} J. Wess and B. Zumino, {\textit{Nucl. Phys.}} {\bf{B70}}, 13 (1974). 这篇文章重印于{\textit{Supersymmetry}}, 参考文献[1].
    \bibitem{3} L. O'Raifeartaigh, {\textit{Nucl. Phys.}} {\bf{B96}}, 331 (1975). 这篇文章重印于{\textit{Supersymmetry}}, 参考文献[1].
    \bibitem{4} F. A. Berezin, {\textit{The Method of Second Quantization}} (Academic Press, New York, 1966).
    \bibitem{5} J. Iliopoulos and B. Zumino, {\textit{Nucl. Phys.}} {\bf{B76}}, 310 (1974); S. Ferrara and B.Zumino, {\textit{Nucl. Phys.}} {\bf{B87}}, 207 (1975). 这些文章重印于{\textit{Supersymmetry}}, 参考文献[1].
    \bibitem{6} 这一节沿用的是\,S. Ferrara\,和\,B. Zumino\,的方法, 参考文献[5].
    \bibitem{7} X. Gr\`{a}cia and J. Pons, {\textit{J. Phys.}} {\bf{A25}}, 6357 (1992). 在此感谢\,J. Gomis\,建议我使用比对$\,\ddot{q}^{n}\,$系数的方程.
    \bibitem[7a]{7a} M. Grisaru, in {\textit{Recent Developments in Gravitation - Garg\`{e}se 1978}}, M. L\'{e}vy and S. Deser\,编辑(Plenum Press, New York, 1979): 577.
    \bibitem{8} B. Zumino, {\textit{Phys. Lett.}} {\bf{87B}}, 203 (1979). 这篇文章重印于{\textit{Supersymmetry}}, 参考文献[1].
    \bibitem{9} P. Griffiths and J. Harris, {\textit{Principles of Algebraic Geometry}} (Wiley, New York, 1978): 109. 在此感谢\,D. Freed\,告诉我这个一般定理的应用.
\end{thebibliography}


\renewcommand\thesection{\thechapter.\arabic{section}}
%===============================================================================%
\fancyhf{}
\fancyhead[CE]{\leftmark}
\fancyhead[CO]{\rightmark}
\fancyhead[RO,LE]{$\cdot$\ \thepage\ $\cdot$}
\renewcommand{\headrulewidth}{0.8pt}
\renewcommand\theequation{\arabic{chapter}.\arabic{section}.\arabic{equation}}
\pagestyle{fancy}
\renewcommand{\chaptermark}[1]{\markboth{第\,\thechapter\,章\ #1}{}}
\renewcommand{\sectionmark}[1]{\markright{\thesection\ #1}{}}
%===========================章节的标题格式====================================%
\titleformat{\chapter}{\centering\CJKfamily{zhhei}\huge}{\chaptertitlename}{1em}{}
\titlespacing{\chapter}{0pt}{3.5ex plus .1ex minus .2ex}{10\wordsep}
\titleformat{\section}{\centering\CJKfamily{zhhei}\Large}{\thesection}{1em}{}
\titlespacing{\section}{2em}{3.5ex plus .1ex minus .2ex}{1.5\wordsep}
\titleformat{\subsection}{\centering\CJKfamily{zhkai}\large}{}{0em}{}
\titlespacing{\subsection}{2em}{1.5ex plus .1ex minus .2ex}{\wordsep}
\renewcommand{\captionfont}{\small}

%第二十七章 超对称规范理论

\chapter{超对称规范理论} \label{cha:27}

前两卷中成功描述强, 弱和电磁相互作用的理论都是规范理论. 因此, 为了看到简单超对称形如何与现实世界联系起来, 我们必须要考虑如何构造既满足超对称性又满足规范不变性的作用量.\cite{1}

\section{手征超场的规范不变作用量} \label{sec:27.1}

考虑一组保持超对称性生成元$\,Q\,$不变的阿贝尔或非阿贝尔规范变换. (简单超对称性只有一个\,Majorana\,旋量超对称性生成元, 对于任何半单规范群, 它只能构成这个群的平庸表示.) 同一个超多重态中的各个分量场在这种规范变换下必须以相同的方式变换. 特别地, 对于一个左手征超场, 我们有
\begin{align}
    &\phi_{n}(x) \to \sum_{m} \Biggl[\exp\Bigl(\mi\sum_{A}t_{A}\Lambda^{A}(x)\Bigr) \Biggr]_{nm}\phi_{m}(x) \:,\nonumber\\
    &\psi_{nL}(x) \to \sum_{m} \Biggl[\exp\Bigl(\mi\sum_{A}t_{A}\Lambda^{A}(x)\Bigr) \Biggr]_{nm}\psi_{mL}(x) \:,
    \label{27.1.1} \\
    &\mathscr{F}_{n}(x) \to \sum_{m} \Biggl[\exp\Bigl(\mi\sum_{A}t_{A}\Lambda^{A}(x)\Bigr) \Biggr]_{nm}\mathscr{F}_{m}(x) \:, \nonumber
\end{align}
其中$\,t_{A}\,$是代表规范代数生成元的厄米矩阵, $\Lambda^{A}(x)\,$是$\,x^{\mu}\,$的实函数, 参数化了一个有限大的规范变换. (我们对规范变换函数使用的符号约定几乎与\,15.1\,节相同, 不同之处只有, 为了避免与\,Dirac\,指标产生混淆, 取代$\,\alpha,\,\beta\,$等, 我们用字母$\,A,B\,$等来标记规范生成元和规范变换参量.)

左手征超场(\ref{26.3.11})包含一些分量场的导数, 所以它的变换要比方程(\ref{27.1.1})复杂. 然而, 方程(\ref{26.3.21})表明, 如果用$\,\theta_{L}\,$和方程(\ref{26.3.23})定义的变量$\,x_{+}\,$表示超场, 那么超场中就没有导数. 因此它有变换规则
\begin{equation}
    \Phi_{n}(x,\theta) \to \sum_{m}\biggl[\exp\Bigl(\mi\sum_{A}t_{A}\Lambda^{A}(x_{+})\Bigr)\biggr]_{nm}
    \Phi_{m}(x,\theta) \:. \label{27.1.2}
\end{equation}
如果作用量中的一项只依赖左手征超场而不依赖左手征超场的导数或者复共轭, 例如方程(\ref{26.3.30}) 中的$\,\int\dif^{4}x\,[f(\Phi)]_{\mathscr{F}}$, 那么只要它在$\,\Lambda^{A}(x)\,$与$\,x^{\mu}\,$无关的整体规范变换下不变, 那么它(和它的复共轭)在定域规范变换下不变. 在手征超场的可重整理论中引入规范场的需求仅来源于既包含$\,\Phi_{n}$又包含$\,\Phi_{n}\,$的$\,D\,$-项. 因为矩阵$\,t_{A}\,$是厄米的, 所以方程(\ref{27.1.2})的厄米伴是
\begin{equation}
    \Phi^{\dag}_{n}(x,\theta) \to \sum_{m}\Phi_{m}^{\dag}(x,\theta)
    \biggl[\exp\Bigl(-\mi\sum_{A}t_{A}\Lambda^{A}(x_{+})^{\ast}\Bigr)\biggr]_{mn} \:. \label{27.1.3}
\end{equation}
要不是$\,\Lambda^{A}(x_{+})^{\ast}=\Lambda^{A}(x_{-})\,$和$\,\Lambda^{A}(x_{+})\,$之间有差异, 这几乎就是说$\,\Phi^{\dag}\,$变换遵循的规范群表示与$\,\Phi\,$ 构成的表示逆步, 以及对于$\,\Phi\,$和$\,\Phi^{\dag}\,$的任意函数, 只要它在整体规范变换下不变, 那么它就在定域规范变换下不变. 由于$\,x_{+}\,$和$\,x_{-}\,$不同, 我们必须引入有如下变换性质的规范联络矩阵$\,\Gamma_{nm}(x,\theta)$,
\begin{equation}
    \Gamma(x,\theta) \to \exp\Bigl(+\mi\sum_{A}t_{A}\Lambda^{A}(x_{+})^{\ast}\Bigr)\,\Gamma(x,\theta)\,
    \exp\Bigl(-\mi\sum_{A}t_{A}\Lambda^{A}(x_{+})\Bigr) \:. \label{27.1.4}
\end{equation}
这样, 通过给$\,\Phi^{\dag}\,$右乘$\,\Gamma$, 我们就获得了有如下变换性质的超场
\begin{equation}
    \Bigl[\Phi^{\dag}(x,\theta)\,\Gamma(x,\theta)\Bigr]_{n} \to
    \sum_{m}\Bigl[\Phi^{\dag}(x,\theta)\,\Gamma(x,\theta)\Bigr]_{m}
    \biggl[\exp\Bigl(-\mi\sum_{A}t_{A}\Lambda^{A}(x_{+})\Bigr)\biggr]_{mn} \:, \label{27.1.5}
\end{equation}
这使得任何用$\,\Phi\,$和$\,\Phi^{\dag}\Gamma\,$(不包含它们的导数或复共轭)构建的整体规格不变函数同时也是定域规范不变的. 一个显然的例子是$\,26.4\,$节中构建的拉格朗日量中$\,D\,$-项的规范不变版本$\,(\Phi^{\dag}\Gamma\Phi)_{D}$.

任何像方程(\ref{27.1.4})那样变换的$\,\Gamma(x,\theta)\,$都能让我们构建手征超场的规范不变拉格朗日量. 选择不是唯一的; 如果$\,\Gamma\,$像方程(\ref{27.1.4})那样变换, 我们给它右乘一个有如下变换性质的左手征超场$\,\Upsilon_{L}$
\[
    \Upsilon_{L}(x,\theta) \to  \exp\Bigl(+\mi\sum_{A}t_{A}\Lambda^{A}(x_{+})\Bigr)\,\Upsilon_{L}(x,\theta)\,
    \exp\Bigl(-\mi\sum_{A}t_{A}\Lambda^{A}(x_{+})\Bigr) \:,
\]
那么我们就得到了一个也满足方程(\ref{27.1.4})的新规范联络. 一种简化方式是把$\,\Gamma(x,\theta)\,$取成厄米的:
\begin{equation}
    \Gamma^{\dag}(x,\theta) = \Gamma(x,\theta) \:. \label{27.1.6}
\end{equation}
这总是可能的: 如果存在任何满足方程(\ref{27.1.4})的$\,\Gamma(x,\theta)$, 那么通过取方程(\ref{27.1.4})的厄米伴, 我们可以很容易地看到$\,\Gamma^{\dag}(x,\theta)\,$的变换方式与$\,\Gamma(x,\theta)\,$相同, 所以, 如果$\,\Gamma(x,\theta)\,$不是厄米的, 我们就可以用它的厄米部分$\,(\Gamma+\Gamma^{\dag})/2\,$替换它(如果这部分为零, 那么就用反厄米部分$\,(\Gamma-\Gamma^{\dag})/2\mi\,$ 替换它.) 另一个有重要物理意义的简化是将$\,\Gamma(x,\theta)\,$表示成那些规范变换性质不依赖于超场$\,\Phi(x,\theta)\,$所属的规范代数特定表示$\,t_{A}\,$的场, 这使得对于按照规范群的任意表示变换的手征超场, 这些场可以被用来构造一个合适的$\,\Gamma(x,\theta)$. 对于这个目的, 回忆起\,Baker-Hausdorff\,公式是有用的, 这一公式表述了, 对于任意矩阵$\,a\,$和$\,b$,
\begin{equation}
    \me^{a}\,\me^{b} = \exp\Bigl(a+b+\tfrac{1}{2}[a,b]+\tfrac{1}{12}[a,[a,b]]+\tfrac{1}{12}[b,[b,a]]+\cdots\Bigr)\:,
    \label{27.1.7}
\end{equation}
其中``$\cdots$''表示可以被写为$\,a\,$和$\,b\,$的多重对易子的高阶项, 其中的二阶项和三阶项已经显式地写出来. 由此得出, 对于一个\,Lie\,代数的任意表示, 我们有
\begin{equation}
    \exp\Bigl(\sum_{A}a^{A}t_{A}\Bigr)\,\exp\Bigl(\sum_{A}b^{A}t_{A}\Bigr)=\exp\Bigl(\sum_{A}f^{A}(a,b)t_{A}\Bigr)\:,
    \label{27.1.8}
\end{equation}
其中
\begin{align}
    f^{A}(a,b)&=a^{A}+b^{A}+\tfrac{1}{2}\mi\sum_{BC}C^{A}{}_{BC}a^{B}b^{C}
    -\tfrac{1}{12}\sum_{BCDE}C^{A}{}_{BC} C^{C}{}_{DE}\,a^{B}\,a^{D}\,b^{E} \nonumber \\
    &\quad -\tfrac{1}{12}\sum_{BCDE}C^{A}{}_{BC} C^{C}{}_{DE}\,b^{B}\,b^{D}\,a^{E}+\cdots \:, \label{27.1.9}
\end{align}
它通过结构常数$\,C^{A}{}_{BC}\,$依赖于这个\,Lie\,代数, 结构常数像往常一样定义成
\[
[t_{B},t_{C}] = \mi\sum_{A}C^{A}{}_{BC}\,t_{A} \:,
\]
但是它不依赖由$\,t_{A}\,$构成的特定表示. 我们将取$\,\Gamma(x,\theta)\,$为如下的形式
\begin{equation}
    \Gamma(x,\theta)=\exp\Biggl(-2\sum_{A}t_{A}\,V^{A}(x,\theta)\Biggr)\:, \label{27.1.10}
\end{equation}
其中$\,V^{A}(x,\theta)\,$是一组实超场(这使得$\,\Gamma\,$是厄米的), 它们不依赖由$\,t_{A}\,$构成的规范代数表示.

注意到超对称规范理论有一个额外的对称性, 它使我们能够进一步做一个重要的简化. 如果$\,\Phi\,$和$\,\Phi^{\dag}\Gamma\,$在整体规范变换下不变, 那么它不仅自动在定域规范变换下不变, 同时也在一个更大的扩充规范变换
\begin{equation}
    \Phi_{nL}(x,\theta)\to\sum_{m}\Biggl[\exp\Bigl(\mi\sum_{A}t_{A}\Omega^{A}(x,\theta)\Bigr)\Biggr]_{nm}
    \Phi_{mL}(x,\theta) \label{27.1.11}
\end{equation}
和
\begin{equation}
    \Gamma(x,\theta)\to \exp\Bigl(-\mi\sum_{A}t_{A}\Omega^{A}(x,\theta)\Bigr)\,\Gamma(x,\theta)\,
    \exp\Bigl(+\mi\sum_{A}t_{A}\Omega^{A}(x,\theta)^{\ast}\Bigr)  \label{27.1.12}
\end{equation}
下不变, 其中$\,\Omega^{A}(x,\theta)\,$是一个任意的左手征超场------即, $\theta_{L}\,$和$\,x_{+}\,$的一个任意函数. 在这个变换下,
\begin{equation}
    V^{A}(x,\theta)\to V^{A}(x,\theta)+\frac{\mi}{2}\Bigl[\Omega^{A}(x,\theta)-\Omega^{A}(x,\theta)^{\ast}\Bigr]
    +\cdots \:, \label{27.1.13}
\end{equation}
其中``$\cdots$''代表的项来自方程(\ref{27.1.7})中的对易子, 它们是规范耦合常数的一阶项或高阶项. 作为一个一般的左手征超场, $\Omega\,$可以写成(\ref{26.3.11})的形式
\begin{align}
    \Omega^{A}(x,\theta) &= W^{A}(x) -\sqrt{2}\biggl(\bar{\theta}\biggl(\frac{1+\gamma_{5}}{2}\biggr)w^{A}(x)\biggr)
    +\mathscr{W}^{A}(x)\biggl(\bar{\theta}\biggl(\frac{1+\gamma_{5}}{2}\biggr)\theta\biggr)\nonumber \\
    &\quad+\frac{1}{2}\Bigl(\bar{\theta}\gamma_{5}\gamma_{\mu}\theta\Bigr)\partial^{\mu}W^{A}(x)
    -\frac{1}{\sqrt{2}}\Bigl(\bar{\theta}\gamma_{5}\theta\Bigr)\biggl(\bar{\theta}\,\slashed{\partial}
    \biggl(\frac{1+\gamma_{5}}{2}\biggr)w^{A}(x)\biggr) \nonumber \\
    &\quad-\frac{1}{8}\Bigl(\bar{\theta}\gamma_{5}\theta\Bigr)^{2}\square W^{A}(x) \:, \label{27.1.14}
\end{align}
其中$\,W^{A}(x)\,$和$\,\mathscr{W}(x)\,$是$\,x^{\mu}\,$的任意复函数, 并且我们引入了\,Majorana\,旋量$\,w^{A}(x)$, 定义它成使得超场的左手旋量部分是$\,\frac{1}{2}(1+\gamma_{5})w^{A}(x)$. 利用\,Majorana\,双线性型的复共轭性质(\ref{26.A.21}), 方程 (\ref{27.1.14})的复共轭给出
\begin{align}
    \Omega^{A}(x,\theta)^{\ast} &= W^{A\ast}(x) -\sqrt{2}\biggl(\bar{\theta}\biggl(\frac{1-\gamma_{5}}{2}\biggr)w^{A}(x)\biggr)
    +\mathscr{W}^{A\ast}(x)\biggl(\bar{\theta}\biggl(\frac{1-\gamma_{5}}{2}\biggr)\theta\biggr)\nonumber \\
    &\quad-\frac{1}{2}\Bigl(\bar{\theta}\gamma_{5}\gamma_{\mu}\theta\Bigr)\partial^{\mu}W^{A\ast}(x)
    +\frac{1}{\sqrt{2}}\Bigl(\bar{\theta}\gamma_{5}\theta\Bigr)\biggl(\bar{\theta}\,\slashed{\partial}
    \biggl(\frac{1+\gamma_{5}}{2}\biggr)w^{A}(x)\biggr) \nonumber \\
    &\quad-\frac{1}{8}\Bigl(\bar{\theta}\gamma_{5}\theta\Bigr)^{2}\square W^{A\ast}(x) \:, \label{27.1.15}
\end{align}
我们像方程(\ref{26.2.10})中那样将实超场$\,V^{A}(x,\theta)\,$写成分量场的形式:
\begin{align}
    V^{A}(x,\theta) &= C^{A}(x) -\mi\Bigl(\bar{\theta}\,\gamma_{5}\,\omega^{A}(x)\Bigr)
    -\frac{\mi}{2}\Bigl(\bar{\theta}\gamma_{5}\theta\Bigr)M^{A}(x)
    -\frac{1}{2}\Bigl(\bar{\theta}\theta\Bigr)N^{A}(x)  \nonumber \\
    &\quad + \frac{\mi}{2}\Bigl(\bar{\theta}\gamma_{5}\gamma^{\mu}\theta\Bigr)V^{A}_{\mu}(x)
    -\mi\Bigl(\bar{\theta}\gamma_{5}\theta\Bigr)\Biggl(\bar{\theta}
    \Bigl[\lambda^{A}(x)+\frac{1}{2}\slashed{\partial}\omega^{A}(x)\Bigr]\Biggr)\nonumber \\
    &\quad -\frac{1}{4}\Bigl(\bar{\theta}\gamma_{5}\theta\Bigr)^{2}
    \Biggl(D^{A}(x)+\frac{1}{2}\square C^{A}(x)\Biggr) \:, \label{27.1.16}
\end{align}
其中$\,C^{A}(x),\,M^{A}(x),\,N^{A}(x)\,$和$\,V_{\mu}^{A}(x)\,$都是实的, 而$\,\omega^{A}(x)\,$和$\,\lambda^{A}(x)\,$则都是\,Majorana\,旋量. 在方程(\ref{27.1.13})中使用方程(\ref{27.1.14})---(\ref{27.1.16}), 我们发现规范超场的分量场进行如下的扩充规范变换
\begin{align}
    & C^{A}(x)\to C^{A}(x)-\operatorname{Im}W^{A}(x)+\cdots \:, \nonumber \\
    & \omega^{A}(x)\to \omega^{A}(x)+\frac{1}{\sqrt{2}}w^{A}(x)+\cdots \:,\nonumber \\
    & V^{A}_{\mu}(x)\to V_{\mu}^{A}(x)+\partial_{\mu}\operatorname{Re}W^{A}(x)+\cdots \:,\nonumber \\
    & M^{A}(x)\to M^{A}(x)-\operatorname{Re}\mathscr{W}^{A}(x)+\cdots \:, \label{27.1.17} \\
    & N^{A}(x)\to N^{A}(x)+\operatorname{Im}\mathscr{W}^{A}(x)+\cdots \:,  \nonumber \\
    & \lambda^{A}(x)\to \lambda^{A}(x) +\cdots \:, \nonumber \\
    & D^{A}(x)\to D^{A}(x)+\cdots \:, \nonumber
\end{align}
其中``$\cdots$''依旧表示由方程(\ref{27.1.9})中的结构常数产生的项, 因此这些项正比于一个或多个耦合常数. %
我们可以使用这样的扩充规范变换将规范超场变成一种方便的形式, 称为\,\textit{Wess-Zumino}\,{\kai{规范}}, %
在这个规范下
\begin{equation}
    C^{A}(x)=\omega^{A}(x)=M^{A}(x)=N^{A}(x)=0 \:, \label{27.1.18}
\end{equation}
进而使得
\begin{align}
    V^{A}(x,\theta) &= \frac{\mi}{2}\Bigl(\bar{\theta}\,\gamma_{5}\,\gamma^{\mu}\,\theta\Bigr)V_{\mu}^{A}(x)
    -\mi\Bigl(\bar{\theta}\,\gamma_{5}\,\theta\Bigr)\Bigl(\bar{\theta}\lambda^{A}(x)\Bigr) \nonumber \\
    &\quad+\Bigl(\bar{\theta}\,\gamma_{5}\,\theta\Bigr)^{2}D^{A}(x) \:. \label{27.1.19}
\end{align}
为了在耦合常数的零阶实现这点, 只需令$\,\operatorname{Im}W^{A}(x)=C^{A}(x)$, $w^{A}(x)=-\sqrt{2}\omega^{A}(x)\,$%
以及$\,\mathscr{W}^{A}(x)=M^{A}(x)-\mi N^{A}(x)$. 对于阿贝尔规范理论, 结构常数为零, 我们的任务就结束了. %
对于非阿贝尔规范理论, 就得给\,$\operatorname{Im}W^{A}(x)$, $w^{A}(x)\,$和$\,\mathscr{W}^{A}(x)\,$加上规范耦合常数的一阶项来抵消零阶项的对易子产生的项, 然后给\,$\operatorname{Im}W^{A}(x)$, $w^{A}(x)\,$和$\,\mathscr{W}^{A}(x)\,$加上规范耦合常数的二阶项来抵消零阶项和一阶项的对易子产生的项, 以此类推. 计算\,$\operatorname{Im}W^{A}(x)$, $w^{A}(x)\,$和%
$\,\mathscr{W}^{A}(x)\,$级数展开中的项使得在耦合常数的所有阶都满足规范条件(\ref{27.1.18})并不容易, %
但没必要这么做------重要的是这样做是可行的.

对变换规则(\ref{26.2.11})---(\ref{26.2.14})的观察表明, 除非$\,V_{\mu}^{A}=\lambda^{A}=0$, %
否则\,Wess-Zumino\,规范条件(\ref{27.1.18})在超对称变换下不是不变的, 另外, 除非$\,D^{A}=0$, %
否则条件$\,\lambda^{A}=0\,$不是超对称的, 在这种情况下, 整个超场为零. 一旦我们采取了\,Wess-Zumino\,规范, %
那么在一般的扩充规范变换或超对称下, 作用量都不再是不变的, 但是, 在做一个超对称变换后, 这使我们脱离了\,Wess-Zumino\,规范,
然后在接上合适的扩充规范变换使得我们回到\,Wess-Zumino\,规范, 作用量在这样的组合变换下是不变的. %
(我们会在\,\ref{sec:27.8}\,节明确地探究这点.) 正如我们现在将要看到的, %
在保持\,Wess-Zumino\,规范的普通规范变换(\ref{27.1.2})---(\ref{27.1.4})下, 作用量也是不变的.

当规范超场满足\,Wess-Zumino\,规范条件(\ref{27.1.18})后, 推导它在普通的无限小规范变换下的行为变得相对容易. 在这种情况下, %
$\Omega^{A}(x_{+})\,$是形如(\ref{26.3.11})的左手征超场, 但是没有$\,\psi_{L}\,$-分量和$\,\mathscr{F}\,$-分量, %
并且$\,\phi\,$-分量由{\kai{实}}无限小函数$\,\Lambda^{A}(x)\,$给出:
\begin{equation}
    \Omega^{A}(x_{+})=\Lambda^{A}(x)+\frac{1}{2}\Bigl(\bar{\theta}\gamma_{5}\gamma_{\mu}\theta\Bigr)
    \partial^{\mu}\Lambda^{A}(x)-\frac{1}{8}\Bigl(\bar{\theta}\gamma_{5}\theta\Bigr)^{2}\square\Lambda^{A}(x)\:.
    \label{27.1.20}
\end{equation}
为了计算变换规则(\ref{27.1.4})中的指数乘积, 我们使用\,Baker-Hausdorff\,公式的如下版本:
\begin{equation}
    \exp(a)\exp(X)\exp(b)=\exp\Bigl[X+L_{X}\cdot (b-a)+ (L_{X}\coth L_{X})\cdot(b+a)+\cdots\Bigr]\:,\label{27.1.21}
\end{equation}
其中\,$a$, $b\,$和$\,X\,$是任意矩阵, $L_{X}\,$是算符
\begin{equation}
    L_{X}\cdot f=\tfrac{1}{2}[X,f] \:, \label{27.1.22}
\end{equation}
而这里的``$\cdots$''代表$\,a\,$和(或)$\,b\,$的二阶项和高阶项. 在我们的情况中有
\begin{align*}
    &b+a = 2\sum_{A}t_{A}\operatorname{Im}\Lambda^{A}(x_{+})=-\mi\Bigl(\bar{\theta}\gamma_{5}\gamma_{\mu}\theta\Bigr)
    \sum_{A}t_{A}\partial^{\mu}\Lambda^{A}(x)\:, \\
    &b-a=-2\mi\sum_{A}t_{A}\operatorname{Re}\Lambda^{A}(x_{+})=-2\mi\sum_{A}t_{A}
    \biggl[\Lambda^{A}(x)-\frac{1}{8}\Bigl(\bar{\theta}\gamma_{5}\theta\Bigr)^{2}\square\Lambda^{A}(x)\biggr]\:,\\
    &X=-2\sum_{A}t_{A}V^{A}(x,\theta)=-2\sum_{A}t_{A}\Biggl[
    \frac{\mi}{2}\Bigl(\bar{\theta}\,\gamma_{5}\,\gamma^{\mu}\,\theta\Bigr)V_{\mu}^{A}(x) \\
    &\phantom{X=-2\sum_{A}t_{A}V^{A}(x,\theta)=}-\mi\Bigl(\bar{\theta}\,\gamma_{5}\,\theta\Bigr)
    \Bigl(\bar{\theta}\lambda^{A}(x)\Bigr)-\frac{1}{4}\Bigl(\bar{\theta}\,\gamma_{5}\,\theta\Bigr)^{2}
    D^{A}(x)\Biggr] \:.
\end{align*}
现在, $X\,$中的每一项包含至少一个$\,\theta_{L}\,$因子和至少一个$\,\theta_{R}\,$因子, %
而$\,a+b\,$则只有一个$\,\theta_{L}\,$因子和一个$\,\theta_{R}\,$因子, %
所以我们可以扔掉$\,L_{X}\coth L_{X}\,$中$\,L_{X}\,$的任何二阶或更高阶项. %
由于$\,L_{X}\coth L_{X}\,$是$\,L_{X}\,$\\ 的{\kai{偶}}函数, 这意味着我们可以将它本身换成它关于$\,L_{X}\,$的零阶项, %
而这一项就是\,1. 另外, 我们可以扔掉$\,b-a\,$中正比于$\,(\bar{\theta}\gamma_{5}\theta)^{2}\,$的项, %
这是因为当$\,L_{X}\,$作用在该项上时, 这一项至少会产生\,3\,个$\,\theta_{L}\,$或$\,\theta_{R}$ 因子. %
因此方程(\ref{27.1.21})右边的指数变量可以被换成
\begin{align*}
    X+\tfrac{1}{2}[X,b-a]+b+a &= -2\sum_{A}t_{A}\Biggl[V^{A}(x,\theta)+\sum_{BC}C\indices{^A_B_C}\,V^{B}(x,\theta)\,
    \Lambda^{C}(x) \\
    &\quad +\tfrac{1}{2}\mi\Bigl(\bar{\theta}\gamma_{5}\gamma_{\mu}\theta\Bigr)\partial^{\mu}\Lambda^{A}(x)
    \Biggr] \:.
\end{align*}
因此对于无限小规范变换, 变换规则(\ref{27.1.4})给出
\begin{equation}
    V^{A}(x,\theta)\to V^{A}(x,\theta)+\sum_{BC}C\indices{^A_B_C}\,V^{B}(x,\theta)\,\Lambda^{C}(x)
    +\frac{1}{2}\mi\Bigl(\bar{\theta}\gamma_{5}\gamma_{\mu}\theta\Bigr)\partial^{\mu}\Lambda^{A}(x)\:.\label{27.1.23}
\end{equation}
值得注意的是, 在普通的规范变换下, Wess-Zumino\,规范下的规范超场依旧会处在\,Wess-Zumino 规范下. %
用方程(\ref{27.1.19})中的分量场表示, 方程(\ref{27.1.23})是
\begin{align}
    & V^{A}_{\mu}(x) \to \sum_{BC}C\indices{^A_B_C}\,V^{B}_{\mu}(x)\,\Lambda^{C}(x)
    +\partial_{\mu}\Lambda^{A}(x) \:, \label{27.1.24} \\
    &\lambda^{A}(x) \to \sum_{BC}C\indices{^A_B_C}\,\lambda^{B}(x)\,\Lambda^{C}(x) \:,\label{27.1.25} \\
    &D^{A}(x)\to \sum_{BC}C\indices{^A_B_C}\,D^{B}(x)\,\Lambda^{C}(x) \:. \label{27.1.26}
\end{align}
我们可以认为方程(\ref{27.1.24})是规范场通常的\,Yang-Mills\,规范变换规则(\textcolor{foo}{15.1.9}), %
而方程(\ref{27.1.25})和 (\ref{27.1.26})告诉我们场$\,\lambda^{A}(x)\,$和$\,D^{A}(x)\,$像规范群伴随表示%
下的``物质''场那样变换. Majorana\,旋量 $\lambda^{A}\,$被称为{\kai{规范微子}}场, 而实标量$\,D^{A}\,$将会变成另一组辅助场.

接下来我们必须要计算构建手征超场的规范不变函数时需要的矩阵$\,\Gamma$. 由于所有含有超过\,4 个$\,\theta\,$因子的项为零, %
在\,Wess-Zumino\,规范下, 指数展开非常简单:
\begin{align*}
    \Gamma(x,\theta) &= \exp\Bigl(-2\sum_{A}t_{A}V^{A}(x,\theta)\Bigr) \\
    &= 1- \mi\Bigl(\bar{\theta}\,\gamma_{5}\,\gamma^{\mu}\,\theta\Bigr)\sum_{A}t_{A}V^{A}_{\mu}(x) \\
    &\quad -\frac{1}{2}\Bigl(\bar{\theta}\,\gamma_{5}\,\gamma^{\mu}\,\theta\Bigr)
    \Bigl(\bar{\theta}\,\gamma_{5}\,\gamma^{\nu}\,\theta\Bigr)\sum_{AB}t_{A}\,t_{B}\,V_{\mu}^{A}(x)\,V_{\nu}^{B}(x)\\
    &\quad+2\mi\Bigl(\bar{\theta}\,\gamma_{5}\,\theta\Bigr)\sum_{A}t_{A}\Bigl(\bar{\theta}\,\lambda^{A}(x)\Bigr)
    +\frac{1}{2}\Bigl(\bar{\theta}\,\gamma_{5}\,\theta\Bigr)^{2}\sum_{A}t_{A}D^{A}(x) \:.
\end{align*}
通过给矩阵$\,\Gamma\,$右乘一个形如(\ref{26.3.11})的左手征超场的列矢量:
\begin{align*}
    \Phi_{n}(x,\theta) &= \phi_{n}(x) -\sqrt{2}\Bigl(\bar{\theta}\psi_{nL}(x)\Bigr)
    +\mathscr{F}_{n}(x)\biggl(\bar{\theta}\biggl(\frac{1+\gamma_{5}}{2}\biggr)\theta\biggr) \\
    &\quad + \frac{1}{2}\Bigl(\bar{\theta}\gamma_{5}\gamma_{\mu}\theta\Bigr)\partial^{\mu}\phi_{n}(x)
    -\frac{1}{\sqrt{2}}\Bigl(\bar{\theta}\gamma_{5}\theta\Bigr)
    \Bigl(\bar{\theta}\,\slashed{\partial}\psi_{nL}(x)\Bigr) \\
    &\quad -\frac{1}{8}\Bigl(\bar{\theta}\gamma_{5}\gamma_{\mu}\theta\Bigr)^{2} \square\phi_{n}(x)\:,
\end{align*}
再左乘列矢量
\begin{align*}
    \Phi_{n}(x,\theta)^{\ast} &= \phi_{n}^{\ast}(x) -\sqrt{2}\Bigl(\overline{\psi_{nL}(x)}\theta\Bigr)
    +\mathscr{F}^{\ast}_{n}(x)\biggl(\bar{\theta}\biggl(\frac{1-\gamma_{5}}{2}\biggr)\theta\biggr) \\
    &\quad - \frac{1}{2}\Bigl(\bar{\theta}\gamma_{5}\gamma_{\mu}\theta\Bigr)\partial^{\mu}\phi^{\ast}_{n}(x)
    -\frac{1}{\sqrt{2}}\Bigl(\bar{\theta}\gamma_{5}\theta\Bigr)\partial_{\mu}
    \Bigl(\overline{\psi_{nL}(x)}\gamma^{\mu}\theta\Bigr) \\
    &\quad -\frac{1}{8}\Bigl(\bar{\theta}\gamma_{5}\gamma_{\mu}\theta\Bigr)^{2} \square\phi^{\ast}_{n}(x)\:.
\end{align*}
我们可以构建一个规范不变的密度. 这个乘积中$\,\theta\,$的\,4\,阶项是
\begingroup
\allowdisplaybreaks
\begin{align*}
    \Bigl[\Phi^{\dag}\,\Gamma\,\Phi\Bigr]_{\theta^{4}} &= -\frac{1}{8}\Bigl(\bar{\theta}\gamma_{5}\gamma_{\mu}\theta\Bigr)^{2}\Biggl\{\Bigl[\phi^{\dag}\square\phi\Bigr]
    +\Bigl[\Bigl(\square\phi^{\dag}\Bigr)\phi\Bigr]\Biggr\} \\
    &\quad +\Bigl(\bar{\theta}\gamma_{5}\gamma_{\mu}\theta\Bigr)\Biggl\{ \Bigl[\Bigl(\overline{\psi_{L}}\,\theta\Bigr)
    \Bigl(\bar{\theta}\gamma^{\mu}\partial_{\mu}\psi_{L}\Bigr)\Bigr]
    +\Bigl[\Bigl((\partial_{\mu}\overline{\psi_{L}})\,\gamma^{\mu}\theta\Bigr)
    \Bigl(\bar{\theta}\,\psi_{L}\Bigr)\Bigr]\Biggr\} \\
    &\quad +\frac{1}{4}\Bigl(\bar{\theta}(1-\gamma_{5})\theta\Bigr)\Bigl(\bar{\theta}(1+\gamma_{5})\theta\Bigr)
    \Bigl[\mathscr{F}^{\dag}\mathscr{F}\Bigr] \\
    &\quad -\frac{1}{4}\Bigl(\bar{\theta}\gamma_{5}\gamma^{\mu}\theta\Bigr)
    \Bigl(\bar{\theta}\gamma_{5}\gamma^{\nu}\theta\Bigr)\Bigl[\partial_{\mu}\phi^{\dag}\partial_{\nu}\phi\Bigr] \\
    &\quad -\frac{\mi}{2}\Bigl(\bar{\theta}\gamma_{5}\gamma^{\mu}\theta\Bigr)
    \Bigl(\bar{\theta}\gamma_{5}\gamma^{\nu}\theta\Bigr)\sum_{A}V_{\mu}^{A}\Biggl\{
    \Bigl[\phi^{\dag}\,t_{A}\,\partial_{\nu}\phi\Bigr]-\Bigl[(\partial_{\nu}\phi^{\dag})\,t_{A}\,\phi\Bigr]\Biggr\} \\
    &\quad -\frac{1}{2}\Bigl(\bar{\theta}\gamma_{5}\gamma^{\mu}\theta\Bigr)
    \Bigl(\bar{\theta}\gamma_{5}\gamma^{\nu}\theta\Bigr)\sum_{AB}V^{A}_{\mu}V^{B}_{\nu}
    \Bigl[\phi^{\dag}\,t_{A}\,t_{B}\,\phi\Bigr] \\
    &\quad -2\mi\Bigl(\bar{\theta}\gamma_{5}\gamma^{\mu}\theta\Bigr)\sum_{A}V_{\mu}^{A}
    \Bigl[\Bigl(\overline{\psi_{L}}\,\theta\Bigr)\,t_{A}\,\Bigl(\bar{\theta}\,\psi_{L}\Bigr)\Bigr] \\
    &\quad -2\mi\sqrt{2}\Bigl(\bar{\theta}\,\gamma_{5}\,\theta\Bigr)
    \sum_{A}\Bigl[\Bigl(\overline{\psi_{L}}\,\theta\Bigr)\,t_{A}\,\Bigl(\bar{\theta}\lambda^{A}\Bigr)\phi\Bigr]\\
    &\quad -2\mi\sqrt{2}\Bigl(\bar{\theta}\,\gamma_{5}\,\theta\Bigr)
    \sum_{A}\Bigl[\phi^{\dag}\,\Bigl(\overline{\lambda^{A}}\theta\Bigr)\,t_{A}\,
    \Bigl(\bar{\theta}\,\psi_{L}\Bigr)\Bigr] \\
    &\quad +\frac{1}{2}\Bigl(\bar{\theta}\,\gamma_{5}\,\theta\Bigr)^{2}
    \sum_{A}D_{A}\Bigl[\phi^{\dag}\,t_{A}\,\phi\Bigr]\:,
\end{align*}
\endgroup
其中我们用方括号表示味指标$\,n,m\,$的标量积, 并继续用圆括号表示\,Dirac\,指标的标量积. 同\,\ref{sec:26.4} 节一样, %
我们可以使用恒等式(\ref{26.A.17})---(\ref{26.A.19})把对$\,\theta\,$的所有依赖写成%
总因子$\,(\bar{\theta}\gamma_{5}\theta)^{2}\,$的形式:
\begin{align*}
    \Bigl[\Phi^{\dag}\,\Gamma\,\Phi\Bigr]_{\theta^{4}} &= \Bigl(\bar{\theta}\gamma_{5}\theta\Bigr)^{2}
    \Biggl\{-\frac{1}{8}\Bigl[\phi^{\dag}\square\phi\Bigr]-\frac{1}{8}\Bigl[\Bigl(\square\phi^{\dag}\Bigr)\phi\Bigr] \\
    &\quad +\frac{1}{4}\Bigl[\Bigl(\overline{\psi_{L}}\,\gamma^{\mu}\partial_{\mu}\psi_{L}\Bigr)\Bigr]
    -\frac{1}{4}\Bigl[\Bigl((\partial_{\mu}\overline{\psi_{L}})\,\gamma^{\mu}\,\psi_{L}\Bigr)\Bigr] \\
    &\quad -\frac{1}{2}\Bigl[\mathscr{F}^{\dag}\mathscr{F}\Bigr]
    +\frac{1}{4}\Bigl[\partial_{\mu}\phi^{\dag}\partial^{\mu}\phi\Bigr] \\
    &\quad +\frac{\mi}{2}\sum_{A}V_{\mu}^{A}\Bigl[\phi^{\dag}\,t_{A}\,\partial^{\mu}\phi\Bigr]
    -\frac{\mi}{2} \sum_{A}V_{\mu}^{A}\Bigl[(\partial^{\mu}\phi^{\dag})\,t_{A}\,\phi\Bigr] \\
    &\quad+\frac{1}{2}\sum_{AB}V_{\mu}^{A}V^{B\mu}\Bigl[\phi^{\dag}\,t_{A}\,t_{B}\,\phi\Bigr]
    -\frac{\mi}{2}\sum_{A}V_{\mu}^{A}\Bigl[\Bigl(\overline{\psi_{L}}\gamma^{\mu}t_{A}\,\psi_{L}\Bigr)\Bigr]\\
    &\quad -\frac{\mi}{\sqrt{2}}\sum_{A}\Bigl[\Bigl(\overline{\psi_{L}}\,t_{A}\,\lambda^{A}\Bigr)\phi\Bigr]
    +\frac{\mi}{\sqrt{2}}\sum_{A}\Bigl[\phi^{\dag}\,\Bigl(\overline{\lambda^{A}}\,t_{A}\,\psi_{L}\Bigr)\Bigr]\\
    &\quad +\frac{1}{2}\sum_{A}D_{A}\,\Bigl[\phi^{\dag}\,t_{A}\,\phi\Bigr]  \Biggr\} \:.
\end{align*}
$D\,$-项是$\,-\tfrac{1}{4}(\bar{\theta}\gamma_{5}\theta)^{2}\,$的系数减去$\,\frac{1}{2}\square$作用在与$\,\theta\,$无关的项上, $[\Phi^{\dag}\Gamma\Phi]\,$中与$\,\theta\,$无关的项是$\,[\phi^{\dag}\phi]$, 所以
\begin{align*}
    \Bigl[\Phi^{\dag}\,\Gamma\,\Phi\Bigr]_{D} &= -2[\partial_{\mu}\phi^{\dag}\partial^{\mu}\phi] \\
    &\quad -\Bigl[\Bigl(\overline{\psi_{L}}\,\gamma^{\mu}\partial_{\mu}\psi_{L}\Bigr)\Bigr]
    +\Bigl[\Bigl((\partial_{\mu}\overline{\psi_{L}})\,\gamma^{\mu}\,\psi_{L}\Bigr)\Bigr]
    +2\Bigl[\mathscr{F}^{\dag}\mathscr{F}\Bigr] \\
    &\quad -2\mi\sum_{A}V_{\mu}^{A}\Bigl[\phi^{\dag}\,t_{A}\,\partial^{\mu}\phi\Bigr]
    +2\mi\sum_{A}V_{\mu}^{A}\Bigl[(\partial^{\mu}\phi^{\dag})\,t_{A}\,\phi\Bigr] \\
    &\quad -2\sum_{AB}V_{\mu}^{A}V^{B\mu}\Bigl[\phi^{\dag}\,t_{A}\,t_{B}\,\phi\Bigr]
    +2\mi\sum_{A}V_{\mu}^{A}\Bigl[\Bigl(\overline{\psi_{L}}\gamma^{\mu}t_{A}\psi_{L}\Bigr)\Bigr] \\
    &\quad +2\mi\sqrt{2}\sum_{A}\Bigl[\Bigl(\overline{\psi_{L}}\,t_{A}\,\lambda^{A}\Bigr)\phi\Bigr]
    -2\mi\sqrt{2}\sum_{A}\Bigl[\phi^{\dag}\Bigl(\overline{\lambda^{A}}\,t_{A}\,\psi_{L}\Bigr)\Bigr] \\
    &\quad -2\sum_{A}D_{A}\Bigl[\phi^{\dag}\,t_{A}\,\phi\Bigr] \:.
\end{align*}
为了看到这{\kai{是}}规范不变的, 我们注意到它可以写成
\begin{align}
    \frac{1}{2}\Bigl[\Phi^{\dag}\,\Gamma\,\Phi\Bigr]_{D} &= -\Bigl[(D_{\mu}\phi)^{\dag}D^{\mu}\phi\Bigr] \nonumber \\
    &\quad -\frac{1}{2}\Bigl[\Bigl(\overline{\psi_{L}}\,\gamma^{\mu}D_{\mu}\psi_{L}\Bigr)\Bigr]
    +\frac{1}{2}\Bigl[\Bigl(\overline{(D_{\mu}\psi_{L})}\,\gamma^{\mu}\,\psi_{L}\Bigr)\Bigr]
    +\Bigl[\mathscr{F}^{\dag}\mathscr{F}\Bigr] \nonumber \\
    &\quad +\mi\sqrt{2}\sum_{A}\Bigl[\Bigl(\overline{\psi_{L}}\,t_{A}\,\lambda^{A}\Bigr)\phi\Bigr]
    -\mi\sqrt{2}\sum_{A}\Bigl[\phi^{\dag}\Bigl(\overline{\lambda^{A}}\,t_{A}\,\psi_{L}\Bigr)\Bigr] \nonumber \\
    &\quad-\sum_{A}D_{A}\Bigl[\phi^{\dag}\,t_{A}\,\phi\Bigr] \:, \label{27.1.27}
\end{align}
其中$\,D_{\mu}\,$是规范不变导数(\textcolor{foo}{15.1.10}):
\begin{equation}
    D_{\mu}\psi_{L}\equiv\partial_{\mu}\psi_{L}-\mi\sum_{A}t_{A}\,V^{A}_{\mu}\,\psi_{L}\:,\qquad
    D_{\mu}\phi\equiv\partial_{\mu}\phi -\mi\sum_{A}t_{A}\,V_{\mu}^{A}\,\phi \:. \label{27.1.28}
\end{equation}
因此, 方程(\ref{27.1.27})现在是左手征超场的标量分量和旋量分量一个合适的规范不变动能拉格朗日量, %
再加上规范微子场与手征超场的标量分量和旋量分量的\,Yukawa\,耦合以及包含辅助场$\,\mathscr{F}_{n}\,$和 $D_{A}\,$的项.

\section{阿贝尔规范超场的规范不变作用量} \label{sec:27.2}


我们现在要考虑如何给包含规范场$\,V_{\mu}^{A}\,$的规范超场$\,V^{A}(x,\theta)\,$构建一个规范不变的超对称作用量. %
为了启发这个构造, 我们将首先考虑单个阿贝尔规范场(扔掉下标\,$A$)的情况, 然后在下一节回到一般情况.

在量子电动力学这样的阿贝尔规范场论中, 用$\,V_{\mu}(x)\,$构造的规范不变场是熟悉的场强张量
\begin{equation}
    f_{\mu\nu}(x) = \partial_{\mu}V_{\nu}(x) - \partial_{\nu}V_{\mu}(x) \:. \label{27.2.1}
\end{equation}
那么, $f_{\mu\nu}(x)\,$的超对称变换规则就由$\,V_{\mu}(x)\,$的变换规则(\ref{26.2.15})给定为
\begin{equation}
    \delta f_{\mu\nu} = \Bigl(\bar{\alpha}(\partial_{\mu}\gamma_{\nu}-\partial_{\nu}\gamma_{\mu})\lambda\Bigr)\:.
    \label{27.2.2}
\end{equation}
方程(\ref{26.2.16})给出了$\,\lambda(x)$,的变换规则
\begin{equation}
    \delta\lambda = \Bigl(-\tfrac{1}{4}f_{\mu\nu}[\gamma^{\mu},\gamma^{\nu}]+\mi\,\gamma_{5}\,D\Bigr)\alpha \:,
    \label{27.2.3}
\end{equation}
而方程(\ref{26.2.17})给出了$\,D(x)\,$的变换规则:
\begin{equation}
    \delta D =\mi\Bigl(\bar{\alpha}\,\gamma_{5}\,\slashed{\partial}\lambda\Bigr) \:. \label{27.2.4}
\end{equation}
其中没有一个依赖于超场$\,V^{A}(x,\theta)\,$是否被取在了\,Wess-Zumino\,规范中. 我们看到场$\,f_{\mu\nu}(x)$, %
$\lambda(x)$ 和$\,D(x)\,$构成了完备的超对称多重态.

给这个超多重态中的场构造一个合适的动能拉格朗日密度并不困难. 这些场的\,Lorentz\,不变, 宇称守恒, %
规范不变且量纲为\,4\,的函数只能是$\,f_{\mu\nu}f^{\mu\nu}$, $\bar{\lambda}\,\slashed{\partial}\lambda\,$和$\,D^{2}$. %
通过取$\,f_{\mu\nu}f^{\mu\nu}\,$的系数为$\,-\frac{1}{4}$, 我们可以使$\,V^{\mu}\,$是按习惯归一化的矢量场, %
所以我们暂且可以将动能拉格朗日密度取成
\[
    \mathscr{L}_{\mathrm{gauge}} = -\tfrac{1}{4}f_{\mu\nu}f^{\mu\nu}-c_{\lambda}\Bigl(\bar{\lambda}\,\slashed{\partial}\lambda\Bigr)
    -c_{D}D^{2} \:,
\]
其中系数$\,c_{\lambda}\,$和$\,c_{D}\,$由$\,\int\mathscr{L}_{\text{guage}}\,\dif^{4}x\,$是超对称的这一条件决定. %
利用方程(\ref{27.2.2})---(\ref{27.2.4}), 无限小超对称变换对拉格朗日中的算符的改变是
\begin{align*}
    \delta\Bigl(f_{\mu\nu}f^{\mu\nu}\Bigr) &= 2f^{\mu\nu}\Bigl(\bar{\alpha}(\gamma_{\nu}\partial_{\mu}-\gamma_{\mu}\partial_{\nu})\lambda\Bigr)\:, \\
    \delta (\bar{\lambda}\,\slashed{\partial}\lambda) &=
    2\Bigl(\bar{\alpha}\Bigl[+\tfrac{1}{4}f_{\mu\nu}[\gamma^{\mu},\gamma^{\nu}]+\mi\,\gamma_{5}\,D\Bigr]
    \slashed{\partial}\lambda\Bigr) \:, \\
    \delta D^{2} &= 2\mi\,D\,\Bigl(\bar{\alpha}\,\gamma_{5}\,\slashed{\partial}\lambda\Bigr) \:,
\end{align*}
其中我们扔掉了对作用量变分无贡献的导数项. 为了看到这些项是如何抵消的, 需要使用$\,\gamma\,$-矩阵的恒等式\footnote{为了推导这个恒等式, 要用到任何$\,4\times4\,$矩阵都能表示成\,5.4\,节描述的\,16\,个独立的协变矩阵的线性组合这个事实, 而在我们的情况中, %
由于\,Lorentz\,不变性和空间反演不变性的限制, 这些项变成了现在的样子. %
这些项的系数可以通过给$\,\mu\nu\rho\,$赋值$\,121\,$和$\,123\,$计算出来.}
\begin{equation}
    [\gamma^{\mu},\gamma^{\nu}]\,\gamma^{\rho} = -2\eta^{\mu\rho}\gamma^{\nu}
    +2\eta^{\nu\rho}\gamma^{\mu}-2\mi\epsilon^{\mu\nu\rho\sigma}\gamma_{\sigma}\gamma_{5} \:. \label{27.2.5}
\end{equation}
$-\mi\epsilon^{\mu\nu\rho\sigma}f_{\mu\nu}(\bar{\alpha}\gamma_{\sigma}\gamma_{5}\partial_{\rho}\lambda)\,$分部积分产生的贡献正比于$\,\epsilon^{\mu\nu\rho\sigma}\partial_{\rho}f_{\mu\nu}$, 这一项由于$\,f_{\mu\nu}\,$的形式(\ref{27.2.1})为零, %
所以$\,-\mi\epsilon^{\mu\nu\rho\sigma}f_{\mu\nu}(\bar{\alpha}\gamma_{\sigma}\gamma_{5}\partial_{\rho}\lambda)\,$这一项对%
$\,\int\dif^{4}x\,\delta\mathscr{L}\,$没有贡献. 这样一来, 这个恒等式使得我们能够将$\,\lambda\,$-项的变分重写成
\[
\delta\Bigl(\bar{\lambda}\,\slashed{\partial}\lambda\Bigr)
=-f^{\mu\nu}\Bigl(\bar{\alpha}(\gamma_{\nu}\partial_{\mu}-\gamma_{\mu}\partial_{\nu})\lambda\Bigr)
+2\mi D\Bigl(\bar{\alpha}\,\gamma_{5}\,\slashed{\partial}\lambda\Bigr) \:.
\]
抵消正比$\,f^{\mu\nu}\lambda\,$的项要求$\,c_{\lambda}=1/2$, 而抵消正比于$\,D\lambda\,$的项要求$\,c_{D}=-c_{\lambda}$, %
所以超对称拉格朗日密度采取如下的形式
\begin{equation}
    \mathscr{L}_{\mathrm{guage}}=-\tfrac{1}{4}f_{\mu\nu}f^{\mu\nu}-\tfrac{1}{2}
    \Bigl(\bar{\lambda}\,\slashed{\partial}\lambda\Bigr) + \tfrac{1}{2}D^{2} \:. \label{27.2.6}
\end{equation}
这表明, 在场$\,V^{\mu}\,$正则归一化的情况下, %
通过变换规则(\ref{27.2.2})和(\ref{27.2.3})与$\,V^{\mu}\,$相关联的场$\,\lambda\,$也是正则归一化的.

另外, 阿贝尔规范理论还有一个可重整项, 称为\,\textit{Fayet-Iliopoulous}\,{\kai{项}}:\cite{2}
\begin{equation}
    \mathscr{L}_{\mathrm{FI}} = \xi\,D \:, \label{27.2.7}
\end{equation}
其中$\,\xi\,$是任意常数. 通过方程(\ref{27.2.4})可以证明这一项在超对称变换下的变分是导数, %
这使得它产生了作用量中另外一个超对称项. 正如我们将在\,\ref{sec:27.5}\,节看到的, %
这种项的出现为超对称的自发破缺提供了一个机制.

哪类超场是以$\,f_{\mu\nu}$, $\lambda\,$和$\,D\,$为分量场, 提这个问题是有益的, 这一方面是因为是其自身, %
另一方面是它可以作为工具来构建包含这些场的超对称相互作用. 有些出人意料的是, 结果是{\kai{旋量}}超场$\,W_{\alpha}(x)$, %
它的分量场(在方程(\ref{26.2.10})的符号约定下)是
\begin{align}
    &C_{(\alpha)}(x) = \lambda_{\alpha}(x) \:, \nonumber \\
    &\omega_{(\alpha)\beta}= \tfrac{1}{2}\mi\Bigl(\gamma^{\mu}\gamma^{\nu}\epsilon\Bigr)_{\alpha\beta}f_{\mu\nu}(x)
    +(\gamma_{5}\epsilon)_{\alpha\beta}D(x) \:, \nonumber \\
    &V_{(\alpha)\mu}(x) = -\mi\partial_{\mu}\Bigl(\gamma_{5}\lambda(x)\Bigr)_{\alpha} \:,\label{27.2.8} \\
    &M_{(\alpha)}(x)=-\mi\Bigl(\slashed{\partial}\gamma_{5}\lambda(x)\Bigr)_{\alpha} \:,\qquad
    N_{(\alpha)}(x)=- \Bigl(\slashed{\partial}\lambda(x)\Bigr)_{\alpha} \:,  \nonumber \\
    &\lambda_{(\alpha)\beta}(x) = D_{(\alpha)}(x)=0 \:. \nonumber
\end{align}
(把这些分量场的下标$\,\alpha\,$放在括号里面是为了强调它标记的是整个超场.) 可以直接用方程(\ref{27.2.2}) ---(\ref{27.2.4})验证方程(\ref{27.2.8})给出的超场分量确实像方程(\ref{26.2.11})---(\ref{26.2.17})那样中变换.

将分量场(\ref{27.2.8})代入方程(\ref{26.2.10})并使用方程(\ref{26.A.5}), 我们发现超场$\,W_{\alpha}\,$采取如下的形式
\begin{align}
    W_{\alpha}(x,\theta) &=\Biggl[\lambda(x)+\tfrac{1}{2}\gamma^{\mu}\gamma^{\nu}\,\theta\,f_{\mu\nu}(x)
    -\mi\gamma_{5}\theta\,D(x) - \tfrac{1}{2}\Bigl(\theta^{\mathrm{T}}\epsilon\theta\Bigr)\,
    \slashed{\partial}\gamma_{5}\lambda(x) \nonumber \\
    &\quad+\tfrac{1}{2}\Bigl(\theta^{\mathrm{T}}\epsilon\gamma_{5}\theta\Bigr)\,\slashed{\partial}\lambda(x)
    +\tfrac{1}{2}\Bigl(\theta^{\mathrm{T}}\epsilon\gamma^{\mu}\theta\Bigr)\gamma_{5}\partial_{\mu}\lambda(x) \nonumber\\
    &\quad-\tfrac{1}{4}\Bigl(\theta^{\mathrm{T}}\epsilon\theta\Bigr)\gamma_{5}\gamma^{\mu}\gamma^{\nu}\gamma^{\sigma}\,
    \theta\,\partial_{\sigma}f_{\mu\nu}(x)  \nonumber \\
    &\quad +\tfrac{1}{2}\mi\Bigl(\theta^{\mathrm{T}}\epsilon\theta\Bigr)\gamma^{\sigma}\,\theta\,\partial_{\sigma}D(x)
    -\tfrac{1}{8}\Bigl(\theta^{\mathrm{T}}\epsilon\theta\Bigr)^{2}\square\lambda(x)\Biggr]_{\alpha} \:. \label{27.2.9}
\end{align}
就像我们在\,\ref{sec:26.3}\,节证明过的, 像这样没有$\,\lambda\,$-分量和$\,D\,$-分量的超场是{\kai{手征的}}------即, %
它是左手征超场和右手征超场的和
\begin{equation}
    W(x,\theta)=W_{L}(x,\theta)+W_{R}(x,\theta) \:. \label{27.2.10}
\end{equation}
这里的左手征超场和右手征超场分别是$\,W\,$在$\,\gamma_{5}=+1\,$的超空间和在$\,\gamma_{5}=-1\,$的超空间上的投影:
\begin{align}
    W_{L}(x,\theta)&=\tfrac{1}{2}(1+\gamma_{5})W(x,\theta) \nonumber \\
    &=\lambda_{L}(x_{+})+\tfrac{1}{2}\gamma^{\mu}\gamma^{\nu}\theta_{L}f_{\mu\nu}(x_{+})
    +\Bigl(\theta_{L}^{\mathrm{T}}\epsilon\theta_{L}\Bigr)\,\slashed{\partial}\lambda_{R}(x_{+})
    -\mi\theta_{L}D(x_{+}) \:, \label{27.2.11} \\
    W_{R}(x,\theta)&=\tfrac{1}{2}(1-\gamma_{5})W(x,\theta) \nonumber \\
    &=\lambda_{R}(x_{-})+\tfrac{1}{2}\gamma^{\mu}\gamma^{\nu}\theta_{R}f_{\mu\nu}(x_{-})
    -\Bigl(\theta_{R}^{\mathrm{T}}\epsilon\theta_{R}\Bigr)\,\slashed{\partial}\lambda_{L}(x_{-})
    -\mi\theta_{R}D(x_{-}) \:, \label{27.2.12}
\end{align}
其中$\,x_{\pm}^{\mu}\,$由方程(\ref{26.3.23})给出.

正如我们在\,\ref{sec:26.3}\,节看到的, 我们可以用一个左手征超场的任意函数的\,$\mathscr{F}$\,-项和它的厄米共轭来构建合适的拉格朗日密度. 左手征超场(\ref{27.2.11})的最简单标量函数是$\,\sum_{\alpha\beta}\epsilon_{\alpha\beta}W_{L\alpha}W_{L\beta}$. %
为了计算$\,\mathscr{F}\,$-项, 我们注意到, 当表示成$\,\theta_{L}\,$和$\,x_{+}\,$的函数时, %
$\sum_{\alpha\beta}\epsilon_{\alpha\beta}W_{L\alpha}W_{L\beta}\,$中$\,\theta_{L}\,$的二阶项是
\begin{align*}
    {-}\Bigl[\sum_{\alpha\beta}\epsilon_{\alpha\beta}W_{L\alpha}W_{L\beta}\Bigr]_{\theta_{L}^{2}}
    &= \Bigl(\theta_{L}^{\mathrm{T}}\epsilon\theta_{L}\Bigr)
    \Bigl[-2\Bigl(\lambda_{L}^{\mathrm{T}}(x)\epsilon\,\slashed{\partial}\lambda_{R}(x)\Bigr)+D^{2}(x)\Bigr] \\
    &\quad +\frac{1}{16}\Bigl(\overline{\theta_{L}}[\gamma^{\mu},\gamma^{\nu}]\,
    [\gamma^{\rho},\gamma^{\sigma}]\theta_{L}\Bigr) f_{\mu\nu}(x)f_{\rho\sigma}(x) \:.
\end{align*}
(场变量取成了$\,x^{\mu}\,$而不是$\,x_{+}^{\mu}\,$是因为差产生的项至少包含\,3\,个$\,\theta_{L}\,$的因子, 因此为零.) %
$(\bar{s}[\gamma_{\mu},\gamma_{\nu}]s)\,$和\\$\,(\bar{s}[\gamma_{\mu},\gamma_{\nu}]\gamma_{5}s)\,$对于%
任何\,Majorana\,旋量$\,s\,$都为零这个性质, 再加上\,Lorentz\,不变性告诉我们, %
双线性型$\,(\overline{\theta_{L}}[\gamma^{\mu},\gamma^{\nu}]\,[\gamma^{\rho},\gamma^{\sigma}])\theta_{L}\,$必须正比于%
$\,(\overline{\theta_{L}}\theta_{L})(\eta^{\mu\rho}\eta^{\nu\sigma}-\eta^{\mu\sigma}\eta^{\nu\rho})\,$和%
$\,(\overline{\theta_{L}}\theta_{L})\epsilon^{\mu\nu\rho\sigma}\,$的线性组合. %
通过给$\,\mu\nu\rho\sigma\,$赋值$\,1212\,$或$\,1230$, 我们可以找到系数, 并且以这种方式, 我们发现
\[
\Bigl(\overline{\theta_{L}}[\gamma^{\mu},\gamma^{\nu}]\,[\gamma^{\rho},\gamma^{\sigma}]\theta_{L}\Bigr)
=4\Bigl(\overline{\theta_{L}}\theta_{L}\Bigr)\Bigl[-\eta^{\mu\rho}\eta^{\nu\sigma}+\eta^{\mu\sigma}\eta^{\nu\rho}
 +\mi\epsilon^{\mu\nu\rho\sigma}\Bigr] \:.
\]
$\mathscr{F}\,$-项是$\,(\overline{\theta_{L}}\theta_{L})\,$的系数, 所以
\begin{equation}
    {-}\Bigl[\sum_{\alpha\beta}\epsilon_{\alpha\beta}W_{L\alpha}W_{L\beta}\Bigr]_{\mathscr{F}}
    =-2\Bigl(\overline{\lambda_{R}}\,\slashed{\partial}\lambda_{R}\Bigr)-\frac{1}{2}f_{\mu\nu}f^{\mu\nu}
    +\frac{\mi}{4}\epsilon^{\mu\nu\rho\sigma}f_{\mu\nu}f_{\rho\sigma}+D^{2} \:. \label{27.2.13}
\end{equation}
方程(\ref{26.A.1})表明$\,(\bar{\lambda}\,\slashed{\partial}\lambda)\,$是实的, %
而$\,(\bar{\lambda}\,\slashed{\partial}\gamma_{5}\lambda)\,$是虚的, %
所以方程(\ref{27.2.13})给出了规范场和规范微子场的拉格朗日量(\ref{27.2.6})
\begin{equation}
{-}\frac{1}{2}\operatorname{Re}\Bigl[\sum_{\alpha\beta}\epsilon_{\alpha\beta}W_{L\alpha}W_{L\beta}\Bigr]_{\mathscr{F}}
=-\frac{1}{2}\Bigl(\bar{\lambda}\,\slashed{\partial}\lambda\Bigr)-\frac{1}{4}f_{\mu\nu}f^{\mu\nu}+\frac{1}{2}D^{2}\:.
\label{27.2.14}
\end{equation}
在下一节, 虚部的物理意义会在更普遍的背景下进行讨论.

推导旋量超场的形式有另一种方法, 而这个方法在非阿贝尔规范理论中推导规范超场的分量时提供了一个更加方便的方法. %
一个繁琐但直接的计算表明规范不变超场(\ref{27.2.9})可以用规范超场(\ref{27.1.16})表示成
\begin{equation}
    W_{\alpha}(x,\theta)=\frac{\mi}{4}\Bigl(\mathscr{D}^{\mathrm{T}}\epsilon\mathscr{D}\Bigr)\mathscr{D}_{\alpha}\,
    V(x,\theta) \:, \label{27.2.15}
\end{equation}
其中$\,\mathscr{D}_{\alpha}\,$是方程(\ref{26.2.26})中引入的超导数:
\[
\mathscr{D}_{\alpha} \equiv \sum_{\beta}(\gamma_{5}\epsilon)_{\alpha\beta}\frac{\partial}{\partial\theta_{\beta}}
-(\gamma^{\mu}\theta)_{\alpha}\frac{\partial}{\partial x^{\mu}} =-\frac{\partial}{\partial\bar{\theta}_{\alpha}}
-(\gamma^{\mu}\theta)_{\alpha}\frac{\partial}{\partial x^{\mu}} \:.
\]
得到这个结果(除了归一化因子)的一个方法是注意到函数(\ref{27.2.15})拥有称为规范不变手征旋量超场的所需性质. %
首先, 注意到方程(\ref{27.2.15}){\kai{是}}一个超场, 这是因为它是通过用超导数作用在超场$\,V\,$上形成的. %
另外, 从$\,\mathscr{D}\,$的反对易可以得出任意三个或多个$\,\mathscr{D}_{L}\,$的乘积或者%
任意三个或多个$\,\mathscr{D}_{R}\,$的乘积为零, 这使得
\begin{equation}
    \Bigl(\mathscr{D}^{\mathrm{T}}\epsilon\mathscr{D}\Bigr)\mathscr{D}
    =\Bigl(\mathscr{D}_{L}^{\mathrm{T}}\epsilon\mathscr{D}_{L}\Bigr)\mathscr{D}_{R}
    +\Bigl(\mathscr{D}_{R}^{\mathrm{T}}\epsilon\mathscr{D}_{R}\Bigr)\mathscr{D}_{L} \:. \label{27.2.16}
\end{equation}
因为$\,\mathscr{D}_{L}(\mathscr{D}_{L}^{\mathrm{T}}\epsilon\mathscr{D}_{L})=
\mathscr{D}_{R}(\mathscr{D}_{R}^{\mathrm{T}}\epsilon\mathscr{D}_{R})=0$, 超场(\ref{27.2.15})是手征的, 且有
\begin{equation}
    W_{L\alpha}(x,\theta) = \frac{\mi}{4}\Bigl(\mathscr{D}_{R}^{\mathrm{T}}\epsilon\mathscr{D}_{R}\Bigr)
    \mathscr{D}_{L\alpha}\,V(x,\theta)\:,\qquad
    W_{R\alpha}(x,\theta)=\frac{\mi}{4}\Bigl(\mathscr{D}^{\mathrm{T}}_{L}\epsilon\mathscr{D}_{L}\Bigr)
    \mathscr{D}_{R\alpha}\,V(x,\theta) \:. \label{27.2.17}
\end{equation}
最后, 我们可以证明(\ref{27.2.15})在推广的规范变换(\ref{27.1.13})下不变, 对于单个阿贝尔规范场, 这个规范变换就是
\begin{equation}
    V(x,\theta) \to V(x,\theta)+\frac{\mi}{2}\Bigl[\Omega(x,\theta)-\Omega^{\ast}(x,\theta)\Bigr]\:, \label{27.2.18}
\end{equation}
其中$\,\Omega(x,\theta)\,$是一个任意的左手征超场. 由于$\,\mathscr{D}_{L}\Omega^{\ast}=0$, %
$W_{L\alpha}\,$的变化正比于$\,(\mathscr{D}_{R}^{\mathrm{T}}\epsilon\mathscr{D}_{R})\mathscr{D}_{L\alpha}\Omega$. %
但是$\,\mathscr{D}_{R}\Omega=0\,$且
\[
\Bigl[(\mathscr{D}_{\mathrm{R}}^{\mathrm{T}}\epsilon\mathscr{D}_{R}),\mathscr{D}_{L\alpha}\Bigr]
=-2\Bigl[(1+\gamma_{5})\,\slashed{\partial}\mathscr{D}_{R}\Bigr]_{\alpha} \:,
\]
所以$\,W_{L\alpha}\,$的变化为零. 类似的讨论表明$\,W_{R\alpha}\,$也是规范不变的. %
(通过使用这一规范不变性质将 $V(x,\theta)\,$变到\,Wess-Zumino\,规范下, 验证方程(\ref{27.2.15})的工作量被极大地简化.)

手征超场(\ref{27.2.11})和(\ref{27.2.12})显然不是左手征超场和右手征超场的最一般形式. 为了把这些超场满足的约束变成明显超对称的形式, 我们通过使用反对易关系(\ref{26.2.30})注意到
\begin{align}
    \epsilon_{\alpha\beta}\mathscr{D}_{L\alpha}\Bigl(\mathscr{D}_{R}^{\mathrm{T}}\epsilon\mathscr{D}_{R}\Bigr)
    \mathscr{D}_{L\beta}&=-2\mathscr{D}_{R\alpha}\mathscr{D}_{L\beta}
    \Bigl(\epsilon(1+\gamma_{5})\,\slashed{\partial}\Bigr)_{\beta\alpha}
    +\Bigl(\mathscr{D}_{R}^{\mathrm{T}}\epsilon\mathscr{D}_{R}\Bigr)
    \Bigl(\mathscr{D}_{L}^{\mathrm{T}}\epsilon\mathscr{D}_{L}\Bigr) \nonumber \\
    &= \epsilon_{\alpha\beta}\mathscr{D}_{R\alpha}\Bigl(\mathscr{D}_{L}^{\mathrm{T}}\epsilon\mathscr{D}_{L}\Bigr)
    \mathscr{D}_{R\beta} \:. \label{27.2.19}
\end{align}
那么从方程(\ref{27.2.17})得出$\,W_{L}\,$和$\,W_{R}\,$通过如下的约束相关联:
\begin{equation}
    \epsilon_{\alpha\beta}\mathscr{D}_{L\alpha}W_{L\beta}=\epsilon_{\alpha\beta}\mathscr{D}_{R\alpha}W_{R\beta}\:.
    \label{27.2.20}
\end{equation}
可以直接证明满足方程(\ref{27.2.20})的最一般手征旋量超场是(\ref{27.2.11})和(\ref{27.2.12})的形式, 且其中的 $f_{\mu\nu}\,$满足\,``Bianchi''\,恒等式$\,\epsilon^{\mu\nu\rho\sigma}\partial_{\rho}f_{\mu\nu}=0\,$的约束.

\section{一般规范超场的规范不变作用量} \label{sec:27.3}

我们上一节对超对称阿贝尔规范理论的经验表明, 在一般的非阿贝尔规范理论中, 场$\,V_{\mu}^{A}(x)$, $\lambda^{A}(x)\,$%
和$\,D^{A}(x)\,$的动能拉格朗日量应该作为方程(\ref{27.2.6})的规范不变推广的一部分出现:
\begin{equation}
    \mathscr{L}_{\mathrm{gauge}}=-\tfrac{1}{4}\sum_{A}f_{A\mu\nu}f^{\mu\nu}_{A}
    -\tfrac{1}{2}\sum_{A}\Bigl(\overline{\lambda_{A}}(\slashed{D}\lambda)_{A}\Bigr)
    +\tfrac{1}{2}\sum_{A}D_{A}D_{A} \:. \label{27.3.1}
\end{equation}
在我们现在使用的\,Lie\,代数的基中, 结构常数是全反对称的, 因此我们不区分上群指标和下群指标, 将所有指标$\,A,B\,$等写为下标. %
另外, $f_{A\mu\nu}\,$是规范协变的场强张量
\begin{equation}
    f_{A\mu\nu}=\partial_{\mu}V_{A\nu}-\partial_{\nu}V_{A\mu}+\sum_{BC}C_{ABC}V_{B\mu}V_{C\nu} \:, \label{27.3.2}
\end{equation}
$D_{\mu}\lambda\,$是规范微子场的规范协变导数, 它在伴随表示下是
\begin{equation}
    (D_{\mu}\lambda)_{A} = \partial_{\mu}\lambda_{A}+\sum_{BC}C_{ABC}V_{B\mu}\lambda_{C} \:. \label{27.3.3}
\end{equation}
问题是: 方程(\ref{27.3.1})是否给出了一个超对称的作用量?

由于拉格朗日密度(\ref{27.3.1})是明显规范不变的, 我们可以在任何方便的规范下检验这个作用量是否是超对称的. %
为了查明$\,\delta\mathscr{L}_{\text{gauge}}\,$是否在某个点$\,X^{\mu}\,$是导数, 采取\,Wess-Zumino\,规范的一个特定版本, %
$V_{A}^{\mu}(X)=0$, 将是方便的. 那么在$\,X\,$处, 分量场的变化由方程(\ref{26.2.15})---(\ref{26.2.17})取在$\,x=X\,$处给出
\begin{gather}
    \delta V_{A\mu} = \Bigl(\bar{\alpha}\,\gamma_{\mu}\lambda_{A}\Bigr) \:, \label{27.3.4} \\
    \delta\lambda_{A} =\Bigl(\frac{1}{4}f_{A\mu\nu}\,[\gamma^{\nu},\gamma^{\mu}]+\mi\gamma_{5}D_{A}\Bigr)\alpha \:, \label{27.3.5} \\
    \delta D_{A}=\mi\Bigl(\bar{\alpha}\,\gamma_{5}\,\slashed{\partial}\lambda_{A}\Bigr) \:. \label{27.3.6}
\end{gather}
(我们必须在计算超对称变换下的变化{\kai{之后}}, 而不是之前, 令这些表达式中的$\,x^{\mu}\,$等于$\,X^{\mu}$.) 另外, $f_{A}^{\mu\nu}\,$中的非线性项关于$\,V\,$是二次的, 因此它们在$\,x=X\,$处的变分为零, 所以在$\,x=X\,$处
\begin{equation}
    \delta f_{A\mu\nu} =\Bigl(\bar{\alpha}\,(\gamma_{\nu}\partial_{\mu}-\gamma_{\mu}\partial_{\nu})\lambda_{A}\Bigr)\:.
    \label{27.3.7}
\end{equation}
除了一个例外, 方程(\ref{27.3.1})中的项和它们在超对称变换下的变换就是上节讨论的阿贝尔理论的数个副本(由$\,A\,$标记), %
因此给出了一个超对称的作用量. %
可能会扰乱这个作用量的超对称性的那个例外来源于规范微子的规范协变导数(\ref{27.3.3})的第二项:
\begin{equation}
    \mathscr{L}_{\lambda\lambda V}= -\tfrac{1}{2}\sum_{ABC}C_{ABC}\Bigl(\overline{\lambda_{A}}\,\slashed{V}_{B}\lambda_{C}\Bigr) \:, \label{27.3.8}
\end{equation}
它在$\,x=X\,$处的变分是
\begin{equation}
    \delta \mathscr{L}_{\lambda\lambda V} = -\tfrac{1}{2}\sum_{ABC}C_{ABC}\Bigl(\overline{\lambda_{A}}(\delta\slashed{V}_{B})\lambda_{C}\Bigr)
    =-\tfrac{1}{2}\sum_{ABC}C_{ABC}\Bigl(\overline{\lambda_{A}}\gamma_{\mu}\lambda_{C}\Bigr)\,
    \Bigl(\bar{\alpha}\gamma^{\mu}\lambda_{B}\Bigr) \:. \label{27.3.9}
\end{equation}
我们可以将右边双线性型的乘积写为两项的和
\begin{equation}
    \Bigl(\overline{\lambda_{A}}\gamma_{\mu}\lambda_{C}\Bigr)\Bigl(\bar{\alpha}\gamma^{\mu}\lambda_{B}\Bigr)
    =X_{ABC}+Y_{ABC} \:,
\end{equation}
其中
\begin{align*}
    X_{ABC} &\equiv \tfrac{1}{4}\sum_{\pm}\Bigl(\overline{\lambda_{A}}(1\pm\gamma_{5})\gamma_{\mu}\lambda_{C}\Bigr)
    \Bigl(\bar{\alpha}\gamma^{\mu}(1\pm\gamma_{5})\lambda_{B}\Bigr) \:,\\
    Y_{ABC} &\equiv \tfrac{1}{4}\sum_{\pm}\Bigl(\overline{\lambda_{A}}(1\pm\gamma_{5})\gamma_{\mu}\lambda_{C}\Bigr)
    \Bigl(\bar{\alpha}\gamma^{\mu}(1\mp\gamma_{5})\lambda_{B}\Bigr) \:.
\end{align*}
通过使用标准\,Fierz\,恒等式和旋量场的反对易子, 我们有
\begin{align*}
   \Bigl(\overline{\lambda_{A}}(1\pm\gamma_{5})\gamma_{\mu}\lambda_{B}\Bigr)
    \Bigl(\bar{\alpha}\gamma^{\mu}(1\pm\gamma_{5})\lambda_{C}\Bigr) &=
    \Bigl(\overline{\lambda_{A}}(1\pm\gamma_{5})\gamma_{\mu}\lambda_{C}\Bigr)
    \Bigl(\bar{\alpha}\gamma^{\mu}(1\pm\gamma_{5})\lambda_{B}\Bigr)
    \:,\\
  \Bigl(\overline{\lambda_{A}}(1\pm\gamma_{5})\gamma_{\mu}\lambda_{B}\Bigr)
    \Bigl(\bar{\alpha}\gamma^{\mu}(1\mp\gamma_{5})\lambda_{C}\Bigr) &=
    \Bigl(\overline{\lambda_{C}}(1\pm\gamma_{5})\gamma_{\mu}\lambda_{B}\Bigr)
    \Bigl(\bar{\alpha}\gamma^{\mu}(1\mp\gamma_{5})\lambda_{A}\Bigr)\:.
\end{align*}
(为了推导这些关系中的第一个, %
我们注意到$\,[(1\pm\gamma_{5})\gamma_{\mu}]_{\alpha\gamma}[(1\pm\gamma_{5})\gamma^{\mu}]_{\alpha\beta}\,$可以认为是一个依赖于$\,\delta\,$和$\,\gamma\,$的矩阵的矩阵元$\,\alpha\beta$, 因此可以展成$\,1_{\alpha\beta}$, $\gamma^{\mu}_{\alpha\beta}$, %
$[\gamma^{\mu},\gamma^{\kappa}]_{\alpha\beta}$, $(\gamma_{5}\gamma^{\mu})_{\alpha\beta}\,$和$\,(\gamma_{5})_{\alpha\beta}$. 由于因子$\,(1\pm\gamma_{5})$, 展开中只有正比于$\,[(1\pm\gamma_{5})\gamma^{\mu}]_{\alpha\beta}\,$的项. %
Lorentz\,不变性和以及另一个$\,1\pm\gamma_{5}\,$因子的出现告诉我们这个展开采取如下的形式
\[
[(1\pm\gamma_{5})\gamma_{\mu}]_{\alpha\gamma}[(1\pm\gamma_{5})\gamma^{\mu}]_{\delta\beta}
=k[(1\pm\gamma_{5})\gamma_{\mu}]_{\alpha\beta}[(1\pm\gamma_{5})\gamma^{\mu}]_{\delta\gamma} \:.
\]
为了确定比例常数$\,k$, 我们可以用$\,(\gamma_{\nu})_{\gamma\alpha}\,$收缩两边, 并发现$\,k=-1$. %
这个负号被$\,\lambda_{C}\,$和$\,\bar{\alpha}\,$反对易产生的负号抵消了. %
除了要使用\,Majorana\,双线性型的对称性质(\ref{26.A.7}), 证明另一个\,Fierz\,恒等式的方法是相同的.) %
因此$\,X_{ABC}\,$关于$\,B\,$和$\,C\,$的交换是对称的, 而$\,Y_{ABC}\,$关于$\,A\,$和$\,B\,$的交换是对称的. %
由于$\,C_{ABC}\,$是全反对称的, $X_{ABC}\,$和$\,Y_{ABC}\,$对方程(\ref{27.3.9})中的和没有贡献, %
留给我们$\delta \mathscr{L}_{\lambda\lambda V}=0$, 使得方程(\ref{27.3.1})给出的作用量是超对称的, 而这正是我们所要证明的.

通过找出那个拥有$\,f_{A\mu\nu}$, $\lambda_{A}\,$和$\,D_{A}\,$作为分量场的超场, %
我们可以理解{\kai{为什么}}方程(\ref{27.3.1})给出了一个超对称作用量. 回忆, 在一个推广的规范变换下, %
矢量超场$\,V_{A}(x,\theta)\,$有变换性质(\ref{27.1.12}):
\begin{align}
    &\exp\Biggl(-2\sum_{A}t_{A}V_{A}(x,\theta)\Biggr)\to \exp\Biggl(-\mi\sum_{A}t_{A}\Omega_{A}(x,\theta)\Biggr)\nonumber\\
    &\quad \times \exp\Biggl(-2\sum_{A}t_{A}V_{A}(x,\theta)\Biggr)
    \exp\Biggl(+\mi\sum_{A}t_{A}\Omega_{A}^{\ast}(x,\theta)\Biggr) \:, \label{27.3.10}
\end{align}
其中$\,\Omega_{A}(x,\theta)\,$是一个一般的左手征超场. 因为$\,\Omega^{\ast}_{A}\neq\Omega_{A}$, %
所以这不是一个规范协变的变换规则. 为了消除包含$\,\Omega_{A}^{\ast}\,$的因子, %
我们注意到$\,\Omega^{\ast}_{A}\,$是右手征超场, 这使得$\,\mathscr{D}_{L\alpha}\Omega_{A}^{\ast}=0$, 因此
\begin{align}
    &\exp\Biggl(-2\sum_{A}t_{A}V_{A}(x,\theta)\Biggr)\,\mathscr{D}_{L\alpha}
    \exp\Biggl(+2\sum_{A}t_{A}V_{A}(x,\theta)\Biggr) \nonumber \\
    &\quad \to \exp\Biggl(-\mi\sum_{A}t_{A}\Omega_{A}(x,\theta)\Biggr)
    \exp\Biggl(-2\sum_{A}t_{A}V_{A}(x,\theta)\Biggr) \nonumber \\
    &\quad \times \mathscr{D}_{L\alpha}\Biggl[\exp\Biggl(+2\sum_{A}t_{A}V_{A}(x,\theta)\Biggr)
    \exp\Biggl(+\mi\sum_{A}t_{A}\Omega_{A}(x,\theta)\Biggr)    \Biggr] \:. \label{27.3.11}
\end{align}
因为左超导数$\,\mathscr{D}_{L\alpha}\,$既作用在$\,\exp(+\mi\sum_{A}t_{A}\Omega_{A}(x,\theta))\,$上%
又作用在$\,\exp(+2\sum_{A}t_{A}V_{A}(x,\theta)),$上, 这仍然不是规范协变的. 如果我们沿用上一节讨论的阿贝尔理论的推导, %
并定义旋量超场
\begin{align}
    2\sum_{A}t_{A}W_{AL\alpha}(x,\theta) &\equiv \sum_{\beta\gamma}\epsilon_{\beta\gamma}\mathscr{D}_{R\beta}
    \mathscr{D}_{R\gamma}\,\Biggl[\exp\Biggl(-2\sum_{A}t_{A}V_{A}(x,\theta)\Biggr) \nonumber \\
    &\quad \times \mathscr{D}_{L\alpha}\,\exp\Biggl(+2\sum_{A}t_{A}V_{A}(x,\theta)\Biggr)\Biggr] \:,\label{27.3.12}
\end{align}
这个因子可以被消掉. 因为任何三个$\,\mathscr{D}_{R}\,$的乘积为零, $W_{AL\alpha}\,$是左手征的
\begin{equation}
    \mathscr{D}_{R\beta}W_{AL\alpha}(x,\theta) = 0 \:, \label{27.3.13}
\end{equation}
又因为$\,\mathscr{D}_{R\beta}\mathscr{D}_{R\gamma}\mathscr{D}_{L\alpha}\Omega_{A}\propto\mathscr{D}_{R\delta}\Omega_{A}=0$,
对于一个推广的规范变换
\begin{align}
    \sum_{A}t_{A}W_{AL\alpha}(x,\theta) &\to \exp\Bigl(-\mi\sum_{A}t_{A}\Omega_{A}(x,\theta)\Bigr)\,
    \sum_{A}t_{A}W_{AL\alpha}(x,\theta) \nonumber \\
    &\quad \times \exp\Bigl(+\mi\sum_{A}t_{A}\Omega_{A}(x,\theta)\Bigr) \:, \label{27.3.14}
\end{align}
所以$\,W_{AL\alpha}\,$在如上的意义下是规范协变的.

为了计算$\,x=X\,$处的旋量场, 我们可以再次使用\,Wess-Zumion\,规范的$\,V_{A}(X)=0\,$的版本, 在一个直接计算后, %
我们发现在这个规范下
\begin{align*}
    W_{AL}(x,\theta) &= \lambda_{AL}(X_{+}) +\tfrac{1}{2}\gamma^{\mu}\gamma^{\nu}\theta_{L}
    \Bigl(\partial_{\mu}V_{A\nu}(X_{+})-\partial_{\nu}V_{A\mu}(X_{+})\Bigr) \\
    &\quad + \Bigl(\theta_{L}^{\mathrm{T}}\epsilon\theta_{L}\Bigr)\,\slashed{\partial}\lambda_{RA}(X_{+})
    -\mi\theta_{L}D_{A}(X_{+}) \:.
\end{align*}
由于$\,W_{AL}\,$是规范协变的, 在一个一般规范下, 它在一般的点上必须有值
\begin{equation}
    W_{AL}(x,\theta) = \lambda_{AL}(x_{+}) + \tfrac{1}{2}\gamma^{\mu}\gamma^{\nu}\theta_{L}f_{A\mu\nu}(x_{+})
    +\Bigl(\theta_{L}^{\mathrm{T}}\epsilon\theta_{L}\Bigr)\slashed{D}\lambda_{RA}(x_{+})
    -\mi\theta_{L}D_{A}(x_{+}) \:. \label{27.3.15}
\end{equation}
由此, 我们可以用$\,W\,$的双线性型构建一个\,Lorentz\,不变且规范不变的$\,\mathscr{F}\,$-项
\begin{align}
    -\Bigl[\sum_{A\alpha\beta}W_{AL\alpha}W_{AL\beta}\Bigr]_{\mathscr{F}} &=
    \sum_{A}\Biggl[-\Bigl(\overline{\lambda_{A}}\,\slashed{D}(1-\gamma_{5})\lambda_{A}\Bigr)
    -\frac{1}{2}f_{A\mu\nu}f_{A}^{\mu\nu} \nonumber \\
    &\phantom{=\sum_{A}}+\frac{\mi}{4}\epsilon_{\mu\nu\rho\sigma}f^{\mu\nu}_{A}f^{\rho\sigma}_{A}+D_{A}^{2}\Biggr]\:.
    \label{27.3.16}
\end{align}
同上一节一样, 从这个$\,\mathscr{F}$-项的实部获得了规范不变拉格朗日量(\ref{27.3.1}):
\begin{equation}
    {-}\frac{1}{2}\operatorname{Re}
    \Bigl[\sum_{A\alpha\beta}\epsilon_{\alpha\beta}W_{AL\alpha}W_{AL\beta}\Bigr]_{\mathscr{F}}
    =\mathscr{L}_{\mathrm{gauge}} \:. \label{27.3.17}
\end{equation}
虚部又怎么样呢? 它是
\begin{equation}
    {-}\operatorname{Im}
    \Bigl[\sum_{A\alpha\beta}\epsilon_{\alpha\beta}W_{AL\alpha}W_{AL\beta}\Bigr]_{\mathscr{F}}
    =-\mi\sum_{A}\Bigl(\overline{\lambda_{A}}\,\slashed{D}\gamma_{5}\lambda_{A}\Bigr)
    +\frac{1}{4}\epsilon_{\mu\nu\rho\sigma}\sum_{A}f_{A}^{\mu\nu}f_{A}^{\rho\sigma} \:. \label{27.3.18}
\end{equation}
方程(\ref{26.A.7})和结构常数的反对易性表明$\,(\overline{\lambda_{A}}\,\slashed{D}\gamma_{5}\lambda_{A})
=\frac{1}{2}\partial_{\mu}(\overline{\lambda_{A}}\gamma^{\mu}\gamma_{5}\lambda_{A})$, 所以第一项是全导数, %
而方程(\textcolor{foo}{23.5.4})告诉我们第二项也是全导数. 在阿贝尔规范理论中, 这意味着像(\ref{27.3.18})这样的项没有效应.
而在非阿贝尔规范理论中, 就像在\,23.5\,节和\,23.6\,节中讨论的那样, 瞬子解得存在使得密度(\ref{27.3.18})对时空的积分可以不为零. 因此, 我们必须考虑到拉格朗日密度可能有如下的新项
\begin{equation}
    \mathscr{L}_{\theta} = -\frac{g^{2}\theta}{16\uppi^{2}}\operatorname{Im}
\Bigl[\sum_{A\alpha\beta}\epsilon_{\alpha\beta}W_{AL\alpha}W_{AL\beta}\Bigr]_{\mathscr{F}} \:, \label{27.3.19}
\end{equation}
其中$\,\theta\,$是一个新的实参量, $g\,$是规范耦合, 对于一个单规范群, 它可以方便地定义成: %
如果$\,t_{A}$, $t_{B}\,$和 $t_{C}\,$处在计算瞬子效应所使用的规范代数的``标准''$\,SU(2)\,$子代数中, %
我们就有$\,C_{ABC}=g\,\epsilon_{ABC}$. 在规范耦合的这个定义下, 方程(\textcolor{foo}{23.5.20})对于单规范群给出
\begin{equation}
    \int \dif^{4}x\:\epsilon_{\mu\nu\rho\sigma}\sum_{A}f_{A}^{\mu\nu}f_{A}^{\rho\sigma}=64\uppi^{2}\nu/g^{2}\:,
    \label{27.3.20}
\end{equation}
其中$\,\nu=0,\,\pm1,\,\pm2,\cdots$是整数, 即缠绕数, 它表征了规范场构形的拓扑类. 因此对于缠绕数为$\,\nu\,$的瞬子, %
拉格朗日密度$\,\mathscr{L}_{\theta}\,$对路径积分贡献一个相位,
\begin{equation}
    \biggl[\exp\biggl(\mi\int\dif^{4}x\:\mathscr{L}_{\theta}\biggr)\biggr]_{\nu}=\exp(\mi\nu\theta) \:, \label{27.3.21}
\end{equation}
所以$\,\mathscr{L}_{\theta}\,$关于$\,\theta\,$是周期的, 周期为$\,2\uppi$.

规范场吸收进一个$\,g\,$因子通常会比较方便, 这使得结构常数不依赖于$\,g$, %
而规范场的拉格朗日密度则要乘以一个总因子$\,1/g^{2}$. 在这个约定下, 规范场的完整拉格朗日密度可以用重新标度过的规范场和耦合常数表示成
\begin{equation}
    \mathscr{L}_{\mathrm{gauge}}+\mathscr{L}_{\theta} = -\operatorname{Re}
    \Bigl[\frac{\tau}{8\uppi\mi}\sum_{A\alpha\beta}\epsilon_{\alpha\beta}W_{AL\alpha}W_{AL\beta}\Bigr]_{\mathscr{F}}\:,
    \label{27.3.22}
\end{equation}
其中$\,\tau\,$是复耦合常数
\begin{equation}
    \tau \equiv \frac{4\uppi\mi}{g^{2}} +\frac{\theta}{2\uppi} \:. \label{27.3.23}
\end{equation}
根据方程(\textcolor{foo}{23.5.19}), 缠绕数为$\,\nu\,$的瞬子对路径积分的贡献被因子%
$\,\exp(-8\uppi^{2}\lvert\nu\rvert/g^{2})\,$压低了, 它与因子(\ref{27.3.21})合在一起产生了总因子
\begin{equation}
    \exp\Biggl[\mi\nu\theta-\frac{8\uppi^{2}\lvert \nu\rvert}{g^{2}}\Biggr]=
    \begin{cases}
    \exp(2\uppi\mi\nu\tau) &\qquad \nu\geq 0 \\
    \exp(2\uppi\mi\nu\tau^{\ast})&\qquad \nu \leq 0
    \end{cases} \:. \label{27.3.24}
\end{equation}


\section{含有手征超场的可重整规范理论} \label{sec:27.4}

我们现在将前三节装配的零件放在一起为与一般规范场相互作用的手征超场构造最一般的可重整作用量. 将(\ref{27.1.27}), %
(\ref{27.2.7}), (\ref{27.3.1})和方程(\ref{26.4.5})中的超势项加在一起给出拉格朗日密度
\begin{align}
    \mathscr{L}&=\frac{1}{2}\Biggl[\Phi^{\dag}\exp\Biggl(-2\sum_{A}t_{A}V_{A}\Biggr)\Phi\Biggr]_{D}
    -\frac{1}{2}\operatorname{Re}\sum_{A}\Bigl(W_{AL}^{\mathrm{T}}\epsilon W_{AL}\Bigr)_{\mathscr{F}} \nonumber \\
    &\qquad -\frac{g^{2}\theta}{16\uppi^{2}}\sum_{A}\operatorname{Im}
    \Bigl(W_{AL}^{\mathrm{T}}\epsilon W_{AL}\Bigr)_{\mathscr{F}}-\sum_{A}\xi_{A}[V_{A}]_{D}
    +2\operatorname{Re}[f]_{\mathscr{F}} \nonumber \\
    &= -\sum_{n}(D_{\mu}\phi)_{n}^{\ast}(D^{\mu}\phi)_{n}-\frac{1}{2}\sum_{n}
    \Bigl(\overline{\psi_{n}}\gamma^{\mu}(D_{\mu}\psi)_{n}\Bigr)+\sum_{n}\mathscr{F}_{n}^{\ast}\mathscr{F}_{n} \nonumber\\
    &\quad-\operatorname{Re}\sum_{nm}\frac{\partial^{2}f(\phi)}{\partial\phi_{n}\partial\phi_{m}}
    \Bigl(\psi_{nL}^{\mathrm{T}}\epsilon\psi_{mL}\Bigr)
    +2\operatorname{Re}\sum_{n}\frac{\partial f(\phi)}{\partial\phi_{n}}\mathscr{F}_{n} \nonumber \\
    &\quad-2\sqrt{2}\operatorname{Im}\sum_{Anm}(t_{A})_{nm}\Bigl(\overline{\psi_{nL}}\lambda_{A}\Bigr)\phi_{m}
    +2\sqrt{2}\operatorname{Im}\sum_{Anm}(t_{A})_{mn}\Bigl(\overline{\psi_{nR}}\lambda_{A}\Bigr)\phi_{m}^{\ast}
    \nonumber\\
    &\quad-\sum_{Anm}\phi_{n}^{\ast}(t_{A})_{nm}\phi_{m}D_{A}
    -\sum_{A}\xi_{A}D_{A}+\frac{1}{2}\sum_{A}D_{A}D_{A} \nonumber \\
    &\quad -\frac{1}{4}\sum_{A}f_{A\mu\nu}f^{\mu\nu}_{A}
    -\frac{1}{2}\sum_{A}\Bigl(\overline{\lambda_{A}}(\slashed{D}\lambda)_{A}\Bigr)
    +\frac{g^{2}\theta}{64\uppi^{2}}\epsilon_{\mu\nu\rho\sigma}\sum_{A}f_{A}^{\mu\nu}f_{A}^{\rho\sigma}\:.\label{27.4.1}
\end{align}
这里的$\,f(\phi)\,$是超势, $\phi_{n}\,$(不是$\,\phi_{n}^{\ast}$)的规范不变复函数, 而可重整性条件要求这是一个三次多项式; %
$\xi_{A}\,$是常数, 除非$\,t_{A}\,$是$\,U(1)\,$生成元, 否则规范不变性要求它为零; 规范协变导数是
\begin{gather}
    D_{\mu}\psi_{L} \equiv \partial_{\mu}\psi_{L}-\mi\sum_{A}t_{A}V_{A\mu}\psi_{L} \:, \label{27.4.2} \\
    D_{\mu}\phi\equiv \partial_{\mu}\phi -\mi\sum_{A}t_{A}V_{A\mu}\phi \:, \label{27.4.3} \\
    (D_{\mu}\lambda)_{A}=\partial_{\mu}\lambda_{A} +\sum_{BC}C_{ABC}V_{B\mu}\lambda_{C} \:, \label{27.4.4}
\end{gather}
$f_{A\mu\nu}\,$是规范协变的规范场强张量
\begin{equation}
    f_{A\mu\nu} =\partial_{\mu}V_{A\nu}-\partial_{\nu}V_{A\mu}+\sum_{BC}C_{ABC}V_{B\mu}V_{C\nu}\:.\label{27.4.5}
\end{equation}

辅助场以二次型的方式进入拉格朗日量, 且二阶项的系数是与场无关的常数, %
所以通过令辅助场等于使得拉格朗日密度稳定的值可以消除它们:
\begin{gather}
    \mathscr{F}_{n} =-\Bigl(\partial f(\phi)/\partial\phi_{n}\Bigr)^{\ast} \:, \label{27.4.6}\\
    D_{A} =\xi_{A} +\sum_{nm}\phi_{n}^{\ast}(t_{A})_{nm}\phi_{m} \:. \label{27.4.7}
\end{gather}
把它们代回方程(\ref{27.4.1}), 拉格朗日密度变成
\begingroup
\allowdisplaybreaks
\begin{align}
    \mathscr{L} &= -\sum_{n}(D_{\mu}\phi)_{n}^{\ast}(D^{\mu}\phi)_{n} \nonumber \\
    &\quad -\frac{1}{2}\sum_{n}\Bigl(\overline{\psi_{nL}}\gamma^{\mu}(D_{\mu}\psi_{L})_{n}\Bigr)
    +\frac{1}{2}\sum_{n}\Bigl(\overline{(D_{\mu}\psi_{L})_{n}}\gamma^{\mu}\psi_{nL}\Bigr) \nonumber \\
    &\quad -\frac{1}{2}\sum_{nm}\frac{\partial^{2}f(\phi)}{\partial\phi_{n}\partial\phi_{m}}
    \Bigl(\psi_{nL}^{\mathrm{T}}\,\epsilon\,\psi_{mL}\Bigr)-\frac{1}{2}\sum_{nm}
    \Biggl(\frac{\partial^{2}f(\phi)}{\partial\phi_{n}\partial\phi_{m}}\Biggr)^{\ast}
    \Bigl(\psi_{nL}^{\mathrm{T}}\,\epsilon\,\psi_{mL}\Bigr)^{\ast} \nonumber \\
    &\quad -\sum_{n}\biggl\lvert \frac{\partial f(\phi)}{\partial\phi_{n}}\biggr\rvert^{2} \nonumber \\
    &\quad +\mi\sqrt{2}\sum_{Anm}\Bigl(\overline{\psi_{nL}}\,(t_{A})_{nm}\,\lambda_{A}\Bigr)\phi_{m}
    -\mi\sqrt{2}\sum_{Anm}\phi_{n}^{\ast}\,\Bigl(\overline{\lambda_{A}}\,(t_{A})_{nm}\,\psi_{mL}\Bigr)\nonumber \\
    &\quad -\frac{1}{2}\sum_{A}\Biggl(\xi_{A}+\sum_{nm}\phi_{n}^{\ast}\,(t_{A})_{nm}\,\phi_{m}\Biggr)^{2}
    -\frac{1}{4}\sum_{A}f_{A\mu\nu}f^{\mu\nu}_{A} \nonumber \\
    &\quad -\frac{1}{2}\sum_{A}\Bigl(\overline{\lambda_{A}}(\slashed{D}\lambda)_{A}\Bigr)
    +\frac{g^{2}\theta}{64\uppi^{2}}\epsilon_{\mu\nu\rho\sigma}\sum_{A}f_{A}^{\mu\nu}f_{A}^{\rho\sigma} \:.\label{27.4.8}
\end{align}
\endgroup
Lorentz\,不变性要求场$\,\psi_{nL}$, $\lambda_{A}\,$和$\,f_{A\mu\nu}\,$的真空期望值为零, $\phi_{n}\,$的树级真空期望值处在势
\begin{equation}
    V(\phi) =\sum_{n}\biggl\lvert \frac{\partial f(\phi)}{\partial\phi_{n}}\biggr\rvert^{2}
    +\frac{1}{2}\sum_{A}\Biggl(\xi_{A}+\sum_{nm}\phi_{n}^{\ast}\,(t_{A})_{nm}\,\phi_{m}\Biggr)^{2} \label{27.4.9}
\end{equation}
的最小值处. 这个势是正的, 所以{\kai{如果}}存在场的一组值使得$\,V(\phi)\,$为零, 那么这组场值同时自动是势的一个最小值点. %
为了使$\,V(\phi)\,$在某个场值$\,\phi_{n}=\phi_{n0}\,$处为零, 充要条件是
\begin{equation}
    \mathscr{F}_{n0}=-\biggl[\frac{\partial f(\phi)}{\partial\phi_{n}}\biggr]^{\ast}_{\phi=\phi_{0}}=0\label{27.4.10}
\end{equation}
和
\begin{equation}
    D_{A0}=\xi_{A}+\sum_{nm}\phi_{n0}^{\ast}\,(t_{A})_{nm}\,\phi_{m0}=0 \:. \label{27.4.11}
\end{equation}
由于方程(\ref{26.3.15})给出$\,\langle\delta\psi_{nL}\rangle_{\text{VAC}}{=}\sqrt{2}
\langle\mathscr{F}_{n}\rangle_{\text{VAC}}\,\alpha_{L}$, 方程(\ref{26.2.16})给出%
$\,\langle\delta\lambda_{A}\rangle_{\text{VAC}}{=}\mi\langle D_{A}\rangle_{\text{VAC}}\,\gamma_{5}\alpha$, %
这转而是超对称不自发破缺的充要条件.


这里值得强调一下, 超对称性的自发对称性破缺要比其他对称性更加困难. 对于作用量的绝大多数对称性, 将会存在场构形使得对称性是不破缺的且势是稳定的, 但是, 如果这些构形中没有一个是势的最小值点, 这个对称性还是会自发破缺的. 反过来, 任何超对称的场构形给出的势的值是零, 它必然要比任何非超对称构形的势的值要低, 所以{\kai{任何}}超对称场构形的存在将会确保超对称是不破缺的. %
我们将在\,\ref{sec:27.6}\,节看到的, 这个结论会超出本节使用的树级近似; 它不被微扰论中任何有限阶的修正影响.


看起来方程(\ref{27.4.10})和(\ref{27.4.11})给标量场附加了太多的条件以至于不给超势做一些精细调节 (fine-tuning)就无法期待有解.
然而, 对于维度为$\,D\,$的规范群, 对所有$\,A\,$和$\,\phi$, 超势$\,f(\phi)\,$要满足$\,D$ 个约束
\begin{equation}
    \sum_{m}\frac{\partial f(\phi)}{\partial \phi_{m}}\Bigl(t_{A}\phi\Bigr)_{m}=0 \:. \label{27.4.12}
\end{equation}
因此, 如果$\,\phi\,$有$\,N\,$个独立分量, 那么{\kai{独立}}条件(\ref{27.4.10})的个数是$\,N-D$, %
而条件(\ref{27.4.11})的个数是$\,D$, 所以总共只有$\,N\,$个条件. 条件的数目等于自由变量的个数, %
因此对于一般的超势是很可能找到解的. 事实上, 找到解比找不到解更普遍些. 例如, 对于处在一个半单群的非平庸表示下的手征超场, %
我们有$\,\xi_{A}=0$, 而$\,f(\phi)\,$不可能有$\,\phi_{n}\,$的线性项, %
所以方程(\ref{27.4.10})和(\ref{27.4.11})在$\,\phi_{n0}=0\,$时均是满足的. 方程(\ref{27.4.10})和(\ref{27.4.11})可能有其它会破缺规范对称性的解, 但在这样的一个理论中, 超对称不会被破缺, 至少不会在树级近似下被破缺, 而我们将在\,\ref{sec:27.6}\,节看到, %
它也不会在微扰论的任何阶被破缺.


更一般地, 很容易看到, 即使规范群有$\,U(1)\,$因子且即使超势包含规范不变的超场, %
假定\,Fayet-Iliopoulos\,常数$\,\xi_{A}\,$都为零, 如果存在一组满足方程(\ref{27.4.10})的标量场值$\,\phi_{n0}$, %
那么就存在另外一组满足方程(\ref{27.4.10}){\kai{和}}(\ref{27.4.11})的标量场值. 为了证明这点, %
我们注意到, 由于超势$\,f(\phi)\,$不涉及$\,\phi^{\ast}$, 它不仅在$\,\Lambda_{A}\,$是实常数的普通规范变换$\,\phi\to\exp(\mi\sum_{A}\Lambda_{A}t_{A})\phi\,$下不变, 而且也在$\,\Lambda_{A}\,$是任意复数的变换下不变. %
在所有这些变换下, 方程(\ref{27.4.10})中的$\,\mathscr{F}\,$-项进行线性变换, 所以如果$\,\phi_{0}\,$满足方程(\ref{27.4.10}), %
那么$\,\phi^{\Lambda}\equiv\exp(\mi\sum_{A}\Lambda_{A}t_{A})\phi_{0}\,$也满足. 另一方面, %
标量积$\,[\phi^{\dag}\phi]\,$在$\,\Lambda_{A}\,$是复数的变换下不是不变的, %
但$\,[\phi^{\Lambda\dag}\phi^{\Lambda}]\,$对于复的$\,\Lambda_{A}\,$依旧是正实的, 所以它下有界, 因此有一个最小值. %
当 $\xi_{A}=0\,$时, $[\phi^{\Lambda\dag}\phi^{\Lambda}]\,$在最小值处为零这个条件就是$\,\phi^{\Lambda}\,$应该满足%
方程(\ref{27.4.11}). 我们由此看到, 当没有\,Fayet-Ilioupoulous\,$D\,$-项时, 规范理论中超对称破不破缺的问题完全就是超势是否允许方程 
(\ref{27.4.10})有解的问题. 相同的结论即使在不可重整理论中也是成立的.\cite{3}

现在我们假定存在一组值$\,\phi_{n0}\,$使得$\,V(\phi_{0})=0$, 使得超对称性是不破缺的. 描述自旋$\,0\,$自由度的是偏移场
\begin{equation}
    \varphi = \phi_{n}-\phi_{n0} \:. \label{27.4.13}
\end{equation}
那么就存在$\,\varphi\,$和规范场之间的交叉项, 来源于方程(\ref{27.4.1})中的第一项:
\[
2\sum_{nA}\operatorname{Im}\Bigl(\partial_{\mu}\varphi_{n}\,(t_{A}\phi_{0})_{n}^{\ast}\Bigr)V_{A}^{\mu}\:.
\]
正如在\,21.1\,节中证明过的, 通过选取一个``幺正规范''总能消除这一项, 在这个规范下, $\phi_{n}\,$满足一个条件使得这项为零:
\begin{equation}
    \sum_{n}\operatorname{Im}\Bigl(\phi_{n}\,(t_{A}\phi_{0})_{n}^{\ast}\Bigr)=0 \:. \label{27.4.14}
\end{equation}
这将会消除破缺规范对称性附带的\,Goldstone\,玻色子.

现在, 在超对称性不破缺的前提下, 考虑到规范对称性可能自发破缺的可能性, 我们将解出这个理论中产生的自旋\,0, %
$\frac{1}{2}\,$和\,1\,粒子的质量.

\subsection{自旋\,0}

因为$\,\partial f(\phi)/\partial\phi_{n}\,$和$\,\xi_{A}+\sum_{nm}\phi_{n}^{\ast}(t_{A})_{nm}\phi_{m}\,$必须在%
$\,\phi_{n}=\phi_{n0}\,$处都为零, $V(\phi)\,$中 $\varphi_{n}\equiv\phi_{n}{-}\phi_{n0}$ 和(或)$\,\varphi_{n}^{\ast}\,$的二阶项是如下的形式
\begin{align}
V_{\mathrm{quad}}(\phi) &= \sum_{nm}(\mathscr{M}^{\ast}\mathscr{M})_{nm}\varphi_{n}^{\ast}\varphi_{m}
+\sum_{Anm}\Bigl(t_{A}\phi_{0}\Bigr)_{n}\Bigl(t_{A}\phi_{0}\Bigr)_{m}^{\ast}\varphi_{n}^{\ast}\varphi_{m} \nonumber\\
&\quad +\frac{1}{2}\sum_{Anm}\Bigl(t_{A}\phi_{0}\Bigr)_{n}^{\ast}\Bigl(t_{A}\phi_{0}\Bigr)_{m}^{\ast}
\varphi_{n}\varphi_{m}+\frac{1}{2}\sum_{Anm}\Bigl(t_{A}\phi_{0}\Bigr)_{n}\Bigl(t_{A}\phi_{0}\Bigr)_{m}
\varphi_{n}^{\ast}\varphi_{m}^{\ast} \:, \label{27.4.15}
\end{align}
其中$\,\mathscr{M}\,$是复对称矩阵(\ref{26.4.11}):
\[
\mathscr{M}_{nm} \equiv \Biggl(\frac{\partial^{2}f(\phi)}{\partial\phi_{n}\partial\phi_{m}}\Biggr)_{\phi=\phi_{0}}\:.
\]
这可以写成
\begin{equation}
    V_{\mathrm{quad}}=\frac{1}{2}\begin{bmatrix}
    \varphi \\ \varphi^{\ast}
\end{bmatrix}^{\dag}
M_{0}^{2}
\begin{bmatrix}
    \varphi \\ \varphi^{\ast}
\end{bmatrix} \:, \label{27.4.16}
\end{equation}
其中$\,M_{0}^{2}\,$是分块矩阵
\begin{equation}
    M_{0}^{2} = \begin{bmatrix}
        \mathscr{M}^{\ast}\mathscr{M}+\sum_{A}(t_{A}\phi_{0})(t_{A}\phi_{0})^{\dag} &
        \sum_{A}(t_{A}\phi_{0})(t_{A}\phi_{0})^{\mathrm{T}} \\[1em]
        \sum_{A}(t_{A}\phi_{0})^{\ast}(t_{A}\phi_{0})^{\dag} &
        \mathscr{M}^{\ast}\mathscr{M}+\sum_{A}(t_{A}\phi_{0})^{\ast}(t_{A}\phi_{0})^{\mathrm{T}}
    \end{bmatrix} \:. \label{27.4.17}
\end{equation}
现在我们必须找到这个质量平方矩阵的本征值. 方程(\ref{27.4.12})对$\,\phi_{n}\,$的微分给出
\begin{equation}
    \sum_{m}\frac{\partial^{2}f(\phi)}{\partial\phi_{n}\partial\phi_{m}}\Bigl(t_{A}\phi\Bigr)_{m}
    +\sum_{m}\frac{\partial f(\phi)}{\partial\phi_{m}}\Bigl(t_{A}\Bigr)_{mn} =0 \:. \label{27.4.18}
\end{equation}
但正如我们已经看到的, $\partial f(\phi)/\partial \phi_{m}\,$在$\,\phi=\phi_{0}\,$处为零, %
所以通过在方程(\ref{27.4.18})中令$\,\phi\,$取在该值处, 我们发现
\begin{equation}
    \sum_{m}\mathscr{M}_{nm}(t_{A}\phi_{0})_{m}=0 \:. \label{27.4.19}
\end{equation}
由此得出
\begin{equation*}
    M_{0}^{2}\begin{bmatrix}
        t_{B}\phi_{0} \\ \pm(t_{B}\phi_{0})^{\ast}
    \end{bmatrix} =
    \sum_{A}\Bigl(\phi_{0}^{\dag}[t_{A}t_{B}\pm t_{B}t_{A}]\phi_{0}\Bigr)
    \begin{bmatrix}
        t_{B}\phi_{0} \\ \pm(t_{B}\phi_{0})^{\ast}
    \end{bmatrix}
\end{equation*}
但$\,D_{A}\,$在$\,\phi=\phi_{0}\,$处为零以及$\,\xi_{A}\,$的整体规范不变性告诉我们
\begin{equation}
    \Bigl(\phi_{0}^{\dag}[t_{A},t_{B}]\phi_{0}\Bigr) = \mi\sum_{C}C_{ABC}\Bigl(\phi_{0}^{\dag}t_{C}\phi_{0}\Bigr)
    =-\mi \Bigl(\phi_{0}^{\dag}\phi_{0}\Bigr)\sum_{C}C_{ABC}\xi_{C} = 0 \:. \label{27.4.20}
\end{equation}
因此矩阵(\ref{27.4.17})对每个规范对称性有一对本征矢量
\begin{equation}
    u=\begin{bmatrix}
        \sum_{B}c_{B}\,t_{B}\phi_{0} \\ \sum_{B}c_{B}\,(t_{B}\phi_{0})^{\ast}
    \end{bmatrix} \:, \qquad
    v=\begin{bmatrix}
        \sum_{B}c_{B}\,t_{B}\phi_{0} \\ -\sum_{B}c_{B}\,(t_{B}\phi_{0})^{\ast}
    \end{bmatrix}\:, \label{27.4.21}
\end{equation}
对于每个本证矢量
\begin{equation}
    M_{0}^{2}u=\mu^{2}u \:, \qquad M_{0}^{2}v=0 \:, \label{27.4.22}
\end{equation}
其中$\,\mu^{2}\,$和$\,c_{A}\,$是如下本征值问题的实解\footnote{方程(\textcolor{foo}{21.1.17})中出现因子$\,1/2\,$而方程(\ref{27.4.23})中没有是因为标量场归一化的方式不同.}
\begin{equation}
    \sum_{B}\Bigl(\phi_{0}^{\dag}\{t_{A},t_{B}\}\phi_{0}\Bigr)c_{B}=\mu^{2}c_{A} \:, \label{27.4.23}
\end{equation}
这里有一个例外, 如果本征值$\,\mu^{2}\,$为零, 那么$\,\sum_{B}c_{B}t_{B}\phi_{0}=0$, 这使得本征值$\,u\,$和$\,v\,$都缺失了. %
与本征矢$\,v\,$相对应的是\,Goldstone\,玻色子, 它们被幺正规范条件(\ref{27.4.14})从物理频谱中消除了. 除了这些有质量的本征态外, 还存在另外一组与所有$\,u\,$和$\,v\,$都正交的本征态, 因此它们取如下的形式
\begin{equation}
    w_{\pm} = \begin{bmatrix}
        \zeta \\ \pm\zeta^{\ast}
    \end{bmatrix} \:, \label{27.4.24}
\end{equation}
其中
\begin{equation}
    \sum_{n}(t_{A}\phi_{0})_{n}^{\ast} \,\zeta_{n} = 0 \:. \label{27.4.25}
\end{equation}
方程(\ref{27.4.19})表明满足方程(\ref{27.4.25})的$\,\xi\,$构成的空间在乘以厄米矩阵$\,\mathscr{M}^{\dag}\mathscr{M}\,$后是不变的, 所以这个空间由这个矩阵的本征矢张开, 满足
\begin{equation}
    \mathscr{M}^{\dag}\mathscr{M}\,\zeta = m^{2}\zeta \label{27.4.26}
\end{equation}
其中$\,m^{2}\,$是一组正实的(或零)本征值. 方程(\ref{27.4.26})和它的复共轭加上方程(\ref{27.4.25})表明%
$\,w_{\pm}\,$是$\,M^{2}_{0}$ 本征值为$\,m^{2}\,$的本征矢:
\begin{equation}
    M_{0}^{2}w_{\pm}=m^{2}w_{\pm} \:. \label{27.4.27}
\end{equation}
因此我们有{\kai{两个}}满足方程(\ref{27.4.27})的质量为$\,m\,$的自荷共轭无自旋玻色子, 以及对于每个非零质量 $\mu\,$有一个满足方程(\ref{27.4.23})的自荷共轭无自旋玻色子.


\subsection{自旋\,1/2}

费米子的质量来源于方程(\ref{27.4.8})中的非导数项, 它们是费米子场$\,\psi_{n}\,$和$\,\lambda_{A}\,$的二阶项:
\begin{equation}
   \mathscr{L}_{1/2}=-\frac{1}{2}\sum_{nm}\mathscr{M}_{nm}\Bigl(\psi_{nL}^{\mathrm{T}}\epsilon\psi_{mL}\Bigr)
   -\mi\sqrt{2}\sum_{Am}(t_{A}\phi_{0})_{m}^{\ast}\,\Bigl(\lambda_{LA}^{\mathrm{T}}\epsilon\psi_{mL}\Bigr)+
   \mathrm{H.c.} \label{27.4.28}
\end{equation}
我们在\,\ref{sec:26.4}\,节看到, 对于一列\,Majorana\,旋量场$\,\chi$, 如果拉格朗日量中的费米子质量项写成了如下的形式
\begin{equation}
    \mathscr{L}_{1/2} = -\frac{1}{2}\Bigl(\chi_{L}^{\mathrm{T}}\epsilon M\chi_{L}\Bigr)+\mathrm{H.c.}\:,
    \label{27.4.29}
\end{equation}
那么费米子质量平方是厄米矩阵$\,M^{\dag}M\,$的本征值. 这里方程(\ref{27.4.28})给出的矩阵$\,M\,$的矩阵元是
\begin{equation}
    M_{nm}=\mathscr{M}_{nm}\:, \qquad M_{nA}=M_{An}=\mi\sqrt{2}(t_{A}\phi_{0})_{n}^{\ast} \:, \qquad
    M_{AB}=0 \:, \label{27.4.30}
\end{equation}
对于这组矩阵元, 利用方程(\ref{27.4.19})和(\ref{27.4.20}),
\begin{align}
    (M^{\dag}M)_{nm}&=(\mathscr{M}^{\dag}\mathscr{M})_{nm}
    +2\sum_{A}(t_{A}\phi_{0})_{n}(t_{A}\phi_{0})_{m}^{\ast}\:,\nonumber \\
    (M^{\dag}M)_{nA} &= (M^{\dag}M)_{An}=0 \:, \label{27.4.31} \\
    (M^{\dag}M)_{AB}&=2(\phi_{0}^{\dag}t_{B}t_{A}\phi_{0})=(\phi_{0}^{\dag}\{t_{B},t_{A}\}\phi_{0})\:.\nonumber
\end{align}
矩阵(\ref{27.4.30})的本征矢有\,3\,类. 第一种的形式是
\begin{equation}
    z=\begin{bmatrix}
        \zeta \\ 0
    \end{bmatrix} \:, \label{27.4.32}
\end{equation}
本征值是$\,m^{2}$, 其中$\,\xi_{n}\,$和$\,m^{2}\,$是$\,\mathscr{M}^{\dag}\mathscr{M}\,$的任何本征矢量以及相应的本征值. 第二种的形式是
\begin{equation}
    g= \begin{bmatrix}
        0 \\ c
    \end{bmatrix} \:, \label{27.4.33}
\end{equation}
本征值是$\,\mu^{2}$, 其中$\,c_{B}\,$和$\,\mu^{2}\,$是矩阵$\,(\phi_{0}^{\dag}\{t_{B},t_{A}\}\phi_{0})\,$的任何本征矢量和相应的本征值. 最后一种的形式是
\begin{equation}
    h= \begin{bmatrix}
        \sum_{B}c_{B}t_{B}\phi_{0} \\ 0
    \end{bmatrix} \:, \label{27.4.34}
\end{equation}
本征值是$\,\mu^{2}$, 其中$\,c_{B}\,$和$\,\mu^{2}\,$依旧是矩阵$\,(\phi_{0}^{\dag}\{t_{B},t_{A}\}\phi_{0})\,$的任何本征矢量和相应的本征值. 唯一的例外是这个矩阵本征值为零的本征矢$\,c\,$有$\,\sum_{A}c_{A}t_{A}\phi_{0}=0$, 对应于未破缺的对称性, %
这使得矢量 (\ref{27.4.34})在这一情况下为零且我们只有本征矢(\ref{27.4.33}). 因此, %
对于每个质量$\,m\,$有一个满足方程 (\ref{27.4.26})的\,Majorana\,费米子, 对于每个非零的质量$\,\mu\,$有两个满足方程(\ref{27.4.22})的\,Majorana\,费米子, 对于每个未破缺的对称性有一个零质量的\,Majorana\,费米子.


\subsection{自旋\,1}


拉格朗日量中规范场的质量项来源于方程(\ref{27.4.1})的第一项中规范场$\,V_{A}^{\mu}\,$的二阶项部分:
\begin{equation}
\mathscr{L}_{V} = -\sum_{nAB}(t_{A}\phi_{0})_{n}^{\ast} (t_{B}\phi_{0})_{n}\,V_{A\mu}V_{B}^{\mu} \:. \label{27.4.35}
\end{equation}
由于场$\,V_{A\mu}\,$是实的, 它们的质量平方矩阵是方程(\ref{27.4.23})中的矩阵:
\begin{equation}
(\mu^{2})_{AB} = \Bigl(\phi_{0}^{\dag}\{t_{B},t_{A}\}\phi_{0}\Bigr) \:. \label{27.4.36}
\end{equation}
对于矩阵(\ref{27.4.36})的每个本征值$\,\mu^{2}\,$有一个质量为$\,\mu\,$的自旋\,1\,粒子.
\\

将这些放在一起, 我们看到, 对于矩阵$\,\mathscr{M}^{\ast}\mathscr{M}\,$的每个本征值$\,m^{2}$, 存在两个质量为$\,m\,$的自荷共轭的无自旋粒子和一个质量为$\,m\,$的\,Majorana\,费米子; 对于矩阵$\,\mu_{AB}^{2}\,$的每个非零本征值, 存在一个自荷共轭的无自旋玻色子, 两个\,Majorana\,费米子和一个自荷共轭的自旋\,1\,玻色子, 质量均为$\,\mu$; 对于这个矩阵的每个非零本征值, 存在一个无质量的\,Majorana\,费米子和一个无质量的自荷共轭的自旋\,1\,玻色子. 每个零质量或非零质量的粒子多重态恰好与我们在%
\,\ref{sec:25.4}\,和\,\ref{sec:25.5}\,节直接用超对称代数发现的相同, 这并不奇怪. 稍微有点让人惊讶的{\kai{是}}, 规范粒子和手征粒子的质量彼此不受影响. 由$\,(\mathscr{M}^{\ast}\mathscr{M})\,$的本征值给出的质量$\,m$和有这些质量的粒子就是没有规范超场的手征超场理论中的那些粒子和质量, 而由矩阵$\,\mu^{2}_{AB}\,$的本征值给出的质量$\,\mu\,$和有这些质量的粒子就是没有手征超场的规范超场理论中的那些粒子和质量.

为了在\,\ref{sec:27.9}\,节的使用, 我们现在要用\,\ref{sec:26.7}\,节描述的方法来为超对称规范拉格朗日量(\ref{27.4.1})构建超对称流. 在之前使用的规范下, 一个无限小超对称变换对$\,V_{A}$, $\lambda_{a}\,$和$\,D_{A}\,$的改变是%
(\ref{27.3.4})---(\ref{27.3.6}). 方程(\ref{26.7.2})给出了这些场的\,Noether\,超对称流, 将这些流与已经在方程(\ref{26.7.8})中%
给出的$\,\phi_{n}$, $\psi_{n}\,$和$\,\mathscr{F}_{n}\,$的流加在一起, 再把导数换成规范协变导数, 这样就给出了总的\,Noether\,超对称流:
\begin{align}
    N^{\mu}&=\sum_{A}f_{A}^{\mu\nu}\gamma_{\nu}\lambda_{A} -\frac{1}{8}\sum_{A}f_{A\rho\sigma}[\gamma^{\rho},\gamma^{\sigma}]\gamma^{\mu}\lambda_{A}
    -\frac{1}{2}\mi\sum_{A}D_{A}\,\gamma_{5}\gamma^{\mu}\,\lambda_{A} \nonumber \\
    &\quad +\frac{1}{\sqrt{2}}\sum_{n}\Bigl[2(D^{\mu}\phi)_{n}^{\ast}\,\psi_{nL}
    +2(D^{\mu}\phi)_{n}\,\psi_{nR}+(\slashed{D}\phi)_{n}\,\gamma^{\mu}\psi_{nR} \nonumber \\
    &\quad +(\slashed{D}\phi)_{n}^{\ast}\,\gamma^{\mu}\psi_{nL}-\mathscr{F}_{n}\,\gamma^{\mu}\psi_{nR}
    -\mathscr{F}_{n}^{\ast}\,\gamma^{\mu}\psi_{nL} \Bigr] \:. \label{27.4.37}
\end{align}
因为拉格朗日密度在超对称下不是不变的, 这不是超对称流; 诚然, 拉格朗日密度的变化是导数
\begin{equation}
    \delta\mathscr{L}=\partial_{\mu}\Bigl(\bar{\alpha}K^{\mu}\Bigr) \:, \label{27.4.38}
\end{equation}
其中\footnote{计算$\,[\Phi^{\dag}\exp(-2\sum_{A}t_{A}V_{A})\Phi]_{D}\,$中的变化的最简单方法是%
计算$\,\Phi^{\dag}\exp(-2\sum_{A}t_{A}V_{A})\Phi$\,的$\,\lambda\,$-分量并使用方程(\ref{26.2.17}). 在以这种方式进行计算时, 方程(\ref{27.4.39})右边第二行这个重要的项来源于$\,\exp(-2\sum_{A}t_{A}V_{A})\,$的$\,\lambda\,$-分量.}
\begin{align}
    K^{\mu}&=\frac{1}{2}\mi\sum_{A}\epsilon^{\rho\sigma\mu\nu}f_{A\rho\sigma}\gamma_{\nu}\gamma_{5}\lambda_{A}
    +\frac{1}{8}\sum_{A}[\gamma^{\rho},\gamma^{\sigma}]\gamma^{\mu}\lambda_{A}f_{A\rho\sigma}
    +\frac{1}{2}\mi\sum_{A}D_{A}\,\gamma_{5}\gamma^{\mu}\,\lambda_{A} \nonumber \\
    &\quad-\mi\sum_{Anm}(t_{A})_{nm}\gamma_{5}\gamma^{\mu}\,\lambda_{A}\phi_{n}^{\ast}\phi_{m} \nonumber \\
    &\quad +\frac{1}{\sqrt{2}}\sum_{n}\gamma^{\mu}\Biggl[-(\slashed{D}\phi)_{n}\,\psi_{nR}
    -(\slashed{D}\phi)_{n}^{\ast}\,\psi_{nL}+\mathscr{F}_{n}^{\ast}\,\psi_{nL}+\mathscr{F}_{n}\,\psi_{nR}\nonumber\\
    &\quad +2\biggl(\frac{\partial f(\phi)}{\partial\phi_{n}}\biggr)\,\psi_{nL}
    +2\biggl(\frac{\partial f(\phi)}{\partial\phi_{n}}\biggr)^{\ast}\,\psi_{nR} \Biggr]\:. \label{27.4.39}
\end{align}
前两项是用恒等式(\ref{27.2.5})导出的. 再次使用同一个恒等式并使用方程(\ref{26.7.4})给出了总的超对称流:
\begin{align}
    S^{\mu}&= N^{\mu}+K^{\mu} \nonumber \\
    &=-\frac{1}{4}\sum_{A}f_{A\rho\sigma}[\gamma^{\rho},\gamma^{\sigma}]\gamma^{\mu}\lambda_{A}
    -\mi\sum_{Anm}(t_{A})_{nm}\gamma_{5}\gamma^{\mu}\lambda_{A}\phi_{n}^{\ast}\phi_{m} \nonumber \\
    &\quad +\frac{1}{\sqrt{2}}\sum_{n}\Biggl[(\slashed{D}\phi)_{n}\,\gamma^{\mu}\psi_{nR}+
    (\slashed{D}\phi^{\ast})_{n}\,\gamma^{\mu}\psi_{nL} \nonumber \\
    &\quad +2\biggl(\frac{\partial f(\phi)}{\partial\phi_{n}}\biggr)\,\gamma^{\mu}\psi_{nL}
    +2\biggl(\frac{\partial f(\phi)}{\partial\phi_{n}}\biggr)^{\ast}\,\gamma^{\mu}\psi_{nR} \Biggr]\:. \label{27.4.40}
\end{align}

\subsection*{* * *}


在\,\ref{sec:26.8}\,节, 我们考虑了一类有超势$\,f(\Phi)\,$和\,Kahler\,势$\,K(\Phi,\Phi^{\ast})\,$的超对称理论, %
其中$\,f(\Phi)\,$以任意的方式依赖于一组左手征超场$\,\Phi_{n}\,$但与它们的导数无关, 而$\,K(\Phi,\Phi^{\ast})\,$以以任意的方式依赖于$\,\Phi_{n}$ 和$\,\Phi_{n}^{\ast}\,$但与它们的导数无关. 我们可以将相同的考虑推广至规范理论, 其中拉格朗日量对手征超场的依赖性依旧只被超对称形限制, 但不引入新的超导数或时间导数. 这样, 可重整的拉格朗日密度就被替换成
\begin{align}
    \mathscr{L} &= \frac{1}{2}\Bigl[K\Bigl(\Phi,\Phi^{\dag}\,\exp(-2\sum_{A}t_{A}V_{A})\Bigr)\Bigr]_{D}
    +2\operatorname{Re}\Bigl[f(\Phi)\Bigr]_{\mathscr{F}} \nonumber \\
    &\quad -\frac{1}{2}\operatorname{Re}\sum_{AB}\Bigl[h_{AB}(\Phi)\,\Bigl(W_{AL}^{\mathrm{T}}\epsilon W_{BL}\Bigr)\Bigr]_{\mathscr{F}} \:, \label{27.4.41}
\end{align}
其中$\,h_{AB}(\Phi)\,$是$\,\Phi_{n}\,$的一个新函数, 但与$\,\Phi_{n}^{\ast}$或导数无关.

手征规范超场和标量规范超场由展开(\ref{26.3.21})和(\ref{27.3.15})给出:
\begin{align*}
    W_{AL}(x,\theta)&=\lambda_{AL}(x_{+})+\frac{1}{2}\gamma^{\mu}\gamma^{\nu}\theta_{L}\,f_{A\mu\nu}(x_{+})
    +\Bigl(\theta_{L}^{\mathrm{T}}\epsilon\theta_{L}\Bigr)\slashed{D}\lambda_{AR}(x_{+}) \\
    &\quad -\mi\theta_{L}D_{A}(x_{+}) \:, \\
    \Phi_{n}(x,\theta) &= \phi_{n}(x_{+})-\sqrt{2}\Bigl(\theta_{L}^{\mathrm{T}}\epsilon\psi_{nL}(x_{+})\Bigr)
    +\mathscr{F}_{n}(x_{+})\Bigl(\theta_{L}^{\mathrm{T}}\epsilon\theta_{L}\Bigr) \:,
\end{align*}
其中$\,x_{+}^{\mu}\,$是偏移坐标(\ref{26.3.23}). 这样, $\sum_{AB}h_{AB}(\Phi)(W_{AL}^{\mathrm{T}}\epsilon W_{BL})\,$中$\,\theta_{L}$(与$\,\theta_{R}\,$独立)的二阶项就是
\begin{align*}
    &-\Biggl[\sum_{AB}h_{AB}(\Phi)\Bigl(W_{AL}^{\mathrm{T}}\epsilon W_{BL}\Bigr)\Biggr]_{\theta_{L}^{2}} = \\
    &\Bigl(\theta_{L}^{\mathrm{T}}\epsilon\theta_{L}\Bigr)
    \sum_{AB}\Bigl(\lambda_{AL}^{\mathrm{T}}\epsilon\lambda_{BL}\Bigr)
    \Biggl[\frac{1}{2}\sum_{nm}\Bigl(\psi_{nL}^{\mathrm{T}}\epsilon\psi_{mL}\Bigr)
    \frac{\partial^{2}h_{AB}(\phi)}{\partial \phi_{n}\partial\phi_{m}}
    -\sum_{n}\mathscr{F}_{n}\frac{\partial h_{AB}(\phi)}{\partial\phi_{n}}\Biggr] \\
    &\quad +\Bigl(\theta_{L}^{\mathrm{T}}\epsilon\theta_{L}\Bigr)\sum_{AB}h_{AB}(\phi)\Biggl[
    -\Bigl(\overline{\lambda_{A}}\slashed{D}(1-\gamma_{5})\lambda_{B}\Bigr)
    -\frac{1}{2}f_{A\mu\nu}f_{B}^{\mu\nu} \\
    &\quad\qquad\qquad  +\frac{\mi}{4}\epsilon_{\mu\nu\rho\sigma}f_{A}^{\mu\nu}f_{B}^{\rho\sigma}+D_{A}D_{B}\Biggr]\\
    &\quad +\sqrt{2}\sum_{ABn}\frac{\partial h_{AB}(\phi)}{\partial\phi_{n}}\,
    \Bigl(\theta_{L}^{\mathrm{T}}\epsilon\psi_{nL}\Bigr)
    \Bigl[-\Bigl(\lambda_{BL}^{\mathrm{T}}\epsilon\gamma^{\mu}\gamma^{\nu}\theta_{L}\Bigr)f_{A\mu\nu}
    +2\mi\Bigl(\lambda_{BL}^{\mathrm{T}}\epsilon\theta_{L}\Bigr)\Bigr]\:,
\end{align*}
现在所有场被理解成在$\,x^{\mu}\,$处计算而不是$\,x_{+}^{\mu}$. (右边的第一项和第二项分别取自方程 (\ref{26.4.4}) 和 (\ref{27.3.16}).) 另外, 
通过将$\,\theta_{L\alpha}\theta_{L\beta}\,$写成%
$\,\frac{1}{2}\epsilon_{\alpha\beta}(\theta_{L}^{\mathrm{T}}\epsilon\theta_{L})$, 右边第三项也可以表示成正比于%
$\,(\theta_{L}^{\mathrm{T}}\epsilon\theta_{L})$:
\begin{align*}
    &\Bigl(\theta_{L}^{\mathrm{T}}\epsilon\psi_{nL}\Bigr)
    \Bigl[\Bigl(\overline{\psi_{B}}\gamma^{\mu}\gamma^{\nu}\theta_{L}\Bigr)f_{A\mu\nu}
    -2\mi\Bigl(\overline{\psi_{B}}\theta_{L}\Bigr)\Bigr] = \\
    &\qquad \frac{1}{2}\Bigl(\theta_{L}^{\mathrm{T}}\epsilon\theta_{L}\Bigr)
    \Bigl[\Bigl(\overline{\psi_{B}}\gamma^{\mu}\gamma^{\nu}\psi_{nL}\Bigr)f_{A\mu\nu}
    -2\mi\Bigl(\overline{\psi_{B}}\psi_{nL}\Bigr)D_{A}\Bigr] \:.
\end{align*}
$\mathscr{F}\,$-项是$\,(\theta_{L}^{\mathrm{T}}\epsilon\theta_{L})\,$的系数, 所以
\begingroup
\allowdisplaybreaks
\begin{align*}
    &-\Biggl[\sum_{AB}h_{AB}(\Phi)\Bigl(W_{AL}^{\mathrm{T}}\epsilon W_{BL}\Bigr)\Biggr]_{\mathscr{F}} = \\
    &\qquad \sum_{AB}\Bigl(\lambda_{AL}^{\mathrm{T}}\epsilon\lambda_{BL}\Bigr)
    \Biggl[\frac{1}{2}\sum_{nm}\Bigl(\psi_{nL}^{\mathrm{T}}\epsilon\psi_{mL}\Bigr)
    \frac{\partial^{2}h_{AB}(\phi)}{\partial \phi_{n}\partial\phi_{m}}
    -\sum_{n}\mathscr{F}_{n}\frac{\partial h_{AB}(\phi)}{\partial\phi_{n}}\Biggr] \\
    &\qquad +\sum_{AB}h_{AB}(\phi)\Biggl[-\Bigl(\overline{\lambda_{A}}\slashed{D}(1-\gamma_{5})\lambda_{B}\Bigr)
    -\frac{1}{2}f_{A\mu\nu}f_{B}^{\mu\nu} +\frac{\mi}{4}\epsilon_{\mu\nu\rho\sigma}f_{A}^{\mu\nu}f_{B}^{\rho\sigma}\\
    &\quad \qquad\qquad\qquad\qquad +D_{A}D_{B}\Biggr]\\
    &\qquad +\frac{\sqrt{2}}{2}\sum_{ABn}\frac{\partial h_{AB}(\phi)}{\partial\phi_{n}}\,
    \Bigl[-\Bigl(\overline{\psi_{B}}\gamma^{\mu}\gamma^{\nu}\psi_{nL}\Bigr)f_{A\mu\nu}
    +2\mi\Bigl(\overline{\psi_{B}}\psi_{nL}\Bigr)D_{A}\Bigr]\:.
\end{align*}
\endgroup
方程(\ref{27.4.41})中的另一项正是由拉格朗日密度(\ref{26.8.6})的规范不变版本给出. 将这些放在一起就给出了拉格朗日密度
\begin{align}
    \mathscr{L} &= \operatorname{Re}\sum_{nm}\mathscr{G}_{nm}(\phi,\phi^{\ast})\Biggl[
    -\frac{1}{2}\Bigl(\overline{\psi_{m}}\,\slashed{D}(1+\gamma_{5})\psi_{n}\Bigr)
    +\mathscr{F}_{n}\mathscr{F}_{m}^{\ast}- D_{\mu}\phi_{n}D^{\mu}\psi_{m}^{\ast}\Biggr] \nonumber \\
    &\quad -2\operatorname{Re}\sum_{i}\frac{\partial K(\phi,\phi^{\ast})}{\partial\phi_{i}^{\ast}}D_{A}(\phi^{\ast}t_{A})_{i}\nonumber \\
    &\quad +\mi\sqrt{2}\sum_{ij}\frac{\partial^{2}K(\phi,\phi^{\ast})}{\partial\phi_{i}\partial\phi_{j}^{\ast}}
    \bigl[(t_{A}\phi)_{i}\overline{\psi_{j}}\lambda_{AR}
    -(\phi^{\ast}t_{A})_{j}\overline{\psi_{i}}\lambda_{AL}\bigr] \nonumber \\
    &\quad-\operatorname{Re}\sum_{nml}\frac{\partial^{3}K(\phi,\phi^{\ast})}{\partial\phi_{n}\partial\phi_{m}
    \partial\phi_{l}^{\ast}}\Bigl(\overline{\psi_{n}}\psi_{mL}\Bigr)\mathscr{F}_{l}^{\ast} \nonumber \\
    &\quad+\operatorname{Re}\sum_{nml}\frac{\partial^{3}K(\phi,\phi^{\ast})}{\partial\phi_{n}
    \partial\phi_{m}\partial\phi_{l}^{\ast}}\Bigl(\overline{\psi_{m}}\gamma^{\mu}\psi_{lR}\Bigr)D_{\mu}\phi_{n} \nonumber \\
    &\quad+\frac{1}{4}\sum_{nmlk}\frac{\partial^{4}K(\phi,\phi^{\ast})}{\partial\phi_{n}\partial\phi_{m}
    \partial\phi_{l}^{\ast}\partial\phi_{k}^{\ast}}\Bigl(\overline{\psi_{n}}\psi_{mL}\Bigr)
    \Bigl(\overline{\psi}_{k}\psi_{lR}\Bigr)\nonumber \\
    &\quad-\operatorname{Re}\sum_{nm}\frac{\partial^{2}f(\phi)}{\partial\phi_{n}\partial\phi_{m}}
    \Bigl(\overline{\psi_{n}}\psi_{mL}\Bigr)+2\operatorname{Re}\sum_{n}\mathscr{F}_{n}\frac{\partial f(\phi)}{\partial\phi_{n}} \nonumber \\
    &\quad+\frac{1}{4}\operatorname{Re}\sum_{ABnm}\Bigl(\overline{\lambda_{A}}\lambda_{BL}\Bigr)
     \Bigl(\overline{\psi_{n}}\psi_{mL}\Bigr)\frac{\partial^{2}h_{AB}(\phi)}{\partial\phi_{n}\partial\phi_{m}}
     -\frac{1}{2}\operatorname{Re}\sum_{ABn}\Bigl(\overline{\lambda_{A}}\lambda_{BL}\Bigr)
     \mathscr{F}_{n}\frac{\partial h_{AB}(\phi)}{\partial\phi_{n}} \nonumber \\
     &\quad +\operatorname{Re}\sum_{AB}h_{AB}(\phi)\Biggl[
     -\Bigl(\overline{\lambda_{A}}\,\slashed{D}\lambda_{BR}\Bigr)-\frac{1}{4}f_{A\mu\nu}f_{B}^{\mu\nu}
     +\frac{1}{8}\mi\,\epsilon_{\mu\nu\rho\sigma}f_{A}^{\mu\nu}f_{B}^{\rho\sigma}+\frac{1}{2}D_{A}D_{B} \Biggr]\nonumber \\
     &\quad\qquad +\frac{\sqrt{2}}{4}\operatorname{Re}\sum_{ABn}\frac{\partial h_{AB}(\phi)}{\partial\phi_{n}}
     \Bigl[-\Bigl(\overline{\lambda_{B}}\gamma^{\mu}\gamma^{\nu}\psi_{nL}\Bigr)f_{A\mu\nu}
    +2\mi\Bigl(\overline{\lambda_{B}}\psi_{nL}\Bigr)D_{A}\Bigr] \:.\label{27.4.42}
\end{align}
这个结果的一个有趣特征是, 当超对称性被$\,\mathscr{F}_{n}\,$的一个非零值破缺时, %
在含有$\,\phi_{n}\,$相关函数$\,h_{AB}(\phi)$ 的理论中出现了规范微子质量. 在一些引力传递的超对称性破缺的理论中, 这个机制被用于生成规范微子质量, 这些将在\,\ref{sec:31.7}\,节进行讨论.

\section{树级重求和中的超对称破缺} \label{sec:27.5}

我们在上一节看到, 如果\,Fayet-Iliopoulos\,常数$\,\xi_{A}\,$全为零并且方程$\,\partial f(\phi)/\partial\phi_{n}=0$\,存在一组解, 那么这些方程也会有规范超场的$\,D\,$-分量全为零的解, 进而使得超对称是没有破缺的. 由此得出, 在规范超场和手征超场的可重整理论中, 超对称性在树级近似下能够自发破缺只有两种(互不排斥的)方式: 超势$\,f(\phi)\,$可以被取成使得所有方程$\,\partial f(\phi)/\partial \phi_{n}=0$\,没有解, 或者, 对于有$\,U(1)\,$因子的规范群, 作用量中含有\,Fayet-Iliopoulos\,项.

我们在\,\ref{sec:26.5}\,节已经看到$\,\phi\,$的任何值都不会使$\,\partial f(\phi)/\partial \phi_{n}=0\,$是如何发生的. %
当手征超场与规范超场相互作用时, 那个讨论也不需要做出任何改变, 所以我们转向另一可能性: %
Fayet-Iliopoulos 项产生的超对称性自发破缺. 由于这只对有$\,U(1)\,$因子的规范群才会发生, %
最简单的情况是只有一个$\,U(1)\,$规范群的理论. 正如在\,22.4\,节讨论过的, 为了避免$\,U(1)\,$-$\,U(1)\,$-$\,U(1)\,$反常和%
$\,U(1)\,$-引力-引力反常, 所有左手征超场的$\,U(1)\,$量子数之和以及它们的立方和必须为零. 我们将考虑最简单的可能性: %
两个左手征超场$\,\Phi_{\pm}$, 带有$\,U(1)\,$量子数$\,\pm e$. (这是量子电动力学的超对称版, %
两个超场的旋量分量$\,\psi_{-L}\,$和$\,\psi_{+L}\,$提供了电子场及其电荷共轭场的左手部分.) 在一个可重整理论中, %
最一般的$\,U(1)\,$-不变超势就是$\,f(\Phi)=m\Phi_{+}\Phi_{-}$. %
那么标量势(\ref{27.4.9})对于这些超场的标量分量$\,\phi_{\pm}\,$是
\begin{equation}
    V(\phi_{+},\phi_{-})=m^{2}\lvert\phi_{+}\rvert^{2}+m^{2}\lvert\phi_{-}\rvert^{2}
    +\Bigl(\xi+e^{2}\lvert\phi_{+}\rvert^{2}-e^{2}\lvert\phi_{-}\rvert^{2} \Bigr)^{2} \:. \label{27.5.1}
\end{equation}
除非\,Fayet-Iliopoulos\,常数$\,\xi\,$为零, 否则明显不可能找到$\,V=0\,$的超对称真空. %
当$\,\xi>m^{2}/2e^{2}\,$或 $\xi<-m^{2}/2e^{2}\,$时, 势(\ref{27.5.1})在$\,\phi_{+}=0\,$和%
$\,\lvert\phi_{-}\rvert^{2}=(2e^{2}\xi-m^{2})/2e^{4}\,$或者在$\,\phi_{-}=0\,$和%
$\,\lvert\phi_{+}\rvert^{2}=(-2e^{2}\xi-m^{2})/2e^{4}\,$处有最小值, 这使得\,$U(1)\,$对称性是伴随超对称性破缺的. %
当$\,\lvert\xi\rvert<m^{2}/2e^{2}\,$时, 势能的最小值处在$\,\phi_{+}=\phi_{-}=0$, 所以这里的规范对称性没有破缺. %
超对称性可能的破缺与规范对称性可能的破缺之间一般没有必然的联系.

无论超对称性是通过这里讨论的\,Fayet-Iliopoulos\,机制亦或是\,\ref{sec:26.5}\,节的\,O'Raifeartaigh\,机制亦或是二者的结合体自发破缺, 超对称都在树级近似质量中留有余影. 对于规范超场和手征超场的一般可重整超对称理论, 对它们的拉格朗日量(\ref{27.4.8})的观察表明, 这个理论中的超对称自发破缺对\,\ref{sec:27.4}\,节计算的质量产生了相应的修正.

\subsection{自旋\,0\,质量}

如果$\,\mathscr{F}\,$-项$\,\mathscr{F}_{n}=-(\partial f(\phi)/\partial \phi_{n})\,$在势能的最小值点$\,\phi_{0}\,$处不为零, 那么除了方程(\ref{27.4.15})中列出的那些项, 势能中$\,\varphi_{n}\equiv\phi_{n}-\phi_{n0}\,$的二阶项有额外的项:
\begin{align}
    V_{\mathrm{quad}} &= \sum_{nm}(\mathscr{M}^{\ast}\mathscr{M})_{nm}\varphi_{n}^{\ast}\varphi_{m}
    +\sum_{Anm}\Bigl(t_{A}\phi_{0}\Bigr)_{m}\Bigl(t_{A}\phi_{0}\Bigr)_{m}^{\ast}
    \varphi_{n}^{\ast}\varphi_{m} \nonumber \\
    &\quad +\frac{1}{2}\sum_{Anm}\Bigl(t_{A}\phi_{0}\Bigr)_{n}^{\ast}
    \Bigl(t_{A}\phi_{0}\Bigr)_{m}^{\ast}\varphi_{n}\varphi_{m}
    +\frac{1}{2}\sum_{Anm}\Bigl(t_{A}\phi_{0}\Bigr)_{n}\Bigl(t_{A}\phi_{0}\Bigr)_{m}
    \varphi_{n}^{\ast}\varphi_{m}^{\ast}\nonumber \\
    &\quad+\frac{1}{2}\sum_{nm}\mathscr{N}_{nm}\varphi_{n}\varphi_{m}
    +\frac{1}{2}\sum_{nm}\mathscr{N}_{nm}^{\ast}\varphi_{n}^{\ast}\varphi_{m}^{\ast} \nonumber \\
    &\quad +\sum_{Anm}D_{A0}(t_{A})_{nm} \varphi_{n}^{\ast}\varphi_{m} \label{27.5.2}
\end{align}
其中$\,\mathscr{M}\,$依旧是复对称矩阵(\ref{26.4.11}):
\[
\mathscr{M}_{nm}\equiv \Biggl(\frac{\partial^{2}f(\phi)}{\partial\phi_{n}\partial\phi_{m}}\Biggr)_{\phi=\phi_{0}}\:,
\]
$\mathscr{N}_{nm}\,$是新元素
\begin{equation}
    \mathscr{N}_{nm} \equiv -\sum_{\ell}\mathscr{F}_{\ell 0}
    \Biggl(\frac{\partial^{3}f(\phi)}{\partial\phi_{n}\partial\phi_{m}\partial\phi_{\ell}}\Biggr)_{\phi=\phi_{0}}\:,
    \label{27.5.3}
\end{equation}
而$\,\mathscr{F}_{0}\,$和$\,D_{A0}\,$依旧是手征标量超场和规范超场在势能最小值点处的$\,\mathscr{F}\,$-项和$\,D\,$-项:
\[
\mathscr{F}_{n0}=-\bigg[\frac{\partial f(\phi)}{\partial\phi_{n}}\biggr]^{\ast}_{\phi=\phi_{0}} \:,\qquad
D_{A0}=\xi_{A} + \sum_{nm}\phi_{n0}^{\ast}\,(t_{A})_{nm}\,\phi_{m0} \:.
\]
如果我们势能的二次部分(\ref{27.5.2})写成(\ref{27.4.16})的形式:
\[
    V_{\mathrm{quad}}=\frac{1}{2}\begin{bmatrix}
    \varphi \\ \varphi^{\ast}
\end{bmatrix}^{\dag}
M_{0}^{2}
\begin{bmatrix}
    \varphi \\ \varphi^{\ast}
\end{bmatrix} \:,
\]
那么取代方程(\ref{27.4.17}), 我们现在有标量质量矩阵
\begin{equation}
    M_{0}^{2} = \begin{bmatrix}
        \mathscr{M}^{\ast}\mathscr{M}+\mathscr{A}+\sum_{A}D_{A0}\,t_{A} &
        \mathscr{B}+\mathscr{N}^{\ast} \\[1em]
        \mathscr{B}^{\ast}+\mathscr{N} &
        \mathscr{M}\mathscr{M}^{\ast}+\mathscr{A}^{\ast}+\sum_{A}D_{A0}\,t_{A}^{\mathrm{T}}
    \end{bmatrix} \:, \label{27.5.4}
\end{equation}
其中
\[
\mathscr{A}\equiv \sum_{A}(t_{A}\phi_{0})(t_{A}\phi_{0})^{\dag} \:, \qquad
\mathscr{B}\equiv \sum_{A}(t_{A}\phi_{0})(t_{A}\phi_{0})^{\mathrm{T}} \:.
\]

\subsection{自旋\,1/2\,质量}


费米子质量矩阵$\,M\,$在这里依旧由方程(\ref{27.4.30})给出
\[
M_{nm}=\mathscr{M}_{nm}\:,\qquad M_{nA}=M_{An}=\mi\sqrt{2}(t_{A}\phi_{0})_{n}^{\ast}\:,\qquad
M_{AB}=0 \:.
\]
然而, 取代方程(\ref{27.4.19}), 规范不变性条件(\ref{27.4.18})现在给出
\begin{equation}
    \sum_{m}\mathscr{M}_{nm}(t_{A}\phi_{0})_{m}=\sum_{m}\mathscr{F}_{m0}(t_{A})_{mn} \:.\label{27.5.5}
\end{equation}
因此本征值是费米子质量平方的厄米正定矩阵是
\begin{align}
    (M^{\dag}M)_{nm} &= (\mathscr{M}^{\dag}\mathscr{M})_{nm}
    +2\sum_{A}(t_{A}\phi_{0})_{n}(t_{A}\phi_{0})_{m}^{\ast} \:, \nonumber \\
    (M^{\dag}M)_{AB} &= 2(\phi_{0}^{\dag}t_{B}t_{A}\phi_{0}) \:, \label{27.5.6}\\[1em]
    (M^{\dag}M)_{An} &= (M^{\dag}M)^{\ast}_{nA} =\mi\sqrt{2}\mathscr{F}_{m0}(t_{A})_{mn}\:. \nonumber
\end{align}


\subsection{自旋\,1\,质量}


矢量玻色子的质量平方依旧由矩阵(\ref{27.4.36})的本征值给出:
\begin{equation}
    (\mu^{2})_{AB} =\Bigl(\phi_{0}^{\dag}\,,\{t_{B},t_{A}\}\phi_{0}\Bigr) \:. \label{27.5.7}
\end{equation}

除了方程(\ref{27.5.4})中的$\,D\,$-项有一个例外外, 质量平方矩阵的变化都在它们的非对角元部分. 因此, 方程(\ref{27.5.4}), %
(\ref{27.5.6})\,和\,(\ref{27.5.7})\,对这些矩阵的迹给出了特别简单的结果: 对于自旋\,0
\begin{equation}
    \operatorname{Tr}M_{0}^{2}=2\operatorname{Tr}(\mathscr{M}^{\ast}\mathscr{M})
    +\operatorname{Tr}\mu^{2}+2\sum_{A}D_{A0}\operatorname{Tr}t_{A} \: \label{27.5.8}
\end{equation}
以及对于自旋\,1/2
\begin{equation}
    \operatorname{Tr}(M^{\dag}M)=\operatorname{Tr}(\mathscr{M}^{\ast}\mathscr{M})+2\operatorname{Tr}\mu^{2}\:.
    \label{27.5.9}
\end{equation}
由于迹是本征值的和, 我们从此获得了一个{\kai{质量求和规则}}:
\begin{equation}
\sum_{\mathrm{spin}\:0}\text{质量}^{2}-2\sum_{\mathrm{spin}\:1/2}\text{质量}^{2}
+3\sum_{\mathrm{spin}\:1}\text{质量}^{2}=-2\sum_{A}D_{A0}\operatorname{Tr}t_{A}\:. \label{27.5.10}
\end{equation}
除非$\,t_{A}\,$是$\,U(1)\,$生成元, 否则$\,t_{A}\,$的迹自动为零, 并且, 正如\,22.4\,节所提及的, %
为了避免引力贡献一个会破坏$\,U(1)\,$流守恒的反常, $U(1)\,$规范生成元的迹(当取遍所有左手费米子时)也必须为零. %
因此(\ref{27.5.10})给出了更简单的结果\cite{4}
\begin{equation}
\sum_{\mathrm{spin}\:0}\text{质量}^{2}-2\sum_{\mathrm{spin}\:1/2}\text{质量}^{2}
+3\sum_{\mathrm{spin}\:1}\text{质量}^{2}=0\:. \label{27.5.11}
\end{equation}
当然, 电荷, 色荷, 重子数和轻子数守恒没有被破坏使得质量矩阵没有矩阵元来连接这些量子数取不同值的粒子, %
所以所有这些结果对每组守恒的量子数分别成立.

在标准模型的最小超对称扩张中, 求和规则(\ref{27.5.11})通常为超对称在树级近似下自发破缺的模型提供了反对的证据. %
我们会在\,\ref{sec:28.3}\,节连同其它讨论来讨论这点.

正如已经在\,\ref{sec:26.5}\,节观察到的(将在\,\ref{sec:29.1}\,节和\,\ref{sec:29.2}\,节进行更普遍的而讨论), %
超对称形的自发破缺必然要求存在无质量费米子, 戈德斯通微子. 对于树级近似下的可重整理论, %
戈德斯通微子场$\,g\,$出现在手征超场和规范超场的旋量分量$\,\psi_{n}\,$和$\,\lambda_{A}\,$中, 系数是
\begin{equation}
\psi_{nL}=\mi\sqrt{2}\mathscr{F}_{n0}\,g_{L}+\cdots \:, \qquad
\lambda_{AL}= D_{A0}\,g_{L}+\cdots\:, \label{27.5.12}
\end{equation}
其中省略号代表与质量明确非零的旋量场相关的项. 为了验证这点, 我们必须证明$\,(\mi\sqrt{2}\mathscr{F}_{n0},D_{A0})\,$\\ 是费米子质量平方矩阵$\,M^{\dag}M\,$本征值为零的本征矢量. 为此, 我们将需要使用势(\ref{27.4.9})在$\,\phi=\phi_{0}\,$处稳定这一条件:
\begin{equation}
0=\frac{\partial V}{\partial \phi_{n}}\biggr\rvert_{\phi=\phi_{0}}=
-\sum_{m}\mathscr{M}_{nm}\mathscr{F}_{m0}+\sum_{A}D_{A0}(\phi_{0}^{\dag}t_{A})_{n} \:. \label{27.5.13}
\end{equation}
我们同时需要规范不变性条件(\ref{27.4.12}), 它在$\,\phi=\phi_{0}\,$处是
\begin{equation}
\sum_{n}\mathscr{F}_{n0}\,(t_{A}\phi_{0})_{n}=0 \:. \label{27.5.14}
\end{equation}
这样, 结合方程(\ref{27.5.13})和(\ref{27.5.14})与方程(\ref{27.5.5})和(\ref{27.5.6})就给出了
\begin{equation}
\mi\sqrt{2}\sum_{m}(M^{\dag}M)_{nm}\mathscr{F}_{m0}=\mi\sqrt{2}\sum_{A}D_{A}(t_{A}\mathscr{F}_{0}^{\ast})_{n}
=-\sum_{A}(M^{\dag}M)_{nA}D_{A0} \label{27.5.15}
\end{equation}
和
\begin{equation}
\mi\sqrt{2}\sum_{m}(M^{\dag}M)_{Am}\mathscr{F}_{m0}=-2\sum_{nm}\mathscr{F}_{n0}\,(t_{A})_{nm}\,\mathscr{F}_{m0}
=-\sum_{B}(M^{\dag}M)_{AB}D_{B0}  \:.\label{27.5.16}
\end{equation}
即,
\begin{equation}
M^{\dag}M \begin{pmatrix} \mi\sqrt{2}\mathscr{F}_{0} \\ D_{0}\end{pmatrix} =0 \:, \label{27.5.17}
\end{equation}
而这正是所要证明的.


\section{微扰无重整定理} \label{sec:27.6}



自一开始, 普通可重整量子场论中的数个发散就被发现在这些理论的超对称版中消失了. 随着\,1975\,年对超图技术的发展, %
证明一些辐射修正不仅有限并且在微扰论中消失了变得可行. 超图将会在第\,\ref{cha:30}\,章进行细致的描述, %
但实际上证明最重要的无重整定理(non-renormaliation theorem)并不需要它们. %
这一节将会给出\,Seiberg\cite{6}在\,1993\,发展的方法的一个版本, %
这个方法将会展示如何从对对称性和解析性的简单考察中得到无重整定理.

考察一个一般的有数个左手征超场$\,\Phi_{n}\,$和(或)规范超场$\,V_{A}\,$的可重整超对称规范理论. 我们在\,\ref{sec:27.3}\,节提到过, 如果我们从$\,t_{A}\,$和$\,C_{ABC}\,$中移出因子$\,g\,$转而把它放进规范超场中, 那么拉格朗日密度有如下的形式
\begin{equation}
\mathscr{L} =\Bigl[\Phi^{\dag}\,\me^{-V}\,\Phi\Bigr]_{D} + 2\operatorname{Re}\Bigl[f(\Phi)\Bigr]_{\mathscr{F}}
+\frac{1}{2g^{2}}\operatorname{Re}
\Bigl[\sum_{A\alpha\beta}\epsilon_{\alpha\beta}W_{A\alpha L}\,W_{A\beta L}\Bigr]_{\mathscr{F}} \:,\label{27.6.1}
\end{equation}
其中超势$\,f(\Phi)\,$是左手征超场规范不变的三次多项式. (我们忽略了可能存在的$\,\theta\,$-项, 它在微扰论中没有任何效应.)

假定我们给圈图中环流的动量附加一个紫外截断$\,\lambda$. 就像在\,12.4\,节讨论过的, %
我们可以找到一个带有这个截断的{\kai{定域}}``威尔逊型''有效拉格朗日量$\,\mathscr{L}_{\lambda}$, %
对于动量低于$\,\lambda\,$的过程的$\,S\,$-矩阵, 它会给出与原始拉格朗日密度相同的结果. %
有效拉格朗日密度的质量和耦合常数现在会依赖于$\,\lambda$, 而且有效拉格朗日量中通常会有无限多个耦合项, %
即理论的对称性允许的所有可能的项. 然而在超对称理论中, 情况要简单的多. 无重整定理告诉我们, %
只要截断不破坏超对称性和规范不变性, 直到微扰论的所有阶, 有效拉格朗日量将会有如下的结构
\begin{align}
\mathscr{L}_{\lambda}&=\Bigl[\mathscr{A}_{\lambda}(\Phi,\Phi^{\dag},V,\mathscr{D},\cdots)\Bigr]_{D}
+2\operatorname{Re}\Bigl[f(\Phi)\Bigr]_{\mathscr{F}} \nonumber \\
&\quad +\frac{1}{2g_{\lambda}^{2}} \operatorname{Re}
\Bigl[\sum_{A\alpha\beta}\epsilon_{\alpha\beta}W_{A\alpha L}\,W_{A\beta L}\Bigr]_{\mathscr{F}} \:,\label{27.6.2}
\end{align}
其中$\,\mathscr{A}_{\lambda}\,$是一般的\,Lorentz\,不变且规范不变的函数; ``$\mathscr{D}\cdots$''代表的项包含了对后继变量的超导数或时空导数; $g_{\lambda}\,$是{\kai{单圈}}有效规范耦合, 给出它的公式与单圈重整化规范耦合常数相同,
\begin{equation}
g_{\lambda}^{-2}=\text{常数}-2b\ln\lambda \:, \label{27.6.3}
\end{equation}
其中$\,b\,$是\,Gell-Mann--Low\,函数$\,\beta(g)\,$中$\,g^{3}\,$的系数, 我们在第\,18\,章讨论过. 这是只有一个规范耦合的单规范群的结果, 但是到单规范群和$\,U(1)\,$规范群直积的推广是平庸的. 特别地, 注意到有效超势不仅在$\,\lambda\to\infty\,$的极限是有限的, 而且至少在微扰论中它不包含原先超势中没有的项, 并且它所包含的那些项的系数没有任何变化.

为了证明这个定理, 我们将会把这个理论解释成有额外两个外规范不变左手征超场的理论的特殊情况, 这个理论的拉格朗日密度是
\begin{equation}
\mathscr{L}^{\sharp}=\frac{1}{2}\Bigl[\Phi^{\dag}\,\me^{-V}\,\Phi\Bigr]_{D}+
2\operatorname{Re}\Bigl[Y\,f(\Phi)\Bigr]_{\mathscr{F}}+\frac{1}{2}\operatorname{Re}
\Bigl[X\sum_{A\alpha\beta}\epsilon_{\alpha\beta}W_{A\alpha L}\,W_{A\beta L}\Bigr]_{\mathscr{F}} \:.\label{27.6.4}
\end{equation}
当$\,X\,$和$\,Y\,$的标量分量$\,x\,$和$\,y\,$被赋予值$\,x=1/g^{2}\,$和$\,y=1\,$且它们的旋量分量和辅助分量被设为零时, %
这个拉格朗日密度就与原始的拉格朗日密度相同. 由于假定了在截断处理中超对称性和规范不变性是被保护的, %
有这些外超场的有效拉格朗日密度必须是一般超场的$\,D\,$-项与左手征超场的$\,\mathscr{F}\,$-项之和:
\begin{equation}
\mathscr{L}_{\lambda}^{\sharp} = \Bigl[\mathscr{A}_{\lambda}(\Phi,\Phi^{\dag},V,X,X^{\dag},Y,Y^{\dag},\mathscr{D},\cdots)\Bigr]_{D}
+2\operatorname{Re}\Bigl[\mathscr{B}_{\lambda}(\Phi,W_{L},X,Y)\Bigr]_{\mathscr{F}} \:, \label{27.6.5}
\end{equation}
其中$\,\mathscr{A}_{\lambda}\,$和$\,\mathscr{B}_{\lambda}\,$均是写出变量的规范不变函数. %
我们不在$\,\mathscr{F}\,$-项引入任何超导数或时空导数, 同 \ref{sec:26.3}\,节一样, %
这是因为包含任何左手征超场或它们共轭的导数的项可以重写为对$\,[\mathscr{A}_{\lambda}]_{D}\,$的贡献. %
(诚然, 方程(\ref{27.3.12})表明$\,W_{L}\,$本身是由两个$\,\mathscr{D}_{R}\,$作用在一个超场%
$\,\exp(-2V)\mathscr{D}_{L}\exp(2V)\,$上给出的, 但这个超场不是规范不变的, %
而我们要求$\,\mathscr{A}_{\lambda}\,$是规范不变的.)

从拉格朗日密度(\ref{27.6.4})获得的两个对称性严格限制了$\,\mathscr{B}_{\lambda}\,$对$\,X\,$和$\,Y\,$的依赖性. %
(这两个对称性都被非微扰效应破缺了, 这将会在第\,\ref{cha:29}\,章考虑.) %
第一个对称性是\,\ref{sec:26.3}\,节讨论的那种微扰论性的$\,U(1)\,R$-对称性, %
其中$\,\theta_{L}\,$和$\,\theta_{R}\,$被赋予$\,R\,$值$\,+1\,$和$\,-1$, 超场$\,\Phi,$ $V\,$和$\,X\,$是$\,R\,$-中性的, %
而$\,Y\,$的$\,R\,$-值是$\,+2$. (回忆, $f_{\mathscr{F}}\,$是$\,f\,$中$\,\theta_{L}^{2}\,$的系数, %
所以为了使$\,f_{\mathscr{F}}\,$的$\,R\,$-值为\,0, 所以$\,f\,$的$\,R\,$-值必须是\,2.) %
因为$\,W_{L}\,$是由两个$\,\mathscr{D}_{R}\,$和一个$\,\mathscr{D}_{L}\,$作用在$\,R\,$-中性超场上给出的, %
它的$\,R\,$-值是$\,+1$. 现在, $R\,$-不变性要求$\,\mathscr{B}_{\lambda}\,$同超势一样有$\,R\,$-值$\,+2$. %
它不能依赖于任何$\,R\,$值为负的超场, 例如左手征超场的共轭, 因为它是{\kai{全纯的}}, %
所以$\,\mathscr{B}_{\lambda}\,$只能是$\,Y\,$的一阶或者是$\,W_{L}\,$的二阶, %
而系数只能依赖于$\,R\,$-中性超场$\,\Phi\,$和(或) $X$:
\begin{equation}
\mathscr{B}_{\lambda}(\Phi,W_{L},X,Y)=Y\,f_{\lambda}(\Phi,X)+\sum_{\alpha\beta AB}
\epsilon_{\alpha\beta}W_{A\alpha L}W_{B\beta L}h_{\lambda AB}(\Phi,X) \:. \label{27.6.6}
\end{equation}
(Lorentz\,不变性要求$\,W_{L}\,$的旋量指标要与$\,\epsilon_{\alpha\beta}\,$收缩.) 另一个对称性是$\,X\,$平移一个虚的数值常数, %
$X\to X+\mi\xi$, 其中$\,\xi\,$是实数. 它对拉格朗日密度(\ref{27.6.4})的改变正比于$\,\operatorname{Im}\sum_{A\alpha\beta}
W_{A\alpha L}W_{A\beta L}$, 而正如我们在\,\ref{sec:27.3}\,节看到的, 这是时空导数, 因此在微扰论中没有效应. %
除了$\,X\,$在原始拉格朗日密度出现的地方, 这个平移对称性使得$\,X\,$无法出现在有效拉格朗日密度(\ref{27.6.5})的其它任何地方. %
因此我们得出$\,f_{\lambda}\,$独立于$\,X$, 而$\,h_{\lambda AB}\,$由一个正比于$\,X\delta_{AB}\,$的$\,\Phi\,$-无关项%
和一个与$\,X\,$独立的项构成. 即,
\begin{equation}
\mathscr{B}_{\lambda}(\Phi,W_{L},X,Y)=Y\,f_{\lambda}(\Phi,X)+\sum_{\alpha\beta AB}
\epsilon_{\alpha\beta}W_{A\alpha L}W_{B\beta L}\Bigl[c_{\lambda}\delta_{AB}X+\ell_{\lambda AB}(\Phi)\Bigr] \:. \label{27.6.7}
\end{equation}
其中$\,c_{\lambda}\,$是截断无关的实常数.

引入外辅助超场$\,X\,$和$\,Y\,$的目的在于, 通过赋予它们合适的值, 我们可以使用弱耦合近似定出方程(\ref{27.6.7})中的系数. %
如果我们令$\,X\,$和$\,Y\,$的旋量分量和辅助分量为零, 并令它们的标量分量分别趋于无穷大和零, %
那么规范耦合常数将以$\,1/\sqrt{x}\,$的速率趋于零, 而从超势导出的所有\,Yukawa\,耦合和标量耦合将以$\,y\,$的速率趋于零. %
在这个极限下, 对(\ref{27.6.7})中正比于$\,Y\,$的项有贡献的只有一个图, 这个图有一个来自于$\,2\operatorname{Re}[Y f(\Phi)]_{\mathscr{F}}\,$的单顶点, 所以
\begin{equation}
f_{\lambda}(\Phi)=f(\Phi) \:. \label{27.6.8}
\end{equation}
另外, 在$\,Y=0\,$时, 有一个守恒律要求$\,\mathscr{L}_{\lambda}^{\sharp}\,$中的所有项%
有相同数目的$\,\Phi\,$和$\,\Phi^{\dag}$, %
又因为$\,\Phi^{\dag}\,$不能出现在$\,\ell_{\lambda AB}\,$中, 所以$\,\Phi\,$也不能. 这样, 对于单群, %
规范不变性就要求常数$\,\ell_{\lambda AB}\,$正比于$\,\delta_{AB}$:
\begin{equation}
\ell_{\lambda AB}=\delta_{AB}L_{\lambda}\:. \label{27.6.9}
\end{equation}
现在, 由于规范传播子趋于$\,1/x\,$而纯规范相互作用趋于$\,x\,$且标量传播子与相互作用与$\,x\,$无关, 在 $y=0\,$时, %
对于有$\,V_{W}\,$个纯规范玻色则顶点, $I_{W}\,$个规范玻色子内线以及任意多个标量-规范玻色子顶点和标量传播子的图, %
这个图中$\,x\,$的幂次是
\begin{equation}
    N_{x}=V_{W}-I_{W} \:. \label{27.6.10}
\end{equation}
圈的个数是
\begin{equation}
    L=I_{W}+I_{\Phi}-V_{W}-V_{\Phi}+1 \:, \label{27.6.11}
\end{equation}
其中$\,I_{\Phi}\,$是内$\,\Phi\,$线的个数, $V_{\Phi}\,$是$\,\Phi$-$V\,$相互作用顶点的个数. %
所有$\,\Phi$-$V$\,顶点有两个$\,\Phi\,$线与其相连, 所以当没有外$\,\Phi\,$线时, $I_{\Phi}\,$等于$\,V_{\Phi}$, %
因而在方程(\ref{27.6.11})中抵消了, 这使得方程(\ref{27.6.10})可以写成
\begin{equation}
    N_{x}=1-L \:. \label{27.6.12}
\end{equation}
因此树级近似正确地给出了$\,X\,$在方程(\ref{27.6.7})中的系数$\,c_{\lambda}$, 因而这个系数也就是它在原始拉格朗日量中的值, %
即$\,c_{\lambda}=1$, 而$\,X\,$-无关项的系数$\,L_{\lambda}\,$仅由单圈图给定. 综上, 我们有
\begin{align}
    \mathscr{L}_{\lambda}^{\sharp}&= \Bigl[\mathscr{A}_{\lambda}(\Phi,\Phi^{\dag},V,X,X^{\dag},Y,Y^{\dag},\mathscr{D}\cdots)\Bigr]_D
    +2\operatorname{Re}\Bigl[Y\,f(\Phi)\Bigr]_{\mathscr{F}} \nonumber \\
    &\qquad +\frac{1}{2}\operatorname{Re}\Biggl[\Bigl(X+L_{\lambda}\Bigr)
    \sum_{A\alpha\beta}\epsilon_{\alpha\beta}W_{A\alpha L}W_{A\beta L}\Biggr]_{\mathscr{F}} \:, \label{27.6.13}
\end{align}
其中$\,L_{\lambda}\,$是单圈贡献. 令$\,Y=1\,$和$\,X=1/g^{2}\,$就给出了方程(\ref{27.6.2}), %
其中$\,g_{\lambda}^{-2}=g^{-2}+L_{\lambda}$. 正如\,18.3\,中展示过的, 无论用何种重整化方案定义耦合$\,g_{\lambda}$, %
对$\,\lambda \dif g_{\lambda}/\dif \lambda\,$的领头阶贡献是$\,g_{\lambda}\,$的相同函数, 所以到单圈阶, 我们必须有
\begin{equation}
    \lambda \,\dif g_{\lambda}/\dif\lambda =b\,g_{\lambda}^{3} \:, \label{27.6.14}
\end{equation}
其中$\,b\,$与\,Gell-Mann\,和\,Low\,的重整化群方程中$\,g^{3}\,$的系数相同. 解是方程(\ref{27.6.3}), 完成了证明.

\begin{figure}[t]
  \centering
   \begin{tikzpicture}[scale=0.65]
  \draw[line width=0.30mm, dashed](-5.5,1.5)--(-0.5,1.5);
  \draw[line width=0.30mm, dashed](1.5,1.5) ellipse (2 and 2);
  \draw[line width=0.32mm](1.1,3.7)--(1.5,3.5)--(1.1,3.3);
  \draw[line width=0.32mm](1.6,-0.3)--(1.2,-0.5)--(1.6,-0.7);
  \draw[line width=0.32mm](-3.3,1.7)--(-2.9,1.5)--(-3.3,1.3);
  \end{tikzpicture}
  \vspace{5 mm}
  \caption{超对称性被标量场和它们的共轭之间的三线性耦合破缺的理论中二次发散的单圈图. 这些线均代表复标量场.}%
  \label{fig:27.1}%
\end{figure}


在有一个$\,U(1)\,$规范超场$\,V_{1}\,$的理论中, 拉格朗日量可能会包含一个\,Fayet--Iliopoulos\,项(\ref{27.2.7}):
\begin{equation}
    \mathscr{L}_{\mathrm{FI}}=\xi\,\Bigl[V_{1}\Bigr]_{D} \:. \label{27.6.15}
\end{equation}
容易看到这种项的系数$\,\xi\,$是没有重整化过的.\cite{7} 如果威尔逊型拉格朗日密度中的相应系数$\,\xi_{\lambda}\,$不依赖与规范耦合或者超势中的耦合, 那么当我们将原始拉格朗日量(\ref{27.6.1})替换成包含外超场$\,X\,$和 $Y\,$的拉格朗日量(\ref{27.6.4})时, 超对称性就会要求威尔逊型拉格朗日量中的这一项取如下的形式
\begin{equation}
    \mathscr{L}_{\mathrm{FI}\,\lambda}^{\sharp}=
    \Bigl[\xi_{\lambda}(X,Y,X^{\ast},Y^{\ast})\,V_{1}\Bigr]_{D} \:, \label{27.6.16}
\end{equation}
其中$\,\xi_{\lambda}\,$是一个以不平庸的方式依赖于$\,X\,$和(或)$\,Y\,$和(或)它们的共轭的函数. 但这样的项不会是规范不变的, %
这是因为, 根据方程(\ref{27.2.18}), 规范变换会使$\,V_{1}\,$偏移一个手征超场$\,\mi(\Omega-\Omega^{\ast})/2$, 而尽管一个手征超场的$\,D\,$-项为零, 如果$\,\xi_{\lambda}\,$依赖于其它超场, $\mi(\Omega-\Omega^{\ast})/2\,$与$\,\xi_{\lambda}\,$的乘积对于一般的规范变换不是手征的. 确实有对$\,\xi_{\lambda}\,$有贡献且独立于所有耦合常数的图. 对于拉格朗日量(\ref{27.6.1}), 在规范超场与手征物质相互左右的顶点没有规范耦合$\,g\,$的因子, 但每个规范传播子有一个因子$\,g^{-2}$, 所以一个没有内规范线且没有手征超场自耦合的图将不会依赖于耦合常数. 这样的对$\,\xi_{\lambda}\,$有贡献的图只有那些单个外规范线与一个手征圈相连的图. %
(参看图\ref{fig:27.1}). 所有这种图的贡献正比于所有手征超场的规范耦合之和------即, 正比于$\,U(1)\,$生成元的迹. %
但就像在\,22.4\,节讨论过的那样, (如果$\,U(1)\,$对称性是不破缺的,) 为了避免破坏$\,U(1)\,$流守恒的引力反常, 这个迹必须为零.

这些定理的最重要应用是一个推论, 这个推论告诉我们, 如果没有\,Fayet--Iliopoulos\,项且如果超势$\,f(\Phi)\,$使得%
方程$\,\partial f(\phi)/\partial \phi_{n}=0\,$有解, 那么超对称性在微扰论的任何有限阶都是不破缺的.

为了检验这点, 我们必须检查\,Lorentz\,不变的场构形, 在这种场构形中, $\Phi_{n}\,$只有常标量分量$\,\phi_{n}$ 和常辅助分量$\,\mathscr{F}_{n}$, 而矩阵规范超场$\,V\,$中规范生成元$\,t_{A}\,$的系数$\,V_{A}\,$(在\,Wess-Zumino\,规范下)%
只有辅助分量$\,D_{A}$. 如果存在$\,\phi_{n}\,$的值使得$\,\mathscr{L}_{\lambda}\,$中没有$\,\mathscr{F}_{n}\,$或$\,D_{A}\,$的一阶项, 这时当然就有$\,\mathscr{F}_{n}=D_{A}=0\,$的平衡解, 那么超对称性就是不破缺的. (在\,\ref{sec:29.2}\,节, 我们将看到这是超对称性不破缺的充分必要条件.) 在没有\,Fayet-Iliopoulos\,项时, 如果对于所有的$\,A\,$有
\begin{equation}
    \sum_{nm}\frac{\partial K_{\lambda}(\phi,\phi^{\ast})}{\partial \phi_{n}^{\ast}}(t_{A})_{mn}\phi_{m}^{\ast}=0
    \label{27.6.17}
\end{equation}
闭对所有的$\,n\,$有
\begin{equation}
    \frac{\partial f(\phi)}{\partial\phi_{n}}=0 \:, \label{27.6.18}
\end{equation}
其中有效\,Kahler\,势$\,K_{\lambda}(\phi,\phi^{\ast})\,$是
\begin{equation}
    K_{\lambda}(\phi,\phi^{\ast})=\mathscr{A}_{\lambda}(\phi,\phi^{\ast},0,0\cdots)\:,\label{27.6.19}
\end{equation}
$A_{\lambda}(\phi,\phi^{\ast},0,0\cdots)\,$从$\,\mathscr{A}_{\lambda}\,$中通过设规范超场和所有超势等于零获得的, %
那么就将是这样的情况. (在超导数被\,Lorentz\,不变性要求为零后, %
$\mathscr{A}_{\lambda}\,$对$\,V\,$的唯一依赖是每个$\,\Phi^{\dag}\,$因子后面的因子$\,\exp(-V)$.) %
我们现在使用在\,\ref{sec:27.4}\,节用过的一个技巧. 如果方程(\ref{27.6.18})有任何解$\,\phi^{(0)}$, %
那么规范对称性告诉我们存在这样一组连续的解, 即$\phi_{n}\,$被替换成
\begin{equation}
    \phi_{n}(z) =\Bigl[\exp(\mi\sum_{A}t_{A}\,z_{A})\Bigr]_{nm}\phi_{m}^{(0)}\:, \label{27.6.20}
\end{equation}
其中(既然$\,f\,$只依赖于$\,\phi\,$而不依赖于$\,\phi^{\ast}$)$\,z_{A}\,$是任意一组{\kai{复}}参量. %
如果$\,K_{\lambda}(\phi,\phi^{\ast})\,$在曲面$\,\phi=\phi(z)\,$的任何一处有一个稳定点, 那么在那一点
\begin{equation}
    0=\sum_{nmA}\frac{\partial K_{\lambda}(\phi,\phi^{\ast})}{\partial\phi_{n}}(t_{A})_{nm}\phi_{m}\,\delta z_{A}
    -\sum_{nmA}\frac{\partial K_{\lambda}(\phi,\phi^{\ast})}{\partial\phi_{n}^{\ast}}
    (t_{A})_{mn}\phi_{m}^{\ast}\,\delta z_{A}^{\ast} \:. \label{27.6.21}
\end{equation}
由于这对所有无限小的{\kai{复}}\,$\delta z_{A}\,$都必须满足, %
$\delta z_{A}\,$和$\,\delta z_{A}^{\ast}\,$的系数都必须为零, %
所以方程(\ref{27.6.17})和方程(\ref{27.6.18})在这一点都是满足的. %
因此$\,K_{\lambda}(\phi,\phi^{\ast})\,$在曲面$\,\phi=\phi(z)\,$上有稳定点表明超对称性在微扰论的所有阶都是不破缺的. %
零阶\,Kahler\,势$\,(\phi^{\dag}\phi)\,$下有解且在$\,\phi\to\infty\,$时趋于无穷, %
所以它肯定在曲面$\,\phi=\phi(z)\,$上有最小值点, 那么显然它在这个点上是稳定的. 如果这个最小值点没有平坦方向, %
即$\,K_{\lambda}\,$在这个方向是常数, 那么对\,Kahler\,势任何充分小的微扰会移动这个最小值点, 但是不会摧毁它. %
Kahler\,势在曲面$\,\phi=\phi(z)\,$上的最小值点有平坦方向: $z_{A}\,$为实数的普通整体规范变换%
$\,\delta \phi=\mi\sum_{A}\delta z_{A}t_{A}\phi$. 但它们同时是微扰$\,K_{\lambda}(\phi,\phi^{\ast})-(\phi^{\dag},\phi)\,$的平坦方向, 所以至少对于任何在有限范围内的微扰, $K_{\lambda}\,$在曲面$\,\phi=\phi(z)\,$上仍然有一个定域最小值点, %
因此对于$\,K_{\lambda}(\phi,\phi^{\ast})\,$中出现的任何耦合常数, 到这些耦合常数的所有阶, %
$K_{\lambda}\,$在曲面$\,\phi=\phi(z)\,$上都有一个定域最小值点. 而正如我们看到的, %
这是使得对于所有$\,n\,$和$\,A\,$都有$\,\mathscr{F}_{n}=0\,$和$\,D_{A}=0\,$的标量场值, 也就意味着超对称性是不破缺的.

\subsection*{* * *}

这些结果也可以被推广至不可重整理论.\cite{3} 在这种理论中, 方程(\ref{27.6.1})中的第一项$\,[\Phi^{\dag}\me^{-V}\Phi]_{D}$ 要被替换成$\,\Phi^{\dag}$, $\Phi$, $V\,$以及它们的超导数和时空导数的一个任意规范不变实标量函数的$\,D\,$-项, %
而方程(\ref{27.6.1})中的第二项和第三项要被换成$\,\Phi_{n}\,$和$\,W_{\alpha}\,$的任意一个整体规范不变标量函数%
$\,f(\Phi,W)$ 的$\,\mathscr{F}\,$-项. 业已证明, 到微扰论的所有阶, 除了$\,W\,$二次项的单圈重整化外, %
威尔逊型拉格朗日量的$\,\mathscr{F}\,$-项中出现的函数$\,f_{\lambda}(\Phi,W)\,$与$\,f(\Phi,W)\,$相同.


\section[超对称软破缺]{超对称软破缺\footnote{本节有些脱离本书的发展主线, 可以在第一次阅读时跳过.}} \label{sec:27.7}

我们会在下一章看到, 即使超对称性是作用量的一个精确对称性, 超对称性在高能的自发破缺会在描述低能物理的有效作用量中产生违反超对称守恒的超可重整项. 这些超可重整项可以解释在可到达能量处没有观测到超对称性的现象. 在这一节, 我们将考虑由这种破坏超对称性的超可重整项产生的辐射修正, 这部分是为了看到这是否为标准模型的超对称版本中引入或排除这种项提供了一个判据.

超对称性破缺的迹象是一般超场的$\,D\,$-项或手征超场的$\,\mathscr{F}\,$-项有期望值. %
拉格朗日密度中破缺超对称性的任何算符$\,\epsilon\mathcal{O}\,$都可以以超对称的形式写成一个$\,D\,$-项
\begin{equation}
    \epsilon\,\mathcal{O}=\Bigl[Z\,S\Bigr]_{D} \:, \label{27.7.1}
\end{equation}
其中$\,S\,$是非手征超场, 它的$\,C\,$-项是$\,\mathcal{O}$, 而$\,Z\,$是非手征外超场, %
它唯一不为零的分量是$\,[Z]_{D}=\epsilon$. 部分但不是全部破坏对称性的算符$\,\epsilon\mathcal{O}\,$也可以写成%
$\,\mathscr{F}\,$-项,
\begin{equation}
    \epsilon \,\mathcal{O}=\Bigl[\Omega\,O\Bigr]_{\mathscr{F}} \:, \label{27.7.2}
\end{equation}
或者它们的共轭, 其中$\,O\,$是$\,\mathscr{F}\,$-项是$\,\mathscr{O}\,$的左手征超场, 而$\,\Omega\,$是外左手征超场, %
它唯一不为零的分量是$\,[\Omega]_{\mathscr{F}}=\epsilon$. 对于有效拉格朗日量中一个会出现的给定修正, %
通过计算以超对称的方式构造这个修正所需要的$\,Z\,$或$\,\Omega\,$的幂次, 我们可以数出$\,\epsilon\,$的阶数. %
对于可以写成(\ref{27.7.2})和(\ref{27.7.1})的形式的那些相互作用, 我们会发现这些相互作用产生的修正上数个有趣的限制.

根据上一节的结果, $\mathscr{F}\,$-项没有辐射修正, 所以所有对威尔逊型拉格朗日量的破缺超对称的辐射修正%
必须采取$\,D\,$-项的形式. 这个定理并不阻止任何给定算符出现在威尔逊型拉格朗日量中, 这是因为, %
即使一个算符$\,\epsilon\Delta\mathscr{L}\,$无法表示成$\,[Z\Lambda]_{D}\,$的形式, %
其中$\,\Lambda\,$是$\,C\,$-项为$\,\Delta\mathscr{L}\,$的一般超场, 但 $\epsilon^{2}\Delta\mathscr{L}\,$依旧{\kai{能}}表示成
\begin{equation}
    \epsilon^{2}\Delta\mathscr{L} = 2\Bigl[\Omega^{\ast}\Omega\Lambda\Bigr]_{D} \:. \label{27.7.3}
\end{equation}
但并非所有的算符是由$\,\Omega\,$或$\,\Omega^{\ast}\,$的{\kai{一阶}}辐射修正产生的. 特别地, %
如果一个函数仅是左手征超场$\,\Phi$ 的$\,\phi\,$项的函数, 而不是$\,\phi^{\ast}\,$的函数, %
那么它无法写为对$\,\Omega\,$是线性的超场的$\,D\,$-项. (注意, $[\Omega h(\Phi)]_{D}\,$是一个导数, %
而$\,[\Omega^{\ast}h(\Phi)]_{D}=2[\Phi]_{\mathscr{F}}\partial h(\phi)/\partial\phi\,$不只是$\,\phi\,$的函数.) %
我们由此得出{\kai{威尔逊型拉格朗日量中只依赖于$\,\phi\,$的超对称型破缺项无法由对形如(\ref{27.7.2})的超对称破缺相互作用是一阶的辐射修正产生.}}


这个结果是重要的, 因为绝大多数发散的辐射修正是超可重整耦合的最低阶. 更精确些, %
对于一个量纲(按能量的幂次)为$\,\mathscr{D}\,$的相互作用, 它的系数的量纲是$\,4-\mathscr{D}$, %
所以对一个量纲为$\,d\,$的相互作用, 量纲分析表明一组量纲为$\,d_{1}$, $d_{2}\,$等的相互作用对这个相互作用的系数的贡献至多包含紫外截断的
\begin{equation}
    p=4-d-(4-d_{1})-(4-d_{2})-\cdots \label{27.7.4}
\end{equation}
次方, 因此在$\,p<0\,$是有限的. (这个讨论忽略了子积分中可能存在的紫外发散; 关于这个问题的全面处理, 参考文献[8].) %
超可重整相互作用是``软''的, 也就是说, 它们会减少它们出现的图的发散度. 特别地, %
在一个所有相互作用都有$\,d_{i}\leq 4\,$且$\,d_{i}=4\,$的严格可重整相互作用是超对称的可重整理论中, 对$\,d=4\,$的相互作用%
一个或多个超可重整相互作用对这个相互作用的系数的贡献总会有$\,p<0$, 所以即使它们不是超对称的, %
超可重整相互作用不会对超对称$\,d=4\,$相互作用的系数产生破缺超对称且紫外发散的修正.

另一方面, 这种力量中或许会有对超可重整相互作用本身的发散辐射修正.\cite{9} 最麻烦的情况是二次(或更高次)发散, %
如果在某个很高的能量标度$\,M_{X}\,$处截断, 那么为了保证超对称性在能量低于$\,M_{X}\,$是一个好的近似对称性, %
这种发散会要求对裸耦合常数的精细调节. 根据方程(\ref{27.7.4}), 在所有$\,d_{i}=4\,$的相互作用都是超对称的理论中, %
仅当辐射修正包含一个量纲$\,d_{1}\geq 2+d\,$的超可重整超对称性破缺的相互作用插入时, 它们才能产生量纲为$\,d\,$的二次发散或高次发散(\,$p\geq 2$\,)的超对称性破缺算符. 这使得要么$\,d=0\,$且$\,d_{1}\geq 2$, 要么$\,d=1\,$且$\,d_{1}=3$, 前者仅在我们计算宇宙常数时出现, 而后者仅在我们计算一个标量线消失于真空中的蝌蚪图时才会出现. 对所有已知的理论, %
宇宙常数都会产生精细调节的问题,\cite{10} 我们这里不会进一步考虑. 蝌蚪图代表$\,\phi\,$或$\,\phi^{\ast}\,$线性的算符, %
而我们已经看到, 对于可以写成(\ref{27.7.2})的超对称破缺相互作用, 直到它的第一阶都无法产生蝌蚪图. %
因此这种超可重整相互作用是``软的'', 也就是说它们无法引入二次或高次发散. 连同$\,d\leq 2\,$的超可重整相互作用, %
其中包括$\,\phi\,$和$\,\phi^{\ast}\,$的任意二次多项式, 超对称破缺相互作用在如下的意义是软的: %
包含可以表示成$\,\phi^{3}=[\Omega\Phi^{3}]_{\mathscr{F}}\,$的$\,\phi\,$的三阶项, 以及类似的$\,\phi^{\ast}\,$的三阶项, %
还有可以写成$\,[\Omega\epsilon_{\alpha\beta}W_{\alpha}W_{\beta}]_{\mathscr{F}}\,$的$\,d=3\,$的规范微子质量项, %
但不包含$\,\phi^{2}\phi^{\ast}\,$或$\,\phi\phi^{2\ast}\,$这样的项, 这样的项一般{\kai{会}}产生二次发散的蝌蚪图.\cite{9}

然而, 蝌蚪图只能伴随对所有精确对称性都是中性的标量场产生. 在没有这种标量场的理论中, 例如下一章要讨论超对称标准模型, %
{\kai{所有}}超可重整相互作用都可以被视为是软的.


\section{另一种方法: 规范不变的超对称变换} \label{sec:27.8}

迄今为止讨论的超对称变换规则包含普通的时空导数但不包含规范不变导数让人有些不安. 例如, 在一个$\,U(1)\,$规范理论中, %
手征标量超场分量场的变换规则由方程(\ref{26.3.15})---(\ref{26.3.17})给出
\begin{align}
    &\delta \psi_{L} =\sqrt{2}\partial_{\mu}\phi\,\gamma^{\mu}\,\alpha_{R}\,\phi
    +\sqrt{2}\mathscr{F}\,\alpha_{L}\:,\nonumber \\
    &\delta \mathscr{F}=\sqrt{2}\Bigl(\overline{\alpha_{L}}\,\slashed{\partial}\psi_{L}\Bigr)\:,\label{27.8.1} \\
    &\delta \phi=\sqrt{2}\Bigl(\overline{\alpha_{R}}\psi_{L}\Bigr) \:. \nonumber
\end{align}
有的人或许会觉得, 在携带$\,U(1)\,$荷$\,q\,$的手征超场的变换中, 方程(\ref{27.8.1})中的普通时空导数应该被换成规范协变导数, %
以$\,U(1)\,$规范场$\,V_{\mu}\,$表示就是
\begin{equation}
    D_{\mu}=\partial_{\mu}-\mi q\,V_{\mu} \:. \label{27.8.2}
\end{equation}
当手征超场的变换是这种规范不变的超对称变换时, 对于只包含物理和辅助场$\,V_{\mu},$ $\lambda\,$和$\,D\,$的规范超多重态, %
我们仍旧尝试将它的超对称变换规则写成:
\begin{align}
    &\tilde{\delta} V_{\mu} = \Bigl(\bar{\alpha}\gamma_{\mu}\lambda\Bigr) \:, \nonumber \\
    &\tilde{\delta} \lambda=\mi D\gamma_{5}\alpha
    +\frac{1}{2}\,\Bigl[\partial_{\mu}\slashed{V},\gamma^{\mu}\Bigr]\alpha\:, \label{27.8.3} \\
    &\tilde{\delta} D =\mi\Bigl(\bar{\alpha}\gamma_{5}\,\slashed{\partial}\lambda\Bigr) \:, \nonumber
\end{align}
其中出现的是普通导数是因为规范超场不携带$\,U(1)\,$荷.

这并不行得通. 这些变换的代数不封闭: 两个修正超对称变换的对易子不是玻色对称变换, 例如时空平移和规范变换, 的线性组合. %
由此得出, 为手征超场和规范超场构造在这些修正超对称变换下不变的拉格朗日量是不可能的, 这是因为, 如果存在这样的拉格朗日量, %
那么它也必须在这些变换的对易子在不变, 这使得这些对易子必须是拉格朗日量的玻色对称性.

在\,1973\,年, de Wit\,和\,Freedman\cite{11}证明了通过修正手征超场的超对称变换性质可以使得超对称代数封闭, %
方法是不仅把普通导数换成规范协变导数, 同时在$\,\mathscr{F}\,$-分量的变换中加入额外一项, 使得对于$\,U(1)\,$规范理论, %
修正超对称变换规则是
\begin{align}
    &\tilde{\delta} \psi_{L} =\sqrt{2}D_{\mu}\phi\,\gamma^{\mu}\,\alpha_{R}\,\phi
    +\sqrt{2}\mathscr{F}\,\alpha_{L}\:,\nonumber \\
    &\tilde{\delta}\mathscr{F}=\sqrt{2}\Bigl(\overline{\alpha_{L}}\,\slashed{D}\psi_{L}\Bigr)
    -2\mi\,q\,\phi\Bigl(\overline{\alpha_{L}}\lambda_{R}\Bigr)\:,\label{27.8.4} \\
    &\tilde{\delta} \phi=\sqrt{2}\Bigl(\overline{\alpha_{R}}\psi_{L}\Bigr) \:. \nonumber
\end{align}
在做了这个改变后, 他们还能够构造出在变换(\ref{27.8.3})---(\ref{27.8.4})下不变的拉格朗日量, %
而这个拉格朗日量正是我们在\,\ref{sec:27.1}\,节和\,\ref{sec:27.2}\,节发现的.

继续使用传统的变换规则(\ref{27.8.1})并没有什么错, 所以我们并不需要用\,de Wit--Freedman\,形式体系处理超对称规范理论. 然而, 这个体系本身是有益的, 因为传统体系在超引力理论中的相应版本非常繁琐. 就像在第\,\ref{cha:31}\,章描述的, 在推导超引力理论中物理上感兴趣的结果时主要使用的形式体系跟随的是\,de Wit\,和\,Freedman\,那样的方法, 即超对称变换规则包含的是协变导数而不是普通导数, 而不是基于类似(\ref{27.8.1})这样的传统超对称变换. 因此在相对简单的$\,U(1)\,$规范理论的框架下理解\,de Wit--Freedman\,形式体系和传统方法之间的关系是有益的, 特别是解释$\,\mathscr{F}\,$的变换规则中额外那一项的起源.

在写下不包含规范超场$\,V\,$的分量$\,C$, $M$, $N$\,或$\,\omega\,$的超对称变换(\ref{27.8.3})时, de Wit\,和\,Freedman\,\\
隐含地采用了\,\ref{sec:27.1}\,节讨论的\,Wess--Zumino\,规范. %
但是\,Wess--Zumino\,规范的选择既不在传统超对称变换(\ref{26.2.11})---(\ref{26.2.17})下不变, %
也不在扩充规范变换(\ref{27.1.17})下不变, 所以一旦我们采用了这个规范, 两个对称性都失去了. 然而, %
我们可以定义一个{\kai{组合}}变换, 它作用在\,Wess-Zumino\,规范中的场上, 由一个传统的超对称变换加上一个规范变换, 使得我们再次回到\,Wess-Zumino\,规范. {\kai{这就是\,}}{\textit{de Wit--Freedman}}\,{\kai{变换}}$\,\tilde{\delta}$.\footnote{de Wit\,%
和\,Freedman\,并没有明确地表明这一点. 然而, 实际上, 尽管这篇文章的重点是强调变换(\ref{27.8.3})和(\ref{27.8.4})的细节可以从要求超对称代数封闭推导出来(对非阿贝尔规范理论同样如此), 但他们指出他们实际上通过找出在\,Wess--Zumino\,规范下幸存的费米变换就发现了这些变换.}

以这种方法构造\,de Wit--Freedman\,变换时, 注意到对于一个满足\,Wess-Zumino\,规范条件$\,C=M=N=\omega=0\,$的规范超场, %
变换规则(\ref{26.2.11})---(\ref{26.2.14})给出
\begin{equation}
    \delta C=0\:,\qquad \delta \omega=\slashed{V}\,\alpha\:,\qquad
    \delta M=-\Bigl(\bar{\alpha}\,\lambda\Bigr)\:, \qquad \delta N=\mi\Bigl(\bar{\alpha}\gamma_{5}\,\lambda\Bigr)\:.
    \label{27.8.5}
\end{equation}
根据方程(\ref{27.1.17}), 通过进行一个无限小扩充规范变换(\ref{27.1.13}):
\begin{equation}
    V\to V +\frac{\mi}{2}\Bigl[\Omega-\Omega^{\ast}\Bigr] \:, \label{27.8.6}
\end{equation}
其中$\,\Omega\,$是分量为
\begin{equation}
    \phi^{\Omega}=0\:,\qquad \psi_{L}^{\Omega}=-\sqrt{2}\,\slashed{V}\,\alpha_{R}\:,\qquad
    \mathscr{F}^{\Omega}=-\Bigl(\bar{\alpha}(1-\gamma_{5})\lambda\Bigr)  \label{27.8.7}
\end{equation}
的左手征超场, 我们能够回到\,Wess--Zumino\,规范. 根据方程(\ref{27.1.11}), 这个扩充规范变换在荷为$\,q$ 的手征超场上诱导出了变换
\begin{equation}
    \delta^{\prime}\Phi =\mi\,q\,\Omega\,\Phi\:. \label{27.8.8}
\end{equation}
利用乘法规则(\ref{26.3.27})---(\ref{26.3.29}), $\Phi\,$的分量的变换是
\begin{align}
    &\delta^{\prime}\psi_{L}=-\mi\sqrt{2}q\phi\,\slashed{V}\,\alpha_{R}\:,\nonumber \\
    &\delta^{\prime}\mathscr{F}=-2\mi q\phi\Bigl(\overline{\alpha_{L}}\lambda_{R}\Bigr)
    -\mi\sqrt{2}q\Bigl(\overline{\alpha_{L}}\,\slashed{V}\,\psi_{L}\Bigr) \:, \label{27.8.9}\\
    &\delta^{\prime}\phi=0 \:.\nonumber
\end{align}
将这个加在方程(\ref{27.8.1})上并与方程(\ref{27.8.4})比较表明\,de Wit--Freedman\,变换确实是一个传统超对称变换与相应扩充规范变换(\ref{27.8.8})的组合:
\begin{equation}
    \tilde{\delta} \Phi=\delta\Phi +\delta^{\prime}\Phi \:. \label{27.8.10}
\end{equation}

\section[带有扩充超对称性的规范理论]{带有扩充超对称性的规范理论\footnote{本节有些脱离本书的发展主线, 可以在第一次阅读时跳过.}} \label{sec:27.9}


因为\,\ref{sec:25.4}\,节讨论过的粒子多重态的非手征性, 带有未破缺扩充超对称性的理论被认为不是标准模型的真实扩张的好候选者.
然而, 由于带有扩充超对称性的规范理论为使用强有力的数学工具解决动力学问题提供了范例, 它们在这里值得考虑一下.

为构造有$\,N=2\,$扩充超对称性的拉格朗日量已经提出了数个特殊的形式体系,\cite{12} 但幸运的是, 我们可以用已有的工具获得它. %
任何带有$\,N=2\,$超对称性的理论同时也有$\,N=1\,$超对称性, 所以它的拉格朗日量必然是本章已经考虑过的拉格朗日量的一个特殊情况. 为了给\,\ref{sec:25.4}\,节和\,\ref{sec:25.5}\,节构造的某组$\,N=2\,$粒子超多重态构造一个有$\,N=2\,$超对称性的拉格朗日量, %
我们只需写下带有$\,N=1\,$超对称性的最一般拉格朗日量, 要求它的\,$N=1\,$超多重态包含$\,N=2\,$超多重态中粒子的场, %
然后给这个拉格朗日量附加一个离散的$\,R\,$-对称性: 在$\,N=2\,$超多重态的不同分量上进行不同作用的对称性. 这样拉格朗日量在第二个超对称性也是不变的, 它的超对重态是通过用$\,R\,$-对称性作用在普通$\,N=1\,$的超多重态上给出的.

选择离散$\,R\,$-对称性使得
\begin{equation}
    Q_{1}\to Q_{2}\:, \qquad Q_{2}\to -Q_{1}  \label{27.9.1}
\end{equation}
将是方便的. 如果中心荷是零, 那么超对称代数在一个$\,SU(2)\,R\,$-对称群将是不变的,  这个对称群把变换(\ref{27.9.1})当做一个有限元$\,\exp(\mi\uppi/\tau_{2})$, 但就我们的目的而言, 离散对称性是足够的, 所以我们无需假定中性荷为零. 事实上, %
我们用这个方法构造的拉格朗日量最后将会有一个\,$SU(2)\,$对称性, 而不只是在离散变换(\ref{27.9.1})下的对称性.

对于一般规范群的规范玻色子, 连同$\,N=2\,$扩充对称性要求的超对称伙伴, 我们先来考虑它们的可重整理论. %
我们在\,\ref{sec:25.4}\,节看到, 在有$\,N=2\,$整体超对称性的理论中, 一个无质量玻色子所属的多重态必须同时含有%
螺旋度$\,\pm1/2\,$的无质量费米子各一对以及一对无自旋玻色子, 前者在$\,SU(2)\,R\,$-对称性按照双重态变换而后者则是$\,SU(2)\,$单态. 既然$\,N=2\,$对称性包含$\,N=1\,$对称性, 这个理论的可重整拉格朗日量必然是一般可重整拉格朗日密度(\ref{27.4.1})的特殊情况. 这个特殊情况的一个特征是, 由于规范玻色子属于规范群的伴随表示, 所以费米场和标量也必须属于这个伴随表示. 为了构建含有正确粒子的场的$\,N=2\,$的超多重态, 对每个$\,N=1\,$规范多重态$\,V_{A}^{\mu}$, $\lambda_{A}$, $D_{A}$, 我们必须有一个$\,N=1\,$手征超场$\,\Phi_{A}$, 它的分量场是$\,\phi_{A}$, $\psi_{A}$, %
$\mathscr{F}_{A}$(其中$\,\psi_{A}\,$是\,Majurana\,费米子而$\,\phi_{A}\,$和$\,\mathscr{F}_{A}\,$均是复的). 我们在变换
\begin{equation}
    \psi_{A}\to \lambda_{A}\:, \qquad \lambda_{A}\to -\psi_{A}  \label{27.9.2}
\end{equation}
(所有其它场不变)下附加一个离散$\,R\,$-对称性, 这是因为这是变换(\ref{27.9.1})的效应. 由于超势给出的$\,\psi_{A}\,$的相互作用或质量项是没有$\,\lambda_{A}\,$的, 超势必须为零. 拉格朗日量(\ref{27.4.1})因此取如下的特殊形式
\begin{align}
    \mathscr{L} &= -\sum_{A}(D_{\mu}\phi)^{\ast}_{A}\,(D^{\mu}\phi)_{A}
    -\frac{1}{2}\sum_{A}\Bigl(\overline{\psi_{A}}\,(\slashed{D}\psi)_{A}\Bigr)
    +\sum_{A}\mathscr{F}_{A}^{\ast}\mathscr{F}_{A} \nonumber \\
    &\quad -2\sqrt{2}\operatorname{Re}\sum_{ABC}C_{ABC}\,\Bigl(\lambda_{AL}^{\mathrm{T}}\,\psi_{CL}\Bigr)
    \phi_{B}^{\ast} \nonumber \\
    &\quad +\mi\sum_{ABC}C_{ABC}\,\phi_{B}^{\ast}\phi_{C}D_{A}-\sum_{A}\xi_{A}\,D_{A}
    +\frac{1}{2}\sum_{A}D_{A}D_{A} \nonumber \\
    &\quad -\frac{1}{4}\sum_{A}f_{A\mu\nu}f_{A}^{\mu\nu}
    -\frac{1}{2}\sum_{A}\Bigl(\overline{\lambda_{A}}\,(\slashed{D}\lambda)_{A}\Bigr)
    +\frac{g^{2}\theta}{64\uppi^{2}}\epsilon_{\mu\nu\rho\sigma}\sum_{A}f_{A}^{\mu\nu}f_{A}^{\rho\sigma}\:,
    \label{27.9.3}
\end{align}
其中
\begin{align}
    (D_{\mu}\psi)_{A} &= \partial_{\mu}\psi_{A} +\sum_{BC}C_{ABC}V_{B\mu}\psi_{C} \:,\label{27.9.4} \\
    (D_{\mu}\lambda)_{A} &= \partial_{\mu}\lambda_{A}+\sum_{BC}C_{ABC}V_{B\mu}\lambda_{C}\:, \label{27.9.5}\\
    (D_{\mu}\phi)_{A} &=\partial_{\mu}\phi_{A}+\sum_{BC}C_{ABC}V_{B\mu}\phi_{C}\:, \label{27.9.6}
\end{align}
以及
\begin{equation}
    f_{A\mu\nu} =\partial_{\mu}V_{A\nu}-\partial_{\nu}V_{A\mu} +\sum_{BC}C_{ABC}V_{B\mu}V_{C\nu} \:. \label{27.9.7}
\end{equation}
(回忆在伴随表示下$\,(t_{A})_{BC}=-\mi C_{ABC}$, 其中$\,C_{ABC}\,$是实结构常数, 同往常一样在本书中定义成包含规范耦合因子, 并取在使得它自身全反对称的基上.) 因为拉格朗日密度(\ref{27.9.3})是方程(\ref{27.4.1}) 的一个特殊情况, %
因此它有多重态为$\,\phi_{A},$ $\psi_{A}$, $\mathscr{F}_{A}\,$和$\,V_{A}^{\mu}$, $\lambda_{A}$, $D_{A}\,$的$\,N=1\,$超对称性, 并且它还有一个旋转$\,\psi_{A}\,$和$\,\lambda_{A}\,$的$\,SU(2)\,$对称性, 其中包含在有限$\,SU(2)\,$变换(\ref{27.9.2})下的不变性, 所以它还有{\kai{第二个}}独立的$\,N=1\,$超对称性, 多重态为$\,\phi_{A}$, $\lambda_{A}$, $\mathscr{F}_{A}$%
和$\,V_{A}^{\mu}$, $-\psi_{A}$, $D_{A}$. 因此它满足$\,N=2\,$超对称性附加的条件.

我们可以通过让辅助场等于使得拉格朗日密度(\ref{27.9.3})稳定的值来消除它们:
\begin{equation}
    \mathscr{F}_{A}=0 \:, \qquad D_{A}=-\mi\sum_{BC}C_{ABC}\phi_{B}^{\ast}\phi_{C}\:. \label{27.9.8}
\end{equation}
(我们现在假定\,Fayet--Iliopoulos\,常数$\,\xi_{A}\,$全为零.) 将这些值代会方程(\ref{27.9.3})将给出一个等价的拉格朗日密度
\begin{align}
    \mathscr{L} &= -\sum_{A}(D_{\mu}\phi)^{\ast}_{A}\,(D^{\mu}\phi)_{A}
    -\frac{1}{2}\sum_{A}\Bigl(\overline{\psi_{A}}\,(\slashed{D}\psi)_{A}\Bigr)  \nonumber \\
    &\quad +\sqrt{2} \sum_{ABC}C_{ABC}\,\biggl(\overline{\psi_{B}}\,
    \biggl(\frac{1-\gamma_{5}}{2}\biggr)\,\lambda_{A}\biggr)\phi_{C} \nonumber \\
     &\quad -\sqrt{2} \sum_{ABC}C_{ABC}\,\biggl(\overline{\lambda_{A}}\,
    \biggl(\frac{1+\gamma_{5}}{2}\biggr)\,\psi_{C}\biggr)\phi^{\ast}_{B} -V(\phi,\phi^{\ast}) \nonumber \\
    &\quad -\frac{1}{4}\sum_{A}f_{A\mu\nu}f_{A}^{\mu\nu}
    -\frac{1}{2}\sum_{A}\Bigl(\overline{\lambda_{A}}\,(\slashed{D}\lambda)_{A}\Bigr)
    +\frac{g^{2}\theta}{64\uppi^{2}}\epsilon_{\mu\nu\rho\sigma}\sum_{A}f_{A}^{\mu\nu}f_{A}^{\rho\sigma}\:,
    \label{27.9.9}
\end{align}
其中势是
\begin{equation}
    V(\phi,\phi^{\ast})= -\frac{1}{2}\sum_{A}\Biggl[\sum_{BC}C_{ABC}\,\phi_{B}^{\ast}\phi_{C}\Biggr]^{2}
    =2\sum_{A}\Biggl[\sum_{BC}C_{ABC}\,\operatorname{Re}\phi_{B}\operatorname{Im}\phi_{C}\Biggr]^{2}\:.\label{27.9.10}
\end{equation}
这个势有一个等于零的最小值, 这个最小值不仅能通过对任何一组$\,\phi\,$令$\,\phi_{A}=0\,$得到, %
也可以通过对所有$\,A\,$令$\,\sum_{BC}C_{ABC}\,\phi_{B}^{\ast}\phi_{C}=0\,$得到, 或者换一种说法, 令
\begin{equation}
    [t\cdot \operatorname{Re}\phi\,,t\cdot \operatorname{Im}\phi]=0 \:, \qquad
    \text{其中}\quad t\cdot v\equiv\sum_{B}t_{B}v_{B} \:. \label{27.9.11}
\end{equation}
即, 当这些标量场的值使得所有生成元$\,t\cdot\operatorname{Re}\phi\,$和$\,t\cdot\operatorname{Im}\phi\,$属于整个规范代数的一个\,Cartan\,子代数时, 即所有生成元彼此都对易的子代数, 势取它的最小值. 尽管所有这样的$\,\phi\,$值都给出零势, 因而也就给出了不破缺的$\,N=2\,$对称性, 它们在物理上是不等价的, 例如对于破缺规范对称性附带的规范玻色子, 它们会赋予这些玻色子不同的质量.

扩充超对称性的显著特征之一是, 超对称代数在任何态中的中心荷可以用那个态中与玻色场耦合的``荷''计算出来.\cite{13} %
进行这个计算的最简单方法是使用扩充超对称流$\,S_{r}^{\mu}(x)\,$在普通$\,N=1\,$超对称性下的变换性质计算反对易子%
$\,\{Q_{1\alpha},S_{r\beta}^{\mu}(x)\}$, 其中$\,r=2,\,3,\,\cdots,N$. 这样我们就能从这个反对易子中计算出中心荷
\begin{equation}
    \{Q_{1\alpha}\,,Q_{r\beta}\} =\int\dif^{3}x\:\{Q_{1\alpha}\,,S_{r\beta}^{0}(x)\} \:. \label{27.9.12}
\end{equation}
右边的被积函数最后表现为一个对时空坐标的导数, 但如果态中有在$\,\mathbf{x}\to\infty\,$时不快速为零的场, 那么这个积分不会为零.

为了细致地看到这是如何运作的, 我们来考虑有一个$\,SU(2)\,$规范对称性和一个$\,N=2\,$规范超多重态但没有额外物质超多重态的%
$\,N=2\,$超对称性. 这里的拉格朗日量由方程(\ref{27.9.3})给出, 其中$\,A$, $B\,$和$\,C\,$在$\,1,\,2,\,3$中取值, 并且
\begin{equation}
    C_{ABC}=e\,\epsilon_{ABC}\:, \xi_{A}=0 \:. \label{27.9.13}
\end{equation}
(这里的耦合常数被记做$\,e\,$是因为它是与未破缺$\,U(1)\,$规范对称性的无质量规范场相互作用的荷.) %
通常的$\,N=1\,$超对称流(现在用一个下标\,1\,进行区分)由方程(\ref{27.4.40})给出
\begin{align}
    S_{1}^{\mu}&=-\frac{1}{4}\sum_{A}f_{A\rho\sigma}\,[\gamma^{\rho}\,,\gamma^{\sigma}]\gamma^{\mu}\lambda_{A}
    -e\sum_{ABC}\epsilon_{ABC}\,\gamma_{5}\gamma^{\mu}\,\lambda_{A}\,\phi_{B}^{\ast}\,\phi_{C} \nonumber \\
    &\quad +\frac{1}{\sqrt{2}}\sum_{A}\Biggl[(\slashed{D}\phi)_{A}\,\gamma^{\mu}\psi_{AR}+
    (\slashed{D}\phi^{\ast})_{A}\,\gamma^{\mu}\psi_{AL}\Biggr] \:. \label{27.9.14}
\end{align}
我们可以通过上面使用的有限$\,SU(2)\,R\,$-对称性作用$\,S_{1}^{\mu}\,$来计算第二个超对称流, %
简单地归结为做替换$\,\psi_{A}\to\lambda_{A},\,\lambda_{A}\to-\psi_{A}$. 这给出
\begin{align}
    S_{2}^{\mu}&=\frac{1}{4}\sum_{A}f_{A\rho\sigma}\,[\gamma^{\rho}\,,\gamma^{\sigma}]\gamma^{\mu}\psi_{A}
    +e\sum_{ABC}\epsilon_{ABC}\,\gamma_{5}\gamma^{\mu}\,\psi_{A}\,\phi_{B}^{\ast}\,\phi_{C} \nonumber \\
    &\quad +\frac{1}{\sqrt{2}}\sum_{A}\Biggl[(\slashed{D}\phi)_{A}\,\gamma^{\mu}\lambda_{AR}+
    (\slashed{D}\phi^{\ast})_{A}\,\gamma^{\mu}\lambda_{AL}\Biggr] \:. \label{27.9.15}
\end{align}
对我们的目的而言(并且也简单一些), 仅计算这个流的右手部分在一个$\,N=1\,$超对称变换下的变化将是足够的. %
令辅助场等于它们的平衡值
\[
\mathscr{F}_{A}=0 \:, \qquad D_{A}=-\mi e\sum_{BC}\epsilon_{ABC}\phi_{B}^{\ast}\phi_{C} \:,
\]
我们发现
\begin{align*}
    \delta S_{2R}^{\mu} &= \frac{\sqrt{2}}{4}\sum_{A}f_{A\rho\sigma}
    [\gamma^{\rho}\,,\gamma^{\sigma}]\gamma^{\mu}(\slashed{D}\phi)_{A}\alpha_{R} \\
    &\quad -\sqrt{2}\,e\sum_{ABC}\epsilon_{ABC}\gamma^{\mu}(\slashed{D}\phi)_{A}\alpha_{R}\phi_{B}^{\ast}\phi_{C}
    -\frac{\sqrt{2}}{4}f_{A\rho\sigma}(\slashed{D}\phi)_{A}\gamma^{\mu}[\gamma^{\rho}\,,\gamma^{\sigma}]\alpha_{R} \\
    &\quad -\sqrt{2}\,e\sum_{ABC}\epsilon_{ABC}\phi_{B}^{\ast}\phi_{C}(\slashed{D}\phi)_{A}\gamma^{\mu}\alpha_{R}
    +\cdots \:,
\end{align*}
其中省略号代表费米场的双线性项, 因为我们这里感兴趣的是长程玻色场的效应, 所以并不关心它们. 通过使用\,Dirac\,反对易关系和恒等式
\[
[\gamma^{\rho}\,,\gamma^{\sigma}]\,[\gamma^{\mu}\,,\gamma^{\nu}]
+[\gamma^{\mu}\,,\gamma^{\nu}]\,[\gamma^{\rho}\,,\gamma^{\sigma}]=-8\eta^{\mu\rho}\eta^{\nu\sigma}
+8\eta^{\sigma\mu}\eta^{\rho\nu}+8\mi\,\epsilon^{\mu\nu\rho\sigma}\gamma_{5}
\]
我们可以组合这些项并发现
\begin{align*}
    \delta S_{2R}^{\mu} &= -2\sqrt{2}\sum_{A}f_{A}^{\mu\nu}(D_{\nu}\phi)_{A}\alpha_{R}
    -\mi\sqrt{2}\sum_{A}\epsilon^{\mu\nu\rho\sigma}f_{A\rho\sigma}(D_{\nu}\phi)_{A}\alpha_{R} \\
    &\quad -2\sqrt{2}\,e\sum_{ABC}\epsilon_{ABC}\phi_{B}^{\ast}\phi_{C}(D^{\mu}\phi)_{A}\alpha_{R}+\cdots \:.
\end{align*}
为了将这个写成一个导数, 我们需要方程(\textcolor{foo}{15.3.6}), (\textcolor{foo}{15.3.7})和(\textcolor{foo}{15.3.9})给出的%
\,Yang--Mills\,场方程:
\begin{align*}
    &D_{\nu}f_{A}^{\mu\nu} = J_{A}^{\mu} =
    e\sum_{BC}\epsilon_{ABC}\Bigl((D_{\mu}\phi)_{B}^{\ast}\phi_{C}-\phi_{B}^{\ast}(D^{\mu}\phi)_{C}\Bigr) \:,\\
    &\epsilon_{\mu\nu\rho\sigma}(D^{\nu}f^{\rho\sigma})_{A} =0 \:.
\end{align*}
它们使得我们可以将$\,\delta S_{2R}^{\mu}\,$写成一个全导数
\begin{equation}
    \delta S_{2R}^{\mu} = D_{\nu}X^{\mu\nu}\,\alpha_{R} \:, \label{27.9.16}
\end{equation}
其中
\begin{equation}
    X^{\mu\nu}=-2\sqrt{2}\sum_{A}f_{A}^{\mu\nu}\phi_{A}
    -\mi\sqrt{2}\sum_{A}\epsilon^{\mu\nu\rho\sigma}f_{A\rho\sigma}\phi_{A}+\cdots \:, \label{27.9.17}
\end{equation}
省略号依旧代表包含费米场的无关项. 方程(\ref{26.1.18})使得我们可以将方程(\ref{27.9.16})写成一个反对易关系
\begin{equation}
    \Bigl\{Q_{R\alpha}\,,S_{R\beta}^{\mu}\Bigr\}
    =\mi\,\biggl[\epsilon\biggl(\frac{1-\gamma_{5}}{2}\biggr)\biggr]_{\alpha\beta}\,
    D_{\nu}X^{\mu\nu} \:. \label{27.9.18}
\end{equation}
由于$\,X^{\mu\nu}\,$是规范不变量, 它的规范协变导数与它的普通导数相同. 另外, $X^{\mu\nu}\,$是反对称的, %
所以$\,D_{\nu}X^{0\nu}=\partial_{i}X^{0i}$. 从方程(\ref{27.9.12})和(\ref{27.9.16}), 我们最后有
\begin{equation}
    \Bigl\{Q_{R\alpha}\,,Q_{R\beta}\Bigr\}
    =\mi\,\biggl[\epsilon\biggl(\frac{1-\gamma_{5}}{2}\biggr)\biggr]_{\alpha\beta}\,
    \int\dif S_{i}\: X^{0i} \:, \label{27.9.19}
\end{equation}
积分取在一个包裹所考察系统的大闭曲面, 而面积微分$\,\dif\mathbf{S}\,$取成曲面的法向. 与方程(\ref{25.2.38})比较给出了中心荷
\begin{equation}
    Z_{12}=-\mi\int \dif S_{i}\:X^{0i} \:. \label{27.9.20}
\end{equation}
如果我们选择的规范中$\,\phi_{A}\,$(几乎处处)只有不为零的常分量$\,\phi_{3}\equiv v$, 那么
\begin{equation}
    \sum_{A}f_{A}^{0i}\phi_{A} = -vE^{i} \:, \qquad
    \frac{1}{2}\sum_{A}\epsilon^{0i\rho\sigma}f_{A\rho\sigma}\phi_{A} =vB^{i} \:, \label{27.9.21}
\end{equation}
其中$\,\mathbf{E}\,$和$\,\mathbf{B}\,$是与$\,SU(2)\,$规范群的未破缺$\,U(1)\,$子群相联系的电场和磁场. %
因此中心荷(\ref{27.9.20})在这里是
\begin{equation}
    Z_{12}=2\sqrt{2}\,v\,\Bigl[\mi q-\mathscr{M}\Bigr] \:, \label{27.9.22}
\end{equation}
其中$\,q\,$和$\,\mathscr{M}\,$是电荷和磁单极距, 定义成
\begin{equation}
    q=\int\dif S_{i}\:E^{i} \:, \qquad \mathscr{M}=\int \dif S_{i}\:B^{i} \:. \label{27.9.23}
\end{equation}
就像在\,23.3\,节讨论过的, 这个理论, 其中$\,SU(2)\,$规范对称性被一个$\,SU(2)\,$三重态标量的期望值自发破缺, 确实有磁单极子.

将\,\ref{sec:27.4}\,节的结果应用到拉格朗日密度(\ref{27.9.3})上表明, 在\,$SU(2)$\,规范对称性自发破缺之后, 这个理论将会包含电荷为$\,\pm e$, 磁单极距为零, 树级近似质量$\,M=\sqrt{2}\lvert e\,v\rvert\,$的基本粒子. 特别地, 对电荷的每一个符号, 有一个自旋\,1\,的粒子, 两个自旋\,1/2\,的, 以及一个自旋\,0\,的. 这里获得结果的一个显著后果是, {\kai{假定$\,v\,$这个量为中心荷由方程(\ref{27.9.22})定义的, 那么质量值$\,\sqrt{2}\lvert e\,v\rvert\,$是精确的, 不受辐射修正或非微扰效应的影响.}}\cite{13}

为了看到这点, 注意到对电荷的每个符号, 有质量单粒子态是``短''$\,N=2\,$超多重态, 而正如\,\ref{sec:25.5}\,节末尾证明过的, 它们的质量抵达下界(\ref{25.5.24}):
\begin{equation}
    M=\lvert Z_{12}\rvert/2 \:. \label{27.9.24}
\end{equation}
即使我们不相信树级近似给出粒子质量的精确值, 但我们也无法期待对这个近似的修正会把这个短多重态变成有更多态且质量更大的完整多重态, 所以我们可以确信方程(\ref{27.9.24})是精确成立的. 对于电荷$\,q=\pm e\,$且磁单极距为零的粒子, %
方程(\ref{27.9.20})给出$\,Z_{12}=\pm2\sqrt{2}\mi\,v\,e$, 所以方程(\ref{27.9.24}) 告诉我们它们的质量是
\begin{equation}
    M=\sqrt{2}\lvert e\,v\rvert \:. \label{27.9.25}
\end{equation}
这是在树级近似下发现的结果, 但现在我们看到它是精确的.

23.3\,节描述的半经典计算表明这个理论中的电中性磁单极子有磁单极强度\footnote{注意方程(\ref{27.9.23})%
定义的磁矩$\,\mathscr{M}\,$与\,23.3\,节定义磁矩$\,g\,$的关系是$\,\mathscr{M}=4\uppi g$.}
\begin{equation}
    \mathscr{M}=\frac{4\uppi \nu}{e} \:, \label{27.9.26}
\end{equation}
其中$\,\nu\,$是缠绕数, 一个正整数或负整数. 因此中心荷的公式(\ref{27.9.22})加上不等式(\ref{25.2.24})给出了单极子质量上的下界
\begin{equation}
    M\geq \frac{4\uppi \sqrt{2}\lvert\nu\,v\rvert}{\lvert e \rvert}\:. \label{27.9.27}
\end{equation}
有趣的是, 这与\,23.3\,节推导过的单极子能量上的\,Bogomol'nyi\,下界\cite{13}相同.\footnote{有非零真空期望值的正则归一化场(对于实的$\,v\,$)是$\,\operatorname{Re}\phi_{3}$, 所以\,Bogomol'nyi\,不等式(\textcolor{foo}{23.3.19})中出现的量%
$\,\langle\phi\rangle\,$是$\,\sqrt{2}v$.} 事实上, 23.3\,节描述过的$\,\nu=1\,$的单极子解处在这个下界上. 更一般地, %
这个理论的``双荷子'',\cite{15} 即既有电荷又有磁矩的粒子, 有质量\cite{16}
\begin{equation}
    M=2\,\lvert v\rvert \sqrt{q^{2}+\mathscr{M}^{2}} \:, \label{27.9.28}
\end{equation}
这依旧是方程(\ref{25.5.24})和(\ref{27.9.20})所允许的最小值. 诚然, 这个理论中{\kai{所有}}已知的粒子均有在半经典极限下由方程(\ref{27.9.28})给出的质量.\cite{17}

现在回到一般的$\,N=2\,$规范理论, 我们也可以在拉格朗日量中引入额外的``物质''场. 简单起见, 我们限制在有质量的``短''多重态(中心荷$\,\mathscr{Z}\,$使不等式(\ref{25.5.24})的等号成立)上, 每个多重态由一个自旋\,1/2\,的费米子和自旋\,0\,粒子的一个%
$\,SU(2)\,$双态再加上可区分的反粒子构成. 这个自旋组成与$\,N=1\,$超对称性下成对左手征标量超场$\,\Phi_{n}^{\prime}\,$%
和$\,\Phi_{n}^{\prime\prime}\,$加上其右手征共轭的自旋组成相同, 后者中的复标量场分量$\,\phi_{n}^{\prime}\,$和%
$\,\phi_{n}^{\prime\prime}\,$与它们的共轭形成了成对的$\,SU(2)\,$双态, 而旋量场都是$\,SU(2)\,$单态. (我们使用撇号和双撇号来区分这些超场和$\,\Phi_{A}\,$以及它们的分量和$\,\Phi_{A}\,$的分量.) 如果其中一些极多重态$\,\Phi_{n}^{\prime}\,$和%
$\,\Phi_{n}^{\prime\prime}\,$在规范群下是中性的, 那么就允许有如下形式的超势:\footnote{由于和之前相同的原因, %
超势中仍然不能有$\,\Phi_{A}\,$的二阶项和高阶项: 这种项将会导致$\,\psi_{A}\,$有标量耦合或质量, %
而它们的$\,SU(2)\,$伙伴$\,\lambda_{A}\,$没有相应的耦合或质量. 另外, 也不能有$\,\Phi_{n}^{\prime}\,$和(或)%
$\,\Phi_{n}^{\prime\prime}\,$的三线性项, 这是因为方程(\ref{27.4.1})中包含费米子双线性型与超势二阶导数之积那一项将会给出%
$\,SU(2)\,$单态费米子和$\,SU(2)\,$双态场$\,\phi_{n}^{\prime}\,$或$\,\phi_{n}^{\prime\prime}\,$的耦合. 因此, 超势中的三线性项必须包含一个$\,\Phi_{A}\,$因子和两个$\,\Phi_{n}^{\prime}\,$和(或)$\,\Phi_{n}^{\prime\prime}\,$因子. %
不可能有包含一个$\,\Phi_{A}\,$和两个$\,\Phi_{n}^{\prime}\,$或两个$\,\Phi_{n}^{\prime\prime}\,$的三线性项, %
这是因为这种项将会给$\,SU(2)\,$单态辅助场$\,\mathscr{F}_{A}\,$一个与$\,SU(2)\,$-三重态乘积%
$\,\phi_{n}^{\prime}\phi_{m}^{\prime}\,$或$\,\phi_{n}^{\prime\prime}\phi_{m}^{\prime\prime}\,$的相互作用, 并且也不能有任何包含两个$\,\Phi_{n}^{\prime}\,$或两个$\,\Phi_{n}^{\prime\prime}\,$的双线性项, 这是因为这将会产生$\,SU(2)\,$-三重态质量项%
$\,(\psi_{n}^{\prime\mathrm{T}}\epsilon\psi_{m}^{\prime})\,$或%
$\,(\psi_{n}^{\prime\prime\mathrm{T}}\epsilon\psi_{m}^{\prime\prime})$. %
剩下允许的双线性项和三线性项就只能是(\ref{27.9.29})的形式.}
\begin{equation}
    f(\Phi,\Phi^{\prime},\Phi^{\prime\prime})
    =\frac{1}{2}\sum_{Anm}(s_{A})_{nm}\Phi_{n}^{\prime}\Phi_{m}^{\prime\prime}\Phi_{A}
    +\frac{1}{2}\sum_{nm}\mu_{nm}\Phi_{n}^{\prime}\Phi_{m}^{\prime\prime} \:. \label{27.9.29}
\end{equation}
这样我们就必须给拉格朗日密度(\ref{27.9.3})加上这些极多重态的拉格朗日密度, 它由方程(\ref{27.4.1})右边的前八项给出, %
并得到总的拉格朗日密度
\begin{align}
    \mathscr{L} &= -\sum_{n}(D_{\mu}\phi^{\prime})_{n}^{\ast}(D^{\mu}\phi^{\prime})_{n}
    -\sum_{n}(D_{\mu}\phi^{\prime\prime})_{n}^{\ast}(D^{\mu}\phi^{\prime\prime})_{n}
    -\sum_{A}(D_{\mu}\phi)_{A}^{\ast}(D^{\mu}\phi)_{A} \nonumber \\
    &\quad -\frac{1}{2}\sum_{n}\Bigl(\overline{\psi^{\prime}_{n}}(\slashed{D}\psi^{\prime})_{n}\Bigr)
    -\frac{1}{2}\sum_{n}\Bigl(\overline{\psi^{\prime\prime}_{n}}(\slashed{D}\psi^{\prime\prime})_{n}\Bigr) \nonumber\\
    &\quad -\frac{1}{2}\sum_{A}\Bigl(\overline{\psi_{A}}(\slashed{D}\psi)_{A}\Bigr)
    -\frac{1}{2}\sum_{A}\Bigl(\overline{\lambda_{A}}(\slashed{D}\lambda)_{A}\Bigr) \nonumber \\
    &\quad +\sum_{n}{\mathscr{F}_{n}^{\prime}}^{\ast}\mathscr{F}_{n}^{\prime}
    +\sum_{n}{\mathscr{F}_{n}^{\prime\prime}}^{\ast}\mathscr{F}_{n}^{\prime\prime}
    +\sum_{n}\mathscr{F}_{A}^{\ast}\mathscr{F}_{A} \nonumber \\
    &\quad-\operatorname{Re}\sum_{Anm}(s_{A})_{nm}\phi_{A}
    \Bigl(\psi_{nL}^{\prime\,\mathrm{T}}\epsilon\psi_{mL}^{\prime\prime}\Bigr)
    -2\sqrt{2}\operatorname{Re}\sum_{ABC}C_{ABC}\,
    \Bigl(\lambda_{AL}^{\mathrm{T}}\epsilon\psi_{CL}\Bigr)\,\phi_{B}^{\ast} \nonumber \\
    &\quad-\operatorname{Re}\sum_{Anm}(s_{A})_{nm}\phi_{n}^{\prime}
    \Bigl(\psi_{mL}^{\prime\prime\,\mathrm{T}}\epsilon\psi_{AL}\Bigr)
    -\operatorname{Re}\sum_{Anm}(s_{A})_{nm}\phi_{m}^{\prime\prime}
    \Bigl(\psi_{nL}^{\prime\,\mathrm{T}}\epsilon\psi_{AL}\Bigr) \nonumber \\
    &\quad +2\sqrt{2}\operatorname{Im}\sum_{Anm}(t_{A}^{\prime})_{mn}
    \Bigl(\psi_{nL}^{\prime\,\mathrm{T}}\epsilon\lambda_{AL}\Bigr)\phi_{m}^{\prime\,\ast}
    +2\sqrt{2}\operatorname{Im}\sum_{Anm}(t_{A}^{\prime\prime})_{mn}
    \Bigl(\psi_{nL}^{\prime\prime\,\mathrm{T}}\epsilon\lambda_{AL}\Bigr)\phi_{m}^{\prime\prime\,\ast} \nonumber \\
    &\quad +\operatorname{Re}\sum_{Anm}(s_{A})_{nm}\phi_{A}\phi_{n}^{\prime}\mathscr{F}_{m}^{\prime\prime}
    +\operatorname{Re}\sum_{Anm}(s_{A})_{nm}\phi_{A}\phi_{m}^{\prime\prime}\mathscr{F}_{n}^{\prime}\nonumber \\
    &\quad +\operatorname{Re}\sum_{Anm}(s_{A})_{nm}\phi_{n}^{\prime}\phi_{m}^{\prime\prime}\mathscr{F}_{A}\nonumber \\
    &\quad +\operatorname{Re}\sum_{Anm}\mu_{nm}\phi_{n}^{\prime}\mathscr{F}_{m}^{\prime\prime}
     +\operatorname{Re}\sum_{Anm}\mu_{nm}\phi_{m}^{\prime\prime}\mathscr{F}_{n}^{\prime}
     -\operatorname{Re}\sum_{nm}\mu_{nm}\Bigl(\psi_{nL}^{\prime\,\mathrm{T}}\epsilon\psi_{mL}^{\prime\prime}\Bigr)
     \nonumber \\
     &\quad-\sum_{Anm}(t_{A}^{\prime})_{nm}\phi_{n}^{\prime\,\ast}\phi_{m}^{\prime}D_{A}
     -\sum_{Anm}(t_{A}^{\prime\prime})_{nm}\phi_{n}^{\prime\prime\,\ast}\phi_{m}^{\prime\prime}D_{A}
     +\mi\sum_{ABC}C_{ABC}\,\phi_{B}^{\ast}\phi_{C}D_{A} \nonumber \\
     &\quad -\sum_{A}\xi_{A}\,D_{A}+\frac{1}{2}\sum_{A}D_{A}D_{A} \nonumber \\
     &\quad -\frac{1}{4}\sum_{A}f_{A\mu\nu}f^{\mu\nu}_{A}
     +\frac{g^{2}\theta}{64\uppi^{2}}\epsilon_{\mu\nu\rho\sigma}\sum_{A}f_{A}^{\mu\nu}f_{A}^{\rho\sigma}\:,
     \label{27.9.30}
\end{align}
其中\,$(t_{A}^{\prime})_{nm}$\,和\,$(t_{A}^{\prime\prime})_{nm}$\,分别代表左手征标量超场$\,\Phi_{n}^{\prime}\,$%
和$\,\Phi_{n}^{\prime\prime}\,$上的规范群的矩阵(包含耦合常数因子). 费米子和标量之间的\,Yukawa\,耦合在变换
\begin{equation}
    \lambda_{AL}\to -\psi_{AL} \:, \qquad \psi_{AL}\to +\lambda_{AL}\:,\qquad
    \phi_{n}^{\prime\prime} \to -\phi_{n}^{\prime\,\ast}\:,\qquad
    \phi_{n}^{\prime}\to \phi_{n}^{\prime\prime\,\ast}  \label{27.9.31}
\end{equation}
下有一个离散的$\,R\,$-对称性, 如果假定
\begin{equation}
    s_{A} = -2\sqrt{2}\,\mi\,t_{A}^{\prime\,\mathrm{T}} = +2\sqrt{2}\,\mi\,t_{A}^{\prime\prime} \:. \label{27.9.32}
\end{equation}
(特别地, 注意到方程(\ref{27.9.32})要求$\,\Phi_{n}^{\prime}\,$和$\,\Phi_{n}^{\prime\prime}\,$构成的规范群表示互为复共轭.) %
除了包含辅助场的那些项, 这同时也是拉格朗日密度(\ref{27.9.30})中所有其它项的一个对称性.

将变换(\ref{27.9.31})下的对称性推广至辅助场是不可能的, 但是对称性会在消掉辅助场后出现.\footnote{在消掉辅助场后, 相应的作用量在原始的$\,N=2\,$超对称变换下仅是``在壳''不变的------即, 相差一些场满足相互作用场方程时为零的项. 这并不会造成任何损害, 因为依旧有两个守恒的超对称流, 当组成它们的场被要求满足\,Heisenberg\,绘景的场方程时, 它们的时间分量积分满足$\,N=2\,$超对称反对易关系. 确实有$\,N=2\,$超对称性的``离壳''形式体系, 但它们有各种各样的复杂性.} 在令$\,D_{A}$, $\mathscr{F}_{n}^{\prime}\,$%
和$\,\mathscr{F}_{n}^{\prime\prime}\,$等于使得拉格朗日量稳定的值, 并组合$\,D\,$-项和$\,\mathscr{F}\,$-项后, %
拉格朗日密度(其中 $s_{A}\,$和$\,t_{A}^{\prime\prime}\,$由方程(\ref{27.9.32})给定而$\,\xi_{A}\,$取为零)取如下的形式
\begin{align}
    &\mathscr{L} = -\sum_{n}(D_{\mu}\phi^{\prime})_{n}^{\ast}(D^{\mu}\phi^{\prime})_{n}
    -\sum_{n}(D_{\mu}\phi^{\prime\prime})_{n}^{\ast}(D^{\mu}\phi^{\prime\prime})_{n}
    -\sum_{A}(D_{\mu}\phi)_{A}^{\ast}(D^{\mu}\phi)_{A} \nonumber \\
    & -\frac{1}{2}\sum_{n}\Bigl(\overline{\psi^{\prime}_{n}}(\slashed{D}\psi^{\prime})_{n}\Bigr)
    -\frac{1}{2}\sum_{n}\Bigl(\overline{\psi^{\prime\prime}_{n}}(\slashed{D}\psi^{\prime\prime})_{n}\Bigr) \nonumber\\
    & -\frac{1}{2}\sum_{A}\Bigl(\overline{\psi_{A}}(\slashed{D}\psi)_{A}\Bigr)
    -\frac{1}{2}\sum_{A}\Bigl(\overline{\lambda_{A}}(\slashed{D}\lambda)_{A}\Bigr) \nonumber \\
    &-2\sqrt{2}\operatorname{Im}\sum_{Anm}(t_{A}^{\prime})_{mn}\phi_{A}\,
   \Bigl(\psi_{nL}^{\prime\,\mathrm{T}}\epsilon\psi_{mL}^{\prime\prime}\Bigr)
   -2\sqrt{2}\operatorname{Re}\sum_{ABC}C_{ABC}\,\Bigl(\lambda_{AL}^{\mathrm{T}}\epsilon\psi_{CL}\Bigr)
   \,\phi_{B}^{\ast} \nonumber \\
    &-2\sqrt{2}\operatorname{Im}\sum_{Anm}(t_{A}^{\prime})_{mn}\phi_{n}^{\prime}\,
   \Bigl(\psi_{mL}^{\prime\prime\,\mathrm{T}}\epsilon\psi_{AL}\Bigr)
   -2\sqrt{2}\operatorname{Im}\sum_{ABC}(t_{A}^{\prime})_{mn}\,\phi_{m}^{\prime\prime}
   \Bigl(\psi_{nL}^{\prime\,\mathrm{T}}\epsilon\psi_{AL}\Bigr)  \nonumber \\
    &+2\sqrt{2}\operatorname{Im}\sum_{Anm}(t_{A}^{\prime})_{mn}\,
   \Bigl(\psi_{nL}^{\prime\,\mathrm{T}}\epsilon\lambda_{AL}\Bigr) \,\phi_{m}^{\prime\,\ast}
   -2\sqrt{2}\operatorname{Im}\sum_{Anm}(t_{A}^{\prime})_{nm}\,
   \Bigl(\psi_{nL}^{\prime\prime\,\mathrm{T}}\epsilon\lambda_{AL}\Bigr)\,\phi_{m}^{\prime\prime\,\ast}  \nonumber \\
   &-\frac{1}{4}\sum_{A}f_{A\mu\nu}f_{A}^{\mu\nu}
   + \frac{g^{2}\theta}{64\uppi^{2}}\epsilon_{\mu\nu\rho\sigma}\sum_{A}f_{A}^{\mu\nu}f_{A}^{\rho\sigma}\nonumber\\
   &-\sum_{ABnm}\{t_{A}^{\prime}\,,t_{B}^{\prime}\}_{mn}\phi_{A}\phi_{B}^{\ast}
   \Bigl(\phi_{n}^{\prime}\phi_{m}^{\prime\,\ast}+\phi_{n}^{\prime\prime\,\ast}\phi_{m}^{\prime\prime}\Bigr)\nonumber\\
    & -\frac{1}{2}\sum_{A}\Biggl[\sum_{nm}(t_{A}^{\prime})_{nm}
    \Bigl(\phi_{n}^{\prime\,\ast}\phi_{m}^{\prime}-
    \phi_{n}^{\prime\prime}\phi_{m}^{\prime\prime\,\ast}\Bigr)\Biggr]^{2} \nonumber \\
    &+\frac{1}{2}\sum_{ABCDE}C_{ABC}C_{ADE}\phi_{B}^{\ast}\phi_{C}\phi_{D}^{\ast}\phi_{E}
    -2\sum_{A}\Biggl\lvert\sum_{nm}(t_{A}^{\prime})_{nm}\phi_{n}^{\prime}\phi_{m}^{\prime\prime}\Biggr\rvert^{2}
    \nonumber \\
    &-4\operatorname{Re}\sum_{nm}(t_{A}^{\prime}\mu)_{nm}\phi_{n}^{\prime\,\ast}\phi_{m}^{\prime}
    -4\operatorname{Re}\sum_{nm}(\mu\,t_{A}^{\prime})_{nm}\phi_{n}^{\prime\prime}\phi_{m}^{\prime\prime\,\ast}
    \nonumber \\
    &-2\sum_{nm}(\mu^{\dag}\mu)_{nm}\phi_{n}^{\prime\,\ast}\phi_{m}^{\prime}
    -2\sum_{nm}(\mu\,\mu^{\dag})_{nm}\phi_{n}^{\prime\prime}\phi_{m}^{\prime\prime\,\ast} \:.  \label{27.9.33}
\end{align}
右边的后五行来自于方程(\ref{27.9.30})中包含辅助场的那些项, 现在给定
\begin{equation}
    [t_{A}^{\prime}\,,\mu]=[\mu^{\dag}\,,\mu]=0 \:, \label{27.9.34}
\end{equation}
它们在离散变换(\ref{27.9.31})下也将是不变的.

我们现在可以更进一步并考虑$\,N=4\,$扩充整体超对称性的情况. (正如\,{\ref{sec:25.4}}\,节评述的, %
$N=3\,$超对称与$\,N=4\,$超对称性相同) $N=4\,$超对称性的无质量多重态中不含引力子或引力微子的只能由一个螺旋度为\,1\,的粒子, 一个螺旋度\,1/2\,粒子的$\,SU(4)\,$四重态, 以及一个零螺旋度粒子的 $SU(4)\,$六重态, 加上与它们螺旋度相反的$\,\mathsf{CPT}\,$共轭构成. 对规范群的每个生成元$\,t_{A}\,$都有这样一个超多重态. 这些粒子可以被分组成$\,N=2\,$超对称性的超多重态: %
对每个$\,t_{A}\,$有一个规范超多重态, 这个规范多重态的$\,\mathsf{CPT}\,$共轭, 以及两个极多重态, 规范多重态由螺旋度为\,1\,的一个粒子, 螺旋度为$\,\pm1/2\,$的两个粒子和螺旋度为\,0\,的一个粒子构成, $\mathsf{CPT}\,$共轭则由螺旋度相反的粒子构成, %
而每个极多重态由两个螺旋度各为$\,\pm 1/2\,$的粒子和两个零螺旋度的粒子构成. %
$N=2\,$规范超场由一个$\,N=1\,$规范超场$\,V_{A}\,$和一个左手征标量超场$\,\Phi_{A}\,$以及它们的复共轭构成, %
而两个$\,N=2\,$的极多重态由两个另外的左手征标量超场$\,\Phi_{A}^{\prime}\,$和$\,\Phi_{A}^{\prime\prime}\,$以及它们的复共轭构成.

由于$\,N=4\,$超对称性包含$\,N=2\,$超对称性, 在消掉$\,N=1\,$超对称性的辅助场后, 它的拉格朗日密度必然是方程(\ref{27.9.33})的一个特殊情况, 只不过指标$\,n,\,m\,$等现在在伴随表示的指标$\,A,\,B,\,C$ 中取值. 另外, %
超势(\ref{27.9.29})中的系数$\,\mu_{nm}\,$在这里必须为零, 否则方程(\ref{27.9.33})将会包含费米场$\,\psi_{A}^{\prime}$ %
和$\,\psi_{A}^{\prime\prime}\,$的二次项, 但它们的$\,N=4\,$超对称伙伴$\,\lambda_{A}\,$和$\,\psi_{A}\,$没有这样的项. %
再令\,$(t_{A}^{\prime})_{BC}$\,等于伴随表示中的生成元$\,-\mi\,C_{ABC}$, 我们发现拉格朗日密度必须取如下的形式
\begin{align}
    &\mathscr{L} = -\sum_{A}(D_{\mu}\phi^{\prime})_{A}^{\ast}(D^{\mu}\phi^{\prime})_{A}
    -\sum_{A}(D_{\mu}\phi^{\prime\prime})_{A}^{\ast}(D^{\mu}\phi^{\prime\prime})_{A}
    -\sum_{A}(D_{\mu}\phi)_{A}^{\ast}(D^{\mu}\phi)_{A} \nonumber \\
    & -\frac{1}{2}\sum_{A}\Bigl(\overline{\psi^{\prime}_{A}}(\slashed{D}\psi^{\prime})_{A}\Bigr)
    -\frac{1}{2}\sum_{A}\Bigl(\overline{\psi^{\prime\prime}_{A}}(\slashed{D}\psi^{\prime\prime})_{A}\Bigr) \nonumber\\
    & -\frac{1}{2}\sum_{A}\Bigl(\overline{\psi_{A}}(\slashed{D}\psi)_{A}\Bigr)
    -\frac{1}{2}\sum_{A}\Bigl(\overline{\lambda_{A}}(\slashed{D}\lambda)_{A}\Bigr) \nonumber \\
    & -2\sqrt{2}\operatorname{Re}\sum_{ABC}C_{ABC}\phi_{A}
    \Bigl(\psi_{BL}^{\prime\,\mathrm{T}}\epsilon\psi_{CL}^{\prime\prime}\Bigr)
    -2\sqrt{2}\operatorname{Re}\sum_{ABC}C_{ABC}\,
    \Bigl(\lambda_{AL}^{\mathrm{T}}\epsilon\psi_{CL}\Bigr)\,\phi_{B}^{\ast} \nonumber \\
    &-2\sqrt{2}\operatorname{Re}\sum_{ABC}C_{ABC}\phi_{B}^{\prime}
    \Bigl(\psi_{CL}^{\prime\prime\,\mathrm{T}}\epsilon\psi_{AL}\Bigr)
    -2\sqrt{2}\operatorname{Re}\sum_{ABC}C_{ABC} \phi_{C}^{\prime\prime}
    \Bigl(\psi_{BL}^{\prime\,\mathrm{T}}\epsilon\psi_{AL}\Bigr) \nonumber \\
    & +2\sqrt{2}\operatorname{Re}\sum_{ABC}C_{ABC}
    \Bigl(\psi_{BL}^{\prime\,\mathrm{T}}\epsilon\lambda_{AL}\Bigr)\phi_{C}^{\prime\,\ast}
    +2\sqrt{2}\operatorname{Re}\sum_{ABC}C_{ABC}
    \Bigl(\psi_{BL}^{\prime\prime\,\mathrm{T}}\epsilon\lambda_{AL}\Bigr)\phi_{C}^{\prime\prime\,\ast}  \nonumber \\
    &-\frac{1}{4}\sum_{A}f_{A\mu\nu}f_{A}^{\mu\nu}
    +\frac{g^{2}\theta}{64\uppi^{2}}\epsilon_{\mu\nu\rho\sigma}\sum_{A}f_{A}^{\mu\nu}f_{A}^{\rho\sigma}
    -V\:,      \label{27.9.35}
\end{align}
其中势是
\begin{align}
    V&=\sum_{ABCDE} C_{ADE}C_{BCE}\Bigl(\phi_{A}\phi_{B}^{\ast}+\phi_{B}\phi_{A}^{\ast}\Bigr)
    \Bigl(\phi_{C}^{\prime}\phi_{D}^{\prime\,\ast}
    +\phi_{C}^{\prime\prime\,\ast}\phi_{D}^{\prime\prime}\Bigr)\nonumber\\
    &\quad +\frac{1}{2}\sum_{A}\Biggl\lvert\sum_{BC}
    \Bigl(\phi_{B}^{\prime\,\ast}\phi_{C}^{\prime}-\phi_{B}^{\prime\prime}\phi_{C}^{\prime\prime\,\ast}\Bigr)
    \Biggr\rvert^{2}\nonumber \\
    &\quad -\frac{1}{2}\sum_{ABCDE}C_{ABC}C_{ADE}\phi_{B}^{\ast}\phi_{C}\phi_{D}^{\ast}\phi_{E}
    +2\sum_{A}\Biggl\lvert\sum_{BC} C_{ABC}\phi_{B}^{\prime}\phi_{C}^{\prime\prime} \Biggr\rvert^{2}\label{27.9.36}
\end{align}

不需要进一步的约束, 这个拉格朗日量就有\,$SU(4)\,R$\,-对称性, 这暗示了它在\,$N=4$\,超对称性下是不变的. 为了看到这点, %
我们需要用\,Jacobi\,恒等式将方程(\ref{27.9.36})右边第二行中的交叉项写成如下形式
\begin{align*}
    \sum_{ABCDE}C_{ABC}C_{ADE}\phi_{B}^{\prime\,\ast}\phi_{C}^{\prime}
    \phi_{D}^{\prime\prime\,\ast}\phi_{E}^{\prime\prime}
    &=-\sum_{ABCDE}C_{ABC}C_{ADE}\phi_{B}^{\prime\,\ast}\phi_{D}^{\prime}
    \phi_{E}^{\prime\prime\,\ast}\phi_{C}^{\prime\prime}  \\
    &\quad -\sum_{ABCDE}C_{ABC}C_{ADE}\phi_{B}^{\prime\,\ast}\phi_{E}^{\prime}
    \phi_{C}^{\prime\prime\,\ast}\phi_{D}^{\prime\prime} \:,
\end{align*}
这使得我们可以将势写成对标量和它们的共轭对称的形式
\begin{align}
    V&= \sum_{A}\Biggl\lvert \sum_{BC}C_{ABC}\phi_{B}^{\ast}\phi_{C}^{\prime}\Biggr\rvert^{2}
    + \sum_{A}\Biggl\lvert \sum_{BC}C_{ABC}\phi_{B}^{\ast}\phi_{C}^{\prime\prime\,\ast}\Biggr\rvert^{2}
    + \sum_{A}\Biggl\lvert \sum_{BC}C_{ABC}\phi_{B}\phi_{C}^{\prime}\Biggr\rvert^{2} \nonumber \\
    &\quad  \sum_{A}\Biggl\lvert \sum_{BC}C_{ABC}\phi_{B}\phi_{C}^{\prime\prime\,\ast}\Biggr\rvert^{2}
    + \sum_{A}\Biggl\lvert \sum_{BC}C_{ABC}\phi_{B}^{\prime\,\ast}\phi_{C}^{\prime\prime}\Biggr\rvert^{2}\nonumber\\
    &\quad + \sum_{A}\Biggl\lvert \sum_{BC}C_{ABC}\phi_{B}^{\prime}\phi_{C}^{\prime\prime}\Biggr\rvert^{2}
    +\frac{1}{2}\sum_{A}\Biggl\lvert \sum_{BC}C_{ABC}\phi_{B}^{\prime}\phi_{C}^{\prime\,\ast}\Biggr\rvert^{2}\nonumber\\
    &\quad+\frac{1}{2}\sum_{A}\Biggl\lvert \sum_{BC}C_{ABC}\phi_{B}^{\prime\prime}\phi_{C}^{\prime\prime\,\ast}\Biggr\rvert^{2}
    +\frac{1}{2}\sum_{A}\Biggl\lvert \sum_{BC}C_{ABC}\phi_{B}\phi_{C}^{\ast}\Biggr\rvert^{2} \label{27.9.37}
\end{align}
现在, 为了显现\,$SU(4)$\,对称性, 我们引入场的\,$SU(4)$\,记法. 我们把左手费米场组装成一个\,$SU(4)$\,矢量:
\begin{equation}
    \psi_{1AL}\equiv \psi_{AL}\:, \qquad \psi_{2AL}\equiv\lambda_{AL} \:,\qquad
    \psi_{3AL}\equiv \psi_{AL}^{\prime}\:,\qquad\psi_{4AL}\equiv\psi_{AL}^{\prime\prime} \:.\label{27.9.38}
\end{equation}
为了使拉格朗日密度中的费米子动能项是\,$SU(4)$\,-不变的, 我们必须把右手费米场组装成一个逆变矢量:
\begin{equation}
    \psi_{AR}^{1}\equiv \psi_{AR}\:,\qquad  \psi_{AR}^{2}\equiv \lambda_{AR} \:, \qquad
    \psi_{AR}^{3}\equiv \psi_{AR}^{\prime} \:, \qquad \psi_{AR}^{4}\equiv \psi_{AR}^{\prime\prime}\:. \label{27.9.39}
\end{equation}
这样, 费米场上的\,Majorana\,条件就取\,$SU(4)$\,-不变的形式
\begin{equation}
    (\psi_{iAL})^{\ast}=-\beta\epsilon\psi_{AR}^{i} \:, \label{27.9.40}
\end{equation}
其中指标\,$i,\,j$\,等在\,1,\,2,\,3,\,4\,中取值. 为了使费米场和标量场之间的\,Yukawa\,耦合是\,$SU(4)$\,不变的, 我们必须给标量赋予反对称\,$SU(4)$\,张量的变换性质
\begin{equation}
    \begin{split}
        &\phi_{A}^{12}\equiv \phi_{A}^{\ast} \:,\qquad \phi_{A}^{13}\equiv\phi_{A}^{\prime\prime}\:,\qquad
        \phi_{A}^{14}\equiv -\phi_{A}^{\prime} \:, \\
        &\phi_{A}^{23}\equiv -\phi_{A}^{\prime\,\ast} \qquad \phi_{A}^{24}\equiv-\phi_{A}^{\prime\prime\,\ast}\:,
        \qquad \phi_{A}^{34} \equiv \phi_{A} \:,
    \end{split} \label{27.9.41}
\end{equation}
这同时还服从一个\,$SU(4)$\,-不变的实条件
\begin{equation}
    \Bigl(\phi_{A}^{ij}\Bigr)^{\ast} = \frac{1}{2}\sum_{kl}\epsilon_{ijkl}\,\phi_{A}^{kl} \:. \label{27.9.42}
\end{equation}
这样, 整个拉格朗日密度(\ref{27.9.35})就可以写成一个显然\,$SU(4)$\,-不变的形式
\begin{align}
    \mathscr{L} &= -\frac{1}{2}\sum_{Aij}(D_{\mu}\phi^{ij})_{A}(D^{\mu}\phi^{ij})^{\ast}_{A} \nonumber \\
    &\quad-\frac{1}{2}\sum_{Ai}\Bigl(\psi_{iAL}^{\mathrm{T}}\epsilon(\slashed{D}\psi_{R}^{i})_{A}\Bigr)
    +\frac{1}{2}\sum_{Ai}\Bigl(\psi_{AR}^{i\,\mathrm{T}}\epsilon(\slashed{D}\psi_{iL})_{A}\Bigr) \nonumber \\
    &\quad -\sqrt{2}\operatorname{Re}\sum_{ABCij}C_{ABC}\phi_{A}^{ij}
    \Bigl(\psi_{iBL}^{\mathrm{T}}\,\epsilon\psi_{jCL}\Bigr) - V  \nonumber \\
    &\quad -\frac{1}{4}\sum_{A}f_{A\mu\nu}f^{\mu\nu}_{A}+
    \frac{g^{2}\theta}{64\uppi^{2}}\epsilon_{\mu\nu\rho\sigma}\sum_{A}f_{A}^{\mu\nu}f_{A}^{\rho\sigma}\:,
    \label{27.9.43}
\end{align}
其中势是
\begin{equation}
    V=\frac{1}{8}\sum_{Aijkl}\Biggl\lvert \sum_{BC}C_{ABC}\phi_{B}^{ij}\phi_{C}^{kl}\Biggr\rvert^{2} \:. \label{27.9.44}
\end{equation}
这个势有一个为零的最小值, 这使得这个理论中的超对称性是不破缺的. 当生成元\,$\sum_{A}t_{A}\phi_{A}^{ij}$\,全部彼此对易时, 势能到达这个最小值.

当$\,\theta\,$角为零时, 无论是\,$N=2$\,还是\,$N=4$\,的超对称性, 只有一个单规范群的规范理论只有一个耦合常数, %
规范耦合常数\,$g$. 由于这些理论有\,$N=1$\,规范对称性, 它们享有\,\ref{sec:27.6}\,节讨论过的性质, 在微扰论的高阶中, 无穷大只出现在对这个耦合的单圈修正中.\footnote{尽管有\,\ref{sec:27.6}\,节的无重整定理, 超势(\ref{27.9.29})中的三线性项正比于规范耦合, 因此它是重整化的. 这是因为, 这里我们重整化了左手征标量超场\,$\Phi_{A},\,\Phi_{n}^{\prime}\,$和%
$\,\Phi_{n}^{\prime\prime}\,$以及规范超场$\,V_{A}\,$以保持它们是正则归一的. 由于相同的原因, 方程(\ref{27.9.29})中的双线性项也是重整化的.} 那么到微扰论的所有阶, 重整化群方程$\,\mu\dif g/\dif \mu=\beta(g)\,$中的函数$\,\beta(g)\,$就由单圈公式%
(\textcolor{foo}{18.7.2})加上因出现标量场的合适修正给出:
\begin{equation}
    \beta(g) = -\frac{g^{3}}{4\uppi^{2}}\,\biggl(\frac{11}{12}C_{1}-\frac{1}{6}C_{2}^{f}
    -\frac{1}{12}C_{2}^{s}\biggr) \:, \label{27.9.45}
\end{equation}
其中
\begin{align}
    &\sum_{AB}C_{ABC}C_{ABD}=g^{2}C_{1}\delta_{CD} \:, \nonumber \\
    &\Bigl[\operatorname{Tr}(t_{C}t_{D})\Bigr]_{\text{Majorana fermions}} = g^{2}C_{2}^{f}\delta_{CD} \:,\label{27.9.46} \\
    &\Bigl[\operatorname{Tr}(t_{C}t_{D})\Bigr]_{\text{complexs scalars}}=g^{2}C_{2}^{s}\delta_{CD} \:.
\end{align}
在有$\,N=2\,$超对称性的一般理论中, 我们有两个处在伴随表示下的\,Majorana\,费米子$\,\lambda_{A}\,$和$\,\psi_{A}\,$以及%
$\,H\,$对\,Majorana\,费米子$\,\psi_{n}^{\prime}\,$和$\,\psi_{n}^{\prime\prime}$, %
它们的左手和右手部分处在生成元为$\,t_{A}^{\prime}\,$或$\,-t_{A}^{\prime\mathrm{T}}\,$的表示中, 所以
\begin{equation}
    C_{2}^{f}=2C_{1}+2HC_{2}^{\prime} \:, \label{27.9.47}
\end{equation}
其中$\,C_{2}^{\prime}\,$被定义成
\begin{equation}
    \operatorname{Tr}t_{C}^{\prime}t_{D}^{\prime}=g^{2}C_{2}^{\prime}\delta_{CD} \:. \label{27.9.48}
\end{equation}
另外, 我们有一个处在伴随表示下的复标量$\,\phi_{A}\,$和$\,H\,$对处在生成元为$\,t_{A}^{\prime}\,$或$\,-t_{A}^{\prime\mathrm{T}}\,$的表示下的$\,\phi_{n}^{\prime}\,$和 $\phi_{n}^{\prime\prime}$, 所以
\begin{equation}
    C_{2}^{s} =C_{1} +2 HC_{2}^{\prime} \:. \label{27.9.49}
\end{equation}
因此\,$\beta$\,函数(\ref{27.9.45})是
\begin{equation}
    \beta(g) =-\frac{g^{2}}{8\uppi^{2}}\Bigl(C_{1}-HC_{2}^{\prime}\Bigr) \:. \label{27.9.50}
\end{equation}
$N=4\,$超对称性就是$\,H=1\,$对$\,N=2\,$极多重态处在伴随表示下的特殊情况, 即有$\,C_{2}^{\prime}=C_{1}$, %
所以这一情况下的$\,\beta\,$函数为零. 因此这是一个根本没有重整化的有限理论.\cite{19}

有$\,N=4\,$超对称性的规范理论有另外一个显著性质, 称为{\kai{对偶}}. %
这最早是\,Montonen\,和\,Olive 对单规范群自发破缺到$\,U(1)\,$电磁规范群的纯玻色理论做出的猜想. 他们注意到, %
对于电荷$\,q=ne\,$且磁单极距$\,\mathscr{M}=4\uppi m/e\,$($\,n\,$和$\,m\,$是符号任意的整数)的粒子, %
(23.3\,节描述过的那类)半经典计算给出的粒子质量是
\begin{equation}
    M=\sqrt{2}\biggl\lvert v\,\biggl(ne + \frac{4\uppi\mi m}{e}\biggr)\biggr\rvert \:, \label{27.9.51}
\end{equation}
它在变换
\begin{equation}
    m\to n\:, \qquad n\to -m \:, \qquad  e \leftrightarrow 4\uppi/e . \label{27.9.52}
\end{equation}
下是不变的. 以此为基础, 他们认为一个有弱规范耦合$\,e\,$的理论完全等于一个有强规范耦合$\,4\uppi/e$ 的理论. %
纯玻色理论或$\,N=1\,$和$\,N=2\,$扩充超对称理论的最简单版本实际上都没有这个性质;\cite{20} 首先, 破缺规范对称性的有质量带电荷矢量玻色子有自旋\,1, 而所有磁单极子和双荷子有自旋$\,1/2$ 或\,0. (我们将在\,\ref{sec:29.5}\,节看到, $N=2\,$的理论确实一类更加巧妙的对偶性质.) 但对于$\,N=4\,$超对称性, 单极子态构成了有一个自旋\,1\,粒子, 4\,自旋\,1/2\,粒子和两个自旋\,0\,粒子的多重态, 就像基本粒子一样.\cite{20} $N=4\,$超对称规范理论在电量子数和磁量子数以及$\,e\,$和$\,4\uppi/e\,$交换下不变这一点已有证据.\cite{21} 强耦合理论和弱耦合理论的等价性已经变成弦论中日趋重要的课题, 但这超出本书的讨论范围.

\section*{习题}
\noindent 1. 到规范耦合常数的第二阶, 计算出使用变换(\ref{27.1.12})将规范超场$\,V^{A}\,$变到\,Wess-Zumino\,规范下所需要的%
超场$\,\Omega^{A}\,$的分量. \\

\noindent 2. 证明满足条件(\ref{27.2.20})的最一般手征线性超场$\,W_{\alpha}\,$拥有满足齐次\,Maxwell\,方程$\,\epsilon^{\mu\nu\rho\sigma}\partial_{\rho}f_{\mu\nu}=0\,$的分量$\,f_{\mu\nu}$. 方程(\ref{27.2.20})给$\,W_{\alpha}\,$的其它分量附加的条件是什么? \\

\noindent 3. 考虑有一个\,$SU(2)$\,规范群和一个手征超场的一般可重整$\,N=1\,$超对称规范理论, 其中手征超场属于$\,SU(2)\,$%
的\,3\,-矢表示. 这个理论最一般的超势是什么? 清楚地构造出整个理论的拉格朗日密度. 消掉辅助场. 证明这个理论中的超对称性是不破缺的. 这个理论中粒子的质量是什么? \\

\noindent 4. 将\,27.5\,节描述过的量子电动力学超对称版中的规范微子场和费米场表示成戈德斯通微子场以及其它有明确质量的旋量场?\\

\noindent 5. 考虑有一个\,$SU(3)$\,规范对称性但没有极多重态的可重整$\,N=2\,$超对称理论. 使得势为零的标量场的值是什么? 对这些标量不为零的值, 无质量规范场是什么? 计算出中心荷, 并用与这些无质量玻色场耦合的量表示它.

%++++++++++++++++++参考文献+++++++++
\renewcommand{\sectionmark}[1]{\markright{ #1}{}}
\renewcommand{\bibname}{参考文献}

\begin{thebibliography}{99}
    \bibitem{1} 超对称性第一次被应用于阿贝尔规范理论时没有使用超场形式体系, J. Wess and B. Zumino, {\textit{Nucl. Phys.}} {\bf{B78}}, 1 (1974). 然后它被推广至非阿贝尔规范理论, S. Ferrara and B. Zumino, {\textit{Nucl. Phys.}} {\bf{B79}}, 413 (1974); A. Salam and J. Strathdee, {\textit{Phys. Lett.}} {\bf{51B}}, 353 (1974). 这些文章重印于{\textit{Supersymmetry}}, S. Ferrar\,编辑(North Holland/World Scientific, Amsterdam/Singapore, 1987).
    \bibitem{2} P. Fayet and J Iliopoulos, {\textit{Phys. Lett.}} {\bf{51B}}, 461 (1974). %
    这篇文章重印于\,{\textit{Supersymmetry}}, 参考文献[1].
    \bibitem{3} S. Weinberg, {\textit{Phys. Rev. Lett.}} {\bf{80}}, 3702 (1998).
    \bibitem{4} S. Ferrara, L. Girardello, and F. Palumbo, {\textit{Phys. Rev.}} {\bf{D20}}, 403 (1979). %
    这篇文章重印于\,{\textit{Supersymmetry}}, 参考文献[1]. P. Fayet\,给出了这个求和规则的特殊情况, {\textit{Phys. Lett.}} {\bf{84B}}, 416 (1979).
    \bibitem{5} M. T. Grisaru, W. Siegel, and M. Ro$\check{\text{c}}$ek, {\textit{Nucl. Phys.}} {\bf{B159}}, 429 (1979).
    \bibitem{6} N. Seiberg, {\textit{Phys. Lett.}} {\bf{B318}}, 469 (1993).
    \bibitem{7} 这最先是在超图形式理论中证明的, W. Fischler, H. P. Nilles, J. Polchinski, S. Raby, and L. Susskind, %
    {\textit{Phys. Rev. Lett.}} {\bf{47}}, 757 (1981). 这里给出的证明来自\,M. Dine, 收录于\,{\textit{Fields, Strings, and Duality: TASI 96}}, C. Efthimiou and B. Greene\,编辑(World Scientific, Singapore, 1997); S. Weinberg, 参考文献[3].
    \bibitem{8} 破坏某个整体对称性的超可重整项不会对可重整相互作用的系数引入无限大的对称性破缺辐射修正的细致证明是%
    由\,K. Symanzik\,给出的, 收录于\textit{Carg\'{e}se Lectures in Physics,} Vol. 5, D. Bessis\,编辑(Gordon and Breach, New York, 1972). 这在卷\,I\,第\,507\,页的脚注中简要讨论过.
    \bibitem{9} L. Girardello and M. T. Grisaru, {\textit{Nucl. Phys.}} {\bf{B194}}, 65 (1982), 重印于\,%
    {\textit{Supersymmetry}}, 参考文献[1]; K. Harada and N. Sakai, {\textit{Prob. Theor. Phys.}} {\bf{67}}, 67 (1982).
    \bibitem{10} 关于综述, 参看\,S. Weinberg, {\textit{Rev. Mod. Phys.}} {\bf{61}}, 1-23 (1989).
    \bibitem{11} B. de Wit and D. Z. Freedman, {\textit{Phys. Rev.}} {\bf{D12}}, 2286 (1975). 这篇文章重印于\,%
    {\textit{Supersymmetry}}, 参考文献[1].
    \bibitem{12} 首例有$\,N=2\,$扩充超对称性的规范理论是\,P. Fayet\,给出的, {\textit{Nucl. Phys.}} {\bf{B113}}, 135 (1976); 重印于\,{\textit{Supersymmetry}}, 参考文献[1]. 这里给出的方法与\,P. Fayet\,的类似. R. Grimm, M. Sohnius\,和\,J. Scherk\,随后给出了超场形式体系, {\textit{Nucl. Phys.}} {\bf{B113}}, 77 (1977). 四维时空中的$\,N=2\,$和$\,N=4\,$超对称规范理论可以通过对高维时空的简单超对称理论做维度约化得到, L. Brink, J. H. Schwarz, and J. Scherk, {\textit{Nucl. Phys.}} {\bf{B113}}, 77 (1977); M. F. Sohnius, K. S. Stelle, and P. C. West, {\textit{Nucl. Phys.}} {\bf{B113}}, 127 (1980); 关于其它方法, 参看\,M. F. Sohnius, {\textit{Nucl. Phys.}} {\bf{B138}}, 109 (1979); A. Halperin, E. A. Ivanov, and V. I. Ogievetsky, {\textit{Prima JETP}} {\bf{33}}, 176 (1981); P. Breitenlohner and M. F. Sohnius, {\textit{Nucl. Phys.}} {\bf{B178}}, 151 (1981); P. Howe, K. S. Stelle, and P. K. Townsend, {\textit{Nucl. Phys.}} {\bf{B214}}, 519 (1983).
    \bibitem{13} E. Witten and D. Olive, {\textit{Phys. Lett.}} {\bf{78B}}, 97 (1978). 另见\,H. Osborn, {\textit{Phys. Lett.}} {\bf{83B}}, 321 (1979).
    \bibitem{14} E. B. Bogomol'nyi, {\textit{Sov. J. Nucl. Phys.}} {\bf{24}}, 449 (1976).
    \bibitem{15} D. Zwanziger, {\textit{Phys. Rev.}} {\bf{176}}, 1480 (1968); J. Schwinger, {\textit{Phys. Rev.}} {\bf{144}}, 1087 (1966); {\bf{173}}, 1536 (1968); B. Julia and A. Zee, {\textit{Phys. Rev.}} {\bf{D11}}, 2227 (1974); F. A. Bais and J. R. Primack, {\textit{Phys. Rev.}} {\bf{D13}}, 819 (1975). (在卷\,II\,第一次印刷的版本中, 第\,23\,章中错误地将最后一篇的作者写成\,Julia\,和\,Zee.)
    \bibitem{16} M. K. Prasad and C. M. Sommerfield, {\textit{Phys. Rev. Lett.}} {\bf{35}}, 760 (1975); E. B. Bogomol'nyi, 参考文献[14]; S. Coleman, S. Parke, A. Neveu, and C. M. Sommerfield, {\textit{Phys. Rev.}} {\bf{D15}}, 544 (1977).
    \bibitem{17} C. Montonen\,和\,D. Olive\,注意到了这一点, {\textit{Phys. Lett.}} {\bf{72B}}, 117 (1977). 证明质量没有单圈修正的是, A. D'Adda, R. Horsley, and P. Di Vecchia, {\textit{Phys. Lett.}} {\bf{76B}}, 298 (1978).
    \bibitem{18} W. Siegel\,和\,M. Ro$\check{\text{c}}$ek分析了对$\,N=4\,$超对称性用辅助场形式化理论的障碍所在, {\textit{Phys. Lett.}} {\bf{105B}}, 275 (1981).
    \bibitem{19} 证明$\,N=4\,$理论有限性的是, M. F. Sohnius and P. C. West, {\textit{Nucl. Phys.}} {\bf{B100}}, 245 (1981); P. S. Howe, K. S. Stelle, and P. Townsend, {\textit{Nucl. Phys.}} {\bf{B214}}, 519 (1983); S. Mandelstam, {\textit{Nucl. Phys.}} {\bf{B213}}, 149 (1983); L. Brink, O. Lindgren, and B. E. W. Nilsson, {\textit{Nucl. Phys.}} {\bf{B212}}, 401 (1983); {\textit{Phys. Lett.}} {\bf{123B}}, 328 (1983). N. Seiberg\,把这个证明推广至非微扰效应, {\textit{Phys. Lett.}} {\bf{B206}}, 75 (1988). 另见\,S. Kovacs, hep-th/9902047, 即将发表. %
        有一大类紫外有限的$\,N=2\,$理论; 参看\,P. S. Howe, K. S. Stelle, and P. C. West, {\textit{Phys. Lett.}} {\bf{B124}}, 55 (1983).
    \bibitem{20} H. Osborn, 参考文献[13].
    \bibitem{21} A. Sen, {\textit{Phys. Lett}}. {\bf{B329}}, 217 (1994); C. Vafa and E. Witten, {\textit{Nucl. Phys.}} {\bf{B431}}, 3 (1994); L. Girardello, A. Giveon, M. Porrati, and A. Zaffaroni, {\textit{Phys. Lett.}} {\bf{B334}}, 331 (1994).
\end{thebibliography}


%第二十八章 标准模型的超对称版本

\chapter{标准模型的超对称版}  \label{cha:28}

在现今加速器实验室能达到的能量处, 物理现象由标准模型描述, 即夸克, 轻子和规范玻色子被规范群$\,SU(3)\times SU(2)\times U(1)\,$ 控制的可重整理论, 我们在\,18.7\,节和\,21.3\,节讨论过. 标准模型现在通常\cite{1}通常被理解成一个未知理论的低能近似, 而在这个未知理论中, 引力在\,$10^{16}$\,到$\,10^{18}\,\mathrm{GeV}$ 之间的某个能量处于强力和电弱力统一在一起. %
这产生了{\kai{等级问题}}(\emph{hierarchy problem}): 如何解释这一基本能标与表征标准模型的能标%
$\,\approx 300\,\mathrm{GeV}\,$之间这个庞大的比值?

理论上提出超对称的最强动机是它为解决等级问题提供了希望. 夸克, 轻子和规范玻色子被\,$SU(3)\times SU(2)\times U(1)\,$对称性要求以零质量的方式出现在标准模型的拉格朗日量中, 这使得这些粒子的物理质量正比于电弱破缺标度, 而这个标度反过来又正比于引起电弱对称性破缺的标量场的质量. 等级问题的关键在于\cite{1a}玻色场, 不像费米子和规范玻色场, 标准模型的对称性并没有阻止标量场获得非常大的裸质量, 所以很难看到为什么它们的质量以及其它所有质量没有落在\,$10^{16}$\,到$\,10^{18}\,\mathrm{GeV}\,$附近.

曾经希望通过将标准模型嵌入一个超对称理论中来解决这个问题. 如果标量场和某个规范群的手征表示下的费米子共同处在一个超多重态中, 那么超对称性将会要求标量和费米子的裸质量为零. 这样标准模型的所有质量会被绑在超对称性破缺的那个能标处. 沿着这些路线解决等级问题的希望曾经是尝试将超对称性融入现实理论的最强动机.

不幸的是, 超对称性要求的新粒子一个都没有探测到, 并且迄今为止, 标准模型完全让人满意的超对称版还没有出现. 本章将会描述在这个方向上做过的尝试.


\section{超场, 反常和守恒律} \label{sec:28.1}

在本节, 我们将尝试至少试验性地定出标准模型的超对称版应该出现什么元素.

标准模型的夸克和轻子场没有一个属于\,$SU(3)\times SU(2)\times U(1)$\,规范群的伴随表示, 所以它们不可能是已知规范玻色子的超对称伴, 因而它们必须放在手征标量超场中. 我们定义左手征超场\,$U_{i}$, $D_{i}$, $\bar{U}_{i}$, $\bar{D}_{i}$, $N_{i}$, %
$E_{i}$\,和$\,\bar{E}_{i}$, 它们的$\,\psi_{L}\,$分量分别是电荷为$\,2e/3\,$和$\,{-}e/3\,$的夸克的左手场, %
电荷为$\,{-}2e/3\,$和$\,{+}e/3\,$的反夸克的左手场, 电荷为$\,0\,$和$\,-e\,$的轻子的左手场, 以及电荷为$\,{+}e\,$的反轻子的左手场, 而代指标\,$i$\,在\,1, 2\, 和\,3\,中取值. (例如, $U_{1}$, $U_{2}$\,和$\,U_{3}\,$的旋量分量分别是夸克%
\,$u$, $c\,$和$\,t\,$的左手场.) 在这些超场中, $U_{i}\,$和$\,D_{i}\,$构成$\,SU(2)\,$双重态, %
$N_{i}\,$和$\,E_{i}\,$也构成$\,SU(2)\,$双重态, 而其它是$\,SU(2)\,$单态. %
夸克超场构成\,$SU(3)\,$三重态而反夸克超场构成$\,SU(3)\,$反三重态. 正如前面提到过的, 有这些超场的标量分量描述的粒子被称为标量夸克, 反标量夸克, 标量轻子和反标量轻子. 同时还有规范微子, $SU(3)$, $SU(2)\,$和$\,U(1)\,$规范玻色子的超对称伙伴, 分别被称为胶微子(gluino), $W$\,微子(wino)和$\,B\,$微子(bino).\footnote{就像\,\ref{sec:28.3}\,节讨论的, 表征超对称性破缺的能标预期是远大于表征\,$SU(2)\times U(1)\,$\,破缺的能标$\,\approx 300\,\mathrm{GeV}$, 所以在一个很大的能量范围内, %
超对称性被认为是破缺的而$\,SU(2)\times U(1)\,$则不是. 在这个范围内, 规范微子的质量被$\,SU(2)\times U(1)\,$对称性决定, %
所以有确定质量的中性电弱规范微子是$\,SU(2)\,$三重态$\,W^{0}\,$和$\,SU(2)\,$单态$\,B\,$的超对称伙伴, %
被称为中性\,$W$\,微子和$\,B\,$微子, 而不是$\,Z^{0}\,$和光子的超对称伙伴. 当$\,SU(2)\times U(1)\,$破缺被考虑在内时, %
中性\,$W$\,微子和$\,B\,$微子之间有一个微小的混合.}

我们还必须加上某个机制来产生$\,SU(2)\times U(1)\,$的自发破缺并赋予所有夸克, 轻子以及$\,W^{\pm}\,$和 $Z^{0}\,$质量. %
最简单的可能性是再假定存在两个左手征超场的$\,SU(2)\,$双重态:
\begin{equation}
    H_{1}=\begin{pmatrix}
    H_{1}^{0} \\ H_{1}^{-}
    \end{pmatrix} \:, \qquad
    H_{2}=\begin{pmatrix}
    H_{2}^{+} \\ H_{2}^{0}
    \end{pmatrix} \:, \label{28.1.1}
\end{equation}
它们出现在拉格朗日密度中的方式是$\,SU(3)\times SU(2)\times U(1)\,$-不变$\,\mathscr{F}\,$-项的线性组合:
\begin{equation}
    \Bigl[\Bigl(D_{i}H_{1}^{0}-U_{i}H_{1}^{-}\Bigr)\bar{D}_{j}\Bigr]_{\mathscr{F}} \:,\qquad
    \Bigl[\Bigl(E_{i}H_{1}^{0}-N_{i}H_{1}^{-}\Bigr)\bar{E}_{j}\Bigr]_{\mathscr{F}} \:, \label{28.1.2}
\end{equation}
和
\begin{equation}
    \Bigl[\Bigl(D_{i}H_{2}^{+}-U_{i}H_{2}^{0}\Bigr)\bar{U}_{j}\Bigr]_{\mathscr{F}} \:, \label{28.1.3}
\end{equation}
其中色指标显然收缩掉了. 根据方程(\ref{26.4.11}), $H_{1}^{0}\,$的非零真空期望值将赋予带电荷的轻子和带 $-e/3\,$电荷的夸克以质量, 而$\,H_{2}^{0}\,$的非零真空期望值将带$\,+2e/3\,$电荷的夸克赋予质量. 当然, %
这些期望值也赋予了\,$W^{\pm}\,$和$\,Z^{0}\,$矢量玻色子以质量, 并且, 由于$\,H_{1}\,$和$\,H_{2}\,$是$\,SU(2)\,$双重态, %
我们自动得到了和\,21.3\,节中发现的质量同样成功的结果. 注意到, 超对称性不允许左手征超场$\,H_{1}\,$和$\,H_{2}\,$的复共轭出现在拉格朗日密度中, 所以$\,H_{1}^{0}\,$的标量分量的真空期望值并不能给带电荷$\,+2e/3\,$的夸克赋予质量, %
而$\,H_{2}^{0}\,$的标量分量的真空期望值不能给带电荷$\,-2e/3\,$的夸克或带电荷的轻子赋予质量, 这就是为了给所有夸克和轻子赋予质量赋予质量为什么既需要$\,H_{1}\,$又需要$\,H_{2}\,$的原因.

当然, $\,H_{1}$和(或)$\,H_{2}\,$双重态可能不止一个. 它们的个数部分被反常相消的条件限制. 我们在\,22.4\,节看到, %
非超对称标准模型的规范对称性是没有反常的, 这也是为了量子力学上的相容性所必须的, 但现在拉格朗日量中有额外的旋量场. 规范微子场不会产生任何问题, 这是因为它们的左手分量属于规范群的伴随表示, 而这个表示对于所有规范群都是实的. %
唯一的问题只能来自于希格斯微子(higgsinos)------超场$\,(H_{1}^{0},H_{1}^{-})\,$和$\,(H_{2}^{+},H_{2}^{0})\,$的%
自旋$\,1/2\,$分量. 超场的每个$\,(H_{1}^{0},H_{1}^{-})\,$双重态的自旋$\,1/2\,$分量会产生一个正比于%
$\,\sum t_{3}^{2}y=(\frac{1}{2}g)^{2}(\frac{1}{2}g^{\prime})+(-\frac{1}{2}g)^{2}(\frac{1}{2}g^{\prime})
=\frac{1}{2}g^{2}g^{\prime}\,$的$\,SU(2)$-$SU(2)$-$U(1)\,$反常, %
而超场的每个$\,(H_{2}^{+},H_{1}^{0})\,$双重态的自旋$\,1/2\,$分量会产生一个正比于%
$\,\sum t_{3}^{2}y=(\frac{1}{2}g)^{2}(-\frac{1}{2}g^{\prime})+(-\frac{1}{2}g)^{2}(-\frac{1}{2}g^{\prime})
=-\frac{1}{2}g^{2}g^{\prime}\,$的$\,SU(2)$-$SU(2)$-$U(1)\,$反常. %
{\kai{因此反常相消要求$\,(H_{1}^{0},H_{1}^{-})\,$和$\,(H_{2}^{+},H_{2}^{0})\,$双重态的个数相等.}} 在这种情况下, %
所有反常都抵消了, 包括$\,U(1)^{3}\,$和$\,U(1)$-引力-引力反常. 下一节将给出一个讨论来论证每种类型的双重态实际上只出现了一个.

在沿着这些路线构建的理论中, 我们必须放弃非超对称标准模型的一个吸引人的特征, 即, 它{\kai{自动}}排除了任何违反重子数和轻子数守恒的可重整相互作用. 拉格朗日密度中有几个可重整的超对称$\,SU(3)\times SU(2)\times U(1)\,$-不变$\,\mathscr{F}\,$-项会破坏%
重子数和(或)轻子数守恒但不破坏$\,SU(3)\times SU(2)\times U(1)\,$规范对称性:
\begin{equation}
    \Bigl[\Bigl(D_{i}N_{j}-U_{i}E_{j}\Bigr)\bar{D}_{k}\Bigr]_{\mathscr{F}} \:,\qquad
    \Bigl[\Bigl(E_{i}N_{j}-N_{i}E_{j}\Bigr)\bar{E}_{k}\Bigr]_{\mathscr{F}} \:, \label{28.1.4}
\end{equation}
以及
\begin{equation}
    \Bigl[\bar{D}_{i}\bar{D}_{j}\bar{U}_{k}\Bigr]_{\mathscr{F}} \:, \label{28.1.5}
\end{equation}
其中方程(\ref{28.1.5})中未写出的三个色指标被理解成与一个反对称$\,\epsilon\,$符号收缩以给出色单态. %
当所有这些相互作用出现时, 没有一种明智的方法给标量夸克和标量轻子赋予重子数和轻子数来避免对重子数和%
轻子数守恒一个未压低的破坏. 例如, 在相互作用(\ref{28.1.4})和(\ref{28.1.5})的顶点之间交换$\,\bar{D}\,$超场的标量玻色子将会%
导致仅被数个耦合常数因子压低的过程$\,u_{L}d_{R}u_{R}\to \overline{e_{R}}\,$以灾难性的速率发生, %
例如会作为$\,p\to\pi^{0}+e^{+}\,$被观测到. 为了避免这点, 必须做出一个独立假定来排除相互作用(\ref{28.1.4})---(\ref{28.1.5})中的一些或全部.

注意到, 没有必要排除掉全部相互作用(\ref{28.1.4})和(\ref{28.1.5}). 例如, 假如我们只假定在惯用的重子数赋值下重子数是守恒的: %
左手征超场$\,U_{i}\,$和$\,D_{i}\,$被赋予重子数$\,+1/3$; $\bar{U}_{i}\,$和$\,\bar{D}_{i}\,$被赋予重子数 $-1/3$; %
而$\,L_{i}$, $\bar{E}_{i}$, $H_{1}\,$和$\,H_{2}\,$均被赋予重子数\,0. %
这将允许相互作用(\ref{28.1.4})但禁止相互作用(\ref{28.1.5}).
尽管出现了, 如果超场的标量分量被赋予合适的轻子数, 仅相互作用(\ref{28.1.4})自己是不能破坏轻子数守恒的. %
通过给超场$\,N_{i}\,$和$\,E_{i}\,$赋予轻子数\,0, 给超场$\,U_{i}$, $D_{i}$, $\bar{U}_{i}\,$和%
$\,\bar{D}_{i}\,$赋予轻子数$\,-1$, 给超场$\,\bar{E}_{i}\,$赋予轻子数$\,-2$, 给超场$\,H_{1}\,$和$\,H_{2}\,$赋予轻子数$\,0$,
以及分别给$\,\theta_{L}\,$和$\,\theta_{R}\,$赋予轻子数$\,-1\,$和$\,+1$可以实现这点. (回忆, 这种$\,\theta\,$进行非平庸变换的对称性被称为$\,R$-{\kai{对称性}}.) 这样所有的夸克和轻子就会有以往的轻子数: %
费米分量$\,\nu_{iL}\,$和$\,e_{iL}\,$是$\,N_{i}\,$和$\,E_{i}\,$中$\,\theta_{L}\,$的系数, 它们有轻子数$\,0+1=+1\,$; $\bar{E}_{i}\,$超场的费米分量$\,\overline{e_{iR}}\,$有轻子数$\,-2+1=-1$, 而夸克和反夸克有轻子数$\,-1+1=0$. %
希格斯微子($\,H_{1}\,$和$\,H_{2}\,$的费米分量)有轻子数$\,0+1=+1$. 另一方面, 超场的标量分量和超场本身有相同的轻子数, %
这与以往不同. 更进一步, 左手征超场的$\,\mathscr{F}\,$-项是$\,\theta_{L}^{2}\,$的系数, %
所以方程(\ref{28.1.4})中的相互作用分别有轻子数$\,-1+0-1+2=0\,$和$\,0+0-2+2=0$\,; %
$H_{1}\,$相互作用(\ref{28.1.2})分别有轻子数$\,-1+0-1+2=0\,$和$\,0+0-2+2=0$\,; %
而$\,H_{2}\,$相互作用(\ref{28.1.3})有轻子数$\,-1+0-1+2=0$\,; 所以这些相互作用没有一个破坏轻子数守恒. %
另外, $H_{1}\,$和$\,H_{2}\,$的标量分量有轻子数$\,0$, 所以它们的真空期望值也不破坏轻子数守恒. 在这种轻子数赋值下, %
轻子数守恒排除了所有会破坏重子数守恒的可重整相互作用: 相互作用(\ref{28.1.5})有轻子数$\,-1-1-1+2=-1$, 所以被禁止了.

相互作用(\ref{28.1.4})会给出另一个破缺$\,SU(2)\times U(1)\,$的机制并赋予带电轻子以及电荷为$\,-e/3\,$的夸克以质量: %
中微子超场$\,N_{i}\,$的标量分量会有不为零的真空期望值. (当轻子数按上一段所说的被赋值后, %
因为这些标量分量有超场$\,N_{i}\,$的轻子数\,0, 所以这个期望值不会破坏轻子数守恒.) 但我们无法在没有$\,H_{1}\,$超场的情况下用这个机制来实现这点, 这是因为我们仍然需要$\,H_{2}\,$相互作用(\ref{28.1.3}) 来赋予电荷为$\,+2e/3\,$的夸克以质量, 而正如我们已经看到的, 反常相消要求超场$\,H_{1}\,$和$\,H_{2}\,$的个数相等.

取而代之, 通常假定某些对称性既禁止了相互作用(\ref{28.1.4})又禁止了相互作用(\ref{28.1.5}). 显然, %
这些对称性可能是惯用重子数和轻子数赋值下的重子数和轻子数守恒: $U_{i}$\,和$\,D_{i}\,$有重子数$\,B=1/3\,$和轻子数$\,L=0$, %
$\bar{U}_{i}\,$和$\,\bar{D}_{i}\,$有重子数$\,B=-1/3\,$和轻子数$\,0$, $N_{i}\,$和$\,E_{i}\,$有轻子数$\,L=+1\,$和重子数$\,0$, $\bar{E}_{i}\,$有轻子数$\,-1\,$和重子数$\,0$, 而$\,H_{1}^{0}, H_{1}^{-}\,$和$\,H_{2}^{+},H_{2}^{0}\,${\kai{以及}}$\,\theta_{L}\,$和$\,\theta_{R}\,$的重子数和轻子数均为\,0. 如果我们仅假定重子数和轻子数的某个线性组合是守恒的, %
例如\,22.4\,节讨论过的无反常组合$,B-L$, 相同的结果依旧成立.

对是否可能存在精确的连续对称性有广泛的怀疑, 这是因为在弦论中, 存在任何精确连续对称性意味着存在一个与这个对称流耦合的无质量自旋\,1\,粒子, 这使得这个对称性必须是定域而非整体对称性.\cite{1b} 但是相互作用(\ref{28.1.4})和(\ref{28.1.5})也可以通过假定%
一个{\kai{离散}}整体对称性被禁止掉, 这个对称性被称为$\,R\,${\kai{宇称}}守恒.\cite{2} 对夸克, 轻子, %
规范玻色子以及\,Higgs\,标量, $R\,$宇称被定义成\,$+1$, 对它们的超对称伙伴则是$\,-1$. 这个$\,R\,$宇称等于
\begin{equation}
    \Pi_{R} = (-1)^{F}\,(-1)^{3(B-L)} \:, \label{28.1.6}
\end{equation}
其中$\,(-1)^{F}\,$是费米子宇称, 对于所有玻色子是$\,+1$, 对于所有费米子是$\,-1$. %
费米子宇称的符号与一个$\,2\uppi\,$旋转产生的相同, 因而总是守恒的, 所以如果$\,B-L\,$是守恒的, %
那么$\,R\,$宇称也是.\footnote{$(-1)^{3(B-L)}\,$的值对于夸克和轻子超场是$\,-1$, 对于其它所有超场则是$\,+1$, %
所以$\,R\,$宇称守恒等价于在一个所有夸克超场和轻子超场改变符号而其它超场不改变符号的变换下不变. %
这个不变性原理在参考文献[3]中被引入以排除相互作用(\ref{28.1.4})和(\ref{28.1.5}).} 即使$\,B-L\,$不守恒, %
$R\,$宇称也有可能是守恒的, 但实际上相互作用(\ref{28.1.4})和(\ref{28.1.5})被$\,R\,$宇称守恒禁止了, %
所以只要可重整相互作用被认为是$\,R\,$宇称的, 那么它就意味着重子数和轻子数是守恒的. 对于在很高能量处推测由物理过程产生的不可重整的超对称相互作用, 这是不正确的. 这种相互作用产生的重子数和轻子数不守恒过程将在\,\ref{sec:28.7}\,节进行讨论.

所有被超对称理论要求的新``超粒子(sparticles)''(标量夸克, 标量轻子, 规范微子和希格斯微子)有负的$\,R\,$宇称, %
所以如果$\,R\,$宇称是精确且不破缺的, 那么超对称性要求的新粒子中最轻的那一个必须是绝对稳定的. 这样, %
所有其它新粒子将经历一系列的衰变, 最终产生普通粒子和最轻的新粒子. %
各种超对称模型的唯象学绝大部分被选择哪个新粒子是最轻的所决定.

当有超对称性以及$\,R\,$宇称或$\,B-L\,$守恒时, 上面讨论的超场的最一般可重整拉格朗日量的组成部分有, 手征超场通常的规范不变动能部分, 由形如$\,(\Phi^{\ast}\exp(-V)\Phi)_{D}\,$的项的和给出, 对于每个夸克, 轻子和\,Higgs\,手征超场都有这样的一项, 加上规范超场通常的规范不变动能项, 由形如$\,\epsilon_{\alpha\beta}(W_{\alpha}W_{\beta})_{\mathscr{F}}\,$的项的和给出, %
对于每个$\,SU(3)$, $SU(2)\,$和$\,U(1)\,$场强超场都有这样的一项, 以及超对称\,Yukawa\,耦合, 由相互作用(\ref{28.1.2}), %
(\ref{28.1.3}), 以及一个新的耦合$\,H_{1}\,$和$\,H_{2}\,$的$\,\mathscr{F}\,$-项的线性组合给出:
\begin{align}
    \mathscr{L}_{Y} &=
    \sum_{ij}h_{ij}^{D}\Bigl[\Bigl(D_{i}H_{1}^{0}-U_{i}H_{1}^{-}\Bigr)\bar{D}_{j}\Bigr]_{\mathscr{F}}
    +\sum_{ij}h_{ij}^{E}\Bigl[\Bigl(E_{i}H_{1}^{0}-N_{i}H_{1}^{-}\Bigr)\bar{E}_{j}\Bigr]_{\mathscr{F}} \nonumber \\
    &\quad +\sum_{ij}h_{ij}^{U}\Bigl[\Bigl(D_{i}H_{2}^{+}-U_{i}H_{2}^{0}\Bigr)\bar{U}_{j}\Bigr]_{\mathscr{F}}
    +\mu\Bigl[H_{2}^{+}H_{1}^{-}-H_{2}^{0}H_{1}^{0}\Bigr]_{\mathscr{F}} + \mathrm{H.c.} \label{28.1.7}
\end{align}
我们将会在\,\ref{sec:28.3}\,节看出, 为了解释超对称破缺, 更多的项要加在这个拉格朗日量上.

方程(\ref{28.1.7})中的系数$\,\mu\,$有质量的量纲, 并且是进入标准模型拉格朗日量的超对称版中唯一有量纲参量. %
发现仍然允许有这样的项让人有些失望, 因为它再次引起了等级问题: 为什么$\,\mu\,$不是$\,10^{16}\,$至%
$\,10^{18}\,\mathrm{GeV}\,$阶的? 如果我们假定轻子数守恒, 并且赋予轻子数的值是上面讨论过的允许相互作用(\ref{28.1.4})%
但不允许相互作用(\ref{28.1.5})的非传统值. 在这一情况下, $\mu\,$-项携带轻子数$\,+2$, 因而也是被禁止的. %
如果我们假定$\,U(1)\,$``Peccei--Quinn\,对称性'',\cite{4} 对这个对称性, 超场$\,H_{1}\,$和$\,H_{2}\,$携带相同的量子数, %
例如$\,+1$, 而$\,\theta_{L}\,$和$\,\theta_{R}\,$是中性的, 这一项也可以被禁止掉. %
赋予夸克和轻子质量的相互作用(\ref{28.1.2})和(\ref{28.1.3})也将是被允许的, 例如, %
如果我们给反标量夸克和反标量轻子的左手征超场赋予\,Peccei--Quinn\,量子数$\,-1\,$而标量夸克和标量轻子的左手征超场则被取成中性就能实现这点. 这样, 这个选择也禁止了危险的相互作用(\ref{28.1.4})和(\ref{28.1.5}). 不幸的是, %
我们将在\,\ref{sec:28.4}\,节看到, 由于唯象上的原因, 方程(\ref{28.1.7})中的$\,\mu\,$-项似乎是需要的. %
\ref{sec:31.7}节讨论的引力传递的超对称破缺将为产生一个量级可接受的$\,\mu\,$项提供一个自然的机制.

通过假定超对称性确实解决了本章开头讨论的等级问题, 我们可以获得新粒子质量的一个粗糙的上界. %
与\,\ref{sec:27.6}\,节的定理一致, 如果超对称没有破缺, 中间态是夸克, 轻子, %
$W\,$或$\,Z\,$的单圈图对$\,H_{1}\,$或$\,H_{2}\,$的标量分量质量的贡献会被中间态是标量夸克, 标量轻子, %
$W\,$微子或$\,B\,$微子的单圈图抵消掉. 因此当超对称性破缺时, 这种图对$\,H_{1}\,$和$\,H_{2}\,$标量的质量平方的贡献%
$\,\delta m_{H}^{2}\,$是$\,(\mathscr{G}_{s}^{2}8\uppi^{2})\Delta m_{s}^{2}$ 阶的项之和, %
其中$\,\mathscr{G}_{s}\,$是\,Higgs\,标量与超多重态$\,s\,$的\,Yukawa\,或规范耦合, $\Delta m_{s}^{2}\,$是超多重态内部的质量平方分裂. 为了避免对这些修正做精细调节, 与标准模型拉格朗日密度中给出树级近似下观测到的$\,SU(2)\times U(1)\,$破缺的项相比, %
我们需要$\,\delta m_{H}^{2}\,$比该项量级为$\,(300\,\mathrm{GeV})^{2}\,$的系数大不了多少, %
所以我们将假定$\,\delta m_{H}^{2}<(1\,\mathrm{TeV})^{2}$. 例如, 顶夸克和顶标量夸克与$\,H_{2}\,$的耦合是单位阶的, %
所以我们预期分裂$\,\Delta m^{2}\,$应该小于$\,8\uppi^{2}\,\mathrm{TeV}^{2}$, %
顶标量夸克的质量则因此应该小于$\,10\,\mathrm{TeV}$. 我们将在\,\ref{sec:28.4}\,节看到, 通过取标量夸克的质量是近乎相等的, %
味改变的过程的速率可以被拉到实验上界以内, 在这种情况下, 这个可以取成所有标量夸克质量的一个粗略上界. %
(然而, 如果标量夸克的前两代质量反而取得非常大而顶标量夸克依旧低于$\,10\,\mathrm{TeV}\,$自然性上界,\cite{4a} %
这些过程的速率也可能被压低.) 这类讨论在其它$\,R=1\,$的粒子的质量上设置的限制或多或少要弱些, %
但至少在\,\ref{sec:28.6}\,节讨论的那类流行理论中, 这些粒子的质量没有一个预期会远大于标量夸克的质量, %
所以$\,10\,\mathrm{TeV}\,$可以被取为所有它们的一个上界. 另一方面, 这些粒子没有一个被观测到的事实仅表明它们的质量可能远大于%
$\,100\,\mathrm{GeV}$, 所以有一个充足的质量范围去找到它们.


\subsection*{* * *}

如果$\,R\,$宇称守恒或者其它某个守恒使得超对称预言的新粒子中最轻的那些稳定, 那么其中一些粒子可能从早期宇宙中分离出来. %
通过使用最初用于有质量中微子的宇宙密度的技术,\cite{4b} 这些遗留物的数密度可以被估计出来. 为了给出这种计算的一个粒子, %
对于一个宽广的合理质量的范围, 我们将证明超对称理论的新稳定粒子不能是带电或无色粒子, 像带电的标量轻子, %
$W\,$微子或$\,B\,$微子.\cite{4c}

一旦宇宙温度$\,T\,$(单位为能量, 且玻尔兹曼常数设为\,1)掉到任何稳定带电未陷俘粒子的质量 $m\,$以下, %
它们在体积$\,R^{3}\,$内的数量$\,nR^{3}\,$在体积随着宇宙膨胀的时候以$\,\overline{v\sigma}n\,$的每粒子速率因湮灭而衰减,
其中$\,\overline{v\sigma}\,$是相对速率和湮灭截面之积的平均值. 即,
\[
\frac{\dif (nR^{3})}{\dif t} = -\overline{v\sigma}n^{2}R^{3} \:,
\]
这使得
\begin{equation}
    \frac{1}{n R^{3}} = \biggl(\frac{1}{nR^{3}}\biggr)_{0}+\int_{t_{0}}^{t}\frac{\overline{v\sigma}}{R^{3}}\:\dif t\:,
    \label{28.1.8}
\end{equation}
其中$\,0\,$标记$\,T\simeq m\,$的时间点. 湮灭过程是放热的, 所以$\,\overline{v\sigma}\,$在$\,v\ll1\,$时趋于一个常数. %
另外, 在宇宙膨胀的辐射主导阶段, $R\propto t^{1/2}$, 所以积分收敛, 并给出
\begin{align}
    \biggl(\frac{1}{nR^{3}}\biggr)_{t\to\infty} &= \biggl(\frac{1}{nR^{3}}\biggr)_{0}
    +\overline{v\sigma}\int_{t_{0}}^{\infty}\frac{\dif t}{R_{0}^{3}\,(t/t_{0})^{3/2}} \nonumber \\
    &=\biggl(\frac{1}{nR^{3}}\biggr)_{0}+\frac{2\,\overline{v\sigma}\,t_{0}}{R_{0}^{3}} \:. \label{28.1.9}
\end{align}
重子数(重子减去反重子)的密度$\,n_{B}\,$趋于$\,R^{-3}$, 所以这可以重写为新粒子与重子的当前比值的一个公式:
\begin{equation}
    (n/n_{B})_{\infty} = \bigl[(n_{B}/n)_{0}+2\,\overline{v\sigma}\,n_{B0}\,t_{0}\bigr]^{-1} \:. \label{28.1.10}
\end{equation}
我们预期在$\,T\,$掉落到约等于$\,m\,$时的比值$\,(n/n_{B})_{0}\,$在量级上大约为\,1, 并且由于任何真实理论中的当前比值%
$\,(n/n_{B})_{\infty}\,$都必须远小于\,1, 我们可以忽视方程(\ref{28.1.10})右边分母中的第一项, 转而写下
\begin{equation}
    (n/n_{B})_{\infty} \simeq \frac{1}{\overline{v\sigma}\,n_{B0}\,t_{0}} \:. \label{28.1.11}
\end{equation}
$\overline{v\sigma}\,$的精确值依赖粒子自旋和它的相互作用; 仅保留因子$\,2\uppi$, 粒子质量$\,m\,$以及电荷的踪迹, %
我们可以一般地估计出它在量级上是
\begin{equation}
    \overline{v\sigma} \approx \frac{e^{4}\mathscr{N}}{2\uppi m^{2}} \approx 10^{-3}\frac{\mathscr{N}}{m^{2}} \:,
    \label{28.1.12}
\end{equation}
其中$\,\mathscr{N}\,$是质量小于$\,m\,$的带电粒子自旋态的数目, 即这个粒子可以湮灭到的态. 另外, %
宇宙在温度$\,T_{0}\simeq m\,$时的年龄是$\,t_{0}\approx m^{4}/m_{PL}$, 其中$\,m_{PL}\simeq 10^{18}\,\mathrm{GeV}$, %
而重子数的密度大约是$\,10^{-9}\,$乘以质子数密度, 而后者的量级是$\,T^{3}$, 这使得$\,n_{B0}\approx 10^{-9}m^{3}$. %
将这些放在一起, 我们发现了新带电粒子与重子的当前比值:
\begin{equation}
    (n/n_{B})_{\infty} \approx 10^{12}\,\frac{m}{m_{PL}\mathscr{N}}\approx 10^{-6}\,\frac{m\,(\mathrm{GeV})}{\mathscr{N}} \:. \label{28.1.13}
\end{equation}
这些新的带电粒子会同普通重子一样经历相同的凝聚融入星系, 恒星和行星, 所以这将是今天在地球上观测到的比值. %
但是对重水类分子被电解极大富集的水样本进行质谱分析的实验为陆地物质中$\,6\,\mathrm{GeV}<m<330\,\mathrm{GeV}\,$的新带电粒子%
设置了约为$\,10^{-21}n_{B}\,$的限制. 因此, 即使$\,\mathscr{N}\,$有\,1000\,那么大, 这些实验果断地排除了任何新带电未陷俘粒子在这个质量范围内以及在这个从早期宇宙中剩下来的数目下存在的可能性.

另一方面, 中性未陷俘粒子将被留在星系际空间. 这样的粒子或许为解释控制星系在星系团中运动所必须的引力场提供了``缺失的质量''.
其中一个可能的中性粒子是引力微子, 它的宇宙学丰度将在\,\ref{sec:28.3}\,节讨论. Ellis\,等人\cite{4c}将宇宙学考察推广至了超对称要求的所有新粒子.


\section{超对称和强-电弱统一} \label{sec:28.2}

直到我们准备好考虑超对称性是如何破缺的, 我们将不得不推迟对粒子物理的超对称模型的细致评估. %
在这一节, 我们将考虑超对称性在超对称破缺不那么重要的情形下的定量应用, 这也是超对称取得最大经验性成功的地方.

如果强和电弱相互作用的$\,SU(3)\times SU(2)\times U(1)\,$规范群被嵌入到夸克和轻子(可能还有一些$\,SU(3)\times SU(2)\times U(1)$\,-中性的费米子)作为其表示的某个单群$\,G\,$中, 那么, 就像\,21.5\,节描述的那样, %
当能量处在或超过$\,G\,$自发破缺的标度$\,M_{X}\,$时, $SU(3)\times SU(2)\times U(1)\,$耦合常数的关系是
\begin{equation}
g_{s}^{2}=g^{2} =\frac{5g^{\prime2}}{3} \quad \text{在能量}\,\geq M_{X}\text{\,时 .} \label{28.2.1}
\end{equation}
当能量远低于$\,M_{X}\,$时, 这些耦合严格收到重整化修正的影响. 如果在标度$\,\mu<M_{X}\,$处测量, %
耦合将会有值$\,g_{s}^{2}(\mu)$, $g^{2}(\mu)$, $g^{\prime2}(\mu)$, 被单圈重整化群方程
\begin{equation}
    \mu \frac{\dif}{\dif \mu}g^{\prime}(\mu)=\beta_{1}\Bigl(g^{\prime}(\mu)\Bigr)\:,\qquad
    \mu \frac{\dif}{\dif \mu}g(\mu)=\beta_{2}\Bigl(g(\mu)\Bigr)\:,\qquad
    \mu \frac{\dif}{\dif \mu}g_{s}(\mu)=\beta_{3}\Bigl(g_{s}(\mu)\Bigr)\label{28.2.2}
\end{equation}
控制, 而在$\,M_{X}\,$处的初值条件满足方程(\ref{28.2.1}). 在\,21.5\,节讨论过的最早使用这些重整化群方程的地方,\cite{5} %
直到单圈阶的\,$\beta\,$函数计算出来是
\begin{align}
    \beta_{1} &= \frac{5 n_{g}g^{\prime3}}{36\uppi^{2}} \:, \label{28.2.3} \\
    \beta_{2} &= \frac{g^{3}}{4\uppi^{2}} \biggl(-\frac{11}{6}+\frac{n_{g}}{3}\biggr) \:, \label{28.2.4} \\
    \beta_{3} &= \frac{g_{s}^{3}}{4\uppi^{2}}\biggl(-\frac{11}{4}+\frac{n_{g}}{3}\biggr) \:, \label{28.2.5}
\end{align}
其中$\,n_{g}\,$是夸克和轻子的代数, 相对较小的标量场贡献在这里被忽略了. 由于$\,M_{X}\,$结果会比当今加速器能达到的能量在量级上大很多阶, 假定超对称在$\,M_{X}\,$一下一个很大的范围内都未破缺看起来似乎是合理的, 在这种情况下, %
计算方程(\ref{28.2.1})中的$\,\beta\,$函数时需要纳入上节讨论的所有新场. 这些新场在计算$\,\beta\,$函数引入了三个主要的变化:

\noindent\textbf{1.} 对每个规范玻色子, 存在一个有相同$\,SU(3)\times SU(2)\times U(1)\,$量子数的\,Majorana\,规范微子. %
方程(\textcolor{foo}{17.5.41})表明, 对任何规范耦合的$\,\beta\,$函数, %
如果一个\,Dirac\,费米子贡献到生成元为$\,t_{A}\,$的规范群上,%\footnoteB{原书此处疑为误植成贡献(contribution).\qquad------译者注} %
它的贡献与相应规范玻色子的贡献之比是$\,-4C_{2}/11C_{1}$, %
而根据方程(\textcolor{foo}{17.5.33})和(\textcolor{foo}{17.5.34}), $C_{1}\,$和 $C_{2}\,$的比值被下式给出:
\begin{equation}
    \sum_{AB}C_{CAB}C_{DBA} = -(C_{1}/C_{2})\operatorname{Tr}(t_{C}t_{D}) \:. \label{28.2.6}
\end{equation}
对于伴随表示, $(t_{C})_{AB}=\mi\,C_{ABC}$, 所以$\,C_{1}=C_{2}$, 因此伴随表示下的一个\,Dirac\,费米子做出的贡献是规范玻色子的%
$\,-4/11$. 但规范微子是\,Majorana\,费米子, 所以它们的贡献是规范玻色子的$\,-2/11$. 因此方程(\ref{28.2.4})和(\ref{28.2.5})中的项$\,11/6\,$和$\,11/4\,$分别被约化至$\,9/6\,$和$\,9/4$.

\noindent\textbf{2.} 对每个左手夸克, 轻子, 反夸克或反轻子的场, 存在一个有相同$\,SU(3)\times SU(2)\times U(1)\,$量子数的复标量场. 沿用\,17.5\,节的方法, 对于生成元为$\,t_{A}\,$的规范群, 一个属于该群表示的复标量场对规范耦合$\,g_{i}\,$的$\,\beta\,$函数的贡献是
\begin{equation}
    [\beta_{i}(g_{i})]_{\text{scalar}}= \frac{g_{i}^{3}C_{2i}}{48\uppi^{2}} \:, \label{28.2.7}
\end{equation}
其中$\,\operatorname{Tr}(t_{A}t_{B})=g_{i}^{2}C_{2i}\delta_{AB}$. %
这是属于相同表示的\,Dirac\,旋量场由方程(\textcolor{foo}{18.7.2})给出的贡献的$\,1/4$, %
因此标量场的贡献是每个左手旋量场贡献的$\,1/2$ (包含\,Dirac\,场右手分量的复共轭). %
方程 (\ref{28.2.3}) ---(\ref{28.2.5})中$\,n_{g}\,$的系数因此应该乘以$\,3/2$.

$\beta\,$函数的负规范玻色子项减少为原来的$\,9/11\,$以及正标量夸克和标量轻子项增长为原来的$\,3/2$ 均导致了三个耦合在%
$\,M_{X}\,$以下从比值(\ref{28.2.1})处分散速率的减缓, 然而, 正如我们将要看到的, 以其本身而言, %
它不会影响对电弱混合参量$\,\sin^{2}\theta\,$的预测. 但这些变化确实增强了在方程(\ref{28.2.3})---(\ref{28.2.5})中忽略掉%
的\,Higgs\,玻色子的相对贡献, 而现在相伴这个贡献而来的还有更大的希格斯微子贡献. 当有$\,n_{s}\,$个上节讨论的超场%
\,$(H_{1}^{0},H_{1}^{-})\,$或$\,(H_{2}^{+},H_{2}^{0})\,$时, 方程(\ref{28.2.7})中的常数$\,C_{2i}\,$对于$\,SU(2)$ 是%
$\,[(1/2)^{2}+(-1/2)^{2}]n_{s}=n_{s}/2$, 对于$\,U(1)\,$则是$\,2n_{s}(\pm1/2)^{2}=n_{s}/2$. 根据方程(\ref{28.2.7}), %
这些超场的标量分量对$\,\beta_{1}\,$的贡献等于$\,n_{s}g^{\prime3}/96\uppi^{2}$, 对$\,\beta_{2}\,$的贡献也等于$\,n_{s}g^{3}/96\uppi^{2}$. 正如我们已经看到的, Majorana\,希格斯微子对$\,\beta\,$函数的贡献是量子数相同的复标量的两倍, %
所以超场$\,(H_{1}^{0},H_{1}^{-})\,$或$\,(H_{2}^{+},H_{2}^{0})$ 对$\,\beta_{1}\,$和$\,\beta_{2}\,$的贡献是\,Higgs\,标量%
的\,3/2, 因此分别等于$\,n_{s}g^{\prime3}/32\uppi^{2}\,$和$\,n_{s}g^{3}/32\uppi^{2}$.

在$\,\beta\,$函数中做出所有这些改变, 我们现在有
\begin{align}
    \beta_{1} &= \frac{g^{\prime3}}{4\uppi^{2}}\biggl(\frac{5n_{g}}{6}+\frac{n_{s}}{8}\biggr) \:, \label{28.2.8} \\
    \beta_{2} &= \frac{g^{3}}{4\uppi^{2}} \biggl(-\frac{9}{6}+\frac{n_{g}}{2}+\frac{n_{s}}{8}\biggr) \:, \label{28.2.9} \\
    \beta_{3} &= \frac{g_{s}^{3}}{4\uppi^{2}}\biggl(-\frac{9}{4}+\frac{n_{g}}{2}\biggr) \:, \label{28.2.10}
\end{align}
重整化群方程(\ref{28.2.2})的解就是
\begin{align}
    \frac{1}{g^{\prime2}(\mu)} &= \frac{1}{g^{\prime2}(M_{X})} +\frac{1}{2\uppi^{2}}
    \biggl(\frac{5n_{g}}{6}+\frac{n_{s}}{8}\biggr) \,\ln\biggl(\frac{M_{X}}{\mu}\biggr) \:,\label{28.2.11}\\
    \frac{1}{g^{2}(\mu)} &= \frac{1}{g^{2}(M_{X})} +\frac{1}{2\uppi^{2}}
    \biggl(-\frac{3}{2}+\frac{n_{g}}{2}+\frac{n_{s}}{8}\biggr) \,\ln\biggl(\frac{M_{X}}{\mu}\biggr) \:,\label{28.2.12}\\
    \frac{1}{g^{2}_{s}(\mu)} &= \frac{1}{g^{2}_{s}(M_{X})} +\frac{1}{2\uppi^{2}}
    \biggl(-\frac{9}{4}+\frac{n_{g}}{2}\biggr) \,\ln\biggl(\frac{M_{X}}{\mu}\biggr) \:,\label{28.2.13}
\end{align}
取$\,\mu=m_{Z}\,$是方便的, 这使得$\,SU(2)\times U(1)\,$在我们使用公式(28.2.11)---(\ref{28.2.13})的几乎所有能量范围内都可以认为是未破缺的. 使用方程(\ref{28.2.1}), 方程(\ref{28.2.12})和(\ref{28.2.13})的差给出
\begin{equation}
    \frac{1}{g^{2}(m_{Z})} - \frac{1}{g_{s}^{2}(m_{Z})} =\frac{1}{2\uppi^{2}}\biggl(\frac{3}{4}+\frac{n_{s}}{8}\biggr)\,\ln\biggl(\frac{M_{X}}{m_{Z}}\biggr) \:, \label{28.2.14}
\end{equation}
而方程(\ref{28.2.12})与\,3/5\,倍的方程(\ref{28.2.11})的差给出
\begin{equation}
    \frac{1}{g^{2}(m_{Z})} - \frac{3}{5g^{\prime2}(m_{Z})} = \frac{1}{2\uppi^{2}}\biggl(-\frac{3}{2}+\frac{n_{s}}{20}\biggr)\ln\biggl(\frac{M_{X}}{m_{Z}}\biggr)\:.
    \label{28.2.15}
\end{equation}
方程(\textcolor{foo}{21.3.19})使得我们可以用电弱混合角$\,\theta\,$和正电子荷$\,e\,$表示电弱耦合:
\begin{equation}
    g(m_{Z}) = -e(m_{Z})/\sin\theta \:, \qquad g^{\prime}(m_{Z})=-e(m_{Z})/\cos\theta \:. \label{28.2.16}
\end{equation}
这样我们就可以用输入参量$\,e(m_{Z})\,$和$\,g_{s}(m_{Z})\,$解出未知的$\,\ln(M_{X}/m_{Z})\,$以及$\,\sin^{2}\theta$:
\begin{equation}
    \sin^{2}\theta = \frac{18+3n_{s}+(e^{2}(m_{Z})/g_{s}^{2}(m_{Z}))(60-2n_{s})}{108+6n_{s}} \:, \label{28.2.17}
\end{equation}
\begin{equation}
    \ln\biggl(\frac{M_{X}}{m_{Z}}\biggr) = \Biggl(\frac{8\uppi^{2}}{e^{2}(m_{Z})}\Biggr)
    \Biggl(\frac{1-(8e^{2}(m_{Z})/3g_{s}^{2}(m_{Z}))}{18+n_{s}}\Biggr) \:. \label{28.2.18}
\end{equation}
当$\,n_{s}=0\,$时, 方程(\ref{28.2.17})给出的$\,\sin^{2}\theta\,$的结果与原先在非超对称理论中(忽略\,Higgs\,标量给出的小贡献)%
计算出的结果(\textcolor{foo}{21.5.15})相同, 而$\,\ln(M_{X}/m_{Z})\,$的值(\ref{28.2.18})是原始结果%
(\textcolor{foo}{21.5.16})的$\,11/9\,$倍, 而正如我们看到的, 这来自规范微子对$\,\beta\,$函数的贡献.


使用同\,21.5\,节相同的输入参量$\,e^{2}(m_{Z})/4\uppi=(128)^{-1}$, $g_{s}^{2}(m_{Z})/4\uppi=0.118$, %
$m_{Z}=91.19\,\mathrm{GeV}$, 表\ref{tab:28.1}给出了此时的数值结果. 就像上一节讨论过的, 电弱流中反常相消的需要要求%
$\,(H_{1}^{0},H_{1}^{-})\,$和\\$\,(H_{2}^{+},H_{2}^{0})\,$双重态的个数相等, 所以我们只考虑这些场的数目$\,n_{s}\,$是偶数的情况.
\begin{table}[t]
  \caption{就作为左手征超场双重态$\,(H_{1}^{0},H_{1}^{-})\,$或$\,H_{2}^{+},H_{2}^{0}\,$的数目$\,n_{s}\,$的函数而言, %
  方程(\ref{28.2.17})和(\ref{28.2.18})给出的电弱混合参量$\,\sin^{2}\theta\,$和统一质量$\,M_{X}\,$的值.}
  \label{tab:28.1}%
  \centering
  \begin{tabular}[c]{cll}
  \hline\hline
   $n_{s}$ & $\sin^{2}\theta$ & $M_{X}$\:($\mathrm{GeV}$) \\ \hline
   0 & $0.203$ & $8.7\times 10^{17} $ \\
   2 & $0.231$ & $2.2\times 10^{16} $ \\
   4 & $0.253$ & $1.1\times 10^{15} $ \\
  \hline\hline
  \end{tabular}
\end{table}

显著地, 最简单的合理理论的值$\,n_{s}=2\,$给出了$\,\sin^{2}\theta=0.231$,\cite{6} 这与实验上的观测值$\,\sin^{2}\theta =0.23\,$完美契合. $M_{X}\,$的值是以这种方法在非超对称理论中计算出来的值\,20\,倍大,\cite{7} 导致了$\,p\to\pi^{0}+e^{+}\,$这样的质子衰变过程的速率被减缓为原来的$\,20^{-4}$, 因此去除了与实验上未发现此类过程的矛盾. (质子衰变会在\,\ref{sec:28.7}\,节进行更细致的讨论.) $M_{X}\,$值的增长使得它更加接近引力与其它相互作用有相同强度的能量标度$\,\approx 10^{18}\,\mathrm{GeV}$. %
这个遗留的空隙可能被引力相互作用在极端高能时的变化所填补.\cite{7a}

$n_{s}=4\,$时给出的$\,\sin^{2}\theta\,$的值与实验严重不符, 而$\,M_{X}\,$的值又低到重新引起它与质子衰变期望值的矛盾. %
这强烈支持了每个超场$\,(H_{1}^{0},H_{1}^{-})\,$和$\,(H_{2}^{+},H_{2}^{0})\,$只有一个.

不像$\,\sin^{2}\theta\,$和$\,M_{X}\,$的计算值, 在$\,M_{X}\,$处的共有规范耦合(\ref{28.2.1})不依赖代的数目和标量双重态的个数. 当$\,n_{g}=3$, $n_{s}=2\,$且初入参量和前面相同时, 方程(\ref{28.2.13})给出
\begin{equation}
    \frac{g^{2}(M_{X})}{4\uppi} = \frac{g_{s}^{2}(M_{X})}{4\uppi} =\frac{1}{17.5} \:. \label{28.2.19}
\end{equation}


\section{超对称在哪里破缺?} \label{sec:28.3}

超对称如果有效的话, 那么它在已知粒子的选单中一定不明显, 所以任何将超对称性应用于普通能量处的考虑都要求我们对超对称破缺的机制做出某个假定. 最简单的假定是超对称性像 $SU(2)\times U(1)\,$那样被超对称标准模型树级近似下的效应所破缺. 这种可能可以确实地被排除掉.

一个反对超对称的树级近似破缺的论证基于质量求和规则(\ref{27.5.11}), 它对于未破缺守恒量色和电荷的每个值分别成立. %
在电荷为$\,-e/3\,$的色三重态中, 已知的费米子只有$\,d$, $s\,$和$\,b\,$夸克, 对于它们
\begin{equation}
    m_{d}^{2}+m_{s}^{2}+m_{b}^{2}\simeq (5\,\mathrm{GeV})^{2} \:. \label{28.3.1}
\end{equation}
根据求和规则, 如果没有其它费米子带有这个色和电荷, 那么所有带相同色和电荷的玻色子的质量平方和(对每个自旋态分别计数)必须%
约等于$\,2(5\,\mathrm{GeV})^{2}$. 特别地, 每个带有这个颜色和电荷的标量夸克的质量必须不超过$\,7\,\mathrm{GeV}$. 这么轻的标量夸克的存在性在实验上已经被排除了; 它们本应出现, 例如, 它们会贡献到电子-正电子湮灭成强子的速率中, 而这样的过程在它们会发生的能量处已经进行了彻底地研究.


如果存在一个重的第四代夸克, 这个讨论是不成立的. Dimopoulos\,和\,Georgi\cite{3}给出了另一个讨论, 无论有多少重夸克, 这个讨论也是成立的, 并且它甚至给最轻的标量夸克施加了一个更强的上界. 电荷和颜色守恒没有破缺告诉我们超对称标准模型中唯一非零%
的$\,D_{A0}\,$-项是$\,U(1)\,$生成元$\,y\,$和$\,SU(2)\,$生成元$\,t_{3}\,$的项, 我们分别称它们为$\,D_{1}\,$和$\,D_{2}$. %
这些生成元的值对于电荷为$\,2e/3\,$的左手夸克是$\,y=-g^{\prime}/6\,$和$\,t_{3}=+g/2$; 对于电荷为$\,-e/3\,$的左手夸克是%
$\,y=-g^{\prime}/6\,$和$\,t_{3}=-g/2$; 对于电荷为$\,2e/3\,$的右手夸克是$\,y=2g^{\prime}/3\,$和$\,t_{3}=0$; %
对于电荷为$\,-e/3\,$的右手夸克是$\,y=-g^{\prime}/3\,$和$\,t_{3}=0$. 另外, 标量夸克场是色三重态, 因而不能有真空期望值. %
根据方程(\ref{27.5.4}), 电荷为$\,2e/3\,$的色三重态(不是反三重态)标量夸克的质量平方矩阵是
\begin{equation}
    M_{0U}^{2} = \begin{bmatrix}
        \mathscr{M}_{U}^{\ast}\mathscr{M}_{U}-g^{\prime}D_{1}/6+gD_{2}/2 &
        \mathscr{F}_{U}^{\ast}\\[1em]
        \mathscr{F}_{U} &
        \mathscr{M}_{U}\mathscr{M}_{U}^{\ast}+2 g^{\prime}D_{1}/3
    \end{bmatrix} \:, \label{28.3.2}
\end{equation}
而电荷为$\,-e/3\,$的色三重标量夸克的质量平方矩阵是
\begin{equation}
    M_{0D}^{2} = \begin{bmatrix}
        \mathscr{M}_{D}^{\ast}\mathscr{M}_{D}-g^{\prime}D_{1}/6-gD_{2}/2 &
        \mathscr{F}_{D}^{\ast}\\[1em]
        \mathscr{F}_{D} &
        \mathscr{M}_{D}\mathscr{M}_{D}^{\ast}- g^{\prime}D_{1}/3
    \end{bmatrix} \:. \label{28.3.3}
\end{equation}
另外, 当没有与规范微子混合时, 电荷为$\,2e/3\,$和$\,-e/3\,$的夸克的质量平方矩阵在这里由方程(\ref{27.5.6}) 给出, %
分别就是$\,\mathscr{M}^{\ast}_{U}\mathscr{M}_{U}\,$和$\,\mathscr{M}^{\ast}_{D}\mathscr{M}_{D}$.

现在令$\,v_{u}\,$和$\,v_{d}\,$是夸克质量平方矩阵$\,\mathscr{M}_{U}^{\ast}\mathscr{M}_{U}\,$和%
$\,\mathscr{M}_{D}^{\ast}\mathscr{M}_{D}\,$的与质量最低夸克$\,u\,$和$\,d\,$对应的归一化本征矢, %
并考虑相应标量夸克质量平方矩阵的期望值
\begin{align}
    \begin{bmatrix}
        0 \\ v_{u}^{\ast}
    \end{bmatrix}^{\dag} M_{0U}^{2}
    \begin{bmatrix}
        0 \\ v_{u}^{\ast}
    \end{bmatrix}&= m_{u}^{2}+\frac{2g^{\prime}D_{1}}{3} \:, \label{28.3.4} \\
    \begin{bmatrix}
        0 \\ v_{d}^{\ast}
    \end{bmatrix}^{\dag} M_{0D}^{2}
    \begin{bmatrix}
        0 \\ v_{d}^{\ast}
    \end{bmatrix}&= m_{d}^{2}-\frac{g^{\prime}D_{1}}{3} \:. \label{28.3.5}
\end{align}
这些期望值分别是电荷为$\,2e/3\,$和$\,-e/3\,$的标量夸克的质量平方的加权平均, 所以电荷为$\,2e/3\,$的标量夸克中至少有一个质量平方必须小于$\,m_{u}^{2}+2g^{\prime}D_{1}/3$, 而电荷为$\,-e/3\,$的标量夸克中至少有一个质量平方必须小于%
$\,m_{d}^{2}-g^{\prime}D_{1}/3$. 因此, 取决于$\,D_{1}\,$的符号, {\kai{要么存在一个电荷为$\,2e/3\,$且比$\,u\,$夸克更轻的标量夸克, 要么存在一个电荷为$\,-e/3\,$且比$\,d\,$夸克更轻的标量夸克.}}

无需而言, 如果这样轻的带电荷色三重态标量存在, 那么它将会彻底改变强相互作用的唯象学. 同$\,u\,$和$\,d\,$夸克一样, %
这种带色标量将会作为强子的组成元素出现且其作为``组分''的质量有几百个$\,\mathrm{MeV}$, 而这一点当然没有看到. %
由于这个标量是带电荷的, 它也可以在能量在几百个$\,\mathrm{MeV}\,$以上的$\,e^{+}\,$--$\,e^{-}\,$湮灭中成对产生, %
而它对这个湮灭截面的贡献将会摧毁这一截面在理论和实验上的精确一致. 更糟的是, 由于$\,u\,$和$\,d\,$夸克太轻了, %
而$\,D_{1}\,$又被预期有超对称破缺标度的量级, 方程(\ref{28.3.4})和 (\ref{28.3.5})表明其中一个标量夸克的质量平方为负, %
这意味着这个标量夸克场的期望值不为零, 破坏了颜色和电荷守恒. 我们被迫要放弃这个超对称性在标准模型超对称版的树级近似下自发破缺的简单模型.

跳出这个结论的一种方法是给这个理论加上另外一个$\,U(1)\,$规范超场. %
如果所有夸克超场对这个新$\,U(1)\,$生成元均携带相同的值$\,\tilde{g}$, %
那么相应的$\,D\,$-项$\,\tilde{D}\,$会对方程(\ref{28.3.4})和(\ref{28.3.5})的右边有额外一个贡献$\,\tilde{g}\tilde{D}$. %
如果这一项足够大, 那么它会给所有标量夸克的质量平方赋予一个很大的正值, 规避掉了上面提及的所有问题. %
但是在可达到能量处还没有看到这种性中性规范玻色子的迹象, 并且无论如何对所有电荷为$\,-e/3\,$的标量夸克, %
我们仍有一个$\,7\,\mathrm{GeV}\,$的上界.

我们必须要在标准模型超对称版的树级近似以外寻找超对称破缺不一定是件坏事. %
如果超对称在这一近似下破缺, 那么设定超对称性破缺标度的表征质量将是拉格朗日量的某个质量参量, %
而这个质量参量反过来会给标准模型中的所有其它质量设定标度. 这样我们依旧面临等级问题: %
为什么这个质量标度远小于$\,10^{16}\,$--$\,10^{18}\,\mathrm{GeV}$?

有一个已知的方法来解释这么大的质量比值. 无论在某个很高的质量标度$\,M_{X}\,$统一所有相互作用的是何种场论, %
如果超对称在树级近似下没有自发破缺, 那么就像\,\ref{sec:27.6}\,节中证明的那样, 它也不会在微扰论的任意阶破缺. %
但是它可以被非微扰效应破缺. 特别地, 如果存在某个规范场在重整化标度$\,\mu\,$处有渐进自由的规范耦合$\,\mathscr{G}(\mu)$, %
并且如果$\,\mu\approx M_{X}\,$的$\,\mathscr{G}^{2}(\mu)/8\uppi^{2}\,$充分小于\,1, 那么就像\,18.3\,节中讨论的那样, %
这个规范相互作用在能量量级为$\,M_{S}=M_{X}\exp(-8\uppi^{2}b/\mathscr{G}^{2}(M_{X}))\,$处会变强, %
其中$\,b\,$是量级为\,1\,的数. 为了让$\,M_{S}\,$比$\,M_{X}\,$小好几个量级不一定非要把%
$\,\mathscr{G}^{2}(M_{X})/8\uppi^{2}\,$取得非常小. 我们将在\,\ref{sec:29.4}\,节看到, %
超对称性确实可以以这种方式被在某个能量$\,M_{S}\ll M_{X}\,$处变得很强的规范耦合破缺. %
诚然, 这就是在量子色动力学中发生在手征对称性上的事; 对于质子质量(或者至少它的主要部分, %
这部分来自手征对称性的动力学破缺而非$\,u\,$夸克和$\,d\,$夸克的微小质量)为什么远小于统一标度$\,M_{X}\,$并没有什么疑惑. %
换一种说法, 在能量$\,M_{S}\,$处很强的力会给标量场产生一个势, 而这个势的真空期望值会破缺超对称性.

没有任何迹象表明已知夸克和轻子有任何新的强相互作用, 所以我们不得不假定标准模型中观测到粒子相对于破缺超对称性的那个力是中性的. 因此超对称破缺发生在确实能感应到这些力的粒子的一个``隐藏区域(hidden sector)''. 这样一来剩下的问题就是, 在这个隐藏区域破缺的超对称性是通过什么机制与标准模型的已知粒子交互的? 正如我们将看到的, 我们对超对称性在唯象学上的应用绝大部分取决于这个问题的答案, 而不是超对称性破缺本身的细节.

当然, 超对称性破缺与观测到的粒子的交互必须是这些粒子感受到的某类相互作用. 有两个主要的候选者. %
一个机制由$\,SU(3)\times SU(2)\times U(1)\,$规范相互作用自身提供, 这将在\,\ref{sec:28.6}\,节进行讨论. %
另一个是引力, 而不是作为引力场超对称伙伴的辅助场, 这将在\,\ref{sec:31.4}\,和\,\ref{sec:31.7}\,节进行讨论.

不在这里进入细节, 我们可以对这两种可能性下的超对称性破缺标度$\,M_{S}\,$做一个粗略的估计. 对于规范传递的超对称性破缺, %
对于观测到的夸克, 轻子和规范玻色子, 取决于考虑的超超多重态是什么, 我们预期它们与它们的超对称伙伴的质量分裂是%
$\,g_{s}^{2}/16\uppi^{2}\,$或$\,g^{\prime2}/16\uppi^{2}\,$或$\,g^{2}/16\uppi^{2}\,$阶的(其中$\,g_{s}$, %
$g\,$和$\,g^{\prime}\,$是\,$SU(3)$, $SU(2)\,$和$\,U(1)\,$规范耦合). (这个猜测会在\,\ref{sec:28.6}\,节中证实.) %
因此, 如果标量夸克, 标量轻子和规范微子的质量像\,\ref{sec:28.1}\,节末尾论证的那样处在\,$100\,\mathrm{GeV}\,$到%
$\,10\,\mathrm{TeV}\,$的范围内, 那么超对称破缺的标度$\,M_{S}\,$要高两到三个数量级------例如是$\,100\,\mathrm{TeV}\,$阶的. %
另一方面, 如果引力充当了超对称性破缺的媒介, 那么基于量纲分析, 我们将期待已观测到的粒子与它们的超对称伙伴之间的能量分裂是%
$\,\sqrt{G}M_{S}^{2}\,$阶的, 亦或是$\,GM_{S}^{3}\,$阶的. (这两类结果均会在\,\ref{sec:31.7}\,节描述的模型中遇到.) 如果标量夸克, 标量轻子和规范微子的质量处在\,$100\,\mathrm{GeV}\,$到$\,10\,\mathrm{TeV}\,$的范围内, 那么$\,G_{S}\,$对于$\,\Delta m\approx \sqrt{G}M_{S}^{2}\,$将是$\,10^{11}\,\mathrm{GeV}\,$阶的, 对于$\,\Delta m\approx G M_{S}^{3}\,$将是%
$\,10^{13}\,\mathrm{GeV}\,$阶的.

超对称性破缺标度$\,M_{S}\,$的估计值在规范传递和引力传递的超对称性破缺之间的巨大差异对粒子唯象学和宇宙学有一个重要影响. %
正如数次提及的, 超对称性表明引力子必须有一个自旋$\,3/2\,$的伙伴, 引力微子. 当超对称性在标度$\,M_{S}\,$处破缺时, %
引力微子获得的质量$\,m_{g}\,$是$\,\sqrt{G}M_{S}^{2}\,$阶的. (\ref{sec:31.3} 节会给出一个精确公式.) 对于规范传递的超对称性破缺, 这个质量非常小; 如果$\,M_{S}\approx 100\,\mathrm{TeV}$, 那么$\,m_{g}\approx 1\,\mathrm{eV}$, 所以引力微子将是迄今为止超对称性要求的新粒子中最轻的那一个------即, 有负$\,R\,$宇称(\ref{28.1.6})的最轻粒子. 对于引力传递的超对称性破缺, 引力微子与$\,\sqrt{G}M_{S}^{2}\,$也即已知粒子与它们的超对称伴之间的质量分裂处在同一阶上, 所以引力微子的质量粗略与标量夸克, 标量轻子以及规范微子相等. 这样引力微子有可能是也有可能不是带有负\,$R$\,宇称的最轻粒子, 但它与已知粒子以及它们的超对称伴的相互作用在这一情况下是引力的强度, 这使得引力微子不会在基本粒子的实验中起到直接作用.


\subsection{* * *}

在能够从大爆炸中能够幸存下来的引力微子数目上有数个限制, 这些限制为超对称破缺的标度$\,M_{S}\,$设置了有用的约束. %
在遥远过去的某个时刻, 温度$\,T\,$被推测足够高以至于即使纯引力相互作用将会保持引力微子与其它粒子处在热平衡态, %
在这一情况下, 引力微子的数密度是$\,T^{3}\,$阶, 大约与光子的数密度相同. (我们使用的单位制中玻尔兹曼常数$\,k_{B}$, %
$\hbar\,$和$\,c\,$均等于\,1.) 如果引力微子不湮灭或不衰变, 那么宇宙膨胀将会用以相同的方式降低光子和引力微子的数密度, %
所以即使在引力微子离开平衡态后, 它们的呈现出的数目也与光子不相上下. 更准确些, 目前引力微子的数密度$\,n_{g0}\,$在量级上将比宇宙微波背景辐射中光子数密度$\,n_{\gamma0}\,$小一或两阶. 为了使引力微子的质量密度$\,m_{g}n_{g0}\,$不超过由\,Hubble\,常数的观测值设置的宇宙质量密度的上界, $m_{g}\,$将不得不必须约小于\cite{8}$\,1\,\mathrm{keV}$. 正如我们已经看到的, %
这个限制在规范传递的超对称破缺的理论中被很好地满足了, 那里的引力微子对于宇宙引力微子而言太轻以至于无法对宇宙数密度有可观的贡献. 为了使已知粒子显现出超对称破缺的效应, 在这些理论中破缺超对称的一些场必须与已知夸克, 轻子以及规范场至少有间接的相互作用, 所以引力微子与已知粒子以及它们的超对称伴的相互作用仅被规范耦合和\,Yukawa\,耦合常数的幂次压低, 进而使得夸克, 轻子以及规范玻色子的所有超对称伴快速地衰变到这些已知粒子和引力微子. 因此这些粒子也不会为宇宙学家为``缺失质量''寻找的模型中提供候选者粒子. %
(守恒律有可能使得超对称破缺区域的一些粒子稳定, 在这种情况下, 它们能够让人信服地充当缺失质量.)

另一方面, 对于引力传递的超对称性破缺, 引力微子足够重以至于是不稳定的(尽管引力微子湮灭依旧被忽略), %
所以上述限制不必成立.\cite{9} 我们会在\,\ref{sec:31.3}\,节看到, 引力微子与其它场的耦合正比于$\,\sqrt{G}$, %
所以基于量纲分析, 一个静止引力微子的衰变速率$\,\Gamma_{g}\,$粗略是$\,Gm_{g}^{3}\,$阶的. %
这要与宇宙膨胀的速率进行比较, 后者在温度为$\,T\,$时是$\,\sqrt{GT^{4}}\,$阶的. (这里我们忽略了量级为$\,10\,$--$\,100\,$的因子, 包括那些与非引力耦合常数以及粒子种类的数目相关的因子.) 当宇宙温度掉到$\,T\approx m_{g}\,$以下, 这时引力微子是非相对论性的, 它们的衰变速率与膨胀速率的比值在量级上是$\,\sqrt{G}m_{g}=m_{g}/m_{\text{Planck}}\ll1$, %
所以引力微子衰变变得重要是在这个时候之后, 这时引力微子是高度非相对论性的. 正如我们已经看到的, %
它们的数密度将是$\,T^{3}\,$阶的, 所以它们的能量密度是$\,m_{g}T^{3}\,$阶的, %
这远大于光子以及处在热平衡态的其它粒子在温度$\,T\,$时的$\,T^{4}\,$阶的能量密度, %
因而对控制宇宙膨胀速率的宇宙引力场给出主导贡献. 在这些条件下的膨胀速率因此是$\,\sqrt{Gm_{g}T^{3}}\,$阶的, %
引力微子衰变变得重要是在这等于$\,Gm_{g}^{3}\,$阶的引力微子衰变速率时, 因而就是在温度
\[
T_{g}\approx G^{1/3}m_{g}^{5/3} \:.
\]

正如我们已经看到的, 如果这些引力微子在此之前没有衰变, 那么它们的质量最好小于$\,1\,\mathrm{keV}$, %
但即使它们在此之前衰变了, 它们也会导致宇宙学上的困难. 在它们衰变之后, 它们的能量必须要进入光子和其它相对论性粒子的能量中, %
所以衰变之后的温度$\,T_{g}^{\prime}\,$与上面计算的温度$\,T_{g}\,$通过能量守恒条件%
$\,m_{g}T_{g}^{3}\approx T^{\prime4}_{g}\,$相关联, 因此
\[
T_{g}^{\prime}\approx G^{1/4}m_{g}^{3/2} \:.
\]
特别地, 由于$\,T_{g}\ll m_{g}$, 我们有$\,T_{g}^{\prime}\gg T_{g}$. 如果$\,T_{g}\,$小于宇宙核合成能够发生的温度%
$\,T_{n}\simeq 0.1\,\mathrm{MeV}$, 那么引力微子在核合成之前仍旧是丰富的, 给出了更高的能量密度以及随之而来的更快的膨胀, %
这使得自由中子在被并入复核之前衰变时间更少, 因而使得在核合成发生时会有更多的氦元素产生. 另外, %
光子和重子数密度之比也会随着引力微子衰变而上升, 所以这个比值在核合成时期要远小于通常从当前微波背景辐射温度估计出的值, %
所以核反应会更完整地把中子并入氦元素, 留给当下的氘元素会变少. 目前关于宇宙氦元素和氘元素丰度的理论和实验上的一致性因此会被摧毁. 如果$\,T_{g}>0.1\,\mathrm{MeV}$, 这个问题可以被避免, 但也可以在更弱的条件$\,T_{g}^{\prime}>0.4\,\mathrm{MeV}\,$下被避免, 这是因为, 这样一来, 在引力微子衰变之后, 温度将高到足以破坏多余的氦元素并重启宇宙核合成作为二次冷却. %
这个条件要求$\,m_{g}>10\,\mathrm{TeV}$, 而\,\ref{sec:28.1}\,节对已知夸克, 轻子以及规范玻色子的超对称伴的质量推导出的上界对于引力传递的超对称性破缺是$\,m_{g}\,$阶, 二者勉强一致. $m_{g}\,$上的限制对应的超对称破缺标度对于$\,m_{g}\approx \sqrt{G}M_{S}^{2}\,$是$\,M_{S}>10^{11}\,\mathrm{GeV}\,$或对于$\,m_{g}\approx G M_{S}^{3}\,$是%
$\,M_{S}>10^{13}\,\mathrm{GeV}$.


\section{最小超对称标准模型}  \label{sec:28.4}

在上一节, 我们找出两种不同在很高的能量标度$\,M_{S}\,$处破缺超对性且可以与已知夸克和轻子交互的方法: 通过规范超场或引力超场. %
相应低能有效拉格朗日量中的超对称破缺项就会被规范耦合或牛顿常数的幂次压低. 因此这些项中的大多数相对较小, 但有一个例外, %
连同规范耦合或牛顿常数因子, 基于量纲分析, 有效拉格朗日量中的质量项和其它超可重整项将正比于一个或多个超对称破缺标度因子%
$\,M_{S}$, 而与已知粒子质量相比这相当大. 由此我们得出, 在规范传递的超对称破缺一个恰当的近似以及对引力传递的超对称破缺一个非常好的近似下, 超对称破缺的主要效应将在超对称标准模型有效作用量的超可重整项中. 标准模型的这个版本\cite{10}中除了超可重整项外均是超对称的, 通常被称为{\kai{最小超对称标准模型}}.

当$\,R\,$宇称或$\,B-L\,$守恒时, $SU(3)\times SU(2)\times U(1)\,$规范对称性允许的最一般超可重整拉格朗日密度取如下的形式
\begin{align}
    \mathscr{L}_{SR} &= -\sum_{ij}M_{ij}^{2}{}^{Q}\Bigl(\mathscr{Q}_{i}^{\dag}\mathscr{Q}_{j}\Bigr)
    -\sum_{ij}M_{ij}^{2}{}^{\bar{U}}\Bigl(\bar{\mathscr{U}}_{i}^{\dag}\bar{\mathscr{U}}_{j}\Bigr)
     -\sum_{ij}M_{ij}^{2}{}^{\bar{D}}\Bigl(\bar{\mathscr{D}}_{i}^{\dag}\bar{\mathscr{D}}_{j}\Bigr) \nonumber \\
     &\quad -\sum_{ij}M_{ij}^{2}{}^{L}\Bigl(\mathscr{L}_{i}^{\dag}\mathscr{L}_{j}\Bigr)
     -\sum_{ij}M_{ij}^{2}{}^{\bar{E}}\Bigl(\bar{\mathscr{E}}_{i}^{\dag}\bar{\mathscr{E}}_{j}\Bigr) \nonumber \\
     &\quad -\Bigl(\overline{\lambda_{3}}\,m_{\text{gluino}}\,\lambda_{3}\Bigr)
     -\Bigl(\overline{\lambda_{2}}\,m_{\text{wino}}\,\lambda_{2}\Bigr)
     -\Bigl(\overline{\lambda_{1}}\,m_{\text{bino}}\,\lambda_{1}\Bigr) \nonumber \\
     &\quad-\sum_{ij}A_{ij}^{D}\,h_{ij}^{D}\Bigl(\mathscr{Q}_{i}^{\mathrm{T}}\,e\,\mathscr{H}_{1}\Bigr)\bar{\mathscr{D}}_{j}
     -\sum_{ij}A_{ij}^{E}\,h_{ij}^{E}\Bigl(\mathscr{L}_{i}^{\mathrm{T}}\,e\,\mathscr{H}_{1}\Bigr)\bar{\mathscr{E}}_{j} \nonumber \\
     &\quad -\sum_{ij}A_{ij}^{U}\,h_{ij}^{U}\Bigl(\mathscr{Q}_{i}^{\mathrm{T}}\,e\,\mathscr{H}_{2}\Bigr)\bar{\mathscr{U}}_{j}
     -\sum_{ij}C_{ij}^{D}\,h_{ij}^{D}\Bigl(\mathscr{Q}_{i}^{\mathrm{T}}\,\mathscr{H}_{2}^{\ast}\Bigr)\bar{\mathscr{D}}_{j}
     \nonumber\\
     &\quad -\sum_{ij}C_{ij}^{E}\,h_{ij}^{E}\Bigl(\mathscr{L}_{i}^{\mathrm{T}}\,\mathscr{H}_{2}^{\ast}\Bigr)\bar{\mathscr{E}}_{j}
     -\sum_{ij}C_{ij}^{U}\,h_{ij}^{U}\Bigl(\mathscr{Q}_{i}^{\mathrm{T}}\,\mathscr{H}_{1}^{\ast}\Bigr)\bar{\mathscr{U}}_{j} \nonumber\\
     &\quad-B\mu\Bigl(\mathscr{H}_{2}^{\mathrm{T}}\,e\,\mathscr{H}_{1}\Bigr)+\mathrm{H.c.} \:. \label{28.4.1}
\end{align}
花体字母在这里用来表明是左手征超场的标量分量. 对$\,SU(2)\,$和色指标求和是为了在$\,SU(3)\times SU(2)\times U(1)\,$下不变%
所必需的, 而$\,e\,$是通常的反对称$\,2\times2\,$矩阵$\,\mi\sigma_{2}$. 所有系数可以是复的, %
规范微子的质量可以包含正比于$\,\gamma_{5}\,$的项和正比于单位矩阵的项.

在这里我们沿用传统, 将包含标量场但不包含标量伴随场的项的系数写成方程(\ref{28.1.7})中相应超对称%
$\,\mathscr{F}\,$-项的系数乘以$\,A_{ij}^{D}$, $A_{ij}^{E}$, $A_{ij}^{U}\,$以及$\,B$. %
写成这样是基于如下的考虑: 方程(\ref{28.1.7})中轻夸克的\,Yukawa\,耦合很小反应了数个近似手征对称性, 而如果被推广至整个超对称态, 这也会使得方程(\ref{28.4.1})中相应的三线性项很小, 而方程(\ref{28.1.7})中出现的$\,\mu\,$-项违反了可能的%
\,Peccei--Quinn\cite{4}对称性, 如果这个对称性近似成立, 它会使得$\,\mu\,$和$\,B\mu\,$都很小. %
类似的考虑表明我们应该把既包含标量又包含它们复共轭的项的系数写成上面的形式. %
另外, 在\,\ref{sec:31.4}\,节我们将描述对$\,Ah\,$和$\,B\mu\,$的确实分别正比于$\,h\,$和$\,\mu\,$的贡献. %
然而在这里, 系数$\,Ah$, $Ch\,$和$\,B\mu\,$在方程(\ref{28.1.7})中的相应系数$\,h\,$和$\,\mu\,$很小时是否一定很小, %
我们将其留作一个开放问题.

在最小超对称标准模型的讨论中一般会省略掉方程(\ref{28.4.1})中的$\,Ch\,$-项. 这部分是因为, 正如在\,\ref{sec:27.7}节讨论过的, %
像这种既包含左手征标量超场的$\,\phi\,$-分量又包含它们的复共轭的项有可能会产生平方发散因而引起精细调节问题. %
但我们在\,\ref{sec:27.7}\,节看到, 平方发散仅出现在``蝌蚪''图中, 在这样的图中标量场线消失于真空中, %
而在最小超对称标准模型中没有在所有规范对称性都是中性的标量, 因而没有标量蝌蚪图. %
$Ch\,$-项在\,\ref{sec:31.6}\,节中讨论的引力传递的超对称破缺理论中被禁止了, 而在\,\ref{sec:28.6}\,节讨论的规范传递的超对称破缺理论中又很小, 但没有什么理由认为总将是这样的情况.

即使不是超对称的, 在\,\ref{sec:27.7}\,中证明了方程(\ref{28.4.1})中的那些超可重整相互作用不会对超对称 $,d=4\,$相互作用的系数产生破坏超对称的紫外发散修正. 因此, 附加在最小超对称标准模型的无量纲耦合上的超对称条件不会妨碍通过耦合常数的重整化来消除紫外发散. 正是这个性质, 而非任何在一个高能隐藏区域破缺超对称的理论, 促使引入了参考文献[10]中的最小超对称模型.

当前研究超对称标准模型的应用的最好理由是, 正如已经提到的, 超对称在某个很高的能量标度自发破缺的理论在能量很低时将自然地由最小超对称标准模型描述. 我们可以研究超对称标准模型的唯象应用, 并合理地相信无论超对称破缺的细致模型和它的媒介哪个是对的, %
结果总是相关的.

即使没有$\,Ch\,$-项, 如果拉格朗日量中的其它所有系数仅被规范对称性和$\,R\,$宇称守恒约束, %
最小超对称标准模型的自由参量也会有\,100\,多个.\cite{11} 这里的``最小''除了理论只包含最小份超场以外没有其它含义. %
有时, 基于某个底层理论或经验约束, ``最小超对称标准模型''这个术语是给还满足超可重整项系数上的约束的模型保留的. 例如,
有时乐观地假定最小超对称标准模型满足通用条件
\begin{align}
    &M_{ij}^{2}{}^{Q}=M_{ij}^{2}{}^{\bar{D}}=M_{ij}^{2}{}^{\bar{U}} =M_{ij}^{2}{}^{L}=M_{ij}^{2}{}^{\bar{E}}=M^{2}\delta_{ij}\:,\nonumber\\
    & m_{\text{gluino}} = m_{\text{wino}} = m_{\text{bino}} \:, \label{28.4.2} \\
    & A_{ij}^{D}=A_{ij}^{E}=A_{ij}^{U}=A\:, \qquad C_{ij}^{D}=C_{ij}^{E}=C_{ij}^{U}=0\:. \nonumber
\end{align}
这些条件通常被加在耦合常数统一的标度$\,M_{X}\approx 10^{16}\,\mathrm{GeV}\,$上, 而对低能的修正仅由重整化群流产生.
我们在这里不做这样的假定.

在分析最小超对称标准模型的唯象应用时, 我们必须不仅要对搜寻新粒子进行处理, 还要处理在含有已知粒子的过程上的两类严格经验约束:
各种味不守恒过程以及各种\,\textsf{CP}\,不守恒模型上的经验上界.

\subsection*{味改变过程}

我们在\,21.3\,节看到, 在非超对称标准模型中, 对$\,K^{0}$--$\bar{K}^{0}\,$振荡和$\,K^{0}\to\mu^{+}\mu^{-}\,$这样的味改变过程有一个自动的压低. 这是因为理论的如下特征, 使得夸克可以区分进而每个味分别守恒仅是夸克质量分裂, %
所以这些味改变过程的振幅必须正比于数个非常小的夸克质量因子. 另外, 轻子味在这个理论中自动守恒, %
这使得$\,\mu\to e\gamma\,$这样的过程被绝对禁止了. 这些令人满意的结果由于超对称标准模型中出现了标量夸克和标量轻子而被至于危险的境地, 这是因为没有什么理由期待标量夸克和标量轻子的质量矩阵在夸克和轻子的质量矩阵的基下是对角的. %
在这些粒子和与味无关的规范玻色子的相互作用中, 这不会引入味改变, 但是在标量夸克或标量轻子变成夸克或轻子并伴随规范微子的发射或吸收的过程中, 它会产生味改变跃迁. 当然, 如果标量夸克和标量轻子简并就没有问题, 这种情况下它们的质量矩阵在任何基下是对角的.

\begin{figure}[t]
  \centering
  \begin{tikzpicture}[scale=1.3]

  \draw[thick] (-4,1) -- (-2.2,1);
  \draw[thick, dashed] (-2.2,1) -- (-0.2,1);
  \draw[thick] (-0.2,1) -- (1.5,1);
  \draw[thick] (-4,-0.5) -- (-2.2,-0.5);
  \draw[thick, dashed] (-2.2,-0.5) -- (-0.2, -0.5);
  \draw[thick] (-0.2,-0.5) -- (1.5,-0.5);


  \draw (0.5,1.1) -- (0.7,1) -- (0.5,0.9);
  \draw (-1.3,1.1) -- (-1.1,1) -- (-1.3,0.9);
  \draw (-3.1,1.1) -- (-2.9,1) -- (-3.1,0.9);

  \draw (0.7,-0.4) -- (0.5,-0.5) -- (0.7,-0.6);
  \draw (-1.1,-0.4) -- (-1.3,-0.5) -- (-1.1,-0.6);
  \draw (-2.9,-0.4) -- (-3.1,-0.5) -- (-2.9,-0.6);

  \draw[thick,decorate, decoration={snake,amplitude=2pt,segment length=13pt}] (-2.2,1) -- (-2.2,-0.5);
  \draw[thick] (-2.2,1) -- (-2.2,-0.5);
  \draw[thick,decorate, decoration={snake,amplitude=2pt,segment length=13pt}] (-0.2,1) -- (-0.2,-0.5);
  \draw[thick] (-0.2,1) -- (-0.2,-0.5);

  \node at (-3.3,-0.8) {$s_L$};
  \node at (1.0 , -0.8) {$d_{L}$};
  \node at (1.0 , 1.3) {$s_{L}$};
  \node at (-3.3, 1.3) {$d_L$};
  \node at (-1.2, 1.3) {$\mathcal{D}_{1}\:\:\mathcal{D}_{2}\:\:\mathcal{D}_{3}$};
  \node at (-1.2, -0.8) {$\mathcal{D}_{1}\:\:\mathcal{D}_{2}\:\:\mathcal{D}_{3}$};
  \node at (-2.8, 0.25) {胶微子};
  \node at (0.5, 0.25) {胶微子};
  \end{tikzpicture}
  \vspace{5 mm}
  \caption{对超对称标准模型的$\,\Delta S=2\,$有效相互作用$\,(\overline{s_{L}}\gamma^{\mu}d_{L})(\overline{s_{L}}\gamma_{\mu}d_{L})\,$有贡献的单圈图. %
  这里的实线是夸克; 虚线是标量夸克; 实线和波浪线的组合线是胶微子.}%
  \label{fig:28.1}%
\end{figure}


标量夸克质量分裂和(或)混合角上的最严格限制由对$\,K^{0}$--$\bar{K}^{0}\,$跃迁的测量给出.\cite{12} %
这些转变由低能有效拉格朗日密度中$\,(\overline{s_{L}}\gamma^{\mu}d_{L})(\overline{s_{L}}\gamma_{\mu}d_{L})\,$这样的算符产生,
而这样的算符可以通过像图\,\ref{fig:28.1} 那样的图产生. 夸克$\,d_{L}\,$和$\,s_{L}\,$的超对称伴一般是质量确定的%
标量夸克$\,\mathscr{D}_{i}\,$的线性组合$\,\sum_{i}V_{di}\mathscr{D}_{i}\,$和 $\sum_{i}V_{si}\mathscr{D}_{i}$, %
其中$\,V_{ji}\,$是$\,3\times3\,$幺正矩阵, 所以这个图中的两个标量夸克传播子贡献了一个因子
\[
\sum_{i}\frac{V_{di}V_{si}^{\ast}}{k^{2}+M_{i}^{2}-\mi\epsilon}\times \sum_{j}\frac{V_{dj}V_{sj}^{\ast}}{k^{2}+M_{j}^{2}-\mi\epsilon}\:,
\]
其中$\,k\,$是圈中环流的\,4\,-动量. 因为$\,V_{ji}\,$是幺正的, 如果三个标量夸克质量$\,M_{i}\,$都相等, 这为零. %
如果标量夸克质量与某个共用值$\,M^{2}_{\text{squark}}\,$差一个相对小的量$\,\Delta M_{i}^{2}$, 那么这变成
\[
\Biggl(\frac{1}{k^{2}+M^{2}_{\text{squark}}-\mi\epsilon}\Biggr)^{4}
\Biggl(\sum_{i}V_{di}V_{si}^{\ast}\Delta M_{i}^{2}\Biggr)^{2} \:.
\]
$d_{L}\overline{s_{L}}\to s_{L}\overline{d_{L}}\,$的振幅拥有的量纲是$\text{质量}^{-2}$, %
所以乘以胶微子传播子以及\,4\,个强耦合因子$\,g_{s}$, 再对$\,k\,$积分之后, 我们得到的振幅必然正比于
\begin{equation}
    \frac{g_{s}^{4}}{\tilde{M}^{6}}\Biggl(\sum_{i}V_{di}V_{si}^{\ast}\Delta M_{i}^{2}\Biggr)^{2} \:, \label{28.4.3}
\end{equation}
其中的$\,\tilde{M}\,$是$\,M_{\text{squark}}\,$和$\,m_{\text{gluino}}\,$中较大的那一个. %
这可与这个振幅在非超对称标准模型中的结果进行比较, 后者由交换$\,W\,$产生, 如图\,\ref{fig:28.2}\,所示. %
忽略第三代夸克, 它到前两代夸克只有非常小的跃迁振幅, 因发射$\,W^{-}\,$产生的$\,d\to u$, $d\to c$, $s\to u\,$以及%
$\,s\to c\,$振幅分别是$\,\cos\theta_{c}$, $-\sin\theta_{c}$, $\sin\theta_{c}\,$和$\,cos\theta_{c}$, 其中$\,\theta_{c}\,$是\,21.3\,节中定义的\,Cabibbo\,角. 因此, 取代这里的标量夸克传播子, 我们有夸克传播子
\[
\sin\theta_{c}\cos\theta_{c}\biggl(\frac{\mi\slashed{k}+m_{u}}{k^{2}+m_{u}^{2}-\mi\epsilon}-
\frac{\mi\slashed{k}+m_{c}}{k^{2}+m_{c}^{2}-\mi\epsilon}\biggr) \:,
\]
取代强耦合$\,g_{s}$, 我们这里有$\,SU(2)\,$耦合$\,g$. 因此在非超对称标准模型中, $d_{L}\overline{s_{L}}\to s_{L}\overline{d_{L}}\,$的振幅正比于
\begin{equation}
    \frac{g^{4}\sin^{2}\theta_{c}\cos^{2}\theta_{c}}{m_{W}^{4}}\Bigl(m_{c}-m_{u}\Bigr)^{2} \:, \label{28.4.4}
\end{equation}
其比例系数与方程(\ref{28.4.3})中的同阶. 在对从$\,d_{L}\overline{s_{L}}\to s_{L}\overline{d_{L}}\,$的振幅如何计算出%
$\,K^{0}$--$\bar{K}^{0}\,$的跃迁振幅有一个合理的猜测后, 已经知道与图\,\ref{fig:28.2}\,对应的振幅会给出与实验吻合很好的结果.
(事实上, 在$\,c\,$夸克被发现之前, Gaillard\,和\,Lee\cite{13}利用这个计算预测了$\,m_{c}\approx 1.5\,\mathrm{GeV}$.) %
因此, 要求交换标量夸克的结果(\ref{28.4.3})要小于交换夸克的结果(\ref{28.4.4})看起来是合理的. 这给出条件
\begin{equation}
    \Biggl\lvert\sum_{i}V_{di}V_{si}^{\ast}\frac{\Delta M_{i}^{2}}{\tilde{M}^{2}}\Biggr\rvert <
    \frac{g^{2}\sin\theta_{c}\cos\theta_{c}}{g_{s}^{2}}\,\frac{(m_{c}-m_{u})\tilde{M}}{m_{W}^{2}} \:. \label{28.4.5}
\end{equation}
取$\,g^{2}/4\uppi=0.036$, $g_{s}^{2}/4\uppi=0.118$, $\sin\theta_{c}=0.22$, $m_{W}=80.4\,\mathrm{GeV}$, %
$m_{c}=1.5\,\mathrm{GeV}\,$以及$\,m_{u}\ll m_{c}$, 我们发现
\begin{equation}
    \Biggl\lvert\sum_{i}V_{di}V_{si}^{\ast}\frac{\Delta M_{i}^{2}}{\tilde{M}^{2}}\Biggr\rvert <
   1.5\times 10^{-3}\times (\tilde{M}/100\,\mathrm{GeV})  \:. \label{28.4.6}
\end{equation}
标量夸克的质量不太可能远小于$\,m_{\text{gluino}}$, 所以我们可以得出如下的结论, 要么标量夸克质量的分割没有超过$\,10^{3}\,$之一份, 要么混合矩阵$\,V_{ji}\,$的非对角项约小于$\,10^{-3}$, 要么标量夸克约重于$\,10\,\mathrm{TeV}$, 要么我们有%
近兼并标量夸克, 近零混合角和重标量夸克的某个线性组合. 就这个结果自身而言, %
它仅约束了电荷为$\,-e/3\,$的左手夸克的超对称伴$\,\mathscr{D}_{i}$, 但通过考虑$\,d_{R}\overline{s_{R}}\to s_{R}\overline{d_{R}}$, 我们可以对$\,\bar{\mathscr{D}}_{i}\,$标量夸克的质量和混合角得到类似的限制. %
通过考虑交换$\,W\,$微子而非胶微子产生的振幅, 我们可以对$\,\mathscr{U}_{i}\,$标量夸克的质量和混合角获得较弱一些的限制. %
然而, 应该注意的是, 这些讨论没有给不同电荷的标量夸克的质量之差给出任何限制, 也没有对左手夸克和反夸克的超对称版%
$\,\mathscr{Q}_{i}\,$和$\,\bar{\mathscr{Q}}_{i}\,$的质量之差给出任何限制.


\begin{figure}[t]
  \centering
  \begin{tikzpicture}[scale=1.3]

    \draw[thick] (-4,1) -- (-2.2,1);
    \draw[thick] (-2.2,1) -- (-0.2,1);
    \draw[thick] (-0.2,1) -- (1.5,1);
    \draw[thick] (-4,-0.5) -- (-2.2,-0.5);
    \draw[thick] (-2.2,-0.5) -- (-0.2, -0.5);
    \draw[thick] (-0.2,-0.5) -- (1.5,-0.5);

    \draw (0.5,1.1) -- (0.7,1) -- (0.5,0.9);
    \draw (-1.3,1.1) -- (-1.1,1) -- (-1.3,0.9);
    \draw (-3.1,1.1) -- (-2.9,1) -- (-3.1,0.9);

    \draw (0.7,-0.4) -- (0.5,-0.5) -- (0.7,-0.6);
    \draw (-1.1,-0.4) -- (-1.3,-0.5) -- (-1.1,-0.6);
    \draw (-2.9,-0.4) -- (-3.1,-0.5) -- (-2.9,-0.6);
    \draw[thick,decorate, decoration={snake,amplitude=1.5pt,segment length=13pt}] (-2.2,1) -- (-2.2,0.19);
    \draw[ thick, decorate, decoration={markings,mark=at position 1 with {\arrow[scale=2]{>}}},shorten >=0.3pt]
    (-2.2,1) -- (-2.2,0.19);
    %\draw[thick,decorate, decoration={snake,amplitude=1.5pt,segment length=13pt},->] (-2.2,1) -- (-2.2,0.19);
    \draw[thick,decorate, decoration={snake,amplitude=1.5pt,segment length=13pt}] (-2.2,0.2) -- (-2.2,-0.5);
    %\draw[thick] (-2.2,1) -- (-2.2,-0.5);

      \draw[thick,decorate, decoration={snake,amplitude=1.5pt,segment length=13pt}] (-0.2,1) -- (-0.2,0.19);
    \draw[ thick, decorate,
    decoration={snake,amplitude=1.5pt,segment length=13pt,markings,mark=at position 1 with {\arrow[scale=2]{>}}},
    shorten >=0.3pt
    ]
    (-0.2,1) -- (-0.2,0.19);
    %\draw[thick,decorate, decoration={snake,amplitude=1.5pt,segment length=13pt},->] (-2.2,1) -- (-2.2,0.19);
    \draw[thick,decorate, decoration={snake,amplitude=1.5pt,segment length=13pt}] (-0.2,0.2) -- (-0.2,-0.5);
   % \draw[thick,decorate, decoration={snake,amplitude=1.5pt,segment length=13pt},->] (-0.2,1) -- (-0.2,0.19);

    %\draw[thick,decorate, decoration={snake,amplitude=1.5pt,segment length=13pt}] (-0.2,0.2) -- (-0.2,-0.5);
    %\draw[thick] (-0.2,1) -- (-0.2,-0.5);

    %\draw (-2.41,0.30) -- (-2.28,0.18) -- ( -2.1,0.30);

    \node at (-3.3,-0.8) {$s_L$};
    \node at (1.0 , -0.8) {$d_{L}$};
    \node at (1.0 , 1.3) {$s_{L}$};
    \node at (-3.3, 1.3) {$d_L$};
    \node at (-1.2, 1.3) {$u_{L}\:\:c_{L}\:\:d_{L}$};
    \node at (-1.2, -0.8) {$u_{L}\:\:c_{L}\:\:d_{L}$};
    \node at (-2.8, 0.25) {$W^{-}$};
    \node at (0.5, 0.25) {$W^{+}$};


    \end{tikzpicture}
  \vspace{5 mm}
  \caption{对超对称和非超对称标准模型中的$\,\Delta S=2\,$有效相互作用$\,(\overline{s_{L}}\gamma^{\mu}d_{L})(\overline{s_{L}}\gamma_{\mu}d_{L})\,$均有贡献的单圈图. 这里的实线是夸克, 波浪线是$\,W^{\pm}\,$玻色子.}%
  \label{fig:28.2}%
\end{figure}


和标量夸克的情况一样, 有确定质量的标量轻子被期待是轻子超对称伴的非对角线性组合. %
这会通过图\,\ref{fig:28.3}\,这样的图给出衰变过程$\,\mu\to e+\gamma$. %
实验在这一过程的分支比上给出的上界$\,4.9\times 10^{-11}\,$在电荷相同, 代数不同且混合角为一般值的标量轻子的分数质量劈裂, 或者不简并标量轻子的混合角上设置了约为$\,10^{-3}\,$的限制.\cite{14}

曾经有过用连接不同代的规范对称性解释标量夸克和标量轻子的兼并性的尝试.\cite{14a} %
在\,\ref{sec:28.6} 节, 我们将描述一种超对称破缺的途径, 在这种途径中不需要附加这种对称性, 简并性也会出现.


\subsection*{\textsf{CP}\,破坏}

关于已知粒子的实验信息所提供的第二类重要约束与$\,\mathsf{CP}\,$-破坏效应有关, 例如中子和电子的电偶极矩.\cite{15} %
我们在\,21.3\,节看到, 除去\,23.6\,节讨论过的一个量子色动力学的参量$\,\theta\,$引起的问题, %
这些效应在只有一个标量双重态的非对称标准模型中相当弱. 这是因为, 如果只有两代夸克和轻子, 那么夸克和轻子的质量矩阵以及它们与规范玻色子的相互作用中的所有$\,\mathsf{CP}\,$-破坏相位可以被吸收进夸克和轻子场的定义中, 而尽管有第三代, %
它与前两代的混合(由于未知的原因)相当弱. (这个论证不适用于直接包含第三代夸克的过程, 例如$\,B^{0}$--$\bar{B}^{0}\,$混合, %
这将在计划中的``B\,工厂''中测量.) 因此在标准模型中的这个简单的非超对称版本中, 中子的电偶极矩被预期\cite{16}要约小于%
$\,10^{-30}\,e\:\mathrm{cm}$, 远小于实验给出的上界, $6.3\times 10^{-26}\,e\:\mathrm{cm}$.\cite{16a}

\begin{figure}[t]
  \centering
  \begin{tikzpicture}[scale=1.3]
  \draw[thick] (-3,1) --   (-2,1) node [below=8pt,left=-4pt] {$\mu$};
  \draw[thick, decorate, decoration={markings,mark=at position 1 with {\arrow[scale=1.5]{>}}},shorten >=0.2pt] (-3,1) --(-2,1);
  \draw[thick] (-2.2,1) -- (-1,1);
  \draw[thick,dashed] (-1,1) node [below=7pt,right] {$\mathcal{E}_{i}\:\:\mathcal{L}_{i}^{-}$} -- (0.6,1) ;
  \draw[thick, decorate, decoration={markings,mark=at position 1 with {\arrow[scale=1.5]{>}}},shorten >=0.2pt] (-1,1)--  (0.6,1);
  \draw[thick,dashed] (0.6,1) -- (2,1) ;
  \draw[thick,] (2,1)--  (3,1) node [below=8pt,left=-4pt] {$e$};
  \draw[thick, decorate, decoration={markings,mark=at position 1 with {\arrow[scale=1.5]{>}}},shorten >=0.2pt] (2,1)--  (3,1);
  \draw[thick] (2.9,1) -- (4,1);

  %\draw[thick] (-1,1) to [bend left=50] (2,1);
   \draw[thick] (-1,1) .. controls (0,2) and (1,2) ..  node [above] {bino,$\text{wino}^{0}$}(2,1);
  \draw[thick,decorate, decoration={snake}] (-1,1) .. controls (0,2) and (1,1.8) .. (2,1);
  %\draw[thick,decorate, decoration={snake}] (-1,1) to [bend left=50] (1.96,1);
  \draw[thick,decorate, decoration={snake,amplitude=1pt}] (1.2,1) -- (3.0,-0.3) node [right,above] {$\gamma$};
  \end{tikzpicture}

  \vspace{10mm}

  \begin{tikzpicture}[scale=1.3]
  \draw[thick] (-3,1) --   (-2,1) node [below=8pt,left=-4pt] {$\mu$};
  \draw[thick, decorate, decoration={markings,mark=at position 1 with {\arrow[scale=1.5]{>}}},shorten >=0.2pt] (-3,1) --(-2,1);
  \draw[thick] (-2.2,1) -- (-1,1);
  \draw[thick,dashed] (-1,1) node [below=7pt,right] {$\mathcal{N}_{i}\:\:\mathcal{L}_{i}^{0}$} -- (0.6,1) ;
  \draw[thick, decorate, decoration={markings,mark=at position 1 with {\arrow[scale=1.5]{>}}},shorten >=0.2pt] (-1,1)--  (0.6,1);

  \draw[thick,dashed] (0.6,1) -- (2,1) ;
  \draw[thick] (2,1)--  (3,1) node [below=8pt,left=-4pt] {$e$};
  \draw[thick, decorate, decoration={markings,mark=at position 1 with {\arrow[scale=1.5]{>}}},shorten >=0.2pt] (2,1)--  (3,1);
  \draw[thick] (2.9,1) -- (4,1);

  %\draw[thick] (-1,1) to [bend left=50] (2,1);
   \draw[thick] (-1,1) .. controls (0,2) and (1,2) ..  node [above] {$\text{wino}^{-}$}(2,1);
  \draw[thick,decorate, decoration={snake}] (-1,1) .. controls (0,2) and (1,1.8) .. (2,1);
  %\draw[thick,decorate, decoration={snake}] (-1,1) to [bend left=50] (1.96,1);
  \draw[thick,decorate, decoration={snake,amplitude=1pt}] (1.4,1.45) -- (3.0,2.3) node [right,below] {$\gamma$};

  \end{tikzpicture}
  \caption{过程$\,\mu\to e+\gamma\,$的单圈图. 这里的实线是轻子; 虚线是标量轻子; 实线和波浪线的组合线是规范微子; 波浪线是光子.}%
  \label{fig:28.3}%
\end{figure}


与之相反, 最一般形式的最小超对称标准模型的\,100\,多个参量包含数十个$\,\mathsf{CP}\,$-破坏相对相位. %
在积掉已知粒子的超对称伴后, 这些相位产生了数个要加在标准模型拉格朗日量上的$\,\mathsf{CP}\,$-破坏有效相互作用. %
基于量纲分析, 量纲最小的那些可能是最有用的, 其中包括夸克和轻子的电偶极矩,\cite{17} %
类似的对胶子和夸克相互作用的``色电(chromoelectric)''偶极矩贡献,\cite{18} %
一个$\,\mathsf{CP}\,$-破坏的纯胶子相互作用,\cite{19} %
以及最轻\,Higgs\,标量与轻子的一个$\,\mathsf{CP}\,$-破坏相互作用.\cite{20}

举一个粒子, 考虑夸克的色电偶极矩, 在一些模型中, 这对中子的电偶极矩贡献最大. $\,\mathsf{CP}\,$-破坏的色电偶极矩算符是%
$\,(\bar{q}\gamma_{5}[\gamma_{\mu},\gamma_{\nu}]\lambda_{a}q)f_{a}^{\mu\nu}\,$(其中$\,q\,$是一个$\,u\,$或$\,d\,$色三重态夸克场, $f^{\mu\nu}_{a}\,$是$\,SU(3)\,$场强张量, $\lambda_{a}\,$是$\,SU(3)\,$的$\,3\times 3\,$生成元). %
由于$\,\gamma_{5}[\gamma_{\mu},\gamma_{\nu}]\,$仅在$\,\bar{q}_{L}\,$和$\,q_{R}\,$或$\,\bar{q}_{R}\,$和$\,q_{L}\,$之间有矩阵元, 为了使单圈图贡献到色电偶极矩中, 一个外左手$\,u\,$或$\,d\,$夸克线必须发射一个内胶微子线并转变成一个$\,\mathscr{U}\,$%
或$\,\mathscr{D}\,$标量夸克线, 接下来转变成一个$\,\bar{\mathscr{U}}^{\ast}\,$或$\,\bar{\mathscr{D}}^{\ast}\,$标量夸克线, %
然后再通过吸收内胶微子线转变成一个右手$\,u\,$或$\,d\,$夸克, 其中外胶子线要接到内胶微子线或一个内标量夸克线上. %
(见图\,\ref{fig:28.4}.)


为了计算这个, 我们需要知道左手征夸克超场$\,Q_{i}\,$的标量分量$\,\mathscr{U}_{i}\,$(或$\,\mathscr{D}_{i}\,$)与左手征反夸克超场 $\bar{U}_{j}\,$(或$\,\bar{D}_{j}\,$)的标量分量的复共轭%
$\,\bar{\mathscr{U}}_{j}^{\ast}\,$(或$\,\bar{\mathscr{D}}_{j}^{\ast}\,$)的混合, 这由$\,SU(2)\times U(1)\,$的自发破缺产生, %
在图\,\ref{fig:28.4} 中用\,X\,表示. 这个矩阵的一部分来源于超对称$\,\mathscr{F}$-项相互作用(\ref{28.1.7})对方程%
(\ref{26.4.7})最后一项的贡献:
\begin{equation}
    \mathscr{L}_{\mathscr{Q}\bar{\mathscr{Q}}\mathscr{H}} =
    -\Biggl\lvert \sum_{ij}h_{ij}^{U}\mathscr{U}_{i}\bar{\mathscr{U}}_{j}+\mu\mathscr{H}_{1}^{0}\Biggr\rvert^{2}
    -\Biggl\lvert \sum_{ij}h_{ij}^{D}\mathscr{D}_{i}\bar{\mathscr{D}}_{j}+\mu\mathscr{H}_{2}^{0}\Biggr\rvert^{2} \:. \label{28.4.7}
\end{equation}
还有一个贡献来自于方程(\ref{28.4.1})中的$\,A$-项和$\,C$-项:
\begin{align}
    \mathscr{L}_{\mathscr{Q}\bar{\mathscr{Q}}\mathscr{H}}^{\prime} &=
    -\sum_{ij}h_{ij}^{D}\mathscr{D}_{i}\bar{\mathscr{D}}_{j}\Bigl[-A_{ij}^{D}\mathscr{H}_{1}^{0}+C_{ij}^{D}\mathscr{H}_{2}^{0\ast}\Bigr]
    \nonumber \\
    &\quad  -\sum_{ij}h_{ij}^{U}\mathscr{U}_{i}\bar{\mathscr{U}}_{j}\Bigl[A_{ij}^{U}\mathscr{H}_{2}^{0}+C_{ij}^{U}\mathscr{H}_{1}^{0\ast}\Bigr] - \text{H.c.} \label{28.4.8}
\end{align}
将中性\,Higgs\,标量场换成它们的期望值给出了二次项
\begin{align}
    \mathscr{L}_{\mathscr{Q}\bar{\mathscr{Q}}} &= -2\operatorname{Re}\sum_{ij}m_{ij}^{U}\mathscr{U}_{i}\bar{\mathscr{U}}_{j}
    \Bigl(\mu^{\ast}\cot\beta +A_{ij}^{U}+C_{ij}^{U}\cot\beta\Bigr) \nonumber \\
    &\quad -2\operatorname{Re}\sum_{ij}m_{ij}^{D}\mathscr{D}_{i}\bar{\mathscr{D}}_{j}
    \Bigl(\mu^{\ast}(\tan\beta)^{\ast} +A_{ij}^{D}-C_{ij}^{D}(\tan\beta)^{\ast}\Bigr) \:, \label{28.4.9}
\end{align}
其中$\,m_{ij}^{U}=\langle\mathscr{H}_{2}^{0} \rangle h_{ij}^{U}\,$和$\,m_{ij}^{D}=-\langle \mathscr{H}_{1}^{0}\rangle h_{ij}^{D}\,$是电荷为$\,2e/3\,$和$\,-e/3\,$的夸克的质量矩阵, 以及
\begin{equation}
    \tan\beta \equiv \langle\mathscr{H}_{2}^{0}\rangle /\langle \mathscr{H}_{1}^{0}\rangle^{\ast} \:. \label{28.4.10}
\end{equation}


\newcommand{\Cross}{$\mathbin{\tikz [x=1.4ex,y=1.4ex,line width=.3ex] \draw (0,0) -- (1,1) (0,1) -- (1,0);}$}
\begin{figure}[t]
  \centering
  \begin{tikzpicture}[scale=1.3]
  \draw[thick] (-3,1) --   (-2,1) node [below] {$u_{L},d_{L}$};
  \draw[thick, decorate, decoration={markings,mark=at position 1 with {\arrow[scale=1.5]{>}}},shorten >=0.2pt] (-3,1) --(-2,1);
  \draw[thick] (-2.2,1) -- (-1,1);
  \draw[thick,dashed] (-1,1) node [below=7pt,right] {$\mathcal{U}\:\:\mathcal{D}$} -- (2,1) node [below=7pt,left] {$\overline{\mathcal{U}}\:\:\overline{\mathcal{D}}$} ;
  %\draw[thick,dashed] (0.6,1) -- (2,1) ;
  \draw[thick] (2,1)--  (3,1) node [below] {$u_{R},d_{R}$};
   \draw[thick, decorate, decoration={markings,mark=at position 1 with {\arrow[scale=1.5]{>}}},shorten >=0.2pt] (2,1) --(3,1);
  \draw[thick] (2.9,1) -- (4,1);

  %\draw[thick] (-1,1) to [bend left=50] (2,1);
   \draw[thick] (-1,1) .. controls (0,2) and (1,2) ..  node [above] {$\text{wino}^{-}$}(2,1);
  \draw[thick,decorate, decoration={snake}] (-1,1) .. controls (0,2) and (1,1.8) .. (2,1);
  %\draw[thick,decorate, decoration={snake}] (-1,1) to [bend left=50] (1.96,1);
  \draw[thick,decorate, decoration={snake,amplitude=1pt}] (1.4,1.45) -- (3.0,2.3) node [right=3pt,below] {gluon};


  \node at (0.5,1) {\Cross};
  \node at (-1.3, 1.5) {gluino};
  \node at (2.3, 1.5) {gluino};
  \end{tikzpicture}
  
  \vspace{5mm}
  
  \caption{$u\,$夸克或$\,d\,$夸克的色电偶极矩的单圈图. 这里实线是夸克; 虚线是标量夸克; 实线和波浪线的组合线是胶微子; 波浪线是胶子. %
  这里\,X\,代表插入一个双线性相互作用, 它来自伴随$\,SU(2)\times U(1)\,$自发破缺而来的三线性标量相互作用. 还有另外一个胶子线连在内部标量夸克线而非胶微子线的图.}%
  \label{fig:28.4}%
\end{figure}

忽略掉\,Cabibbo\,混合, 明确起见, 取$\,A\,$和$\,C\,$是对角的, 图\,\ref{fig:28.4}\,对$\,u\,$和$\,d\,$夸克的色电偶极矩的贡献是如下的形式
\begin{align}
    d_{u}^{ce} &= \frac{g_{s}^{3}}{16\uppi^{2}}\operatorname{Im}\Bigl[m_{u}\,A_{u}^{\prime}\,I(m_{\mathscr{U}},m_{\bar{\mathscr{U}}},m_{\text{gluino}})\Bigr] \:, \label{28.4.11} \\
  d_{d}^{ce} &= \frac{g_{s}^{3}}{16\uppi^{2}}\operatorname{Im}\Bigl[m_{d}\,A_{d}^{\prime}\,I(m_{\mathscr{D}},m_{\bar{\mathscr{D}}},m_{\text{gluino}})\Bigr] \:, \label{28.4.12}
\end{align}
其中
\begin{equation}
    A_{u}^{\prime}\equiv (\mu^{\ast}+C_{u})\cot\beta+A_{u} \:, \qquad
    A_{d}^{\prime}\equiv (\mu^{\ast}-C_{d})(\tan\beta)^{\ast}+A_{d} \:, \label{28.4.13}
\end{equation}
而$\,I\,$是其变量的一个复杂的无量纲函数, 来源于对虚\,4\,-动量的积分. %
当$\,m_{\mathscr{Q}}\simeq m_{\bar{\mathscr{Q}}}\,$以及胶微子场的定义使得$\,m_{\text{gluion}}\,$和$\,m_{\mathscr{Q}}\,$都是实的,  函数$\,I\,$取如下的形式
\begin{equation}
    I(m_{\mathscr{Q}},m_{\mathscr{Q}},m_{\text{gluino}}) = m_{\text{gluino}}^{-3}\,
    J\Biggl(\frac{m_{\text{gluino}}^{2}}{m_{\mathscr{Q}}^{2}-m_{\text{gluino}}^{2}}\Biggr) \:, \label{28.4.14}
\end{equation}
其中\cite{21}
\begin{equation}
    J(z)=2\biggl(-z^{4}+\frac{4}{3}z^{3}+z^{2}\biggr)\ln\biggl(\frac{1+z}{z}\biggr)+2z^{3}-\frac{11}{3}z^{2} \:. \label{28.4.15}
\end{equation}

这类计算的困难部分总是在于估计一个像色电偶极相互作用的算符对中子电偶极矩这样的强子矩阵元的贡献. %
由于是在中子质量而非标量夸克和胶微子质量这个能量量级使用这个算符, 所以我们预期将会要求重整化群的修正. %
更重要的是量纲因子和$\,4\uppi\,$因子要搞对. 对于这个目的, 通常会使用被称为``朴素量纲分析''的计数规则.\cite{22} %
有$\,V_{i}\,$个$\,i\,$类顶点和$\,I\,$条内线的连通图有$\,L=I-\sum_{i}V_{i}+1\,$个圈. 如果有$\,N_{i}\,$条线与一个$\,i\,$类顶点相连而整个图有$\,N\,$个外线, 那么$\,2I+N=\sum_{i}V_{i}N_{i}$, 所以
\[
L=1-\frac{N}{2}+\sum_{i}V_{i}\biggl(\frac{N_{i}}{2}-1\biggr) \:.
\]
我们对每个圈预期有一个$\,1/16\uppi^{2}\,$阶的因子, 所以在低能有效拉格朗日量中一个有$\,N\,$个场因子的算符$\,\mathcal{O}\,$的系数会包含一个总因子
\[
(4\uppi)^{N-2} \prod_{i}(4\uppi)^{(2-N_{i})V_{i}} \:.
\]
如果算符$\,\mathcal{O}\,$的量纲为$\,d\,$而$\,i\,$类相互作用$\,\mathcal{O}_{i}\,$的量纲为$\,d_{i}$, %
那么$\,\mathcal{O}\,$的系数有量纲$\,4-d-\sum_{i}(4-d_{i})$, 所以这个系数也会有一个因子$\,M^{4-d}\prod_{i}M^{d_{i}-4}$, %
其中$\,M\,$是表征强子物理的某个标度, 例如核子质量或能量$\,2\uppi F_{\pi}\simeq 1200\,\mathrm{MeV}$, \,19.5\,节讨论的低能展开在这里开始失效. 最后, 某个图对$\,\mathcal{O}\,$的系数的贡献显然正比于这个图中顶点所对应的所有算符$\,\mathcal{O}_{i}\,$的耦合. 这些评述可以总结为定义一个``约化耦合'': 对于有$\,N_{i}\,$个场因子, 量纲为$\,d_{i}\,$且耦合为$\,g_{i}\,$的任何算符$\,\mathcal{O}_{i}$, 与它对应的约化耦合是:
\begin{equation}
    g_{i}^{\mathrm{reduced}} \equiv g_{i}(4\uppi)^{2-N_{i}}M^{\mathscr{D}_{i}-4} \:. \label{28.4.16}
\end{equation}
上述估计给出了朴素量纲分析的规则: 在有效拉格朗日量中, 任何算符$\,\mathcal{O}\,$的约化耦合粗略等于贡献到这个有效耦合的相互作用的约化耦合之积.


中子电偶极矩是有一个光子场和两个中子场且量纲为$\,5\,$的一个算符的系数, 所以它的约化耦合是$\,Md_{n}^{e}/4\uppi$. %
类似的, 夸克色电偶极矩的约化耦合是$\,Md_{q}^{ce}/4\uppi$. 除了这个约化耦合因子外, 中子电偶极算符的约化耦合必须有一个电磁耦合的约化耦合$\,e/4\uppi\,$以及数量待定的约化强耦合因子 $g_{4}/4\uppi$, 这些因子在低能标度$\,M\,$处与\,1\,相差不大因而会被忽略掉. 取$\,d\,$夸克的贡献为$\,u\,$和$\,d\,$夸克的总贡献的代表性贡献, 那么结果是
\begin{equation}
    d_{n}^{e}\approx \frac{e\,d_{d}^{ce}}{4\uppi}\approx e\,\biggl(\frac{g_{s}}{4\uppi}\biggr)^{3}
    \operatorname{Im}\Bigl[m_{d}\,A_{d}^{\prime}\Bigr] I(m_{\mathscr{D}},m_{\bar{\mathscr{D}}},m_{\text{gluino}}) \:.
    \label{28.4.17}
\end{equation}
通过设$\,m_{\text{gluino}}\simeq m_{\mathscr{D}}\simeq m_{\bar{\mathscr{D}}}\,$, 这使得$\,J=7/18$, 并在标量夸克和胶微子质量标量处取$\,g_{s}^{2}/4\uppi\,$的值为在$\,m_{Z}\,$处的值$\,0.12$, 以及$\,\lvert m_{d}\rvert\approx 7\,\mathrm{MeV}\,$来进一步简化, 我们就有
\begin{equation}
    \lvert d_{n}^{e}\rvert \approx 0.5\times 10^{-23}\,e\,\mathrm{cm}\,\frac{\lvert A_{d}^{\prime}\rvert\,\lvert\sin\varphi\rvert\times (100\,\mathrm{GeV})^{2}}{m_{\text{gluino}}^{3}} \:, \label{28.4.18}
\end{equation}
其中$\,\varphi\,$是$\,A_{d}^{\prime}\,$的相位, 所取的约定是胶微子, 夸克和标量夸克质量均取为实的. 夸克的电偶极矩贡献要大一些, 而纯胶子性的$\,\mathsf{CP}\,$-奇算符的贡献要小得多.\cite{23}


为了避免与实验上界$\,0.97\times 10^{-25}\,e\,\mathrm{cm}\,$相矛盾, 要么超对称标准模型中与$\,\mathsf{CP}\,$破坏相联系的相位必须要约小于$\,10^{-2}$, 要么这个模型中的一些新粒子必须要重于$\,1\,\mathrm{TeV}$. 通过计算原子和分子的电偶极矩可以得到类似的结论.\cite{23} 通过考虑图\,\ref{fig:28.1}\,对精确测量的那个$\,\mathsf{CP}\,$-破坏效应------$\,K^{0}$--$\bar{K}^{0}\,$振荡振幅的虚部------的贡献, 对$\,\mathsf{CP}\,$-破坏相位更严格的限制条件也被推导出来了.\cite{24}


\section{零轻子数和零重子数的区域} \label{sec:28.5}

尽管超对称标准模型的参量很多, 但在某些情形下它的预测性惊人的好. 特别是当我们考虑真空期望值自发破缺$\,SU(2)\times U(1)\,$规范对称性的标量场时. 在这一节, 我们将考虑这些标量连同其它重子数和轻子数为零的场: 在荷共轭下为奇的中性标量, 带荷标量, 这些标量以及$\,W^{\pm}\,$和$\,Z^{0}\,$的超对称伴.

对于标量模型的超对称版而言, 为了解释电磁和弱相互作用的$\,SU(2)\times U(1)\,$规范群破缺, 一个关键的要求是它们应该包含质量和相互作用参量正确地标量双重态``Higgs''超场. 我们在\,\ref{sec:28.1}\,节看到, 为了给电荷为$\,2e/3\,$和$\,-e/3\,$的夸克以及带电轻子赋予质量, 至少需要两个左手征标量双重态, 而我们在\,\ref{sec:28.2}\,发现两个双重态正是把$\,SU(3)$, $SU(2)\,$和$\,U(1)\,$规范耦合在某个很高的能量汇合在一起所需要的. 因此我们已经假定存在两个左手征标量$\,SU(2)\,$双重态
\begin{equation}
    H_{1}=\begin{pmatrix}
    H_{1}^{0} \\ H_{1}^{-}
    \end{pmatrix} \:, \qquad
    H_{2}=\begin{pmatrix}
    H_{2}^{+} \\ H_{2}^{0}
    \end{pmatrix} \:, \label{28.5.1}
\end{equation}
它们有由(\ref{27.4.1})给出的$\,SU(2)\,$和$\,U(1)\,D\,$-项(假定\,Fayet--Iliopoulos\,常数为零)
\begin{align}
    \vec{D}&=\frac{g}{2}\Bigl(\mathscr{H}_{1}^{\dag}\,\cvec{\tau}\,\mathscr{H}_{1}\Bigr)
    +\frac{g}{2}\Bigl(\mathscr{H}_{2}^{\dag}\,\cvec{\tau}\,\mathscr{H}_{2}\Bigr) \:, \label{28.5.2}\\
    D_{y}&=\frac{g^{\prime}}{2}\Bigl(\mathscr{H}_{1}^{\dag}\mathscr{H}_{1}\Bigr)
    -\frac{g^{\prime}}{2}\Bigl(\mathscr{H}_{2}^{\dag}\mathscr{H}_{2}\Bigr)  \:, \label{28.5.3}
\end{align}
其中$\,\mathscr{H}_{1,2}\,$是标量双重态$\,H_{1,2}\,$的标量分量而$\,\tau_{r}\,$是满足$\,\tau_{r}^{2}=1\,$的\,Pauli\,矩阵. %
正如方程(\ref{27.4.9})中说明的那样, 在可重整理论中, 这给出了标量场势的$\,D\,$-项贡献
\begin{align}
    V_{D} &=\frac{1}{2}\vec{D}^{2} + \frac{1}{2}D_{y}^{2} \nonumber \\
    &=\frac{g^{2}}{8}\Biggl[\Bigl(\mathscr{H}_{1}^{\dag}\,\cvec{\tau}\,\mathscr{H}_{1}\Bigr)
    +\Bigl(\mathscr{H}_{2}^{\dag}\,\cvec{\tau}\,\mathscr{H}_{2}\Bigr)\Biggr]^{2}
    +\frac{g^{\prime2}}{8}\Biggl[\Bigl(\mathscr{H}_{1}^{\dag}\mathscr{H}_{1}\Bigr)
    -\Bigl(\mathscr{H}_{2}^{\dag}\mathscr{H}_{2}\Bigr) \Biggr]^{2} \:. \label{28.5.4}
\end{align}
通过使用关系
\begin{equation}
    (\cvec{\tau})_{i\ell}\cdot (\cvec{\tau})_{kj}=2\delta_{ij}\delta_{k\ell}-\delta_{i\ell}\delta_{kj} \:, \label{28.5.5}
\end{equation}
这可以被变成更方便的形式. (为了证明这点, 用有理不变性证明$\,\delta_{ij}\delta_{k\ell}\,$可以表示成%
$\,(\cvec{\tau})_{i\ell}\cdot(\cvec{\tau})_{kj}\,$和 $\delta_{i\ell}\delta_{kj}\,$的线性组合, %
并通过对指标$\,i,j\,$和$\,k,\ell\,$取迹来计算系数.) 以这种方法, 我们可以将标量场势的\,$D$\,-项部分重写为
\begin{equation}
    V_{D}=\frac{g^{2}}{2}\Biggl\lvert\Bigl(\mathscr{H}_{1}^{\dag}\mathscr{H}_{2}\Bigr)\Biggr\rvert^{2} +
    \frac{g^{2}+g^{\prime2}}{8}\Biggl[\Bigl(\mathscr{H}_{1}^{\dag}\mathscr{H}_{1}\Bigr)
    -\Bigl(\mathscr{H}_{2}^{\dag}\mathscr{H}_{2}\Bigr)\Biggr]^{2} \:. \label{28.5.6}
\end{equation}

正如\,\ref{sec:28.1}\,节所提及的, 这两个左手征双重态的超势中只有一种可能的可重整项, 形如
\begin{equation}
    f(H_{1},H_{2})=\mu\Bigl(H_{1}^{\mathrm{T}}eH_{2}\Bigr) \:, \label{28.5.7}
\end{equation}
其中$\,\mu\,$是有质量量纲的常数而$\,e\,$是反对称矩阵$\,\mi\tau_{2}$. 根据方程(\ref{27.4.9}), 这给标量场势一个额外的贡献
\begin{align}
    V_{\mu} &= \sum_{r}\biggl\lvert \frac{\partial f(\mathscr{H}_{1},\mathscr{H}_{2})}{\partial\mathscr{H}_{1r}}\biggr\rvert^{2}
    +\sum_{r} \biggl\lvert \frac{\partial f(\mathscr{H}_{1},\mathscr{H}_{2})}{\partial\mathscr{H}_{2r}}\biggr\rvert^{2} \nonumber \\
    &=\lvert\mu\rvert^{2} \Biggl[\Bigl(\mathscr{H}_{1}^{\dag}\mathscr{H}_{1}\Bigr)
    +\Bigl(\mathscr{H}_{2}^{\dag}\mathscr{H}_{2}\Bigr)\Biggr] \:. \label{28.5.8}
\end{align}
当$\,\mu\neq 0\,$时, 势$\,V_{D}+V_{\mu}\,$显然有一个为零的最小值, 仅在$\,\mathscr{H}_{1}=\mathscr{H}_{2}=0\,$这个点处达到. %
当势中仅有这些项时, $SU(2)\times U(1)\,$和超对称是不自发破缺的. ($\mu=0\,$的情况并没有好多少; 这时有无限多个连续的超对称不破缺而$\,SU(2)\times U(1)\,$破缺至电磁规范不变性且有所有可能强度的真空态, 其中场强有可能为零.) 和在\,\ref{sec:28.3}\,节中看到的一样, 这是另一个体现在标准模型下构造超对称自发破缺的相对论性理论的一般性困难的例子.

在上一节超对称在有效拉格朗日量中仅被超可重整项破缺的假定下, 这种包含标量双重态的超对称破缺项的最一般形式是
\[
V_{m}=m_{1}^{2} \Bigl(\mathscr{H}_{1}^{\dag}\mathscr{H}_{1}\Bigr)
     +m_{2}^{2} \Bigl(\mathscr{H}_{2}^{\dag}\mathscr{H}_{2}\Bigr)
     +\operatorname{Re}\Bigl\{B\mu\Bigl(\mathscr{H}_{1}^{\mathrm{T}}e\mathscr{H}_{2}\Bigr)\Bigr\} \:,
\]
其中$\,m_{1}^{2}\,$和$\,m_{2}^{2}\,$是实参量(不一定为正)而$\,B\mu\,$是一个相位任意的参量. 我们将调整超场$\,H_{1}\,$和$\,H_{2}\,$的{\kai{总}}相位使得$\,B\mu\,$是正实的, 这样
\begin{equation}
    V_{m}=m_{1}^{2} \Bigl(\mathscr{H}_{1}^{\dag}\mathscr{H}_{1}\Bigr)
     +m_{2}^{2} \Bigl(\mathscr{H}_{2}^{\dag}\mathscr{H}_{2}\Bigr)
     + B\mu\operatorname{Re}\Bigl(\mathscr{H}_{1}^{\mathrm{T}}e\mathscr{H}_{2}\Bigr)\:. \label{28.5.9}
\end{equation}
那么总的标量势在树级近似下是
\begin{align}
    V &= V_{D}+V_{\mu}+V_{m} \nonumber \\
      &=\frac{g^{2}}{2}\Biggl\lvert\Bigl(\mathscr{H}_{1}^{\dag}\mathscr{H}_{2}\Bigr)\Biggr\rvert^{2} +
    \frac{g^{2}+g^{\prime2}}{8}\Biggl[\Bigl(\mathscr{H}_{1}^{\dag}\mathscr{H}_{1}\Bigr)
    -\Bigl(\mathscr{H}_{2}^{\dag}\mathscr{H}_{2}\Bigr)\Biggr]^{2}\nonumber \\
     &\quad +(m_{1}^{2}+\lvert\mu\rvert^{2}) \Bigl(\mathscr{H}_{1}^{\dag}\mathscr{H}_{1}\Bigr)
     +(m_{2}^{2}+\lvert\mu\rvert^{2}) \Bigl(\mathscr{H}_{2}^{\dag}\mathscr{H}_{2}\Bigr) \nonumber \\
      &\quad + B\mu\operatorname{Re}\Bigl(\mathscr{H}_{1}^{\mathrm{T}}e\mathscr{H}_{2}\Bigr) \: \label{28.5.10}
\end{align}
特别注意到, $\mu^{2}$, $m_{1}^{2}\,$和$\,m_{2}^{2}\,$仅出现$\,m_{1}^{2}+\lvert\mu\rvert^{2}\,$和%
$\,m_{2}^{2}+\lvert\mu\rvert^{2}\,$的组合中.

从势有下界这个条件出发可以推导出超对称破缺参量$\,m_{i}^{2}\,$上的一个条件. 当标量场从一般方向趋于无穷时, %
势由四次项$\,V_{D}\,$主导. 存在$\,V_{D}\,$为零的特殊方向: 在这些方向上(相差一个$\,SU(2)\times U(1)\,$规范变换)
\[
    \mathscr{H}_{1}=\begin{pmatrix}
    \phi \\ 0
    \end{pmatrix} \:, \qquad
    \mathscr{H}_{2}=\begin{pmatrix}
      0 \\ \phi
    \end{pmatrix} \:,
\]
其中$\,\phi\,$是一个任意的复量. 对于这样的方向, $V=(2\lvert\mu\rvert^{2}+m_{1}^{2}+m_{2}^{2})\lvert\phi\vert^{2}-B\mu\phi^{2}$, 所以(由于$\,B\mu\,$已经被定义成正的)为了使其在$\,\phi\to+\infty\,$ 时不趋于$\,-\infty$, 必有
\begin{equation}
    2\lvert\mu\rvert^{2} + m_{1}^{2} + m_{2}^{2} \geq B\mu \:. \label{28.5.11}
\end{equation}

我们希望在电磁规范不变性没有破缺的地方找到势的最小值, 所以让我们来考虑带荷标量场被设为零时, 势作为中性标量场的函数的行为. 在这一情况下, 方程(\ref{28.5.10})给出了中性标量的势
\begin{align}
    V^{\mathrm{N}} &=\frac{g^{2}+g^{\prime2}}{8} \Bigl[\bigl\lvert \mathscr{H}_{1}^{0}\rvert^{2}-
   \lvert\mathscr{H}_{2}^{0}\bigr\rvert^{2}\Bigr]^{2}
    + (m_{1}^{2}+\lvert\mu\rvert^{2})\,\bigl\lvert\mathscr{H}_{1}^{0}\bigr\rvert^{2} \nonumber \\
    &\quad +  (m_{2}^{2}+\lvert\mu\rvert^{2})\,\bigl\lvert \mathscr{H}_{2}^{0}\bigr\rvert^{2}
    -B\mu\operatorname{Re}\Bigl(\mathscr{H}_{1}^{0}\mathscr{H}_{2}^{0}\Bigr) \:. \label{28.5.12}
\end{align}
为了找到稳相点, 我们在常数值$\,\mathscr{H}_{i}^{0}=v_{i}\,$附近展开$\,V^{\mathrm{N}}$, 写成
\begin{equation}
    \mathscr{H}^{0}_{i} = v_{i} +\varphi_{i} \:. \label{28.5.13}
\end{equation}
到$\,\varphi_{i}\,$的第二阶, 方程(\ref{28.5.12})给出
\begin{align}
    V^{\mathrm{N}}_{\mathrm{quad}} &= \frac{g^{2}+g^{\prime2}}{4}(\lvert v_{1}\rvert^{2}-\lvert v_{2}\rvert^{2})
    \Bigl[2\operatorname{Re}(v_{1}^{\ast}\varphi_{1}-v_{2}^{\ast}\varphi_{2})+\lvert\varphi_{1}\rvert^{2}
    -\lvert \varphi_{2}\rvert^{2}\Bigr] \nonumber \\
    &\quad +\frac{g^{2}+g^{\prime2}}{2}
    \Bigl[\operatorname{Re}(v_{1}^{\ast}\varphi_{1}-v_{2}^{\ast}\varphi_{2})\Bigr]^{2}
    +(m_{1}^{2}+\lvert\mu\rvert^{2})\,\Bigl(2\operatorname{Re}v_{1}^{\ast}\varphi_{1}
    +\lvert\varphi_{1}\rvert^{2}\Bigr) \nonumber \\
    &\quad+(m_{2}^{2}+\lvert\mu\rvert^{2})\,\Bigl(2\operatorname{Re}v_{2}^{\ast}\varphi_{2}
     +\lvert\varphi_{2}\rvert^{2}\Bigr)-B\mu\operatorname{Re}
     \Bigl(v_{1}\varphi_{2}+v_{2}\varphi_{1}+\varphi_{1}\varphi_{2}\Bigr) \nonumber \\
    &\quad +\text{常数}  \:. \label{28.5.14}
\end{align}
为了使$\,v_{i}\,$是场的平衡态值, $\varphi_{i}\,$的一阶项必须为零
\begin{align}
    \Bigl(m_{1}^{2}+\lvert\mu\rvert^{2}\Bigr) v_{1}^{\ast} + \frac{g^{2}+g^{\prime2}}{4}
    \Bigl(\lvert v_{1}\rvert^{2}-\lvert v_{2}\rvert^{2}\Bigr) v_{1}^{\ast} - \frac{1}{2}B\mu \,v_{2} &=0 \:, \label{28.5.15} \\
    \Bigl(m_{2}^{2}+\lvert\mu\rvert^{2}\Bigr) v_{2}^{\ast} + \frac{g^{2}+g^{\prime2}}{4}
    \Bigl(\lvert v_{2}\rvert^{2}-\lvert v_{1}\rvert^{2}\Bigr) v_{2}^{\ast} - \frac{1}{2}B\mu\, v_{1} &=0 \:. \label{28.5.16}
\end{align}
无需改变$\,\varphi_{i}\,$的总相位, 我们可以调整它们的相对相位使得$\,v_{1}\,$是实的. 那么方程(\ref{28.5.15})和(\ref{28.5.16})\\%
表明$\,v_{2}\,$也是实的, 这使得这些方程变成
\begin{align}
    \Bigl(m_{1}^{2}+\lvert\mu\rvert^{2}\Bigr) v_{1} + \frac{g^{2}+g^{\prime2}}{4}
    \Bigl(\lvert v_{1}\rvert^{2}-\lvert v_{2}\rvert^{2}\Bigr) v_{1} - \frac{1}{2}B\mu \,v_{2} &=0 \:, \label{28.5.17} \\
    \Bigl(m_{2}^{2}+\lvert\mu\rvert^{2}\Bigr) v_{2} + \frac{g^{2}+g^{\prime2}}{4}
    \Bigl(\lvert v_{2}\rvert^{2}-\lvert v_{1}\rvert^{2}\Bigr) v_{2} - \frac{1}{2}B\mu\, v_{1} &=0 \:. \label{28.5.18}
\end{align}
我们可以用这些条件把势中的质量参量表示方便的量
\begin{equation}
    \tan \beta \equiv v_{2}/v_{1} \:, \label{28.5.19}
\end{equation}
\begin{equation}
    m_{Z}^{2} = \tfrac{1}{2}(g^{2}+g^{\prime2})\,(v_{1}^{2}+v_{2}^{2}) \:, \label{28.5.20}
\end{equation}
以及
\begin{equation}
    m_{A}^{2}\equiv 2\lvert\mu\rvert^{2} + m_{1}^{2} + m_{2}^{2} \:. \label{28.5.21}
\end{equation}
(参量$\,m_{Z}\,$是$\,Z\,$矢量玻色子的质量.\footnote{方程(\ref{28.5.20})和(\textcolor{foo}{21.3.30})给出的$\,m_{Z}^{2}\,$的公式差一个因子\,2, 这是因为这里对标量场做的归一化和\,\textcolor{foo}{21.3}\,节中的不同.} 我们不久就会看到$\,m_{A}\,$ 是其中一个物理标量的质量.) 给方程(\ref{28.5.17})和(\ref{28.5.18})分别乘以$\,v_{2}\,$和$\,v_{1}$, 并取和与差, 给出
\begin{equation}
    B\mu = m_{A}^{2}\sin 2\beta \:, \label{28.5.22}
\end{equation}
和
\begin{equation}
    m_{1}^{2}-m_{2}^{2} = -(m_{A}^{2}+m_{Z}^{2})\cos 2\beta \:, \label{28.5.23}
\end{equation}
这与方程(\ref{28.5.21})一起给出
\begin{equation}
    m_{1}^{2}+\lvert\mu\rvert^{2} = \tfrac{1}{2}m_{A}^{2}-\tfrac{1}{2}(m_{A}^{2}+m_{Z}^{2})\cos 2\beta \:,\qquad
    m_{2}^{2}+\lvert\mu\rvert^{2} = \tfrac{1}{2}m_{A}^{2}+\tfrac{1}{2}(m_{A}^{2}+m_{Z}^{2})\cos 2\beta \:.    \label{28.5.24}
\end{equation}
由于线性项相互抵消, 中性标量势的二次部分(\ref{28.5.14})可以写成
\begin{align}
    V^{\mathrm{N}}_{\mathrm{quad}} &= \frac{g^{2}+g^{\prime2}}{4} (v_{1}^{2}-v_{2}^{2})
    \Bigl[\lvert\varphi_{1}\rvert^{2}-\lvert\varphi_{2}\rvert^{2}\Bigr]
    +\frac{g^{2}+g^{\prime2}}{2} \Bigl[\operatorname{Re}(v_{1}\varphi_{1}-v_{2}\varphi_{2})\Bigr]^{2}\nonumber\\
    &\quad +(m_{1}^{2}+\lvert\mu\rvert^{2})\lvert\varphi_{1}\rvert^{2}
    +(m_{2}^{2}+\lvert\mu\rvert^{2})\lvert\varphi_{2}\rvert^{2}
    -B\mu\operatorname{Re}\Bigl(\varphi_{1}\varphi_{2}\Bigr) + \text{常数} \nonumber \\
    &= \tfrac{1}{2}m_{Z}^{2}\cos 2\beta\Bigl[\lvert\varphi_{1}\rvert^{2}-\lvert\varphi_{2}\rvert^{2}\Bigr]
    +m_{Z}^{2}\Bigl[\operatorname{Re}(\cos\beta\varphi_{1}-\sin\beta\varphi_{2})\Bigr]^{2} \nonumber\\
    &\quad+\tfrac{1}{2}m_{A}^{2}\Bigl(\lvert\varphi_{1}\rvert^{2}+\lvert\varphi_{2}\rvert^{2}\Bigr)
    -\tfrac{1}{2}(m_{A}^{2}+m_{Z}^{2})\cos2\beta
    \Bigl[\lvert\varphi_{1}\rvert^{2}-\lvert\varphi_{2}\rvert^{2}\Bigr] \nonumber \\
    &\quad-m_{A}^{2}\sin2\beta\operatorname{Re}\Bigl(\varphi_{1}\varphi_{2}\Bigr)+\text{常数} \:.\label{28.5.25}
\end{align}

我们从方程(\ref{28.5.25})中看到$\,\varphi_{i}\,$的实部和虚部退耦了. (这是因为势(\ref{28.5.12})在荷共轭或$\,\mathsf{CP}\,$变换$\,\varphi_{i}\to\varphi_{i}^{\ast}\,$下不变.) $\varphi_{i}\,$虚部的质量平方矩阵是
\begin{equation}
    M_{\operatorname{Im}\varphi}^{2} = \begin{pmatrix}
        \tfrac{1}{2} m_{A}^{2}(1-\cos2\beta) &
        \tfrac{1}{2} m_{A}^{2}\sin 2\beta \\[1em]
        \tfrac{1}{2} m_{A}^{2}\sin 2\beta &
        \tfrac{1}{2} m_{A}^{2}(1+\cos2\beta)
    \end{pmatrix} \:. \label{28.5.26}
\end{equation}
行列式为零, 所以一个本征值为零, 而另一个等于迹, 正是$\,m_{A}^{2}$. 零质量标量当然是与$\,SU(2)\times U(1)\,$到电磁规范不变性的自发破缺相关的中性\,Goldstone\,玻色子, 而正如第\,\textcolor{foo}{21}\,章中所讨论的, 它被\,Higgs\,机制消除了. 正如所承诺的, $m_{A}\,$是物理标量中那个$\,\mathsf{C}\,$为负的非\,Goldstone\,玻色子的质量. 这表明, %
为了使场值$\,\varphi_{i}=v_{i}\,$至少是势能的一个局域最小值点, 由方程(\ref{28.5.21})定义的参量$\,m_{A}^{2}\,$必须是正的. %
使得势在大场强处有好行为的条件(\ref{28.5.11})表明方程(\ref{28.5.22})在这里对于在$\,0\leq\beta\leq\uppi/2\,$这个范围内的%
$\,\beta\,$有一个解.

特别地, 如果$\,B\mu=0\,$且$\,0<\beta<\uppi/2$, 那么方程(\ref{28.5.22})表明$\,m_{A}=0$. 在这一情况下, %
粒子$\,A\,$是势(\ref{28.5.12})在$\,\mathscr{H}_{1}^{0}\,$和$\,\mathscr{H}_{2}^{0}\,$的等相位变化下的$\,U(1)\,$Peccei-Quinn\, 对称性\cite{4}的\,Goldstone\,玻色子, 它在$\,v_{1}\neq0\,$和$\,v_{2}\neq0\,$时会自发破缺到使得这个对称性和电弱$\,U(1)\,$对称性的任何组合都不是不破缺的. 这是轴子的原始版本,\cite{25} 而正如我们在\,\textcolor{foo}{23.6}\,节看到的, %
它仅从标量与夸克的\,Yukawa\,相互作用获得一个小质量并在实验上被排除了. 因此我们可以得出结论: $B\mu\,$绝对不为零.

对实标量, 方程(\ref{28.5.25})给出的质量平方矩阵是
\begin{align}
    (M_{\operatorname{Re}\varphi}^{2})_{11} &= \tfrac{1}{2}m_{A}^{2}(1-\cos 2\beta)+\tfrac{1}{2}m_{Z}^{2}(1+\cos2\beta)\:, \nonumber \\
    (M_{\operatorname{Re}\varphi}^{2})_{12} &= (M_{\operatorname{Re}\varphi}^{2})_{21} =
    -\tfrac{1}{2}(m_{A}^{2}+m_{Z}^{2})\sin 2\beta \:, \label{28.5.27} \\
    (M_{\operatorname{Re}\varphi}^{2})_{22} &= \tfrac{1}{2}m_{A}^{2}(1+\cos 2\beta)
    +\tfrac{1}{2}m_{Z}^{2}(1-\cos2\beta) \:. \nonumber
\end{align}
解久期方程给出本征值
\begin{align}
    m_{H}^{2} &= \frac{1}{2} \Bigl[m_{A}^{2}+m_{Z}^{2}
    +\sqrt{(m_{A}^{2}+m_{Z}^{2})^{2}-4m_{A}^{2}m_{Z}^{2}\cos^{2}2\beta}\Bigr] \:,\label{28.5.28} \\
    m_{h}^{2} &= \frac{1}{2} \Bigl[m_{A}^{2}+m_{Z}^{2}
    -\sqrt{(m_{A}^{2}+m_{Z}^{2})^{2}-4m_{A}^{2}m_{Z}^{2}\cos^{2}2\beta}\Bigr] \:.\label{28.5.29}
\end{align}

为了计算带荷标量的质量, 设中性标量为它们的真空值:
\begin{equation}
    \mathscr{H}_{1}=\begin{pmatrix}
    v_{1} \\  \mathscr{H}_{1}^{-}
    \end{pmatrix} \:, \qquad
    \mathscr{H}_{2}=\begin{pmatrix}
    \mathscr{H}_{2}^{+} \\ v_{2}
    \end{pmatrix} \:, \label{28.5.30}
\end{equation}
我们来计算此时的势$\,V$. 在方程(\ref{28.5.10})使用上式就给出了带荷标量势的二次部分
\begin{align}
    V_{\mathrm{quad}}^{\mathrm{C}} &= \frac{g^{2}}{2}\Bigl\lvert v_{2}(\mathscr{H}_{1}^{-})^{\ast}
    +v_{1}\mathscr{H}_{2}^{2}\Bigr\rvert^{2} + \frac{g^{2}+g^{\prime2}}{4} (v_{1}^{2}-v_{2}^{2})
    \Bigl(\lvert\mathscr{H}_{1}^{-}\rvert^{2}-\lvert\mathscr{H}_{2}^{+}\rvert^{2}\Bigr) \nonumber \\
    &\quad +(m_{1}^{2}+\lvert\mu\rvert^{2})\lvert \mathscr{H}_{1}^{-}\rvert^{2}
    +(m_{2}^{2}+\lvert\mu\rvert^{2})\lvert \mathscr{H}_{2}^{+}\rvert^{2}
    +B\mu\mathscr{H}_{1}^{-}\mathscr{H}_{2}^{+} \:. \label{28.5.31}
\end{align}
利用方程(\ref{28.5.22})和(\ref{28.5.24}), 这可以写成
\begin{align}
    V_{\mathrm{quad}}^{\mathrm{C}} &= \frac{1}{2}(m_{W}^{2}+m_{A}^{2})
    \Bigl[\lvert\mathscr{H}_{1}^{-}\rvert^{2}(1-\cos 2\beta)+\lvert\mathscr{H}_{2}^{+}\rvert^{2}
    (1+\cos 2\beta)\nonumber \\
    &\quad +2\sin2\beta\mathscr{H}_{1}^{-}\mathscr{H}_{2}^{+}\Bigr] \:, \label{28.5.32}
\end{align}
其中$\,m_{W}\,$是带荷规范玻色子的质量:
\begin{equation}
    m_{W}^{2} = \frac{1}{2}g^{2}\Bigl(\lvert v_{1}\rvert^{2}+\lvert v_{2}\rvert^{2}\Bigr)\:.\label{28.5.33}
\end{equation}
带荷标量的质量矩阵则是
\begin{equation}
    M_{C}^{2} = \frac{1}{2}(m_{W}^{2}+m_{A}^{2})
    \begin{pmatrix}
    1-\cos 2\beta & \sin 2\beta \\
    \sin 2\beta & 1+\cos 2\beta
    \end{pmatrix} \:. \label{28.5.34}
\end{equation}
它的行列式为零, 所以它的有一个本征值等于零而另一个本征值等于迹
\begin{equation}
    m_{C}^{2} = m_{W}^{2} + m_{A}^{2} \:. \label{28.5.35}
\end{equation}
零质量带荷标量当然是与$\,SU(2)\times U(1)\,$自发破缺相关的另一个\,Goldstone\,玻色子, 同之前发现的中性\,Goldstone\,玻色子一样, 它被\,Higgs\,机制消除了.

即使不知道参量$\,m_{A}\,$和$\,\beta$, 这些结果就标量玻色子质量的相对大小告诉了我们很多信息. 我们可以将方程(\ref{28.5.28})和(\ref{28.5.29})重写成如下形式
\begin{align}
    m_{H}^{2} &= \frac{1}{2} \Bigl[m_{A}^{2}+m_{Z}^{2}
    +\sqrt{(m_{A}^{2}-m_{Z}^{2})^{2}+4m_{A}^{2}m_{Z}^{2}\sin^{2}2\beta}\Bigr] \:,\label{28.5.36} \\
    m_{h}^{2} &= \frac{1}{2} \Bigl[m_{A}^{2}+m_{Z}^{2}
    -\sqrt{(m_{A}^{2}-m_{Z}^{2})^{2}+4m_{A}^{2}m_{Z}^{2}\sin^{2}2\beta}\Bigr] \:.\label{28.5.37}
\end{align}
我们看到{\kai{较重的中性标量质量$\,m_{H}\,$要大于$\,m_{Z}\,$和$\,m_{A}\,$中较大的那一个, %
而较轻的中性质量$\,m_{h}\,$要小于$\,m_{Z}\,$和$\,m_{A}\,$中较小的那一个.}} 如果顶夸克和底夸克的质量之比很大是因为标量场真空期望值之比$\,v_{2}/v_{1}=\tan\beta\,$很大, 而非\,Yukawa\,耦合之比很大, 那么我们可以预期$\,\beta\,$接近$\,\uppi/2$, 在这些情况下, 这些不等式变成了近似等式. 更进一步, 方程(\ref{28.5.35})表明{\kai{带荷标量的质量既大于$\,m_{A}\,$也大于$\,m_{W}$.}}

这些结果定量上被标准模型内的各种辐射修正所修正(与之相反的是, 在规范传递的超对称破缺理论中, %
辐射修正产生的是输入参量$\,m_{i}^{2}$.) 最重要的修正由单个顶夸克或底夸克圈与外标量场相互作用任意多次构成的图%
在标量势$\,V\,$ 中产生的项给出. 这是因为顶夸克和底夸克分别与$\,\mathscr{H}_{2}\,$和$\,\mathscr{H}_{1}\,$的耦合到目前为止是最强的. (这里将顶夸克圈和底夸克圈都计入在内是明智的, 这是因为, 正如前面所提及的, %
顶夸克的大质量可能来源于大比值$\,v_{2}/v_{1}\,$而非\,Youkawa\,耦合的大比值, 但即使是在这一情况下, 我们将会看到主导贡献仍然来自于顶夸克圈.)

我们先来考虑中性标量, 它们中至少有一个在没有辐射修正时将轻于$\,Z\,$玻色子. 这些顶夸克圈和底夸克圈的效应是给%
$\,V^{\mathrm{N}}\,$贡献形如$\,U_{t}(\lvert \mathscr{H}_{2}^{0}\rvert^{2})+U_{b}(\lvert \mathscr{H}_{1}^{0}\rvert^{2})\,$的一项. 我们将会把$\,U_{b}\,$或$\,U_{t}\,$中任何关于$\,\lvert \mathscr{H}_{1}^{0}\rvert^{2}-v_{1}^{2}\,$或$\,\lvert \mathscr{H}_{2}^{0}\rvert^{2}-v_{2}^{2}\,$线性的项吸收进输入参量$\,m_{1}^{2}\,$和$\,m_{2}^{2}\,$中, 使得
\begin{equation}
    U_{b}^{\prime}(v_{1}^{2})= U_{t}^{\prime}(v_{2}^{2}) =0 \:. \label{28.5.38}
\end{equation}
这样, 我们之前对$\,m_{1}^{2}+\lvert\mu\rvert^{2}$, $m_{2}^{2}+\lvert\mu\rvert^{2}\,$和$\,B\mu\,$的结果(\ref{28.5.24})和(\ref{28.5.22})就保持不变. 另外, $\mathsf{C}\,$为奇的中性标量的质量平方矩阵仍由方程(\ref{28.5.26})给出. 另一方面, %
对于$\,\mathsf{C}\,$为偶的中性标量, 质量平方矩阵现在是
\begin{align}
    (M_{\operatorname{Re}\varphi}^{2})_{11} &= \tfrac{1}{2}m_{A}^{2}(1-\cos 2\beta)+\tfrac{1}{2}m_{Z}^{2}(1+\cos2\beta) + \Delta_{b}\:, \nonumber \\
    (M_{\operatorname{Re}\varphi}^{2})_{12} &= (M_{\operatorname{Re}\varphi}^{2})_{21} =
    -\tfrac{1}{2}(m_{A}^{2}+m_{Z}^{2})\sin 2\beta \:, \label{28.5.39} \\
    (M_{\operatorname{Re}\varphi}^{2})_{22} &= \tfrac{1}{2}m_{A}^{2}(1+\cos 2\beta)
    +\tfrac{1}{2}m_{Z}^{2}(1-\cos2\beta) +\Delta_{t}\:. \nonumber
\end{align}
其中
\begin{equation}
    \Delta_{b} = 2 v_{1}^{2}\,U_{b}^{\prime\prime}(v_{1}^{2}) \:, \qquad
    \Delta_{t} = 2 v_{2}^{2}\,U_{t}^{\prime\prime}(V_{2}^{2}) \:. \label{28.5.40}
\end{equation}
那么久期方程的解就是
\begin{align}
    m_{H}^{2} &= \frac{1}{2}\Biggl[ m_{A}^{2}+m_{Z}^{2} +\Delta_{t}+\Delta_{b} \nonumber \\
    &\quad +\sqrt{\Bigl((m_{A}^{2}-m_{Z}^{2})\cos2\beta+\Delta_{t}-\Delta_{b}\Bigr)^{2}
    +\Bigl(m_{A}^{2}+m_{Z}^{2}\Bigr)^{2}\sin^{2}2\beta} \Biggr] \:, \label{28.5.41}
\end{align}
\begin{align}
    m_{h}^{2} &= \frac{1}{2}\Biggl[ m_{A}^{2}+m_{Z}^{2} +\Delta_{t}+\Delta_{b} \nonumber \\
    &\quad -\sqrt{\Bigl((m_{A}^{2}-m_{Z}^{2})\cos2\beta+\Delta_{t}-\Delta_{b}\Bigr)^{2}
    +\Bigl(m_{A}^{2}+m_{Z}^{2}\Bigr)^{2}\sin^{2}2\beta} \Biggr] \:. \label{28.5.42}
\end{align}
在考虑搜寻这些粒子时, 很重要的一件事是注意到, 轻\,Higgs\,质量$\,m_{h}\,$随着未知质量$\,m_{A}\,$的增长而增长, %
而在$\,m_{A}\to\infty\,$时达到上界
\begin{equation}
    m_{h}\leq m_{h}(m_{A}\to\infty) = m_{Z}^{2}\cos^{2}2\beta +\Delta_{t}\sin^{2}\beta
    +\Delta_{b} \cos^{2}\beta \:. \label{28.5.43}
\end{equation}

为了计算$\,\Delta_{b}\,$和$\,\Delta_{t}$, 我们回忆在\,\textcolor{foo}{16.2}\,节势$\,U_{b}\,$和$\,U_{t}\,$给定为
\begin{align}
    U_{b}(\lvert\mathscr{H}_{1}^{0}\rvert^{2}) &= -\frac{3}{16\uppi^{2}}\Bigl\lvert \lambda_{b}\mathscr{H}_{1}^{0}\Bigr\rvert^{4}\left[\ln\frac{\Bigl\lvert \lambda_{b}\mathscr{H}_{1}^{0}\Bigr\rvert^{2}}{M_{sb}^{2}}-\frac{3}{2}\right] +\text{线性项}\:, \label{28.5.44} \\
    U_{t}(\lvert\mathscr{H}_{2}^{0}\rvert^{2}) &= -\frac{3}{16\uppi^{2}}\Bigl\lvert \lambda_{t}\mathscr{H}_{2}^{0}\Bigr\rvert^{4}\left[\ln\frac{\Bigl\lvert \lambda_{t}\mathscr{H}_{2}^{0}\Bigr\rvert^{2}}{M_{st}^{2}}-\frac{3}{2}\right] +\text{线性项}\:, \label{28.5.45}
\end{align}
其中$\,\lambda_{t}=m_{t}/v_{2}\,$和$\,\lambda_{b}=m_{b}/v_{1}\,$是顶夸克和底夸克的\,Yukawa\,耦合; $M_{st}\,$和$\,M_{sb}\,$是标量顶夸克和标量底夸克(顶夸克和底夸克的标量超对称伴)的质量; 这些质量以及方括号中的项$\,-3/2\,$被选择成满足如下条件: 如果质量相等, 那么标量顶夸克圈和标量底夸克圈引起的超对称破缺修正将会抵消顶夸克圈和底夸克圈引起的修正; %
并且``线性项''关于$\,\lvert \mathscr{H}_{2}^{0}\rvert^{2}\,$或$\,\lvert \mathscr{H}_{1}^{0}\rvert^{2}\,$是线性的, 而系数被调整成满足方程(\ref{28.5.38}). (因子\,3\,是因为将\,3\,个夸克色考虑在内.) 那么方程(\ref{28.5.40})给出
\begin{equation}
    \Delta_{b} = -\frac{3}{4\uppi^{2}}\lvert\lambda_{b}\rvert^{4}v_{1}^{2}
    \ln\biggl(\frac{\lambda_{b}v_{1}^{2}}{M_{sb}^{2}}\biggr)= \frac{3\sqrt{2}\,m_{b}^{4}\,G_{F}}{2\uppi^{2}\cos^{2}\beta} \ln\biggl(\frac{M_{sb}^{2}}{m_{b}^{2}}\biggr)\:,\label{28.5.46}
\end{equation}
\begin{equation}
    \Delta_{t} = -\frac{3}{4\uppi^{2}}\lvert\lambda_{t}\rvert^{4}v_{2}^{2}
    \ln\biggl(\frac{\lambda_{t}v_{2}^{2}}{M_{st}^{2}}\biggr)= \frac{3\sqrt{2}\,m_{t}^{4}\,G_{F}}{2\uppi^{2}\sin^{2}\beta} \ln\biggl(\frac{M_{st}^{2}}{m_{t}^{2}}\biggr)\:,\label{28.5.47}
\end{equation}
其中$\,G_{F}=1.17\times 10^{-5}\,\mathrm{GeV}^{-2}\,$是\,Fermi\,耦合常数, %
由方程(\textcolor{foo}{21.3.34})给定为$\,G_{F}=g^{2}/4\sqrt{2}m_{W}^{2}.$ 取 $m_{b}=4.3\,\mathrm{GeV},$ $m_{t}=180\,\mathrm{GeV}$, $M_{st}\sim M_{sb}\sim 1\,\mathrm{TeV}\,$以及$\,m_{Z}=91.2\,\mathrm{GeV}\,$给出$\,\Delta_{b}\sim1.1\times 10^{-6}m_{Z}^{2}/\cos^{2}\beta\,$和$\,\Delta_{t}\sim1.1m_{Z}^{2}/\sin^{2}\beta$. %
我们看到即使$\,\tan\beta\,$和$\,m_{t}/m_{b}\,$一样大, 顶夸克修正$\,\Delta_{t}\,$仍然要比$\,\Delta_{b}\,$大得多.

$\Delta_{t}\,$的效应是{\kai{增加}}$\,m_{H}\,$和$\,m_{h}$. 将这个以及其它辐射修正考虑在内,\cite{26} 当$\,\tan\beta>10\,$时, 对于处在$\,300\,\mathrm{GeV}\,$和$\,1\,\mathrm{TeV}\,$之间的标量夸克质量, 最轻中性标量质量上的上界(\ref{28.5.42})从$\,m_{Z}\,$以下被提高至$\,100\,\mathrm{GeV}\,$和$\,110\,\mathrm{GeV}\,$之间. 作为比较, $e^{+}e^{-}\,$在\,130\,至$\,172\,\mathrm{GeV}\,$之间碰撞没有产生$\,hA\,$或$\,HA\,$末态给$\,m_{h}$, %
$m_{H}\,$和$\,m_{A}\,$设置了实验上的下界\cite{27}$\,62.5\,\mathrm{GeV}$. 另外, %
包含\,Higgs\,标量的辐射修正的计算与$\,m_{h}\,$处在$\,27\,$至$\,140\,\mathrm{GeV}\,$范围内的电弱现象的精确测量一致.\cite{27a}

辐射修正对带荷标量是次要的. 由于连接带荷标量线将会允许顶夸克和底夸克之间的跃迁, 这里对标量势的修正采取一个更加普遍的形式, %
被$\,SU(2)\times U(1)\,$约束为
\begin{equation}
    \Delta V = U(\mathscr{H}_{2}^{\dag}\mathscr{H}_{2},\mathscr{H}_{1}^{\dag}\mathscr{H}_{1},
    \mathscr{H}_{2}^{\dag}\mathscr{H}_{1},\mathscr{H}_{1}^{\dag}\mathscr{H}_{2},
    \mathscr{H}_{1}^{\mathrm{T}}e\mathscr{H}_{2}) \:. \label{28.5.48}
\end{equation}
(夸克圈实际上不产生任何对$\,\mathscr{H}_{1}^{\mathrm{T}}e\mathscr{H}_{2}\,$的依赖.) %
每个$\,\mathscr{H}_{1}\,$或$\,\mathscr{H}_{2}\,$出现时总分别伴随着一个$\,\lambda_{b}\,$或$\,\lambda_{t}$ 因子, %
所以包含$\,\mathscr{H}_{1}\,$的项将被压低, 正如我们在计算中性标量质量时所看到的. 这样, 作为一个好的近似, 有效势的修正是如下的形式
\begin{equation}
    \Delta V \simeq U(\mathscr{H}_{2}^{\dag}\mathscr{H}_{2},0,0,0)
    =U(\lvert v_{2}+\varphi_{2}\rvert^{2}+\lvert\mathscr{H}_{2}^{-}\rvert^{2},0,0,0) \:. \label{28.5.49}
\end{equation}
通过回到带荷标量场为零的情况, 我们看到函数$\,U\,$必须是我们之前称作$\,U_{t}\,$的函数. 在$\,U\,$关于 $\lvert v_{2}+\varphi_{2}\rvert^{2}+\lvert\mathscr{H}_{2}^{-}\rvert^{2}-v_{2}^{2}\,$的幂级数展开中, %
任何一阶项将用来重新定义常数$\,m_{2}^{2}$, 并被约定(\ref{28.5.38})所消除. $\,U\,$中$\,\lvert v_{2}+\varphi_{2}\rvert^{2}+\lvert\mathscr{H}_{2}^{-}\rvert^{2}-v_{2}^{2}\,$的二阶项是真正的辐射修正, 然而, 尽管这些项中包含会影响中性标量质量的$\,\lvert\varphi_{2}\rvert^{2}\,$的二阶项, 但它们不包含能够偏移非\,Goldstone\,带荷标量质量的$\,\lvert\mathscr{H}_{2}^{-}\rvert^{2}\,$的二阶项. 幸运的是, 辐射修正不需要避免与实验产生矛盾, 这是因为当$\,m_{A}\,$没有上界时, 带荷标量质量(\ref{28.5.35})上没有上界. 在$\,181\,$至$\,184\,\mathrm{GeV}\,$之间没有观测到过程%
$\,e^{+}e^{-}\to\mathscr{H}^{+}\mathscr{H}^{-}\,$给出实验下界\cite{28}$\,m_{C}\geq 59 \,\mathrm{GeV}$. 过程$\,b\to s\gamma\,$的速率(在$\,B\to K^{\ast}\gamma\,$这样的衰变中测量)给$\,m_{C}\,$提供了一个严格得多的下界, 这样的过程可以通过跃迁到一个$\,\mathscr{H}^{-}u\,$或$\,\mathscr{H}^{-}c\,$中间态发生, 其中光子从虚夸克或$\,\mathscr{H}^{-}\,$辐射出去. 当前理论与实验在这一过程上的一致性给$\,m_{C}\,$设置了约为$\,150\,\mathrm{GeV}\,$的下界\cite{29}(在$\,\tan\beta<1\,$时会更高). %
利用方程(\ref{28.5.35}), 这给出了一个重要的下界$\,m_{A}>125\,\mathrm{GeV}$.

为了给出电弱对称性破缺一个成功的解释, 在任何超对称破缺的模型中, 质量$\,m_{i}^{2}\,$都必须满足两个条件. 其中一个由势下有界这个要求提供, 而正如我们看到的, 这要求
\[
    2\lvert\mu\rvert^{2}+m_{1}^{2}+m_{2}^{2} > B\mu \:.
\]
由于$\,B\mu\,$被定义成正的, 这确保了$\,\mathsf{C}\,$为奇的中性标量的平方质量(\ref{28.5.21})为正. %
另一个条件由方程(\ref{28.5.22})和(\ref{28.5.24})给出, 对于$\,\beta\,$的任意值, 它要求
\begin{equation}
    4\,(m_{1}^{2}+\lvert\mu\rvert^{2})\,(m_{2}^{2}+\lvert\mu\rvert^{2})\leq (B\mu)^{2} \:. \label{28.5.50}
\end{equation}
从方程(\ref{28.5.10})中我们可以轻易地看到这个条件确保了势的二阶导数矩阵在$\,\mathscr{H}_{1}=\mathscr{H}_{2}=0\,$处有一个负的本征值, 所以这个$\,SU(2)\times U(1)\,$-不变点是一个不稳定的平衡态, 因此$\,SU(2)\times U(1)\,$必须自发破缺. %
如果$\,\beta\,$非常接近$\,\uppi/2$, 那么方程(\ref{28.5.24})告诉我们这个条件在$\,m_{1}^{2}+\lvert\mu\rvert^{2}\,$为正时且$\,m_{2}^{2}+\lvert\mu\rvert^{2}\,$为{\kai{负}}时被满足. 正如我们将在下一节看到的, 标量场拉格朗日量中参量的重整化群流提供了一个机制将$\,m_{2}^{2}+\lvert\mu\rvert^{2}\,$带到负值.

即使只有最小的一组超场, 在超对称理论中, 存在数对$\,SU(2)\times U(1)\,$变换性质不同但电荷, 色以及重子数和轻子数相同的粒子, 而当$\,SU(2)\times U(1)\,$自发破缺时, 这些粒子开始混合. 我们在上一节已经看到了这样的一个例子, 在这种情况下, %
我们不得不处理左手夸克的标量超对称伴与左手反夸克标量超对称伴的复共轭的混合. 类似的混合发生在希格斯微子和规范微子之间, 二者啊要么都带荷要么都中性; 有确定质量但既非希格斯微子又非规范微子而是混合物的被称为{\kai{带荷微子}}(\emph{charginos})和{\kai{中性微子}}(\emph{neutralinos}). 我们来考虑带荷微子, 它给$\,\mu\,$提供了一个有用的上界. 根据方程(\ref{27.4.8}), 拉格朗日密度中有非对角的超对称质量项
\[
-\operatorname{Re}\Bigl[\mu\Bigl(h_{1L}^{-\mathrm{T}}\epsilon h_{2L}^{+}\Bigr)
+\mi\sqrt{2}m_{W}\cos\beta\Bigl(w_{L}^{-\mathrm{T}}\epsilon h_{2L}^{+}\Bigr)
+\mi\sqrt{2}m_{W}\sin\beta \Bigl(w_{L}^{+\mathrm{T}}\epsilon h_{1L}^{-}\Bigr)\Bigr] \:,
\]
对这一项, 我们应该给它加上由规范相互作用在超对称破缺区域生成的$\,W\,$微子的质量项
\[
-m_{\text{wino}}\operatorname{Re}\Bigl(w_{L}^{+\mathrm{T}}\epsilon w_{L}^{-}\Bigr) \:.
\]
带荷微子的质量平方则是矩阵$\,\mathscr{M}_{C}^{\dag}\mathscr{M}_{C}\,$的本征值, 其中
\begin{equation}
    \mathscr{M}_{C} =
    \begin{pmatrix}
    m_{\text{wino}} & \mi\sqrt{2}m_{W}\sin\beta \\
    \mi\sqrt{2}m_{W}\cos\beta & \mu
    \end{pmatrix} \:. \label{28.5.51}
\end{equation}
这两个本征值是
\begin{align}
    m^{2}_{\text{chargino}} &= \frac{1}{2}\Biggl[ m_{\text{wino}}^{2}+2m_{W}^{2}+\lvert \mu\rvert^{2}
    \pm \Bigl((m^{2}_{\text{wino}}-\lvert\mu\rvert^{2})^{2}+4m_{W}^{4}\cos^{2}2\beta \nonumber \\
    &\quad + 4m_{W}^{2}(m_{\text{wino}}^{2}+\lvert\mu\rvert^{2}-2m_{\text{wino}}\operatorname{Re}\mu\sin2\beta)
    \Bigr)^{1/2}\Biggr]\:. \label{28.5.52}
\end{align}
我们期待$\,W\,$微子的质量$\,m_{\text{wino}}\,$要远大于$\,m_{W}$. 如果它也远大于$\,\lvert\mu\rvert$, 较重的带荷微子多半是一个 $W\,$微子, 质量为$\,m_{\text{wino}}$, 而最轻的带荷微子多半是希格斯微子, 质量为$\,\mu$. 在任何情况下, $\mu\,$要大于最轻的带荷微子的质量, 而在$\,e^{+}$-$e^{-}\,$湮灭中没有出现规范微子告诉我们它要大于$\,60\,\mathrm{GeV}$, 并且可能大于$\,m_{W}$. %
在$\,e^{+}$-$e^{-}\,$湮灭中(在类似方程(\ref{28.4.2})这样的假定下)搜寻中性微子给最轻的中性微子的质量%
设立了一个$\,27\,\mathrm{GeV}\,$ 的下界.\cite{29a}


\section{超对称破缺的规范传递}  \label{sec:28.6}

在这一节, 我们将考虑超对称性破缺通过普通$\,SU(3)\times SU(2)\times U(1)\,$规范玻色子与它们的超对称伴的相互作用传递到已知粒子的可能性.\cite{30} 在这里假定超对称是在超场部分动力学破缺的, 其中{\kai{不}}包括已观测到夸克和轻子的超场, 以及对称性破缺部分中的一些被称为{\kai{信使超场}}(\emph{messenger superfields})的手征超场有不为零的$\,SU(3)\times SU(2)\times U(1)\,$量子数. %
为了使信使超场在不破缺$\,SU(3)$ $\times SU(2)\times U(1)\,$的情形下能够获得(例如, 约$\,1\,\mathrm{TeV}\,$的)大质量, 必然要使它们构成$\,SU(3)\times SU(2)\times U(1)\,$的一个实(或赝实)表示, 这同时自动意味着它们不会引入任何新的反常. 尽管文献中大多数对规范传递的超对称破缺的处理还对信使超场与响应超对称破缺的其它超场的相互作用做了具体的假定, 但这类理论的最重要预测实际上不依赖这些假定. 因此对信使超场与对称性破缺部分其他超场的相互作用, 我们将推迟做出任何假定. 然而, 基于一个唯象学上很强的原因, 我们将对信使超场$\,SU(3)\times SU(2)\times U(1)\,$性质做出另一个假定. 为了使信使粒子不干扰\,\ref{sec:28.2}\,节讨论的耦合常数统一, 我们假定它们对所有$\,SU(3)\times SU(2)\times U(1)\,$生成元平方的总迹的比值与普通夸克和轻子相同. %
如果信使超场(或许再加上一些$\,SU(3)\times SU(2)\times U(1)\,$中性的手征超场)构成某个包含$\,SU(3)\times SU(2)\times U(1)\,$的单群$\,G\,$的表示, 使得夸克和轻子(同上, 或许再加上一些$\,SU(3)\times SU(2)\times U(1)\,$中性的手征超场)也构成该群的一个完整表示, 那么这个条件将是自动满足的. (例如, 这些手征超场可能构成$\,N\,$个电荷为$\,e/3\,$的$\,SU(2)\,$单态$\,SU(3)\,$三重态和$\,N\,$个电荷为$\,0\,$和$\,-e\,$的$\,SU(2)\,$双重态$\,SU(3)$ 单态, 再加上同样多的处在$\,SU(3)\times SU(2)\times U(1)\,$复共轭表示下的手征超场, 前者构成$\,N\,$个$\,SU(5)$ 的$\,\bm{5}\,$表示, 后者构成$\,N\,$个$\,\bar{\bm{5}}\,$表示.) 然而, 对于我们当前的目的, 我们即无需假定$\,G\,$是理论的真实的对称群, 也不需要指定具体的$\,G\,$或信使粒子构成的表示.

\begin{figure}[t]
  \centering
  \begin{tikzpicture}[scale=1.4]
  \draw[thick,decorate, decoration={snake}] (-1,1) -- (1,1);
  \draw[thick] (2,1) circle (1) node [above=40pt] {$n$} ;
  \draw[thick,decorate, decoration={snake}] (3,1) -- (5,1);
  \draw[thick,dotted] (2.3,0.8) circle (0.25) ;
  \draw[thick,dotted] (1.3,1.7)-- (1.8,0.07);
  \draw[thick,dotted] (1.43,1.32)-- (1.15,0.5);
  \draw[thick,dotted] (2.05,0.6)-- (1.75,0.3);
  \draw[thick,dotted] (2.25,1.9)-- (2.3,1.05);
  \draw[thick,dotted] (2.55,0.57)-- (2.8,0.39);
  \node at (0,1.4) {$i,c$};
  \node at (4,1.4) {$i,c$};
\end{tikzpicture}

  \vspace{5mm}

  \caption{在规范微子传播子中引入超对称破缺的一类图. 这里的波浪线是规范超场的任意分量场; 实线是信使超场的分量场; %
  虚线是超对称破缺部分的$\,SU(3)\times SU(2)\times U(1)\,$中性超场的分量场.}%
  \label{fig:28.5}%
\end{figure}


信使超场与超对称破缺部分的其它手征和(或)规范超场以及$\,SU(3)\times SU(2)\times U(1)\,$规范超场的相互作用预期会在$\,SU(3)\times SU(2)\times U(1)\,$规范超场的分量场的传播子中产生超对称破缺. 到$\,SU(3)\times SU(2)\times U(1)\,$耦合的最低阶, 对传播子的领头阶贡献来自于图\,\ref{fig:28.5}\,中的图, 在这个图中, 一对规范线, 规范微子线或辅助$\,D\,$-场线被连到了信使场的圈上, 而这个圈又可以有任意多个与超对称破缺部分的$\,SU(3)\times SU(2)\times U(1)\,$中性场的相互作用. 因此对规范超场传播子的超对称破缺修正$\,\Delta_{ic}\,$(其中对$\,SU(3)$, $SU(2)\,$和$\,U(1)$, $i=1,2,3$, 而$\,c=V,\lambda,D\,$标记每个规范超场的不同分量)有如下的形式
\begin{align}
    \Delta_{3c}(q) &= (g_{s}^{2}/16\uppi^{2})\sum_{n} T_{3n}\Pi_{cn}(q) \:, \nonumber \\
    \Delta_{2c}(q) &= (g^{2}/16\uppi^{2})\sum_{n} T_{2n}\Pi_{cn}(q) \:, \label{28.6.1} \\
    \Delta_{1c}(q) &= (g^{\prime2}/16\uppi^{2})\sum_{n} T_{1n}\Pi_{cn}(q) \:, \nonumber
\end{align}
其中$\,n\,$标记不同的信使超场; $\Pi_{cn}(q)\,$是\,4\,-动量$\,q\,$多少有些复杂的函数; %
$T_{3n}\,$和$\,T_{2n}\,$分别是$\,SU(3)\,$和\\$\,SU(2)\,$任何生成元平方在第$\,n\,$个信使超场构成的表示中的迹(归一化成在定义表示中有$\,T_{3}=T_{2}=1/2$); 而$\,T_{1n}\,$是第$\,n\,$个信使超场的电弱超荷(hypercharge)的平方和. 一个立即的结果是规范微子获得相同形式的质量:\footnote{回忆, $B\,$微子是出现在标准模型拉格朗日量中的$\,U(1)\,$规范场的超对称伴. 我们还没有把$\,SU(2)\times U(1)\,$破缺考虑在内, 所以这里计算的规范微子, 标量夸克以及标量轻子的质量应该被理解成标准模型的$\,SU(3)\times SU(2)\times U(1)\,$-不变有效拉格朗日量中出现的参量.}
\begin{align}
    m_{\text{gluino}} &= (g_{s}^{2}/16\uppi^{2})\sum_{n} T_{3n} M_{gn} \:, \nonumber \\
    m_{\text{wino}} &= (g^{2}/16\uppi^{2})\sum_{n} T_{2n} M_{gn} \:, \label{28.6.2} \\
    m_{\text{bino}} &= (g^{\prime2}/16\uppi^{2})\sum_{n} T_{1n} M_{gn} \:, \nonumber
\end{align}
其中$\,M_{gn}\,$是表征不同信使超场的质量. 正如之前提及的, 为了保证耦合在很高的能量处统一, 我们假定$\,T_{n}\,$的和对已观测到的夸克和轻子有相同的比值:
\begin{equation}
    \sum_{n}T_{3n} =\sum_{n}T_{2n} =\sum_{n}3T_{1n}/5\equiv T \:. \label{28.6.3}
\end{equation}

超对称性在这些传播子中的破缺通过图\,\ref{fig:28.6}\,中所示的图与超对称标准模型中的标量夸克和标量轻子交互, 在这样的图中, %
一个$\,SU(3)\times SU(2)\times U(1)\,$规范玻色子或规范微子或辅助$\,D\,$场被标量夸克或标量轻子发射并再吸收. 我们要计算的低能理论还没有把$\,SU(3)\times SU(2)\times U(1)\,$考虑在内, 所以$\,SU(3)$, $SU(2)\,$和$\,U(1)\,$传播子之间没有混合, 并且每个传播子在规范指标的上的作用就像一个单位矩阵. 因此, 任何标量夸克和标量轻子被赋予的质量平方将正比于所有$\,SU(3)\times SU(2)\times U(1)\,$生成元(包含耦合常数)在那个标量夸克或标量轻子构建的表示中的平方和. $SU(2)\,$和$\,SU(3)$ 生成元在定义表示中的平方和是
\[
\sum_{a=1}^{3}\Bigl(g\sigma_{a}/2\Bigr)^{2}=\frac{3g^{2}}{4}\cdot 1 \:, \qquad
\sum_{\alpha=1}^{8}\Bigl(g_{s}\lambda_{\alpha}/2\Bigr)^{2}=\frac{4g_{s}^{2}}{3}\cdot 1 \:,
\]
其中$\,\sigma_{a}\,$是\,Pauli\,同位旋矩阵(\textcolor{foo}{5.4.18})而$\,\lambda_{a}\,$是\,Gell-Mann\,矩阵%
(\textcolor{foo}{19.7.2}). 对于$\,U(1)$, 生成元就是包含一个因子$\,g^{\prime}\,$在内的弱超荷%
(hypercharge)(\textcolor{foo}{21.3.7}). 标量夸克和标量轻子的质量平方因此有如下的形式
\begin{align}
    M_{Q}^{2}&=2\sum_{n}M_{sn}^{2}\Biggl[\frac{4}{3}\biggl(\frac{g_{s}^{2}}{16\uppi^{2}}\biggr)^{2}T_{3n}
    +\frac{3}{4}\biggl(\frac{g^{2}}{16\uppi^{2}}\biggr)^{2}T_{2n}
    +\Bigl(\frac{1}{6}\Bigr)^{2}\biggl(\frac{g^{\prime2}}{16\uppi^{2}}\biggr)^{2}T_{1n}\Biggr]\:, \nonumber \\
    M_{\bar{U}}^{2}&=2\sum_{n}M_{sn}^{2}\Biggl[\frac{4}{3}\biggl(\frac{g_{s}^{2}}{16\uppi^{2}}\biggr)^{2}T_{3n}
    +\Bigl(\frac{2}{3}\Bigr)^{2}\biggl(\frac{g^{\prime2}}{16\uppi^{2}}\biggr)^{2}T_{1n}\Biggr]\:, \nonumber \\
    M_{\bar{D}}^{2}&=2\sum_{n}M_{sn}^{2}\Biggl[\frac{4}{3}\biggl(\frac{g_{s}^{2}}{16\uppi^{2}}\biggr)^{2}T_{3n}
    +\Bigl(-\frac{1}{3}\Bigr)^{2}\biggl(\frac{g^{\prime2}}{16\uppi^{2}}\biggr)^{2}T_{1n}\Biggr]\:, \label{28.6.4} \\
    M_{L}^{2}&=2\sum_{n}M_{sn}^{2}\Biggl[\frac{3}{4}\biggl(\frac{g^{2}}{16\uppi^{2}}\biggr)^{2}T_{2n}
    +\Bigl(\frac{1}{2}\Bigr)^{2}\biggl(\frac{g^{\prime2}}{16\uppi^{2}}\biggr)^{2}T_{1n}\Biggr]\:, \nonumber \\
    M_{\bar{E}}^{2}&=2\sum_{n}M_{sn}^{2}\biggl(\frac{g^{\prime2}}{16\uppi^{2}}\biggr)^{2}T_{1n}\:, \nonumber
\end{align}
其中$\,Q$, $\bar{U}$, $\bar{D}$, $L\,$和$\,\bar{E}\,$是左手夸克双重态, 电荷为$\,-2e/3\,$和$\,+e/3\,$的左手反夸克, 左手轻子双重态以及左手带电反轻子的标量超对称伴, 而$\,M_{sn}\,$是一些用来表征第\,n\,个信使超场的新质量. %
(从$\,M_{sn}^{2}\,$中抽出因子\,2\,是为了将来的方便.) 以这种方式产生的标量夸克和标量轻子质量在所有三代中将是自动相等的, 因此避免了\,\ref{sec:28.4}\,节讨论的味改变过程伴随的问题.

我们期待所有$\,M_{gn}\,$和$\,M_{sn}\,$在量级上粗略相同, 进而使得胶微子和标量夸克有可比拟的质量, 而$\,W\,$微子, $B\,$微子和标量轻子则要轻得多, 它们的质量被电弱耦合常数压低了.

\begin{figure}[t]
  \centering
  \begin{tikzpicture}[scale=1.4]

  \draw[thick, dashed] (-1,1)--(5,1);
  %\draw[thick,dashed] (3.6,1)--(5,1);
  %\draw[thick] (0.4,1)--(3.6,1);
  \node at (2,2) [rectangle, draw,thick,minimum size=14] {};
  \draw[thick,decorate, decoration={snake,amplitude=3pt,segment length=13.5pt}] (0.4,1) .. controls (0.5,1.3) and (1,1.7) .. (1.81,2);
  \draw[thick,decorate, decoration={snake,amplitude=3pt,segment length=13.5pt}] (2.19,2) .. controls (3,1.8) and (3.5,1.25) .. (3.6,1);
  %\draw[thick] (0.4,1) .. controls (0.5,1.4) and (1,1.8) .. (1.81,2);
  %\draw[thick] (2.19,2) .. controls (3,1.8) and (3.5,1.4) .. (3.6,1);

\end{tikzpicture}

  \vspace{7mm}

\begin{tikzpicture}[scale=1.4]

\draw[thick, dashed] (-1,1)--(0.4,1);
\draw[thick,dashed] (3.6,1)--(5,1);
\draw[thick] (0.4,1)--(3.6,1);
\node at (2,2) [rectangle, draw,thick,minimum size=14] {};
\draw[thick,decorate, decoration={snake,amplitude=3pt,segment length=13.5pt}] (0.4,1) .. controls (0.5,1.3) and (1,1.7) .. (1.81,2);
\draw[thick,decorate, decoration={snake,amplitude=3pt,segment length=13.5pt}] (2.19,2) .. controls (3,1.8) and (3.5,1.25) .. (3.6,1);
\draw[thick] (0.4,1) .. controls (0.5,1.4) and (1,1.8) .. (1.81,2);
\draw[thick] (2.19,2) .. controls (3,1.8) and (3.5,1.4) .. (3.6,1);

\end{tikzpicture}

  \vspace{5mm}

  \caption{将超对称破缺传递至标量夸克和标量轻子的图. 这里的虚线是标量夸克或标量轻子; %
  波浪线是$\,SU(3)\times SU(2) \times U(1)\,$规范玻色子或辅助$\,D\,$ 场; 实线是夸克或轻子; %
  实线和波浪线的组合线是$\,SU(3)\times SU(2)\times U(1)\,$规范微子; 而方框代表插入图\,\ref{fig:28.5}\,中所示的超对称破缺传播子修正.}
  \label{fig:28.6}%
\end{figure}

在一些合理的动力学假定下, 我们可以更进一步. 假定超对称性破缺在信使超场上的效应可以通过将这些超场以及一组$\,SU(3)\times SU(2)\times U(1)\,$-中性手征超场$\,S_{n}\,$(不一定全不相同)并进超势
\begin{equation}
    f(\Phi,\bar{\Phi},S)=\sum_{n}\lambda_{n}S_{n}\,\Phi_{n}\bar{\Phi}_{n}  \label{28.6.5}
\end{equation}
来模型化, 其中$\,\bar{\Phi}_{n}\,$和$\,\Phi_{n}\,$是$\,SU(3)\times SU(2)\times U(1)\,$复共轭表示下的左手征超场, 而$\,\lambda_{n}\,$是一组耦合常数. (这里以及后面我们将隐去在计算$\,\Phi_{n}\bar{\Phi}_{n}\,$这样的标量积时要求和的$\,SU(3)\times SU(2)\,$指标.) 对超场$\,S_{n}\,$的标量分量和辅助分量, 预期分别会有不为零的真空期望值$\,\mathscr{S}_{n}\,$和%
$\,\mathscr{F}_{n}$. 在这些模型中, 正是$\,\mathscr{F}_{n}\,$的非零值引入了$\,\Phi_{n}\,$和$\,\bar{\Phi}_{n}\,$粒子质量中的超对称破缺. \ref{sec:26.4}\,节证明了, 当规范耦合被忽略时, $\Phi_{n}\,$(和$\,\bar{\Phi}_{n}\,$)的旋量分量的质量平方是矩阵$\,\mathscr{M}_{n}^{\dag}\mathscr{M}_{n}\,$的本征值, 其中$\,\mathscr{M}_{n}\,$由方程(\ref{26.4.11})定义, 它给出
\[
\mathscr{M}_{n} =
\begin{pmatrix}
0 & \lambda_{n}\mathscr{S}_{n} \\ \lambda_{n}\mathscr{S}_{n} & 0
\end{pmatrix} \:,
\]
使得信使费米子有质量$\,\lvert\lambda_{n}\mathscr{S}_{n}\rvert$. 为了找到超场$\,\Phi_{n}\,$和$\,\bar{\Phi}_{n}\,$的标量分量%
$\,\phi_{n}\,$和$\,\bar{\phi}_{n}\,$的质量项, 我们注意到积掉$\,\Phi_{n}\,$和$\,\bar{\Phi}_{n}\,$的辅助场后给出一个势
\[
\sum_{n}\Biggl\lvert \frac{\partial f(\phi,\bar{\phi},\mathscr{S})}{\partial\phi_{n}}\Biggr\rvert^{2}
+\sum_{n}\Biggl\lvert \frac{\partial f(\phi,\bar{\phi},\mathscr{S})}{\partial\bar{\phi}_{n}}\Biggr\rvert^{2}
=\sum_{n}\lvert \lambda_{n}\mathscr{S}_{n}\rvert^{2}
\Bigl[\lvert\phi_{n}\rvert^{2}+\lvert\bar{\phi}_{n}\rvert^{2}\Bigr] \:,
\]
我们现在必须给这个势加上$\,S_{n}\,$的辅助分量的贡献, 由方程(\ref{26.4.4})的第二项给出:
\[
2\operatorname{Re}\sum_{n}\Biggl[\lambda_{n}\mathscr{F}_{n}\frac{\partial f(\phi,\bar{\phi},\mathscr{S})}{\partial\mathscr{S}_{n}}\Biggr]
=2\operatorname{Re} \sum_{n}\Bigl[\mathscr{F}_{n}\lambda_{n}\phi_{n}\bar{\phi}_{n}\Bigr] \:.
\]
这样, 有确定质量的复标量场就是$\,(\phi_{n}\pm\me^{-\mi\alpha_{n}}\bar{\phi}_{n})/\sqrt{2}$, 其中$\,\alpha_{n}\,$是%
$\lambda_{n}\mathscr{F}_{n}\,$的相位, 它们的质量平方是$\,\lvert\lambda_{n}\mathscr{F}_{n}\rvert^{2}\pm\lvert\lambda_{n}\mathscr{F}_{n}\rvert$. (注意到这种一对复标量的质量平方以一个\,Majorana\,费米子的质量平方为中心等间隔排布正是我们从求和规则(\ref{27.5.11})中所期待的.) 既然这些质量平方必须是正的, 由此得出
\begin{equation}
    \lvert\mathscr{F}_{n}\rvert \leq \lvert\lambda_{n}\rvert \lvert \mathscr{S}_{n}\rvert^{2}\:. \label{28.6.6}
\end{equation}


模型中基于方程(\ref{28.6.5})的胶微子质量由图\,\ref{fig:28.5}\,中所示的那种图给出, 只不过现在只有一个圈, %
不包含图\,\ref{fig:28.5}\,中的虚线表示的部分. 一个细致的计算给出方程(\ref{28.6.2})中的系数$\,M_{gn}\,$是\cite{31}
\begin{equation}
    M_{gn} = \frac{\lvert \mathscr{F}_{n}\rvert}{\lvert\mathscr{S}_{n}\rvert}\,g\,
    \biggl(\frac{\lvert \mathscr{F}_{n}\rvert}{\lvert\lambda_{n}\rvert\lvert\mathscr{S}_{n}\rvert^{2}}\biggr)\:,
    \label{28.6.7}
\end{equation}
其中
\begin{align}
    g(x) &= \frac{1}{2x^{2}} \Bigl[(1+x)\ln(1+x)+(1-x)\ln(1-x)\Bigr] \nonumber \\
    &= 1 + \frac{x^{2}}{6} + \frac{x^{4}}{15} + \frac{x^{6}}{28} + \cdots \:. \label{28.6.8}
\end{align}
标量夸克和标量轻子的质量由图\,\ref{fig:28.6}\,中的图给出, 它现在只包含两个圈. %
另一个细致的计算给出方程(\ref{28.6.4})中的质量参量$\,M_{sn}^{2}\,$是\cite{31}
\begin{equation}
    M_{sn}^{2}=\frac{\lvert \mathscr{F}_{n}\rvert^{2}}{\lvert\mathscr{S}_{n}\rvert^{2}}\,
    f\biggl(\frac{\lvert \mathscr{F}_{n}\rvert}{\lvert\lambda_{n}\rvert\lvert\mathscr{S}_{n}\rvert^{2}}\biggr)
    \:, \label{28.6.9}
\end{equation}
其中
\begin{align}
    f(x) &= \frac{1+x}{x^{2}}\,\biggl[\ln(1+x)-2\operatorname{Li}_{2}\biggl(\frac{x}{1+x}\biggr)
    +\frac{1}{2}\operatorname{Li}_{2}\biggl(\frac{2x}{1+x}\biggr) \biggr] + x\to -x  \nonumber \\
    &= 1+ \frac{1}{36}x^{2}-\frac{11}{450}x^{4}-\frac{319}{11760}x^{6}+\cdots, \label{28.6.10}
\end{align}
而$\,\operatorname{Li}_{2}\,$是二重对数
\begin{equation}
    \operatorname{Li}_{2}(x) \equiv - \int_{0}^{x} \frac{\ln(1-t)}{t}\,\dif t\:. \label{28.6.11}
\end{equation}
特别地, 如果(也是通常假定的)各个$\,S_{n}\,$的值都相同, 并且如果对所有$\,n\,$有$\,\lvert\mathscr{F}\rvert\ll\lvert\lambda_{n}\rvert\lvert\mathscr{S}\rvert^{2}$, 那么方程(\ref{28.6.9})和(\ref{28.6.7})中的$\,f\,$和$\,g\,$可以被设为\,1, 使得
\begin{equation}
    M_{gn}=M_{sn}=\lvert \mathscr{F}\rvert / \lvert \mathscr{S} \rvert \equiv M \:. \label{28.6.12}
\end{equation}
利用方程(\ref{28.6.3}), 我们可以把规范微子质量(\ref{28.6.2})表示成
\begin{align}
    & m_{\text{wino}}= (g^{2}/16\uppi^{2}) TM \:, \nonumber \\
    & m_{\text{bino}}= (5/3)(g^{\prime2}/16\uppi^{2})TM \:, \label{28.6.13} \\
    & m_{\text{gluino}} = (g_{s}^{2}/16\uppi^{2})TM \:, \nonumber
\end{align}
标量夸克和标量轻子的质量平方(\ref{28.6.4})则变成
\begin{align}
    M_{Q}^{2}&=2TM^{2}\Biggl[\frac{4}{3}\biggl(\frac{g_{s}^{2}}{16\uppi^{2}}\biggr)^{2}
    +\frac{3}{4}\biggl(\frac{g^{2}}{16\uppi^{2}}\biggr)^{2}
    +\frac{5}{3}\Bigl(\frac{1}{6}\Bigr)^{2}\biggl(\frac{g^{\prime2}}{16\uppi^{2}}\biggr)^{2}\Biggr]\:, \nonumber \\
    M_{\bar{U}}^{2}&=2TM^{2}\Biggl[\frac{4}{3}\biggl(\frac{g_{s}^{2}}{16\uppi^{2}}\biggr)^{2}
    +\frac{5}{3}\Bigl(\frac{2}{3}\Bigr)^{2}\biggl(\frac{g^{\prime2}}{16\uppi^{2}}\biggr)^{2}\Biggr]\:, \nonumber \\
    M_{\bar{D}}^{2}&=2TM^{2}\Biggl[\frac{4}{3}\biggl(\frac{g_{s}^{2}}{16\uppi^{2}}\biggr)^{2}
    +\frac{5}{3}\Bigl(-\frac{1}{3}\Bigr)^{2}\biggl(\frac{g^{\prime2}}{16\uppi^{2}}\biggr)^{2}\Biggr]\:, \label{28.6.14} \\
    M_{L}^{2}&=2TM^{2}\Biggl[\frac{3}{4}\biggl(\frac{g^{2}}{16\uppi^{2}}\biggr)^{2}
    +\frac{5}{3}\Bigl(\frac{1}{2}\Bigr)^{2}\biggl(\frac{g^{\prime2}}{16\uppi^{2}}\biggr)^{2}\Biggr]\:, \nonumber \\
    M_{\bar{E}}^{2}&=2TM^{2}\frac{5}{3}\biggl(\frac{g^{\prime2}}{16\uppi^{2}}\biggr)^{2}\:, \nonumber
\end{align}
没有什么特别的理由期待会有$\,\lvert\mathscr{F}\rvert\ll\lvert \lambda_{n}\mathscr{S}\rvert^{2}$, 但这个假定的限制性实际上不是很强, 这是因为方程(\ref{28.6.6})已经要求$\,\lvert\mathscr{F}\rvert\leq\lvert\lambda_{n}\rvert\lvert\mathscr{S}\rvert^{2}$, 而除非$\,x\,$非常接近于\,1, 否则函数$\,f(x)\,$和$\,g(x)\,$在$\,x<1\,$时与\,1\,相差不大.

通过使用\,\ref{sec:27.6}\,节描述过的\,Seiberg\,的全纯讨论,\cite{33} Giudice\,和\,Rattazzi\cite{32}解释了结果(\ref{28.6.13}) 和(\ref{28.6.14})为何如此简单. 假如我们对信使超场引入像(\ref{28.6.5})这样的超势, 但只有一个外单态超场$\,S$:
\begin{equation}
    f(S,\Phi)=S\sum_{n}\lambda_{n}\,\Phi_{n}\bar{\Phi}_{n} \:. \label{28.6.15}
\end{equation}
此外, 对于在重整化标度$\,\mu\,$处的威尔逊型有效拉格朗日量, 其中的规范超场$\,V_{i}\,$(对于$\,SU(3)$, $SU(2)$ 和$\,U(1)$, $i=3,2,1$)的动能项现在将采取形式
\begin{equation}
    \mathscr{L}_{\text{gauge},\mu}=\operatorname{Re}
    \Biggl[\sum_{i}N_{i}(S,\mu)\sum_{\alpha\beta}(W_{iL\alpha}\epsilon_{\alpha\beta}W_{iL\beta})\Biggr]_{\mathscr{F}}\:,
    \label{28.6.16}
\end{equation}
其中一些函数$\,N_{i}(S,\mu)\,$被替换成了方程(\ref{27.3.22})中的因子$\,1/2g_{i}^{2}(\mu)$. (这里扔掉了$\,\theta\,$-项是因为它在微扰论中没有任何效应. 对$\,W_{iL\alpha}\,$上标记$\,SU(3)\,$和$\,SU(2)\,$伴随表示不同成员的指标暗含了一个求和, 这里没有显式地写出.) 规范耦合常数现在通过令超场$\,S\,$等于它的标量分量的真空期望值$\,\mathscr{S}\,$给出
\begin{equation}
    \frac{1}{2g_{i}^{2}(\mu)} = N_{i}(\mathscr{S},\mu) \:. \label{28.6.17}
\end{equation}
另外, 回忆起$\,W_{iL\alpha}=\lambda_{iL\alpha}+O(\theta)\,$并使用方程(\ref{27.2.11}), 拉格朗日密度(\ref{28.6.16})中规范微子场的二阶项在相差一些导数项的意义下是
\[
-2\sum_{i}\operatorname{Re}\Bigl[N_{i}(\mathscr{S},\mu)
\Bigl(\bar{\lambda}_{iR}\,\slashed{\partial}\lambda_{iR}\Bigr)
+[N_{i}(S,\mu)]_{\mathscr{F}}\Bigl(\lambda_{iL}^{\mathrm{T}}\epsilon\lambda_{iL}\Bigr)\Bigr] \:.
\]
这给出了规范微子质量
\begin{equation}
    m_{gi}(\mu) = \biggl\lvert \frac{[N_{i}(s,\mu)]_{\mathscr{F}}}{2N_{i}(\mathscr{S},\mu)}\biggr\rvert
    = g^{2}_{i}(\mu) \Bigl\lvert [N_{i}(S,\mu)]_{\mathscr{F}} \Bigr\rvert \:. \label{28.6.18}
\end{equation}
现在我们来考虑$\,N_{i}(\mathscr{S},\mu)\,$作为正实$\,\mathscr{S}\,$的函数的行为, 其中信使超场的相位都被调整成是的所有 $\lambda_{n}\,$都是正实的. 假定我们在所有信使粒子质量之上的某个标度$\,\mu=K\,$处固定规范耦合的值$\,g_{i}(\mu)$. 将重整化群方程$\,\mu\dif g_{i}/\dif\mu=b_{i}g_{i}^{3}\,$中的常数$\,b_{i}\,$在$\,\mu\,$穿过不同信使质量时发生的变化考虑在内, %
当$\,\mu$ {\kai{低于}}低于所有信使粒子的质量时, 这个方程有如下形式的解
\[
    \frac{1}{g_{i}^{2}(\mu)} =\frac{1}{g_{i}^{2}(K)} - 2b_{i}^{(0)}\ln\biggl(\frac{M_{1}}{K}\biggr)
    - 2b_{i}^{(1)}\ln\biggl(\frac{M_{2}}{M_{1}}\biggr) - \cdots
    - 2b_{i}^{(N)}\ln\biggl(\frac{M_{N}}{\mu}\biggr) \:,
\]
其中我们标记信使粒子使得它们的质量$\,M_{n}=\lambda_{n}\mathscr{S}\,$满足
\[
M_{1}>M_{2}>\cdots> M_{N} \:,
\]
而在计算$\,b_{i}^{(n)}\,$时则只考虑质量小于$\,M_{n}\,$的粒子. 由于所有$\,M_{n}\,$都正比于$\,\mathscr{S}$, %
我们看到$\,N_{i}(\mathscr{S},\mu)\,$对 $\mathscr{S}\,$的依赖是
\begin{equation}
    N_{i}(\mathscr{S},\mu) = -b_{i}^{\text{messenger}} \ln\mathscr{S}+\mathscr{S}\text{-无关项} \:,\label{28.6.19}
\end{equation}
其中$b_{i}^{\text{messenger}}=b_{i}^{(0)}-b_{i}^{(N)}\,$是所有信使超场对$\,b_{i}\,$的贡献. %
根据方程(\ref{27.9.45})(其中$\,C_{i1}=0\,$且$\,C_{i2}^{f}=C_{i2}^{s}=\sum_{n}T_{in}$), 这是
\begin{equation}
    b_{i}^{\text{messenger}}=\frac{1}{16\uppi^{2}}\sum_{n}T_{in} \:. \label{28.6.20}
\end{equation}
由于超对称性要求$\,N_{i}(S,\mu)\,$是$\,S\,$的全纯函数, 我们看到
\begin{equation}
    N_{i}(\mathscr{S},\mu) = -\frac{1}{16\uppi^{2}}\sum_{n}T_{in}\ln S+S\text{-无关项}\:.\label{28.6.21}
\end{equation}
在$\,S=\mathscr{S}\,$附近展开, 到$\,\mathscr{F}\,$的一阶, 我们有$\,[\ln\mathscr{S}]_{\mathscr{F}}=\mathscr{F}/\mathscr{S}$, 这样方程(\ref{28.6.18})给出规范微子质量
\begin{equation}
    m_{gi}(\mu)=\frac{g_{i}^{2}(\mu)}{16\uppi^{2}}\sum_{n}T_{in}\,\biggl\lvert\frac{\mathscr{F}}{\mathscr{S}}\biggr\rvert\:. \label{28.6.22}
\end{equation}
利用方程(\ref{28.6.3}), 我们看到这与我们前面的结果(\ref{28.6.13})相同. 代替规范超场, Giudice\,和\,Rattazzi\,\\ 通过类似的方法研究夸克和轻子超场的动量项获得了标量夸克和标量轻子的方程(\ref{28.6.14}).

附带地, 如果我们假定超对称破缺部分在某个以已知夸克和轻子以及信使场作为完整表示的大统一群$\,G\,$下是不变的, %
不做像方程(\ref{28.6.5})这样的特殊动力学假定也可获得方程(\ref{28.6.13})和 (\ref{28.6.14})(一般情况下, %
方程(\ref{28.6.13})和(\ref{28.6.14})中的$\,M\,$值不同). 在这一情况下, 方程(\ref{28.6.2})和 (\ref{28.6.4})中的%
系数$\,M_{gn}\,$和$\,M_{sn}\,$将分别有值$\,M_{g}(d)\,$和$\,M_{s}(d)$, 它们仅依赖于第$\,n\,$个信使场所属的$\,G\,$的%
不可约表示$\,d$. 对属于$\,G\,$的任何不可约表示$\,d\,$的$\,n$, $T_{in}\,$对其求和与方程(\ref{28.6.3})中的和有相同的比值, %
所以$\,\sum_{n\in d} T_{in}=k_{i}T(d)$, 其中$\,k_{3}=k_{2}=1$, $k_{1}=5/3$, 因此
\[
\sum_{n}T_{in}M_{gn}=\sum_{d}M_{g}(d)\sum_{n\in d}T_{in} = k_{i}M_{g} \:,
\]
其中$\,M_{g}=\sum_{d}M_{g}(d)T(d)$. 同样的,
\[
\sum_{n}T_{in}M_{sn}^{2} =\sum_{d}M_{s}^{2}(d)\sum_{n\in d}T_{in} = k_{i}M_{s}^{2} \:,
\]
其中$\,M_{s}^{2}=\sum_{d}M_{s}^{2}(d)T(d)$. 这样, 除了$\,TM\,$要被换成$\,M_{g}\,$以及$\,2TM^{2}\,$要被换成$\,M_{s}^{2}$, %
方程(\ref{28.6.2})和\\(\ref{28.6.4})将会给出方程(\ref{28.6.13})和(\ref{28.6.14}). 然而, 如果信使的质量标度远低于大统一的标度, $M_{gn}$ 和$\,M_{sn}\,$遵循$\,G\,$下的不变性这个假定是不合道理的, 这是因为, 无论我们假定大统一群是什么, %
对于像$\,\lambda_{n}\,$这样的耦合常数, 如果同属大统一群同一表示的$\,\Phi_{n}\,$有不相同的%
$\,SU(3)\times SU(2)\times U(1)\,$的量子数, 那么$SU(3)\times SU(2)\times U(1)\,$规范相互作用会使得这些耦合常数的跑动不相同.

这些结果要做各种辐射修正, 其中最重要的一个是我们必须使用$\,g_{s}$, $g\,$和$\,g^{\prime}\,$在与要计算的质量可比拟的标度重整化的值. 诚然, 方程(\ref{28.6.13})中给出的规范微子的质量比值也可以从完全不同的假定中推导出来: 所有规范微子的质量在耦合常数之间的关系是$\,g_{s}^{2}=g^{2}=5g^{\prime2}/3\,$的大统一标度处相等, 并且就像重整化方程描述的那样在低能标处变得不同.

数值结果的一个例子是, 假定信使超场构成电荷为$\,e/3\,$的$\,SU(2)\,$单态$\,SU(3)\,$三重态以及电荷为$\,0\,$和$\,-e\,$的%
$\,SU(2)\,$双重态$\,SU(3)\,$单态, 以及处在$\,SU(3)\times SU(2)\times U(1)\,$复共轭表示下的左手征超场. 那么正如前面提及的, %
方程(\ref{28.6.3})满足于$\,T=2\times 1/2=1$, 所以通过使用规范耦合的正确值, 计算可得标量夸克, 胶微子, $L\,$标量轻子, $W\,$微子, $E\,$标量轻子以及$\,B\,$微子的质量比值是\cite{34}$11.6::7.0::2.5::2::1.1::1.0$.

除了耦合常数跑动, 还有额外的修正. 根据一个计算,\cite{34} 在有一个电荷为$\,e/3\,$的$\,SU(2)\,$单态 $SU(3)\,$三重态以及%
电荷为$\,0\,$和$\,-e\,$的$\,SU(2)\,$双重态$\,SU(3)\,$单态, 以及处在$\,SU(3)\times SU(2)\times U(1)\,$复共轭表示下的左手征超场的模型中, 辐射修正给出的标量夸克, 胶微子, $L\,$标量轻子, $W\,$微子, %
$E\,$标量轻子以及$\,B\,$微子的质量比值是$\,9.3::6.4::2.6::1.9::1.35::1.0$.

正如我们在上一节看到的, $W\,$微子和$\,B\,$微子可以与带荷以及中性希格斯微子混合, 所以这里计算出的$\,W\,$微子和$\,B\,$微子质量必须视为计算称为{\kai{带荷微子}}和{\kai{中性微子}}的混合物的物理质量的输入量, 而不是物理质量本身.

现在我们来考虑这些模型中\,Higgs\,标量的质量. 如果我们仅考虑这些标量通过与超对称破缺部分的规范相互作用的两圈图获得质量, 那么由于它们与左手轻子双重态(除符号外)有相同的 $SU(3)\times SU(2)\times U(1)\,$量子数, 它们的质量将由像方程组(\ref{28.6.4})中的第四个方程那样的公式给出
\begin{align}
    [m_{1}^{2}]_{\text{2 loop}} &= [m_{2}^{2}]_{\text{2 loop}} = M_{L}^{2} \nonumber \\
    &=\sum_{n}M_{sn}^{2}\Biggl[\frac{3}{4} \biggl(\frac{g^{2}}{16\uppi^{2}}\biggr)^{2} T_{2n}
    +\Bigl(\frac{1}{2}\Bigr)^{2}\biggl(\frac{g^{\prime 2}}{16\uppi^{2}}\biggr)^{2} T_{1n}\Biggr]^{2} \:. \label{28.6.23}
\end{align}
如果这就是整个故事, 那么(除非$\,\tan\beta\,$非常接近于一)就不可能达到上一节发现的$\,SU(2)\times U(1)\,$破缺条件, %
即$\,m_{1}^{2}+\lvert\mu\rvert^{2}\,$和$\,m_{2}^{2}+\lvert\mu\rvert^{2}\,$中有一个为负. 幸运的是, 顶夸克和标量夸克的大质量给了$\,m_{2}^{2}\,$一个负贡献, 这自然地导致了电弱对称性的自发破缺. Higgs\,双重态与第三代夸克超场的耦合由如下超势描述
\begin{equation}
    f_{\text{3rd gen}}= \lambda_{b}\Bigl(H_{1}^{\mathrm{T}}eQ\Bigr)\bar{B}
    +\lambda_{t} \Bigl(H_{2}^{\mathrm{T}}eQ\Bigr) \bar{T} \:, \label{28.6.24}
\end{equation}
其中$\,Q\,$是$\,SU(2)\,$夸克左手征超场双重态$\,(T,B)$, 而$\,\bar{T}\,$和$\,\bar{B}\,$是左手反顶夸克和反底夸克的左手征超场, $\lambda_{t}\,$和$\,\lambda_{b}\,$是\,Yukawa\,耦合, 与$\,t\,$夸克和$\,b\,$夸克的质量关系是$\,m_{t}=\lambda_{t}v_{2}\,$和%
$\,m_{b}=\lambda_{b}v_{1}$. 这样超势中与标量夸克和\,Higgs\,场相互作用相关的项就由方程(\ref{26.4.7})的最后一项给出
\begin{align}
    V_{sq\:H}&=\Bigl\lvert \lambda_{b}\mathscr{H}_{1}^{-}\bar{\mathscr{B}}
    +\lambda_{t}\mathscr{H}_{2}^{0}\bar{\mathscr{T}}\Bigr\rvert^{2}
    +\Bigl\lvert \lambda_{b}\mathscr{H}_{1}^{0}\bar{\mathscr{B}}
    +\lambda_{t}\mathscr{H}_{2}^{+}\bar{\mathscr{T}}\Bigr\rvert^{2} \nonumber \\
    &\quad +\lvert\lambda_{b}\rvert^{2}\Bigl\lvert \mathscr{H}_{1}^{0}\mathscr{B}-\mathscr{H}_{1}^{-}\mathscr{T}\Bigr\rvert^{2}
    +\lvert\lambda_{t}\rvert^{2}\Bigl\lvert \mathscr{H}_{2}^{+}\mathscr{B}
    -\mathscr{H}_{2}^{0}\mathscr{T}\Bigr\rvert^{2}  \:, \label{28.6.25}
\end{align}
其中下标代表超场的标量场分量. 这样标量夸克圈对$\,\mathscr{H}_{1}\,$和$\,\mathscr{H}_{2}\,$的贡献就是
\begin{equation}
    V_{H}^{\text{squark loop}} = 3\langle \mathscr{S}\mathscr{S}^{\ast}\rangle
    \Bigl[2\lvert\lambda_{b}\rvert^{2}\Bigl(\mathscr{H}_{1}^{\dag}\mathscr{H}_{1}\Bigr)
    +2\lvert\lambda_{t}\rvert^{2}\Bigl(\mathscr{H}_{2}^{\dag}\mathscr{H}_{2}\Bigr)\Bigr] \:, \label{28.6.26}
\end{equation}
其中$\,\langle\mathscr{S}\mathscr{S}^{\ast}\rangle\,$是任何一个标量夸克场与它的复共轭在同一时空点的真空期望值. (为了使上式对所有标量夸克类型都相同, 我们在这里使用了方程(\ref{28.6.4}), 它告诉我们标量夸克质量$\,M_{Q}\,$在$\,\mathscr{T}$, $\mathscr{B}$, $\bar{\mathscr{T}}\,$和$\,\bar{\mathscr{B}}\,$标量夸克之间变化不大). %
真空期望值$\,\langle\mathscr{S}\mathscr{S}^{\ast}\rangle\,$在最低阶是
\begin{equation}
    \langle \mathscr{S}\mathscr{S}^{\ast}\rangle \equiv \langle \mathscr{S}(x)\mathscr{S}^{\ast}(x)\rangle_{\text{VAC}}
    =\frac{-\mi}{(2\uppi)^{2}}\int\frac{\dif^{4}p}{p^{2}+M_{Q}^{2}-\mi\epsilon} \:. \label{28.6.27}
\end{equation}
这当然是发散的, 但如果标量夸克有夸克的零裸质量, 那么它对超对称破缺系数的贡献将被夸克圈抵消, 所以夸克圈的效应就是给(\ref{28.6.7})减去形式相同但是$\,M_{Q}\,$被替换成零的式子, 在\,Wick\,旋转后变成
\begin{align*}
    \langle \mathscr{S}\mathscr{S}^{\ast}\rangle &\to \frac{M_{Q}^{2}\,\mi}{(2\uppi)^{4}}
    \int\frac{\dif^{4}p}{(p^{2}+M_{Q}^{2}-\mi\epsilon)(p^{2}-\mi\epsilon)} \\
    &=-\frac{M_{Q}^{2}}{(2\uppi)^{4}} \int^{M^{2}}_{0}\frac{\uppi^{2}\dif p^{2}}{p^{2}+M_{Q}^{2}}
    \simeq -\frac{M_{Q}^{2}}{16\uppi^{2}}\ln \Biggl(\frac{M^{2}}{M_{Q}^{2}}\Biggr) \:.
\end{align*}
我们在信使质量$\,M\,$处插入了一个紫外截断, 因为当动量在$\,M\,$之上时, 标量夸克质量必须被换成一个动量相关的质量, 而这个质量在非常高的动量处趋于超对称值零. 在方程(\ref{28.6.26})中做这个减除给出了标量夸克圈和夸克圈对势的净贡献:
\begin{equation}
    (V_{m})^{\text{3 loop}}= \frac{3M_{Q}^{2}}{16\uppi^{2}}\ln \Biggl(\frac{M^{2}}{M_{Q}^{2}}\Biggr)\,
    \Biggl[2\lvert\lambda_{b}\rvert^{2}\Bigl(\mathscr{H}_{1}^{\dag}\mathscr{H}_{1}\Bigr)
    +2\lvert\lambda_{t}\rvert^{2}\Bigl(\mathscr{H}_{2}^{\dag}\mathscr{H}_{2}\Bigr) \Biggr]  \:. \label{28.6.28}
\end{equation}
(因为标量夸克质量由两圈图给出, 这是一个三圈贡献. 势中还有\,Higgs\,和标量夸克超场的二次项, 来源于\,Higgs\,和标量夸克项在$\,SU(2)\times U(1)\,$规范场$\,D\,$-分量平方中的乘积. 它们对\,Higgs\,质量没有三圈贡献, 这是因为标量夸克的每个$\,SU(2)\times U(1)\,$量子数之和为零, 这使得它们对$\,SU(2)\times U(1)\,$ 规范场$\,D\,$-分量的贡献有一个零期望值.) %
比较方程(\ref{28.6.28})与(\ref{28.5.9})并给质量加上两圈贡献(\ref{28.6.23}), 我们就看到
\begin{equation}
    m_{1}^{2} \simeq M_{L}^{2}-\frac{3M_{Q}^{2}\lvert\lambda_{b}\rvert^{2}}{8\uppi^{2}}
    \ln \Biggl(\frac{M^{2}}{M_{Q}^{2}}\Biggr) \:, \label{28.6.29}
\end{equation}
\begin{equation}
    m_{2}^{2} \simeq M_{L}^{2}-\frac{3M_{Q}^{2}\lvert\lambda_{t}\rvert^{2}}{8\uppi^{2}}
    \ln \Biggl(\frac{M^{2}}{M_{Q}^{2}}\Biggr) \:, \label{28.6.30}
\end{equation}
使用方程(\ref{28.6.14})以及$\,\lvert\lambda_{t}\rvert=m_{t}/v_{2}=m_{t}(2\sqrt{2}G_{F})^{1/2}/\sin\beta$, %
我们可以将方程(\ref{28.6.30})写成
\begin{align}
    m_{2}^{2} &\simeq 2TM^{2}\Biggl[\frac{3}{4}\Biggl(\frac{g^{2}}{16\uppi^{2}}\Biggr)^{2}
    +\frac{5}{12}\Biggl(\frac{g^{\prime2}}{16\uppi^{2}}\Biggr)^{2} \nonumber \\
    &\quad -\frac{\sqrt{2}G_{F}m_{t}^{2}}{\uppi^{2}\sin^{2}\beta}\Biggl(\frac{g^{2}_{s}}{16\uppi^{2}}\Biggr)^{2}
    \ln\Biggl(\frac{3}{8T(g_{s}^{2}/16\uppi^{2})}\Biggr)\Biggr]\:. \label{28.6.31}
\end{align}
对$\,T=1$, $g_{s}^{2}/4\uppi=0.118$, $g^{2}/4\uppi=0.0340$, $g^{\prime2}/4\uppi=0.0101$, 以及$\,m_{t}=180\,\mathrm{GeV}$, %
这是
\begin{equation}
    m_{2}^{2}\simeq M_{L}^{2}\Biggl[ 1-\frac{3.06}{\sin^{2}\beta}\Biggr] \:, \label{28.6.32}
\end{equation}
它对于所有$\,\beta\,$值都是负的, 因此提供了电弱规范对称性自发破缺的一个自然机制. 另外, $M_{L}^{2}=(0.91\times 10^{-4})M^{2}/8\uppi^{2}.$ 除非$\,\tan\beta\,$非常大, 我们有$\,\lvert\lambda_{b}\rvert\ll\lvert\lambda_{t}\rvert$, %
所以方程(\ref{28.6.27})给出
\begin{equation}
    m_{1}^{2}\simeq m_{L}^{2} \:. \label{28.6.33}
\end{equation}


在电弱唯象学方面, 规范传递的超对称模型给出的预测中最不确定且最不让人满意与超对称保留项%
$\,\mu[(H_{1}^{\mathrm{T}}eH_{2})]_{\mathscr{F}}\,$中的参量$\,\mu\,$以及拉格朗密度中相关的超对称破缺项$\,B\mu\,$有关.
它们相互关联是因为\,Higgs\,超场与规范超场, 轻子超场以及夸克超场的相互作用在如下的对称性下是不变的,
\begin{align}
    &H_{1}\to \me^{\mi\varphi}H_{1}\:, \qquad H_{2} \to \me^{\mi\varphi}H_{2} \:, \nonumber \\
    &Q\to \me^{-\mi\varphi}Q \:, \qquad \:\: V_{i}\to V_{i} \:, \label{28.6.34} \\
    &\bar{D}\to \bar{D} \:, \qquad \qquad \bar{U}\to\bar{U} \:, \nonumber
\end{align}
当没有超势项$\,\mu(H_{1}^{\mathrm{T}}eH_{2})\,$时, 这将阻止辐射修正在标量场势中产生%
$\,B\mu\operatorname{Re}(\mathscr{H}_{1}^{\mathrm{T}}e\mathscr{H}_{2})$.

$B\mu\,$不可能为零, 否则方程(\ref{28.5.22})和(我们看到的)$\,m_{A}\neq=0\,$将给出$\,\sin2\beta=0$, 换句话说, %
要么$\,v_{1}=0\,$要么$\,v_{2}=0$, 而这将给出要么所有带$\,-e/3\,$电荷的夸克以及带电轻子是无质量的, 要么所有带$\,+2e/3\,$电荷的夸克是无质量的. (如果$\,B\mu=0\,$且$\,\mu=0$, 那么这个问题将延续至所有阶, 这是因为$\,v_{1}\,$和$\,v_{2}\,$均有非零的真空期望值意味着在变换(\ref{28.6.34})和电弱$\,U(1)\,$规范变换的任意组合变换下的对称性将是自发破缺的, 所以$\,\mathsf{C}\,$为奇的中性标量将是$\,m_{A}=0\,$的\,Goldstone\,玻色子.) 自然想法是尝试将$\,B\mu\,$的非零期望值解释为如下理论中的辐射修正: %
变换(\ref{28.6.34})下的对称性被拉格朗日密度中的一个超对称项$\,\mu[(H_{1}^{\mathrm{T}}e H_{2})]_{\mathscr{F}}\,$明显破缺了.
这会在信使标度给$\,B\mu\,$一个非常小的值, 但是重整化群效应会在低能处极大地增强这个值. 根据方程(\ref{28.5.22}), %
一个相对小的$\,B\mu\,$与顶夸克的大质量来自于大的$\,\tan\beta\,$值这个想法很相符. 无论如何, %
带荷微子质量上的实验下界告诉我们$\,\mu\,$至少约$\,60\,\mathrm{GeV}$.

$\mu\,$的非零值引起的麻烦是它带回了超对称本要解决的等级问题: 取代为什么拉格朗日密度中的\,Higgs\,质量项为什么远小于普朗克质量或者规范耦合统一时的质量, 我们现在必须要问为什么$\,\mu\,$这么小?

如果\,Higgs\,超场与超对称破缺部分的相互作用方式使得$\,\mu[(H_{1}^{\mathrm{T}}eH_{2})]\,$被某个对称性禁止掉, 而在这个对称性自发破缺时出现, 那么等级问题就能得以解决. 如果对称性是离散而非连续的, 那么就可以避免出现无质量\,Goldstone\,玻色子. %
最简单的可能性就是扩张对称变换(\ref{28.6.34})以包含变换
\[
S\to \me^{-2\mi\varphi}S \:,
\]
这个变换将会允许超势中有如下形式的项
\[
\lambda^{\prime}S(\mathscr{H}_{1}^{\mathrm{T}}e\mathscr{H}_{2}) \:.
\]
我们可以通过在超势中再引入一个$\,S^{3}\,$项来避免连续对称性, 这使得仅在$\,\varphi\,$是$\,2\uppi/3\,$的倍数时, 拉格朗日量在变换下才是不变的, 而这足以禁止$\,\mu\,$有非零的裸值. 在这一情况下, %
$S\,$的标量分量和辅助分量的非零期望值$\,\mathscr{S}\,$和$\,\mathscr{F}\,$给出
\[
B\mu=\lvert \lambda^{\prime}\mathscr{F}\rvert \:, \qquad \mu=\lvert \lambda^{\prime}\mathscr{S}\rvert \:.
\]
这将导致$\,B\,$有一个非常大的$\,\lambda\,$-相关值$\,M$, 由方程(\ref{28.6.12})给定, %
它要比标量夸克或胶微子的质量大一个约为$\,(g_{s}^{2}/16\uppi)^{-1}\simeq 100\,$的因子. 但另一方面, %
由于方程(\ref{28.6.32})和(\ref{28.6.33})给出$\,m_{1}^{2}+m_{2}^{2}<0$, 稳定性条件(\ref{28.5.11})将要求$\,\lvert\mu\rvert \geq M/2$, 因此方程(\ref{28.5.22})将要求$\,m_{A}\,$也远大于标量夸克和标量轻子的质量. 除非$\,\tan\beta\,$非常接近于一, %
否则这被关系(\ref{28.5.23})以及$\,m_{1}^{2}\,$和$\,m_{2}^{2}\,$的估计值(\ref{28.6.33})和 (\ref{28.6.32})排除了.

我们会在\,\ref{sec:31.6}\,节看到, 引力传递的超对称破缺理论自然地给出了可接受的$\,B\mu\,$值和$\,\mu\,$值. 这种理论由一个非常高的超对称破缺能标表征, 在各个版本中不是$\,10^{11}\,\mathrm{GeV}\,$就是$\,10^{13}\,\mathrm{GeV}$. 有数个方案\cite{35}谋求在超对称性在相对低的能量处破缺的理论中获得可接受的$\,B\mu\,$和$\,\mu\,$值, 例如规范传递的超对称破缺理论, 但没有一个特别引人注目. 另外, 由于我们不知道$\,\mu\,$来自哪里, 我们没有任何理由认为它是实的, %
所以规范传递超对称破缺的理论有着产生太多$\,\mathsf{CP}\,$破坏的风险, 和$\,\ref{sec:28.4}\,$节更一般框架中的情况一样.

同标量质量平方$\,m_{1}^{2}\,$和$\,m_{2}^{2}\,$一样, 方程(\ref{28.4.1})中的参量$\,A_{ij}\,$和$\,C_{ij}\,$由两圈图给出. 然而它们的量纲是质量而非质量平方, 并且它们远小于标量和规范微子的质量, 所以它们对超对称破缺的贡献相对而言就不那么重要.

就像任何超对称在能量远低于$\,10^{10}\,\mathrm{GeV}\,$处破缺的模型一样, 在所有基于规范传递超对称破缺的模型中, %
$R\,$为奇的最轻粒子是引力微子. 而我们将在\,\ref{sec:31.3}\,节看到, 引力微子质量的量级是$\,\sqrt{G}\,$乘以表征超对称破缺的能量平方$\,F$, %
$F\,$的定义使得真空能是$\,F^{2}/2$. 当超对称是被$\,SU(3)\times SU(2)\times U(1)\,$-中性手征超场$\,S_{n}\,$的%
$\,\mathscr{F}\,$-项$\,\mathscr{F}_{n0}\,$破缺时, 我们有$\,F^{2}=\sum_{n}\lvert\mathscr{F}_{n0}\rvert^{2}$. %
如果$\,S_{n}\,$的拉格朗日量中没有大的无量纲参量, 那么这一模型中的标量夸克质量是$\,g_{s}^{2}\sqrt{F}/16\uppi^{2}\approx10^{-2}\sqrt{F}\,$阶的, %
所以为了使其小于自然性给出的上界$\,10^{4}\,\mathrm{GeV}$, 我们必须有$\,\sqrt{F}<10^{6}\,\mathrm{GeV}$, 这给出的引力微子质量小于$\,1\,\mathrm{keV}$. %
引力耦合在可达到能标处是如此的弱以至于真正能够产生的是引力微子螺旋度为$\,\pm1/2$ 的态, 而这些态的行为就像戈德斯通微子态. %
正如方程(\ref{27.5.12})表明的, 在这里讨论的模型中, 戈德斯通微子场伴随着系数$\,\mi\sqrt{2}\mathscr{F}_{n}\,$出现在$\,S_{n}\,$的费米分量$\,\psi_{n}\,$中. %
戈德斯通微子在$\,R\,$-奇粒子到标准模型中相应的$\,R\,$-偶粒子的衰变中通过辐射修正发射出来, 在这个过程中, %
戈德斯通微子从连接内$\,\psi_{n}\,$线和内$\,\mathscr{S}_{n}\,$线的顶点中浮现出来. 根据方程(\ref{29.2.10}), 戈德斯通微子的发散振幅与$\,F\,$成反比, %
这使得这些衰变变慢很多, 尽管如此, 这些衰变速率仍可能快到足以探测.

由于$\,R\,$-奇粒子到戈德斯通微子的衰变很慢, 唯象学上很重要的一件事是找出这些模型中次轻的$\,R\,$-奇粒子, %
所有更重的$\,R\,$-奇粒子在衰变到戈德斯通微子之前都会先衰变到这个粒子上. 而我们已经看到, 次轻的$\,R\,$-奇粒子通常是标量轻子, $W\,$微子或$\,B\,$微子. %
(在信使超场拥有与\,Higgs\,双重态相同$\,SU(3)\times SU(2)\times U(1)\,$量子数的模型中, 这些超场的混合可能极大地拉低双重态信使的质量以至于最轻的$\,R\,$-奇粒子是胶微子.\cite{35a}) 将$\,SU(2)\times U(1)\,$破缺效应计入在内的细致计算表明, %
次轻的$\,R\,$-奇粒子是两个$\,\tau\,$标量轻子之一的参数空间很大.\cite{35b}

\section{重子和轻子不守恒} \label{sec:28.7}

超对称模型中的额外粒子为重子和轻子不守恒提供了数个新机制. 我们在\,\ref{sec:28.1}\,节看到, %
有数个量纲为\,4\,的重子数和轻子数不守恒超对称算符(\ref{28.1.2})和(\ref{28.1.3})可以被引入到可重整的$\,SU(3)\times SU(2)\times U(1)\,$理论中, %
并且这些算符能够导致质子衰变这样的过程有一个灾难性的速率. 这些项可以通过附加$\,R\,$宇称守恒(或者等效地, %
在所有夸克和轻子手征超场的符号改变下的不变性)从拉格朗日量中排除出去, %
但这并不会排除各种$\,SU(3)\times SU(2)\times U(1)\,$不变但量纲$\,d>4\,$的重子数和轻子数不守恒算符. %
就像\,\textcolor{foo}{21.3}\,节中讨论的那样, 如果重子数和轻子数不守恒有一个由很高质量标度$\,M\,$表征的底层机制, %
那么这些算符将带着正比于$\,M^{4-d}\,$的系数出现在标准模型的有效拉格朗日量中. 当只有非超对称标准模型中的场时, %
能够破坏重子数守恒的算符的最小量纲是\,6,\cite{36} 因此重子数不守恒振幅将正比于$\,M^{-2}$. 对于质子和受约束中子衰变这样的重子数不守恒过程, %
超对称要求的新场会导致对它们的估计有两个重要的改变. 就像我们在\,\ref{sec:28.2}\,节看到的, 重整化群方程中的改变给出了对$\,M\,$更大的估计, %
这降低了量纲为\,6\,的算符的效应. 与此同时, 这些新场使得我们可以构建量纲为\,5\,的新算符, 而这会给出正比于$\,M^{-1}\,$的重子数不守恒振幅, %
因而很可能对质子和受约束中子的衰变给出主导贡献.


\begin{figure}[t]
  \centering

\begin{tikzpicture}

  \draw[thick] (0,0)--(1,0.5) ;
  \draw[thick, decorate, decoration={markings,mark=at position 1 with {\arrow[scale=1.5]{>}}},shorten >=0.2pt] (0,0) --(1,0.5);
  \draw[thick] (1,0.5)--(2,1) ;
  \draw[thick] (1,1.5)--(2,1) ;
  \draw[thick] (0,2)--(1,1.5) ;
  \draw[thick, decorate, decoration={markings,mark=at position 1 with {\arrow[scale=1.5]{>}}},shorten >=0.2pt]  (0,2)--(1,1.5);
  \draw[thick,dashed] (2,1) -- (4,2);
  \draw[thick,dashed] (2,1) -- (4,0);
  \draw[thick] (4,2)-- (4,0);
  \draw[thick,decorate, decoration={snake,amplitude=3pt,segment length=13.5pt}] (4,2)-- (4,0);
  \draw[thick] (4,2) -- (4.6,2.3);
  \draw[thick] (4,0) -- (4.6,-0.3);
  \draw[thick] (5.6,2.8) -- (4.6,2.3);
  \draw[thick, decorate, decoration={markings,mark=at position 1 with {\arrow[scale=1.5]{>}}},shorten >=0.2pt]  (5.6,2.8) -- (4.6,2.3);
  \draw[thick] (5.6,-0.8) -- (4.6,-0.3);
  \draw[thick, decorate, decoration={markings,mark=at position 1 with {\arrow[scale=1.5]{>}}},shorten >=0.2pt] (5.6,-0.8) -- (4.6,-0.3);
  \node at (2,1) [circle, fill=black, inner sep=0pt,minimum size=4pt] {} ;
\end{tikzpicture}

  \vspace{5mm}

  \caption{可以在夸克和(或)轻子之间产生破坏重子和轻子数守恒的四费米子相互作用的图. 这里实线是夸克和(或)轻子; 虚线是标量夸克和(或)标量轻子; 实线和波浪线的组合线是规范微子; 圆点是直接来自于$\,\mathscr{F}$-项相互作用(\ref{28.7.3})的顶点.}
  \label{fig:28.7}%
\end{figure}


由手征超场(一般记做$\,\Phi\,$)构成的量纲为\,5\,的超对称算符有如下形式: $(\Phi^{\ast}\Phi\Phi)_{D}\,$和$\,(\Phi\Phi\Phi\Phi)_{\mathscr{F}}$, 以及它们的复共轭. (我们不考虑含有导数或规范场的算符, 因为它们不对重子数或轻子数不守恒提供额外的可能性.) 按照\,\ref{sec:28.1}\,节的符号约定, %
量纲为$\,5\,$同时保留$\,R\,$宇称的$\,SU(3)\times SU(2)\times U(1)$-不变算符是
\begin{equation}
    (LLH_{2}H_{2})_{\mathscr{F}} \:, \label{28.7.1}
\end{equation}
\begin{equation}
    (L\bar{E}H_{2}^{\ast})_{D}\:,\quad (Q\bar{D}H_{2}^{\ast})_{D} \:, \quad (Q\bar{U}H_{1}^{\ast})_{D}\:,\quad
    (QQ\bar{U}\bar{D})_{\mathscr{F}}\:,\quad (Q\bar{U}L\bar{E})_{\mathscr{F}} \:,\label{28.7.2}
\end{equation}
以及
\begin{equation}
    (QQQL)_{\mathscr{F}}\:, \qquad (\bar{U}\bar{U}\bar{D}\bar{E})_{\mathscr{F}}\:, \label{28.7.3}
\end{equation}
其中, 如$\,SU(3)\,$和$\,SU(2)\,$守恒表明的, 相应的指标收缩掉了. %
相互作用(\ref{28.7.1})是一些理论中量纲为\,5\,的算符的超对称版, 原始版将产生微小中微子质量.\cite{38} %
相互作用(\ref{28.7.2})仅为超对称标准模型的可重整项中已经发生的过程提供一个小修正. 真正为重子数和轻子数不守恒提供新机制是相互作用(\ref{28.7.3}).


根据方程(\ref{26.4.4}), 夸克和轻子通过包含一对夸克和(或)轻子场以及一对标量夸克和(或)标量轻子场的项进入到相互作用(\ref{28.7.3})中. %
为了生成仅发生在夸克和轻子之间的反应, 标量夸克和(或) 标量轻子对必须像图\,\ref{fig:28.7}\,中所示的那样通过在单圈图中交换一个规范微子转换成夸克和(或)轻子对. 这将会在三个夸克和一个轻子之间产生一个$\,d=6\,$的有效四费米子$\,qqq\ell\,$相互作用. 这些相互作用的耦合将正比于规范微子的规范耦合$\,g\,$%
或$\,g^{\prime}\,$或$\,g_{s}\,$的平方, 正比于胶微子的超对称破缺质量, 平方反比于胶微子和标量夸克或标量轻子中质量较大的那一个(这被用来赋予耦合正确的量纲), 以及正比于一个来自圈积分的$\,1/8\uppi^{2}\,$因子.


可以认为是胶微子耦合强度更大使得对$\,g_{6}\,$的主导贡献是胶微子交换. (诚然, 在规范传递超对称破缺的理论中, %
对于迹(\ref{28.6.3})的恰当值再加上$\,g\approx g'$, 方程(\ref{28.6.13})---(\ref{28.6.14})给出
\begin{align*}
    &m_{\text{gluino}}\approx m_{\text{squark}} \approx \frac{g_{s}^{2}}{16\uppi^{2}}M_{\ast} \:, \nonumber \\
    &m_{\text{wino}}\approx m_{\text{slepton}} \approx m_{\text{bino}}\approx \frac{g^{2}}{16\uppi^{2}}M_{\ast} \:, \nonumber \\
\end{align*}
其中$\,M_{\ast}\,$是表征信使部分的质量. 因此在这样的理论中, 各个胶微子交换图给出的贡献正比于 $g_{s}^{2}/m_{\text{gluino}}$, %
而$\,W\,$微子或$\,B\,$微子交换(或者更准确些, 带荷微子或中性微子交换)给出的贡献正比于$\,g^{2}m_{\text{wino}}/m^{2}_{\text{squark}}$, %
它要小一个约等于$\,m_{\text{wino}}g^{2}/m_{\text{gluino}}\approx g^{4}/g_{s}^{4}\,$的因子.) 然而胶微子交换图之间有一个抵消强烈地压低了它们的贡献.
这最早是通过在四费米子算符之间使用一个\,Fierz\,恒等式显现出来的,\cite{39} 但即使没有任何方程也可以获得相同的结果. %
为了使色守恒, 算符$\,(QQQL)_{\mathscr{F}}\,$和 $(\bar{U}\bar{U}\bar{D}\bar{E})_{\mathscr{F}}\,$的系数关于三个夸克或反夸克超场的色必须是反对称的, %
而由于这些超场是玻色的, 它们同时关于这些超场的味也必须是反对称的. 胶微子相互作用与味无关, 所以如果我们能够忽略标量夸克质量的味相关性, %
那么$\,d=6\,$的四费米子算符的系数对于味和色也将是全反对称的. 这样, 费米统计将要求这些算符的系数对夸克和反夸克场的自旋指标也是全反对称的. %
但$\,d=6\,$算符中通过胶微子交换从$\,(QQQL)_{\mathscr{F}}\,$和$\,(\bar{U}\bar{U}\bar{D}\bar{E})_{\mathscr{F}}\,$派生出来的三个夸克或反夸克场%
全部都是左手的因而只有两个独立的自旋指标, 所以没有系数对所有三个自旋是反对称的. 因此, 如果标量夸克质量都相等, %
那么胶微子交换对$\,d=6\,$算符的贡献将是零, 所以这个贡献被不同夸克之间的微小质量差压低了. 根据方程(\ref{28.6.4}), 在规范传递超对称破缺的理论中, %
$\bar{U}\,$和$\,\bar{D}\,$标量夸克之间的微小质量差是$\,g'^{4}/g_{s}^{4}\,$阶的, %
所以$\,(\bar{U}\bar{U}\bar{D}\bar{E})_{\mathscr{F}}\,$算符中的反标量夸克之间交换胶微子生成了一个量纲为\,6\,的四费米子相互作用, %
其系数与交换$\,B\,$微子生成的相互作用的系数处于同一量级. 然而, 由于胶微子保持味不变, %
这个算符同$\,(\bar{U}\bar{U}\bar{D}\bar{E})_{\mathscr{F}}\,$算符一样对于反夸克味也必须是全反对称的, %
这使得它必须包含$\,c\,$夸克或$\,t\,$夸克因而无法直接贡献到质子或束缚中子的衰变上. 另一方面, %
方程(\ref{28.6.4})表明不同味的$\,Q\,$夸克之间的微小质量差远小于$\,g^{4}/g_{s}^{4}\,$阶, %
这使得算符$\,(QQQL)_{\mathscr{F}}\,$中的标量夸克之间交换胶微子对$\,g_{5}\,$的贡献远小于交换$\,W\,$微子或$\,B\,$微子. %
我们由此得出结论: 至少在规范传递超对称破缺的理论中, 胶微子交换对质子或束缚中子衰变的贡献要小于$\,W\,$微子或$\,B\,$微子交换. %
在其它模型中, 胶微子交换可能对其它过程有可观的贡献.\cite{40}

当$\,g\approx g'\,$且$\,m_{\text{wino}}\approx m_{\text{bino}}\,$时, 交换$\,W\,$微子或$\,B\,$微子对量纲为\,6\,的算符的贡献在量级上是
\begin{equation}
    g_{6} \approx \frac{g^{2}\,g_{5}\,m_{\text{wino}}}{8\uppi^{2}m^{2}_{\text{squark}}} \:, \label{28.7.4}
\end{equation}
其中$\,g_{5}\,$是$\,d=5\,$有效相互作用(\ref{28.7.3})的耦合的特征值. %
如果$\,W\,$微子和标量夸克质量有着在规范传递超对称破缺理论中的比值$\,(g^{2}/g_{s}^{2})$, 那么这给出
\begin{equation}
    g_{6} \approx \frac{g^{4}\,g_{5}\,}{8\uppi^{2}g_{s}^{2}m^{2}_{\text{squark}}} \:. \label{28.7.5}
\end{equation}

有效拉格朗日中量纲为\,6\,的四费米子$\,qqq\ell\,$-项与那些在非超对称理论中用来生成质子衰变这种过程的项相同.\cite{36} %
基于量纲分析, 它们产生质子和束缚中子衰变的速率必是如下的形式
\begin{equation}
    \Gamma_{N}=c_{N}\,m_{N}^{5}\,g_{6}^{2} \:, \label{28.7.6}
\end{equation}
其中$\,c_{N},$是必须在量子色动力学中通过非围绕计算得出的纯数. 在这些计算中投入了大量的工作, %
而结果\cite{41}一般处在$\,c_{N}\approx 3\times 10^{-3\pm0.7}\,$的范围内.

为了估计$\,g_{5}$, 我们注意到在树级近似下通过交换规范超多重态是不可能产生像(\ref{28.7.3})这样只包含左手征超场的$\,\mathscr{F}\,$项, %
规范超多重态总是既和左手征超场又和它们的右手征共轭相互作用. 因此在树级近似下, 相互作用(\ref{28.7.3})仅来自于交换手征超场的粒子, %
所以$\,g_{5}\,$是$\,g_{T}^{2}/M_{T}\,$阶的,\footnoteB{原书为$\,g_{T}^{2}/M_{T}^{2}$, 疑有笔误.\qquad ------译者注.} %
其中$\,g_{T}\,$是某个质量为$\,M_{T}\,$的超重左手征超场与夸克和轻子超场的重子和轻子不守恒特征耦合. %
为了产生相互作用(\ref{28.7.3}), 这些超重粒子必须是色三重态或反三重态, 以及$\,SU(2)\,$三重态或单态. 无论统一强和电弱作用的规范群是什么, %
它都大概指定了超重色三重态$\,T\,$与熟悉的色单态$\,H_{1}\,$和$\,H_{2}$ 的相互作用之间的某个关系. %
这样, $\,g_{T}\,$将与相互作用(\ref{28.1.2})和(\ref{28.1.3})中的\,Yukawa\,耦合处在同一量级, 而这两个相互作用给已知夸克和轻子赋予质量, %
这两个耦合等于夸克或轻子的质量除以 $\mathscr{H}_{1}^{0}\,$或$\,\mathscr{H}_{2}^{0}\,$的真空期望值, 约为$\,G_{F}^{-1/2}\simeq 300\,\mathrm{GeV}$. %
我们因此取
\begin{equation}
    g_{5} \approx \frac{G_{F}\,m_{f}^{2}}{M_{T}} \:, \label{28.7.7}
\end{equation}
其中$\,m_{f}\,$是夸克或轻子的某个特征质量. 我们已经看到量纲为\,5\,的算符关于夸克味是反对称的, %
所以为了协调$\,s\,$夸克与$\,u\,$或$\,d\,$夸克的质量, 我们将取$\,m_{f}=30\,\mathrm{MeV}$. %
结合方程(\ref{28.7.5})---(\ref{28.7.7}), 并取$\,M_{T}=2\times 10^{16}\,\mathrm{GeV}\,$(这是\,\ref{sec:28.2}\,节结果给出的值), %
$c_{N}=0.003$, $g_{s}^{2}/4\uppi=0.118$, $g^{2}/4\uppi=1/(0.23\times 137)$, 以及$\,m_{\text{squark}}=1\,\mathrm{TeV}$, %
我们发现质子(或束缚中子)的寿命$\,\Gamma_{N}^{-1}\,$大约是$\,2\times 10^{31}\,$年.\cite{42} %
对于质子衰变预期的领头模型给出的分部寿命(partial lifetimes), 实验给出的下界一般分布在$\,10^{31}\,$至$\,5\times 10^{32}\,$年之内, %
上面给出的估计与此相差不大. 在本书写作之际, 日本的超级神岗中微子探测器没有观测到质子衰变对此给出了最严格的限制:\cite{42a} %
衰变$\,p\to e^{+}\pi^{0}\,$和$\,p\to \bar{\nu}K^{+}\,$的分部寿命分别要大于$\,2.1\times 10^{33}\,$年和$\,5.5\times 10^{32}\,$年. %
上面对理论寿命的估计至少有一个为\,100\,的不确定性, 它仅来自于标量夸克质量的不确定性, 所以现在说实验和理论预期之间有任何矛盾还为时尚早. %
另一方面, 超对称提高了提前发现重子不守恒的可能性.


就各种质子和束缚重子衰变模型预期的分支比, 我们也可以谈论一点一般性质. 正如我们前面提及的, %
量纲为\,5\,的算符(\ref{28.7.3})关于夸克超场的味必须是全反对称的, 所以我们只关心形如 $(U_{i}D_{j}D_{k}N_{\ell})_{\mathscr{F}}$, %
$(D_{i}U_{j}U_{k}E_{\ell})_{\mathscr{F}}\,$以及$\,(\bar{D}_{i}\bar{U}_{j}\bar{U}_{k}\bar{E}_{\ell})_{\mathscr{F}}\,$的算符, %
其中$\,i,j,k,\ell\,$是代指标, 且每个情况下有$\,j\neq k$. 这样交换中性$\,W\,$微子或$\,B\,$微子会产生形如$\,u_{i}d_{j}d_{k}v_{\ell}$, %
$d_{i}u_{j}u_{k}e_{\ell}\,$以及$\,\bar{d}_{i}\bar{u}_{j}\bar{u}_{k}\bar{e}_{\ell}\,$这样$\,d=6\,$的四费米子算符, %
其中$,j\neq k\,$而$\,i\,$和$\,\ell\,$任意, 交换带荷$\,W\,$微子会产生相同形式但$\,i\neq j\,$而$\,k\,$和$\,\ell\,$任意的四费米子算符.
足够轻以至于和质子衰变有关的夸克只有$\,u$, $s\,$和$\,d$; 忽视其它所有夸克以及第三代中的小混合角, 我们有
\[
    u_{1}=u \:, \qquad d_{1}=d\cos\theta_{c}+s\sin\theta_{c}\:,  \qquad d_{2}=-d\sin\theta_{c}+s\cos\theta_{c} \:,
\]
其中$\,\theta_{c}\,$是\,Cabibbo\,角, 而$\,u_{2}$, $u_{3}\,$和$\,d_{3}\,$可以被忽略掉. 因此, %
能够通过交换$\,W\,$微子和$\,B\,$微子以及能够贡献到质子或束缚中子衰变上的四费米子算符是$\,u\,d\,s\,\nu_{\ell}\cos(2\theta_{c})$, %
$u\,d\,d\,\nu_{\ell}\sin(2\theta_{c})$, $u\,u\,s\,e_{\ell}\cos\theta_{c}$ 和$\,u\,u\,d\,e_{\ell}\sin\theta_{c}$, %
再加上夸克和轻子被换成反夸克和反轻子的其它算符. 所有其它的事情是相同的, 因此主导衰变模型是$\,p\to K^{+}\bar{\nu}$, %
$n\to K^{0}\bar{\nu}$, $p\to K^{0}e^{+}$, 以及$\,p\to K^{0}\mu^{+}$, 而对于衰变模型$\,p\to \pi^{+}\bar{\nu}$, $n\to \pi^{0}\bar{\nu}$ %
$p\to \pi^{0}e^{+}$, $p\to\pi^{0}\mu^{+}\,$和$\,n\to\pi^{-}e^{+}$的速率, 尽管会被较大的可用相空间稍许扩大, %
但是会被因子$\,\sin^{2}\theta_{c}=0.05\,$压低.

这些讨论并不会对分支比给出明确的预测, 这是因为除了前面提到的所有因子以外, 算符(\ref{28.7.3}) 的系数, 一般被称为$\,g_{5}$, %
可能对出现这些算符中的超场味有一个很强的依赖. 为了更进一步, 我们需要一个特定的理论来生成量纲为\,5\,的算符. 基于$\,SU(5)\,$理论的超对称版, %
参考文献[42]中的大多数作者得出的结论是, 主导质子和束缚中子衰变的是过程$\,p\to K^{+}\bar{\nu}\,$和$\,n\to K^{0}\bar{\nu}$, %
但对于基于$\,SO(10)\,$的一个模型, 带荷轻子模会变得显著.\cite{43} 另外, 在一些模型中, 希格斯微子交换会与$\,W\,$微子和$\,B\,$微子交换相竞争,\cite{44} %
提高$\,p\to K^{+}\bar{\nu}\,$的速率. 看起来在搜寻重子不守恒时的一个好注意是对质子或束缚中子衰变中的衰变模敞开想法.

当然, 所有这些重子不守恒过程有可能被某个守恒律禁止. 就像\,\ref{sec:28.1}\,节中提到过的, 弦论反对重子守恒成为一个基本的整体对称性, %
但重子不守恒算符(\ref{28.7.3})可能被一个称为重子宇称的$\,\mathds{Z}_{3}\,$的乘性对称性给禁止掉, $Q\,$超场在这个对称性是中性的; %
$H_{2}\,$和$\,\bar{D}\,$超场则要乘以相位$\,\exp(\mi\uppi/3)$, 而$\,L$, $H_{1}$, $\bar{U}\,$和$\,\bar{E}\,$超场则要乘以相反的相位$\,\exp(-\mi\uppi/3)$. %
这个对称性使得基本\,Yukawa\,耦合(\ref{28.1.2}) 和(\ref{28.1.3})以及$\,\mu\,$-项(\ref{28.5.7})和轻子不守恒项(\ref{28.1.4})和(\ref{28.7.1})可以存在, %
但排除了量纲为\,4\,的重子不守恒项(\ref{28.1.5})和量纲为\,5\,的重子不守恒项(\ref{28.7.3}). %
这个对称性因$\,\mathscr{H}_{1}^{0}\,$和$\,\mathscr{H}_{2}^{0}\,$(可能还有标量中微子场$\,\mathscr{N}$)有真空期望值自发破缺, %
且当$\,R\,$宇称没有守恒律时, 没有什么东西可以保持最轻的超对称粒子是稳定的.


\section*{习题}
\noindent 1. 假定相互作用(\ref{28.1.4})和(\ref{28.1.5})真的出现超对称版标准模型的拉格朗日量中. 粗略地估计一下为了不与质子寿命的实验下界产生矛盾, %
标量夸克和标量轻子要有多重. \\

\noindent 2. 假定规范微子, 希格斯微子, 标量夸克和标量轻子的特征质量$\,m\,$远小于$\,m_{Z}$. %
给出跑动耦合常数在能量处在$\,m,\,$之上和之下时的重整化群方程. 利用这个结果, 再加上\,\ref{sec:28.2}\,节采用的统一假定, 给出用$\,m$, $m_{Z}$, %
$e(m_{Z})$, $g_{s}(m_{Z})$和$\,n_{s}\,$表示$\,\sin^{2}\theta\,$和统一标度$\,M\,$的公式. 在不违反实验给出$\,\sin^{2}\theta\,$和$\,M$ 的上下界的情况下, $m\,$能有多大? \\

\noindent 3. 给出夸克和轻子与最小超对称标准模型中最轻的$\,CP\,$-偶中性标量粒子的耦合的公式, %
用参量$\,m_{A}$, $m_{Z}$, $\beta$, $G_{F}\,$以及夸克和轻子质量表示. \\

\noindent 4. 在一个规范传递超对称破缺的理论中, 其中信使超场$\,\Phi_{n}\,$和$\,\bar{\Phi}_{n}\,$从%
$\,\sum_{n}\lambda_{n}S_{n}(\bar{\Phi}_{n}\Phi_{n})\,$中获得它们的质量, %
将单态超场$\,S_{n}\,$的$\,\phi\,$-分量和$\,\mathscr{F}\,$-分量的真空期望值记做$\,\mathscr{S}_{n}\,$和$\,\mathscr{F}_{n}$, %
在$\,\lvert\mathscr{F}_{n}\rvert\ll\lvert\lambda_{n}\rvert\lvert\mathscr{S}_{n}\rvert^{2}\,$的极限下, %
利用全纯讨论推导出胶微子质量用$\,\mathscr{S}_{n}\,$和$\,\mathscr{F}_{n}\,$表示的单圈公式. \\

\noindent 5. 将标量夸克质量对味有一个小依赖性的可能性考虑在内, 估计胶微子交换对夸克和轻子之间的重子和轻子不守恒四费米子相互作用的贡献. %
利用从$\,K^{0}\to \overline{K}^{0}$转化率得出的标量夸克质量劈裂上下界, 给这些贡献设立一个上界.

%++++++++++++++++++参考文献+++++++++
\renewcommand{\sectionmark}[1]{\markright{ #1}{}}
\renewcommand{\bibname}{参考文献}

\begin{thebibliography}{99}
    \bibitem{1} 我们这里不会考虑在低得多的能量处统一的可能性, 这由以下诸人提出: I. Antoniadis, {\textit{Phys. Lett.}} {\bf{B246}}, 377 (1990); J. Lykken, {\textit{Phys. Rev.}} {\bf{D54}}, 3693 (1996), 并被以下诸人复兴: N. Arkani-Hamed, S. Dimopoulos, and G. Dvali, {\textit{Phys. Lett.}} {\bf{B429}}, 263 (1998); K. R. Dienes, E. Dudea, and T. Gherghetta, {\textit{Phys. Lett.}} {\bf{B436}}, 55 (1998); I. Antoniadis, N. Arkani-Hamed, S. Dimopoulos, and G. Dvali, {\textit{Phys. Rev. Lett.}} {\bf{B436}}, 257 (1998).
    \bibitem[1a]{1a} S. Weinberg, 收录于\,{\textit{Proceedings of the XVII International Conference on High Energy Physics, London, 1974}}, J. R. Smith\, 编辑\,(Rutherford Laboratory, Chilton, Didcot, England, 1974); S. Weinberg, {\textit{Phys. Rev.}} {\bf{D13}}, 974 (1976); E. Gildener and S. Weinberg, {\textit{Phys. Rev.}} {\bf{D13}}, 3333 (1976).
    \bibitem[1b]{1b} T. Banks and L. Dixon, {\textit{Nucl. Phys.}} {\bf{B307}}, 93 (1988). 细致的讨论参看, J. Polchinski, {\textit{String Theory}} (Cambridge University Press, Cambridge, 1998): Chapter 18.
    \bibitem{2} 加性\,$R$\,-守恒律由以下诸人引入, A. Salam and J. Strathdee, {\textit{Nucl. Phys.}} {\bf{B87}}, 85 (1975); P. Fayet, {\textit{Nucl. Phys.}} {\bf{B90}}, 104 (1975), 重印于{\textit{Supersymmetry}}, S. Ferrar\,编辑%
        (North Holland/World Scientific, Amsterdam/Singapore, 1987). $R\,$宇称亦可以以一个$\,R\,$量子数的形式定义成%
        $\,\exp(\mi\uppi R)$, 因而即使$\,R\,$不是加性守恒的也可以认为是乘性守恒的; 参看\,G. Farrar and P. Fayet, {\textit{Phys. Lett.}} {\bf{76B}}, 575 (1978); P. Fayet, 收录于\,{\textit{Unification of the Fundamential Particle Interactions}}, S. Ferrata, J. Ellis, and P. van Nieuwenhuizen\,编辑(Plenum, New York, 1980); S. Dimopoulos, S. Raby, and F. Wilczek, {\textit{Phys. Lett.}} {\bf{112B}}, 133 (1982); G. Farrar and S. Weinberg, {\textit{Phys. Rev.}} {\bf{D27}}, 1731 (1983), 重印于{\textit{Supersymmetry}}.
    \bibitem{3} S. Dimopoulos and H. Georgi, {\textit{Nucl. Phys.}} {\bf{B193}}, 150 (1981), 重印于{\textit{Supersymmetry}}, 参考文献[2].
    \bibitem{4} R. D. Peccei and H. Quinn, {\textit{Phys. Rev. Lett.}} {\bf{38}}, 1440 (1977); {\textit{Phys. Rev.}} {\bf{D16}}, 1791 (1977).
    \bibitem[4a]{4a} S. Dimopoulos and G. F. Giudice, {\textit{Phys. Lett.}} {\bf{B357}}, 573 (1995); A. Pomerol and D. Tommasini, {\textit{Nucl. Phys.}} {\bf{B466}}, 3 (1996); G. Dvali and A. Pomerol, {\textit{Phys. Rev. Lett.}} {\bf{77}}, 3728 (1996); {\textit{Nucl. Phys.}} {\bf{B522}}, 3 (1998); A. G. Cohen, D. B. Kaplan, and A. E. Nelson, {\textit{Phys. Lett.}} {\bf{B388}}, 588 (1996); R. N. Mohapatra and A. Riotto, {\textit{Phys. Rev.}} {\bf{D55}}, 1 (1997); R.-J. Zhang, {\textit{Phys. Lett.}} {\bf{B402}}, 101 (1997); H-P. Nilles and N. Polonsky, {\textit{Phys. Lett.}} {\bf{B412}}, 69 (1997). D. E. Kaplan, F. Lepeintre, A. Masiero, A. E. Nelson, and A. Riotto, hep-ph/9806430, 将发表; J. Hisano, K. Kurosawa, and Y. Nomura, {\textit{Phys. Lett.}} {\bf{B445}}, 316 (1999). 辐射修正可以自然地产生这个质量排布: 参看, J. L. Feng, C. Kolda, and N. Polonsky, {\textit{Nucl. Phys.}} {\bf{B546}}, 3 (1999); J. Bagger, J. L. Feng, and N. Polonsky, hep-ph/9905292, 将发表.
    \bibitem[4b]{4b} B. W. Lee and S. Weinberg, {\textit{Phys. Rev. Lett.}} {\bf{39}}, 165 (1977); D. A. Dicus, E. W. Kolb, and V. L. Teplitz, {\textit{Phys. Rev. Lett}}. {\bf{39}}, 168 (1977).
    \bibitem[4c]{4c} S. Wolfram, {\textit{Phys. Lett.}} {\bf{82B}}, 65 (1979); J. Ellis, J. S. Hagelin, D. V. Nanopoulos, K. Olive, and M. Srednicki, {\textit{Nucl. Phys.}} {\bf{B238}}, 453 (1984).
    \bibitem[4d]{4d} P. F. Smith and J. R. J. Bennett, {\textit{Nucl. Phys.}} {\bf{B149}}, 525 (1979).
    \bibitem{5} H. Georgi, H. R. Quinn, and S. Weinberg, {\textit{Phys. Rev. Lett.}} {\bf{33}}, 451 (1974).
    \bibitem{6} S. Dimopoulos and H. Georgi, 参考文献[3]; J. Ellis, S. Kelley, and D. V. Nanopoulos, {\textit{Phys. Lett.}} {\bf{B260}}, 131 (1991); U. Amaldi, W. de Boer, and H. Furstmann, {\textit{Phys. Lett.}} {\bf{B260}}, 447 (1991); C. Giunti, C. W. Kim and U. W. Lee, {\textit{Mod. Phys. Lett.}} {\bf{16}}, 1745 (1991); P. Langacker and M.-X. Luo, {\textit{Phys. Rev.}} {\bf{D44}}, 817 (1991). 关于其它参考文献以及对数据的更细致分析, 参看, P. Langacker and N. Polonsky, {\textit{Phys. Rev.}} {\bf{D47}}, 4028 (1993); {\bf{D49}}, 1454 (1994); L. J. Hall and U. Sarid, {\textit{Phys. Rev. Lett.}} {\bf{70}}, 2673 (1993).
    \bibitem{7} S. Dimopoulos, S. Raby, and F. Wilczek, {\textit{Phys. Rev.}} {\bf{D24}}, 1681 (1981), 重印于{\textit{Supersymmetry}}, 参考文献[2].
    \bibitem[7a]{7a} P. Ho\v{r}ava and E. Witten, {\textit{Nucl. Phys.}} {\bf{B460}}, 506 (1996); {\textit{ibid}}. {\bf{B475}}, 94 (1996); E. Witten, {\textit{Nucl. Phys.}} {\bf{B471}}, 135 (1996); P. Ho\v{r}ava, {\textit{Phys. Rev.}} {\bf{D54}}, 7561 (1996).
    \bibitem{8} H. Pagels and J. R. Primack, {\textit{Phys. Rev. Let.}} {\bf{48}}, 223 (1982).
    \bibitem{9} S. Weinberg, {\textit{Phys. Rev. Lett.}} {\bf{48}}, 1303 (1983).
    \bibitem{10} S. Dimopoulos and H. Georgi, 参考文献[3]; N. Sakai, {\textit{Z. Phys. C}} {\bf{11}}, 153 (1981). 关于综述, 参看\,H. E. Haber and G. L. Kane, {\textit{Phys. Reports}} {\bf{117}}, 75 (1985); J. A. Bagger, 收录于\,{\textit{QCD and Beyond: Proceedings of the Theoretical Advanced Study Institute in Elementary Particle Physics, University of Colorado, June 1995}}, D. E. Soper\,编辑(World Scientific, Singapore, 1996); V. Barger, 收录于\,{\textit{Fundamental Particles and Interactions: Proceedings of the FCP Workshop on Fundamental Particles and Interactions, Vanderbilt University, May 1997}}, R. S. Panvini, T. J. Weiler\,编辑(American Institute of Physics, Woodbury, NY, 1998); J. F. Gunion, 收录于\,{\textit{Quantum Effects in the MSSM -- Proceedings of the International Workshop on Quantum Effects in the MSSM, Barcelona, September 1997}}, J. Sol\`{a}\,编辑(World Scientific Publishing, Singapore, 1998); S. Dawson, 收录于\,{\textit{Proceedings of the 1997 Theoretical Advanced Study Institute on Supersymmetry, Supergravity, and Supercolliders}}, J. Bagger\,编辑(World Scientific, Singapore, 1998); S. P. Martin, %
        收录于\,{\textit{Perspectives on Supersymmetry,}} G. L. kane\,编辑(World Scientific, Singapore, 1998); K. R. Dienes and C. Kolda, 收录于\,{\textit{Perspectives on Supersymmetry}}, 同上.
    \bibitem{11} S. Dimopoulos and D. Sutter, {\textit{Nucl. Phys.}} {\bf{B194}}, 65 (1995); H. Haber, {\textit{Nucl. Phys. Proc. Suppl.}} {\bf{62}}, 469 (1998).
    \bibitem{12} S. Dimopoulos and H. Georgi, 参考文献[3]; J. Ellis and D. V. Nanopoulos, {\textit{Phys. Lett.}} {\bf{110B}}, 44 (1982); J. F. Donoghue, H-P. Nilles, and D. Wyler, {\textit{Phys. Lett.}} {\bf{128B}}. 55 (1983). 关于这些计算的强相互作用修正, 参看\,J. A. Bagger, K. T. Matchev, and R.-J. Zhang, {\textit{Phys. Lett.}} {\bf{B412}}, 77 (1997). 味改变过程上的限制不约束标量夸克质量的条件在下面几处地方进行了讨论, R. Barbieri and R. Gatto, {\textit{Phys. Lett.}} {\bf{110B}}, 211 (1981); Y. Nir and N. Seiberg, {\textit{Phys. Lett.}} {\bf{B309}}, 337 (1993). 关于详细的综述, 参看\,F. Gabbiani, E. Gabrielli, A. Masiero, and L. Silvestrini, {\textit{Nucl. Phys.}} {\bf{B447}}, 321 (1996).
    \bibitem{13} M. K. Gaillard and B. W. Lee, {\textit{Phys. Rev.}} {\bf{D10}}, 897 (1974).
    \bibitem{14} J. Ellis and D. V. Nanopoulos, 参考文献[12]. 详细的结果见\,F. Gabbiani and A. Masiero, {\textit{Nucl. Phys.}} {\bf{B322}}, 235 (1989); J. S. Hagelin, S. Kelley, and T. Tanaka, {\textit{Nucl. Phys.}} {\bf{B415}}, 293 (1994). 最完整的处理是\,D. Sutter, Stanford University Ph. D. thesis (未发表)和\,S. Dimopoulos and D. Sutter, 参考文献[11].
    \bibitem[14a]{14a} M. Dine, R. Leigh, and A. Kagan, {\textit{Phys. Rev.}} {\bf{D48}}, 4269 (1993).
    \bibitem{15} 较新的综述参看\,Y. Grossman, Y. Nir and R. Rattazzi, 收录于{\textit{Heavy Flavours II}}, A. J. Buras and M. Lindner\,编辑(World Scientific, Singapore, 1998); A. Masiero and L. Silvestrini, 收录于\,{\textit{Perspectives on Supersymmetry}}, 参考文献[10].
    \bibitem{16} J. Ellis and M. K. Gaillard, {\textit{Nucl. Phys.}} {\bf{B150}}, 141 (1979); D. V. Nanopoulos, A. Yildiz, and P. H. Cox, {\textit{Ann. Phys. (N.Y.)}} {\bf{127}}, 126 (1980); M. B. Gavela, A. Le Yaouanc, L. Oliver, O. P\`{e}ne, J.-C. Raynal, and T. N. Pham, {\textit{Phys. Lett.}} {\bf{109B}}, 215 (1982); B. H. J. McKellar, S. R. Choudhury, X-G. He, and S. Pakvasa, {\textit{Phys. Lett.}} {\bf{B197}}, 556 (1987).
    \bibitem[16a]{16a} P. G. Harris {\textit{et al.,}} {\textit{Phys. Rev. Lett.}} {\bf{82}}, 904 (1999).
    \bibitem{17} J. Ellis, S. Ferrara, and D. V. Nanopoulos, {\textit{Phys. Lett.}} {\bf{114B}}, 231 (1982); J. Polchinski and M. B. Wise, {\textit{Phys. Lett.}} {\bf{125B}}, 393 (1983); M. Dugan, B. Grinstein, adn L. Hall, {\textit{Nucl. Phys.}} {\bf{B255}}, 413 (1985).
    \bibitem{18} R. Arnowitt, J. Lopez, and D. Nanopoulos, {\textit{Phys. Lett.}} {\bf{D42}}. 2423 (1990); R. Arnowitt, M. Duff, and K. Stelle, {\textit{Phys. Rev.}} {\bf{D43}}, 3085 (1991); Y. Kizuri and N. Oshimo, {\textit{Phys. Rev.}} {\bf{D45}}, 1806 (1992).
    \bibitem{19} S. Weinberg, {\textit{Phys. Rev. Lett.}} {\bf{63}}, 2333 (1989); D. Dicus, {\textit{Phys. Rev.}} {\bf{D41}}, 999 (1990); J. Dai, H. Dykstra, R. G. Leigh, S. Paban, and D. A. Dicus, {\textit{Phys. Lett.}} {\bf{B237}}, 216 (1990); E. Braaten, C. S. Li, and T. C. Yuan, {\textit{Phys. Rev. Lett.}} {\bf{64}}, 1709 (1990); A. De R\'{u}jula, M. B. Gavela, O. P\`{e}ne, and F. J. Vegas, {\textit{Phys. Lett.}} {\bf{B245}}, 640 (1990); R. Arnowitt, M. J. Duff, and K. S. Stelle, 参考文献[18]; T. Ibrahim and P. Nath, {\textit{Phys. Lett.}} {\bf{148B}}, 98 (1998).
    \bibitem{20} K. S. Babu, C. Kolda, J. March-Russell, and F. Wilczek, {\textit{Phys. Rev.}} {\bf{D59}}, 016004 (1999).
    \bibitem{21} R. Arnowitt, M. J. Duff, and K. S. Stelle, 参考文献[18]. %
    如果用这个文献中的函数$\,J_{1}\,$和$\,J_{2}\,$表示, 再假定建设性地加上胶子连在标量夸克和胶微子线上的图后, %
    这里的函数$\,J\,$就取做$\,2J_{1}+\frac{2}{3}J_{2}$.
    \bibitem{22} H. Georgi and L. Randall, {\textit{Nucl. Phys.}} {\bf{B276}} 241 (1980); A. Manohar and H. Georgi, {\textit{Nucl. Phys.}} {\bf{B238}}, 189 (1984); S. Weinberg, 参考文献[19].
    \bibitem{23} W. Fischler, S. Paban, and S. Thomas, {\textit{Phys. Lett.}} {\bf{B289}}, 373 (1992).
    \bibitem{24} J. Ellis and D. V. Nanopoulos, 参考文献[12]; F. Gabbiani and A. Masiero, 参考文献[14]; F. Dine, A. Kagan, and S. Samuel, {\textit{Phys. Lett.}} {\bf{B243}} 250 (1990); F. Gabbiani, E. Gabrielli, A. Masiero, and L. Silvestrini, 参考文献[12].
    \bibitem{25} S. Weinberg, {\textit{Phys. Rev. Lett.}} {\bf{40}}, 223 (1978); F. Wilczek, {\textit{Phys. Rev. Lett.}} {\bf{40}}, 279 (1978).
    \bibitem{26} A. Brignole, J. Ellis, G. Ridolfi, and F. Zwirner, {\textit{Phys. Lett.}} {\bf{B271}}, 123 (1991); M. Carena, M. Quiros, and C. E. M. Wagner, {\textit{Nucl. Phys.}} {\bf{B461}}, 407 (1996); S. Heinemayer, W. Hollik, and G. Weiglein, hep-ph/9812472, hep-ph/9903404, hep-ph/9903504, 待发表. 这里对$\,m_{h}\,$引用的数值结果取自\,S. Dawson\,引用的计算, 参考文献[10].
    \bibitem{27} R. Barate {\textit{et al.}} (ALEPH\,合作组), {\textit{Phys. Lett.}} {\bf{B412}}, 173 (1997).
    \bibitem[27a]{27a} M. Gr\"{u}newald and D. Karlen, 收录于\,{\textit{Proceedings of the XXIX International Conference on High Energy Nuclear Physics,}} A. Astbury, D. Axen, and J. Robinson, 编辑(TRIUMF, Vancouver, 1999).
    \bibitem{28} R. Barate {\textit{et al.}} (ALEPH\,合作组), 1999 CERN preprint EP-99-011, 将发表于\,{\textit{Phys. Lett.}} $54.5\,\mathrm{GeV}\,$的下界在之前由\,P. Abreu\,等人在$\,130\,$至$\,172\,\mathrm{GeV}\,$之间的$\,e^{+}$--$e^{-}\,$湮灭试验中获得, (DELPHI\,合作组), {\textit{Phys. Lett.}} {\bf{B420}}, 140 (1998).
    \bibitem{29} A. J. Buras, M. Misiak, M. M\"{u}nz, and S. Pokorski, {\textit{Nucl. Phys.}} {\bf{B424}}, 374 (1994).
    \bibitem[29a]{29a} R. Barate {\textit{et al.}} (ALEPH\,合作组), 1999 CERN preprint EP-99-014, 将发表于\,{\textit{Eur. Phys. J.}}
    \bibitem{30} M. Dine, W. Fischler, and M. Srednicki, {\textit{Nucl. Phys.}} {\bf{B189}}, 575 (1981); S. Dimopoulos and S. Raby, {\textit{Nucl. Phys.}} {\bf{B192}}, 353 (1982); M. Dine and W. Fischler, {\textit{Phys. Lett.}} {\bf{110B}}, 227 (1982); {\textit{Nucl. Phys.}} {\bf{B204}}, 346 (1982); C. Nappi and B. Ovrut, {\textit{Phys. Lett.}} {\bf{113B}}, 175 (1982); L. Alvarez-Gaum\'{e}, M. Claudson, and M. Wise, {\textit{Nucl. Phys.}} {\bf{B207}}, 96 (1982); S. Dimopoulos and S. Raby, {\textit{Nucl. Phys.}} {\bf{B219}}, 479 (1983). 这类模型被\,M. Dine\,和\,A. E. Nelson\,复兴, {\textit{Phys. Rev.}} {\bf{D48}}, 1277 (1993); {\bf{D51}}, 1362 (1995); J. Bagger, 参考文献[10]; M. Dine, A. Nelson, and Y. Shirman, {\textit{Phys. Rev.}} {\bf{D51}}, 1362 (1995); M. Dine, A. Nelson, Y. Nir, and Y. Shirman, {\textit{Phys. Rev.}} {\bf{D53}}, 2658 (1996). 关于综述, 参看\,C. Kolda, {\textit{Nucl. Phys. Proc. Suppl.}} {\bf{62}}, 266 (1998); G. F. Giudice and R. Rattazzi, hep-ph/9801271, 将发表于{\textit{Phys. Rep}}; S. L. Dubovsky, D. S. Gorbunov, and S. V. Troitsky, hep-ph/9905466, 待发表. S. Dimopoulos, S. Thomas\,和\,J. D. Wells\, 描述了这些模型的唯象学, {\textit{Nucl. Phys.}} {\bf{B488}}, 39 (1997).
    \bibitem{31} S. Dimopoulos, G. F. Giudice, and A. Pomerol, {\textit{Phys. Lett.}} {\bf{389B}}, 37 (1997); S. P. Martin, {\textit{Phys. Rev.}} {\bf{D55}}, 3177 (1997).
    \bibitem{32} G. F. Giudice and R. Rattazi, {\textit{Nucl. Phys.}} {\bf{B511}}, 25 (1998). 这个结果被\,N. Arkani-Hamed, G. F. Giudice, M. A. Luty\, 和\,R. Rattazzi\,做了推广, {\textit{Phys. Rev.}} {\bf{D58}}, 115005 (1998).
    \bibitem{33} N. Seiberg, {\textit{Phys. Lett.}} {\bf{B318}}, 469 (1993).
    \bibitem{34} K. S. Babu, C. Kolda, and F. Wilczek, {\textit{Phys. Rev. Lett.}} {\bf{77}}, 3070 (1996).
    \bibitem{35} J. E. Kim and H-P. Nilles, {\textit{Phys Lett.}} {\bf{138B}}, 150 (1984); J. Ellis, J. F. Gunion, H. E. Haber, L. Roszkowski, and F. Zwirner, {\textit{Phys. Rev.}} {\bf{D39}}, 844 (1989); E. J. Chun, J. E. Kim, and H-P. Nilles, {\textit{Nucl. Phys.}} {\bf{B370}}, 105 (1992); M. Dine and A. E. Nelson, 参考文献[30]; M. Dine, A. E. Nelson, Y. Nir, and Y. Shirman, 参考文献[30]; G. Dvali, G. F. Giudice, and A. Pomerol, {\textit{Nucl. Phys.}} {\bf{B478}}, 31 (1996); S. Dimopoulos, G. Dvali, and R. Rattazzi, {\textit{Phys. Lett.}} {\bf{413B}}, 336 (1997); H-P. Nilles and N. Polonsky, {\textit{Nucl. Phys.}} {\bf{B484}}, 33 (1997); G. Cleaver, M. Cveti\v{c}, J. R. Espinosa, L. Everett, and P. Langacker, {\textit{Phys. Rev.}} {\bf{D57}}, 2701 (1998); P. Langacker, N. Polonsky, and J. Wang, hep-th/9905252, 待发表; J. E. Kim, hep-th/9901204, 待发表.
    \bibitem[35a]{35a} S. Raby, {\textit{Phys. Lett.}} {\bf{B422}}, 158 (1998).
    \bibitem[35b]{35b} D. A. Ricus, B. Dutta, and S. Nandi, {\textit{Phys. Rev. Lett.}} {\bf{78}}, 3055 (1997); {\textit{Phys. Rev.}} {\bf{D56}}, 5748 (1997).
    \bibitem{36} S. Weinberg, {\textit{Phys. Rev. Lett.}} {\bf{43}}, 1566 (1979); F. Wilczek and A. Zee, {\textit{Phys. Rev. Lett.}} {\bf{43}}, 1571 (1979).
    \bibitem{37} S. Weinberg, {\textit{Phys. Rev.}} {\bf{D26}}, 287 (1982); N. Sakai and T. Yanagida, {\textit{Nucl. Phys.}} {\bf{B197}}, 533 (1982). 这些文章重印于\,\textit{Supersymmetry}, 参考文献[2].
    \bibitem{38} S. Weinberg, 参考文献[36].
    \bibitem{39} J. Ellis, J. S. Hagelin, D. V. Nanopoulos, and K. Tamvakis, {\textit{Phys. Lett.}} {\bf{124B}}, 484 (1983); V. M. Belyaev and M. I. Vysotsky, {\textit{Phys. Lett.}} {\bf{127B}}, 215 (1983).
    \bibitem{40} V. Lucas and S. Raby, {\textit{Phys. Rev.}} {\bf{D55}}, 6986 (1997).
    \bibitem{41} 这个估计取自将总两体质子衰变速率写成超重规范玻色子质量$\,M_{X}\,$的非超对称计算的一个汇总, 由\,P. Langacker\,完成, %
    收录于\,\textit{Proceedings of the 1983 Annual Meeting of the Division of Particles and Fields of the American Physical Society} (American Institute of Physics, New York, 1983): 251. 为了将结果表示成$\,g_{6}$, 我实际上已经假定了这些计算中所用的$\,g_{6}\,$由$\,g_{6}=g^{2}(M_{X})/M_{X}^{2}\,$给出, 其中$\,g(M_{X}\,)$的值近似等于非超对称理论中的值, 约为$\,g^{2}(M_{X})/4\uppi \simeq 1/41$.
    \bibitem{42} 对于更细致(大多模型相关)的计算, 包含对$\,g_{5}\,$的重整化群修正, 参看\,S. Dimopoulos, S. Raby, and F. Wilczek, %
    {\textit{Phys. Lett.}} {\textbf{112B}}, 133 (1982); J. Ellis, D. V. Nanopoulos, and S. Rudaz, {\textit{Nucl. Phys.}} {\textbf{B202}}, 43 %
    (1982); W. Lang, {\textit{Nucl. Phys.}} {\bf{B203}}, 277 (1982); J. Ellis, J. S. Hagelin, D. V. Nanopoulos, and K. Tamvakis, 参考文献[39]; %
    V. M. Belyaev and M. I. Vysotsky, 参考文献[39]; L. E. Ib\'{a}\~{n}ez and C. Mu\~{n}oz, {\textit{Nucl. Phys.}} {\bf{B245}}, 425 (1984); P. Nath, %
    A. H. Chamseddine, and R. Arnowitt, {\textit{Phys. Rev.}} {\bf{D32}}, 2385 (1985); J. Hisano, H. Murayama, and T. Yanagida, %
    {\textit{Nucl. Phys.}} {\bf{B402}}, 46 (1993); V. Lucas and S. Raby, 参考文献[40]. 关于综述, 参看\,P. Nath and R. Arnowitt, %
    {\textit{Phys. Atom. Nucl.}} {\bf{61}}, 975 (1997).
    \bibitem[42a]{42a} M. Takita {\textit{et al.,}} 收录于\,{\textit{Proceedings of the XXIX International Conference on High Energy Nuclear Physics}}, 参考文献[27a].
    \bibitem{43} K. S. Babu, J. C. Pati, and F. Wilczek, {\textit{Phys. Lett.}} {\bf{423B}}, 337 (1998).
    \bibitem{44} V. Lucas and S. Raby, 参考文献[40]; T. Goto and T. Nihei, {\textit{Phys. Rev.}} {\bf{D59}}, 115009 (1999).
    \bibitem{45} L. Ib\'{a}\~{n}ez and G. Ross, {\textit{Nucl Phys.}} {\bf{B368}}, 3 (1991).
\end{thebibliography}


%第二十九章 超出微扰论

\chapter{超出微扰论} \label{cha:29}


迄今为止所讨论的超对称应用大多数是用微扰论推导出来的. 在这一章, 我们将考虑一些即使把非微扰效应考虑在内也依旧成立的结果.

\section{超对称破缺的一般性质} \label{sec:29.1}

在已知粒子的谱中并没有观测到超对称, 所以它肯定破缺了. 我们在上一章看到, 超对称在标准模型的树级近似下破缺在实验上被排除了, %
并且电弱破缺标度与\,Planck\,或大统一标度之间的巨大差异指出超对称很可能是在某个跑动耦合常数变强时破缺的. %
因此对我们来说一件很重要的事是不借助微扰论来探索自发超对称破缺.

我们在\,\ref{sec:26.7}\,节看到, 作用量的超对称性表明存在守恒流$\,S^{\mu}(x)$. 从方程(\ref{26.A.2})的意义上说, 这个流是一个\,Majorana\,旋量:
\begin{equation}
    S^{\mu}(x)^{\ast} = -\beta \gamma_{5}\epsilon S^{\mu}(x) \:; \label{29.1.1}
\end{equation}
它是守恒的,
\begin{equation}
    \partial_{\mu}S^{\mu}_{\beta}(x) = 0 \: ; \label{29.1.2}
\end{equation}
并且它的时间分量积分是超对称生成元
\begin{equation}
    \int \dif^{3}x\: S_{\beta}^{0}(x) = Q_{\beta} \:, \label{29.1.3}
\end{equation}
使得$\,-\mi(\bar{\alpha}Q)\,$与任何算符的对易子给出这个算符在一个无限小\,Majorana\,旋量参量为$\,\alpha\,$的超对称变换下的变化.


给出这些结果的讨论依赖于作用量的超对称性; 但没有什么依赖于超对称性是否自发破缺, 除了积分(\ref{29.1.3})是否可能存在这个假定. 事实上, 这个假定在有无质量费米子的理论中会被破坏掉, 这种理论中有长程效应(或者, 等价地, 零\,4\,-动量处的极点)会使得这个积分不收敛. 我们在这里会看到这种无质量费米子是超对称破缺的必然的结果. 为了避免这个积分即使在有无质量费米子的理论中的收敛性问题, 在有限体积$\,V\,$的空间中进行处理将是非常方便的. 我们可以通过附加周期性边界条件来保持平移不变性: 所有的场被假定成在任何空间坐标$\,x^{i}\,$平移$\,V^{1/3}\,$下不变.

假定存在一个零\,3\,-动量的超对称真空态$\,\lvert \text{VAC}\rangle$, 而多重态可以通过作用场算符得到, 那么如果存在一个算符$\,Q_{\alpha}\,$诱导出量子场上的超对称变换, 这就使得我们可以导出超对称的所有结果. 如果$\,\lvert \text{VAC}\rangle\,$是超对称的, 也就是说$\,Q_{\alpha}\lvert\text{VAC}\rangle=0$, %
那么由反对易关系(\ref{25.2.36})可以得出这个态的能量和动量都为零. 相反, 通过取正定算符$\,\{\mathcal{Q}_{a},\mathcal{Q}_{a}^{\ast}\}\,$(暂时回到二分量的符号约定)的真空期望值, 我们看到, 如果真空的能量为零, 那么它必须被$\,\mathcal{Q}_{a}\,$和$\,\mathcal{Q}_{a}^{\ast}\,$湮灭掉, 因此是超对称的, %
如果它不是超对称的, 那么它的能量必然是正定的. {\kai{因此, 超对称是否破缺的问题完全是真空的能量是正定还是零的问题.}}


相同的推理让\,Witten\,得出如下的结论: 有$\,N>1\,$个二分量旋量生成元$\,\mathcal{Q}_{ar}\,$及其共轭的扩充超对称性无法自发破缺到生成元较少的扩充超对称性或简单超对称, 这是因为如果任何一个生成元没有湮灭真空, 那么真空能就不为零, 由此可以得出没有一个生成元可以湮灭真空.\cite{1} 真空态的能量通常是作为所有态的能量中一个定义不明确的附加常数出现的, 而在这里, 这个常数因超对称的反对易关系中出现了能动量\,4\,-矢而被赋予了意义. 在有限体积下进行处理的一个好处就是使得讨论真空的总能量是有意义的.

Hughes, 刘俊和\,Polchinski\,指出存在呈现出一类部分破缺超对称性的理论.\cite{1a} 这些理论没有第\,\ref{cha:25}\,章描述的那类超对称代数. %
取而代之, 它们有一个{\kai{流}}的代数, 基于像方程(\ref{26.7.45})这样的反对易关系:
\[
\int \dif^{3}x\:\Bigl\{S_{r\alpha}^{0}(x),\bar{S}_{s\beta}^{\mu}(y)\Bigr\}
=-2\mi\,\delta_{rs}\gamma_{\nu}\Theta^{\mu\nu}(y)+2\mi\gamma_{\alpha\beta}^{\mu}C_{rs} \:,
\]
其中$\,\Theta^{\mu\nu}\,$是满足守恒条件$\,\partial_{\mu}\Theta^{\mu\nu}=0\,$的一个能动量张量, 而$\,C_{rs}\,$是一个新元素, 一个常数. 对$\,N=1$, %
这个常数可以认为是$\,\Theta^{\mu\nu}\,$中的一项$\,-\eta^{\mu\nu}C$, 但除非$\,C_{rs}\propto \delta_{rs}$, 这对于扩充超对称性是不可能的, %
而实际情况不需要是这样. 因为它不可能是$\,S\,$-矩阵的对称性, 所以这个代数没有被\,\ref{sec:25.2}\,节证明的\,Haag--Lopuszanski--Sohnius\,定理排除. %
自发破缺的对称性确实不可能是$\,S\,$-矩阵的对称性, 但通常假定它们所基于的代数或超代数{\kai{能够}}成为某些理论的某些相中的$\,S\,$-矩阵的对称性. %
当$\,C_{rs}\,$不正比于$\,\delta_{rs}\,$时, 流的超代数在任何理论的任何相中无法生成一个对称性. 这里我们仅考虑第\,\ref{cha:25}\,章讨论的那类超代数, %
Witten\,的讨论对于这类代数是成立的.

在有限体积中进行处理的另一个好处是所有态都是离散且可归一化的. $Q_{\alpha}\,$与$\,P_{\mu}\,$对易的一个立即的结果是任何能量非零的态都会伴随着另一个能量和动量相同但统计相反的态. 为了看到这一点, 注意到对于任何\,3\,-动量$\,\mathbf{p}$, 我们可以找到一个二分量旋量$\,u_{a}\,$使得$\sum_{ab}u_{a}^{\ast}\bm{\sigma}_{ab}\cdot\mathbf{p}u_{b}=0\,$和$\,\sum_{a}\lvert u_{a}\rvert^{2}=1$. (当$\,\mathbf{p}\,$处在\,3\,-方向上, %
取$\,u=(1,1)/\sqrt{2}$. 当$\,\mathbf{p}\,$处在任何其它方向上时, 对这个$\,u\,$作用从$\,3\,$-方向到$\,\mathbf{p}\,$方向的旋转的自旋$\,1/2\,$表示). %
这样, 在有\,4\,-动量$\,p^{\mu}\,$的空间中, 反对易关系(\ref{25.2.31})和(\ref{25.2.32})给出
\begin{equation}
    \mathcal{Q}^{2}(p) = p^{0}\:, \label{29.1.4}
\end{equation}
其中$\,\mathcal{Q}(p)\,$是超对称生成元的厄米线性组合:
\begin{equation}
    \mathcal{Q}(p) = \sum_{a} u_{a}\mathcal{Q}_{a} +\sum_{a}u_{a}^{\ast} \mathcal{Q}_{a}^{\ast} \:. \label{29.1.5}
\end{equation}
用$\,\mathcal{Q}(p)\,$作用任何\,4\,-动量为$\,p^{\mu}\,$且$\,p^{0}>0$\, 的归一化态$\,\lvert X\rangle$, %
我们会得到另一个\,4\,-动量相同但统计相反的归一化态\,$\lvert Y\rangle=\mathcal{Q}(p)\lvert X\rangle/\sqrt{p^{0}}$. %
更进一步, $\lvert X\rangle\,$是唯一一个以这种方式与$\,\lvert Y\rangle\,$关联的态, 这是因为, 如果$\,\lvert Y\rangle=\mathcal{Q}(p)\lvert X\rangle/\sqrt{p^{0}}$, 那么根据方程(\ref{29.1.4}), $\lvert X\rangle=\mathcal{Q}(p)\lvert Y\rangle/\sqrt{p^{0}}$. %
超对称生成元和自旋态的多重性通常将引起这些玻色态和费米态的对与其它对组合在一起, 且它们有相同的\,4\,-动量, %
但目前知道所有能量非零的态至少可以被分入到这些对中就已经足够了.

当超对称破缺时, 我们不期待粒子个数确定的态与其它统计相反, 4\,-动量相同且粒子个数相同的态构成超多重态. 这样态的配对就会要求存在一个无质量费米子, %
使得与一个$\,n\,$-粒子态配对的态能动量相同但统计相反的态, 并由同样$\,n\,$个粒子和一个零能动量的无质量费米子构成. %
这个无质量费米子被称为{\kai{戈德斯通微子}}(\textit{goldstino}). 更确切些, 任何$\,n\,$-粒子态都伴随着两个能动量相同但统计相反的态, %
这两个态包含一个额外的自旋为上或下的零动量戈德斯通微子, 以及另一个能动量相同且统计{\kai{相同}}的态, 而这个态包含两个额外的自旋相反的零动量戈德斯通微子. %
特别地, 当超对称自发破缺时, 真空态的能量非零, 所以它必须与一个能量相同且动量为零的费米态配对; 更确切些, 真空和包含两个零动量戈德斯通微子的态与两个零动量单戈德斯通微子态配对. 仅当超对称性没有破缺时, 才存在一个未配对的零能量态, 真空.

能量非零态的配对提供了一个很有用的诊断超对称是否自发破缺的工具, 在一些情况下, 即使微扰论无法足以判断这个问题, 这个工具也可以告诉我们这个问题的答案. %
当所有相互作用都很弱时, 我们可以依靠微扰论给频谱一个{\kai{定性}}的图景. 如果我们在树级近似下发现有$\,n\,$个能量为零且没有无质量费米子的真空态, %
那么我们可以非常放心地说, 在弱耦合时, 由于没有零能量费米态可以与这$\,n\,$个真空态配对, 所以这些未配对的态的能量必须为零. 这样, %
当我们提高耦合的强度或者以其它任何方式变化理论的参量时, 态可以从正能量变成零能量或者从零能量变成正能量, 但它们一般不会突然出现或消失. %
(有一个例外, 当改变理论的参量使得大场的拉格朗日量的渐近行为发生变化; 我们不久之后就会看到, 这{\kai{能}}产生或消灭态.) %
由于每个能量非零的态都会伴随着另一个统计相反的态, 它们可以在这种对中从零能量跃迁到非零能量或者从非零能量跃迁到零能量, 所以, 当我们改变理论的参量时, %
只要拉格朗日量的大场行为没有改变, 那么零能量玻色态的个数减去零动量费米态的个数就不会改变. 这个差称为\,\emph{Witten}\,{\kai{指标}}.\cite{2} %
形式上, 这个指标是$\,\operatorname{Tr}(-1)^{F}$, 其中$\,F\,$是费米数; 前面讨论的态的配对确保了非零能量的态不会贡献到这个迹上. %
如果\,Witten\,指标非零, 那么必存在{\kai{一些}}能量为零的态, 因而超对称不可能破缺. 特别地, %
在树级近似下给出$\,n\,$个零能量真空态且不给出零能量费米子的理论中, 在弱耦合时, Witten\,指标是$\,n$, 这时可以相信树级近似给出了频谱的一个定性图景,
而当耦合强度提高时, 这个指标会一直等于$\,n$, 由此我们可以确信高阶效应甚至非微扰效应不会破坏超对称.

作为使用\,Witten\,指标的一个例子, 考察带一个三次多项式超势的单手征超场的\,Wess-Zumino 理论, 其中超势有(\ref{26.4.16})的形式:
\[
f(\phi)=\tfrac{1}{2}m^{2}\phi^{2}+\tfrac{1}{6}g\phi^{3} \:,
\]
其中$\,\phi\,$是超场的复标量场分量. 我们在\,\ref{sec:26.4}\,节看到这个模型在树级近似并不会呈现出超对称破缺, 但微扰论的高阶会怎么样, %
非微扰效应又会怎么样? 当$\,m\,$很大且$\,g\,$很小时, 微扰论给能谱一个非常好的近似; 它告诉我们在这一情况下在零能量附近有两个玻色态, %
对应于方程$\,\partial f(\phi)/\partial \phi=0\,$的两个解$\,\phi=0\,$和$\,\phi=-2m^{2}/g$, 而在零能量附近没有费米态; %
能量最低的费米态是能量在$\,\lvert m\rvert\,$附近的零动量单费米子态. 在典范的标量场论中, 我们不期待有两个玻色态有精确为零的能量; %
即使每个态在树级近似下能量为零, 高阶效应(计入隧穿$\,\phi=0\,$和$\,\phi=-2m^{2}/g\,$之间的势垒)预期会混合二者并偏移它们的能量使之远离零.
(仅在无限体积的极限下, 这个势垒才会变得不可穿越.) 但在超对称理论中, 由于没有可以与这些态配对的低能费米态, 这些态的能量必须精确为零. %
因此当$\,m\,$很大且$\,g\,$很小时, Witten\,指标为\,2. 因为\,Witten\,指标在理论参量的变化下不变, 即使当$\,g\,$很大, 微扰论失效时, 或者$\,m\,$为零, %
两个势阱融合时, Witten\,指标依旧会等于\,2. (在这种情况下, 由于在树级近似下既出现了无质量玻色子又出现了无质量费米子, 直接计算\,Witten\,指标并不容易.) %
由于\,Witten\,指标不为零, 无论理论参量取何值, Wess-Zumino\,模型中的超对称性都是严格不破缺的.


相同的讨论可以广泛地用于有数个手征标量超场的理论来证明\,Witten\,指标是正的, 以及超对称性不是自发破缺的. \ref{sec:26.5}\,节讨论的\,O'Raifeartaigh\,模型是一个例外, 因为这个理论中有一个平坦方向, 在这个方向上, 势随着场趋于无穷会一直是个常数, 而不是随着场的某个幂次增长. 这些模型提供了一个非常好的粒子来说明: %
尽管为了使超对称破缺\,Witten\,指标必须为零, 但是\,Witten\,指标为零并不一定意味着超对称{\kai{是}}破缺的. 例如, 如果我们将\,\ref{sec:26.5}\,节用做例子的超势用正则归一化超场写出来, 那么它有如下的形式
\[
f(X,Y_{1},Y_{2})=mY_{1}(X-a)+gY_{2}X^{2} \:,
\]
其中$\,m$, $g\,$和$\,a\,$是任意参量. 这样势就是
\[
U(x,y_{1},y_{2})=\lvert g\rvert^{2}\lvert x\rvert^{4} +\lvert m\rvert^{2}\lvert x-a\rvert^{2}
+\lvert my_{1}+2gxy_{2}\rvert^{2} \:,
\]
其中小写字母用来指明这是左手征超场的标量分量. 当$\,m\,$和$\,a\,$非零且$\,g\,$很小时, 微扰论给出了频谱一个很好的估计, %
并告诉我们, 当$\,my_{1}+2xy_{2}=0\,$时, 势在$\,x=a-2\lvert g\rvert^{2}\lvert a\rvert^{4}/\lvert m\rvert^{2}\,$附近有一个最小值点, %
真空能在这一点近似等于$\,\lvert ga^{2}\rvert^{2} V$. 由于这个能量作为一个额外常数出现在所有态中, 所有没有能量为零的态并且\,Witten\,指标为零. %
超势的导数矩阵$\,\mathscr{M}\,$(行和列按照$\,x$, $y_{1}$, $y_{2}\,$的顺序标记)在这里是
\[
\mathscr{M}=\begin{pmatrix}
2gy_{2} & m & 2gx \\
m& 0 & 0 \\
2gx & 0 & 0
\end{pmatrix} \:.
\]
这个矩阵有一个本征值为零的本征矢$\,(0,2gx,-m)$, 所以这里有一个无质量费米子; 这个费米子是与超对称破缺相关的戈德斯通微子. %
与真空简并的费米态由一个零能动量的戈德斯通微子构成. (和之前一样, 有两个真空能费米态, 定向与戈德斯通微子自旋相反, 两个真空能玻色态: 真空, 以及一个由两个自旋相反的戈德斯通微子构成的态.) 现在, 超对称在$\,a\to0\,$时可能是不破缺的(我们会在后面看到确实是这样的情况), 但\,Witten\,指标必须保持为零; 在这种情况下,
无质量费米子不再是戈德斯通微子, 但连续性要求它必须保持零质量, 所以它仍和真空态配对. 这显然使超对称在参量的孤立值点被修复的理论的一般特征; %
连续性要求超对称破缺时起到戈德斯通微子作用的无质量费米子在超对称被恢复的参量值处保持无质量, 这使得真空依旧与无质量费米子态配对, %
而\,Witten\,指标保持为零.

这个模型提供了一个很好的例子说明我们为什么要限定\,Witten\,指标在我们改变超对称理论的参量时不发生变化这个陈述, 而要加上参量改变的方式不能改变大场的拉格朗日量的渐近行为这个要求. 假定我们通过给超势加上一个小项来扰乱这个模型的平坦方向, 使得它现在变成
\[
f(X,Y_{1},Y_{2})=mY_{1}(X-a)+gY_{2}X^{2}+\tfrac{1}{2}\epsilon(Y_{1}^{2}+Y_{2}^{2}) \:,
\]
其中$\,\epsilon\,$是一个小的质量参量. 现在保护超对称的条件有两个解:
\[
0=\frac{\partial f}{\partial x} =\frac{\partial f}{\partial y_{1}} =\frac{\partial f}{\partial y_{2}} \:.
\]
在这些解上, $x\,$是二次方程$\,2g^{2}x^{2}+m(x-a)=0\,$的一个根, 而$\,y_{1}\,$和$\,y_{2}\,$是$\,1/\epsilon\,$阶的: %
$y_{1}=-m(x-a)/\epsilon\,$而$\,y_{2}=-gx^{2}/\epsilon$. 我们看到当我们打开小参量$\,\epsilon\,$后\,Witten\,指标从\,0\,变到\,2\,的原因是势从无限大场值处得到两个新的最小值.

在\,Witten\,指标为零的理论中, 为了确定超对称是否破缺, 使用守恒律限制可能发生的配对并定义一类新的指标通常是有用的. %
如果$\,K\,$是一个与超对称性生成元$\,Q_{\alpha}\,$对易的量子算符(这样它也与哈密顿量对易), 那么所有能量非零且对$\,K\,$有确定值的态都会与统计相反, %
能动量相同{\kai{且}}$\,K\,$值相同的态配对. 另外, 不仅\,Witten\,指标$\,\operatorname{Tr}(-1)^{F}\,$与理论的参量无关(只要它们变化的方式不改变%
拉格朗日量的大场渐进行为)------{\kai{加权}}\,{\it{Witten}}\,{\kai{指标}}也是如此, 这个指标由$\,\operatorname{Tr}g(K)(-1)^{F}\,$给出, %
其中$\,g(K)\,$是守恒量的一个任意函数. 当以这种方式使用守恒律时, 它在体积$\,V\,$趋于无穷大时不必是未破缺的; 唯一必须的是$\,K\,$与超对称性生成元对易.

在判断超对称破缺的几率时, 用多个不同守恒量的加权\,Witten\,指标的线性组合进行处理通常是有益的. 特别的, 考虑
\begin{equation}
    W_{G}= \sum_{h\in G}\operatorname{Tr}\Bigl\{h(-1)^{F}\Bigr\} \:, \label{29.1.6}
\end{equation}
其中求和取遍某个对称群$\,G\,$的所有元素. (对于紧致连续群, 这个求和应该被理解成在加上合适测度后对群体积积分.) 在除了恒等表示以外的任何不可约表示中, %
当对一个有限或紧群求和时, ``特征标''$\,\operatorname{Tr}h\,$加起来为零, 所以
\begin{equation}
    W_{G}=\sum_{f}N(f)(-1)^{f} \:, \label{29.1.7}
\end{equation}
其中$\,N(f)\,$是$\,G\,$的恒等表示在费米子数为$\,f\,$的态中出现的次数. 换句话说, $W(G)\,$就是\,Witten\,指标, 只不过只用$\,G\,$-不变态计算. %
只要$\,G\,$守恒, $W(G)\,$就不会依赖理论的参量, 并且如果它不为零则表明超对称未破缺.

\ref{sec:29.4}\,节将用守恒律来研究规范理论中的超对称性自发破缺, 但前面讨论的\,O'Raifeartaigh\,-类模型提供了一个更加简单(尽管理论化)的粒子, %
只不过这时要将参量$\,a\,$设为零. 这时超势是
\begin{equation}
    f(X,Y_{1},Y_{2})=m Y_{1}X+g Y_{2}X^{2} \:, \label{29.1.8}
\end{equation}
它给出了一个树级近似势
\begin{equation}
    U(x,y_{1},y_{2})=\lvert m\rvert^{2}\lvert x\rvert^{2}+ \lvert g\rvert^{2}\lvert x\rvert^{4}
+\lvert my_{1}+2gxy_{2}\rvert^{2} \:. \label{29.1.9}
\end{equation}
因为我们已经看到\,Witten\,指标在$\,a\neq0\,$时为零, 所以我们知道它在这里为零, 但超对称是否破缺? 当$\,a=0\,$时, 存在新的场值$\,x=y_{1}=0\,$使得树级近似下的势为零, 但我们该如何分辨$\,g\,$的高阶效应甚至非微扰效应是否赋予了相应态一个微小的能量? 为了回答这个问题, %
我们注意到这个超势(拉格朗日密度也因此)在一个离散群$\,K\,$下不变, 超场在这个群下的变换是
\begin{equation}
    KXK^{-1}=\mi X \:, \qquad KY_{1}K^{-1}=-\mi Y_{1} \:, \qquad KY_{2}K^{-1}=-Y_{2}\:. \label{29.1.10}
\end{equation}
(要注意的是这个对称性被原始超势中的$\,-maY_{1}\,$项破坏了, 所以使用$\,K\,$获得的结果对那个超势不会成立.) %
由于超势对$\,x=y_{1}=0$以及任意的$\,y_{2}\,$为零, 当$\,g\,$很小时, 我们可以用微扰论告诉我们对每个$\,y_{2}\,$存在一个能量{\kai{接近}}为零的玻色真空态.
当$\,y_{2}=0\,$时, 这个真空在$\,K\,$下为偶. 对于$\,\lvert y_{2}\rvert\,$的任何非零值, %
我们可以取$\,y_{2}=\pm\lvert y_{2}\rvert\,$的两个零能量态的线性组合, 其中一个态在$\,K\,$下为偶另一个则为奇. 正如我们所看到的, 这里还有一个无质量费米子,
$Y_{2}\,$的费米分量, 但这个费米子在$\,K\,$下为奇, 所以它不能与偶真空态配对. 理论中仅剩的另一个费米子拥有树级近似质量$\,\lvert m\rvert$, %
所以当$\,g\,$很小时, 它们也不能与偶真空态配对. 我们由此得出结论: 当$\,g\,$很小时, 偶真空态的能量必须精确为零, 并且超对称不破缺. %
计算这里的加权\,Witten\,指标并不容易, 这是因为存在无限多个在$\,K\,$为偶的玻色零能量态, %
它们由零个或两个零动量$\,Y_{2}\,$-费米子加任意多个零动量$\,Y_{2}\,$玻色子构成, 但显然有$\,\operatorname{Tr}K(-1)^{F}>0$, %
并且由于它独立于$\,g\,$(只要$\,g\neq 0\,$), 超对称性对任何有限的$\,g\,$ 都不可能是破缺的.

\section{超对称流的求和规则}  \label{sec:29.2}

我们现在转向研究一些求和规则, 它们能给出真空能与描述超对称破缺强度的参量之间的精确定量关系.

我们再次假定世界被放在一个体积为$\,V\,$的箱子中, 并附上周期性边界条件以保持平移不变性. 这样, %
反对易关系(\ref{25.2.36})的真空期望值可以表示成对离散态$\,\lvert X,\text{Box}\rangle\,$求和:
\begin{align}
    &\sum_{X} \langle \text{VAC}\vert Q_{\alpha} \vert X,\text{Box}\rangle
    \langle \text{VAC}\vert Q_{\beta}\vert X,\text{Box} \rangle^{\ast} \nonumber \\
    &+\sum_{X} \langle \text{VAC}\vert Q_{\beta}^{\ast} \vert X,\text{Box}\rangle
    \langle \text{VAC}\vert Q_{\alpha}\vert X,\text{Box} \rangle^{\ast} \nonumber \\
    &\qquad \qquad =-2\mi(\gamma_{\mu}\beta)_{\alpha\beta} \langle \text{VAC}\vert P^{\mu} \vert\text{VAC}\rangle \:,
    \label{29.2.1}
\end{align}
其中指标$\,X\,$是指态被归一化成它们的标量积是克罗内克-\,$\delta\,$而不是$\,\delta\,$-函数. 令$\,\beta=\alpha$, %
对$\,\alpha\,$求和并使用方程(\ref{25.2.37})给出
\begin{equation}
    \sum_{X,\alpha} \Bigl\lvert \langle \text{VAC}\vert Q^{\alpha} \vert\text{VAC}\rangle \Bigr\rvert^{2}
    =4 \langle \text{VAC}\vert P^{0} \vert\text{VAC}\rangle  \:. \label{29.2.2}
\end{equation}
由于$\,Q_{\alpha}\,$与\,4\,-动量对易, 只有\,3\,-动量为零且能量与真空相同的态才会对这个求和有贡献. %
为了得到在方程(\ref{29.2.2})中真正有贡献的矩阵元对体积的依赖, 我们注意到, 根据方程(\textcolor{foo}{3.4.3}), %
一个有$\,N_{X}$ 个粒子的箱归一化态$\,\lvert X,\text{Box}\rangle\,$与相应的连续归一化态$\,\lvert X\rangle\,$的关系是
\begin{equation}
    \lvert X,\text{Box} \rangle = \Bigl((2\uppi^{3})/V\Bigr)^{N_{X}/2}\lvert X\rangle \:.\label{29.2.3}
\end{equation}
对于$\,\mathrm{p}_{X}=0\,$的态, 对超对称流的时间分量$\,S_{\alpha}^{0}\,$的空间积分给出另一个$\,V\,$因子, 所以对于$\,\mathrm{p}_{X}=0\,$的箱归一化态,
我们得出
\begin{equation}
    \langle \text{VAC}\vert Q_{\alpha} \vert X,\text{Box}\rangle =
    (2\uppi)^{3N_{X}/2}V^{1-N_{X}/2}\langle \text{VAC}\vert S_{\alpha}^{0} \vert X\rangle\:.\label{29.2.4}
\end{equation}
由于在$\,2\uppi\,$-旋转下的不变性使得$\,X\,$不能是零粒子态, 所以在$\,V\to\infty\,$时, 方程(\ref{29.2.2})中的主导项将是那些来自单粒子态的项.
在这个极限下, 方程(\ref{29.2.2})变成
\begin{equation}
    (2\uppi)^{3}\sideset{}{^{(0)}}\sum_{X,\alpha}\Bigl\lvert
   \langle \text{VAC}\vert S_{\alpha}^{0} \vert X\rangle \Bigr\rvert^{2} = 4\,\rho_{\text{VAC}} \:, \label{29.2.5}
\end{equation}
其中$\,\rho_{\text{VAC}}\,$是真空能密度
\begin{equation}
    \rho_{\text{VAC}}\equiv \langle \text{VAC}\vert P^{0} \vert\text{VAC}\rangle/V \:,\label{29.2.6}
\end{equation}
而上标$\,(0)\,$是指方程(\ref{29.2.5})中的求和只取\,4\,-动量为零的单粒子态. 它们显然使戈德斯通微子的两个螺旋度态.

我们从方程(\ref{29.2.5})中再次看到, 如果真空能密度不为零, 那么真空在超对称变换下不是不变的, 而是变成一个单戈德斯通微子态.
相反, 方程(\ref{29.2.2})表明, 如果真空在超对称变换下不变, 那么根据方程(\ref{29.2.2}), 它在有一个有限大的箱子中的能量不可能为零,
尽管超对称变换可能将真空变成多粒子态, 而在这种情况下真空能密度将在大体积极限下为零.

为了计算方程(\ref{29.2.5})中的单戈德斯通微子贡献, 我们用\,Lorentz\,不变性将超对称流在真空和动量为$\,\mathbf{p}\,$且%
螺旋度为$\,\lambda\,$的单戈德斯通微子态$\,\lvert \mathbf{p},\sigma\rangle\,$之间的矩阵元写成如下的形式,\footnote{Lorentz\,不变性自身给出的公式中,
对分别正比于$\,(1+\gamma_{5})/2\,$和$\,(1-\gamma_{5})/2\,$的矩阵, 其系数$\,F_{L},F_{L}'$和$\,F_{R},F_{R}'\,$是独立的.
附加$\,F_{R}=F_{L}^{\ast}\,$和$\,F_{R}'=F_{L}'^{\ast}\,$这两个关系的是$\,\mathsf{CPT}\,$不变性. 为了看到这点, %
我们必须使用超对称流的$\,\mathsf{CPT}\,$-变换性质
\[
\mathsf{CPT}\,S^{\mu}(x)(\mathsf{CPT})^{-1}=-\gamma_{5}S^{\mu}(-x)^{\ast}=-\beta\epsilon S^{\mu}(-x)
\]
(参看\,\textcolor{foo}{5.8}\,节)以及单粒子态的$\,\mathsf{CPT}\,$-变换性质
\[
\mathsf{CPT}\lvert \mathbf{p},\lambda\rangle =\chi_{\lambda}\lvert \mathbf{p},\lambda\rangle
\]
其中$\,\chi_{\lambda}\,$是一个相位因子, 它依赖于我们如何定义螺旋度态的相对相位. 我们还需要系数函数$\,u(\mathbf{p},\lambda)\,$的实性质.
它们通过定义
\[
\langle\text{VAC}\vert\psi_{\text{REN}}(x)\vert \mathbf{p},\lambda\rangle =(2\uppi)^{-3/2}\exp(\mi p\cdot x)\,u(\mathbf{p},\lambda)
\]
依赖于单利态的那些性质, 其中$\,\psi_{\text{REN}}(x)\,$是重整化\,Majorana\,场, 它的$\,\mathsf{CPT}\,$-变换性质是
\[
\mathsf{CPT}\,\psi_{\text{REN}}(x)(\mathsf{CPT})^{-1} =\gamma_{5}\,\psi_{\text{REN}}(-x)^{\ast}=\beta\,\epsilon\,\psi_{\text{REN}}(-x)\,,
\]
这给出$\,u(\mathbf{p},\lambda)=\chi_{\lambda}^{\ast}\,\beta\,\epsilon\,u^{\ast}(\mathbf{p},-\lambda)$. %
再加上超对称流和单粒子态的$\,\mathsf{CPT}\,$-变换性质, 这给出了关系$\,F_{R}=F_{L}^{\ast}\,$和$\,F_{R}'=F_{L}'^{\ast}$.
}
\begin{align}
    \langle \text{VAC}\vert S^{\mu}(0) \vert \mathbf{p},\lambda\rangle
    &=(2\uppi)^{-3/2} \Biggl[ \biggl(\frac{1+\gamma_{5}}{2}\biggr)\Bigl(\gamma^{\mu}F+\mi p^{\mu}F^{\prime}\Bigr)
    \nonumber \\
    &\quad + \biggl(\frac{1-\gamma_{5}}{2}\biggr)\Bigl(\gamma^{\mu}F^{\ast}+\mi p^{\mu}F^{\prime\ast}\Bigr)\Biggr]
    u(\mathbf{p},\lambda) \:, \label{29.2.7}
\end{align}
其中$\,u(\mathbf{p},\lambda)\,$是\,\textcolor{foo}{5.5}\,节引入的无质量\,Dirac\,场的系数函数, 而$\,F\,$和$\,F'\,$是未知常数. %
因为$\,u(\mathbf{p},\lambda)\,$满足质量为零的动量空间\,Dirac\,方程(\textcolor{foo}{5.5.42}), 再加上$\,p^{\mu}\,$在光锥上,
所以矩阵元(\ref{29.2.7})满足守恒条件$\,p_{\mu}\langle \text{VAC}\vert S^{\mu}(0)\vert\mathbf{p},\lambda\rangle=0$. 对螺旋度的求和给出
\[
\sum_{\lambda} u(\mathbf{p},\lambda)\,\bar{u}(\mathbf{p},\lambda) = -\mi \,\slashed{p}/2p^{0} \:.
\]
(对于动量为$\,\mathbf{p}\,$且自旋为$\,\lambda\,$的无质量粒子, Dirac\,旋量$\,u(\mathbf{p},\lambda)\,$在$\,\mathbf{p}\to0\,$时不是合理定义的,
但如果我们认为方程(\ref{29.2.5})中对$\,X\,$的求和视为对固定方向的小动量$\,\mathbf{p}\,$的螺旋度求和, 那就没有问题了.)
在一个直接计算后, 我们就有
\[
4\,\rho_{\text{VAC}} = \operatorname{Tr}\Biggl\{
 \biggl(F\frac{1+\gamma_{5}}{2}+F^{\ast}\frac{1-\gamma_{5}}{2}\biggr)\gamma^{0}\frac{-\mi\slashed{p}\beta}{2p^{0}}
 (\gamma^{0})^{\dag}\biggl(F^{\ast}\frac{1+\gamma_{5}}{2}+F\frac{1-\gamma_{5}}{2}\biggr)\Biggr\} \:,
\]
因而
\begin{equation}
    \rho_{\text{VAC}} = \lvert F \rvert^{2}/2 \:. \label{29.2.8}
\end{equation}
本节末尾会给出这个公式的另一证明.

参量$\,F\,$在软戈德斯通微子相互作用中起到的作用与(\textcolor{foo}{19.4}\,节引入的)参量$\,F_{\pi}\,$在软$\,\pi\,$相互作用中起到的作用相差无几. %
超对称流在任意两个态$\,X\,$和$\,Y\,$之间的矩阵元可以分成在动量传递$\,p\equiv p_{X}-p_{Y}\,$的$\,p^{\mu}=0\,$处有极点的项和没有这个极点的项
\begin{align}
    \langle X\vert S^{\mu}(0)\vert Y\rangle &= \Biggl\{ \biggl(\frac{1+\gamma_{5}}{2}\biggr)\,[\gamma^{\mu}F+\mi p^{\mu}F^{\prime}]+\biggl(\frac{1-\gamma_{5}}{2}\biggr)\,[\gamma^{\mu}F^{\ast}+\mi p^{\mu}F^{\prime\ast}]\Biggr\}\nonumber\\
    &\quad \times \biggl(\frac{-\mi\,\slashed{p}}{p^{2}}\biggr)\,M(X\to Y+g)
    +\langle X\vert S^{\mu}(0)\vert Y \rangle_{\text{no pole}}\:, \label{29.2.9}
\end{align}
其中$\,\bar{u}M(X\to Y+g)\,$是发射一个\,4\,-动量为$\,p\,$且有\,Dirac\, 波函数$\,u\,$的戈德斯通微子的振幅,
而下标``no pole''(无极点)是指矩阵元中在\,4\,-动量$\,p\,$上没有单戈德斯通微子极点的项. 流$\,S^{\mu}\,$的守恒告诉我们如果与$\,p_{\mu}\,$收缩,
所以在$\,p^{\mu}\to0\,$的极限下, 发射戈德斯通微子的振幅是\footnote{振幅$\,\langle X\vert S^{\mu}(0)\vert Y\rangle_{\text{no pole}}\,$%
在$\,p^{\mu}\to0\,$时可能会有$\,1/p\cdot k\,$这样的极点; 这样的极点不会来自于戈德斯通微子传播子, 这样的传播子已经明确地从这个矩阵元中排除了,
而是来自于在过程$\,X\to Y\,$的动量为$\,k\,$的外线上插入超对称流时产生的另一粒子传播子. 在$\,p^{\mu}\to 0\,$的极限下, %
这些极点在方程(\ref{29.2.10})中的贡献将主导发散或吸收软戈德斯通微子的振幅. 但为了产生这样的极点, %
戈德斯通微子只能在一对统计相反的{\kai{简并}}粒子之间的跃迁中产生, 而这样的跃迁不可能出现在超对称自发破缺的理论中.
软戈德斯通微子的相互作用在这方面与软$\,\pi\,$子, 光子或引力子的相互作用有所不同.}
\begin{equation}
    M(X\to Y+g) \to -\mi \biggl\{\biggl(\frac{1+\gamma_{5}}{2F}\biggr)+\biggl(\frac{1-\gamma_{5}}{2F^{\ast}}\biggr)
    \biggr\}\,p_{\mu}\,\langle X\vert S^{\mu}(0)\vert Y \rangle_{\text{no pole}}\:. \label{29.2.10}
\end{equation}

另外一个求和规则提供了在超对称性自发破缺时会出现戈德斯通微子的另一证明, 而且这个证明也将参量$\,F\,$及真空能密度联系到了表征超对称破缺强度的$\,D\,$-项%
和$\,\mathscr{F}\,$-项. (求和规则的这种用法类似于\,\textcolor{foo}{19.2}\,节中第二种证明在普通对称性破缺时会出现戈德斯通玻色子的方法.) %
为了推导这个求和规则, 我们将舍弃有限体积这个手段, 而为了避免积分(\ref{29.1.3})的收敛性问题, 则将使用作用量超对称性的定域结果
\begin{equation}
    \Bigl[\Bigl(\bar{S}^{0}(\mathbf{x},t)\,\alpha \Bigr),\chi(\mathbf{y},t)\Bigr] =
    \Bigl[\bar{\alpha}\,\Bigl(\bar{S}^{0}(\mathbf{x},t)\Bigr),\chi(\mathbf{y},t)\Bigr] =
    \mi\,\delta^{3}(\mathbf{x}-\mathbf{y})\,\delta\chi(\mathbf{x},t)+\cdots \:, \label{29.2.11}
\end{equation}
其中$\,\chi(x)\,$是一个任意的费米或玻色场, $\delta\chi(x)\,$是$\,\chi(x)\,$在带有无限小参量$\,\alpha\,$的超对称变换下的变化, %
这个超对称变换则保持作用量不变, 而省略号代表包含$\,\delta^{3}(\mathbf{x}-\mathbf{y})\,$的导数的项. %
我们来考虑一个任意左手旋量场$\,\psi_{L}(x)\,$与超对称流协变共轭$\,\bar{S}^{\mu}(x)\,$的反对易子的真空期望值.
通过对中间态的完备集$\,\lvert X\rangle\,$求和(包含对粒子动量的积分), 这个真空期望值可以变成如下的形式
\begin{equation}
    \Bigl\langle \text{VAC} \Bigl\vert \Bigl\{\psi_{L\alpha}(x),\bar{S}^{\mu}_{\beta}(y)\Bigr\}\Bigr\vert \text{VAC}\Bigr\rangle = \int \dif^{4}p\:\me^{\mi p\cdot (x-y)}\,
    \Bigl[G_{\alpha\beta}^{\mu}(p)+\tilde{G}_{\alpha\beta}^{\mu}(-p) \Bigr] \:, \label{29.2.12}
\end{equation}
其中
\begin{align}
    G_{\alpha\beta}^{\mu}(p) &= \sum_{X}\delta^{4}(p-p_{X})\,\langle \text{VAC} \vert \psi_{L\alpha}(0)\vert X\rangle\,
    \langle X\vert \bar{S}^{\mu}_{\beta}(0)\vert \text{VAC}\rangle \:, \label{29.2.13} \\
    \tilde{G}_{\alpha\beta}^{\mu}(p)&= \sum_{X}\delta^{4}(p-p_{X})\,\langle \text{VAC} \vert \bar{S}^{\mu}_{\beta}(0)\vert X\rangle\,
    \langle X\vert  \psi_{L\alpha}(0) \vert \text{VAC}\rangle \:. \label{29.2.14}
\end{align}
Lorent\,不变性要求矩阵$\,G^{\mu}(p)\,$和$\,\tilde{G}^{\mu}(p)\,$必须采取如下的形式:
\begin{align}
    G^{\mu}(p) &= \theta(p^{0})\,\biggl(\frac{1+\gamma_{5}}{2}\biggr)\Biggl[ \gamma^{\mu}G^{(1)}(-p^{2})
    +\slashed{p}p^{\mu}G^{(2)}(-p^{2}) \nonumber \\
    &\quad +p^{\mu} G^{3}(-p^{2})+\slashed{p}\gamma^{\mu}G^{(4)}(-p^{2}) \Biggr] \label{29.2.15}
\end{align}
和
\begin{align}
    \tilde{G}^{\mu}(p) &= \theta(p^{0})\,\biggl(\frac{1+\gamma_{5}}{2}\biggr)\Biggl[ \gamma^{\mu}\tilde{G}^{(1)}(-p^{2})
    +\slashed{p}p^{\mu}\tilde{G}^{(2)}(-p^{2}) \nonumber \\
    &\quad +p^{\mu} \tilde{G}^{3}(-p^{2})+\slashed{p}\gamma^{\mu}\tilde{G}^{(4)}(-p^{2}) \Biggr] \label{29.2.16}
\end{align}
对方程(\ref{29.2.15})和(\ref{29.2.16})的右手边, 如果我们我们将$\,-p^{2}\,$换成$\,m^{2}\,$, 再乘上一个因子$\,\delta(p^{2}+m^{2})\,$并对$\,m^{2}\,$积分, %
这两个方程是不受影响的. 以这种方式, 方程(\ref{29.2.12})变成
\begin{align}
    &(2\uppi)^{-3}\Bigl\langle \text{VAC}\Bigr\vert \Bigl\{\psi_{L}(x),\bar{S}^{\mu}(y)\Bigr\}\Bigr\vert \text{VAC}\Bigr\rangle = \biggl(\frac{1+\gamma_{5}}{2}\biggr) \int^{\infty}_{0}\dif m^{2}\:\Biggl[\gamma^{\mu}
    G^{(1)}(m^{2}) \nonumber \\
    &-\slashed{\partial}\partial^{\mu}G^{(2)}(m^{2})-\mi\partial^{\mu}G^{(3)}(m^{2})
    -\mi\,\slashed{\partial}\gamma^{\mu} G^{(4)}(m^{2})\Biggr]\Delta_{+}(x-y,m) \nonumber \\
    &\quad + \biggl(\frac{1+\gamma_{5}}{2}\biggr)\int^{\infty}_{0}\dif m^{2}\:\Biggl[\gamma^{\mu}
    \tilde{G}^{(1)}(m^{2})  -\slashed{\partial}\partial^{\mu}\tilde{G}^{(2)}(m^{2})-\mi\partial^{\mu}\tilde{G}^{(3)}(m^{2})
    \nonumber \\
    &\quad\qquad -\mi\,\slashed{\partial}\gamma^{\mu} \tilde{G}^{(4)}(m^{2})\Biggr]\Delta_{+}(y-x,m) \:, \label{29.2.17}
\end{align}
其中$\,\Delta_{+}(x,m)\,$是标准函数
\begin{equation}
    \Delta_{+}(x,m) \equiv (2\uppi)^{-3}\int \dif^{4}p\:\theta(p^{0})\,\delta(p^{2}+m^{2})\,\exp(\mi p\cdot x)\:.\label{29.2.18}
\end{equation}
因果律要求方程(\ref{29.2.17})左边的反对易子对于类空间隔的$\,x-y\,$应该为零. 对这样的间隔, $\Delta_{+}(x-y,m^{2})\,$是$\,x-y\,$的偶函数, %
所以方程(\ref{29.2.17})对于所有类空间隔为零当且仅当
\begin{equation}
    \begin{split}
        G^{(1)}(m^{2})&= -\tilde{G}^{(1)}(m^{2})\:, \qquad G^{(2)}(m^{2})=-\tilde{G}^{(2)}(m^{2})\:, \\
        G^{(3)}(m^{2})&= +\tilde{G}^{(3)}(m^{2})\:, \qquad G^{(4)}(m^{2})=+\tilde{G}^{(4)}(m^{2})\:,
    \end{split} \label{29.2.19}
\end{equation}
这使得对于一般的$\,x-y$, 方程(\ref{29.2.17})现在变成
\begin{align}
    &(2\uppi)^{-3}\Bigl\langle \text{VAC}\Bigr\vert \Bigl\{\psi_{L}(x),\bar{S}^{\mu}(y)\Bigr\}\Bigr\vert \text{VAC}\Bigr\rangle = \int_{0}^{\infty}\dif m^{2}\:\Biggl[ G^{(1)}(m^{2})\gamma^{\mu}
    -G^{(2)}(m^{2})\,\slashed{\partial}\partial^{\mu} \nonumber \\
    & -\mi G^{(3)}(m^{2})\partial^{\mu}-\mi G^{(4)}(m^{2})\,\slashed{\partial}\gamma^{\mu}\Biggr]\,
    \Delta(x-y,m)\,\biggl(\frac{1+\gamma_{5}}{2}\biggr) \:, \label{29.2.20}
\end{align}
其中, 像往常一样,
\begin{equation}
    \Delta(x-y,m)\equiv \Delta_{+}(x-y,m)-\Delta_{+}(y-x,m) \:. \label{29.2.21}
\end{equation}
我们接下来附加超对称流守恒条件(\ref{29.1.2}), 这(因为$\,\square \Delta=m^{2}\Delta$)给出
\begin{equation}
    G^{(1)}(m^{2})=m^{2}G^{(2)}(m^{2})\:, \qquad m^{2}G^{(3)}(m^{2})=-m^{2}G^{(4)}(m^{2}) \:. \label{29.2.22}
\end{equation}
最后, 我们必须将这些谱函数联系到超对称性的破缺. 回忆起对$\,x^{0}=y^{0}$,
\[
    \Delta(\mathbf{x}-\mathbf{y},0,m)=0 \:, \qquad \dot{\Delta}(\mathbf{x}-\mathbf{y},0,m)=-\mi\delta^{3}(\mathbf{x}-\mathbf{y}) \:.
\]
因此在方程(\ref{29.2.20})中令$\,x^{0}=y^{0}=t\,$以及$\,\mu=0\,$给出
\begin{align}
    &\Bigl\langle \text{VAC}\Bigr\vert \Bigl\{\psi_{L}(\mathbf{x},t),\bar{S}^{0}(\mathbf{y},t)\Bigr\}\Bigr\vert \text{VAC}\Bigr\rangle = (2\uppi)^{3}\delta^{3}(\mathbf{x}-\mathbf{y})\biggl(\frac{1+\gamma_{5}}{2}\biggr)
    \nonumber \\
    &\phantom{\Bigl\langle \text{VAC}\Bigr\vert \Bigl\{\psi_{L}} \times \int_{0}^{\infty}\dif m^{2}\:\biggl[G^{(3)}(m^{2})+G^{(4)}(m^{2})\biggr] \:. \label{29.2.23}
\end{align}
然后在右边与一个无限小的\,Majorana\,费米型超对称变换参量$\,\alpha\,$收缩并使用方程(\ref{29.2.11}), 这给出
\begin{equation}
    \mi\langle \delta\psi_{L}\rangle_{\text{VAC}} = (2\uppi)^{3}\biggl(\frac{1+\gamma_{5}}{2}\biggr)\,\alpha\,
    \int_{0}^{\infty}\dif m^{2}\:\Bigl[G^{(3)}(m^{2})+G^{(4)}(m^{2})\Bigr] \:. \label{29.2.24}
\end{equation}
但方程(\ref{29.2.22})表明, 除了有可能在$\,m^{2}=0\,$处不为零, 方程(\ref{29.2.24})中对$\,m^{2}\,$积分的被积函数处处为零, 所以我们可以得出
\begin{equation}
    G^{(3)}(m^{2})+G^{(4)}(m^{2}) = \delta(m^{2})\mathscr{G} \:, \label{29.2.25}
\end{equation}
其中常数系数$\,\mathcal{G}\,$由
\begin{equation}
     \mi\langle \delta\psi_{L}\rangle_{\text{VAC}} = (2\uppi)^{3}\mathscr{G}\alpha_{L} \label{29.2.26}
\end{equation}
给出. 正如我们在\,\ref{sec:26.4}\,和\,\ref{sec:27.4}\,节中所看到的, 超对称性破缺的标志是一个或多个旋量场$\,\psi\,$在超对称变换下的变化$\,\delta\,$有真空期望值. 方程(\ref{29.2.25})和(\ref{29.2.26})表明对于任何这样的旋量场, 谱函数$\,G^{(3)}(m^{2})+G^{(4)}(m^{2})\,$在$\,m^{2}=0\,$处有一个\,$\delta\,$-函数奇异性, 而这仅能来自于在方程(\ref{29.2.13})和(或)\\
(\ref{29.2.14})的求和中出现了一个无质量单粒子态$\,\lvert g\rangle$. %
为了使矩阵元$\,\langle\text{VAC}\vert\psi\vert g\rangle\,$或$\,\langle g\vert \psi\vert \text{VAC}\rangle\,$不为零, 这个无质量粒子的自旋必须是$\,1/2$. %
这即是戈德斯通微子.

现在我们来计算动量为$\,\mathbf{p}\,$且螺旋度为$\,\lambda\,$的单戈德斯通微子态$\,\lvert \mathbf{p},\lambda\rangle\,$对谱函数$\,G^{(i)}(m^{2})\,$的贡献. %
Lorentz\,不变性告诉我们方程(\ref{29.2.13})中一般费米子场(不仅仅是重整化戈德斯通微子场)的矩阵元采取如下的形式\footnote{像前面一样, %
Lorentz\,不变性将分别对$\,(1+\gamma_{5})\,$和$\,(1-\gamma_{5})\,$给出两个独立的系数$\,N_{L}\,$和$\,N_{R}$. 通过使用本节第一个脚注中讨论的单粒子态%
的$\,\mathsf{CPT}\,$变换性质和系数函数的实性质, 以及一般费米场的$\,\mathsf{CPT}\,$变换, $\mathsf{CPT}\psi(x)\mathsf{CPT}^{-1}=\beta\epsilon\psi(-x)$, %
我们发现$\,N_{L}=N_{R}^{\ast}$.}
\begin{equation}
    \langle \text{VAC}\vert \psi(0)\vert \mathbf{p},\lambda \rangle = \frac{1}{(2\uppi)^{3/2}}
    \biggl[ N\biggl(\frac{1+\gamma_{5}}{2}\biggr) + N^{\ast}\biggl(\frac{1-\gamma_{5}}{2}\biggr)\biggr]\,
    u(\mathbf{p},\lambda) \:, \label{29.2.27}
\end{equation}
其中$\,N\,$是表征特定费米场的常数. 另外, 方程(\ref{29.2.13})中的$\,\delta\,$-函数是

\[
    \delta^{4}(p-p_{g})=2p^{0}\delta(p^{2})\theta(p^{0})\delta^{3}(\mathbf{p}-\mathbf{p}_{g}) \:.
\]
加上方程(\ref{29.2.7}), 这给出
\[
\Bigl[G_{\alpha\beta}^{\mu}(p)\Bigr]_{1} = \frac{N}{(2\uppi)^{3}} \biggl(\frac{1+\gamma_{5}}{2}\biggr)\,
\theta(p^{0})\delta(p^{2})\,\slashed{p}\Bigl[\gamma^{\mu}F+\mi p^{\mu}F^{\prime}\Bigr] \:,
\]
下标$\,1\,$是指这是单戈德斯通微子贡献. 与方程(\ref{29.2.15})比较表明
\begin{equation}
    \begin{split}
        \Bigl[G^{(2)}(m^{2})\Bigr]_{1} &= \mi(2\uppi)^{-3}NF^{\prime}\delta(m^{2})\:,\qquad
        \Bigl[G^{(4)}(m^{2})\Bigr]_{4} = (2\uppi)^{-3}NF\delta(m^{2})\:, \\
        \Bigl[G^{(1)}(m^{2})\Bigr]_{1} &=  \Bigl[G^{(3)}(m^{2})\Bigr]_{1}=0 \:.
    \end{split} \label{29.2.28}
\end{equation}
因此方程(\ref{29.2.25})和(\ref{29.2.26})给出
\begin{equation}
     \mi\langle \delta\psi_{L}\rangle_{\text{VAC}} = NF \alpha_{L}\:. \label{29.2.29}
\end{equation}
显然, 如果任何费米子场在一个超对称变换下的变化有非零的真空期望值, 那么这个场的因子$\,N$ 不可能为零, 所以对这些谱函数必然有一个单戈德斯通微子的贡献.

更具体一些, 回忆起左手征标量超场$\,\Phi_{Ln}\,$的费米分量$\,\psi_{Ln}\,$服从超对称变换规则(\ref{26.3.15}):
\begin{equation}
    \delta\psi_{Ln}=\sqrt{2}\partial_{\mu}\phi_{n}\,\gamma^{\mu}\,\alpha_{R}+\sqrt{2}\mathscr{F}_{n}\,\alpha_{L}\:,
    \label{29.2.30}
\end{equation}
而方程(\ref{27.3.5})给出的规范微子的超对称变换是
\begin{equation}
    \delta\lambda_{A}=\Bigl(\tfrac{1}{4}f_{A\mu\nu}\,[\gamma^{\nu},\gamma^{\mu}]+\mi\gamma_{5}D_{A}\Bigr)\alpha\:.
    \label{29.2.31}
\end{equation}
因此, 对$\psi_{n}\,$和$\,\lambda_{A}\,$在真空和一个单戈德斯通微子态之间的矩阵元, 其中的$\,N\,$-因子给定为
\begin{equation}
    N_{n} =\mi F^{-1}\,\sqrt{2}\langle \mathscr{F}_{n}\rangle_{\text{VAC}}\:,\qquad
    N_{A}=-F^{-1}\,\langle D_{A}\rangle_{\text{VAC}} \:. \label{29.2.32}
\end{equation}


我们来看一下结果(\ref{29.2.32})是如何出现在树级近似中的. 对于规范超场和手征超场的一个可重整理论, 方程(\ref{27.4.30})给出了费米子质量矩阵的左手部分
\begin{equation}
    \begin{split}
        M_{nm}&=\Biggl(\frac{\partial^{2}f(\phi)}{\partial\phi_{n}\partial\phi_{m}}\Biggr)_{\phi=\phi_{0}} \:, \\
        M_{nA}&=M_{An}=\mi\sqrt{2}(t_{A}\phi_{0})_{n}^{\ast} \:, \qquad M_{AB}=0 \:.
    \end{split} \label{29.2.33}
\end{equation}
真空场$\,\phi_{n0}\,$处在方程(\ref{27.4.9})给出的势的极小值点, 所以
\begin{equation}
    0=\biggl(\frac{\partial V(\phi)}{\partial \phi_{n}}\biggr)_{0} =
    -\sum_{m}M_{nm}\mathscr{F}_{m0}+\sum_{A}\Bigl(\phi_{0}^{\dag}t_{A}\Bigr)_{n} D_{A0} \:, \label{29.2.34}
\end{equation}
其中$\,\mathscr{F}\,$和$\,D\,$由方程(\ref{27.4.6})和(\ref{27.4.7})给出
\[
\mathscr{F}_{n}=-\Bigl(\partial f(\phi)/\partial \phi_{n}\Bigr)^{\ast} \:, \qquad
D_{A}=\xi_{A}+\sum_{nm}\phi^{\ast}_{n}(t_{A})_{nm}\phi_{m} \:,
\]
而下标\,0\,是指我们设了$\,\phi_{n}=\phi_{n0}$. 更进一步, 超势的规范不变性要求, 对$\,\phi\,$的所有值,
\begin{equation}
    \sum_{n}\mathscr{F}_{n}\Bigl(t_{B}\phi\Bigr)_{n}^{\ast} =0\:. \label{29.2.35}
\end{equation}
因此左手夸克质量矩阵$\,M\,$有一个满足$\,Mv=0\,$的本征矢$\,v$, 其中
\begin{equation}
    v_{n}=\sqrt{2}\mathscr{F}_{n0}\:, \qquad v_{A}=\mi D_{A0} \:. \label{29.2.36}
\end{equation}
因此在树级近似下, 如果我们把左手费米子场展到有确定质量的粒子的重整化场上, 那么戈德斯通微子场在$\,\psi_{Ln}\,$和$\,\lambda_{AL}\,$中的系数将分别正比于%
$\,\sqrt{2}\mathscr{F}_{n0}\,$和$\,\mi D_{A0}$, 这与方程(\ref{29.2.32}) 一致.

如果我们对旋量场$\,\psi_{Ln}\,$和$\,\lambda_{A}\,$做归一化使得将这些场联系到有确定质量的粒子的重整化场的矩阵是幺正的, 那么
\begin{equation}
    \sum_{n}\lvert N_{n}\rvert^{2} +\sum_{A}\lvert N_{A}\rvert^{2}=1 \:, \label{29.2.37}
\end{equation}
所以方程(\ref{29.2.32})给出了非微扰结果
\begin{equation}
    \lvert F\rvert^{2}= 2\sum_{n}\Bigl\lvert \langle \mathscr{F}_{n}\rangle_{\text{VAC}} \Bigr\rvert^{2}
    +\sum_{A}\Bigl\lvert \langle D_{A}\rangle_{\text{VAC}} \Bigr\rvert^{2} \:. \label{29.2.38}
\end{equation}
$\lvert F\rvert^{2}\,$的结果(\ref{29.2.38})使得我们可以将真空期望值密度(\ref{29.2.8})表示成辅助场的真空期望值
\begin{equation}
    \rho_{\text{VAC}}= \sum_{n}\Bigl\lvert \langle \mathscr{F}_{n}\rangle_{\text{VAC}} \Bigr\rvert^{2}
    +\frac{1}{2}\sum_{A}\Bigl\lvert \langle D_{A}\rangle_{\text{VAC}} \Bigr\rvert^{2} \:. \label{29.2.39}
\end{equation}
这是零阶结果(\ref{27.4.9})的非微扰推广. 它证实了\,\ref{sec:27.6}\,节中使用的一个结果, 即$\,\langle \mathscr{F}_{n}\rangle=\langle D_{A}\rangle=0\,$是超对称不破缺的充要条件.

\subsection*{* * *}

在不做有限体积这个假定下看到方程(\ref{29.2.9})是如何推导出来的是有益的. 对于这个目的, 考虑两个超对称流的反对易子的真空期望值. %
同上一节一样, 通过使用\,Lorentz\,不变性以及反对易子在类空间隔上为零, 我们发现
\begin{align}
     &\Bigl\langle \text{VAC}\Bigr\vert \Bigl\{S^{\mu}(x),\bar{S}^{\nu}(y)\Bigr\}\Bigr\vert \text{VAC}\Bigr\rangle
     = -\mi \int \dif m^{2}\, \Bigl[ H^{(1)}(m^{2})\gamma^{\mu}\partial^{\nu}
     +H^{(2)}(m^{2})\gamma^{\nu}\partial^{\mu}\nonumber\\
     &\quad +H^{(3)}(m^{2})\,\slashed{\partial}\partial^{\mu}\partial^{\nu}
    +H^{(4)}(m^{2})\,\slashed{\partial}\eta^{\mu\nu}+H^{5}(m^{2})\epsilon^{\mu\nu\lambda\rho}
    \partial_{\lambda}\gamma_{\rho}\Bigr]\Delta(x-y,m^{2}) \nonumber \\
    &\quad +\cdots \label{29.2.40}
\end{align}
其中
\begin{align}
    &\int \dif X\:\delta^{4}(p-p_{X})\,\langle\text{VAC}\vert S_{\alpha}^{\mu}(0)\vert X\rangle \,
    \langle X \vert \bar{S}^{\nu}_{\beta}(0)\vert \text{VAC}\rangle = H^{(1)}(-p^{2})\gamma^{\mu}p^{\nu} \nonumber \\
    &+H^{2}(-p^{2})\gamma^{\nu}p^{\mu} - H^{(3)}(-p^{2})\,\slashed{p}p^{\mu}p^{\nu}
    +H^{(4)}(-p^{2})\,\slashed{p}\eta^{\mu\nu} \nonumber \\
    &+H^{(5)}(-p^{2})\epsilon^{\mu\nu\lambda\rho}p_{\lambda}\gamma_{\rho} +\cdots \:,\label{29.2.41}
\end{align}
而方程(\ref{29.2.40})和(\ref{29.2.41})代表其它独立\,Dirac\,协变矩阵$1$, $\gamma_{5}$, $\gamma_{5}\gamma_{\sigma}$\,和%
$\,[\gamma_{\sigma},\gamma_{\tau}]\,$的线性组合, 这些矩阵在这里对我们的问题没有影响. 流的\,Majorana\,特性加上方程(\ref{26.A.20})告诉我们%
谱函数$\,H^{(i)}\,$都是实的, 而超对称流的守恒律则表明
\begin{equation}
    H^{(1)}(m^{2}) = H^{(2)}(m^{2})=-m^{2}\,H^{(3)}(m^{2})-H^{(4)}(m^{2}) \:, \label{29.2.42}
\end{equation}
和
\begin{equation}
    m^{2}H^{(1)}(m^{2})=0 \:. \label{29.2.43}
\end{equation}
在方程(\ref{29.2.40})中令$\,\mu=\nu=0\,$和$\,x^{0}=y^{0}$, 对$\,\mathbf{x}\,$积分, 并使用方程(\ref{29.2.42}), 这样就给出
\begin{align}
    \Bigl\langle \text{VAC}\Bigr\vert \Bigl\{Q,\bar{S}^{0}(0)\Bigr\}\Bigr\vert \text{VAC}\Bigr\rangle &=
    (2\uppi)^{3}\beta\int\dif m^{2}\,\Bigl[H^{(1)}(m^{2})+H^{(2)}(m^{2}) \nonumber \\
    +m^{2}\,H^{(3)}(m^{2})+H^{(4)}(m^{2})\Bigr]+\cdots &= (2\uppi)^{3}\beta\int\dif m^{2}\: H^{(1)}(m^{2})+\cdots \:.
    \label{29.2.44}
\end{align}
为了计算这个反对易子, 我们首先将其写成
\begin{equation}
    \Bigl\{ Q,\bar{S}^{\nu}(x)\Bigr\}=-2\mi\gamma_{\mu}T^{\mu\nu}(x)+\cdots \:, \label{29.2.45}
\end{equation}
其中$\,T^{\mu\nu}(x)\,$是某个张量算符而省略号依旧表示其它独立协变\,Dirac\,矩阵的线性组合. $Q\,$和$\,S^{\mu}\,$的\\\,Majorana\,性质告诉我们它是守恒的, %
也就是说
\begin{equation}
    \partial_{\nu}T^{\mu\nu}=0 \:, \label{29.2.46}
\end{equation}
而反对易关系(\ref{25.2.36})告诉我们
\begin{equation}
    \int \dif^{3}x\: T^{\mu0}(x) =P^{\mu} \:. \label{29.2.47}
\end{equation}
这些性质让我们认出$\,T^{\mu\nu}\,$是能动量张量. 一般情况下, 它不等于\,\textcolor{foo}{7.4}\,节中讨论的{\kai{对称}}能动量张量 $\Theta^{\mu\nu}$, %
但方程(\ref{29.2.47})表明能量密度$\,T^{00}\,$和$\,\Theta^{00}\,$的差异仅是无法贡献到零\,3\,-动量态上的空间导数项, 所以真空能密度给定为
\begin{equation}
    \rho_{\text{VAC}}=\langle \text{VAC} \vert T^{00}\vert \text{VAC} \rangle \:. \label{29.2.48}
\end{equation}
因此方程(\ref{29.2.44})和(\ref{29.2.45})给出
\begin{equation}
    2\rho_{\text{VAC}} = (2\uppi)^{3}\int \dif m^{2}\: H^{(1)}(m^{2}) \:. \label{29.2.49}
\end{equation}
而方程(\ref{29.2.43})告诉我们, 除了可能在$\,m^{2}=0\,$处, $H^{(1)}(m^{2})\,$处处为零, 所以
\begin{equation}
    H^{(1)}(m^{2})=2(2\uppi)^{-3}\delta(m^{2})\rho_{\text{VAC}} \:. \label{29.2.50}
\end{equation}
因此我们再次看到, 真空能密度不为零暗指存在无质量费米子, 即戈德斯通微子. 使用方程(\ref{29.2.7}), 直接计算单戈德斯通微子对谱函数的贡献, 这给出
\begin{equation}
    H^{(1)}(m^{2}) = (2\uppi)^{-3}\delta(m^{2})\,\lvert F\rvert^{2} \:. \label{29.2.51}
\end{equation}
将其与方程(\ref{29.2.50})进行对比就给出我们之前对真空能密度的结果(\ref{29.2.8}).


\section{超势的非微扰修正}  \label{sec:29.3}

我们在\,\ref{sec:27.6}\,节看到, 规范和手征超场的一般超对称理论中的超势到微扰论的任意阶都不是重整化的, 这使得如果超对称在树级近似下不破缺,
那么它只能被对威尔逊型有效拉格朗日量的非微扰修正破缺. 我们在对这些修正做一个普遍分析. 在\,20\,世纪\,80\,年代早期, Affleck, Davis, Dine\,和\,Seiberg%
在一系列文章中\cite{3}彻底地研究这些修正, 他们尤其关注了有任意多个色和味的超对称版量子色动力学的情况. 这里我们给出对一个简化的对一般超对称规范理论的分析, 这个分析基于我们在\,\ref{sec:27.7}\,节使用过的由\,Seiberg\,给出的一个更加现代的全纯讨论.\cite{4}


为了研究非微扰效应, 我们应该在这里考虑一个一般的可重整超对称理论, 但现在要在拉格朗日密度引入一个可能的$\,\theta\,$-项
\begin{equation}
    \mathscr{L}=\Bigl[\Phi^{\dag}\,\me^{-V}\,\Phi\Bigr]_{D} + 2\operatorname{Re}\Bigl[f(\Phi)\Bigr]_{\mathscr{F}}
    +\operatorname{Re}\Biggl[\frac{\tau}{8\uppi\mi}\sum_{A\alpha\beta}\epsilon_{\alpha\beta}W_{A\alpha L}W_{A\beta L}\Biggr]_{\mathscr{F}} \:, \label{29.3.1}
\end{equation}
其中超势$\,f(\Phi)\,$是左手征超场的规范不变三次多项式而$\,\tau\,$是参量(\ref{27.3.23}):
\begin{equation}
    \tau= \frac{4\uppi\mi}{g^{2}} + \frac{\theta}{2\uppi} \:. \label{29.3.2}
\end{equation}
和\,\ref{sec:27.6}\,节一样, 我们引入一对规范不变的外左手征超场, 现在记做$\,Y\,$和$\,T$, 并将拉格朗日密度换成
\begin{equation}
    \mathscr{L}^{\sharp} = \Bigl[\Phi^{\dag}\,\me^{-V}\,\Phi\Bigr]_{D} + 2\operatorname{Re}\Bigl[Y\,f(\Phi)\Bigr]_{\mathscr{F}}
    +\operatorname{Re}\Biggl[\frac{T}{8\uppi\mi}\sum_{A\alpha\beta}\epsilon_{\alpha\beta}W_{A\alpha L}W_{A\beta L}\Biggr]_{\mathscr{F}} \:. \label{29.3.3}
\end{equation}
当我们把$\,Y\,$和$\,T\,$的旋量分量和辅助分量设为零, 并把它们的标量分别取为$\,y=1\,$和$\,t=\tau$, 这就变成了方程(\ref{29.3.1}).
基于\,\ref{sec:27.6}\,节的分析, 非微扰效应一般会使得两个对称性失效. 平移算符在我们现在的记法下是$\,T\to T+\xi$, 其中$\,\xi\,$是实的,
由于$\,\sum_{A}\epsilon_{\mu\nu\rho\sigma}f_{A}^{\mu\nu}f_{A}^{\rho\sigma}\,$ 对时空的积分不为零, 平移不再是对称性. %
因为第\,\textcolor{foo}{22}\,章讨论的反常赋予了\,$R$\,-流不为零的散度, 原始的$\,R\,$不变性($\,T\,$和$\,Y\,$的$\,R\,$值是\,0\,和$\,+2$)不再是一个对称性. %
$\theta_{L}\,$和$\,\theta_{R}\,$分别有$\,R=+1\,$和$\,R=-1$, $V_{A}\,$和$\,\Phi_{n}\,$是$\,R\,$中性的, %
费米子场$\,\lambda_{AL}\,$和$\,\psi_{nR}\,$分别有$\,R=+1\,$和$\,R=-1$, 所以方程(\textcolor{foo}{22.2.26})在这里给出
\begin{equation}
    \partial_{\mu}J^{\mu}_{R}=-\frac{1}{32\uppi^{2}}\Bigl(C_{1}-C_{2}\Bigr)\sum_{A}\epsilon_{\mu\nu\rho\sigma}
    f_{A}^{\mu\nu}f_{A}^{\rho\sigma} \:, \label{29.3.4}
\end{equation}
其中$\,C_{1}\,$和$\,C_{2}\,$是方程(\textcolor{foo}{17.5.33})和(\textcolor{foo}{17.5.34})是方程中定义的常数:
\begin{equation}
    \sum_{CD}C_{ACD}C_{BCD} = C_{1}\,\delta_{AB}\:, \qquad
    \operatorname{Tr}\{t_{A}t_{B}\}=C_{2}\,\delta_{AB} \:, \label{29.3.5}
\end{equation}
其中的迹取遍所有手征超场的种类.\footnote{方程(\ref{29.3.4})中分母里面出现的不是\,16\,而是\,32\,是因为规范微子没有可区分的反粒子,
而我们现在在方程(\ref{29.3.5})中取迹的时候对粒子和反粒子分别计数. 另外, 我们现在采取的约定是\,\ref{sec:27.3}\,节末尾描述的约定, %
即将规范耦合因子吸收进规范常数而非结构常数以及矩阵生成元$\,t_{A}$. 因此规范生成元的归一化是使得对规范代数的标准$\,SU(2)\,$子代数中的生成元$\,t_{A}$, %
$t_{B}\,$和$\,t_{C}\,$有$\,C_{ABC}=\epsilon_{ABC}$.} 例如, 在参考文献[3]中研究的推广的量子色动力学超对称版中, 规范群是$\,SU(N_{c})$, %
有$\,N_{f}\,$处在定义表示下的左手征夸克超场$\,Q_{n}\,$和$\,\overline{Q}_{n}$以及它们的复共轭, %
这些常数的值由方程(\textcolor{foo}{17.5.35})(其中$\,n_{f}=2N_{f}$\,)给出
\[
    C_{1}=N_{c}\:, \qquad C_{2}=N_{f} \:.
\]

尽管非微扰效应使得$\,T\,$平移和$\,R\,$不变性失去作用, 但还有一个几乎同等效力的对称性. 考虑一般的$\,R\,$变换
\begin{equation}
    \theta_{L}\to \me^{\mi\varphi}\theta_{L}\:, \quad \Phi\to \Phi \:, \quad
    V_{A}\to V_{A}\:, \quad Y\to\me^{2\mi\varphi}Y \:, \label{29.3.6}
\end{equation}
其中$\,\varphi\,$是任意实数. 拉格朗日密度(\ref{29.3.3})中与$\,T\,$无关的项在这个变换下不变, 但根据方程(\ref{29.3.4}), 量子效应破坏了对称性,
就好像拉格朗日密度中有一项$\,\Delta\mathscr{L}$, 其有如下的变换
\[
\Delta\mathscr{L}\to \Delta\mathscr{L} -\frac{1}{32\uppi^{2}}\Bigl(C_{1}-C_{2}\Bigr)\sum_{A}\epsilon_{\mu\nu\rho\sigma}
f_{A}^{\mu\nu}f_{A}^{\rho\sigma} \,\varphi \:.
\]
回忆方程(\ref{27.3.18}), 如果我们赋予$\,T\,$变换
\begin{equation}
    T \to T + (C_{1}-C_{2})\varphi/\uppi \:,\label{29.3.7}
\end{equation}
这会被抵消. 因为$\,W_{A\alpha L}\,$有$\,R=2$, 含有非微扰效应的整个理论在(\ref{29.3.6})和(\ref{29.3.7})的{\kai{组合}}变换下不变. 特别地,
超场$\,\exp(2\mi\uppi T)\,$在$\,T=\tau\,$时关于$\,\theta\,$是周期的, 这个超场的$\,R=2(C_{1}-C_{2})$.

我们再次引入一个紫外截断, 并考虑有效``威尔逊型''拉格朗日量
\begin{align}
    \mathscr{L}_{\lambda}^{\sharp} &= \Bigl[\mathscr{A}_{\lambda}(\Phi,\Phi^{\dag},V,T,T^{\dag},Y,Y^{\dag},\mathscr{D}\cdots)\Bigr]_{D} \nonumber \\
    &\quad +2\operatorname{Re}\Biggl[\frac{T}{8\uppi\mi}\sum_{A\alpha\beta}\epsilon_{\alpha\beta}W_{A\alpha L}W_{A\beta L}+\mathscr{B}_{\lambda}(\Phi,W_{L},T,Y)\Biggr]_{\mathscr{F}} \:, \label{29.3.8}
\end{align}
其中$\,\mathscr{A}_{\lambda}\,$和$\,\mathscr{B}_{\lambda}\,$是所列变量的规范不变函数. %
为了使$\,T\,$的变换(\ref{29.3.7})继续抵消$\,R\,$变换(\ref{29.3.6})中的反常, 正比于$\,T\,$的项从函数$\,\mathscr{B}_{\lambda}\,$中分离了出来. %
这样, 在组合变换(\ref{29.3.6}), (\ref{29.3.7})下的不变性就告诉我们函数$\,\mathscr{B}_{\lambda}\,$中的项必须正比于$\,\exp(2\mi\uppi T)\,$的幂函数, %
这样的函数有确定的$\,R\,$值.

更进一步, 可以出现在$\,\mathscr{B}_{\lambda}\,$中的只能是$\,\exp(2\mi\uppi)\,$的{\kai{正}}幂次. 根据方程(\ref{27.3.24}), %
只有缠绕数为$\,\nu>0\,$的瞬子才会对有效拉格朗日量有关于$\,T\,$而非$\,T^{\ast}\,$全纯的贡献, 而它们给出了因子$\,\exp(2\mi\uppi\nu T)$. %
更一般地, 当$\,T=\tau\,$时, 任意幂次的$\,\exp(2\mi a\uppi T)\,$将通过因子$\,\exp(-8\uppi^{2}a/g^{2})\,$依赖于规范耦合, %
为了使非微扰效应在$\,g\,$很小时被压低, 这使得$\,a\,$必须为正. 结果是, 取决于$\,C_{1}>C_{2}$, $C_{1}=C_{2}$\,还是$\,C_{1}<C_{2}$, %
通过算符$\,\exp(2\mi\uppi aT)\,$进入$\,\mathscr{L}^{\sharp}\,$的非微扰效应的$\,R\,$值必须是正定, 零或负定的. %
(在上面描述的推广的量子色动力学超对称版中, 这分别对应于$\,N_{c}>N_{f}$, $N_{c}=N_{f}$\,和$\,N_{c}<N_{f}$.) 我们现在将考虑各个情况.

\subsection{$C_{1}>C_{2}$}

这时, $a>0\,$的$\,\exp(2\mi a\uppi T)\,$有正定的$\,R\,$值, 由方程(\ref{29.3.7})给定为$\,R=2(C_{1}-C_{2})a$. Lorentz\,不变性告诉我们, %
如果$\,\mathcal{B}_{\lambda}\,$中的任何项包含一个$\,W_{A\alpha L}\,$因子, 那么它至少要包含两个这样的因子, %
所以在$\,\mathscr{B}_{\lambda}\,$中构造$\,R=2\,$的项的唯一方法是用两个$\,W\,$而不用$\,Y\,$或$\,T$, 或者是用一个$\,Y\,$但不用$\,W\,$或$\,T$, %
或者是用一个$\,\exp(2\mi\uppi T/(C_{1}-C_{2}))\,$因子但不用$\,W\,$或$\,Y$:
\begin{equation}
    \mathscr{B}_{\lambda} = Y\,f_{\lambda}(\Phi) + \sum_{\alpha\beta AB}\epsilon_{\alpha\beta}W_{A\alpha L}W_{B\beta L}
    \ell_{\lambda AB}(\Phi) +\exp\biggl(\frac{2\mi\uppi T}{C_{1}-C_{2}}\biggr)\,v_{\lambda}(\Phi)\:. \label{29.3.9}
\end{equation}
因为$\,f_{\lambda}(\Phi)\,$不依赖$\,Y\,$或$\,T$, 它只能是树级近似势
\begin{equation}
    f_{\lambda}(\Phi)= f(\Phi) \:, \label{29.3.10}
\end{equation}
就像在微扰论中一样. 同理, 因为$\,\ell_{\lambda AB}(\Phi)\,$不依赖$\,Y\,$或$\,T$, 它必须有一样多的$\,\Phi\,$和$\,\Phi^{\dag}$, 所以, %
既然它不依赖$\,\Phi^{\dag}$, 它不依赖于$\,\Phi$. 这样, 规范不变性(对一个单规范群)就要求$\,\ell_{\lambda AB}\,$正比于$\,\delta_{AB}$, %
又因为它不依赖$\,T\,$或$\,Y$, \ref{sec:27.6}\,节的幂次计数讨论表明$\,\delta_{AB}\,$的系数只能是对按平方反比跑动的威尔逊型规范耦合的单圈贡献.

为了对规范跑动耦合说的更明确一些, 回忆在有费米子的非超对称规范理论中, 方程(\textcolor{foo}{18.7.2}) 给出的单圈重整化群方程时
\begin{equation}
    \lambda\frac{\dif g_{\lambda}}{\dif\lambda} =bg_{\lambda}^{3} \:, \label{29.3.11}
\end{equation}
其中
\begin{equation}
    b=-\frac{1}{4\uppi^{2}}\biggl(\frac{11}{12}C_{1}-\frac{1}{6}C_{2}\biggr) \:, \label{29.3.12}
\end{equation}
因为我们现在在粒子的左手征态之外对反费米子的左手征态另外计数, 这里的系数$\,C_{2}\,$取成$\,-1/6$ 而不是$\,-1/3$. %
而正如我们在\,\ref{sec:28.2}\,节中所看到的, 规范微子的效应就是给$\,C_{1}\,$-项乘以$\,9/11$, %
而左手征超场的标量分量(例如标量夸克和标量轻子)的效应是给$\,C_{2}\,$-项乘以$\,3/2$, 所以在超对称理论中, 方程(\ref{29.3.12})变成
\begin{equation}
    b=-\frac{1}{16\uppi^{2}}(3 C_{1}-C_{2}) \:. \label{29.3.13}
\end{equation}
这样, 跑动规范耦合的方程(\ref{29.3.11})的解就是
\begin{equation}
    g_{\lambda}^{-2}=g^{-2}+\frac{3 C_{1}-C_{2}}{8\uppi^{2}}\ln\biggl(\frac{\lambda}{K}\biggr)\:, \label{29.3.14}
\end{equation}
其中$\,K\,$是一个紫外截断, 引入它是为了另一个紫外发散的裸耦合$\,g\,$以意义.

为了总结迄今为止有的结果, 令$\,T=\tau\,$和$\,Y=1$, 威尔逊型有效拉格朗日量在$\,C_{1}>C_{2}\,$时取如下的形式
\begin{align}
    \mathscr{L}_{\lambda}^{\sharp} &= \Bigl[\mathscr{A}_{\lambda}(\Phi,\Phi^{\dag},V,\tau,\tau^{\dag},\mathscr{D}\cdots)\Bigr]_{D}
    +2\operatorname{Re}\Biggl[\frac{\tau_{\lambda}}{8\uppi\mi}\sum_{A\alpha\beta}\epsilon_{\alpha\beta}W_{A\alpha L}W_{A\beta L}\Biggr]_{\mathscr{F}}\nonumber \\
    &\quad +2\operatorname{Re}[f(\Phi)]_{\mathscr{F}}+\exp\biggl(\frac{2\mi\uppi\tau_{\lambda}}{C_{1}-C_{2}}\biggr)
    [v_{\lambda}(\Phi)]_{\mathscr{F}} \:, \label{29.3.15}
\end{align}
其中
\begin{equation}
    \tau_{\lambda}=\frac{4\uppi\mi}{g_{\lambda}^{2}}+\frac{\theta}{2\uppi} \:. \label{29.3.16}
\end{equation}
我们已经可以把方程(\ref{29.3.15})中最后一项指数中的$\,\tau\,$换成$\,\tau_{\lambda}$, 能这样做的原因是它们的差是一个常数乘以$\,\ln\lambda$, %
而它所产生的$\,\lambda\,$的幂函数能被吸收进$\,v_{\lambda}\,$的定义中.

非微扰效应现在已经被分离到了方程(\ref{29.3.15})的最后一项中. 如果$\,C_{1}-C_{2}=1/\nu$, 这一项可以被缠绕数为$\,\nu>0\,$的瞬子产生. %
($C_{1}-C_{2}\,$一般是有理数. 对于推广的量子色动力学超对称版, $C_{1}-C_{2}=N_{c}-N_{f}\,$是整数, 所以$\,C_{1}-C_{2}=1/\nu\,$的条件要求$\,N_{c}=N_{f}-1$, 以及只有$\,\nu=1\,$的瞬子才有贡献. 这一模型中的细致计算\cite{5}表明瞬子确实给出了这样的贡献.) 无论是否是瞬子生成了非微扰贡献$\,v_{\lambda}(\Phi)$, %
我们可以通过考虑理论的无反常对称性定出它的形式. 由于这个函数独立于$\,Y$, 可以设$\,Y=0\,$来计算它, %
这样它就分享了方程(\ref{29.3.1})第一项的所有无反常对称性. 这些对称性中包括规范对称性自身以及一个在$\,\prod_{d}SU(n(d))\,$下的整体对称性, %
其中$\,d\,$标记的是由左手征超场构建的规范群的不同不可约表示, 而$\,n(d)\,$是表示$\,d\,$出现的次数. (例如, 在推广的超对称量子色动力学中, %
$d\,$取两个值, $\,SU(N_{c})\,$的$\,N_{c}\,$和$\,\bar{N}_{c}\,$表示, 并有$\,n(N_{c})=n(\bar{N}_{c})=N_{f}$.) %
我们用$\,\Phi_{ai}^{(d)}\,$标记$\,\Phi\,$中$\,n(d)\,$个不同的在规范群下按照表示$\,d\,$变换的$\,\Phi\,$, 其中$\,a\,$是规范指标, $i\,$是``味''指标.
构建$\,\Phi\,$的函数并使它在整体对称群$\,\prod_{d}SU(n(d))\,$下不变的唯一方式是用$\,\Phi\,$的乘积, 对每个表示$\,d$, %
$\,n_{d}\,$个味指标与反对称$\,SU(n(d))\,$张量$\,\epsilon_{i_{1}\ldots i_{n(d)}}\,$收缩, 而规范指标与规范群的结构张量收缩.
(例如, 对推广的超对称量子色动力学, $v_{\lambda}\,$必须是唯一一个不变量
\[
D\equiv \operatorname{Det}_{ij}\sum_{a}Q_{ai}\overline{Q}_{aj}
\]
的函数, 而这个不变量仅在$\,N_{c}\geq N_{f}\,$时不为零.)

除了无反常的$\,SU(n(d))\,$味对称性, 每个由$\,\Phi\,$构成的不可约表示$\,d\,$还有一个$\,U_{d}(1)\,$对称性, %
给定$\,d\,$的所有$\,\Phi_{ai}^{(d)}\,$在这个对称性进行变换
\begin{equation}
\Phi_{ai}^{(d)} \to \me^{\mi\varphi_{d}}\Phi_{ai}^{(d)} \:. \label{29.3.17}
\end{equation}
这个对称性是有反常的, 而反常的效果和拉格朗日量进行变换
\begin{equation}
    \mathscr{L}\to \mathscr{L}-\sum_{d}\frac{n(d)C_{2d}}{32\uppi^{2}}\sum_{A}\epsilon_{\mu\nu\rho\sigma}f_{A}^{\mu\nu}
    f_{A}^{\rho\sigma}\,\varphi_{d} \label{29.3.18}
\end{equation}
的效果相同, 其中$\,C_{2d}\,$是属于规范群不可约表示$\,d\,$的任何左手征标量超场对$\,C_{2}\,$的贡献. 如果我们赋予$\,T\,$变换性质
\begin{equation}
    T\to T+n(d)C_{2d}\varphi_{d}/\uppi \:, \label{29.3.19}
\end{equation}
这个对称性可以被修复. 由于方程(\ref{29.3.9})中的$\,v_{\lambda}(\Phi)\,$ 伴随着因子$\,\exp(2\mi\uppi T/(C_{1}-C_{2}))$, 而这个因子进行如下的变换
\begin{equation}
    \exp\biggl(\frac{2\mi\uppi T}{C_{1}-C_{2}}\biggr) \to
    \prod_{d}\exp\biggl(\frac{+2\mi n(d)C_{2d}\varphi_{d}}{C_{1}-C_{2}}\biggr)
    \exp\biggl(\frac{2\mi\uppi T}{C_{1}-C_{2}}\biggr) \:, \label{29.3.20}
\end{equation}
我们由此得出: {\kai{对于每个由左手征标量构成的规范群表示$\,d$, $v_{\lambda}(\Phi)\,$必须是$\,\Phi_{ai}^{(d)}\,$的%
阶数为$\,-2n(d)$ $C_{2d}/(C_{1}-C_{2})\,$的齐次函数.}} (例如, 在一般超对称量子色动力学中, 我们有$\,SU(N_{c})\,$的两个不可约表示, %
定义表示和反定义表示, 每个表示有$\,n(d)=N_{f}\,$和$\,C_{2d}=1/2$, 所以$\,v_{\lambda}\,$是属于定义表示的$\,Q\,$和属于反定义表示的$\,\bar{Q}\,$ 的%
$\,-N_{f}/(N_{c}-N_{f})\,$阶齐次函数. 因此它必须正比于$\,D^{-1/(N_{c}-N_{f})}$, 其中$\,D\,$是前面定义的行列式.) 一般情况下, %
$C_{2}=\sum_{d}n(d)C_{2d}$, 所以$\,v_{\lambda}(\Phi)\,$是所有$\,\Phi\,$ 的$\,-2C_{2}/(C_{1}-C_{2})\,$阶齐次函数.

这个结果满足一个重要的自洽性检验. 回忆\,\ref{sec:27.4}\,节, 任何超势的量纲为\,+3\,(设$\,\hbar=c=1$, 按质量幂次计数), %
而标量超场$\,\Phi\,$和普通标量场一样有量纲$\,+1$, 所以$\,v_{\lambda}\,$ 中与$\,\Phi\,$相关的部分, 其系数的量纲是
\[
3+\frac{2C_{2}}{C_{1}-C_{2}}=\frac{3C_{1}-C_{2}}{C_{1}-C_{2}} \:.
\]
这个系数不依赖规范耦合或者超势中其它任何耦合或质量, 并且由于我们将方程(\ref{29.3.11})第二项中的裸耦合$\,g\,$换成了$\,g_{\lambda}$, %
它也不能依赖于用来定义$\,g\,$的紫外截断$\,K$, 所以它只能依赖于$\,\lambda$. 因此
\begin{equation}
    v_{\lambda}(\Phi)=\lambda^{(3C_{1}-C_{2})/(C_{1}-C_{2})}H(\Phi) \:, \label{29.3.21}
\end{equation}
其中$\,H(\Phi)\,$是属于规范群表示$\,d\,$的$\,\Phi\,$的$\,-2n(d)C_{2d}/(C_{1}-C_{2})\,$阶齐次多项式, 它独立于理论的任何参量, %
所以它只能依赖于$\,\lambda$. 我们可以将方程(\ref{29.3.14})重写成如下的形式
\begin{equation}
    \tau_{\lambda}=\mi \frac{3C_{1}-C_{2}}{2\uppi}\,\ln\biggl(\frac{\lambda}{\Lambda}\biggr)
    +\frac{\theta}{2\uppi} \:, \label{29.3.22}
\end{equation}
其中$\,\Lambda\,$是表征跑动规范耦合的能量参量, 就像量子色动力学中的$\,\Lambda\approx 200\,\mathrm{MeV}$. 因此方程(\ref{29.3.15}) 中的最后一项是
\begin{equation}
    \exp\biggl(\frac{2\mi\uppi\tau_{\lambda}}{C_{1}-C_{2}}\biggr)\,v_{\lambda}(\Phi) =
    \exp\biggl(\frac{\mi\theta}{C_{1}-C_{2}}\biggr)\,\Lambda^{(3C_{1}-C_{2})/(C_{1}-C_{2})}\,H(\Phi)\:. \label{29.3.23}
\end{equation}
{\kai{因此, 包含非微扰贡献(\ref{29.3.23})的整个有效超势独立于滑动截断$\,\lambda$.}}


函数$\,H(\Phi)\,$关于$\,\Phi\,$是齐次且是负数阶的, 所以当没有裸超势时, 势在标量场的有限值处是正定的而仅在无限大的场值处为零. 在这种理论中, %
稳定的真空态不存在, 超对称破缺的问题是没有意义的. 通过增加合适的裸超势, 真空态可以被稳定下来. 例如, 在$\,N_{f}<N_{c}\,$的一般超对称量子色动力学中, %
唯一可重整的裸超势是质量项的和
\begin{equation}
    f(Q,\overline{Q}) = \sum_{ija}m_{ij}\overline{Q}_{ai}Q_{aj} \:. \label{29.3.24}
\end{equation}
为了寻找超对称真空态, 我们首先需要找到满足方程(\ref{27.4.11})的所有标量分量$\,q_{ai}\,$和$\,\overline{q}_{ai}$, %
在这里对于$\,SU(N_{c})\,$的所有生成元$\,t_{A}$, 这个方程是
\begin{equation}
    \sum_{abi}q_{ai}^{\ast}(t_{A})_{ab}q_{bi} -
    \sum_{abi}\overline{q}_{ai}^{\ast}(t_{A})_{ab}\overline{q}_{bi} = 0 \:. \label{29.3.25}
\end{equation}
规范相互作用(而非超势)在$\,q_{ai}\,$和$\,\overline{q}_{ai}\,$的色指标$\,a\,$上的联立$\,SU(N_{c})\,$变换下不变; %
且在$\,q_{ai}\,$和$\,\overline{q}_{ai}\,$的味指标$\,i\,$上两个独立的$\,SU(N_{f})\,$和$\,\overline{SU(N_{f})}\,$变换下分别是不变的,
且在将$\,q_{ai}\,$和$\,\overline{q}_{ai}\,$分别变换相反相位的$\,U(1)\,$ 变换下不变. 利用这些对称性, 有可能将这些条件的一般解写成如下的形式
\begin{equation}
    q_{ai} =\overline{q}_{ai} =
     \begin{cases}
    u_{i}\delta_{ai} &\quad a\leq N_{f} \\
    0 &\quad a>N_{f}
    \end{cases}  \:, \label{29.3.26}
\end{equation}
其中$\,u_{i}\,$是相位相同的复数. (下面是证明. $SU(N_{c})\,$生成元$\,t_{A}\,$张开了所有无迹厄米矩阵的空间, 所以方程(\ref{29.3.25})等价于要求
\begin{equation}
    \sum_{i}q_{ai}^{\ast}q_{bi}-\sum_{i}\overline{q}_{ai}^{\ast}\overline{q}_{bi}=k\delta_{ab}\:,\label{29.3.27}
\end{equation}
其中$\,k\,$是某个常数. 通过味和色的组合表换$\,q\to UqV$, 其中$\,U\,$和$\,V\,$是幺正且幺模的, 我们可以将矩阵$\,q\,$变成对角形式(\ref{29.3.26}),
而通过对对角元的幺模相位变换, 我们可以重组它们使得它们有相同的相位. 这样方程(\ref{29.3.27})就变成
\[
    \sum_{i}\overline{q}_{ai}^{\ast}\overline{q}_{bi} =
    \begin{cases}
    \Bigl(u_{a}^{2}-k\Bigr)\delta_{ab} &\quad a\leq N_{f} \\
    -k\delta_{ab} &\quad a>N_{f}
    \end{cases}  \:.
\]
$a>N_{f}\,$的条件表明$\,k\leq 0$. 如果$\,k\,$非零, 那么$\,\overline{q}_{ai}\,$将构成$\,N_{c}\,$个有$\,N_{f}\,$个分量的非零正交矢量,
而这对于$\,N_{f}<N_{c}\,$是不可能的, 所以$\,k=0$. 这样我们就可以通过一系列的幺正味表换将$\,\overline{q}_{ai}\,$变成对角形式(\ref{29.3.26}):
首先将$\,\overline{q}_{1i}\,$转至\,1\,-方向; 然后保持\,1\,-方向不动, 旋转与这个方向垂直的空间将$\,\overline{q}_{2i}\,$转至\,2\,-方向; 以此类推;
然后做一个幺模相位变化使得所有对角元有相同的相位. 这样方程(\ref{29.3.27})就表明$\,q_{ai}\,$和$\,\overline{q}_{ai}\,$的对角元的绝对值相等,
而通过$\,q_{ai}\,$和$\,\overline{q}_{ai}\,$的无反常相反相位变换, 我们可以整理它们的共同相位使得它们相等, 这正是所要证明的.)

方程(\ref{29.3.23})中的函数$\,H\,$在这里是
\begin{equation}
    H(q,\overline{q}) = \mathscr{J}\Bigl[\operatorname{Det}_{ij}\sum_{a}q_{ai}\overline{q}_{aj}\Bigr]^{-1/(N_{c}-N_{f})}
    =\mathscr{J}\Bigl[\prod_{i}u_{i}\Bigr]^{-2/(N_{c}-N_{f})} \:, \label{29.3.28}
\end{equation}
其中$\,\mathscr{J}\,$是纯数值常数. (细致的计算表明对$\,N_{f}=2\,$和$\,N_{c}=3$, $\mathscr{J}=2$.) 加上(\ref{29.3.23})和(\ref{29.3.24}), 完整的有效超势现在是
\begin{equation}
    f_{\text{total}}(q,\overline{q})=\mathscr{K}\Bigl[\prod_{i}u_{i}\Bigr]^{-2/(N_{c}-N_{f})}
    + \sum_{i}m_{i}u_{i}^{2} \:, \label{29.3.29}
\end{equation}
其中
\begin{equation}
    \mathscr{K} \equiv \mathscr{J} \,\exp\biggl(\frac{\mi\theta}{N_{c}-N_{f}}\biggr)\,\Lambda^{(3N_{c}-N_{f})/(N_{c}-N_{f})} \label{29.3.30}
\end{equation}
而$\,m_{i}\,$是当对原始质量矩阵进行将标量变成形如(\ref{29.3.26})的$\,SU(N_{f})\times \overline{SU(N_{f})}\,$变换时, 所给出质量矩阵的对角元.
$f_{\text{total}}(q,\overline{q})\,$应该稳定的条件(\ref{27.4.10})有解
\begin{equation}
    u_{i}^{2}=\frac{1}{m_{i}}\biggl(\frac{\mathscr{K}}{N_{c}-N_{f}}\biggr)^{1-N_{f}/N_{c}}
    \Biggl(\prod_{j}m_{j}\Biggr)^{-1/N_{c}} \:. \label{29.3.31}
\end{equation}
因为我们已经将标量场变到了$\,u_{i}\,$有共同相位的基上, $m_{i}\,$在这个基上也必须有共同相位. 但$\,u_{i}^{2}\,$的共同相位是不唯一的------%
方程(\ref{29.3.31})中的幂次$\,1/N_{c}\,$告诉我们解因为因子$\,\exp(2\mi \uppi n/N_{c})\,$是未定的, %
其中$\,n\,$是一个在$\,0\,$到$\,N_{c}-1\,$之间取值的整数. (对于给定的$\,u_{i}^{2}$, 符号相反的两个$\,u_{i}\,$在物理上是等价的, %
这是因为整个理论在$\,q_{ai}\to \me^{\mi\uppi}q_{ai}\,$和$\,\overline{q}_{ai}\to\me^{-\mi\uppi}\overline{q}_{ai}\,$的无反常对称性是不变的.) %
在下一节我们讨论\,Witten\,指标时, 存在$\,N_{c}\,$个物理上不等价的解这个现象将再次出现.


\subsection{$C_{1}=C_{2}$}


这个情况值得考虑是因为, 正如我们在方程(\ref{27.9.3})中看到的, 当用$\,N=1\,$超场表示时, %
最简单的$\,N=2\,$超对称杨-米尔斯理论包含一个处在伴随表示下的左手征超场, 对其\footnote{注意, 这里的$\,C_{2}\,$是指由左手征超场构成的规范群表示,
所以它与\,\ref{sec:27.9}\,节中的$\,C_{2}^{b}\,$相同但只是$\,C_{2}^{f}\,$的一半, 前者是指复标量构成的表示, %
而后者是包含规范微子在内的所有旋量场构成的表示.}当然有$\,C_{2}=C_{1}$.

当$\,C_{2}=C_{1}\,$时, 函数$\,\exp(2\mi\uppi T)\,$有$\,R=0$, 这使得它在$\,\mathscr{L}^{\sharp}_{\lambda}\,$中的形式不受$\,R\,$不变性的限制. %
$\mathscr{L}_{\lambda}^{\sharp}\,$中$\,F\,$-项的一般形式在这里由方程(\ref{29.3.9})给出, 只不过没有了最后一项:
\begin{equation}
    \mathscr{B}_{\lambda} = Y\,f_{\lambda}(\Phi,\exp(2\mi\uppi T))
    +\sum_{\alpha\beta AB}\epsilon_{\alpha\beta}W_{A\alpha L}\,W_{B\beta L}\,\ell_{\lambda AB}(\Phi,\exp(2\mi\uppi T))\:.    \label{29.3.32}
\end{equation}
因为$\,f_{\lambda}\,$可能依赖于$\,T$, 我们现在还不能得出它等于裸超势的结论, 但可以得到: 对于任何在裸超势中以线性方式出现的耦合系数和质量, %
它将会线性地依赖于这些参量和系数. 特别地, {\kai{如果没有一个超势作为出发点, 那么非微扰效应也不会产生什么.}}


为了更进一步, 我们需要使用在全体$\,\Phi_{n}\,$的$\,U(1)\,$变换下反常的手征对称性. 为了使整个理论在这个对称性不变, 对于裸超势中$\,\Phi_{n}\,$ 的%
$\,r\,$阶项, 我们必须引入一个单独的外左手征超场$\,Y_{r}$. 这样, 这个理论就在组合变换
\begin{equation}
    \Phi_{n} \to \me^{\mi\varphi}\Phi_{n} \:, \qquad T\to T+ C_{2}\varphi/\uppi\:, \qquad
    Y_{r}\to\me^{-\mi r\varphi}Y_{r}  \label{29.3.33}
\end{equation}
下不变. 这个对称性告诉我们: 对于$\,\mathscr{B}_{\lambda}\,$中的一项, 如果它是超势中$\,\Phi\,$的$\,r\,$阶项系数的$\,\mathscr{N}_{r}\,$阶并且正比于%
$\,\exp(2a\mi\uppi T)$, 那么它关于$\,\Phi\,$必是$\,\mathscr{N}_{\Phi}\,$阶的, $\mathscr{N}_{\Phi}\,$给定为
\begin{equation}
    \mathscr{N}_{\Phi} = \sum_{r}r\mathscr{N}_{r}-2C_{2}a\:. \label{29.3.34}
\end{equation}


在方程(\ref{29.3.32})中, $\mathscr{B}_{\lambda}\,$中$\,W\,$的二次项系数$\,\ell_{\lambda AB}\,$在$\,C_{1}=C_{2}\,$时被证明独立于超势中的参量, %
所以方程(\ref{29.3.34})在这一情况下变成
\begin{equation}
    \mathscr{N}_{\Phi} = -2C_{2}a\:. \label{29.3.35}
\end{equation}
因此, $\ell_{\lambda AB}\,$中没有$\,\Phi_{r}\,$的正阶项, 并且$\,\ell_{\lambda AB}\,$中任何独立于$\,\Phi_{r}\,$的项也必须独立于$\,T$. %
所以, $\ell_{\lambda AB}\,$中这些与$\,\Phi\,$无关的项仍然只是对跑动耦合参量$\,\tau_{\lambda}\,$的单圈贡献.

方程(\ref{29.3.32})表明有效超势关于超势中的参量是线性的, 所以它的所有项中只有一个$\,\mathscr{N}_{r}=1\,$而其它的为零. 对于这样的项,
方程(\ref{29.3.34})给出
\begin{equation}
    \mathscr{N}_{\Phi} = r-2C_{2}a\:. \label{29.3.36}
\end{equation}
因此, 有效超势中$\,\Phi\,$的$\,\mathscr{N}_{\Phi}\,$次方项只能来自于裸超势中$\,\Phi\,$的$\,r\geq \mathscr{N}_{\Phi}\,$次方项. %
$r=\mathscr{N}_{\Phi}\,$的项有$\,a=0$, 所以树级近似给出它们就是裸超势. 仅剩的项是$\,r>\mathscr{N}_{\Phi}\,$的非微扰修正. %
这种非微扰项是$\,\Phi\,$的给定阶项, 只能来自于裸超势中$\,\Phi\,$的{\kai{高}}阶项.

\subsection{$C_{1}<C_{2}$}

这时$\,\exp(2\mi\uppi T)\,$的$\,R\,$值是{\kai{负}}的, 所以$\,\exp(2\mi\uppi T)\,$的正幂次可以抵消$\,Y\,$和$\,W_{\alpha}\,$的正$\,R\,$值,
这样$\,\mathscr{B}_{\lambda}$ 就能包含$\,Y\,$和$\,W_{\alpha}\,$的任意阶项. 然而, 利用手征对称性条件(\ref{29.3.34})和$\,R\,$不变性条件
\begin{equation}
    2 = \mathscr{N}_{W} + 2\sum_{r}\mathscr{N}_{r}-2a(C_{2}-C_{1}) \:, \label{29.3.37}
\end{equation}
对于$\,\Phi\,$的给定阶$\,\mathscr{N}_{\Phi}$, 我们可以对这些项的结构施加一些限制. 方程(\ref{29.3.34})和(\ref{29.3.37})有一个平庸解, %
这个解对$\,r=\mathscr{N}_{\Phi}\,$有$\,\mathscr{N}_{r}=1$, 对$\,r\,$的其它值有$\,\mathscr{N}_{r}=0$, 以及$\,a=\mathscr{N}_{W}=0$; %
这个解就表示$\,\mathscr{B}_{\lambda}\,$出现在原始裸超势中而没有任何辐射修正. 如果没有超势作为出发点, 那么对所有$\,r\,$有$\,\mathscr{N}_{r}=0$,
所以方程(\ref{29.3.35})将不会允许威尔逊型拉格朗日量中有任何$\,\mathscr{N}_{W}=0\,$的项, {\kai{因此无法生成超势}}.
(在量子色动力学的这个超对称版中, 导出这个结论的通常方法是通过注意到超势中的项没有可能与所有对称性相容, 但正如我们看到的, 这个结论其实要更加普遍.) %
对于可重整的渐进自由理论, 在$\,\mathscr{B}_{\lambda}\,$的$\,\Phi\,$-无关项上有一个有用的限制. 可重整性条件告诉我们在$\,r>3\,$时$\,\mathscr{N}_{r}=0$, %
所以, 通过从方程(\ref{29.3.37})中减掉\,2/3\,倍的方程(\ref{29.3.34}), 我们发现
\begin{equation}
    2\geq \frac{2}{3}\mathscr{N}_{\Phi}+\mathscr{N}_{W} + 2a\biggl(C_{1}-\frac{1}{3}C_{2}\biggr)\:. \label{29.3.38}
\end{equation}
渐进自由要求$\,3C_{1}>C_{2}$, 所以当$\,\mathscr{N}_{\Phi}=0\,$(或$\,\mathscr{N}_{\Phi}>0$)时, 右边的每一项都是正的. %
我们由此得出$\,\Phi\,$-无关项不能超过$\,W\,$的二阶, 而这些项又独立于$\,T$, 所以它们再次表示对跑动耦合参量$\,\tau_{\lambda}\,$的单圈贡献. %
当在$\,C_{1}<C_{2}\,$时, 没有什么禁止条件阻止$\,W\,$的二阶或高阶以及$\,\Phi\,$的负数阶项出现.

\section{规范理论的超对称破缺} \label{sec:29.4}

我们现在转向物理上极度感兴趣的问题: 超对称性会在哪种规范理论中自发破缺?

我们从规范理论出发, 即\,\ref{sec:27.5}\,节中描述的超对称版量子电动力学. 这是有两个手征超场$\,\Phi_{\pm}\,$的 $U(1)\,$规范理论, %
其中$\,\Phi_{\pm}\,$携带量子数$\,\pm e$, 而超势$\,f(\Phi)=m\Phi_{+}\Phi_{-}$. 我们在\,\ref{sec:27.5}\,节看到, %
如果我们在拉格朗日密度中引入\,Fayet-Iliopoulos\,项$\,\xi[V]_{D}$, 超对称就会在树级近似下破缺, 所以\,Witten\,指标在$\,\xi\neq0\,$且$\,e\,$很小时为零, %
因而也就对于$\,e\,$和$\,\xi\,$的包括$\,\xi=0\,$的所有值都为零. 超对称在$\,\xi=0\,$处是不破缺的吗? 它在树级近似下不破缺, %
但我们如何分辨高阶修正或非微扰效应在这一情况下是否赋予了真空有限的能量?

为了回答这个问题, 我们将使用这个理论在$\,\xi=0\,$时的一个对称性原理, 利用在\,\ref{sec:29.1}\,节概述的方法. 这里的对称性是电荷守恒: %
整个拉格朗日量在手征和规范超场的电荷共轭变换下是不变的:
\begin{equation}
    \mathsf{C}\Phi_{\pm}\mathsf{C}^{-1} = \Phi_{\mp} \:, \qquad
    \mathsf{C}V\mathsf{C}^{-1} = -V \:. \label{29.4.1}
\end{equation}
在树级近似下有无质量费米子, 光微子, 但如果我们取真空在$\,\mathsf{C}\,$ 下为偶, 那么单光微子态在$\,\mathsf{C}\,$下为奇, 所以这些态不是通过乘以超对称生成元彼此关联. 这里的手征费米子在树级近似下有质量$\,m$, 所以在$\,e\,$很小时, 它也不与真空配对. 没有了可以配对的费米态, %
真空态的能量至少在$\,e\,$足够小时必须严格为零, 以使得微扰论给频谱一个好的定性图像. 正如我们在\,\ref{sec:29.1}\,节中看到的, %
真空态能量为零意味着超对称没破缺. 同理, 由于没有可以和光微子配对的玻色态, 所以它的质量必须严格为零.%

现在, 如果$\,e\,$的值非常大以至于微扰论的结果完全无法信任, 这时改怎么办? 因为\,Witten\,指标为零, 所以它本身在这里是没有帮助的. %
取而代之, 我们来考虑加权\,Witten\,指标, $\operatorname{Tr}\mathsf{C}(-1)^{F}$. 我们已经看到在$\,e\,$很小且$\,\xi=0\,$时, 真空能为零, %
而且还有两个包含一个自旋上或下的零动量光微子的零能量态, 以及一个包含两个自旋相反的零动量光微子的玻色零能量态. %
真空对$\,\operatorname{Tr}\mathsf{C}(-1)^{F}\,$的贡献是$\,+1$; 两个单光微子态的贡献是$\,+2\,$(因为$\,\mathsf{C}\,$和$\,(-1)^{F}\,$都是\,-1); %
而双光微子态的贡献是$\,+1$, 这给出了加权\,Witten\,指标$\,\operatorname{Tr}\mathsf{C}(-1)^{F}=4$. 这与$\,e\,$的值无关, 所以即使对于强耦合, %
加权\,Witten\,指标也是\,4, 所以超对称没有破缺.

这里有这样一个问题.\cite{2} 在树级近似下对零动量态计数时, 我们没有考虑规范场$\,V_{\mu}(x)\,$的零动量分量. $V_{0}(x)\,$中的常数项不会引起问题,
这是因为它可以被规范变换
\begin{equation}
    V_{\mu}(x) \to V_{\mu}(x)+\partial_{\mu}\Lambda(x)  \label{29.4.2}
\end{equation}
移除, 其中规范参量$\,\Lambda(x)\,$正比于$\,x^{0}$. 另一方面, 我们无法简单地移除$\,V_{i}(x)\,$中的常数项, %
这是因为这将会要求$\,\Lambda(x)\,$正比于$\,x^{i}\,$的规范变换, 而这样的变换将与场平移盒子维度$\,L\equiv V^{1/3}\,$之后不变的周期性假定矛盾. %
在这里考虑的特殊模型中, 所有场的电荷是$\,\pm e\,$或零, 所以, 如果我们将自身限制在点阵规范变换
\begin{equation}
    \Lambda(x) = \frac{2\uppi}{eL}\sum_{i}\ell_{i}x^{i} \:, \label{29.4.3}
\end{equation}
其中$\,\ell_{i}\,$是三个正或负整数, 这个周期性将得以保留. 因此, 尽管我们不能通过规范变换移除$\,V_{i}(x)$ 的零动量分量, %
我们可以自由地将它们偏移$\,2\uppi \ell_{i}/eL$. 对$\,V_{\mu}\,$中与$\,x_{i}\,$无关的部分, %
树级近似拉格朗日量({\kai{不是}}拉格朗日密度)就是$-\frac{1}{2}L^{3}\sum_{i}(\partial_{0}V_{i})^{2}$, %
所以哈密顿量就是$\,+\frac{1}{2}L^{-3}\sum_{i}(\pi_{i})^{2}$, 其中$\,\pi\,$正则共轭于$\,V_{i}$\,: $\pi_{i}=L^{3}\partial_{0}V_{i}$.
这样, 这个场的波函数$\,\Psi(\mathbf{V})\,$就像三维空间中处在线性尺寸为$\,2\uppi/eL\,$且有周期性边界条件的盒子中的一个单位质量粒子. %
能量确定为$\,\mathbf{k}^{2}/2L^{3}\,$的波函数正比于$\,\exp(\mi\mathbf{k} \cdot \mathbf{V})$, 其中$\,k_{i}=eL\ell_{i}$, $\ell_{i}\,$是整数. %
这个场有一个唯一的零能量态, 它有$\,k_{i}=0\,$且它的归一化波函数等于常数$\,(eL/2\uppi)^{3/2}\,$. 因为这个态是唯一的, %
所以我们对零能量态的计数不受规范自由度的影响, 而加权\,Witten\,指标确实是\,4.

现在我们来考虑没有手征超场的一个非阿贝尔单纯规范超场的理论. Witten\,在\,1982\,年关于 Witten\,指标的论文\cite{2}中给出了一个论证指出, Witten\,指标对于这样的理论是$\,r{+}1\,$(或可能是$\,-r-1$), 其中$\,r\,$是规范群的秩, 即对易生成元的最大数目. 在\,1997\,年, 他发现了对这个计算的一个修正,\cite{6} 其结果是, 对于典型幺正, 正交以及辛群, 指标是\,Casimir\,不变量$\,C_{1}$, 对于幺正群和辛群, 指标确实是$\,r+1$, 但对于$\,N>7\,$的正交群$\,O(N)\,$以及例外群, 这个指标取不同的值. 一般而言, Casimir\,不变量由方程(\textcolor{foo}{17.5.33})定义:
\begin{equation}
    \sum_{CD} C_{ACD}C_{BCD} = g^{2}C_{1}\delta_{AB} \:, \label{29.4.4}
\end{equation}
其中$\,g\,$是耦合常数, 当把生成元限制到在\,23.5\,节被用来计算瞬子效应的``标准''$\,SU(2)\,$子代数的三个生成元上时, %
通过指定结构常数就是$\,g\epsilon_{ABC}$, 可以明确$\,g\,$的定义. 对于典型群, 我们有
\begin{equation}
    C_{1} = \begin{cases}
    \quad N & \quad SU(N) \\
    N-2 & \quad SO(N)\quad \text{当}\: N>3 \\
    N+1 & \quad USp(2N)
    \end{cases} \:. \label{29.4.5}
\end{equation}
对$\,SU(N)\,$秩是$\,r=N-1$, 而对$\,USp(2N)\,$则是$\,r=N$, 所以在这两种情况下均有$\,r+1=C_{1}$. 但对$\,N>6\,$的$\,SO(N)$, %
在$\,N\,$为奇时有秩$\,r=(N-1)/2=(C_{1}+1)/2$, 在$\,N\,$为偶时有秩$\,r=N/2=(C_{1}+2)/2$, 且指标是$\,C_{1}\,$而不是$\,r+1$. 当然, 这不影响主要结论: %
由于\,Witten\,指标不为零, 超对称不是自发破缺的. Kac\,和\,Smilga\,在\,1999\,年证明了这个指标对例外群也等于$\,C_{1}$.\cite{7} 在这里, %
我们将仅对没有手征超场的$\,SU(N)\,$和$\,USp(2N)\,$超对称规范场论计算\,Witten\,指标, 但在计算的过程中, 我们同样会看到为什么正交群和例外群会有特殊的困难.


这里还有另一个困难, 而之前有带荷手征超场的阿贝尔理论的例子中也不会遇到这个困难. 正如通常的构造, 没有手征超场的规范玻色子和规范微子的一般可重整理论根本不会包含任何无量纲参量, 这使得无法调整耦合参量使得耦合变弱. 取而代之, 我们有一个跑动耦合, 它依赖于能量与一个特征能标$\,\Lambda\,$之比, %
例如\,18,7\,节讨论过的量子色动力学标度$\,\Lambda\approx 200\,\mathrm{MeV}$. 对于非阿贝尔规范理论, 物理耦合在高能时有多弱, %
规范耦合常数在能量低于$\,\Lambda\,$时变强, 但这里我们是在一个体积为$\,L^{3}\,$的箱子中进行处理, 这在能量$\,\approx 1/L\,$处提供了一个红外截断, 而规范对称性未破缺的理论中一般没有这样的截断. 当我们讨论一个弱耦合规范理论时, 应该被理解成这个理论在能量高于$\,1/L\,$时都是弱耦合的. 在这种情况下, 为了能够对无限体积的现实极限做出任何结论, 显然要有能力论证超对称对于强耦合和弱耦合都是破缺的.

我们将在$\,V_{A}^{0}=0\,$的瞬时规范下进行处理. 这样拉格朗日密度(\ref{27.3.1})就是
\begin{equation}
    \mathscr{L} = -\tfrac{1}{4}\sum_{Aij}f^{2}_{Aij} - \tfrac{1}{2}\sum_{A}(\partial_{0}V_{Ai})^{2}
    -\tfrac{1}{2}\sum_{A}\Bigl(\bar{\lambda}_{A}(\slashed{D}\lambda)_{A}\Bigr)
    +\tfrac{1}{2}\sum_{A}D_{A}^{2} \:, \label{29.4.6}
\end{equation}
其中, 在瞬时规范下,
\begin{align}
    f_{Aij} &= \partial_{i}V_{Aj} - \partial_{j}V_{Ai} + \sum_{BC}C_{ABC}V_{Bi}V_{Cj} \:, \label{29.4.7} \\
    (D_{i}\lambda)_{A} &= \partial_{i}\lambda_{A} + \sum_{BC}C_{ABC}V_{Bi}\lambda_{C} \:, \label{29.4.8} \\
    (D_{0}\lambda)_{A} &= \partial_{0}\lambda_{A} \:. \label{29.4.9}
\end{align}
(像往常一样, 规范耦合常数被吸收进结构常数$\,C_{ABC}\,$中.) 没有了手征超场, 就不存在对辅助场$\,D_{A}$ 的其它依赖; 由于它以二次型方式进入, %
所以可以令它等于使拉格朗日量稳定的值, 即$\,D_{A}=0$, 然后从此忽略它. 仅保留$\,\mathbf{x}\,$-相关的模, 有效拉格朗日量变成
\begin{align}
    \int \dif^{3}x\: \mathscr{L} &= L^{3}\Biggl[-\frac{1}{4}\sum_{Aij}\Bigl(\sum_{BC}C_{ABC}V_{Bi}V_{Cj}\Bigr)^{2}
    -\frac{1}{2}\sum_{Ai}(\partial_{0}V_{Ai})^{2} \nonumber \\
    &\quad -\frac{1}{2}\sum_{ABCi} C_{ABC}\Bigl(\bar{\lambda}_{A}\gamma_{i}V_{Bi}\lambda_{C}\Bigr)
    -\frac{1}{2}\sum_{A}\bar{\lambda}_{A}\gamma^{0}\partial_{0}\lambda_{A} \Biggr] \:. \label{29.4.10}
\end{align}
那么哈密顿量是
\begin{equation}
    H = \frac{1}{2L^{3}}\sum_{Ai}\pi_{Ai}^{2}+\frac{L^{3}}{4}\sum_{Aij}\Bigl(\sum_{BC}C_{ABC}V_{Bi}V_{Cj}\Bigr)^{2}
    +\frac{L^{3}}{2}\sum_{ABCi}C_{ABC}\Bigl(\bar{\lambda}_{A}\gamma_{i}V_{Bi}\lambda_{C}\Bigr) \:, \label{29.4.11}
\end{equation}
其中$\,\pi_{Ai}=L^{3}\partial_{0}V_{Ai}\,$是$\,V_{Ai}\,$的正则共轭.

在树级近似下, 能量为零的规范场构形是那些使得对于所有$\,A,i\,$和$\,j\,$都有$\,\sum_{BC}C_{ABC}V_{Bi}V_{Cj}$ $=0\,$的场构形. %
如果$\,V_{Bi}\,$在$\,t_{B}\,$不是规范\,Lie\,代数的\,Cartan\,子代数时对于所有$\,i\,$都为零, 这个条件总是满足的.\footnote{{\it{Cartan}\,\kai{子代数}}是任意由$\,r\,$个相互对易的独立生成元张成的子代数, 即, 使得$\,C_{A\mathscr{AB}}\,$在$\,t_{\mathscr{A}}\,$和$\,t_{\mathscr{B}}$都属于\,Cartan\,子代数时对所有%
$\,t_{A}\,$都为零, 其中$\,r\,$是秩, 即这种生成元的最大个数. 例如, 对于\,19.7\,节中讨论的强相互作用的$\,SU(3)\,$对称性, 秩是$\,r=2$, %
而\,Cartan\,子代数可以取为由同位旋第三分量$\,t_{3}\,$和超荷(hypercharge)$\,t_{8}\,$构成的子代数, 它们作为方程(\textcolor{foo}{19.7.2})中%
被记为$\,\lambda_{3}\,$和$\,\lambda_{8}\,$的两个对易矩阵作用在轻夸克上.} 对幺正群和辛群以及它们的直积, 这是满足这个条件的唯一方法. %
对于$\,N\leq6\,$的正交规范代数$\,O(N)$, 由于它们全部等价于辛\,Lie\,群和(或)幺正\,Lie\,群(参看第\,15\,章附录), 这也同样成立, %
但对于$\,N\geq 7\,$的正交规范代数$\,O(N)\,$则不是如此, 这是为什么\,Witten\,的原始计算对这种情况需要修正的原因.

在这个计算的剩余部分, 我们将仅考虑像辛代数和幺正代数这样的规范代数, %
即使得树级近似下零能量条件确实要求代数的$\,V_{Bi}\,$在$\,t_{B}\,$不是规范\,Lie\,代数的\,Cartan\,子代数时对于所有$\,i\,$都为零. %
当除了$\,t_{\mathscr{A}}\,$处在\,Cartan\,子代数的$\,V_{\mathscr{A}i}\,$以外有$\,V_{Ai}=0$时, %
费米子场的零能量模就是那些$\,t_{\mathscr{A}}\,$处在 Cartan\,子代数的$\,\lambda_{\mathscr{A}}$以外的$\,\lambda_{A}=0\,$的模. %
%除了那些$\,t_{\mathscr{A}}\,$处在\,Cartan\,子代数的$\,\lambda_{\mathscr{A}}$.

现在我们必须对这些态计数. 在半单\,Lie\,代数的任何表示下, $t_{\mathscr{A}}\,$的本征值是量子化的, 所以, %
非零规范场$\,V_{\mathscr{A}i}\,$的所有值通过规范变换都等价于在一个有周期性边界条件的有限大箱子中的值. %
这些模的量子化就像上面对$\,U(1)\,$模中的规范场做的量子化, 所以这些场的零能量态依旧是唯一的, 而波函数依旧是箱中的常函数.

在树级近似下, 零能量态的数目完全来自于费米子自由度. 这里使用二分量记法比较方便, 在这个符号约定下, %
取代对\,Cartan\,子代数的每个生成元有一个\,4\,-分量\,Majorana\,场, 我们现在有两个左手场$\,\lambda_{\mathscr{A}La}$, 其中$\,a=\pm1/2,$ %
以及它们的厄米共轭$\,\lambda_{\mathscr{A}La}^{\ast}$. 这些规范微子场满足正则反对易关系
\[
\Bigl\{\lambda_{\mathscr{A}\,L\,a},\lambda_{\mathscr{B}\,L\,b}^{\ast}\Bigr\}
=\delta_{\mathscr{A}\mathscr{B}}\delta_{ab}
\]
和
\[
\Bigl\{\lambda_{\mathscr{A}\,L\,a},\lambda_{\mathscr{B}\,L\,b}\Bigr\} =
\Bigl\{\lambda_{\mathscr{A}\,L\,a}^{\ast},\lambda_{\mathscr{B}\,L\,b}^{\ast}\Bigr\} =0 \:.
\]
按照情况用任意多个$\,\lambda_{\mathscr{A}La}\,$因子作用在任意态矢上, 我们可以构建出被所有$\,\lambda_{\mathscr{A}La}\,$湮灭的态矢$\,\lvert 0\rangle$. %
一般的零能量态矢就是$\,\lambda_{\mathscr{A}La}^{\ast}\,$的乘积作用在$\,\lvert 0\rangle\,$后的线性组合.

为了看到这些态中那些态可以通过超对称生成元彼此配对, 我们必须要把理论的对称性考虑在内. 对原始规范群中由那些保持这个子代数不变的元素$\,h\,$构成的子群, %
即群元$\,h\,$使得$\,h^{-1}t_{\mathscr{A}}h\,$是 $t_{\mathscr{B}}\,$的线性组合, 零能量条件, %
即规范场和规范微子仅处在与\,Cartan\,子代数相对应的方向上这个条件, 在这个群下不变. 它们构成一个有限群, 称为\,{\it{Weyl}\,\kai{群}}. %
例如, 在$\,SU(N)\,$的定义表示中, Weyl\,群由对$\,N\,$个坐标轴的置换和乘上使得变换幺模的相位构成. 它们可以表示成交换第\,$i\,$个和第$\,j\,$个坐标轴的有限规范变换的乘积$\,W(i,j)=\exp(\mi\uppi\sigma(ij)/2)=\mi\sigma(ij)$, 其中$\,i\neq j$, %
而$\,\sigma(ij)\,$是$\,[\sigma(ij)]^{i}{}_{j}=[\sigma(ij)]^{j}{}_{i}=1\,$而其它元素都为零的$\,U(N)\,$生成元. %
这些变换诱导了在$\,t_{\mathscr{A}}\,$张成的空间中的正交变换, 其中$\,t_{\mathscr{A}}\,$是在伴随表示下生成\,Cartan\,子代数的无迹厄米矩阵. %
例如, 对于$\,SU(2)\,$群, 我们可以令\,Cartan\,子代数仅由$\,t_{3}\,$构成, %
这样\,Weyl\,群就由单位元和一个非平庸规范变换$\,W(1,2)=\mi\sigma(1,2)=\mi\sigma_{1}\,$构成, 使得$\,W^{-1}t_{3}W=-t_{3}$. %
对$\,SU(3)$, Cartan\,子代数有两个生成元$\,\lambda_{3}\,$和$\,\lambda_{8}$, 而\,Weyl\,由\,6\,个规范变换组成: 1, $W(1,2)$, $W(2,3)$, $W(1,3)$, %
$W(1,2)W(2,3)\,$和$\,W(2,3)W(1,2)$, 它们在$\,t_{3}\,$和$\,t_{8}\,$张成的空间中生成了$\,60^{\circ}\,$倍角的旋转.

假定真空在\,Weyl\,群下不变, 那么能够通过超对称生成元的作用能与其配对的态也在\,Weyl\,群下不变.\footnote{因为真空和超对称生成元的\,Weyl\,不变性%
意味着只有\,Weyl\,-不变态才与超对称性自发破缺有关, Witten\,评述道: 物理态必然是\,Weyl\,-不变的, 但我们在这里不会进入细节.} %
这可能也可能不包含先前构造的零能量态$\,\lvert 0\rangle$. 态$\,\lvert 0\rangle\,$被所有$\,\lambda_{\mathscr{A}La}\,$湮灭的条件显然在 Weyl\,群下不变, %
所以如果它是唯一的, 那么它必然构成了\,Weyl\,群的一维表示. Weyl\,群总是通过正交变换作用在\,Cartan\,子代数的生成元上, 所以有两个可能性: 不变表示,
在这个表示下, 每个\,Weyl\,变换的表示是单位元, 以及赝不变表示, 在这个表示下, 每个\,Weyl\,变换的表示是它在 Cartan\,子代数生成元上的作用的行列式.

我们首先来考虑$\,\lvert 0\rangle\,$在\,Weyl\,群下不变的情况. 单费米子态$\,\lambda_{\mathscr{A}La}^{\ast}\lvert0\rangle\,$的任何线性组合在\,Weyl\,群下显然不可能是不变的. 双费米子态的\,Weyl\,-不变线性组合只有一个; 它的形式是$\,U\lvert 0\rangle$, 其中
\begin{equation}
    U \equiv \sum_{ab\mathscr{A}} e_{ab}\,\lambda_{\mathscr{A}\,L\,a}^{\ast}\,
    \lambda_{\mathscr{A}\,L\,b}^{\ast} \:. \label{29.4.12}
\end{equation}
(因为反对易关系使得乘积$\,\lambda_{\mathscr{A}\,L\,a}^{\ast}\,\lambda_{\mathscr{A}\,L\,b}^{\ast}\,$关于自旋指标$\,a\,$和$\,b\,$反对称, %
所以指标$\,a\,$和$\,b\,$与方程(\ref{25.2.9}) 定义的反对称张量$\,e_{ab}\,$收缩.) 对于三个或多个\,Cartan\,子代数生成元的乘积, %
Weyl\,-不变的线性组合有多种, 但$\,\lambda^{\ast}_{\mathscr{A}La}\,$的反对易性使得除了$\,U\,$的幂次以外的线性组合都为零.\footnote{为了对$\,SU(N)\,$看到这点, 注意到它的\,Cartan\,子代数的每个生成元$\,t_{\mathscr{A}}\,$都可以写成$\,U(N)\,$Cartan\,子代数的生成元$\,T_{i}\,$的线性组合$\,t_{\mathscr{A}}=\sum_{i}c_{\mathscr{A}i}T_{i}$, 其中每个$\,T_{i}\,$的非零元只有$\,(T_{i})^{i}{}_{i}=1\,$(指标没有求和). %
为了使$\,t_{\mathscr{A}}\,$无迹, 我们必须有$\,\sum_{i}c_{\mathscr{A}i}=0$. 由于场$\,\lambda_{\mathscr{A}La}^{\ast}\,$在\,Weyl\,群下像%
$\,t_{\mathscr{A}}\,$那样变换, 由此得出为了使函数%
$\,\sum_{\mathscr{A}\mathscr{B}\cdots}d_{\mathscr{AB}\cdots}\lambda_{\mathscr{A}La}^{\ast}\lambda_{\mathscr{B}Lb}^{\ast}\cdots\,$是\,Weyl\,不变的,
系数$\,d_{\mathscr{AB}\cdots}\,$必须采取如下的形式
\[
d_{\mathscr{AB}\cdots}=\sum_{ij\cdots}c_{\mathscr{A}i}c_{\mathscr{B}j}\cdots D_{ij\cdots}\:,
\]
其中$\,D_{ij\cdots}\,$是不变张量, 即对于任何矢量$\,u_{i}$, $v_{j}\,$等, 函数$\,D(u,v,\cdots)\equiv\sum_{ij\cdots}D_{ij\cdots}u_{i}v_{j}\cdots\,$在%
坐标轴的置换下不变. 这种函数的最一般形式是函数
\[
S(x,y,z,\cdots)=\sum_{i}x_{i}y_{i}z_{i}\cdots
\]
的乘积的线性组合, 其中变量$\,x,y,z\cdots\,$取为$\,u,v\,$等的各种子集. 但因为$\,\sum_{i}c_{\mathscr{A}i}=0$, 在我们的情况下, 每个矢量的分量之和为零,
所以$\,D(u)=S(u)=0$; $D(u,v)\,$正比于$\,S(u,v)$; $D(u,v,w)\,$正比于$\,S(u,v,w)$; $D(u,v,w,x)\,$是$\,S(u,v,w,x)$, $S(u,v)S(w,x)$, %
$S(u,w)S(v,x)\,$和$\,S(u,x)S(v,w)\,$的线性组合; 以此类推. 关键点在于, 即使函数$\,D(u,v,\cdots)\,$关于它的变量可能不是对称的(因为$\,S\,$的不同乘积可能会伴随着不同的系数), 函数$\,S(u,v,\cdots)\,${\kai{是}}对称的. 在我们的情况下, %
矢量是反对易量$\,u(a)_{i}=\sum_{\mathscr{A}}\lambda_{\mathscr{A}La}^{\ast}c_{\mathscr{A}i}$, 这使得唯一非零的$\,S\,$函数是$\,S(u(1/2),u(-1/2)).$ %
在生成元合适的归一化下, 这就是算符$\,U$.
} 另外, 如果乘积中$\,U\,$的个数超过$\,r$, 某些$\,\lambda_{\mathscr{A}La}^{\ast}\,$就必然会出现两次, 所以$\,U^{r+1}=0$. 我们由此得出\,Weyl\,不变的态被限制在如下的$\,r+1\,$个态:
\begin{equation}
    \lvert 0 \rangle \:, \: U\lvert 0\rangle \:, \: U^{2}\lvert 0\rangle\:,\cdots, U^{r}\lvert 0\rangle \:.
\end{equation}
它们全都是玻色态, 并且在树级近似下没有能与它们配对的零能量费米态, 所以在耦合足够弱时, 这些态的能量必须精确为零, 且超对称是不破缺的. %
另外这里\,Weyl\,-不变的\,Witten\,指标是$\,r+1$, 而它独立于耦合强度, 所以{\kai{无论规范耦合的强度是多少, 超对称性不是自发破缺的.}}

在态$\,\lvert 0\rangle\,$是赝不变的情况下, 唯一的\,Weyl\,不变态的形式是
\begin{equation}
    \sum_{\mathscr{A},\mathscr{B},\cdots}\epsilon_{\mathscr{A},\mathscr{B},\cdots}
    \lambda_{\mathscr{A}\,L,a}^{\ast}\,\lambda_{\mathscr{B}\,L\,b}^{\ast}\cdots \lvert 0\rangle \:, \label{29.4.14}
\end{equation}
其中$\,\epsilon_{\mathscr{A},\mathscr{B},\cdots}\,$是秩为$\,r\,$的全反对称张量. 上式关于$\,a,b,\cdots$是完全对称的, %
所以表征每个态的是指标$\,+1/2\,$的个数而非$\,-1/2$, 而这个数字的取值范围是$\,0\,$到$\,r$, 所以独立态的个数是$\,r+1$. 取决于$\,r\,$是奇还是偶, %
这些态要么都是费米的要么都是玻色的, 所以这时的\,Witten\,指标是$\,\pm(r+1)$, 而超对称依旧是不破缺的.


这里获得的\,Witten\,指标与特定的整体对称性如何破缺的想法有一个有趣的关系. 超对称规范理论的拉格朗日量在整体$\,U(1)\,$``$R$-对称性''的变换下不变, %
这个变换对规范微子的左手部分和右手部分的作用是改变相反的相位:
\begin{equation}
    \lambda_{A\,L}\to \me^{\mi\varphi}\lambda_{A\,L} \:, \qquad
    \lambda_{A\,R}\to \me^{-\mi\varphi}\lambda_{A\,R} \:, \label{29.4.15}
\end{equation}
其中$\,\varphi\,$是任意的实常相位. 与这个对称性对应的流$\,J^{\mu}_{5}\,$的守恒被如下的反常破坏:
\begin{equation}
    \partial_{\mu}J^{\mu}_{5} = -\frac{1}{32\uppi^{2}}\sum_{ABCD}C_{ACD}\,C_{BCD}\,
    f_{A}^{\mu\nu}\,f_{B}^{\rho\sigma}\,\epsilon_{\mu\nu\rho\sigma} \:, \label{29.4.16}
\end{equation}
其中$\,\epsilon_{\mu\nu\rho\sigma}\,$像往常一样是$\,\epsilon^{0123}\equiv 1\,$的全反对称量. (由于规范微子处在规范群的伴随表示下, %
这是从方程(\textcolor{foo}{22.2.24}) 中通过把规范群生成元$\,t_{A}\,$取为$\,(t_{A})_{BC}=-\mi C_{ABC}\,$获得的, 又因为规范微子没有可区分的反粒子, %
所以要给反常乘以$\,1/2$.) 有了上面指定的规范耦合定义, 方程(\textcolor{foo}{23.5.20})给出了对反常中场强乘积的积分
\begin{equation}
    \epsilon_{\mu\nu\rho\sigma}\int \dif^{4}x\: \sum_{A}f_{A}^{\mu\nu}f_{A}^{\rho\sigma}=64\uppi^{2}\nu/g^{2}\:,    \label{29.4.17}
\end{equation}
其中``缠绕数''$\,\nu\,$是表征规范场所属拓扑类的整数. 将方程(\ref{29.4.4}), (\ref{29.4.16})和(\ref{29.4.17})放在一起, %
我们看到缠绕数为$\,\nu\,$的瞬子在$\,R\equiv \int\dif^{3}x\:J_{5}^{0}\,$中引入了一个变分
\begin{equation}
    \Delta R = \int \dif^{4}x\: \partial_{\mu}J^{\mu}_{5}=-2\nu C_{1} \:. \label{29.4.18}
\end{equation}
即, 有效作用量包含$\,(\sum_{Aab}\lambda_{ALa}\lambda_{ALb}e_{ab})^{C_{1}}\,$和它的整数幂次, 这些项不在$\,R\,$-对称变换(\ref{29.4.15})下不变, %
而是在相位$\,\exp(2\mi\varphi C_{1})\,$的整数幂次变换下不变. 因此瞬子破坏了一般$\,U(1)\,$$R\,$-变换(\ref{29.4.15}) 下的不变性, %
将其约化至$\,\varphi\,$是$\,\uppi/C_{1}\,$整数倍的变换(\ref{29.4.15})构成的$\,Z_{2C_{1}}\,$群. 我们可以预期规范耦合在低能处的增长将导致(就像量子色动力学中那样)规范微子双线性型有真空期望值, 而这意味着离散对称群$\,Z_{2C_{1}}\,$自发破缺到它的$\,Z_{2}\,$子群, 这个子群由规范微子的符号变换生成. %
这样就会有$\,C_{1}\,$个零能量态$\,\lvert n\rangle$, 这些态由$\,Z_{2C_{1}}\,$的元素$\,\exp(\mi n\uppi R/C_{1})\,$作用在任何一个真空态%
$\,\lvert X\rangle\,$生成, 其中, 因为我们把任何一对仅相差$\,Z_{2}\,$生成元$\,\exp(\mi\uppi R)$\,的态视为等价的, 所以$\,n\,$仅在$\,0,1,\cdots,C_{1}-1\,$中取值. 由于$\,\lvert X\rangle\,$只能是$\,R\,$为偶数值的态的线性组合, 利用$\,R\,$的所有值$\,\mathscr{R}=0,2,\cdots,2C_{1}-2$, 通过取线性组合
\[
\sum_{n}\exp\biggl(\frac{\mi n\uppi\mathscr{R}}{C_{1}}\biggr) \,\lvert n \rangle \:,
\]
我们可以构造$\,C_{1}\,$个态. 特别地, 对$\,SU(N)\,$和$\,USp(2N)$, 这些态与计算\,Witten\,指标时遇到的\,Weyl\,-不变态相同, 在计算\,Witten\,指标时, $\lvert0\rangle$是$\,R\,$-不变且\,Weyl\,不变的但不是赝实的, 方程(\ref{29.4.12})定义的算符$\,U\,$有$\,R=2$, 所以有$\,r+1\,=C_{1}$个态$\,U^{n}\lvert 0\rangle$, 它们的$\,R=2n\,$在零到$\,2r=2C_{1}-2\,$之间取值. 更一般地, 出现$\,C_{1}\,$个统计相同的零能量态帮助解释了为什么\,Witten\,指标对于例外群和正交群以及幺正群和辛群等于$\,\pm C_{1}$.

Witten\,指标对于所有纯规范超对称理论都不为零这个现象告诉我们, 为了得到一个超对称性自发破缺的例子, 我们必须给理论加上手征超场. %
由于在弱耦合时引入有质量场并不会改变零质量态的选单, 给理论加上{\kai{有质量}}手征超场并没有什么帮助. 我们已经看到这样的一个例子: %
\ref{sec:29.3}\,节回顾的\,Affleck, Dine\,和\,Seiberg\,的工作表明, 有$\,N_{f}\,$个左手征超场$\,Q_{ai}\,$以及相同数目左手征超场%
$\,\overline{Q}_{ai}\,$的$\,SU(N_{c})\,$规范理论, 其中$\,N_{f}<N_{c}$, $Q_{ai}\,$处在$\,SU(N_{c})\,$的定义表示中%
而$\,\overline{Q}_{ai}\,$则处在复共轭表示, 并有质量项$\,\sum_{aij}m_{ij}\overline{Q}_{ai}Q_{aj}$, %
这样的理论有$\,N_{c}\,$个零能量玻色态但没有零能量费米态. (在这个情况下, 当质量项为零时, Witten\,指标并不会保持不变, %
这是因为这个质量项是超势中超场的高阶项, 使得质量为零会改变超势在大场处的行为. 事实上, 方程(\ref{29.3.31})表明, 随着质量趋于零, %
零能量态中的标量场值将趋于无穷.)

另一方面, 不难找到一个有左手征超场的理论使得它们满足一个保持它们无质量的对称性, 在这种情况下, 超对称{\kai{是}}动力学破缺的. 例如, %
考虑有$\,N_{f}\,$个左手征超场$\,Q_{ai}\,$和$\,N_{f}\,$个左手征超场$\,\overline{Q}_{ai}\,$的$\,SU(N_{c})\,$规范理论, %
其中这两个左手征超场依旧处在规范群的$\,N_{c}\,$和$\,\overline{N}_{c}\,$表示中, 但现在还有$\,N_{f}\,$个在规范群%
$\,SU(N_{c})\,$下为中性的左手征超场$\,L_{i}$.  假定有一个整体(或弱耦合定域)$\,SU(N_{f})\,$对称性, 它作用在$\,Q\,$和$\,L\,$的``味''指标$\,i\,$上, %
但不作用在$\,\overline{Q}\,$上, 这将禁止连接$\,Q\,$和$\,\overline{Q}\,$的质量项. 我们将$\,Q_{ai}\,$和$\,L_{i}\,$分别取为$\,SU(N_{f})\,$的%
$\,N_{f}\,$和$\,\overline{N}_{f}\,$表示下, 将$\,\overline{Q}\,$取为$\,SU(N_{f})\,$单态. 这样, 唯一的可重整超场是如下的形式
\begin{equation}
    f(Q,\overline{Q},L) = \sum_{ija}\mathscr{G}_{j}\overline{Q}_{aj}Q_{ai}L_{i} \:, \label{29.4.19}
\end{equation}
其中$\,\mathscr{G}_{j}\,$是一组耦合常数, 通过一个$\,SU(N_{f})\,$旋转可以使其只有一个不为零的分量, 例如$\,j=N_{f}\,$的那个分量, %
而这个分量也可以被选为正的. (这是\,Affleck, Dine\,和\,Seiberg\,处理的模型的一个推广,\cite{8} 在他们处理的模型中, $N_{c}=3\,$和$\,N_{f}=2$.) %
规范中性超场$\,L_{i}\,$对有效超势的非微扰项没有影响, 所以我们可以使用\,\ref{sec:29.3}\,节的结果, 即在$\,N_{c}>N_{f}\,$时, %
规范相互作用会产生一个总的有效超势
\begin{equation}
    f_{\text{total}}(Q,\overline{Q},L) = \sum_{ija}\mathscr{G}_{j}\overline{Q}_{aj}Q_{ai}L_{i}
    +\mathscr{K}\Bigl[\operatorname{Det}_{ij}\sum_{a}\overline{Q}_{aj}Q_{ai}\Bigr]^{-1/(N_{c}-N_{f})}\:, \label{29.4.20}
\end{equation}
其中$\,\mathscr{K}\,$是一个常数 为了使超对称不破缺, 必要(但不充分)条件是手征超场的标量分量$\,q_{aj}$, $\overline{q}_{aj}\,$和 $\ell_{i}\,$%
要满足条件$\,\partial f_{\text{total}}(q,\overline{q},f)/\partial\ell_{i}=0$, 这使得, 对所有$\,i$,
\begin{equation}
    \sum_{ja} \mathscr{G}_{j}\,\overline{q}_{aj}\,q_{ai} =0 \:. \label{29.4.21}
\end{equation}
但这告诉我们矩阵$\,\sum_{a}\overline{q}_{aj}q_{ai}\,$有一个本征值为零, 因此行列式为零, 所以这是超势(\ref{29.4.20})的一个奇点, 在这个奇点上, %
$\partial f_{\text{total}}/\partial q_{ai}\,$或$\partial f_{\text{total}}/\partial \overline{q}_{aj}\,$不可能为零. 因此这类模型中的超对称是破缺的.

例如, Affleck, Dine\,和\,Seiberg\,在他们$\,N_{c}=3\,$和$\,N_{f}=2\,$的模型中发现, 超场$\,\Phi_{ai}$, $\overline{\Phi}_{ai}$, $L_{i}\,$%
的不为零标量分量$\,q_{aj},\overline{q}_{aj}\,$和$\,\ell_{i}\,$在势能的最小值点取如下的值
\begin{align}
    q_{11} &= \overline{q}_{11} = 1.286\,(\mathscr{K}/2\mathscr{G})^{1/7} \:, \nonumber \\
    q_{22} &= \overline{q}_{22} = 1.249\,(\mathscr{K}/2\mathscr{G})^{1/7} \:, \label{29.4.22} \\
    \ell_{1} &= \sqrt{q_{11}^{2}-q_{22}^{2}} \:, \nonumber
\end{align}
其中真空能密度是
\begin{equation}
    \rho_{\text{VAC}} = 3.593\,(16\mathscr{G}^{10}/\mathscr{K}^{4})^{1/7} \:. \label{29.4.23}
\end{equation}
另外, 在$\,\theta=0\,$时$\mathscr{K}=2\Lambda^{7}$. 

超对称性在这个情况下自发破缺的现象激励我们去思考它可能在更大的一类渐进自由规范理论中被强规范力破缺, %
因此导致了对\,\ref{sec:28.3}\,节中关于超对称破缺的推测的合理性.

\section[Seiberg-Witten\,解]{Seiberg-Witten\,解\footnote{本节有些脱离本书的发展主线, 可以在第一次阅读时跳过.}} \label{sec:29.5}


超对称理论中的树级近似势通常会在标量场值的一个连续区间内取零. (可参看方程(\ref{29.1.9}) 作为一个例子.) 在这个情况下, %
这个理论在树级近似下有数个质量为零的标量激发, 又由于超对称没有破缺, 这些激发必须伴随着合适的费米激发. 在能量很低时, %
这个理论将由一族超对称有效拉格朗日量描述, 这组有效拉格朗日量的元素将有一个或多个{\kai{模}}(moduli)来参数化, 即底层理论的标量激发值. %
底层理论中的量子效应会修正有效拉格朗日量对这些模的依赖, 甚至改变模空间的拓扑.\cite{9}

在\,20\,世纪\,90\,年代, 关于超对称理论的一个显著成就是, 对$\,N=2\,$超对称的规范理论, %
Seiberg 和\,Witten\cite{10}能够计算低能有效拉格朗日量对模参量的精确依赖关系. 通过快速浏览最简单的特殊情况, %
即$\,N=2\,$超对称且没有额外物质极多重态的$\,SU(2)\,$规范理论, 可以使得这个计算背后的想法变得显然.

我们在\,\ref{sec:27.9}\,节看到, 在消掉辅助场后, 这个理论的拉格朗日密度是
\begin{align}
    \mathscr{L} &= \frac{1}{e^{2}} \Biggl[ -\sum_{A}(D_{\mu}\phi)_{A}^{\ast}(D^{\mu}\phi)_{A}
    -\frac{1}{2}\sum_{A}\Bigl(\overline{\psi_{A}}(\slashed{D}\psi)_{A}\Bigr) \nonumber \\
    &\quad -2\sqrt{2}\operatorname{Re}\sum_{ABC}\epsilon_{ABC}\,
    \Bigl(\lambda_{AL}^{\mathrm{T}}\,\epsilon\,\psi_{CL}\Bigr)\, \phi_{B}^{\ast}
    -\frac{1}{4}\sum_{A}f_{A\mu\nu}f_{A}^{\mu\nu} \nonumber \\
    &\quad -\frac{1}{2}\sum_{A}\Bigl(\overline{\lambda_{A}}(\slashed{D}\lambda)_{A}\Bigr)\Biggr]
    +\frac{\theta}{64\uppi^{2}}\epsilon_{\mu\nu\rho\sigma}\sum_{A}f_{A}^{\mu\nu}f_{A}^{\rho\sigma}
    -V(\phi,\phi^{\ast}) \:, \label{29.5.1}
\end{align}
其中$\,A$, $B\,$和$\,C\,$现在在$\,1,2,3\,$中取值. 我们现在给每个场乘以一个因子$\,e\,$来重新标度所有场, 使得$\,e\,$不会出现在协变导数中:
\begin{equation}
    (D_{\mu}\psi)_{A}=\partial_{\mu}\psi_{A}+\sum_{BC}\epsilon_{ABC}V_{B\mu}\psi_{C} \:, \label{29.5.2}
\end{equation}
\begin{equation}
    (D_{\mu}\lambda)_{A}=\partial_{\mu}\lambda_{A}+\sum_{BC}\epsilon_{ABC}V_{B\mu}\lambda_{C} \:, \label{29.5.3}
\end{equation}
\begin{equation}
    (D_{\mu}\phi)_{A}=\partial_{\mu}\phi_{A}+\sum_{BC}\epsilon_{ABC}V_{B\mu}\phi_{C} \:, \label{29.5.4}
\end{equation}
\begin{equation}
    f_{A\mu\nu}=\partial_{\mu}V_{A\nu}-\partial_{\nu}V_{A\mu}+\sum_{BC}\epsilon_{ABC}V_{B\mu}V_{C\nu}\:, \label{29.5.5}
\end{equation}
而势是
\begin{equation}
    V(\phi,\phi^{\ast}) = 2\sum_{A}\Biggl[\sum_{BC}\epsilon_{ABC}\operatorname{Re}\phi_{B}
    \operatorname{Im}\phi_{C} \Biggr]^{2} \:. \label{29.5.6}
\end{equation}
这个势对一族标量场激发值为零, 这组激发值(在相差一个规范变换的意义下)可以参数化为
\begin{equation}
    \phi_{1}=\phi_{2}=0\:, \qquad \phi_{3}=a \:, \label{29.5.7}
\end{equation}
其中$\,a\,$是一个复参量, 被称为真空模. 这个真空期望值赋予矢量场$\,V_{1\mu}\,$和$\,V_{2\mu}\,$, 规范微子$\,\lambda_{1}\,$和$\,\lambda_{2}$, 
手征费米子$\,\psi_{1}\,$和$\,\psi_{2}$, 以及标量$\,\phi_{1}\,$和$\,\phi_{2}\,$以质量$\,2\lvert a\rvert$, 保持$\,V_{3\mu}$, $\lambda_{3}$, 
$\psi_{3}\,$和$\,\phi_{3}=a\,$是无质量的.

仅考虑这些无质量模(并扔掉下标\,3), 树级近似给出的有效低能理论就是一个自由场理论, 其拉格朗日密度是
\begin{align}
    \mathscr{L}_{\text{eff}} &= \frac{1}{e^{2}}\Biggl[-(\partial_{\mu}a)^{\ast}\,(\partial^{\mu}a)
    -\frac{1}{2}\Bigl(\overline{\psi}(\slashed{\partial}\psi)\Bigr) \nonumber \\
    &\quad -\frac{1}{4}f_{\mu\nu}f^{\mu\nu}-\frac{1}{2}\Bigl(\overline{\lambda}(\slashed{D}\lambda)\Bigr)\Biggr]
    +\frac{\theta}{64\uppi^{2}}\epsilon_{\mu\nu\rho\sigma}f^{\mu\nu}f^{\rho\sigma} \:, \label{29.5.8}
\end{align}
其中$\,f_{\mu\nu}=\partial_{\mu}V_{\nu}-\partial_{\nu}V_{\mu}$. 事实也的确如此, 对单个规范玻色子和它的规范中性$\,N=2\,$超对称伴的任何可重整理论, %
因为$\,N=2\,$超对称性不允许这个理论有势, 这种理论必须是一个自由场理论.


但这不是整个故事. 在积掉这个理论的有质量自由度时, 量子修正产生了低能有效场论中的不可重整相互作用项. 通过使用我们在\,19.5\,节用来处理低能$\,\pi\,$介子和核子的能量幂次计数方法, 
我们可以对在低能时起主导作用的相互作用项和包含它们的\,Feynman\,图进行分类. 当我们使用有效拉格朗日量微扰式的计算低能散射振幅, 有$\,L\,$个圈, $I_{f}\,$条费米子外线, $I_{b}\,$条玻色子($a\,$或$\,V_{\mu}$)内线, $I_{a}\,$条辅助场内线, 以及$\,V_{i}\,$个$\,i\,$类顶点的连通图对能量幂次$\,\nu\,$的贡献是
\begin{equation}
    \nu = 4L + \sum_{i}V_{i}d_{i}-2I_{b}-I_{f} \:, \label{29.5.9}
\end{equation}
其中$\,d_{i}\,$是$\,i\,$类相互作用中的导数个数. (因为辅助场内线与动量无关, 所以它们对方程(\ref{29.5.9})没有贡献.) 这些量服从拓扑关系
\begin{equation}
    L = I_{b}+I_{f}+I_{a}-\sum_{i}V_{i}+1 \:, \label{29.5.10}
\end{equation}
以及
\begin{equation}
    2I_{b}+E_{b}=\sum_{i}V_{i}b_{i} \:,\qquad 2I_{f}+E_{f}=\sum_{i}V_{i}f_{i}\:, \qquad
    2I_{a}+E_{a}=\sum_{i}V_{i}a_{i} \:, \label{29.5.11}
\end{equation}
其中$\,E_{b}$, $E_{f}\,$和$\,E_{a}\,$是玻色子, 费米子以及辅助场外线的个数, $b_{i}$, $f_{i}\,$和$\,a_{i}\,$是$\,i\,$类相互作用中玻色子, 费米子以及辅助场的个数. 我们因此可以将能量幂次写为
\begin{equation}
    \nu = \sum_{i}V_{i}\,(d_{i}+\tfrac{1}{2}f_{i}+a_{i}-2)+2L-E_{f}-2E_{a}+2\:. \label{29.5.12}
\end{equation}
根据方程(\ref{26.8.4})和(\ref{27.4.42}), $N=1\,$左手征标量超场和它们共轭的函数的$\,D$-项以及一对$\,N=1\,$规范超场$\,W_{\alpha}\,$与$\,N=1\,$左手征标量超场的任意函数之积的$\,\mathscr{F}\,$-项均有$\,d_{i}+\tfrac{1}{2}f_{i}+a_{i}=2$, 而引入额外的$\,W_{\alpha}\,$因子或者超导数$\,\mathscr{D}_{\alpha}\,$将给出$\,d_{i}+\tfrac{1}{2}f_{i}+a_{i}>2$, 所以在我们的情况中, 超对称排除了任何$\,d_{i}+\tfrac{1}{2}f_{i}+a_{i}\,$小于\,2\,的相互作用. 因此, 低能散射振幅的主导贡献就由树级近似下($L=0$), 用只含$\,d_{i}+\tfrac{1}{2}f_{i}+a_{i}=2\,$的项的有效拉格朗日量计算给出, 这中拉格朗日量采取\,\ref{sec:27.4}\,节讨论中讨论的一般形式:
\begin{equation}
    \mathscr{L}_{\text{eff}}=\frac{1}{2}\Bigl[K(\Phi,\Phi^{\ast})\Bigr]_{D}
    -\frac{1}{2}\operatorname{Re}\Bigl[T(\Phi)\Bigl(W_{L}^{\mathrm{T}}\epsilon W_{L}\Bigr)\Bigr]_{\mathscr{F}} \:.
    \label{29.5.13}
\end{equation}
这样方程(\ref{27.4.42})就给出了用分量场表示的拉格朗日量
\begin{align}
    \mathscr{L} &= \frac{\partial^{2}K(a,a^{\ast})}{\partial a\,\partial a^{\ast}}
    \Biggl[-\frac{1}{2}\Bigl(\overline{\psi}\,\slashed{\partial}\psi\Bigr) + \lvert \mathscr{F}\rvert^{2}
    -\partial_{\mu}a\,\partial^{\mu}a^{\ast} \Biggr] \nonumber \\
    &\quad -\operatorname{Re}\Biggl\{\frac{\partial^{3}K(a,a^{\ast})}{\partial^{2}a\,\partial a^{\ast}}
    \Bigl(\overline{\psi}\psi_{L}\Bigr)\mathscr{F}^{\ast}\Biggr\}
    -\frac{1}{2}\operatorname{Re}\Biggl\{\frac{\partial^{3}K(a,a^{\ast})}{\partial^{2}a\,\partial a^{\ast}}
    \Bigl(\overline{\psi}\gamma^{\mu}\gamma_{5}\psi\Bigr)\,\partial_{\mu}a \Biggr\} \nonumber \\
    &\quad +\frac{1}{4}\frac{\partial^{4}K(a,a^{\ast})}{\partial^{2}a\,\partial^{2}a^{\ast}}\,
    \Bigl(\overline{\psi}\psi_{L}\Bigr)\,\Bigl(\overline{\psi}\psi_{R}\Bigr) \nonumber \\
    &\quad +\frac{1}{4}\operatorname{Re}\Biggl\{ \Bigl(\overline{\lambda}\lambda_{L}\Bigr)\,
    \Bigl(\overline{\psi}\psi_{L}\Bigr)\,\frac{\dif^{2}T(a)}{\dif a^{2}} \Biggr\}
    -\frac{1}{2}\operatorname{Re}\biggl\{\Bigl(\overline{\lambda}\lambda_{L}\Bigr)\mathscr{F}
    \frac{\dif T(a)}{\dif a}\biggr\} \nonumber \\
    &\quad +\operatorname{Re}\Biggl\{ T(a)\biggl[-\frac{1}{2}
    \Bigl(\overline{\lambda}\,\slashed{\partial}(1-\gamma_{5})\lambda\Bigr) -\frac{1}{4}f_{\mu\nu}f^{\mu\nu} \nonumber \\
    &\quad\phantom{+\operatorname{Re}\Biggl\{ T(a)\biggl[}
    +\frac{1}{8}\mi\,\epsilon_{\mu\nu\rho\sigma}f^{\mu\nu}f^{\rho\sigma} +\frac{1}{2}D^{2}\biggr]\Biggr\} \nonumber \\
    &\quad +\frac{\sqrt{2}}{4}\operatorname{Re}\biggl\{\frac{\dif T(a)}{\dif a}
    \Bigl[-\Bigl(\overline{\lambda}\gamma^{\mu}\gamma^{\nu}\psi_{L}\Bigr)f_{\mu\nu}
    +2\mi \Bigl(\overline{\lambda}\psi_{L}\Bigr)D\Bigr]\biggr\} \:. \label{29.5.14}
\end{align}


为了实现$\,N=2\,$超对称性, 我们现在希望附加在离散$\,R\,$-对称变换(\ref{27.9.2})
\begin{equation}
    \psi\to\lambda \:, \qquad \lambda \to -\psi \:, \label{29.5.15}
\end{equation}
而$\,a\,$和$\,V_{\mu}\,$保持不变下的不变性. $(\overline{\psi}\slashed{\partial}\psi)\,$和$\,(\overline{\lambda}\slashed{\partial}\lambda)\,$的系数应该相等的条件是
\begin{equation}
    \frac{\partial^{2}K(a,a^{\ast})}{\partial a\,\partial a^{\ast}} = \operatorname{Re}T(a)\:. \label{29.5.16}
\end{equation}
右边是一个$\,a\,$的函数与一个$\,a^{\ast}\,$的函数之和, 所以$\,\partial^{4}K/\partial^{2}a\,\partial^{2}a^{\ast}=0$, 因此$\,\psi\,$的四次项由于没有$\,\lambda\,$配对物将不会出现. 通过分部积分, $\tfrac{1}{2}\operatorname{Re}\{T(a)(\overline{\lambda}\slashed{\partial}\gamma_{5}\lambda)\}\,$可以被替换为$\,-\tfrac{1}{4}\operatorname{Re}\{(\overline{\lambda}\gamma^{\mu}\gamma_{5}\lambda)\partial_{\mu}T(a)\}$. 因此, 如果
\[
    \frac{1}{4}\partial_{\mu}T = \frac{1}{2}\frac{\partial^{3}K}{\partial^{2}a\,\partial a^{\ast}}\partial_{\mu}a\:,
\]
$(\overline{\psi}\slashed{\partial}\psi)\,$和$\,(\overline{\lambda}\slashed{\partial}\lambda)\,$的系数相等的条件将被满足, %
而这又是方程(\ref{29.5.16})的一个自动结果. 根据方程 (\ref{26.A.7}), 正比于$\,f_{\mu\nu}(\overline{\lambda}[\gamma^{\mu},\gamma^{\nu}]\psi)\,$%
或$\,f_{\mu\nu}(\overline{\lambda}[\gamma^{\mu},\gamma^{\nu}]\gamma_{5}\psi)$的项, 以及正比于$\,(\overline{\lambda}\lambda_{L})(\overline{\psi}\psi_{L})\,$或其共轭的项在变换(\ref{29.5.15})下自动是不变的. 另一方面, 正比于$\,(\overline{\lambda}\psi)D\,$以及正比于$\,(\overline{\lambda}\gamma_{5}\psi)D\,$的项的不变性要求我们要扩展变换(\ref{29.5.15})使其包含
\begin{equation}
    D\to -D \:, \label{29.5.17}
\end{equation}
其又保持$\,\tfrac{1}{2}D^{2}\,$不变. 最后, 方程(\ref{29.5.16})告诉我们$\,(\overline{\psi}\psi_{L})\mathscr{F}^{\ast}\,$和$\,(\overline{\lambda}\lambda)\mathscr{F}\,$的系数相等, 所以要扩展变换(\ref{29.5.15})使其包含
\begin{equation}
    \mathscr{F} \to \mathscr{F}^{\ast} \:, \label{29.5.18}
\end{equation}
这又保持$\,\lvert\mathscr{F}\rvert^{2}\,$不变. 我们由此得出, 条件(\ref{29.5.16})确保了从方程(\ref{29.5.14})导出的整个作用量在组合变换(\ref{29.5.15}), (\ref{29.5.17})以及(\ref{29.5.18})下不变. 由于这个拉格朗日量在超多重态是一个左手征超多重态$\,(a,\psi,\mathscr{F})\,$和一个规范超多重态$\,(V_{\mu},\lambda,D)\,$的$\,N=1\,$超对称性下不变, 它在超多重态是一个左手征超多重态$\,(a,\lambda,\mathscr{F}^{\ast})\,$和一个规范超多重态$\,(V_{\mu},-\psi,-D)\,$的$\,N=1\,$超对称性下也是不变的. 因此方程(\ref{29.5.16})足以保证(无需附加场方程)从方程(\ref{29.5.13})或(\ref{29.5.14})获得的作用量在$\,N=2$ 超对称性下是不变的.

方程(\ref{29.5.16})的一般解可以表示成\footnote{引入因子$\,1/4\uppi\mi\,$是为了简化下面要引入的对偶变换. Seiberg\,和\,Witten\,引入了一个称为{\kai{预势}}(prepotential)的函数$\,\mathscr{F}(a)\,$(与辅助场$\,\mathscr{F}\,$无关), 使得$\,h(a)=\dif\mathscr{F}(a)/\dif a$.}
\begin{equation}
    T(a)\to\frac{1}{4\uppi\mi}\frac{\dif h(a)}{a} \:, \qquad
    K(a,a^{\ast}) = \operatorname{Im}\biggl\{\frac{a^{\ast}h(a)}{4\uppi}\biggr\} \:, \label{29.5.19}
\end{equation}
其中$\,h\,$仅是$\,a\,$的函数. 用$\,h\,$表示, 拉格朗日密度(\ref{29.5.14})现在写成
\begin{align}
    \mathscr{L} &= \frac{1}{4\uppi}\operatorname{Im}\Biggl\{ \biggl[\frac{\dif h}{\dif a}\biggr]\,
    \biggl[-\frac{1}{2}\Bigl(\overline{\psi}\,\slashed{\partial}(1-\gamma_{5})\psi\Bigr)
    -\frac{1}{2}\Bigl(\overline{\lambda}\,\slashed{\partial}(1-\gamma_{5})\lambda\Bigr) \nonumber \\
    &\quad -\partial_{\mu}a\,\partial^{\mu}a^{\ast}+\lvert\mathscr{F}\rvert^{2}+\frac{1}{2}D^{2}
    -\frac{1}{4}f_{\mu\nu}f^{\mu\nu}+\frac{1}{8}\mi\,
    \epsilon_{\mu\nu\rho\sigma}f^{\mu\nu}f^{\rho\sigma}\biggr]\Biggr\} \nonumber \\
    &\quad +\frac{1}{4\uppi}\operatorname{Im}\Biggl\{ \biggl[\frac{\dif^{2}h}{\dif a^{2}}\biggr]
    \biggl[-\frac{1}{2}\Bigl(\overline{\psi}\psi_{L}\Bigr)\mathscr{F}^{\ast}
    -\frac{1}{2}\Bigl(\overline{\lambda}\lambda_{L}\Bigr)\mathscr{F}
    +\frac{\sqrt{2}}{2}\,\mi\,\Bigl(\overline{\lambda}\psi_{L}\Bigr) D \nonumber \\
    &\quad \phantom{+\frac{1}{4\uppi}\operatorname{Im}} -\frac{\sqrt{2}}{4}
    \Bigl(\overline{\lambda}\gamma^{\mu}\gamma^{\nu}u\psi_{L}\Bigr)f_{\mu\nu}\biggr]\Biggr\} \nonumber \\
    &\quad +\frac{1}{16\uppi}\operatorname{Im}\Biggl\{\frac{\dif^{3}h(a)}{\dif a^{3}}
    \Bigl(\overline{\lambda}\lambda_{L}\Bigr)\Bigl(\overline{\psi}\psi_{L}\Bigr) \Biggr\} \:. \label{29.5.20}
\end{align}
这在一个$\,SU(2)$ $R\,$-对称性下不变, 在这个对称性下, $(\psi,\lambda)\,$按照双重态变换而$\,(\operatorname{Im}\mathscr{F},\operatorname{Re}\mathscr{F},D/\sqrt{2})$ 按照三重态变换. 离散变换(\ref{29.5.15}), (\ref{29.5.17})和(\ref{29.5.18})是这个$\,SU(2)\,$群的有限元: 绕着\,2\,-轴以弧度$\,\uppi\,$旋转.

比较方程(\ref{29.5.8})和(\ref{29.5.20}), 我们看到在树级近似下
\begin{equation}
    h(a)_{\text{tree}} = \biggl[\frac{4\uppi\mi}{e^{2}}+\frac{\theta}{2\uppi}\biggr]\,a \:. \label{29.5.21}
\end{equation}
Seiberg\,和\,Witten\,的成就就在于精确计算$\,h(a)$.

这个计算的第一步是意识到, 存在数个$\,a\,$和$\,\dif h(a)/\dif a\,$的线性组合使得给出的理论在物理上等价. 这是因为低能有效理论有一个与\,\ref{sec:27.9}\,节简要讨论过的对偶性质相关的一个显著性质. 为了说明这个性质, 我们回到用$\,N=1\,$超场表示的拉格朗日密度(\ref{29.5.13}), 现在使用$\,N=2\,$超对性要求的关系(\ref{29.5.19})
\begin{equation}
    \mathscr{L}_{\text{eff}} = \frac{1}{8\uppi}\operatorname{Im}\Bigl[ \Phi^{\ast}h(\Phi)\Bigr]_{D}
    -\frac{1}{8\uppi}\operatorname{Im}\Bigl[h^{\prime}(\Phi)\Bigl(W_{L}^{\mathrm{T}}\epsilon W_{L}\Bigr)
    \Bigr]_{\mathscr{F}} \:. \label{29.5.22}
\end{equation}
在路径积分中, 旋量场强超场$\,W_{L}\,$受到齐次\,Maxwell\,方程的超对称推广的约束:
\begin{equation}
    \operatorname{Re}\Bigl(\mathscr{D}_{L}^{\mathrm{T}}\epsilon W_{L}\Bigr)=0 \:. \label{29.5.23}
\end{equation}
附加这个约束的惯用方法是要求$\,W_{L}\,$采取(\ref{27.2.15})的形式
\begin{equation}
    W_{L} =\frac{\mi}{4} \Bigl(\mathscr{D}_{R}^{\mathrm{T}}\epsilon\mathscr{D}_{R}\Bigr)\mathscr{D}_{L}V \:,
    \label{29.5.24}
\end{equation}
其中$\,V\,$是一个不受约束的实超场. 取而代之, 我们可以通过在作用量中引入一个拉格朗日乘子项
\begin{equation}
    \Delta I_{\text{eff}} =\frac{1}{8\uppi}\operatorname{Re}\int \dif^{4}x\:
    \Bigl[ \tilde{V}\Bigl(\mathscr{D}_{L}^{\mathrm{T}}\epsilon W_{L}\Bigr)\Bigr]_{D} \label{29.5.25}
\end{equation}
来实现条件(\ref{29.5.23}), 其中$\,\tilde{V}\,$是不受约束的实超场. (数值因子$\,1/8\uppi\,$用来固定$\,\tilde{V}\,$的归一化, 我们会在后面看到这样做的方便之处.) 这样在路径积分中, 除了$\,\tilde{V}\,$是实超场以及$\,W_{L}\,$是满足左手征条件$\,\mathscr{D}_{R\alpha}W_{L\beta}=0\,$的左手旋量超场外, 我们可以全无约束地积掉$\,\tilde{V}\,$和$\,W_{L}$. 超空间中的分部积分使得我们可以将作用量中的新项写为
\begin{equation}
    \Delta I_{\text{eff}} =-\frac{1}{8\uppi}\operatorname{Re}\int \dif^{4}x\:
    \Big[\Bigl( (\mathscr{D}_{L}\tilde{V})^{\mathrm{T}}\epsilon W_{L}\Bigr)\Bigr]_{D} \:, \label{29.5.26}
\end{equation}
或者, 使用方程(\ref{26.3.31})以及$\,W_{L}\,$是左手征的这个性质,
\begin{equation}
    \Delta I_{\text{eff}} = \operatorname{Re}\biggl[ \frac{\mi}{4\uppi}\int\dif^{4}x\:
    \Bigl[\Bigl(\tilde{W}_{L}^{\mathrm{T}}\epsilon W_{L}\Bigr)\Bigr]_{\mathscr{F}}\biggr] \:, \label{29.5.27}
\end{equation}
其中$\,\tilde{W}_{L}\,$是用$\,\tilde{V}_{L}\,$定义的, 方式与前面用$\,V\,$给出$\,W_{L}\,$的关系相同:
\begin{equation}
    \tilde{W}_{L} =\frac{\mi}{4} \Bigl(\mathscr{D}_{R}^{\mathrm{T}}\epsilon\mathscr{D}_{R}\Bigr)
    \mathscr{D}_{L} \tilde{V} \:. \label{29.5.28}
\end{equation}
但现在我们要积掉没有这种约束的$\,W_{L}$, 而它所在的作用量关于它是二次的
\begin{align}
    I_{\text{eff}}+\Delta I_{\text{eff}}&= \operatorname{Im}\int \dif^{4}x\:
    \biggl[ -\frac{1}{8\uppi}\,h^{\prime}(\Phi)\Bigl(W_{L}^{\mathrm{T}}\epsilon W_{L}\Bigr)
    -\frac{1}{4\uppi}\Bigl(\tilde{W}_{L}^{\mathrm{T}}\epsilon W_{L}\Bigr)\biggr]_{\mathscr{F}} \nonumber \\
    &\quad +\frac{1}{8\uppi}\operatorname{Im}\int \dif^{4}x\: \Bigl[\Phi^{\ast}h(\Phi)\Bigr]_{D} \:. \label{29.5.29}
\end{align}
做掉这个积分的方法是令$\,W_{L}\,$等于作用量关于$\,W_{L}\,$稳定的值:
\begin{equation}
    W_{L}=-\frac{\tilde{W}_{L}}{h^{\prime}(\Phi)} \:, \label{29.5.30}
\end{equation}
这样整个有效作用量变成
\begin{equation}
    \tilde{I}_{\text{eff}} = +\frac{1}{8\uppi}\operatorname{Im}\int \dif^{4}x\:
    \biggl[\frac{1}{h^{\prime}(\Phi)}\,\Bigl(\tilde{W}_{L}^{\mathrm{T}}\epsilon \tilde{W}_{L}\Bigr)\biggr]_{\mathscr{F}}
    +\frac{1}{8\uppi}\operatorname{Im}\int \dif^{4}x\: \Bigl[\Phi^{\ast}h(\Phi)\Bigr]_{D} \:. \label{29.5.31}
\end{equation}
现在, 如果我们定义一个新的左手征标量超场和一个新的$\,h\,$函数
\begin{equation}
    \tilde{\Phi}\equiv h(\Phi) \:, \qquad \tilde{h}(\tilde{\Phi})\equiv -\Phi \:, \label{29.5.32}
\end{equation}
那么
\begin{equation}
    \frac{\dif \tilde{h}}{\dif \tilde{\Phi}}\frac{\dif h}{\dif \Phi}
    =-\frac{\dif \Phi}{\dif \tilde{\Phi}}\frac{\dif \tilde{\Phi}}{\dif \Phi}=-1 \:, \label{29.5.33}
\end{equation}
所以作用量(\ref{29.5.31})可以写为
\begin{equation}
    \tilde{I}_{\text{eff}}=-\frac{1}{8\uppi}\operatorname{Im}\int \dif^{4}x\:
    \Bigl[\tilde{h}^{\prime}(\tilde{\Phi})\,\Bigl(\tilde{W}_{L}^{\mathrm{T}}\epsilon \tilde{W}_{L}\Bigr)\Bigr]_{\mathscr{F}}
    +\frac{1}{8\uppi}\operatorname{Im}\int\dif^{4}x\:\Bigl[\tilde{\Phi}^{\ast}\tilde{h}(\tilde{\Phi})\Bigr]_{D}\:.
    \label{29.5.34}
\end{equation}
从我们推导的方法中, 我们知道基于这个有效作用量的理论等价于原始的有效场论, 所以{\kai{标量场值为$\,a\,$以及$\,h\,$函数为$\,h(a)\,$的$\,N=2\,$有效场论在物理上等价于标量场值}}\footnote{下标$\,D\,$代表``对偶(dual)'', 它与超场的$\,D\,$-项无关.}{\kai{$\,a_{D}\equiv h(a)\,$以及$\,h\,$函数 $\tilde{h}(a_{D})\equiv -a\,$的$\,N=2\,$有效场论}}. 这是对偶性在当前框架下的版本.


函数$\,h(\Phi)\,$的另一个变换也会产生等效的拉格朗日量(现在无需变换$\,\Phi\,$或$\,W$), 这个变换可以与变换(\ref{29.5.32})相结合以给出一个更大的对偶变换群. %
假定我们对$\,h(\Phi)\,$偏移一个实系数线性项:
\begin{equation}
    h(\Phi) \to h(\Phi) + b\,\Phi \:, \label{29.5.35}
\end{equation}
其中$\,b\,$是一个实常数. 那么有效拉格朗日密度(\ref{29.5.22})就被偏移了$\,b\operatorname{Im}\{[\Phi^{\ast}\Phi]_{D}\}/8\uppi$, 
由于$\,\Phi^{\ast}\Phi\,$的 $\,D\,$-项是使得, 所以这个偏移项为零. 有效拉格朗日密度(\ref{29.5.22})中的变换因此由第二项中的偏移给出
\begin{equation}
    \mathscr{L}_{\text{eff}} \to \mathscr{L}_{\text{eff}} - \frac{b}{8\uppi}
    \operatorname{Im}\Bigl[\Bigl(W_{L}^{\mathrm{T}}\epsilon W_{L}\Bigr)\Bigr]_{\mathscr{F}} \:, \label{29.5.36}
\end{equation}
或者, 根据方程(\ref{27.2.13})
\begin{equation}
    \mathscr{L}_{\text{eff}} \to \mathscr{L}_{\text{eff}} - \frac{b}{8\uppi}
    \biggl[\mi\,\Bigl(\overline{\lambda}\,\slashed{\partial}\gamma_{5}\lambda\Bigr)
    +\frac{1}{4}\epsilon^{\mu\nu\rho\sigma}f_{\mu\nu}f_{\rho\sigma}\biggr] \:. \label{29.5.37}
\end{equation}
右边括号中的第一项是时空导数, 因此不会影响有效作用量. 括号中的第二项也可以写成一个时空导数$\,(1/2)\partial_{\mu}(\epsilon^{\mu\nu\rho\sigma}A_{\nu}f_{\rho\sigma})$, $f_{\mu\nu}\,$自始至终可以写为$\,\partial_{\mu}A_{\nu}-\partial_{\nu}A_{\mu}$. 然而, 正如\,23.3\,节中所讨论的, 
我们用来把$\,\phi_{A}\,$变到\,3\,-方向上的规范变换在某处必须是奇异的, 所以我们无法将$\,f_{\mu\nu}\,$处处都用单个$\,A_{\mu}\,$表示. 结果是, 
物理量{\kai{会}}被作用量中形如
\begin{equation}
    -\frac{\theta}{64\uppi^{2}}\int\dif^{4}x\:\epsilon^{\mu\nu\rho\sigma}f_{\mu\nu}f_{\rho\sigma}\label{29.5.38}
\end{equation}
的项影响. 特别地, Witten\cite{11}证明了, 在\,23.3\,节讨论的磁单极子的理论中, 单极矩最小的磁单极子在有这种项时拥有电荷$\,e\theta/2\uppi$. %
当正如\,\ref{sec:27.9}\,节所提及的, 这个理论中还有双荷子, 即, 有磁单极距又有$\,e\,$的整数倍电荷的粒子. 所以单极子和双荷子电荷的整个模式关于$\,\theta,$是周期的, 
周期是$\,2\uppi$. 事实上, 由于低能有效理论的$\,\theta\,$是从出现在底层理论拉格朗日密度(\ref{29.5.1})中的$\,\theta\,$继承而来(注意方程(\ref{29.5.21})), 
而我们在\,23.5\,节看到所有物理量关于这个$\,\theta\,$都是周期的, 所以所有物理量都有这个周期性. 根据方程(\ref{29.5.37}), %
变换(\ref{29.5.35})对$\,\theta\,$的改变是$\,2\uppi b$, 因此当$\,b\,$是任意正或负整数时, 它会给出等价的有效作用量.

这类变换与底层理论的一个精确不变性紧密相关. 拉格朗日量(\ref{29.5.1})在一个连续$\,R\,$-对称性下不变, 在这个对称性下
\begin{equation}
    \theta_{L} \to \exp(\mi\alpha)\theta_{L}\:, \qquad
    W_{AL} \to \exp(\mi\alpha)W_{AL} \:, \qquad
    \Phi_{A}\to \exp(2\mi\alpha) \Phi_{A} \:. \label{29.5.39}
\end{equation}
这个对称性被反常破坏: $\lambda_{AL}\,$和$\,\psi_{AL}\,$的$\,R$-量子数为$\,+1$, 而$\,\lambda_{AR}\,$和$\,\psi_{AR}\,$显然有$\,R\,$-量子数$\,-1$, 所以根据方程(\textcolor{foo}{23.5.21})和(\textcolor{foo}{23.5.23}), 积分中对费米子场的测度在变换(\ref{29.5.39})下会有一个$\,\exp(2\mi\alpha N\nu)$ 因子的变化, 其中$\,\nu\,$是一个整数, 即矢量场构形的缠绕数, 而$\,N\,$定义成
\[
\operatorname{Tr}(t_{A}t_{B}) = \frac{1}{2}N\delta_{AB} \:.
\]
(即使我们在规范群同一表示下有两个场$\,\psi_{A}\,$和$\,\lambda_{A}$, 相较于\,23.5\,节中使用的\,Dirac\,场, 因为它们是 Majorana\,场, 所以因子$\,\exp(2\mi\alpha N\nu)\,$与\,23.5\,节中给出的相同.) 这里的生成元是$\,(t_{A})_{BC}=-\mi\epsilon_{ABC}$, 所以$\,N=4$, 因此当$\,\exp(8\mi\alpha)=1\,$时测度保持不变. 换句话说, 连续$\,R\,$-对称性被瞬子破缺至$\,\mathds{Z}_{8}\,$子群, 生成该子群的变换是
\begin{equation}
    \psi_{AL}\to \sqrt{\mi}\,\psi_{AL} \:, \qquad \lambda_{AL}\to \sqrt{\mi}\,\lambda_{AL}\:,\qquad
    \phi_{A} \to \mi\,\phi_{A} \:. \label{29.5.40}
\end{equation}
这样, 这个对称性必须留存于有效低能理论中. 然后, 它{\kai{没有}}告诉我们有效拉格朗日量(\ref{29.5.20})在要求对所有$\,a\,$都有$\,-\mi h(\mi a)=h(a)\,$的离散变换(\ref{29.5.40})下不变. 树级近似(\ref{29.5.21})是满足这个条件的, 但(我们将要看到)这个条件哪怕在单圈阶都会被破坏. 在有效理论中实现$\,\mathds{Z}_{8}$ $R\,$-对称性的方式是$\,h\,$函数是$\,-\mi h(\mi a)\,$的有效理论{\kai{等价于}}$\,h\,$函数是$\,h(a)\,$的理论. 即, $-\mi h(\mi a)\,$必须通过变换(\ref{29.5.32})和 $b$\,是整数的变换(\ref{29.5.35})的某个线性组合与$\,h(a)\,$相关联.

我们已经看到理论在如下两类变换下不变的物理意义: 变换$\,\Phi\to\tilde{\Phi}=h(\Phi)\,$和$\,h(\Phi)\to\tilde{h}(\tilde{\Phi})=-\Phi$, 其可以写为
\begin{equation}
\begin{pmatrix}
\Phi \\ h(\Phi)
\end{pmatrix}  \to
\begin{pmatrix}
 0 & 1 \\ -1 & 0
\end{pmatrix}
\begin{pmatrix}
\Phi \\ h(\Phi)
\end{pmatrix}  \:, \label{29.5.41}
\end{equation}
以及变换$\,h(\Phi)\to h(\Phi)+b(\Phi)$, $\Phi\to\Phi$, 其可以写为
\begin{equation}
\begin{pmatrix}
\Phi \\ h(\Phi)
\end{pmatrix}  \to
\begin{pmatrix}
 1 & 0 \\ b & 1
\end{pmatrix}
\begin{pmatrix}
\Phi \\ h(\Phi)
\end{pmatrix}  \:, \label{29.5.42}
\end{equation}
其中$\,b\,$是一个任意整数. 将这些变换组合在一起, 我们可以给出推广的对偶变换
\begin{equation}
\begin{pmatrix}
\Phi \\ h(\Phi)
\end{pmatrix}  \to
\begin{pmatrix}
 n & m \\ k & l
\end{pmatrix}
\begin{pmatrix}
\Phi \\ h(\Phi)
\end{pmatrix}  \:, \label{29.5.43}
\end{equation}
其中, 同方程(\ref{29.5.41})和(\ref{29.5.42})中的矩阵一样, 方程(\ref{29.5.43})中的矩阵元$\,n,m,k\,$和$\,\ell\,$%
是任何满足使得矩阵行列式为\,1\,的整数:
\begin{equation}
    nl-mk =1 \:. \label{29.5.44}
\end{equation}
这些变换因此构成群$\,SL(2,\mathds{Z})$.



对偶变换的物理意义可以通过考虑$\,N=2\,$超对称代数的中心荷提取出来. 正如\,\ref{sec:25.5}\,节中所证明的, 这个中心荷在任何单粒子态中的本征值$\,Z_{12}\,$为该态的质量设置了下界$\,M\geq\lvert Z_{12}\rvert/2$, 当粒子属于``短''超多重态时, 粒子质量抵达这个下界. 我们在\,\ref{sec:27.9}\,节看到, 在底层的$\,N=2\,$超对称$\,SU(2)\,$规范理论论中, $Z_{12}\,$由方程(\ref{27.9.22})给出:
\[
Z_{12} = 2\sqrt{2}\,v\,[\mi q-\mathscr{M}] \:,
\]
其中$\,q\,$和$\,\mathscr{M}\,$是粒子的电荷和磁矩, 而$\,v\,$是按惯例归一化的中性标量场的真空期望值. 在本节所使用的符号约定下, 我们把因子$\,e\,$吸收进场$\,a\,$的归一化中, 所以这里我们有
\begin{equation}
    Z_{12} =2\sqrt{2}\,a\,[\mi q-\mathscr{M}]/e \:. \label{29.5.45}
\end{equation}
这个理论包含有无质量基本标量, 旋量, 以及矢量玻色子这样的粒子, 它们的电荷是$\,e\,$磁矩为零, 所以这样的粒子是$\,Z_{12}\,$的本征态, 其本征值是
\begin{equation}
    Z_{12}^{\text{charged elementary}} = 2\sqrt{2}\,\mi\,a \:. \label{29.5.46}
\end{equation}
我们已经看到, 标量场为$\,a\,$和$\,h\,$函数为$\,h(a)\,$的理论等价于标量场为$\,n\,a+m\,h(a)\,$以及$\,h\,$函数为$\,k\,a+l\,h(a)\,$的理论. 因此, 对任何整数$\,n\,$和$\,m$, 标量场为$\,a\,$和$\,h\,$函数为$\,h(a)\,$的理论{\kai{还}}必须包含这样的一个粒子, 使得它从标量场为$\,n\,a+m\,h(a)\,$以及$\,h\,$函数为$\,k\,a+l\,h(a)\,$的理论的视角来看有电荷$\,e\,$和零磁矩, 因此有中心荷
\begin{equation}
    Z_{12} = 2\sqrt{2}\,\mi\Bigl[n\,a+ m\,h(a)\Bigr]\:. \label{29.5.47}
\end{equation}
与方程(\ref{29.5.45})相比较, 我们看到, 从标量场为$\,a\,$和$\,h\,$函数为$\,h(a)\,$的理论的视角来看, 这个粒子是一个带有电荷$\,q\,$和磁单极距$\,\mathscr{M}\,$的粒子, 其中
\begin{equation}
    q/e = n + m\operatorname{Re}[h(a)/a] \:, \qquad \mathscr{M}/e=m\operatorname{Im}[h(a)/a]\:. \label{29.5.48}
\end{equation}
这是一个双荷子, 即既有电荷又有磁单极距. 注意到, 这个粒子的电荷公式(\ref{29.5.48})证实了早先 Witten\,对磁单极子电荷的结果: 给函数$\,h(a)\,$加上$\,ba\,$对$\,m{=}1\,$的单极子电荷的改变是$\,be=e\Delta\theta/2\uppi$.

在方程(\ref{29.5.48})中使用$\,h(a)\,$的树级近似结果(\ref{29.5.21})表明, 在这个近似下, 这个理论中的磁单极距是$\,4\uppi/e\,$的整数倍. %
这与\,23.3\,节中推导出的磁单极距相同(回忆, 那里所使用的磁单极矩是 $\mathscr{M}/4\uppi$), 但这是半经典结果, 要做量子修正. 
注意到, $n=0\,$且$\,m=1\,$的对偶变换(\ref{29.5.41})将电荷为$\,e\,$而磁单极距为零的基本粒子变成一个磁单极矩为$\,\operatorname{Im}[h(a)/a]\,$的非基本粒子, 它的磁单极距在树级近似下是$\,4\uppi/e$.


在底层$\,N=2\,$超对称$\,SU(2)\,$规范理论中, 电荷的$\,\beta\,$函数的微扰论结果由单圈结果(\ref{27.9.50})给出, %
其中一阶\,Casimir\,不变量现在取为$\,C_{1}=2e\,$而极多重态的个数为$\,H=0$:
\begin{equation}
    \beta_{\text{perturbative}}(e) = -\frac{e^{3}}{4\uppi^{2}} \:. \label{29.5.49}
\end{equation}
我们这里取$\,a\,$为重振化标度, 所以跑动电荷$\,e(a)\,$满足
\begin{equation}
    a\frac{\dif }{\dif a} e(a) = \beta\Bigl(e(a)\Bigr) \:. \label{29.5.50}
\end{equation}
利用微扰论公式(\ref{29.5.49}), 这给出
\begin{equation}
    [e^{-2}(a)]_{\text{perturbative}}= \frac{1}{2\uppi^{2}}\ln\biggl(\frac{a}{\Lambda}\biggr) \:, \label{29.5.51}
\end{equation}
其中$\,\Lambda\,$是积分常数. 在这里我们所采用的形式体系中, 因子$\,e\,$别吸收进规范场的定义中, %
$e^{-2}(a)\,$作为$\,-\frac{1}{4}f_{\mu\nu}f^{\mu\nu}\,$的系数$\,h'(a)/4\uppi\mi\,$出现在低能有效拉格朗日量(\ref{29.5.20})中, 
所以函数$\,h(a)\,$在微扰论中是\footnote{因为$\,[\Phi^{*}]_{D}\,$是一个导数, 在对$\,h'(a)\,$积分中所需的加性常数对从方程(\ref{29.5.22})中导出的低能有效作用量没有影响. 通过参考前面讨论的$\,\mathds{Z}_{8}\,$-对称性可以定下这个常数. 
方程(\ref{29.5.52})满足$\,-\mi h(\mi a)=h(a)-1$, 这是保持理论的物理意义不变的变换(\ref{29.5.35})在$\,b=-1\,$时的特殊情况. 
如果我们给方程(\ref{29.5.52})加上常数, 则不会是这种情况.}
\begin{equation}
    [h(a)]_{\text{perturbative}} = \frac{2\mi}{\uppi}\biggl[a\ln\biggl(\frac{a}{\Lambda}\biggr) -a\biggr]
    =4\uppi\mi a\biggl[[e^{-2}(a)]_{\text{perturbative}} -\frac{1}{2\uppi^{2}}\biggr] \:. \label{29.5.52}
\end{equation}
在$\,\lvert a\rvert\,$取的值很大时, 方程(\ref{29.5.51})给出的$\,e(a)\,$值很小, 这是一个很好的近似.\footnote{$\beta(e)\,$有来自于瞬子的非微扰修正,\cite{12}, 但这些修正在$\,a\to0\,$是快速为零.} 在这种情况下, 方程(\ref{29.5.48})对单极子和双荷子给出的磁单极距是
\begin{equation}
    \mathscr{M}_{\text{perturbative}}/e = 4\uppi\,m\,a
    \biggl[[e^{-2}(a)]_{\text{perturbative}} -\frac{1}{2\uppi^{2}}\biggr] \:, \label{29.5.53}
\end{equation}
其中$\,m\,$是任意常数. 注意, 这与将半经典公式$\,\mathscr{M}/e=4\uppi m/e^{2}\,$右边的$\,e\,$换成跑动电荷$\,e(a)\,$后给出的结果并{\kai{不}}相同.

当$\,\lvert a\rvert\,$的值充分大时, 方程(\ref{29.5.52})也满足一个必然的自洽性条件: 动能项$\,-\frac{1}{2}(\overline{\psi}\slashed{\partial}\psi)$, $-\frac{1}{2}(\overline{\lambda}\slashed{\partial}\lambda)$, $-\partial_{\mu}a^{\ast}\partial^{\mu}a^{\mu}\,$和$\,-\frac{1}{4}f_{\mu\nu}f^{\mu\nu}\,$的系数$\,\operatorname{Im} h'(a)/4\uppi\,$必须是{\kai{正的}}. 
因为方程(\ref{29.5.52})在$\,\lvert a\rvert\,$很小时给出的 $\operatorname{Im}h'(a)\,$值是负的, 
相同的条件同时告诉我们方程(\ref{29.5.52})对$\,a\,$的任何值不可能都是一个好近似.

在$\,\lvert a \rvert\,$的值很大, 即微扰论成立的区域, 方程(\ref{29.5.52})表明如果$\,a\,$沿一个圆逆时针绕行一圈, 那么$\,h(a)\,$会被偏移$\,-4a$. 
这告诉我们$\,h(a)\,$在$\,a\,$的有限值处必有一个或多个奇点.

我们可以轻易地排除$\,h(a)\,$只有一个奇点的情况, 这是因为这将使得$\,h'(a)\,$不可能满足对所有非奇异的$\,a\,$值都有正的虚部, 而正如我们看到, 
为了使有效拉格朗日量中的动能项系数是正的, 这个条件是必须的. 因为$\,\mathds{Z}_{8}\,$-对称性告诉我们如果$\,h(a)\,$在$\,a=0\,$处奇异, 
那么它在$\,\mi a\,$处也是奇异, 所以单个奇点必落在$\,a=0\,$处. 这样, 除了无穷大以及可能的$\,a=0\,$处, 
函数$\,h(a)-h_{\text{perturbative}}(a)\,$都是解析的, 但这个函数在$\,\lvert a\rvert\to\infty\,$时为零, 特别的, 
当$\,a\,$在大值附近的圆上绕行一圈, 函数值不会改变. 所以它最多只能是$\,1/a\,$的多项式. 如果这个多项式不为零, 那么$\,h'(a)\,$在$\,a\to 0\,$是必然趋于$\,a\,$的负幂次, 这样的函数没有正的虚部, 而如果这个多项式确实为零, 那么$\,h'(a)\,$将等于$\,h'_{\text{perturbative}}(a)$, 这样在$\,a\to0\,$时也没有正的虚部.

在研究$\,h(a)\,$的奇异性结构时, 通过以相同的方式处理$\,a\,$和$\,a_{D}=h(a)$, 将它们均表示为某个复变量$\,u\,$的函数, 这样可以帮助追踪对偶性 Seiberg\,和\,Witten\,将$\,u\,$取为规范不变量$\,\frac{1}{2}\sum_{A}\phi_{A}\phi_{A}\,$的期望值. 
这个量在$\,\mathds{Z}_{8}\,$变换$\,\phi_{A}\to\pm\mi\phi_{A}\,$下会改变符号, 而在这个变换下显然有$\,a\equiv \phi_{3}\to\pm\mi a$, 所以
\begin{equation}
    a(-u) = \pm\mi\,a(u) \:. \label{29.5.54}
\end{equation}
因为$\,a\,$和$\,-a\,$通过一个有限$\,SU(2)\,$规范变换相关联, 这个符号没有物理意义. (在必要的时候, 
通过取符号在$\,\operatorname{Re}u>0\,$和$\,\operatorname{Re}u<0\,$时分别为正号和负号的约定, 这个符号可以确定下来.) 
我们可以$\,\lvert a\rvert\,$的值很大时相信微扰论, 所以在$\,\lvert u\rvert \to \infty\,$时
\begin{equation}
    a\to \sqrt{2u}\:, \qquad a_{D} \to \frac{\mi}{\uppi}
    \biggl[\sqrt{2u}\ln\biggl(\frac{2u}{\Lambda^{2}}\biggr) - 2\sqrt{2u}\biggr] \:. \label{29.5.55}
\end{equation}
注意到, 当$\,u\,$在固定$\,\lvert u\rvert$值附近的一个圆上逆时针跑一圈后, 对数$\,\ln(2u/\Lambda^{2})\,$会被偏移$\,2\mi\uppi$, 
而$\,\sqrt{2u}\,$则改变符号, 所以$\,a\,$和$\,a_{D}\,$的变换可以由一个{\kai{单值矩阵}}给出:
\begin{equation}
    \begin{pmatrix}
    a \\ a_{D}
    \end{pmatrix} \to
    \begin{pmatrix}
    -1 & 0 \\ 2 & -1
    \end{pmatrix}
    \begin{pmatrix}
    a \\ a_{D}
    \end{pmatrix}  \:. \label{29.5.56}
\end{equation}
因此, 函数$\,a(u)\,$和$\,a_{D}(u)\,$必在$\,u\,$的有限值$\,u_{n}\,$处有两个或多个奇点, 使得在每个奇点绕行一圈后的组合效果与(\ref{29.5.56})相同.


\begin{figure}[t]
    \centering
    \begin{tikzpicture}[scale=1.7]
        \draw [<-,thick] (0,0) to [out=80,in=180] (1,1);
        \draw [<-,thick] (0.95,1) to [out=0,in=100] (2,0);
        \draw [thick] (0.005,0.05) to [in=-100, out=-95] (0.85,-1);
        \draw[->, thick] (1.15,-1) to [in=-85,out=-80] (1.995,0.05);
        \draw[->,thick] (0.85,-1) to (0.9,-0.3);
        \draw[thick] (0.9,-0.302) to [in=-120,out=85](1,0.3);
        \draw[->, thick] (1,0.3) to [in=95,out=-60](1.1,-0.302);
        \draw [thick] (1.1,-0.3) to (1.15,-1);
        \filldraw(1,0.3) circle (1.3 pt) node[right=4pt,above] {$P$};
        \filldraw (0.5,0) circle (1 pt) node[below] {$-u_{0}$};
        \filldraw (1.5,0) circle (1 pt) node[below] {$+u_{0}$};
    \end{tikzpicture}
    \vspace{3 mm}
    \caption{复平面大$\,\lvert u\rvert\,$值处的逆时针围道形变到一个从基点$\,P\,$出发, 逆时针绕行$\,-u_{0}\,$处的奇点一圈回到$\,P$, 再以逆时针绕行$\,+u_{0}\,$处的奇点一圈回到$\,P\,$的围道.}%
    \label{fig:29.1}%
  \end{figure}
  

我们来考虑只有两个奇点的可能性. (这被证明是参考文献[10]中\,Seiberg\,和\,Witten\,的第二篇文章所考虑的情况.) 在$\,a_{A}\to\mi a_{A}\,$的$\,\mathds{Z}_{8}\,$-对称性下, 我们有$\,u\to-u$, 所以奇点所在的$\,u\,$值必须是成对的, 记为$\,u_{0}\,$和$\,-u_{0}$. 从$\,u\,$空间中的一个非奇异基点$\,P\,$出发, 
然后以逆时针方向绕行$\,\pm u_{0}\,$处的两个奇点一圈回到$\,P$, 这将会产生一个等价的理论. 因此它必须采取对偶变换的形式, 一般会依赖于$\,P$, 
在这个变换中, 矢量$\,(a,a_{D})\,$要乘以一个$\,SL(2,\mathds{Z})\,$单值矩阵$\,M_{\pm}$, 就像方程(\ref{29.5.43})中那样. 
在固定的大$\,u\,$值处的逆时针围道可以形变成一个从$\,P\,$出发, 逆时针绕行$\,-u_{0}\,$一圈回到$\,P$, 再以逆时针绕行$\,+u_{0}\,$一圈回到$\,P$. 
(参看图\ref{fig:29.1}.) 由于这个形变不会改变积分, 单值矩阵在{\kai{这个顺序}}(从右往左读)下的乘积必须等于方程(\ref{29.5.56})中的矩阵:
\begin{equation}
    M_{+}M_{-} = M_{\infty} \equiv
    \begin{pmatrix}
    -1 & 0 \\ 2 & -1
    \end{pmatrix} \:. \label{29.5.57}
\end{equation}
当$\,a\,$和$\,a_{D}\,$所在的值使得某个粒子的质量为零时, 奇点会出现. 例如, 在$\,h(a)\,$的微扰论公式(\ref{29.5.52}) 中, $a=0\,$处有奇点是因为, 当$\,a\,$取这个值时, 基本带电荷粒子在微扰论中变成无质量的. 我们已经排除了只在$\,a=0\,$处有一个奇点的可能性, 
所以$\,\pm u_{0}\,$处的奇点必然来自于其它一些粒子的质量归零.

Seiberg--Witten\,计算的最显著部分是他们意识到这些粒子是底层$\,SU(2)\,$超对理论中发现的非基本磁单极子或双荷子. 
23.3\,节所做的那类半经典计算表明稳定的磁单极子和双核子有磁荷量子数$\,m=\pm 1\,$而电荷量子数$\,n\,$则是任意整数, 
并且这些粒子属于极多重态, 每个极多重态由一对\,Majorana\,旋量和一对复标量构成. 它们是``短''多重态, 质量由\,BPS\,值给出, 
而根据方程(\ref{27.9.24}) 和(\ref{29.5.47}), 这个值是
\begin{equation}
    M=\lvert Z_{12} \rvert/2 =\sqrt{2}\Bigl\lvert Na + h(a) \Bigr\rvert \:, \label{29.5.58}
\end{equation}
其中$\,N\equiv \pm n$. 计算在这个质量趋于零时发生了什么的最简单方法是考虑我们更加熟悉的问题: 当普通带电粒子的极多重态的质量趋于零时发生了什么, 然后使用对偶性转换会一个轻单极子的情况. $U(1)\,$规范耦合的$\,\beta\,$函数在这里由$\,C_{1}=0,$ $C_{2}^{f}=C_{2}^{s}=2\,$的方程(\ref{27.9.45})给出, 所以
\[
\beta(e)= +\frac{e^{3}}{8\uppi^{2}} \:.
\]
那么重整化群方程的解就是
\[
e^{-2}(a) = -\frac{1}{4\uppi^{2}}\ln \biggl(\frac{a}{\text{常数}}\biggr) \:.
\]
正如我们在推导方程(\ref{29.5.52})时所看到的, 这给出
\[
h^{\prime}(a) = 4\uppi\mi\,e^{-2}(a) = -\frac{\mi}{\uppi} \ln \biggl(\frac{a}{\text{常数}}\biggr) \:.
\]
因为这给出的$\,e^{2}(a)\,$在$\,a\to 0\,$时是一个小的正值, 所以, 如果理论确实包含普通带电粒子的极多重态且极多重态的质量在这个极限下为零, 那么这个公式是{\kai{可信}}的. 事实也的确如此, 我们假定存在质量趋于零的单极子或双荷子的极多重态, 为了处理这个情况, 我们可以使用将$\,a\,$变至$\,N(a)+h(a)\,$的对偶变换(\ref{29.5.43})
\begin{equation}
    \begin{pmatrix}
    a \\ h(a)
    \end{pmatrix} \to
    \begin{pmatrix}
    \hat{a} \\ \hat{h}(\hat{a})
    \end{pmatrix} =
    \begin{pmatrix}
    N & +1 \\ -1 & 0
    \end{pmatrix}
    \begin{pmatrix}
    a \\ h(a)
    \end{pmatrix} \:. \label{29.5.59}
\end{equation}
我们由此得出, 当$\,u\,$趋于使得$\,\hat{a}=Na+h(a)\to0\,$的点$\,u_{0}$, 我们有
\begin{equation}
    \frac{\dif \hat{h}(\hat{a})}{\dif \hat{a}} \to -\frac{\mi}{\uppi}
     \ln \biggl(\frac{\hat{a}}{\text{常数}}\biggr) \:, \label{29.5.60}
\end{equation}
或者, 换另一种形式,
\begin{equation}
    \frac{\dif a}{\dif (h(a)+Na)} \to +\frac{\mi}{\uppi} \ln \biggl(\frac{h(a)+Na}{\text{常数}}\biggr) \:. \label{29.5.61}
\end{equation}
解是
\begin{equation}
    a(u) = a_{0}+\frac{\mi}{\uppi}\Bigl(h(a)+Na(u)\Bigr)\ln\biggl(\frac{h(u)+Na(u)}{\Lambda_{0}}\biggr)\:, \label{29.5.62}
\end{equation}
其中$\,a_{0}\,$和$\,\Lambda_{0}\,$是积分常数. 我们还假定了在$u\to u_{0}\,$时, $h+Na\to0$, 所以我们可以把领头项写成
\begin{equation}
    h(u)+Na(u)\to c_{0}(u-u_{0}) \:. \label{29.5.63}
\end{equation}
因此方程(\ref{29.5.62})有领头项
\begin{equation}
    a(u)\to a_{0} +\frac{\mi\,c_{0}}{\uppi}(u-u_{0}) \ln\biggl(\frac{c_{0}(u-u_{0})}{\Lambda_{0}}\biggr) \:.
    \label{29.5.64}
\end{equation}

当我们让$\,u\,$围绕$\,u_{0}\,$逆时针绕行一圈时, $h(a)+Na\,$没有变化, 但$\,a\,$会被偏移$\,-2(h(a)+Na)$, 
所以这个奇点的单值矩阵是
\begin{equation}
    M_{+} = \begin{pmatrix}
    1-2N & -2 \\ 2N^{2} & 1+2N
    \end{pmatrix} \:. \label{29.5.65}
\end{equation}

同理, 如果$\,-u_{0}\,$处的奇点所对应的质量为零的单极子和双荷子有磁荷量子数$\,\pm'1\,$(撇号是为了与$\,u_{0}\,$的符号相区分)和电荷量子数$\,n'$, 那么在这个奇点处$\,h(u)\to N'a(u)\to 0$, 且
\begin{equation}
    a \to a_{0}^{\prime} + \frac{\mi}{\uppi}\Bigl(h(u) + N^{\prime}a(u)\Bigr)
    \ln\biggl(\frac{h(u)+N^{\prime}a(u)}{\Lambda_{0}^{\prime}} \biggr) \:, \label{29.5.66}
\end{equation}
其中$\,N'\equiv \pm'n'\,$而$\,a_{0}'\,$和$\,\Lambda_{0}'$是新的积分常数. 领头项是
\begin{equation}
    h(u) + N^{\prime}a(u) \to c_{0}^{\prime}(u+u_{0}) \:, \label{29.5.67}
\end{equation}
\begin{equation}
    a \to a_{0}^{\prime}+\frac{\mi\, c_{0}^{\prime}}{\uppi} (u+u_{0})
    \ln\biggl(\frac{c_{0}^{\prime}(u+u_{0})}{\Lambda_{0}^{\prime}}\biggr) \label{29.5.68}
\end{equation}
这个奇点的单值矩阵是
\begin{equation}
    M_{-}=\begin{pmatrix}
    1-2N^{\prime} & -2 \\ 2N^{\prime2} & 1+2N^{\prime}
    \end{pmatrix} \:. \label{29.5.69}
\end{equation}
这样直接就能看到, 当且仅当
\begin{equation}
    N^{\prime} = N-1 \:, \label{29.5.70}
\end{equation}
这些矩阵上的条件(\ref{29.5.57})才是满足的. 我们对$\,N\,$取什么值并不会有影响, 这是因为通过绕着无穷远处的圈$\,M\,$次, 我们可以将其偏移偶数$\,2M$, 而通过反射$\,u\to-u$, 我们可以将其偏移\,1. Seiberg 和\,Witten\,选择取$\,N=0$, 这使得$\,N'=-1$. 那么在$\,u\to u_{0}\,$时, 
\begin{equation}
    h(u)\to c_{0}(u-u_{0}) \:, \label{29.5.71}
\end{equation}
\begin{equation}
    a(u)\to a_{0}+\frac{\mi\,c_{0}}{\uppi}(u-u_{0})\ln\biggl(\frac{u-u_{0}}{\Lambda_{0}}\biggr) \:, \label{29.5.72}
\end{equation}
以及在$\,u\to-u_{0}\,$时
\begin{equation}
    h(u)-a(u)\to c_{0}^{\prime}(u+u_{0}) \:, \label{29.5.73}
\end{equation}
\begin{equation}
    a(u) \to a_{0}^{\prime} + \frac{\mi\, c_{0}^{\prime}}{\uppi}(u+u_{0})
    \ln\biggl(\frac{u+u_{0}}{\Lambda_{0}^{\prime}}\biggr) \:. \label{29.5.74}
\end{equation}

我们现在附加未破缺的$\,\mathds{Z}_{8}\,$-对称性条件(\ref{29.5.54}). 我们可以通过将$\,a(u)\,$写成$\,-\mi a(-u)\,$并使用方程(\ref{29.5.72})来计算它在$\,u\to-u_{0}\,$时的值. 那么, 在$\,u\to -u_{0}\,$时
\begin{equation}
    a(u) \to -\mi a_{0} + \frac{c_{0}}{\uppi}\ln\biggl(\frac{-c_{0}(u+u_{0})}{\Lambda_{0}}\biggr)\:. \label{29.5.75}
\end{equation}
与方程(\ref{29.5.74})相比较, 我们看到$\,c_{0}'=\mi c_{0}$, 所以在$\,u\to -u_{0}\,$时, 方程变成
\begin{equation}
    h(u)-a(u) \to \mi\,c_{0}\,(u+u_{0}) \:. \label{29.5.76}
\end{equation}

场$\,a\,$在定义包含了一个规范耦合因子$\,e$, 通过合适选择计算$\,e\,$的重整化点, $e\,$可以被赋予任何值. 
Seiberg\,和\,Witten\,选择定义$\,a\,$和$\,u\,$(保持$\,u=a^{2}/2\,$在无穷远处)的标度使得$\,u_{0}=1$; 
即奇点处在$\,u=\pm 1\,$处. 在这个约定下, 他们获得了解(定义解的复平面有$\,-1\,$到$\,+1\,$的割线)
\begin{equation}
    a_{\text{SW}}(u) =\frac{\sqrt{2}}{\uppi}\int_{-1}^{1}\dif x\: \sqrt{\frac{u-x}{1-x^{2}}} \:, \label{29.5.77}
\end{equation}
\begin{equation}
    h_{\text{SW}}(u) =\frac{\mi\,\sqrt{2}}{\uppi}\int_{1}^{u}\dif x\: \sqrt{\frac{u-x}{1-x^{2}}} \:. \label{29.5.78}
\end{equation}
最初用来获得这些结果的数学方法超出了本书的范围, 但幸运的是, 检验它们正确并不困难.

首先, 作为正确的解, 我们可以验证$\,a_{\text{SW}}(u)\,$和$\,h_{\text{SW}}(u)\,$在$\,u=\pm 1\,$处的奇点与(\ref{29.5.71}), 
(\ref{29.5.72}), (\ref{29.5.75})和(\ref{29.5.76})形式相同. 当$\,u\to 1\,$时, 方程(\ref{29.5.77})给出 
\begin{align}
    a_{\text{SW}}(u) &\to \frac{\sqrt{2}}{\uppi} \int_{-1}^{1}\frac{\dif x}{\sqrt{x+1}}
    +\frac{u-1}{2\uppi}\int_{-1}^{1}\frac{\dif x}{\sqrt{(1-x)(u-x)}} \nonumber \\
    &= \frac{4}{\sqrt{\uppi}} + \frac{u-1}{2\uppi}
    \ln\Biggl(\frac{u-1}{3+u-2\sqrt{2}\sqrt{1+u}}\Biggr) \nonumber \\
    &\to \frac{4}{\sqrt{\uppi}} - \frac{u-1}{2\uppi} \ln\Bigl(4(u-1)\Bigr) \:. \label{29.5.79}
\end{align}
另外, 方程(\ref{29.5.78})给出
\begin{equation}
    h_{\text{SW}}(u) \to \frac{\mi}{\uppi}\int_{1}^{u}\dif x\:\sqrt{\frac{u-x}{x-1}}
    = \frac{\mi(u-1)}{2} \:. \label{29.5.80}
\end{equation}
方程(\ref{29.5.79})和(\ref{29.5.80})与我们之前的结果(\ref{29.5.71})和(\ref{29.5.72})在$\,c_{0}=\mi\,$和$\,a_{0}=4/\sqrt{\uppi}\,$给出的值一致. 方程(\ref{29.5.77})满足$\,\mathds{Z}_{8}\,$反射性质(\ref{29.5.54}), 所以$\,a_{\text{SW}}(u)\,$在$\,u\to -1\,$时的行为自动是(\ref{29.5.75}), 并且(要注意从$\,u=-1\,$到$\,u=+1\,$的割线对平方根符号的影响)由此可以直接从方程(\ref{29.5.77})和 (\ref{29.5.78})得出$\,h_{\text{SW}}(u)-a_{\text{SW}}(u)\,$在$\,u\,$接近$\,-1\,$时正比于$\,u+1$. 


现在, 由于$\,a_{\text{SW}}\,$和$\,h_{\text{SW}}(u)\,$在$\,u\to\pm 1\,$处的奇点结构与$\,a(u)\,$和$\,h(u)\,$相同, 
它们有相同的单值性($N=0\,$和$\,N'=-1\,$的方程(\ref{29.5.59})和(\ref{29.5.69})): 当$\,u\,$绕点$\,+1\,$一圈, 我们有
\begin{equation}
    \begin{pmatrix}
    a_{\text{SW}} \\ h_{\text{SW}}
    \end{pmatrix} \to
    \begin{pmatrix}
     0 & +1 \\ -1 & 0
    \end{pmatrix}
    \begin{pmatrix}
    a_{\text{SW}} \\ h_{\text{SW}}
    \end{pmatrix} \:, \label{29.5.81}
\end{equation}
而当$\,u\,$绕点$\,-1\,$一圈,
\begin{equation}
    \begin{pmatrix}
    a_{\text{SW}} \\ h_{\text{SW}}
    \end{pmatrix} \to
    \begin{pmatrix}
    3 & -2 \\ 2 & -1
    \end{pmatrix}
    \begin{pmatrix}
    a_{\text{SW}} \\ h_{\text{SW}}
    \end{pmatrix} \:. \label{29.5.82}
\end{equation}


我们现在来考虑
\begin{equation}
    f(u)\equiv a(u) h_{\text{SW}}(u) - a_{\text{SW}}(u)h(u) \:. \label{29.5.83}
\end{equation}
这是一个$\,SL(2,\mathds{Z})\,$不变量, 所以它的单值性是平庸的: 当$\,u\,$绕着$\,+1\,$或$\,-1\,$一圈后, 它的值不变. 
$a(u)$, $h(u)$, $a_{\text{SW}}(u)\,$或$\,h_{\text{SW}}(u)\,$在有限处的奇点只能是在$\,\pm 1\,$处的对数奇点, 
但由于$\,f(u)\,$的单值性是平庸的, 它没有这些奇点, 因此它在所有的有限点处都是解析的.

为了计算全纯函数$\,f(u)$, 我们首先要验证函数$\,a_{\text{SW}}(u)\,$和$\,h_{\text{SW}}(u)\,$的渐进行为中的领头项分别
与$\,a(u)\,$和$\,h(u)\,$相同. 在$\,u\to\infty\,$时, 方程(\ref{29.5.77})给出
\begin{equation}
    a_{\text{SW}}(u)\to \frac{\sqrt{2u}}{\uppi} \int_{-1}^{1}\frac{\dif x}{\sqrt{1-x^{2}}} =\sqrt{2u} \:, \label{29.5.84}
\end{equation}
而方程(\ref{29.5.78})给出
\begin{equation}
    h_{\text{SW}}(u)\to \frac{\mi\sqrt{2u}}{\uppi} \int_{1}^{u}\frac{\dif x}{\sqrt{x^{2}-1}}
    \to  \frac{\mi\sqrt{2u}}{\uppi} \sqrt{u}\ln u  \:, \label{29.5.85}
\end{equation}
这与$\,a(u)\,$和$\,h(u)\,$的领头行为(\ref{29.5.55})相同.

这本身只证明了在$\,u\to\infty\,$时$\,f(u)/u\ln u\to 0$. 但注意到, $a(u)\,$的反射对称性(\ref{29.5.54})以及$\,a_{\text{SW}}(u)\,$的类似对称性告诉我们两个函数中的次领头项是$\,\sqrt{u}/u^{2}\,$阶的. 另外, $h(u)\,$和$\,h_{\text{SW}}(u)\,$中的次领头项是$\,\sqrt{u}\,$阶的. 由此得出我们可以通过令$\,a(u)=a_{\text{SW}}(u)=\sqrt{2u}\,$来计算$\,f(u)\,$渐进行为中的领头项, 
这使得在$\,u\to\infty\,$时$\,f(u)=O(u)$. 由于$\,f(u)\,$是全纯函数, 这意味着$\,f(u)\,$关于$\,u\,$是线性的. 
然而, $f(u)\,$在$\,u=+1\,$处为零, $h(u)\,$和$\,h_{\text{SW}}(u)\,$在这里是$\,O(u-1)$, 而$\,f(u)\,$也在$\,u=-1\,$处为零, 
$a(u)-h(u)\,$和$\,a_{\text{SW}}(u)-h_{\text{SW}}(u)\,$在这里则是$\,O(u-1)$, 所以$\,f(u)\,$必须处处为零. 我们由此得出
\begin{equation}
    \frac{a_{\text{SW}}(u)}{a(u)} = \frac{h_{\text{SW}}(u)}{h(u)} \equiv g(u) \:. \label{29.5.86}
\end{equation}
现在我们必须考虑函数$\,g(u)\,$的性质. 因为$\,h_{\text{SW}}(u)\,$和$\,h(u)\,$在$\,u=-1\,$以外的所有有限点处都是解析的, 
而$\,a(u)-h(u)\,$和$\,a_{\text{SW}}(u)-h_{\text{SW}}(u)\,$在$\,u=+1\,$以外的所有有限点处都是解析的, 
由此得出$\,g(u)\,$(也可以写为$[a_{\text{SW}}(u)-h_{\text{SW}}(u)]/[a(u)-h(u)]$)处处解析. ($a(u)\,$或$\,h(u)\,$在$\,u\neq\pm 1\,$是没有零点, 更何况二者的共同零点, 这是因为, 在$\,a(u)\,$或$\,h(u)\,$的零点处, 带电粒子或单极子的质量为零, 而我们已经假定对任何$\,u\neq \pm 1\,$都不是这样的情况.) 另外, 函数$\,a_{\text{SW}}(u)\,$和$\,h_{\text{SW}}(u)\,$在$\,u\to\infty\,$时的渐进行为中的领头项分别与$\,a(u)\,$和$\,h(u)\,$的相同, 这个性质意味着在$\,u\to\infty\,$时$\,g(u)\to 1$. 由此得出全纯函数$\,g(u)\,$对所有$\,u\,$都必须等于\,1, 因此
\begin{equation}
    a(u)=a_{\text{SW}}(u) \:, \qquad h(u)=h_{\text{SW}}(u) \:, \label{29.5.87}
\end{equation}
这正是所要证明的.



\section*{习题}
\noindent 1. 对于第\,\ref{cha:26}\,章习题\,4\,中的模型, 它在$\,a\neq 0\,$时的\,Witten\,指标是多少? $a=0\,$是多少? 
在$\,a=0\,$时, 这个模型中的超对称性是否会被高阶效应破缺? 试解释. \\

\noindent 2. 考虑这样的可重整超对称理论, 它有$\,SO(N_{c})\,$规范对称性以及$\,N_{f}\,$个处在$\,N\,$-矢量表示下的左手征标量超场$\,\Phi_{n}$. 当裸超势为零时, 你能对非微扰威尔逊型拉格朗日密度的非微扰结构说些什么? 如果有一个裸超势$\,\sum_{n}\Phi_{n}\Phi_{n}\,$又将如何? \\

\noindent 3. 在拉格朗日密度是(\ref{29.5.22})的理论中, 旋量场强超场$\,W_{\alpha}\,$的分量场与其对偶$\,\tilde{W}_{\alpha}\,$的分量场之间有什么关系? \\


%++++++++++++++++++参考文献+++++++++
\renewcommand{\sectionmark}[1]{\markright{ #1}{}}
\renewcommand{\bibname}{参考文献}

\begin{thebibliography}{99}
    \bibitem{1} E. Witten, {\textit{Nucl. Phys.}} {\bf{B185}}, 513 (1981), 这篇文章重印于{\textit{Supersymmetry}}, S. Ferrara\,编辑(North Holland/World Scientific, Amsterdam/Singapore, 1987).
    \bibitem[1a]{1a} J. Hughes and J. Polchinski, {\textit{Nucl. Phys.}} {\bf{B278}}, 147 (1986); J. Hughes, J. Liu, and J. Polchinski, {\textit{Phys. Lett.}} {\bf{B180}}, 370 (1986); J. Bagger and A. Galperin, {\textit{Phys. Lett.}} {\bf{B336}}, 25 (1994); {\textit{Phys. Rev.}} {\bf{D55}}, 1091 (1997); {\textit{Phys. Lett.}} {\bf{412}}, 296 (1997); L. Antoniadis, H. Partouche, and T. R. Taylor, {\textit{Phys. Lett.}} {\bf{B372}}, 83 (1996); S. Ferrara, L. Girardello, and M. Porrati, {\textit{Phys. Lett.}} {\bf{B376}}, 275 (1996).
    \bibitem{2} E. Witten, {\textit{Nucl. Phys.}} {\bf{B202}}, 253 (1982). 这篇文章重印于{\textit{Supersymmetry}}, 参考文献[1].
    \bibitem{3} A. C. Davis, M. Dine, and N. Seiberg, {\textit{Phys. Lett}}. {\bf{125B}}, 487 (1983); I. Affleck, M. Dine, and N. Seiberg, {\textit{Phys. Rev. Lett.}} {\bf{51}}, 1026 (1983); {\textit{Phys. Lett.}} {\bf{137B}}, 187 (1984), 重印于{\textit{Supersymmetry}}, 参考文献[1]; {\textit{Nucl. Phys.}} {\bf{B241}}, 493 (1984); {\textit{Nucl. Phys.}} {\bf{B52}}, 1677 (1984); {\textit{Phys. Lett.}} {\bf{140B}}, 59 (1984).
    \bibitem{4} N. Seiberg, {\textit{Phys. Lett.}} {\bf{B318}}, 469 (1993).
    \bibitem{5} V. A. Novikov, M. A. Shifman, A. I. Vainshtein, and V. I. Zakharov, {\textit{Nucl. Phys.}} {\bf{B229}}, 381 (1983); {\textit{Nucl. Phys.}} {\bf{B260}}, 157 (1985); G. C. Rossi and G. Veneziano, {\textit{Phys. Lett.}} {\bf{138B}}, 195 (1984), 重印于{\textit{Supersymmetry}}, %
        参考文献[1]. 综述可参看\,M. A. Shifman and A. I. Vainshtein, hep-th/9902018, 待发表.
    \bibitem{6} E. Witten, {\textit{J. High Energy Phys.}} {\bf{9802}}, 006 (1998).
    \bibitem{7} V. G. Kac and A. V. Smilga, hep-th/9902029, 待发表.
    \bibitem{8} I. Affleck, M. Dine, and N. Seiberg, {\textit{Nucl. Phys.}} {\bf{B256}}, 557 (1985).
    \bibitem{9} N. Seiberg, {\textit{Phys. Rev.}} {\bf{D49}}, 6857 (1994).
    \bibitem{10} N. Seiberg and E. Witten, {\textit{Nucl. Phys.}} {\bf{B426}}, 19 (1994); {\textit{erratum, Nucl. Phys.}} {\bf{B430}}, 485 (1994). N. Seiberg\,和\,E.Witten\,给出了到有物质极多重态的$\,N=2\,$理论的推广, {\textit{Nucl. Phys.}} {\bf{B431}}, 484 (1994). 综述可以参看: K. Intrilligator and N. Seiberg, {\textit{Nucl. Phys. Proc. Suppl.}} {\bf{45BC}}, 1 (1996); 另发表于\,{\textit{Summer School in High Energy Physics and Cosmology, Trieste, 1995}}, E. Gava\,编辑(World Scientific, Singapore, 1997), 又收录于\,{\textit{QCD and Beyond: Proceedings of the Theoretical Advanced Study Institute in Elementary Particle Physics, Boulder Colorado, 1995}}, D. E. Soper\,编辑(World Scientific, Singapore, 1996).
    \bibitem{11} E. Witten, {\textit{Phys. Lett.}} {\bf{86B}}, 283 (1979).
    \bibitem{12} N. Seiberg, {\textit{Phys. Lett.}} {\bf{B206}}, 75 (1988).
\end{thebibliography}


%第三十章 超图

\chapter{超图}  \label{cha:30}


在\,20\,世纪\,40\,年代, Feynman\,图的引入为微扰计算提供了每一步都保持显式的\,Lorentz\,不变性的优势. 对于这个目的, 
交换任何虚粒子的所有自旋态都必须要由单个传播子描述. 幸运的是, 更进一步, 发展出超图的形式体系, 使得超对称性和\,Lorentz\,不变性在每一步都显然, 
这是有希望的.\cite{1} 为了实现这点, 有必要用单个超传播子来描述交换给定超场所描述的所有粒子.


这里有一个问题: 我们要对其积分的左手征超场$\,\Phi\,$要满足一个微分约束, $\mathscr{D}_{R}\Phi=0$. 这很像电动力学中的那个问题: 
场强张量要满足一个微分约束, 齐次\,Maxwell\,方程. 在电动力学中, 处理这个问题的方式是用矢势表示场强, 然而对矢势而非场强做路径积分. 
我们在这里使用一种非常类似的方法, 我们通过用势超场表示左手征超场的方法来附加这个约束, 然后对势超场做路径积分. 
在这个形式体系中, 我们遇到的问题很像电动力学中规范不变性所产生的那些问题, 并用非常类似的方法来处理这些问题.

超图形式体系所给出的最重要结果是超势的无重整定理.\cite{2} 在\,\ref{sec:27.6}\,节中, 通过使用\,Seiberg\,发展出的一个容易得多的间接技术, 
我们证明了这个定理, 而这个技术也可以推广以描述非微扰效应. 然而, 看到这些重正化效应的神奇相消如何发生在实际的微扰计算中是非常有趣的.


\section{势超场} \label{sec:30.1}

考虑左手征超场$\,\Phi_{n}(x,\theta)\,$及其复共轭的理论, 但简单起见没有规范超场. 所有分量场编时乘积的真空期望值可以从这些超场编时乘积的真空期望值计算出来. 
我们可以尝试用路径积分
\begin{align}
    &\biggl\langle T\Bigl\{\Phi_{n_{1}}(x_{1},\theta_{1}),\Phi_{n_{2}}(x_{2},\theta_{2}),\cdots\Bigr\}\biggr\rangle =
    \int\Biggl[\prod_{n,x,\theta}\dif \Phi_{n}(x,\theta)\Biggr] \,\exp\Bigl(\mi\,I[\Phi]\Bigr) \nonumber\\
    &\qquad \quad \times \Phi_{n_{1}}(x_{1},\theta_{1})\,\Phi_{n_{2}}(x_{2},\theta_{2})\cdots  \label{30.1.1}
\end{align}
来计算它们, 其中$\,I[\Phi]\,$是作用量
\begin{equation}
    I[\Phi] = \frac{1}{2}\int \dif^{4}x\:\Bigl[\sum_{n}\Phi_{n}^{\ast}(x,\theta)\Phi_{n}(x,\theta)\Bigr]_{D}
    +2\operatorname{Re}\int\dif^{4}x\:\Bigl[f(\Phi)\Bigr]_{\mathscr{F}} \:. \label{30.1.2}
\end{equation}
(同方程(\ref{26.4.3})中一样, 在第一项中引入因子$\,1/2\,$是为了使$\,\Phi\,$的分量场是按惯例归一化的.) 
但我们{\kai{不能}}简单地从方程(\ref{30.1.1})中读出\,Feynman\,规则, 这是因为对超场$\,\Phi_{n}\,$的泛函积分要被约束%
以满足左手征条件$\,\mathscr{D}_{R}\Phi_{n}=0$.


这类似于电动力学中的问题. 正如在\,12.3\,节中所讨论的, 当能量低于电子质量时, 软光子之间的相互作用可以被如下的有效拉格朗日量描述:
\[
    I[f] = -\frac{1}{4}\int \dif^{4}x\:\Biggl[f_{\mu\nu}f^{\mu\nu}+c_{1}\Bigl(f_{\mu\nu}f^{\mu\nu}\Bigr)^{2}
    +c_{2}\Bigl(\epsilon_{\mu\nu\rho\sigma}f^{\mu\nu}f^{\rho\sigma}\Bigr)^{2}\Biggr] \:.
\]
但是, 如果不考虑路径积分被齐次\,Maxwell\,方程
\[
\partial_{\mu}f_{\nu\rho}+\partial_{\nu}f_{\rho\mu}+\partial_{\rho}f_{\mu\nu} =0 
\]
所约束这个事实, 我们无法从这个作用量中读出\,Feynman\,规则. 众所周知, 我们处理这个约束的方式是引入一个\,4\,-矢势$\,A_{\mu}$, 
满足$\,f_{\mu\nu}=\partial_{\mu}A_{\nu}-\partial_{\nu}A_{\mu}$, 这使得这个约束自动满足, 然后对$\,A_{\mu}(x)\,$积分而非$\,f_{\mu\nu}(x)$.

以相同的方法, 我们可以采用在\,\ref{sec:26.6}\,节就用来推导超场场方程的那个技巧, 并引入非手征实超场$\,S_{n}(x,\theta)$, 满足
\begin{equation}
    \Phi_{n} = \mathscr{D}_{R}^{2}S_{n} \:. \label{30.1.3}
\end{equation}
其中
\begin{equation}
    \mathscr{D}_{R}^{2} \equiv \sum_{\alpha\beta}\epsilon_{\alpha\beta}\mathscr{D}_{R\alpha}\mathscr{D}_{R\beta}\:,
    \label{30.1.4}
\end{equation}
这使得$\,\Phi_{n}\,$自动满足左手征约束, $\mathscr{D}_{R\alpha}\Phi_{n}=0$. 取代方程(\ref{30.1.1}), 我们有路径积分公式
\begin{align}
    &\biggl\langle T\Bigl\{\Phi_{n_{1}}(x_{1},\theta_{1}),\Phi_{n_{2}}(x_{2},\theta_{2}),\cdots\Bigr\}\biggr\rangle =
    \int\Biggl[\prod_{n,x,\theta}\dif S_{n}(x,\theta)\Biggr] \,\exp\Bigl(\mi\,I[\mathscr{D}_{R}^{2}S]\Bigr) \nonumber\\
    &\qquad \quad \times \mathscr{D}_{R}^{2}S_{n_{1}}(x_{1},\theta_{1})\,
    \mathscr{D}_{R}^{2}S_{n_{2}}(x_{2},\theta_{2})\cdots \:. \label{30.1.5}
\end{align}
当用$\,S_{n}\,$表示作用量(\ref{30.1.2})后, 我们回忆起超导数的$\,D\,$-项对作用量没有贡献, 
所以我们可以将方程(\ref{30.1.2})第一项中作用在$\,S_{n}^{\ast}\,$上的算符$\,(\mathscr{D}_{R}^{2})^{\ast}=\mathscr{D}_{L}^{2}\,$移到$\,S_{n}\,$上, 
这给出
\begin{equation}
    I[\mathscr{D}_{R}^{2}S]=\frac{1}{2}\int\dif^{4}x\:
    \Biggl[\sum_{n}S_{n}^{\ast}\mathscr{D}_{L}^{2}\mathscr{D}_{R}^{2}S_{n}\Biggr]_{D}
    +2\operatorname{Re}\int\dif^{4}x\Bigl[f(\mathscr{D}_{R}^{2}S)\Bigr]_{\mathscr{F}} \:. \label{30.1.6}
\end{equation}
由于$\,\mathscr{D}_{R}\,$作用在$\,\mathscr{D}_{R}^{2}S\,$上给出零, 对于\,$f(\mathscr{D}_{R}^{2}S)\,$的任何一项中的一个$\,\mathscr{D}_{R}^{2}S\,$因子, 我们可以将其中的算符$\,\mathscr{D}_{R}^{2}\,$提出来使之作用在外面. 以这种方法, 我们可以写下
\begin{equation}
    f(\mathscr{D}_{R}^{2}S) = \mathscr{D}_{R}^{2}\tilde{f}(S) \:, \label{30.1.7}
\end{equation}
其中$\,\tilde{f}(S)\,$是通过将$\,f(\mathscr{D}_{R}^{2}S)\,$中的每一项略掉任何{\kai{一个}}算符$\,\mathscr{D}_{R}^{2}\,$获得的. 例如, 
当超场只有一种时, 如果$\,f(\Phi)=\sum_{r}c_{r}\Phi^{r}$, 那么
\[
    \tilde{f}(S) =\sum_{r}c_{r}S\,(\mathscr{D}_{R}^{2}S)^{r-1} \:.
\]
利用方程(\ref{30.1.6}), (\ref{30.1.7})和(\ref{26.3.31}), 我们可以将整个作用量写成$\,D\,$-项的形式
\begin{equation}
    I[\mathscr{D}_{R}^{2}S]=\frac{1}{2}\int\dif^{4}x\:
    \sum_{n}\Bigl[S_{n}^{\ast}\mathscr{D}_{L}^{2}\mathscr{D}_{R}^{2}S_{n}\Bigr]_{D}
    +2\operatorname{Re}\int\dif^{4}x\Bigl[\tilde{f}(S)\Bigr]_{\mathscr{F}} \:. \label{30.1.8}
\end{equation}
方程(\ref{26.6.5})表明这也可以写成一个超空间积分:
\begin{equation}
    I[\mathscr{D}_{R}^{2}S]= -\frac{1}{4}\int \dif^{4}x \int\dif^{4}\theta\:
    \sum_{n}S_{n}^{\ast}\mathscr{D}_{L}^{2}\mathscr{D}_{R}^{2}S_{n}
    -\operatorname{Re}\int \dif^{4}x \int\dif^{4}\theta\: \tilde{f}(S) \:. \label{30.1.9}
\end{equation}
在超图体系的路径积分推导中, 我们将使用这个作用量.

\section{超传播子} \label{sec:30.2}

通常情况下, 可以直接从作用量中场的二次部分直接获得传播子. 如果我们把复标量场$\,\phi_{i}\,$
(其中$\,i\,$是包含时空坐标以及自旋和种类指标的混合指标)的二次部分写成如下形式
\begin{equation}
    I_{\text{quad}}[\phi] = -\sum_{ij}D_{ij}\phi_{i}^{\ast}\phi_{j} \:, \label{30.2.1}
\end{equation}
其中$\,D_{ij}\,$是厄米的, 那么就像\,9.4\,节中解释的那样, 传播子就是$\,\Delta=D^{-1}$. 
当对于某类矢量$\,\xi$, 作用量在(线性化)规范变换
\begin{equation}
    \phi_{i} \to \phi_{i} + \xi_{i}  \label{30.2.2}
\end{equation}
下不变时, 这就会出现问题. 在这个情况下, 我们有
\begin{equation}
    \sum_{i}D_{ij}\xi_{j} = 0 \:, \label{30.2.3}
\end{equation}
显然我们对``矩阵''$\,D_{ij}\,$求逆. 在电动力学中, 产生这个问题是因为拉格朗日密度在规范变换$\,A_{\mu}\to A_{\mu}+\partial_{\mu}\Lambda\,$下不变. 我们这里也有这个问题: 由于作用量实际上是$\,\mathscr{D}_{R}^{2}\,S_{n}\,$而非$\,S_{n}\,$自身的泛函, 
对于任何超场, 在变换
\begin{equation}
    S_{n} \to S_{n} +\mathscr{D}_{R}X_{n} \:, \label{30.2.4}
\end{equation}
作用量是不变的. 

在带电粒子的电动力学中, 因规范不变性引起的问题一般通过选择规范来解决, 例如通过\,15.5 节描述的\,Faddeev--Popov--de Witt\,方法. 
但这里在变换(\ref{30.2.2})下不变而引起的问题更像是光子的有效场论在能量低于产生带电粒子时的问题, 这时理论规范不变就是因为作用量
只包含规范不变的场. 在这种理论中有一个更加简单的做法. 除了$\,D_{ij}\,$本征值为零的本征矢$\,\xi_{i}\,$
(简单起见取在一个单一的方向上), 我们可以找到一组本征值$\,d_{\nu}\neq 0\,$的正交本征矢$\,u_{\nu i}$:
\begin{equation}
    \sum_{j}D_{ij}\,u_{\nu j}= d_{\nu}\,u_{\nu i}\:, \qquad
    \sum_{i}u_{\nu i}^{\ast}\,u_{\nu^{\prime}i} = \delta_{\nu\nu^{\prime}} \:, \qquad
    \sum_{i}u_{\nu i}^{\ast}\,\xi_{i}=0 \:. \label{30.2.5}
\end{equation}
我们可以引入一组新的积分变量$\,\phi'\,$和$\,\phi_{\nu}''$:
\begin{equation}
    \phi_{i}=\phi^{\prime}\,\xi_{i} + \sum_{\nu}\phi_{\nu}^{\prime}\,u_{\nu i} \:. \label{30.2.6}
\end{equation}
在量子期望值的\,Feynman\,图计算中所遇到的那类积分就可以写为
\begin{align}
    &\int \Biggl[\prod_{i}\dif\phi_{i}\,\dif\phi_{i}^{\ast}\Biggr]\,\exp\{\mi I_{\text{quad}[\phi]}\}
    \phi_{a}\cdots\,\phi_{b}^{\ast}\cdots = \mathscr{J}\int\dif\phi^{\prime}\,\dif\phi^{\prime\ast} \nonumber \\
    &\quad\times \int \Biggl[\prod_{\nu}\dif\phi_{\nu}^{\prime}\,\dif\phi_{\nu}^{\prime\ast}\Biggr]
    \exp\Biggl\{-\mi\sum_{\nu}d_{\nu}\,\lvert \phi_{\nu}^{\prime}\rvert^{2}\Biggr\} \:
    \Biggl[\phi^{\prime}\,\xi_{a} + \sum_{\nu}\phi_{\nu}^{\prime} u_{\nu a}\Biggr] \cdots \nonumber \\
    &\quad\qquad \times \Biggl[\phi^{\prime}\,\xi_{b} + \sum_{\nu}\phi_{\nu}^{\prime} u_{\nu b}\Biggr]^{\ast}\cdots \:,
    \label{30.2.7}
\end{align}
其中$\,\mathscr{J}\,$是变换(\ref{30.2.6})的雅克比行列式. 因为$\,\phi'\,$和$\,\phi^{\prime\ast}\,$没有出现在指数变量中, 
所以对$\,\phi'\,$和$\,\phi^{\prime\ast}\,$的积分显然不是合理定义的. 但如果作用量仅含规范不变量, 那么这不成问题, 
这是因为这样$\,\phi_{a},\phi_{b}$ 等就会与``流''$\,J_{a},J_{b}\,$等收缩, 其中流满足
\begin{equation}
    \sum_{a}\xi_{a}J_{a} = 0\:. \label{30.2.8}
\end{equation}
因此我们可以将方程(\ref{30.2.7})写成
\begin{align}
     &\int \Biggl[\prod_{i}\dif\phi_{i}\,\dif\phi_{i}^{\ast}\Biggr]\,\exp\{\mi I_{\text{quad}[\phi]}\}
    \phi_{a}\cdots\,\phi_{b}^{\ast}\cdots = \mathscr{C}\int\Biggl[\prod_{\nu}\dif\phi_{\nu}^{\prime}\,\dif\phi_{\nu}^{\prime\ast}\Biggr] \nonumber \\
    &\qquad \times \exp\Biggl\{-\mi\sum_{\nu}d_{\nu}\,\lvert \phi_{\nu}^{\prime}\rvert^{2}\Biggr\} \:
    \Biggl[\sum_{\nu}\phi_{\nu}^{\prime}u_{\nu a}\Biggr]\cdots
    \Biggl[\sum_{\nu}\phi_{\nu}^{\prime}u_{\nu b}\Biggr]^{\ast}\cdots \nonumber \\
    &\qquad + \xi\,\text{-项} \:, \label{30.2.9}
\end{align}
其中``$\,\xi\,$-项''是指正比一个或多个$\,\xi_{a},\xi_{b}\,$等因子的项, 这些项与满足方程(\ref{30.2.8})的$\,J\,$收缩后为零, 
而 $\mathscr{C}\,$是无限大常数$\,\mathscr{J}\int\dif \phi'$. 这样, 对$\,\phi_{\nu}'\,$的积分就给出
\begin{align}
    &\int \Biggl[\prod_{i}\dif\phi_{i}\,\dif\phi_{i}^{\ast}\Biggr]\,\exp\{\mi I_{\text{quad}[\phi]}\}
    \phi_{a}\cdots\,\phi_{b}^{\ast}\cdots \approx \sum_{\text{pairings}}\Bigl[-\mi\Delta_{ab}\Bigr] \cdots \nonumber \\
    &\qquad +\xi\,\text{-项} \:, \label{30.2.10}
\end{align}
其中对配对(pairings)求和是指对$\,\phi\,$的指标和$\,\phi^{\ast}\,$的指标进行配对的所有方式求和, 而$\,\Delta_{ab}\,$是传播子
\begin{equation}
    \Delta_{ab} = \sum_{\nu}\frac{u_{\nu a}\,u_{\nu b}^{\ast}}{d_{\nu}} \:. \label{30.2.11}
\end{equation}
取代计算求和(\ref{30.2.11}), 我们可以使用它的定义性质
\begin{equation}
    \sum_{c}D_{ac}\Delta_{cb} = \sum_{\nu}u_{\nu a}\,u_{\nu b}^{\ast} \equiv \Pi_{ab} \:, \label{30.2.12}
\end{equation}
其中$\,\Pi\,$是投影到与$\,\xi\,$正交的空间的算符:
\begin{equation}
    \Pi^{2}=\Pi \:, \qquad \Pi\,\xi=0 \:. \label{30.2.13}
\end{equation}
方程(\ref{30.2.12})的解仅在相差$\,\xi\,$-项意义下是唯一的, 但如果场$\,\phi_{i}\,$仅以规范不变组合的方式出现在作用量中, 
那么它们就不会造成问题.

例如, 在电动力学中, 我们可以将作用量的动能部分写成
\[
I_{\text{quad}}[A] = -\frac{1}{4}\int\dif^{4}x\:f_{\mu\nu}f^{\mu\nu}
= +\frac{1}{2}\int\dif^{4}x\: A^{\mu}\Bigl(\square \delta_{\mu}^{\nu}-\partial_{\mu}\partial^{\nu}\Bigr) A_{\nu} \:.
\]
微分算符$\,-\square\delta_{\mu}^{\nu}+\partial^{\nu}\partial_{\mu}\,$不是可逆的, 因为它有一个本征值为零的本征矢, 形如$\,\xi^{\mu}=\partial^{\mu}\Lambda$. 投影到与这些矢量正交的空间的算符是
\[
\Pi\indices{_\mu^\nu}(x,y) = [\delta\indices{_\mu^\nu}-\partial_{\mu}\partial^{\nu}\square^{-1}]\delta^{4}(x-y) \:,
\]
其中$\,\square^{-1}\delta^{4}(x-y)\,$是方程$\,\square[\square^{-1}\delta^{4}(x-y)]=\delta^{4}(x-y)\,$的任意解. 
这个传播子的定义解是
\[
\Bigl(-\square\delta_{\mu}^{\lambda} + \partial_{\mu}\partial^{\lambda}\Bigr) \Delta\indices{_\lambda^\nu}(x,y)
=\Pi\indices{_\mu^\nu}(x,y) \:,
\]
它有解
\[
\Delta\indices{_\lambda^\nu}(x,y) = \delta_{\mu}^{\nu}\,\Delta_{F}(x-y) + \partial_{\mu}\partial^{\nu}\,\text{-项} \:,
\]
其中$\,\Delta_{F}(x-y)\,$是通常的\,Feynman\,传播子(\textcolor{foo}{6.2.16}), 
满足$\,\square \Delta_F(x-y)=-\delta^{4}(x-y)$. (作用量中的\,1/2\,因子没有出现在这里的传播子的定义方程中是因为$\,A_{\mu}\,$是实场. 而对于方程(\textcolor{foo}{6.2.16})中的 Fourier\,积分分母中的$\,-\mi\epsilon$, 9.2\,节中的路径积分形式理论解释了它的起源.) 

对方程(\ref{30.1.9})第一项的观察表明时实超场传播子的定义方程是
\begin{equation}
    -\frac{1}{4}\mathscr{D}_{L}^{2}\,\mathscr{D}_{R}^{2}\,\Delta_{nm}^{S}(x,\theta;x^{\prime},\theta^{\prime})
    =\mathscr{P}\delta^{4}(x-x^{\prime})\delta^{4}(\theta-\theta^{\prime})\delta_{nm} \:, \label{30.2.14}
\end{equation}
其中$\,\mathscr{P}\,$是超空间微分算符, 满足投影算符的条件
\begin{equation}
    \mathscr{P}^{2} = \mathscr{P} \:, \qquad \mathscr{P}\mathscr{D}_{R} = 0 \:, \label{30.2.15}
\end{equation}
而$\,\delta^{4}(\theta-\theta')\,$是方程(\ref{26.6.8})中引入的费米\,$\delta\,$-函数. 解是
\begin{equation}
    \mathscr{P}= \frac{-1}{16\square}\mathscr{D}_{L}^{2}\,\mathscr{D}_{R}^{2}\:. \label{30.2.16}
\end{equation}
(显然有$\,\mathscr{P}\mathscr{D}_{R}=0$. 为了验证$\,\mathscr{P}^{2}=\mathscr{P}$, 我们需要使用(\ref{26.6.12}), 
它表明$\,\mathscr{D}_{R}^{2}\mathscr{D}_{L}^{2}\mathscr{D}_{R}^{2}=-16\square \mathscr{D}_{R}^{2}$.) 
方程(\ref{30.2.14})的解是
\begin{align}
    \Delta_{nm}^{S}(x,\theta;x^{\prime},\theta^{\prime}) &= -\frac{1}{4\square}\delta^{4}(x-x^{\prime})
    \delta^{4}(\theta-\theta^{\prime})\delta_{nm} \nonumber \\
    &= \frac{1}{4}\Delta_{F}(x-x^{\prime})\delta^{4}(\theta-\theta^{\prime})\delta_{nm}
    +\mathscr{D}_{R}\,\text{-项} \:. \label{30.2.17}
\end{align}

这是由势超场$\,S_{m}^{\ast}(x',\theta')\,$产生而被势超场$\,S_{n}(x,\theta)\,$湮灭的线的传播子, 
这个传播子就是我们计算作用量(\ref{30.1.9})的超图时所要使用的. 为了与普通传播子相联系, 
考虑由左手征超场$\,\Phi_{m}^{\ast}(x',\theta')\,$产生而被左手征超场$\,\Phi_{n}(x,\theta)\,$湮灭的线的传播子是有益的. 
通过用$\,\mathscr{D}_{R}^{2}\,$作用$\,S_{n}(x,\theta)\,$并用
$\,{{\mathscr{D}'}_{R}^{2}}^{\ast}={\mathscr{D}'}_{L}^{2}\,$作用$\,S_{m}^{\ast}(x',\theta')\,$可以获得这些手征超场, 
所以左手征超场的传播子是
\begin{equation}
    \Delta_{nm}^{\Phi}(x,\theta;x^{\prime},\theta^{\prime}) = \frac{1}{4}\mathscr{D}_{R}^{2}
    {\mathscr{D}_{L}^{\prime}}^{2}\Delta_{F}(x-x^{\prime})\,\delta^{4}(\theta-\theta^{\prime})\,\delta_{nm} \:.\label{30.2.18}
\end{equation}


例如, 传播子中$\,\theta\,$和$\,\theta'\,$的零阶项是
\begin{equation}
    \Bigl[\Delta_{nm}^{\Phi}(x,\theta;x^{\prime},\theta^{\prime})\Bigr]_{\theta=\theta^{\prime}=0}
    =\frac{1}{4}\biggl(\frac{\partial}{\partial\theta_{R}}\biggr)^{2}
    \biggl(\frac{\partial}{\partial\theta_{L}^{\prime}}\biggr)^{2}\Delta_{F}(x-x^{\prime})\,
    \delta^{4}(\theta-\theta^{\prime}) \,\delta_{nm} \:. \label{30.2.19}
\end{equation}
为了计算它, 我们回忆起费米$\,\delta\,$-函数的方程(\ref{26.6.8}):
\[
    \delta^{4}(\theta-\theta^{\prime})=\frac{1}{4}\Bigl((\theta_{L}-\theta_{L}^{\prime})^{\mathrm{T}}
    \epsilon(\theta_{L}-\theta_{L}^{\prime})\Bigr)\, \Bigl((\theta_{R}-\theta_{R}^{\prime})^{\mathrm{T}}
    \epsilon(\theta_{R}-\theta_{R}^{\prime})\Bigr) \:,
\]
从此我们发现$\,(\partial/\partial\theta_{R})^{2}(\partial/\partial\theta_{L}')^{2}\delta(\theta-\theta')=4$. 
方程(\ref{30.2.19})因此给出
\begin{equation}
    \Bigl[\Delta_{nm}^{\Phi}(x,\theta;x^{\prime},\theta^{\prime})\Bigr]_{\theta=\theta^{\prime}=0}
    =\Delta_{F}(x-x^{\prime})\,\delta_{nm} \:, \label{30.2.20}
\end{equation}
这正是超场标量分量通常的传播子.


\section{用超图进行计算}  \label{sec:30.3}

我们现在来考虑如果用上一节的结果来计算一组经典势超场$\,S_{n}(x,\theta)\,$及其伴随场的量子有效作用量$\,\Gamma(S,S^{\ast})$. 
沿用\,16.1\,节中所讨论的处理方法, 有效作用量可以定义成所有单粒子不可约连通超图的求和, 其中组成单粒子不可约连通超图的顶点直接与内线和外线相连. 
对于每条始于或终于一个顶点被$\,x\,$和$\,\theta\,$所标记的$\,n\,$型外线 , 我们分别引入一个\,c\,-数因子$\,S_{n}(x,\theta)\,$或$\,S_{n}^{\ast}(x,\theta)$(但没有传播子). 一个被$\,x,\theta\,$标记且与$\,N\,$条入线或出线相连的顶点, 设这$\,N\,$条线被记为$\,n_{1},n_{2},\cdots,n_{N}$, 
这个顶点所产生的项就是$\,\mi\,$分别乘以超势$\,\tilde{f}(S)\,$中$\,S_{1}S_{2}\cdots S_{N}\,$项的系数或该系数的复共轭. 
任何一个从$\,x,\theta\,$标记的顶点出发而进入被$\,x',\theta'\,$标记的顶点会产生一个传播子, 由方程(\ref{30.2.10})和(\ref{30.2.17}) 给出
\begin{equation}
-\frac{\mi}{4}\delta^{4}(\theta-\theta^{\prime})\Delta_{F}(x-x^{\prime}) \:. \label{30.3.1}
\end{equation}
另外, 方程(\ref{30.1.7})表明, 超导数$\,\mathscr{D}_{R}^{2}\,$作用在除了进入任意顶点的一条内线或外线以外的所有传播子或者外线的$\,S\,$因子上, 
而超导数$\,\mathscr{D}_{L}^{2}\,$作用在除了进入任意顶点的一条内线或外线以外的所有传播子或者外线的$\,S\,$因子上. 
对这些因子的乘积要积掉所有$\,x\,$和$\,\theta$; 量子有效作用量就是对所有单粒子不可约图做这样的积分并求和.

通过在超空间中分部积分, 与任何一个传播子(例如, 连接被$\,x,\theta\,$和$\,x',\theta'\,$标记的顶点)伴随的算符$\,\mathscr{D}_{L}^{2}\,$和(或)$\,\mathscr{D}_{R}^{2}$可以被挪到其它传播子或外线因子上. 这使得这个内线贡献的因子正比于$\,\delta^{4}(\theta-\theta')$. 
对$\,\theta'\,$的积分就消掉了这个$\,\delta\,$函数,  并把其它所有的$\,\theta'\,$换成$\,\theta$. (在数条内线连接同一对顶点的情况下, 我们可以使用费米$\,\delta\,$函数的性质$\,[\delta^{4}(\theta-\theta')]^{2}=0$.) 以这种方法继续下去, 
我们最终会得到一个\,4\,维$\,\theta\,$积分, 而所有的$\,\mathscr{D}\,$作用在外线因子$\,S_{n}\,$和$\,S_{m}^{\ast}\,$上.
即, 尽管一般不在空间坐标上定域, 但$\,\Gamma[S,S^{\ast}]\,${\kai{在费米坐标上是定域的}}.

在``规范''变换(\ref{30.2.4})下的不变性控制了这个泛函的结构. 这告诉我们每个外线因子$\,S_{n}\,$或$\,S_{n}^{\ast}$ 上分别要
作用算符$\,\mathscr{D}_{R}^{2}\,$或$\,\mathscr{D}_{L}^{2}$, 但会有两个例外. 这两个例外是, 外线都为入线或出线的项, 
其中除了一个外线因子$\,S_{n}\,$或$\,S_{n}^{\ast}\,$外, 其它所有外线因子都要分别用$\,\mathscr{D}_{R}^{2}\,$或$\,\mathscr{D}_{L}^{2}\,$作用
且没有其它超导数. 这样的项尽管不能表示成对仅由$\,\Phi_{n}\,$和(或)$\,\Phi_{n}^{\ast}\,$构成的函数的\,4\,-维$\,\theta\,$积分, 
它在变换(\ref{30.2.4})下仍然是不变的. 这是因为, 这种振幅在这个变换下的改变仅来自于外线因子$\,S_{n}\,$或$\,S_{n}^{\ast}\,$中{\kai{不}}被算符$\,\mathscr{D}_{R}^{2}\,$或$\,\mathscr{D}_{L}^{2}\,$所作用的那部分的变化, 如果我们用分部积分将其它$\,\mathscr{D}_{R}^{2}\,$或$\,\mathscr{D}_{L}^{2}\,$算符中的一个挪到这个改变上, 那么这个改变就被消掉了. 正如我们在\,\ref{sec:30.1}\,节看到的, 
这样的项在$\,\Gamma[S,S^{\ast}]\,$是$\,\Phi_{n}\,$或$\,\Phi_{n}^{\ast}$ 的泛函的$\,\mathscr{F}\,$-项, 因此会对超势或者它的复共轭有一修正. 
除了这些例外外, $\Gamma[S,S^{\ast}]\,$中的每一项都可以写为对仅是$\,\Phi_{n}\,$和(或)$\,\Phi_{n}^{\ast}\,$的泛函的\,4\,维$\,\theta\,$-积分, 
因此对有效作用量的$\,D\,$-项部分有一修正.

另外注意到, 如果作用量中的一项中除了一个外线因子$\,S_{n}\,$或$\,S_{n}^{\ast}\,$外都分别被$\,\mathscr{D}_{R}^{2}\,$或$\,\mathscr{D}_{L}^{2}\,$作用, {\kai{且}}被额外的$\,\mathscr{D}_{R}^{2}\,$或$\,\mathscr{D}_{L}^{2}\,$算符作用(例如$\,\mathscr{D}_{R}^{2}\mathscr{D}_{L}^{2}\mathscr{D}_{R}^{2}S_{n}\,$或$\,\mathscr{D}_{L}^{2}\mathscr{D}_{R}^{2}\mathscr{D}_{L}^{2}S_{n}^{\ast}\,$这样的组合), 那么这样的项也可以通过分部积分写成一个额外的$\mathscr{D}_{R}^{2}\,$或$\,\mathscr{D}_{L}^{2}\,$算符作用在先前未微分的外线因子$\,S_{n}\,$或$\,S_{n}^{\ast}\,$上. 这样的项因此可以表示成对仅是$\,\Phi_{n}\,$和(或)$\,\Phi_{n}^{\ast}\,$的泛函的\,4\,维$\,\theta\,$-积分, 因而可以对作用量的$\,D\,$-项部分有其它修正. $\Gamma[S,S^{\ast}]\,$中唯一无法用这种方式写的项是有$\,E\,$条入外线且{\kai{只有}}$\,E-1\,$个算符$\,\mathscr{D}_{R}^{2}\,$作用在$\,S_{n}\,$外线因子上, 或者$\,E^{\ast}\,$条出外线\footnoteB{原书误植为入外线}且{\kai{只有}}$\,E^{\ast}-1\,$个算符$\,\mathscr{D}_{L}^{2}\,$作用在$\,S_{n}^{\ast}\,$外线因子上. 因此通过计数超图对$\,\Gamma[S,S^{\ast}]\,$中相应的项贡献的算符$\,\mathscr{D}_{R}^{2}\,$或$\,\mathscr{D}_{L}^{2}\,$的个数, 我们可以分辨出超图是否对超势或它的共轭有贡献.

我们来计数这些超导数. 考虑如下的连通超图: $V_{n}\,$个有$\,n\,$条线进入的顶点; $V_{n}^{\ast}\,$个有$\,n\,$条线离开的顶点; $I\,$条内线; $E\,$条入外线; 以及$\,E^{\ast}\,$条出外线. 这些数之间有关系
\begin{equation}
I+E = \sum_{n}n V_{n} \:, \qquad I + E^{\ast}=\sum_{n}nV_{n}^{\ast} \:. \label{30.3.2}
\end{equation}
那么算符$\,\mathscr{D}_{R}^{2}\,$或$\,\mathscr{D}_{L}^{2}\,$的总数就是
\begin{equation}
N_{R} = \sum_{n}V_{n}(n-1)=I+E-V = L+V^{\ast}+E-1 \:, \label{30.3.3}
\end{equation}
和
\begin{equation}
N_{L} = \sum_{n}V_{n}^{\ast}(n-1) = I+E^{\ast}-V^{\ast} = L+V+E^{\ast}-1 \:, \label{30.3.4}
\end{equation}
其中$\,V=\sum_{n}V_{n}\,$是线进入的顶点的总数; $V^{\ast}=\sum_{n}V^{\ast}_{n}\,$是线离开的顶点的总数; 而$\,L=I-V-V^{\ast}+1\,$是圈的个数. 我们看到任何有圈的图有$\,N_{R}\geq E\,$和$\,N_{L}\geq E^{\ast}$, 使得算符$\,\mathscr{D}_{R}^{2}\,$的个数足以将所有$\,S_{n}\,$转换成$\,\Phi_{n}=\mathscr{D}_{R}^{2}S_{n}\,$或它的导数, 以及算符$\,\mathscr{D}_{L}^{2}\,$的个数足以将所有$\,S_{n}^{\ast}\,$转换成$\,\Phi_{n}^{\ast}=\mathscr{D}_{L}^{2}S_{n}^{\ast}\,$或它的导数. {\kai{因此, 任何有圈的图产生的贡献总会正比于左手征超场$\,\Phi_{n}\,$及其伴随场的泛函的\,4\,维$\,\theta\,$-积分------换句话说, $D\,$-项.}}

想要获得对$\,\mathscr{F}\,$项或其共轭的贡献的唯一方法是分别只有$\,N_{R}=E-1\,$个$\,\mathscr{D}_{R}^{2}\,$算法或只有$\,N_{L}=E^{\ast}-1\,$个$\,\mathscr{D}_{L}^{2}\,$算符. 根据方程(\ref{30.3.3})和(\ref{30.3.4}), 这样的图将分别有$\,L=0\,$和$\,V^{\ast}=0\,$或$\,L=0\,$和$\,V=0$. 换句话说, 由于我们只考虑单粒子不可约图, 我们只能从只有入线的顶点获得$\,\mathscr{F}\,$-项, 且只能从只有出线的顶点获得$\,\mathscr{F}\,$-项的伴随. 这个贡献正是原始超势中的被积$\,\mathscr{F}\,$-项, 或它的伴随. 因此我们再次看到: {\kai{直到微扰论的任意阶, 都不存在$\,\mathscr{F}\,$项有限或无限的重正化.}}


\section*{习题}
\noindent 1. 利用方程(\ref{30.2.18})计算手征超场的旋量分量和辅助分量的传播子. \\

\noindent 2. 考虑单个手征超场$\,\Phi\,$的超对称理论, 它有拉格朗日密度
\[
\mathscr{L}=\frac{1}{2}\Bigl[ \Phi^{\ast}\Phi\Bigr]_{D}+2\operatorname{Re}\Bigl(g[\Phi^{3}]_{\mathscr{F}}\Bigr) \:,
\]
其中$\,g\,$是任意复常数. 利用超图形式体系计算量子有效作用量的单圈贡献. 将答案表示成对坐标及单个格拉斯曼坐标$\,\theta\,$的积分.
\\

\noindent 3. 对于动能项是(\ref{27.3.17})的超对称阿贝尔规范理论, 规范超场$\,V(x,\theta)\,$的超传播子是什么? 






%++++++++++++++++++参考文献+++++++++
\renewcommand{\sectionmark}[1]{\markright{ #1}{}}
\renewcommand{\bibname}{参考文献}

\begin{thebibliography}{99}
    \bibitem{1} A. Salam and J. Strathdee, {\textit{Phys. Rev.}} {\bf{D11}}, 1521 (1975); {\textit{Nucl. Phys.}} {\bf{B86}}, 142 (1975); 
    D. M. Capper, {\textit{Nuovo Cimento}} {\bf{25A}}, 259 (1975); R. Delbourgo, {\textit{Nuovo Cimento}} {\bf{25A}}, 646 (1975); D. M. 
    Capper and G. Leibrandt, {\textit{Nucl. Phys.}} {\bf{B85}}, 492 (1975); F. Krause, M. Scheunert, J. Honerkamp, and M. Schlindwein, 
    {\textit{Phys. Lett.}} {\bf{53B}}, 60 (1974); K. Fujikawa and W. Lang, {\textit{Nucl. Phys.}} {\bf{B88}}, 61 (1975); J. Honerkamp, 
    M. Schlindwein, F. Krause, and M. Scheunert, {\textit{Nucl, Phys.}} {\bf{B95}}, 397 (1975); S. Ferrara and O. Piguet, {\textit{Nucl. 
    Phys.}} {\bf{B93}}, 261 (1975); R. Delbourgo, {\textit{J. Phys.}} {\bf{G1}}, 800 (1975). 这个形式体系被\,W. Siegel\,推广至超引力, {\textit{Phys. Lett.}} {\bf{84B}}, 197 (1979).
    \bibitem{2} M. T. Grisaru, W. Siegel, and M. Ro\v{c}ek, {\textit{Nucl. Phys.}} {\bf{B159}}, 429 (1979).
\end{thebibliography}



%第三十一章 超引力
\chapter{超引力} \label{cha:31}

引力确实存在, 所以如何超对称性是真实的, 那么现实世界的任何超对称理论最终都必须被扩展成物质和引力的超对称理论, 这样的理论称为{\kai{超引力}}(supergravity). 尽管没有超引力的超对称形在能量远低于\,Planck\,能标时可能是一个好近似, 但它不是真实世界的可能理论.

构建超引力的理论主要有两种方法. 首先, 超引力可以作为超空间的理论.\cite{1} 这个方法类似于\,\ref{sec:27.1}---\ref{sec:27.3}\,节发展的超对称规范理论; 引力场可以视为一个超场的一个分量, 这个超场既有非物理分量又有物理分量, 就像规范超场$\,V\,$拥有非物理的$\,C$, $M$, $N\,$和$\,\omega\,$分量. 以这种方法推导整个非线性超引力理论的复杂性让人生畏, 而且迄今为止, 按照这种思路的推导无法摆脱显然任意的步骤. 在推导的某个节点, 不得不要声明对引力子超场采取的某些约束是合理正确的.

这里我们将使用第二种方法, 与第一种方法相比, 这种方法不那么优雅但更加易懂.\cite{2} 在我们这里的讨论中, 我们先在\,\ref{sec:31.1}---\ref{sec:31.5}\,节考虑引力场较弱的情况,\cite{3} 用我们在第\,\ref{cha:26}\,和\,\ref{cha:27}\,章用来研究普通超对称理论的平坦空间超场方法来分析超引力. 以这种方法, 我们可以识别出引力超场的物理分量(包括与规范超场$\,V\,$的$\,D$-分量类似的辅助场.) 弱场近似将允许我们获得超引力理论中最重要的一些结果, 包括\,\ref{sec:31.3}\,节中引力微子质量的一般公式, 以及\,\ref{sec:31.4}\,节中由反常引导的超对称破缺给出的引力微子质量, $A\,$和$\,B\,$参量.

在\,\ref{sec:31.6}\,节, 我们给物理场的超对称变换规则以及描述它们相互作用的拉格朗日量加上$\,G\,$的高阶项, 这些项要使得超对称变换和广义坐标变换构成封闭的代数, 并使得拉格朗日量在这些变换下不变. 这个方法有些像\,\ref{sec:27.8}\,节中对超对称规范理论的处理; 我们仅处理引力超场的物理分量, 获得的变换规则包含协变导数而不是普通导数. 推导超引力理论在超出弱场近似以外的最重要结果------在引力引导的超对称破缺理论中推导低能有效作用量------时所使用的就是这个方法. 这个应用是\,\ref{sec:31.7}\,节的主题.

\section{度规超场} \label{sec:31.1}

超引力必然会包含旋量场和张量场, 所以我们必须要用标架(或称四元基)$\,e^{a}{}_{\mu}(x)\,$而不是度规来描述引力场, 度规与标架之间的关系是
\begin{equation}
g_{\mu\nu}(x)=\eta_{ab}\,e^{a}{}_{\mu}(x)e^{b}{}_{\nu}(x)\:.\label{31.1.1}
\end{equation}
指标$\,\mu,\nu\,$等标记广义坐标, 而指标$\,a,b\,$等则标记定域惯性坐标系中的坐标, $\eta_{ab}\,$是通常的对角矩阵, 对角元是$\,+1,+1,+1,-1$. 在标架体系中, 要求作用量在两类不同的对称变换下不变: 一是广义坐标变换$\,x^{\mu}\to x^{\prime\mu}(x)$, 在这个变换下, 标架$\,e^{a}{}_{\mu}(x)\,$变换到$\,e^{\prime a}{}_{\mu}(x)$, 其中
\begin{equation}
e^{\prime a}{}_{\mu}(x)=\frac{\partial x^{\nu}}{\partial x^{\prime \mu}}e^{a}{}_{\nu}(x)\:,\label{31.1.2}
\end{equation}
另一个是定域\,Lorentz\,变换, 在这个变换下,
\begin{equation}
e^{a}{}_{\mu}(x) \to \Lambda^{a}{}_{b}(x)\,e^{b}{}_{\mu}(x) \:,\label{31.1.3}
\end{equation}
其中$\,\Lambda^{a}{}_{b}(x)\,$是满足如下约束的实矩阵
\begin{equation}
\eta_{ab}\Lambda^{a}{}_{c}(x)\Lambda^{b}{}_{d}(x)=\eta_{cd}\:.\label{31.1.4}
\end{equation}
本章附录会给出这一体系的一个基本回顾.

弱引力场是标架接近单位矩阵的引力场. 在这种场中, 标架可以方便地写成
\begin{equation}
e^{a}{}_{\mu}(x) =\delta^{a}_{\mu}+2\kappa\,\phi^{a}{}_{\mu}(x)  \:,\label{31.1.5}
\end{equation}
其中$\,\phi^{a}{}_{\mu}(x)\,$是一小量. 我们会在\,\ref{sec:31.2}\,节看到, 如果$\,\phi^{a}{}_{\mu}\,$是按惯例归一化的场, 那么常数$\,\kappa\,$就应通过$\,\kappa=\sqrt{8\uppi G}\,$表示成牛顿常数$\,G\,$. 在非常小的坐标变换下
\begin{equation}
x^{\mu}\to x^{\mu}+\xi^{\mu}(x)\:, \label{31.1.6}
\end{equation}
以及非常小的\,Lorentz\,变换下
\begin{equation}
\Lambda^{a}{}_{b}(x)=\delta^{a}_{b}+\omega^{a}{}_{b}(x)\:, \label{31.1.7}
\end{equation}
标架场在单位矩阵附近这一点是受到保护的, 这里的$\,\xi^{\mu}(x)\,$和$\,\omega^{a}{}_{b}(x)\,$在量级上与$\,\phi^{a}{}_{\mu}(x)\,$相同, 由于方程(\ref{31.1.4})的限制, $\omega_{ab}(x)\equiv \eta_{ac}\omega^{c}{}_{b}(x)\,$要满足
\begin{equation}
\omega_{ab}(x)=-\omega_{ba}(x)\:.\label{31.1.8}
\end{equation}
这样, (\ref{31.1.2})和(\ref{31.1.3})的组合变换就变成
\begin{equation}
\phi_{\mu\nu}(x)\to\phi_{\mu\nu}(x)+\frac{1}{2\kappa}
\biggl[-\frac{\partial \xi_{\mu}(x)}{\partial x^{\nu}}+\omega_{\mu\nu}(x)
\biggr]\:.\label{31.1.9}
\end{equation}
现在, 我们不在广义坐标指标$\,\mu,\nu\cdots$和定域\,Lorentz\,坐标指标$\,a,b\cdots$之间做区分, 用$\,\eta^{\mu\nu}\,$和$\,\eta_{\mu\nu}\,$升降所有指标. 表示成度规(\ref{31.1.1}), 弱场假设(\ref{31.1.5})变成
\begin{equation}
g_{\mu\nu}(x)=\eta_{\mu\nu}+2\kappa\,h_{\mu\nu}(x)\:,\label{31.1.10}
\end{equation}
其中
\begin{equation}
h_{\mu\nu}(x)\equiv\phi_{\mu\nu}(x)+\phi_{\nu\mu}(x)\:,\label{31.1.11}
\end{equation}
而变换法则(\ref{31.1.9})变成
\begin{equation}
h_{\mu\nu}(x)\to h_{\mu\nu}(x)-\frac{1}{2\kappa}
\biggl[\frac{\partial \xi_{\mu}(x)}{\partial x^{\nu}}+\frac{\partial \xi^{\nu}(x)}{\partial x_{\mu}}
\biggr]\:.\label{31.1.12}
\end{equation}

通过使用超对称代数, 我们在\,\ref{sec:25.4}\,节证明了引力子有费米超对称伴, 引力微子, 这个粒子的螺旋度是$\,\pm3/2\,$. 我们在\,5.9\,节看到, 对于螺旋度为$\,\pm1\,$的自荷共轭粒子, 仅当它由实场$\,A_{\mu}(x)\,$描述且它的相互作用在规范变换$\,A_{\mu}(x)\to A_{\mu}(x)+\partial_{\mu}\Lambda(x)$下不变时, 它才能有低能相互作用. 以相同的方法, 对于螺旋度为$\,\pm3/2\,$的自荷共轭粒子, 若要使得它有低能相互作用, 它要由带额外指标$\,\mu\,$的 Majorana 场$\,\psi_{\mu}(x)\,$描述, 且相互作用要在如下规范变换下不变
\begin{equation}
\psi_{\mu}(x)\to\psi_{\mu}(x)+\partial_{\mu}\psi(x)\:,\label{31.1.13}
\end{equation}
其中$\,\psi(x)\,$是任意的\,Majorana\,场.\cite{3a} 我们现在需要考虑的是: 如何将有上述变换性质的$\,\phi_{\mu\nu}(x)\,$场和$\,\psi_{\mu}(x)\,$场放进一个超场中.

我们在\,\ref{sec:27.1}\,节看到, 规范场$\,V_{\nu}(x)\,$可以看成是实标量超场$\,V(x,\theta)\,$的$\,V_{\nu}$-分量, 按方程(\ref{26.2.10}) 的定义, 它就是$\,\mi(\bar{\theta}\gamma_{5}\gamma_{\nu}\theta)/2\,$的系数. 类似地, 我们希望将标架场$\,\phi_{\mu\nu}(x)\,$和引力微子场$\,\psi_{\mu}(x)\,$放进一个{\kai{矢量}}超场$\,H_{\mu}(x,\theta)\,$中, 这个超场称为{\kai{度规超场}}. 问题是: 如何将$\,\phi_{\mu\nu}(x)\,$和$\,\psi_{\mu}(x)\,$与这个超场的分量关联起来.

为了解决这一问题, 注意到超对称性要求``规范''变换(\ref{31.1.9})和(\ref{31.1.13})是整个度规超场的变换
\begin{equation}
H_{\mu }(x,\theta )\to H_{\mu }(x,\theta )+\Delta _{\mu }(x,\theta )  \label{31.1.14}
\end{equation}%
的特殊情况. 更进一步, 正如我么在\,\ref{sec:26.7}\,节中所看到的, 与一个{\kai{弱}}引力场$\,h_{\mu\nu}\,$相互作用的能动量张量$\,T^{\mu\nu}\,$是一个实矢量超场$\,\Theta^{\mu}\,$的分量的线性组合, 所以我们期待整个超场$\,H_{\mu}\,$与物质的相互作用是如下的形式
\begin{equation}
I_{\mathrm{int}}=2\kappa \int \dif^{4}x\:\Bigl[ H_{\mu }\Theta ^{\mu }\Bigr]_{D}\:.  \label{31.1.15}
\end{equation}%
(本节后面我们将证实当$\,\Theta^{\mu}\,$像\,\ref{sec:26.7}\,节中那样归一化时, 系数$\,2\kappa\,$是正确的.) \ref{sec:26.7}\,节证明了超流$\,\Theta^{\mu}$ 满足守恒条件
\begin{equation}
\gamma ^{\mu }\mathscr{D}\Theta _{\mu }=\mathscr{D}X\:,  \label{31.1.16}
\end{equation}%
其中$\,X\,$是一个实标量超场(左手征标量超场与其复共轭的和)而$\,\mathscr{D}\,$是\,4-分量超导数(\ref{26.2.26}). 由此可以得出, 当$\,\Delta_{\mu}\,$形如
\begin{equation}
\Delta _{\mu }=\Bigl( \bar{\mathscr{D}}\gamma _{\mu }\Xi \Bigr) \:,\label{31.1.17}
\end{equation}%
其中超场$\,\Xi\,$满足显式超对称条件
\begin{equation}
\Bigl( \bar{\mathscr{D}}\mathscr{D}\Bigr) \Bigl( \bar{\mathscr{D}}\Xi \Bigr) =0\:,  \label{31.1.18}
\end{equation}%
这个相互作用在形如(\ref{31.1.14})的变换下不变. 想要看到这点, 回忆起在\,\ref{sec:26.7}\,节, $X\,$上的手征条件使得我们可以将其写成
\begin{equation}
X=\Bigl( \bar{\mathscr{D}}\mathscr{D}\Bigr) \Omega \:,  \label{31.1.19}
\end{equation}%
其中$\,\Omega\,$一般是非定域超场. 那么, 从方程(\ref{31.1.16})我们发现
\begin{align*}
\int \Bigl[ \Theta ^{\mu }\Bigl( \bar{\mathscr{D}}\gamma _{\mu }\Xi \Bigr) %
\Bigr] _{D} &=-\int \Bigl[ \Bigl( (\bar{\mathscr{D}}\Theta ^{\mu })\gamma
_{\mu }\Xi \Bigr) \Bigr] _{D}=-\int \Bigl[ \Bigl( (\bar{\mathscr{D}}X)\Xi
\Bigr) \Bigr] _{D} \\
&=+\int \Bigl[ \Omega \Bigl( \bar{\mathscr{D}}\mathscr{D}\Bigr) \Bigl( %
\bar{\mathscr{D}}\Xi \Bigr) \Bigr] _{D}=0\:,
\end{align*}%
进而使得相互作用(\ref{31.1.15})在变换$\,H_{\mu}\to H_{\mu}+(\bar{\mathscr{D}}\gamma_{\mu}\Xi)\,$下不变. (不使用$\,X\,$的表示(\ref{31.1.19}), 我们也可以得到相同结果, 方法是从方程(\ref{26.2.25})注意到, 像$\,X\,$这样满足方程(\ref{26.3.1})和(\ref{26.3.2})的手征超场与像$\,(\mathscr{D}\Xi)\,$这样满足方程(\ref{26.3.45})的线性超场, 它们乘积的$\,D$-项是时空导数.)

在本节末尾我们将证明方程(\ref{31.1.17})和(\ref{31.1.18})给出超场$\,\Delta_{\mu}\,$分量上的条件:
\begin{equation}
V_{\mu \nu }^{\Delta }(x)+V_{\nu \mu }^{\Delta }(x)=\frac{\partial v_{\mu
}(x)}{\partial x^{\nu }}+\frac{\partial v_{\nu }(x)}{\partial x^{\mu }}%
-2\eta _{\mu \nu }\frac{\partial v^{\lambda }(x)}{\partial x^{\lambda }}\:, \label{31.1.20}
\end{equation}%
\begin{equation}
\lambda _{\mu }^{\Delta }(x)-\tfrac{1}{3}\gamma _{\mu }\gamma ^{\rho }\lambda
_{\rho }^{\Delta }(x)-\tfrac{1}{3}\gamma _{\mu }\partial ^{\rho }\omega
_{\rho }^{\Delta }(x)=\partial _{\mu }\chi (x)\:,  \label{31.1.21}
\end{equation}%
\begin{equation}
-\tfrac{1}{2}\epsilon ^{\nu \mu \kappa \sigma }\partial _{\kappa }V_{\nu \mu
}^{\Delta }(x)=D^{\Delta \,\sigma }(x)+\partial ^{\sigma }\partial ^{\rho
}C_{\rho }^{\Delta }(x)\:,  \label{31.1.22}
\end{equation}%
\begin{equation}
\partial ^{\mu }M_{\mu }^{\Delta }(x)=\partial ^{\mu }N_{\mu }^{\Delta }(x)=0 \:,  \label{31.1.23}
\end{equation}
其中$\,v_{\mu}(x)\,$是实矢量场, 而$\,\chi(x)\,$是\,Majorana\,旋量场. (这里我们将使用一个贯穿本章始终的符号约定; 跟随方程(\ref{26.3.9}), 任意超场$\,S(x,\theta)\,$的分量$\,C^{S}$, $\omega^{S}$, $M^{S}$, $N^{S}$, $V_{\nu}^{S}$, $\lambda^{S}\,$和$\,D^{S}\,$由展开
\begin{align}
S(x,\theta ) &=C^{S}(x)-\mi\Bigl( \bar{\theta}\,\gamma _{5}\,\omega
^{S}(x)\Bigr) -\frac{\mi}{2}\Bigl( \bar{\theta}\,\gamma _{5}\,\theta \Bigr)
M^{S}(x)-\frac{1}{2}\Bigl( \bar{\theta}\,\theta \Bigr) N^{S}(x)  \nonumber \\
&\quad+\frac{\mi}{2}\Bigl( \bar{\theta}\,\gamma_{5}\,\gamma^{\nu }\,\theta \Bigr)
V_{\nu }^{S}(x)-\mi\Bigl( \bar{\theta}\,\gamma _{5}\,\theta \Bigr) \left( \bar{%
\theta}\Bigl[ \lambda ^{S}(x)+\frac{1}{2}\slashed{\partial} \omega ^{S}(x)\Bigr]
\right)   \nonumber \\
&\quad-\frac{1}{4}\Bigl( \bar{\theta}\,\gamma _{5}\,\theta \Bigr)^{2}\left(
D^{S}(x)+\frac{1}{2}\square C^{S}(x)\right) \label{31.1.24}
\end{align}
定义. 另外, $V_{\mu\nu}^{\Delta}(x)\,$是$\,\Delta_{\mu}\,$的$\,V_{\nu}$-分量.) 这引领我们定义场
\begin{equation}
\phi_{\mu\nu}(x)\equiv V^{H}_{\mu\nu}(x)
-\tfrac{1}{3}\eta_{\mu\nu}\,V^{H\,\lambda}{}_{\lambda}(x)\:,\label{31.1.25}
\end{equation}
\begin{equation}
\tfrac{1}{2}\psi_{\mu}(x)\equiv \lambda_{\mu}^{H}(x)-\tfrac{1}{3}\gamma_{\mu}\gamma^{\rho}\lambda_{\rho}^{H}(x)
-\tfrac{1}{3}\gamma_{\mu}\partial^{\rho}\omega_{\rho}^{H}(x)\:,\label{31.1.26}
\end{equation}
\begin{equation}
b^{\sigma}(x)\equiv D^{H\,\sigma}(x)+\tfrac{1}{2}\epsilon^{\nu\mu\kappa\sigma}\partial_{\kappa}V^{H}_{\nu\mu}(x)
+\partial^{\sigma}\partial^{\rho}C_{\rho}^{H}(x)\:.\label{31.1.27}
\end{equation}
(正如我们将在\,\ref{sec:31.3}\,节讨论的, 在方程(\ref{31.1.26})左边引入因子$\,1/2\,$后给出的场$\,\psi_{\mu}\,$是按惯例归一化的.) 从方程(\ref{31.1.20})和(\ref{31.1.21})可以得出, 变换(\ref{31.1.13})在$\,\phi_{\mu\nu}(x)\,$和$\,\psi_{\mu}(x)\,$上诱导出了规范变换 (\ref{31.1.9})和(\ref{31.1.13}), 其中
\begin{equation}
\xi_{\mu}=-2\kappa\,v_{\mu}\:,\qquad
\omega_{\mu\nu}=\kappa\biggl[-\frac{\partial v_{\mu}}{\partial x^{\nu}}+
\frac{\partial v_{\nu}}{\partial x^{\mu}}+V_{\mu\nu}^{\Delta}-V_{\nu\mu}^{\Delta}\biggr]\:,\qquad
\psi=2\chi\:,\label{31.1.28}
\end{equation}
而方程(\ref{31.1.22})表明$\,b_{\mu}(x)\,$是不变量. 另外, 方程(\ref{31.1.23})表明变换(\ref{31.1.14})诱导出的$\,M_{\mu}^{H}(x)\,$和 $N_{\mu}(x)\,$的偏移将保持场
\begin{equation}
    s\equiv \partial^{\mu}M_{\mu}^{H}(x) \:, \qquad p\equiv \partial^{\mu}N_{\mu}^{H}(x)  \label{31.1.29}
\end{equation}
不变. 最后, 由于$\,C^{\Delta}_{\mu}(x)$, $V_{\mu\nu}^{\Delta}(x)-V_{\nu\mu}^{\Delta}(x)\,$和$\,\omega_{\mu}^{\Delta}(x)\,$不受到超对称性的约束, 变换(\ref{31.1.14})使得我们可以将分量$\,C_{\mu}^{H}(x)$, $V_{\mu\nu}^{H}(x)-V_{\nu\mu}^{H}(x)=\phi_{\mu\nu}-\phi_{\nu\mu}\,$和$\,\omega_{\mu}^{H}(x)\,$取成任何我们想要的值. 特别地, 类比\,\ref{sec:27.1}\,节中讨论的规范超场的\,Wess--Zumino\,规范, 我们可以取$\,C_{\mu}^{H}(x)$, $V_{\mu\nu}^{H}(x)-V_{\nu\mu}^{H}(x)=\phi_{\mu\nu}-\phi_{\nu\mu}\,$和$\,\omega_{\mu}^{H}(x)\,$全为零. 根据场$\,h_{\mu\nu}(x)\,$和$\,\psi_{\mu}(x)\,$的变换形式可以看出它们分别是引力子和引力微子, 而$\,b_{\mu}(x)$, $s(x)\,$和$\,p(x)\,$是辅助场\cite{4}, 这些场在理解引力超场与物质的耦合时非常重要.

另外提一句, 在模掉规范变换(\ref{31.1.12})后, 对称张量$\,h_{\mu\nu}\,$的独立分量个数是$\,10-4=6$, 再加上辅助场$\,s$, $p\,$和$\,b_{\mu}\,$就给出总数为$\,6+6=12\,$的物理玻色场, 而\,Majorana\,旋量场$\,\psi_{\mu}\,$在模掉规范变换(\ref{31.1.13})之后的独立物理分量个数是$\,16-4=12$. 这满足\,\ref{sec:26.2}\,节末尾讨论的条件, 即如果场的任何超多重态构成了超对称代数的一个表示, 那么独立玻色场分量的个数与独立费米场的个数必相等.

现在我们回到物质与引力的相互作用. 一般而言, 两个超场$\,\Theta^{\mu}\,$和$\,H_{\mu}\,$之积的$\,D$-项由(\ref{26.2.25}) 给出
\begin{align}
    \int \dif^{4}x\:\Bigl[\Theta^{\mu}H_{\mu}\Bigr]_{D} &= \int\dif^{4}x\:
    \biggl[-\partial_{\mu}C^{H\,\sigma}\,\partial^{\mu}C_{\sigma}^{\Theta} +C^{H\,\sigma}\,D_{\sigma}^{\Theta}
    +D^{H\,\sigma}\,C_{\sigma}^{\Theta} \nonumber \\
    &\quad \Bigl(\overline{\omega}^{H\,\sigma}\,[\lambda_{\sigma}^{\Theta}+
    \tfrac{1}{2}\slashed{\partial}\omega_{\sigma}^{\Theta}]\Bigr)
    -\Bigl( [\overline{\lambda}^{H\,\sigma}+\tfrac{1}{2}\overline{\omega}^{H\,\sigma}\,\slashed{\partial}]
    \omega_{\sigma}^{\Theta}\Bigr) \nonumber \\
    &\quad +M^{H\,\sigma}\,M_{\sigma}^{\Theta} +N^{H\,\sigma}\,N_{\sigma}^{\Theta}
    -V^{H\,\kappa\sigma}\,V_{\kappa\sigma}^{\Theta} \biggr] \:. \label{31.1.30}
\end{align}
利用方程(\ref{31.1.25})---(\ref{31.1.27}), 我们可以把$\,V_{\mu\nu}^{H}$, $\lambda_{\mu}^{H}\,$和$\,D_{\mu}^{H}\,$分别表示成$\,\phi_{\mu\nu}$, $\psi_{\mu}\,$和$\,b_{\mu}$, 并发现
\begin{align}
    I_{\text{int}} &= 2\kappa \int \dif^{4}x\:\Bigl[\Theta^{\mu}H_{\mu}\Bigr]_{D} \nonumber \\
    &=2\kappa\int\dif^{4}x\:\biggl[ C^{H\,\sigma}\,\Bigl[\square C_{\sigma}^{\Theta}
    -\partial_{\sigma}\partial^{\rho}C_{\rho}^{\Theta}+ D_{\sigma}^{\Theta}\Bigr]+b^{\sigma}C_{\sigma}^{\Theta}\nonumber \\
    &\quad + \Bigl(\overline{\omega}^{H\,\sigma}\,\Bigl[-\lambda_{\sigma}^{\Theta}-\slashed{\partial}\omega_{\sigma}^{\Theta}
    +\partial_{\sigma}\gamma^{\rho}\omega_{\rho}^{\Theta}\Bigr]\Bigr) -\Bigl(\bar{\psi}^{\sigma}\,\omega_{\sigma}^{\Theta}\Bigr)
    +\Bigl(\bar{\psi}^{\sigma}\,\gamma_{\sigma}\gamma^{\rho}\omega_{\rho}^{\Theta}\Bigr) \nonumber \\
    &\quad + M^{H\,\sigma}\, M_{\sigma}^{\Theta} + N^{H\,\sigma}N_{\sigma}^{\Theta} \nonumber \\
    &\quad +\tfrac{1}{2}\epsilon^{\nu\mu\kappa\sigma}\,\phi_{\nu\mu}\partial_{\kappa}C_{\sigma}^{\Theta}
    -\phi^{\lambda\sigma}\Bigl[V_{\lambda\sigma}^{\Theta}-\eta_{\lambda\sigma}\,V^{\Theta}{}\indices{^\rho_\rho}\Bigr]
    \biggr] \:. \label{31.1.31}
\end{align}
我们在\,\ref{sec:26.7}\,节看到, 守恒条件(\ref{31.1.16})给出了条件(\ref{26.7.44}), (\ref{26.7.39})和(\ref{26.7.35}):
\begin{align*}
    D_{\mu}^{\Theta} &= -\square C_{\mu}^{\Theta} + \partial_{\mu}\partial^{\nu}C_{\nu}^{\Theta} \:, \\
    \lambda_{\nu}^{\Theta} &= -\slashed{\partial}\omega_{\nu}^{\Theta}+\partial_{\nu}\gamma^{\mu}\omega_{\mu}^{\Theta} \:, \\
    0 &= V_{\mu\nu}^{\Theta} - V_{\nu\mu}^{\Theta}+\epsilon_{\mu\nu\rho\sigma}\partial^{\sigma}C^{\Theta\,\rho} \:,
\end{align*}
这分别告诉我们方程(\ref{31.1.31})中$\,C^{H\sigma}$, $\overline{\omega}^{H\sigma}\,$以及$\,\phi^{\mu\nu}\,$的反对称部分的系数全为零. 另外, 我们可以将方程(\ref{31.1.31})中剩下的项表示成超流(\ref{26.7.20}), 能动量张量(\ref{26.7.42}), $\mathscr{R}$-流(\ref{26.7.51}):
\begin{align*}
    S^{\mu} &= -2\omega^{\Theta\,\mu} + 2\gamma^{\mu}\gamma^{\nu}\,\omega_{\nu}^{\Theta} \:, \\
    T_{\mu\nu} &= -\tfrac{1}{2}V_{\mu\nu}^{\Theta} -\tfrac{1}{2}V_{\nu\mu}^{\Theta}
    +\eta_{\mu\nu}\,V^{\Theta}{}\indices{^\lambda_\lambda} \:, \\
    \mathscr{R}^{\mu} &= 2\,C^{\Theta\,\mu} \:,
\end{align*}
而
\begin{equation}
    M_{\mu}^{\Theta} =\partial_{\mu}\mathscr{M} \:, \qquad N_{\mu}^{\Theta} =\partial_{\mu}\mathscr{N}  \label{31.1.32}
\end{equation}
定义了密度$\,\mathscr{M}\,$和$\,\mathscr{N}$, 方程(\ref{26.7.33})和(\ref{26.7.34})给出了:
\begin{equation}
    \mathscr{N} = -A^{X} \:, \qquad \mathscr{M}=B^{X} \:, \label{31.1.33}
\end{equation}
其中$\,X\,$出现在守恒方程(\ref{31.1.16})右边的实手征超场. (本节将理解超对称流上的标记``new''.) 这样, 物质与度规超场分量之间的一阶相互作用就是
\begin{equation}
    2\kappa\int\dif^{4}x\:\Bigl[\Theta^{\mu}H_{\mu}\Bigr]_{D} = \kappa \int \dif^{4}x\:
    \Bigl[\mathscr{R}_{\sigma}b^{\sigma} + \tfrac{1}{2}\bar{S}^{\sigma}\psi_{\sigma}-
    2\mathscr{M}\,s -2\mathscr{N}\,p + T^{\kappa\sigma}h_{\sigma\kappa} \Bigr] \:. \label{31.1.34}
\end{equation}
我们注意到引力微子场与超对称流的相互作用方式与引力场和能动量张量的相互作用方式非常类似.

我们现在可以检验这个相互作用中出现的常数因子. $T^{\mu\nu}\,$通常定义成物质作用量对度规变化$\,\delta g_{\mu\nu}\,$的变分\cite{5}
\[
\delta I_{M} = \frac{1}{2}\int \dif^{4}x\:\sqrt{\operatorname{Det}g}\,T^{\mu\nu}\,\delta g_{\mu\nu} \:.
\]
这样, 物质与一个弱引力场的相互作用就由方程(\ref{31.1.10})给定为
\[
\kappa \int\dif^{4}x\: T^{\mu\nu}(x)\,h_{\mu\nu}(x) \:,
\]
这与方程(\ref{31.1.34})中与$\,h_{\mu\nu}\,$相关的部分一致, 因此证实了相互作用(\ref{31.1.15})的归一化.

\subsection*{* * *}

我们现在将验证方程(\ref{31.1.17})和(\ref{31.1.18})给出了条件(\ref{31.1.20})---(\ref{31.1.23}). 超场$\,\Upsilon\equiv \Bigl(\bar{\mathscr{D}}\Xi\Bigr)$有分量
\begin{align*}
    C^{\Upsilon} &= -\mi\operatorname{Tr}(\epsilon \omega^{\Xi}) \:, \\
    \omega^{\Upsilon} &= -\mi\gamma_{5}\,\slashed{\partial}C^{\Xi} + M^{\Xi}
    - \mi\gamma_{5}N^{\Xi} + \slashed{V}^{\Xi} \:, \\
    M^{\Upsilon} &= -\operatorname{Tr}(\epsilon\gamma_{5}\lambda^{\Xi}) \:, \\
    N^{\Upsilon} &= \mi\operatorname{Tr}(\epsilon \lambda^{\Xi}) \:, \\
    V_{\nu}^{\Upsilon} &= -\operatorname{Tr}(\epsilon\gamma_{5}\lambda_{\nu}\lambda^{\Xi})
    -\tfrac{1}{2}\operatorname{Tr}(\epsilon\gamma_{5}[\gamma_{\nu},\slashed{\partial}]\lambda^{\Xi}) \:, \\
    \lambda^{\Upsilon} &= -\slashed{\partial}M^{\Xi} -\mi\gamma_{5}\,\slashed{\partial}N^{\Xi}
    -\mi\gamma_{5}(D^{\Xi}+\square C^{\Xi}) - \partial_{\nu}V^{\Xi\,\nu} \:, \\
    D^{\Upsilon} &= \mi\operatorname{Tr}(\epsilon\,\slashed{\partial}\lambda^{\Xi})
    +\mi\operatorname{Tr}(\epsilon\square\omega^{\Xi}) \:,
\end{align*}
而条件(\ref{31.1.18})给出
\[
 M^{\Upsilon}=  N^{\Upsilon} =  D^{\Upsilon} + \square C^{\Upsilon} = \partial^{\lambda}V_{\lambda}^{\Upsilon}
 =\lambda^{\Upsilon} + \slashed{\partial}\omega^{\Upsilon} = 0 \:.
\]
$M^{\Upsilon}\,$和$\,N^{\Upsilon}\,$为零告诉我们$\,\lambda^{\Theta}\,$是形如
\begin{equation}
    \lambda^{\Xi}\epsilon = f_{\mu}\gamma^{\mu} + \gamma_{\mu}\gamma_{5}\gamma^{\mu}+
    k_{\mu\nu}\,[\gamma^{\mu},\gamma^{\nu}]  \label{31.1.35}
\end{equation}
的线性组合. 这样, $\partial^{\lambda}V_{\lambda}^{\Upsilon}\,$和$\,D^{\Upsilon}+\square C^{\Upsilon}\,$为零就给出了
\begin{equation}
    \partial_{\mu}f^{\mu} = \partial_{\mu} g^{\mu}=0 \:. \label{31.1.36}
\end{equation}
另外, $\lambda^{\Upsilon}+\slashed{\partial}\omega^{\Upsilon}\,$为零给出
\begin{equation}
    D^{\Xi} = \frac{\mi}{2}\gamma_{5}\,[\gamma^{\nu},\slashed{\partial}]\,V_{\nu}^{\Xi} \:. \label{31.1.37}
\end{equation}
$\Delta_{\mu}(x,\theta)\,$的分量是
\begin{align}
    C^{\Delta}_{\mu} &= \mi\operatorname{Tr}(\epsilon\gamma_{5}\gamma_{\mu}\omega^{\Xi}) \:, \label{31.1.38} \\
    \omega^{\Delta}_{\mu} &= \mi\,\slashed{\partial}\gamma_{\mu}C^{\Xi} + \gamma_{5}\gamma_{\mu}M^{\Xi}
    - \mi\gamma_{\mu}N^{\Xi} + \gamma_{5}\gamma^{\rho}\gamma_{\mu} V^{\Xi}_{\rho}  \:,\label{31.1.39} \\
    M^{\Delta}_{\mu} &= -\operatorname{Tr}(\epsilon\gamma_{\mu}\lambda^{\Xi}) \:, \label{31.1.40} \\
    N^{\Delta}_{\mu} &= \mi\operatorname{Tr}(\epsilon\gamma_{5}\gamma_{\mu} \lambda^{\Xi}) \:, \label{31.1.41} \\
    V_{\mu\nu}^{\Delta} &= \operatorname{Tr}(\epsilon\gamma_{\nu}\gamma_{\mu}\lambda^{\Xi})
    -\tfrac{1}{2}\operatorname{Tr}(\epsilon\,[\gamma_{\nu},\slashed{\partial}]\,\gamma_{\mu}\omega^{\Xi}) \:,\label{31.1.42} \\
    \lambda^{\Delta}_{\mu} &= \gamma_{5}\,\slashed{\partial}\gamma_{\mu}M^{\Xi}
    +\mi\,\slashed{\partial}\gamma_{\mu} N^{\Xi}
    -\mi\gamma_{\mu}(D^{\Xi}+\square C^{\Xi}) + \gamma_{5}\gamma_{\mu}\partial^{\nu}V^{\Xi}_{\nu} \:,\label{31.1.43} \\
    D^{\Delta}_{\mu} &= \mi\operatorname{Tr}(\epsilon\,\slashed{\partial}\gamma_{5}\gamma_{\mu}\lambda^{\Xi})
    +\mi\,\square \operatorname{Tr}(\epsilon\gamma_{5}\gamma_{\mu}\omega^{\Xi}) \:. \label{31.1.44}
\end{align}
方程(\ref{31.1.42})的对称部分给出了条件(\ref{31.1.20}), 其中
\begin{equation}
    v_{\mu}= -\operatorname{Tr}(\epsilon\gamma_{\mu}\omega^{\Xi}) + \text{常数} \:. \label{31.1.45}
\end{equation}
方程(\ref{31.1.39})和(\ref{31.1.43})的一个线性组合给出了条件(\ref{31.1.21}), 其中
\begin{equation}
    \xi = 2\gamma_{5}M^{\Xi} + 2\mi\,N^{\Xi} + \text{常数} \:. \label{31.1.46}
\end{equation}
我们然后使用方程(\ref{31.1.42})的反对称部分, 连同方程(\ref{31.1.37}), (\ref{31.1.38}), 以及恒等式
\[
[\gamma_{\nu},\gamma_{\rho}] - [\gamma_{\mu},\gamma_{\rho}]\gamma_{\nu}
= 2\eta_{\mu\rho}\gamma_{\nu} - 2\eta_{\nu\rho}\gamma_{\mu} + 2\mi\epsilon_{\nu\rho\mu\lambda}\,\gamma_{5}\gamma^{\lambda}\:,
\]
\[
\epsilon^{\nu\mu\kappa\sigma}[\gamma_{\nu},\gamma_{\mu}] = 2\mi\gamma_{5}[\gamma^{\kappa},\gamma^{\sigma}] \:,
\]
这样就发现了条件(\ref{31.1.22}), 最后, 方程(\ref{31.1.40})和(\ref{31.1.41})连同方程(\ref{31.1.35})和(\ref{31.1.36})给出了条件(\ref{31.1.23}).

\section{引力作用量} \label{sec:31.2}

为了找到一个合适的引力作用量, 我们必须要构建一个在推广了的规范变换$\,H_{\mu}\to H_{\mu}+ \Delta_{\mu}\,$下不变的超场. 作为出发点, 我们回忆起方程(\ref{31.1.27})定义的场$\,b_{\mu}\,$场在这个规范变换下不变. 通过连续做超对称变换, 我们可以看到$\,b_{\mu}\,$是``Einstein''超场$\,E_{\mu}\,$的$\,C$-分量, 它的各个分量是
\begin{align}
    C_{\mu}^{E} &= b_{\mu}  \:, \label{31.2.1} \\
    \omega_{\mu}^{E} &= \frac{3}{2}L_{\mu} - \frac{1}{2}\gamma_{\mu}\gamma^{\nu}L_{\nu} \:, \label{31.2.2} \\
    M_{\mu}^{E} &= \partial_{\mu}s\:, \qquad N_{\mu}^{E} = \partial_{\mu}p \:, \label{31.2.3} \\
    V_{\mu\nu}^{E} &= -\frac{3}{2}E_{\mu\nu}+\frac{1}{2}\eta_{\mu\nu}E^{\rho}{}_{\rho}
    +\frac{1}{2}\epsilon_{\nu\mu\sigma\rho}\partial^{\sigma}b^{\rho} \:, \label{31.2.4} \\
    \lambda_{\mu}^{E} &= \partial_{\mu}\gamma^{\nu}\,\omega_{\nu}^{E} - \slashed{\partial}\omega_{\mu}^{E}\:,\label{31.2.5} \\
    D_{\mu}^{E} &= \partial_{\mu}\partial^{\nu}b_{\nu} - \square b_{\mu} \:, \label{31.2.6}
\end{align}
其中$\,E_{\mu\nu}\,$是线性化的\,Einstein\,张量
\begin{align}
    E_{\mu\nu} &\equiv \frac{1}{2}\Bigl(\partial_{\mu}\partial_{\nu}h^{\lambda}{}_{\lambda}
    +\square h_{\mu\nu} - \partial_{\mu}\partial^{\lambda} h_{\lambda\nu} \nonumber \\
    &\quad -\partial_{\nu}\partial^{\lambda}h_{\lambda\mu} -\eta_{\mu\nu}\square h^{\lambda}{}_{\lambda}
    +\eta_{\mu\nu}\partial^{\lambda}\partial^{\rho}h_{\lambda\rho} \Bigr) \label{31.2.7}
\end{align}
而
\begin{equation}
    L^{\nu} \equiv \mi\,\epsilon^{\nu\mu\kappa\rho}\,\gamma_{5}\,\gamma_{\mu}\,\partial_{\kappa}\psi_{\rho} \:, \label{31.2.8}
\end{equation}
我们将在\,\ref{sec:31.3}\,节证明后者是自旋\,3/2\,无质量自由场的波动方程的左边. 例如, 通过对方程(\ref{31.2.1}) 和(\ref{31.1.27})使用$\,C^{H}$, $V_{\mu}^{H}\,$和$\,D^{H}\,$的超对称变换规则(\ref{26.2.11}), (\ref{26.2.15})和(\ref{26.2.17}), 我们发现
\[
\delta C^{E\,\sigma} = \mi\,\Bigl(\bar{\alpha}\,\gamma_{5}\,\Bigl[\slashed{\partial}\lambda^{H\,\sigma}
-\tfrac{1}{2}\mi\,\epsilon^{\nu\mu\kappa\sigma}\,\gamma_{5}\,\gamma_{\mu}\,\partial_{\kappa}\lambda_{\nu}^{H}
+\partial^{\sigma}\partial^{\kappa}\omega_{\kappa}^{H} \Bigr]\Bigr) \:.
\]
将其与$\,C^{E}\,$的变换规则(\ref{26.2.11})做一比较表明
\[
\omega^{E\,\sigma} = \slashed{\partial}\lambda^{H\,\sigma} -
\tfrac{1}{2}\mi\,\epsilon^{\nu\mu\kappa\sigma}\,\gamma_{5}\,\gamma_{\mu}\,\partial_{\kappa}\lambda_{\nu}^{H}
+\partial^{\sigma}\partial^{\kappa}\omega_{\kappa}^{H} \:.
\]
我们可以使用方程(\ref{31.1.26})把$\,\lambda_{\mu}^{H}\,$表示成$\,\psi_{\mu}\,$和$\,\omega_{\mu}^{H}$:
\[
\lambda_{\mu}^{H} = \psi_{\mu}-\gamma_{\mu}\gamma^{\rho}\psi_{\rho} - \gamma_{\mu}\partial^{\rho}\omega_{\rho}^{H} \:,
\]
并发现仅用$\,\psi_{\sigma}\,$就可以表示出$\,\omega_{\sigma}^{E}$
\[
\omega_{\sigma}^{E} = \slashed{\partial}\psi_{\sigma} - \partial_{\sigma}\gamma^{\rho}\psi_{\rho} -
\tfrac{1}{2}\mi\,\epsilon_{\nu\mu\kappa\sigma}\,\gamma_{5}\,\gamma^{\mu}\,\partial^{\kappa}\psi^{\nu} \:.
\]
利用恒等式
\[
\eta_{\mu\nu}\gamma_{\lambda} - \eta_{\mu\lambda}\gamma_{\nu} = \mi\,\epsilon_{\mu\lambda\nu\rho}\gamma_{5}\gamma^{\rho}
-\tfrac{1}{2}\mi\,\gamma_{\mu}\gamma^{\sigma}\,\epsilon_{\sigma\lambda\nu\rho}\gamma_{5}\gamma^{\rho} \:,
\]
我们发现了$\,\omega_{\mu}^{E}\,$的方程(\ref{31.2.2}). 以这种方法继续下去给出了$\,E_{\mu}\,$其它分量的公式(\ref{31.2.3})---(\ref{31.2.6}), 并证实了这些分量确实构成实超场.

通过将拉格朗日密度取成
\begin{equation}
    \mathscr{L}_{E} = \tfrac{4}{3}\Bigl(E_{\mu}H^{\mu}\Bigr)_{D} = E_{\mu\nu}h^{\mu\nu}
    -\tfrac{1}{2}\bar{\psi}_{\mu}L^{\mu} -\tfrac{4}{3}(s^{2}+p^{2}-b_{\mu}b^{\mu} ) \:, \label{31.2.9}
\end{equation}
我们现在构建出了关于度规超场是二次且在超对称和扩充规范变换下不变的作用量. 加入因子 $4/3\,$是为了使得引力场的动能拉格朗日量在符号和归一化上与传统一致: 除了包含$\,\partial^{\mu}h_{\mu\nu}\,$或\,$h^{\lambda}{}_{\lambda}$ 的项, 它是$\,h_{\mu\nu}\,$分量的\,Klein-Gordon\,拉格朗日量之和. 下一节讨论引力微子场的归一化.

为了看到方程(\ref{31.2.9})中最后一个表达式的前两项在规范变换(\ref{31.1.12})和(\ref{31.1.13})下不变, 我们注意到$\,E_{\mu\nu}\,$和$\,L_{\mu}\,$在这些变换下显然是不变的, 而作用量关于出现在这些项中的两个$\,h_{\mu\mu}\,$或$\,\psi_{\mu}$ 因子是对称的. 方程(\ref{31.2.9})中没有$\,s$, $p\,$和$\,b_{\mu}\,$的导数项表明它们是辅助场. 场方程使得它们在纯引力时为零, 而当引力与物质耦合时不为零.

在处理物质与引力的耦合之前, 我们先来考虑我们应该对方程(\ref{31.1.5})和(\ref{31.1.10})中的归一化常数$\,\kappa\,$取什么值. 纯引力场的\,Einstein--Hilbert\,作用量是
\begin{equation}
    I_{\text{GR}} = -\frac{1}{16\uppi G}\int \dif^{4}x\: \sqrt{g}\,R \:, \label{31.2.10}
\end{equation}
其中$\,G\,$是引力牛顿常数, $g(x)\,$是度规张量$\,g_{\mu\nu}(x)\,$的行列式, 而$\,R(x)\,$是从$\,g_{\mu\nu}(x)\,$计算而得的曲率标量. 为了对$\,g_{\mu\nu}=\eta_{\mu\nu}+2\kappa h_{\mu\nu}\,$的弱引力场计算$\,I_{\text{GR}}$, 我们回忆起对引力场的任意变分$\,\delta g_{\mu\nu}(x)$\cite{6}
\begin{equation}
    \delta  I_{\text{GR}} = \frac{1}{16\uppi G}\int \dif^{4}x\: \sqrt{g}\,
    \biggl[R^{\mu\nu}-\frac{1}{2}g^{\mu\nu}R\biggr] \,\delta g_{\mu\nu} \:, \label{31.2.11}
\end{equation}
其中$\,R^{\mu\nu}(x)\,$是从$\,g_{\mu\nu}(x)\,$计算得出的\,Ricci\,张量, 对于$\,g_{\mu\nu}=\eta_{\mu\nu}+2\kappa h_{\mu\nu}\,$的弱引力场, Ricci\,质量是\cite{7}
\begin{equation}
    R^{\mu\nu} = \kappa \Bigl(\square h^{\mu\nu}-\partial_{\lambda}\partial^{\mu}h^{\lambda\nu}
    -\partial_{\lambda}\partial^{\nu}h^{\lambda\mu} + \partial^{\mu}\partial^{\nu}h^{\lambda}{}_{\lambda}\Bigr)\:,
    \label{31.2.12}
\end{equation}
所以对于弱场
\begin{equation}
    R^{\mu\nu} - \tfrac{1}{2}g^{\mu\nu}R = 2\kappa E^{\mu\nu} \:, \label{31.2.13}
\end{equation}
因而方程(\ref{31.2.10})给出
\[
\delta  I_{\text{GR}} = \frac{\kappa^{2}}{4\uppi G}\int \dif^{4}x\: E^{\mu\nu}\delta h_{\mu\nu} \:.
\]
另一方面, 将$\,\int\dif^{4}x\:E_{\mu\nu}h^{\mu\nu}\,$对它所含的两个$\,h\,$因子的对称性考虑在内, 我们有
\[
\delta \int\dif^{4}x\: E^{\mu\nu}\,h_{\mu\nu} =2 \int \dif^{4}x\: E^{\mu\nu}\delta h_{\mu\nu} \:.
\]
为了使方程(\ref{31.2.9})中的$\,E_{\mu\nu}h^{\mu\nu}\,$项给出了通常的引力拉格朗日密度, 必然要取
\begin{equation}
    \kappa = \sqrt{8\uppi G} \:. \label{31.2.14}
\end{equation}


我们现在把\,Einstein\,拉格朗日密度(\ref{31.2.9}), 引力与物质的相互作用(\ref{31.1.34})以及物质拉格朗日量$\,\mathscr{L}_{M}$结合在一起, 这就给出了总的拉格朗日密度:
\begin{align}
    \mathscr{L} &= \mathscr{L}_{M} + E_{\mu}h^{\mu\nu} - \tfrac{1}{2}\bar{\psi}_{\mu} L^{\mu}
    -\tfrac{4}{3}(s^{2}+p^{2}-b_{\mu}b^{\mu}) \nonumber \\
    &\quad +2\kappa \Bigl[\tfrac{1}{2}\mathscr{R}_{\sigma}b^{\sigma} + \tfrac{1}{4}\bar{S}^{\sigma}\psi_{\sigma}
    -\mathscr{M}s - \mathscr{N}p + \tfrac{1}{2}T^{\kappa\sigma}h_{\sigma\kappa} \Bigr] \:. \label{31.2.15}
\end{align}
辅助场的场方程给出
\begin{equation}
    s=-6\kappa\mathscr{M}/8\:, \qquad p=-6\kappa\mathscr{N}/8 \:, \qquad b_{\mu}=-6\kappa\mathscr{R}_{\mu}/16\:.
    \label{31.2.16}
\end{equation}
利用这些消掉辅助场, 拉格朗日密度(\ref{31.2.15})现在给出
\begin{align}
    \mathscr{L} &= \mathscr{L}_{M} + E_{\mu\nu}h^{\mu\nu} - \tfrac{1}{2}\bar{\psi}_{\mu}L^{\mu}
    +\tfrac{3}{4}\kappa^{2}(\mathscr{M}^{2}+\mathscr{N}^{2}-\tfrac{1}{4}\mathscr{R}_{\mu}\mathscr{R}^{\mu}) \nonumber \\
    &\quad +\frac{1}{2}\kappa\bar{S}^{\sigma}\psi_{\sigma} +\kappa T^{\kappa\sigma}h_{\sigma\kappa}\:. \label{31.2.17}
\end{align}
场$\,\psi_{\mu}\,$和$\,h_{\mu\nu}\,$的源是$\,\kappa\,$阶的, 所以我们可以认为这些场是这一阶的, 这使得方程(\ref{31.2.17})中除$\,\mathscr{L}_{M}$ 以外的项都是$\,\kappa^{2}\,$阶的.

在真空中, 只有标量场$\,s\,$和$\,p\,$有$\,G\,$阶的树级期望值, 这赋予了真空以能量密度:
\begin{equation}
    \rho_{\text{VAC}} = -\mathscr{L}_{\text{VAC}} = -\mathscr{L}_{M\,\text{VAC}}
    -\tfrac{3}{4}\kappa^{2}(\mathscr{M}^{2}+\mathscr{N}^{2}) \:. \label{31.2.18}
\end{equation}
$G\,$阶真空能密度中这些项前面的负号是超引力理论与普通超对称理论之间的一个标志性差异.

例如, 在单个左手征超场$\,\Phi\,$且超势为$\,f(\Phi)\,$的理论中, 零阶真空能是$\,-\mathscr{L}_{M\,\text{VAC}}=\lvert \dif f(\phi)/\dif \phi\rvert^{2}$, 而方程(\ref{31.1.33})和(\ref{26.7.27})给出
\[
\mathscr{M}+\mi\mathscr{N} = -\frac{1}{3}\biggl[\phi\frac{\dif f(\phi)}{\dif\phi}-3\,f(\phi)\biggr] \:,
\]
所以到$\,G\,$的首阶, 总的真空能是
\begin{equation}
    \rho_{\text{VAC}} = \biggl\lvert\frac{\dif f(\phi)}{\dif \phi}\biggr\rvert^{2} -
    \frac{8\uppi G}{3}\biggl\lvert\phi\frac{\dif f(\phi)}{\dif \phi}-3\,f(\phi)\biggr\rvert^{2} \:, \label{31.2.19}
\end{equation}
其中$\,\phi\,$是$\,\Phi\,$的标量分量. 这是针对使得能动量张量由方程(\ref{26.7.42})给出的度规定义, 特别地, 这使得 $T^{\lambda}{}_{\lambda}\,$在$\,f=0\,$时为零. 对于其他定义, 真空能会差一些$\,8\uppi G\lvert\phi\rvert^{2}\lvert \dif f(\phi)/\dif \phi\rvert^{2}\,$阶的项. 然而, 在计算真空能的极小值时这个歧义是不重要的. 对方程(\ref{31.2.19})的观察表明, 如果$\,f(\phi)\,$在某个点$\,\phi_{0}\,$处是稳定的, 那么在$\,G\,$的首阶, $\rho_{\text{VAC}}\,$对$\,\lvert\phi-\phi_{0}\rvert\,$和$\lvert \dif f/\dif \phi\rvert\,$有最小值, 而在任何这样的$\,\phi\,$值处, $G\,$阶真空能是
\begin{equation}
    \rho_{\text{VAC}} = -24\uppi G\lvert f(\phi_{0})\rvert^{2} \:. \label{31.2.20}
\end{equation}

超引力理论中超对称不破缺的真空态为什么不能有正定的能量密度有一个代数上的原因. 对均匀的非零真空态密度$\,\rho_{V}$, Einstein\,场方程的解在$\,\rho_{V}>0\,$时取\,de Sitter\,空间的形式, 而在$\,\rho_{V}<0\,$时取反\,de Sitter\,空间的形式. 这些空间可以被曲面
\begin{equation}
    x_{5}^{2}\pm\eta_{\mu\nu}x^{\mu}x^{\nu} = R^{2} \:, \label{31.2.21}
\end{equation}
描述, 而这些曲面处在一个\,5\,维准欧几里得空间中, 且有线元
\begin{equation}
    \dif s^{2} = \eta_{\mu\nu}x^{\mu}x^{\nu}\pm \dif x_{5}^{2} \:, \label{31.2.22}
\end{equation}
其中正号对应\,de Sitter\,空间, 负号对应反\,de Sitter\,空间. 这些空间的对称性不再是平移和\,Lorentz 变换构成的\,Poincar\'{e}\,群, 对\,de Sitter\,空间, 它是$\,O(4,1)\,$群, 对反\,de Sitter\,空间则是$\,O(3,2)\,$群, 其中$\,O(n,m)\,$群由保持主对角元上有$\,n\,$个正元素和$\,m\,$个负元素的对角度规不变的线性变换构成. 在有\,de Sitter\,或反\,de Sitter\,真空态的理论中, 超对称性是不破缺的, 因此它分别有与时空对称性相同的单群$\,O(4,1)\,$或$\,O(3,2)$. Nahm\cite{7a}\,分类了所有带有单纯时空对称性的超对称性. 有简单$\,O(3,2)$ 超对称性, 也有$\,N$-扩充的$\,O(3,2)\,$超对称性, 但对于$\,O(4,1)\,$则只有一个$\,N=2\,$的超对称性. 我们这里考虑的是\,$N=1\,$的超引力理论, 所以它们可以有超对称性不破缺的真空态以及$\,\rho_{V}<0$, 这给出了$\,O(3,2)\,$时空对称性, 而不是超对称不破缺且$\,\rho_{V}>0\,$的$\,O(4,1)\,$时空对称性.


有可能存在能量密度为负的真空场构型最初看上去威胁了我们宇宙的稳定性. $f(\phi)\,$通常会有数个稳定点, 而它在每个稳定点取不同的值. 
即使我们对$\,f(\phi)\,$中的参量做了精细调节使得$\,f(\phi)\,$在其中一个稳定点上为零, 由于我们观测到的宇宙是近平坦的, 如果$\,f(\phi)\,$在任何其他稳定点上非零, 那么它将会给出{\kai{更低}}的真空能, 提高了塌缩到负能量密度的态的可能性, 而度规将是``反\,de Sitter''的形式而不是平坦空间.


幸运的是, 仅(\ref{31.2.20})本身不能提供为了制造反\,de Sitter\,空间泡沫而要进入表面张力的正能量, 使得来自平坦空间的跃迁从能量的角度来看实际上并不可能发生. Coleman\,和\,de Luccia\cite{8}\,将广义相对论的方程应用到能量密度为零的平坦空间中的一个内部能量密度为$\,-\epsilon\,$的泡沫中, 并证明了, 如果
\begin{equation}
    \epsilon \leq 6\uppi GS^{2} \:, \label{31.2.23}
\end{equation}
那么任何这样的泡沫只要不承担引力奇点, 那么它的能量就是正的.

表明张力是泡沫表面单位面积上的能量, 到$\,G\,$的零阶, 它由能量密度对泡沫壁的积分给出:
\begin{equation}
    S_{1} =\int_{r_{-}}^{r_{+}} \dif r\:\Biggl[\biggl\lvert\frac{\dif \phi}{\dif r}\biggr\rvert^{2}
    +\biggl\lvert \frac{\dif f(\phi)}{\dif \phi}\biggr\rvert^{2}\Biggr] \:, \label{31.2.24}
\end{equation}
其中$\,r_{-}\,$和$\,r_{+}\,$分别是泡沫内壁和外壁到中心的距离. 这可以被重写成
\[
    S_{1} =\int_{r_{-}}^{r_{+}} \dif r\:\Biggl[ \biggl\lvert\frac{\dif \phi}{\dif r}
    +\xi\biggl(\frac{\dif f(\phi)}{\dif \phi}\biggr)^{\ast}\biggr\rvert^{2}
    -2\operatorname{Re}\biggl(\xi^{\ast}\frac{\dif \phi}{\dif r}\frac{\dif f(\phi)}{\dif \phi}\biggr)\Biggr] \:,
\]
其中$\,\xi\,$是$\,\lvert \xi\rvert=1\,$的任意相因子. 对第二项的积分是平庸的: 由于我们假定了$\,\phi(r_{+})\,$的值使得$\,f(\phi)\,$稳定且为零, 而$\,\phi(r_{-})\,$所处的值$\,\phi_{0}\,$使得$\,f(\phi)\,$稳定但不为零, 这个积分是
\[
\int_{r_{-}}^{r_{+}} \dif r\: \frac{\dif \phi}{\dif r}\frac{\dif f(\phi)}{\dif \phi} =-f(\phi_{0}) \:.
\]
为了最大化这一项, 我们取$\,\xi=f(\phi_{0})/\lvert f(\phi_{0})\rvert$, 并获得了不等式\cite{9}
\begin{equation}
    S_{1} \geq 2\,\lvert f(\phi_{0}) \rvert \:, \label{31.2.25}
\end{equation}
其中当微分方程$\,\dif\phi/\dif r=-\xi(\dif f/\dif\phi)^{\ast}\,$在合适的边界条件下有界时, 不等式取等号. 因此当内部能量密度不小于$\,-24\uppi G\lvert f(\phi_{0})\rvert^{2}\,$时, 即精确取值(\ref{31.2.20}), 不等式(\ref{31.2.23})是成立的. 这个计算打开了平坦空间会因辐射修正不稳定的可能性, 但读者无需担心: 业已证明了, 若超引力理论中存在真空能为零的真空场构形, 当限制在有限区域内时, 对该场的任何扰动的能量都是正的.\cite{10}


\subsection*{* * *}


方程(\ref{26.7.48})表明一组左手征标量超场$\,\Phi_{n}\,$的能动量张量会包含一项
\begin{equation}
    \Delta T^{\mu\nu} = \frac{1}{3}\Bigl(\eta^{\mu\nu}\square-\partial^{\mu}\partial^{\nu}\Bigr)
    \sum_{n}\lvert\phi_{n}\rvert^{2} \:. \label{31.2.26}
\end{equation}
分部积分, 相应相互作用$\,\kappa h_{\mu\nu}\Delta T^{\mu\nu}\,$对作用量的贡献形如
\begin{equation}
    \frac{\kappa}{3} \int \dif^{4}x\:\sum_{n}\lvert\phi_{n}\rvert^{2}\Bigl(\eta^{\mu\nu}\square
    -\partial^{\mu}\partial^{\nu}\Bigr)h_{\mu\nu} = \frac{1}{6}R^{(1)}\sum_{n}\lvert\phi_{n}\rvert^{2} \:,
    \label{31.2.27}
\end{equation}
其中$\,R^{(1)}\,$是线性近似下的曲率标量. 这被加到通常的\,Einstein-Hilbert\,作用量$\,-\sqrt{g}R/2\kappa^{2}\,$上(其作为$\,E_{\mu\nu}h^{\mu\nu}\,$出现在方程(\ref{31.2.17})中), 而它的效果是将这项的系数替换成
\begin{equation}
    -\frac{1}{2\kappa^{2}} + \frac{1}{6}\sum_{n}\lvert \phi_{n}\rvert^{2} =
    -\frac{1}{2\kappa^{2}} \biggl(1-\frac{\kappa^{2}}{3} \sum_{n}\lvert\phi_{n}\rvert^{2}\biggr) \:. \label{31.2.28}
\end{equation}
为了恢复通常的引力常数, 我们可以对度规做一个\,\emph{Weyl}\,{\kai{变换}}, 将标架$\,e^{a}{}_{\mu}\,$换成 
\begin{equation}
    \tilde{e}^{a}{}_{\mu} = e^{a}{}_{\mu} \sqrt{1-\frac{\kappa^{2}}{3} \sum_{n}\lvert\phi_{n}\rvert^{2}} \:.
     \label{31.2.29}
\end{equation}
即, 我们将度规$\,g_{\mu\nu}\,$换成
\begin{equation}
    \tilde{g}^{\mu\nu} = \biggl(1-\frac{\kappa^{2}}{3} \sum_{n}\lvert\phi_{n}\rvert^{2}\biggr)\, g_{\mu\nu}\:, \label{31.2.30}
\end{equation}
或者说将弱场换成
\begin{equation}
    \tilde{h}_{\mu\nu} = h_{\mu\nu} - \frac{\kappa}{6}\sum_{n}\lvert \phi_{n}\rvert^{2}\,\eta_{\mu\nu} \:. \label{31.2.31}
\end{equation}
新度规的弱场\,Einstein\,张量(\ref{31.2.7})是
\begin{align}
    \tilde{E}_{\mu\nu} &\equiv \frac{1}{2}\Bigl( \partial_{\mu}\partial_{\nu}\tilde{h}^{\lambda}{}_{\lambda}
    +\square \tilde{h}_{\mu\nu} - \partial_{\mu}\partial^{\lambda}\tilde{h}_{\lambda\nu}
    - \partial_{\nu}\partial^{\lambda}\tilde{h}_{\lambda\mu} - \eta_{\mu\nu}\square \tilde{h}^{\lambda}{}_{\lambda}
    +\eta_{\mu\nu}\partial^{\lambda}\partial^{\rho}\tilde{h}_{\lambda\rho} \Bigr) \nonumber \\
    &= E_{\mu\nu} - \frac{\kappa}{6}\Bigl(\partial^{\mu}\partial^{\nu}-\eta^{\mu\nu}\square\Bigr)
    \sum_{n}\lvert \phi_{n}\rvert^{2} \:. \label{31.2.32}
\end{align}
这样, 原始作用量中\,Einstein\,项与(\ref{31.2.27})的和就是
\begin{align}
    &\int \dif^{4}x\:\Biggl[h_{\mu\nu}E^{\mu\nu}+\frac{\kappa}{3} \sum_{n}\lvert \phi_{n}\rvert^{2}
    \Bigl(\eta^{\mu\nu}\square - \partial^{\mu}\partial^{\nu}\Bigr)h_{\mu\nu} \Biggr]
    = \int \dif^{4}x\: \Biggl[\tilde{h}_{\mu\nu}\tilde{E}^{\mu\nu} \nonumber \\
    &\qquad +\frac{\kappa^{2}}{12}\,\Biggl(\partial_{\mu}\sum_{n}\lvert\phi_{n}\rvert^{2}\Biggr)\,
    \Biggl(\partial^{\mu}\sum_{n}\lvert\phi_{n}\rvert^{2}\Biggr)\Biggr] \:, \label{31.2.33}
\end{align}
所以有效引力常数现在实际是一个常数. 度规的这个重定义也为超势引入了一个变化. 原始拉格朗日密度包含一项$\,-e\sum_{n}\lvert \partial f(\phi)/\partial \phi_{n}\rvert^{2}$, 用新标架表示就是
\begin{align*}
    -e\sum_{n}\lvert\partial f(\phi)/\partial \phi_{n}\rvert^{2} &=
    -\tilde{e} \sum_{n}\lvert\partial f(\phi)/\partial \phi_{n}\rvert^{2}  \\
    &\quad - \frac{2\kappa^{2}}{3} \Biggl(\sum_{n}\lvert\phi_{n}\rvert^{2}\Biggr)\,
    \sum_{n}\lvert\partial f(\phi)/\partial \phi_{n}\rvert^{2} \:.
\end{align*}
在度规的这个新定义下, 势(\ref{31.2.19})因此要被换成
\begin{align}
    \rho_{\text{VAC}} &= \sum_{n}\biggl\lvert\frac{\partial f(\phi)}{\partial\phi_{n}}\biggr\rvert^{2}
    -\frac{\kappa^{2}}{3}
    \Biggl\lvert\sum_{n}\phi_{n}\frac{\partial f(\phi)}{\partial\phi_{n}}-3\,f(\phi)\Biggr\rvert^{2} \nonumber \\
    &\quad + \frac{2\kappa^{2}}{3}\Biggl(\sum_{n}\lvert\phi_{n}\rvert^{2}\Biggr)
    \sum_{n}\biggl\lvert\frac{\partial f(\phi)}{\partial\phi_{n}}\biggr\rvert^{2} \:. \label{31.2.34}
\end{align}
直到$\,\kappa^{2}\,$阶, 新的项并不会改变$\,\rho_{\text{VAC}}\,$在稳定点的值, 所以无需对前面真空稳定性的讨论作出任何改变.



\section{引力微子} \label{sec:31.3}

在本节, 我们将使用\,\ref{sec:31.1}---\ref{sec:31.3}\,节发展的弱场形式理论来推导引力微子的一些性质. 特别地, 通过使用$\,G\to0\,$时的连续性, 我们将获得超对称性自发破缺时的引力微子质量公式, 这个公式到$\,G\,$的一阶是成立但对其它所有相互作用到任意阶都是成立的. (\ref{sec:31.6}节将给出这个公式的原始推导.)

首先, 我们必须要验证方程(\ref{31.2.9})中的$\,-\frac{1}{2}\bar{\psi}_{\mu}L^{\mu}\,$项是自旋\,3/2\,自荷共轭无质量粒子的自由场拉格朗日量. 为有自旋的粒子构建合适的自由场拉格朗日量的一个历史悠久的方法是先猜出拉格朗日量, 然后验证它确实给出物理上令人满意的场方程和传播子. 对自旋\,3/2\,粒子, 这导致了一些不确定性------例如, 超对称的文章中所谓的\,Rarita--Schwinger\,拉格朗日量通常不是\,Rarita\,和 Schwinger\,最早提出的那个拉格朗日量.\cite{11} 这里我们将使用一个与\,6.2\,节思想相同的方法: 我们先从\,Lorentz\,不变性要求出发推导出自旋\,3/2\,有质量粒子的传播子, 然后从这个传播子反推出拉格朗日量. 在这里, 为了简单以及在现实世界中我们必须把超对称破缺考虑在内, 我们着手与有质量传播子, 然而, 通过注意到超对称流守恒使得传播子中的零质量奇点无关紧要, 我们可以把从这种方法获得的结果应用于无质量引力微子的情况.

一个有额外矢量指标的旋量场$\,\psi^{\mu}\,$属于齐次\,Lorentz\,群的$\,[(\frac{1}{2},0)+(0,\frac{1}{2})]\times (\frac{1}{2},\frac{1}{2})\,$表示. 为了分离出自由场的$\,(1,\frac{1}{2})+(\frac{1}{2},1)\,$部分, 我们附加不可约条件
\begin{equation}
    \gamma_{\mu}\psi^{\mu} =0 \:. \label{31.3.1}
\end{equation}
旋转不变性和方程(\ref{31.3.1})告诉我们这个场在真空和动量$\,\mathbf{q}=0\,$且自旋$\,z$-分量为$\,s\,$的有质量自旋 $3/2\,$粒子之间的矩阵元将满足条件
\begin{equation}
    \langle 0 \vert \psi^{0}(0) \vert s\rangle =0  \label{31.3.2}
\end{equation}
和
\begin{equation}
    \sum_{s=-3/2}^{3/2} \langle 0\vert \psi^{i}(0)\vert s \rangle\,\langle 0\vert \psi^{j}(0)\vert s \rangle^{\ast}
    \propto \delta_{ij}-\tfrac{1}{3}\gamma_{i}\gamma_{j} \:, \label{31.3.3}
\end{equation}
其中比例系数可能依赖于在旋转意义下不变的矩阵$\,\beta\equiv\mi\gamma^{0}$. 在通常的$\,\beta\langle 0\vert \psi^{i}(0)\vert s\rangle =\langle 0\vert \psi^{i}(0)\vert s\rangle$ 的\,Dirac\,约定下(选择这个约定是为了简化场的空间反演性质), 以及类比方程(\textcolor{foo}{5.5.23})按惯例选择 $\beta=+1\,$分量的归一化, 方程(\ref{31.3.3})可以写成
\begin{equation}
    \sum_{s=-3/2}^{3/2} \langle 0\vert \psi^{i}(0)\vert s \rangle\,\langle 0\vert \psi^{j}(0)\vert s \rangle^{\ast}
    =(2\uppi)^{-3}\biggl(\frac{1+\beta}{2}\biggr)\Bigl[ \delta_{ij}-\tfrac{1}{3}\gamma_{i}\gamma_{j} \Bigr] \:. \label{31.3.4}
\end{equation}
由此得出质量为$\,m_{g}\,$的自旋\,3/2\,粒子的动量空间传播子形如
\begin{equation}
    \Delta^{\mu\nu}(q) = \frac{P^{\mu\nu}(q)}{q^{2}+m_{g}^{2}-\mi\epsilon} \:, \label{31.3.5}
\end{equation}
其中$\,P^{\mu\nu}(q)\,$是\,4-矢$\,q\,$的\,Lorentz\,协变多项式, 要满足条件: 在$\,\mathbf{q}=0\,$且$\,q^{0}=m_{g}\,$时,
\begin{equation}
    P^{ij} = \biggl(\frac{1+\beta}{2}\biggr)\Bigl[ \delta_{ij}-\tfrac{1}{3}\gamma_{i}\gamma_{j} \Bigr] \:, \qquad
    P^{i0}=P^{0i}=P^{00} =0 \:. \label{31.3.6}
\end{equation}
除了可能有一些在质壳上为零的项(这些项的效果因此就和直接的流-流相互作用相同), 有这个极限的协变函数只能是
\begin{equation}
    P^{\mu\nu}(q) = \Biggl(\eta^{\mu\nu}+\frac{q^{\mu}q^{\nu}}{m_{g}^{2}}\Biggr)\Bigl(-\mi\slashed{q}+m_{g}\Bigr)
    -\frac{1}{3}\Biggl(\gamma^{\mu}-\mi\frac{q^{\mu}}{m_{g}}\Biggr)(\mi\slashed{q}+m_{g})
    \Biggl(\gamma^{\nu}-\mi\frac{q^{\nu}}{m_{g}}\Biggr) \:. \label{31.3.7}
\end{equation}
(对方程与任何其它拥有极限(\ref{31.3.6})的协变函数之差给出的协变函数, 它的分量在$\,\mathbf{q}=0\,$且$\,q^{0}=m_{g}\,$时为零, 因此在质壳上处处为零.) 这样自由场的拉格朗日密度就有形式
\begin{equation}
    \mathscr{L}^{0} = -\tfrac{1}{2}\Bigl(\bar{\psi}^{\mu}\,D_{\mu\nu}(-\mi\partial)\,\psi^{\nu}\Bigr) \:, \label{31.3.8}
\end{equation}
其中
\begin{equation}
    \Delta^{\mu\nu}(q)D_{\nu\lambda}(q) =\delta_{\lambda}^{\mu} \:. \label{31.3.9}
\end{equation}
一个繁琐但直接的计算给出
\begin{equation}
    D_{\nu\lambda}(q) = -\epsilon_{\nu\mu\kappa\lambda}\,\gamma_{5}\gamma^{\mu}\,q^{\kappa}
    -\tfrac{1}{2}m_{g}[\gamma_{\nu},\gamma_{\lambda}] \:, \label{31.3.10}
\end{equation}
这使得拉格朗日密度(\ref{31.3.8})是
\begin{equation}
    \mathscr{L}_{0} = -\tfrac{1}{2}\mi\,\epsilon^{\nu\mu\kappa\lambda}\,
    \Bigl(\bar{\psi}_{\nu}\,\gamma_{5}\gamma_{\mu}\,\partial_{\kappa}\psi_{\lambda}\Bigr)
    +\tfrac{1}{4}m_{g}\Bigl(\bar{\psi}_{\nu}\,[\gamma^{\nu},\gamma^{\lambda}]\,\psi_{\lambda}\Bigr) \:. \label{31.3.11}
\end{equation}
当\,$m_{g}=0\,$时, 这个结果证实了方程(\ref{31.2.9})中的$\,-\frac{1}{2}\bar{\psi}_{\mu}L^{\mu}\,$项是自旋\,3/2\,的自荷共轭粒子按惯例正确归一化的自由场拉格朗日量. 在$\,m_{g}\to 0\,$的极限下, 由方程(\ref{31.3.5})和(\ref{31.3.7})给出的传播子是奇异的(这仅仅反映了\,5.9\,节中用螺旋度$\,\pm 3/2\,$无质量粒子的产生湮灭算符不可能构建出$\,(1,\frac{1}{2})+(\frac{1}{2},1)\,$的场这个事实), 但$\,P^{\mu\nu}(q)\,$的方程(\ref{31.3.7})中随着$\,m_{g}\to 0\,$发散的项全都正比于$\,q^{\mu}\,$和(或)$\,q^{\nu}$, 因此当与$\,\psi_{\mu}\,$相互作用的流守恒时, 这些项没有贡献. 

作为拉格朗日密度(\ref{31.3.11})正确性的进一步检验, 引入看似奇怪的质量项, 我们注意到它给出场方程
\begin{equation}
    {-}\mi\,\epsilon^{\nu\mu\kappa\lambda}\,\gamma_{5}\gamma_{\mu}\,\partial_{\kappa}\psi_{\lambda}
    +\tfrac{1}{2}m_{g}\,[\gamma^{\nu},\gamma^{\lambda}]\,\psi_{\lambda} = 0 \:. \label{31.3.12}
\end{equation}
对方程取散度给出
\[
[\slashed{\partial},\gamma^{\lambda}]\psi_{\lambda} = 0 \:.
\]
另外, 用$\,\gamma_{\nu}\,$收缩方程(\ref{31.3.12})表明
\[
\gamma_{\nu}\psi^{\nu} \propto \epsilon^{\nu\mu\kappa\lambda}\,\gamma_{5}\gamma_{\mu}\,\partial_{\kappa}\psi_{\lambda}
\propto [\slashed{\partial},\gamma^{\lambda}]\psi_{\lambda} = 0 \:,
\]
使得自由场满足不可约条件(\ref{31.3.1})(尽管在考虑相互作用之后这是不必要的). 从这两个结果可以得出另一个不可约条件
\[
\partial_{\lambda}\psi^{\lambda} = \tfrac{1}{2}\{\slashed{\partial},\gamma_{\lambda}\}\psi^{\lambda}
=\tfrac{1}{2}[\slashed{\partial},\gamma_{\lambda}]\psi^{\lambda} =0 \:.
\]
利用这些不可约条件, 我们可以把场方程(\ref{31.3.12})写成\,Dirac\,方程的形式
\begin{equation}
    \Bigl(\slashed{\partial}+m_{g}\Bigr)\psi^{\lambda} =0 \:, \label{31.3.13}
\end{equation}
除此之外, 这证明了它是质量为$\,m_{g}\,$的粒子的自由场.

我们现在将考虑超引力理论中超对称自发破缺的效应. 破缺的整体对称性引起了一个无质量自旋\,1/2\,粒子, 即戈德斯通微子, 但在超引力理论中, %
戈德斯通微子场$\,\chi\,$可以被规范变换$\,\psi_{\mu}\to\psi_{\mu}-\partial_{\mu}\chi\,$消掉. 以这种方法消掉戈德斯通微子场就确定了规范, 
因此规范不变性就不再保持引力微子是无质量的, 它就获得了质量(从此记为\,$m_{g}$), 这很像我们在\,21.3\,节看到的: 矢量玻色子$\,W^{\pm}\,$和 $Z^{0}\,$从电弱相互作用的$\,SU(2)\times U(1)\,$规范对称性的自发破缺中获得质量.

正如我们在第\,\ref{cha:28}\,章开头所讨论的, 如果超对称与可及的现象相关, 那么它破缺的特征能量标度必须远低于\,Planck\,质量. 在这个情况下, 我们可以用一个连续性讨论来给引力微子质量$\,m_{g}\,$一个通用公式. 根据方程(\ref{31.1.34}), 引力微子场$\,\psi_{\mu}\,$与超对称流$\,S^{\mu}\,$耦合的耦合常数是$\,\frac{1}{2}\kappa=\frac{1}{2}\sqrt{8\uppi G}$, 所以在跃迁$\,A+B\to C+D\,$中交换一个\,4-动量为$\,q\,$的虚引力微子对不变振幅的贡献是 
\begin{equation}
    M(A+B\to C+D) = \tfrac{1}{4}(8\uppi G)\,\langle C\vert\bar{S}_{\mu}\vert A\rangle_{N} \,
    \Delta^{\mu\nu}(a)\,\langle D\vert S_{\nu}\vert B\rangle_{N} \:, \label{31.3.14}
\end{equation}
其中下标$\,N\,$是指从超对称流的矩阵元中移除了$\,q^{2}=0\,$处的单戈德斯通微子极点. 当超对称破缺标度充分小于\,Planck\,质量时, 动量传递有一个远大于引力微子质量但远小于\,Planck\,质量的取值范围. 对于这样的动量, 主导矩阵元的是传播子分子(\ref{31.3.7})中的$\,1/m_{g}^{2}\,$项
\begin{equation}
    M(A+B\to C+D) \to \tfrac{1}{4}(8\uppi G)\,\langle C\vert\bar{S}_{\mu}\vert A\rangle_{N} \,
    \Biggl(\frac{-2\mi\,\slashed{q}q^{\mu}q^{\nu}}{3m_{g}^{2}q^{2}}\Biggr)\,\langle D\vert S_{\nu}\vert B\rangle_{N} \:, \label{31.3.15}
\end{equation}
但当\,Planck\,质量充分大时, 就可以忽视引力微子的耦合, 而矩阵元就与没有引力微子的理论中交换戈德斯通微子产生的矩阵元相同. 
根据方程(\ref{29.2.10}), 这是
\begin{equation}
    M(A+B\to C+D) \to \langle C\vert\bar{S}_{\mu}\vert A\rangle_{N} \,
    \biggl(\frac{-\mi\,\slashed{q}}{q^{2}}\biggr)\,\biggl(\frac{q^{\mu}q^{\nu}}{F^{2}}\biggr)\,
    \langle D\vert S_{\nu}\vert B\rangle_{N} \:, \label{31.3.16}
\end{equation}
其中$\,F\,$是表征超对称破缺强度的参量(这里取为实数), 它的定义使得真空能密度是$\,F^{2}/2$. 为了使方程(\ref{31.3.15})和(\ref{31.3.16})一致, 引力微子质量必须有值
\begin{equation}
    m_{g} = \sqrt{\frac{4\uppi\,G\,F^{2}}{3}} \:. \label{31.3.17}
\end{equation}
这个公式仅在$\,GF^{2}\,$的最低阶是成立的, 但是对与超对称自发破缺相关的非引力相互作用, 它直到任意阶(甚至是非微扰的)都是成立的.


由于一些原因, 将$\,m_{g}\,$表示成无自旋辅助引力场的真空期望值$\,\lvert s\rvert$和$\,\lvert p\rvert\,$将是方便的. 
注意到对时空曲率为零的真空态, 物质场的真空能密度必须要与引力的负真空能以及引力与隐藏分区场的相互作用相平衡, 后者用方程(\ref{31.2.18})和(\ref{31.2.16})表示成$\,\lvert s\rvert\,$和$\,\lvert p\rvert\,$是$\,-(4/3)(\lvert s\rvert^{2}+\lvert p\rvert^{2})$, 所以
\begin{equation}
    F^{2}/2 =(4/3)(\langle s\rangle^{2} + \langle p \rangle^{2}) \:.  \label{31.3.18}
\end{equation}
我们因此可以把方程(\ref{31.3.17})写成
\begin{equation}
    m_{g} = \frac{2\kappa}{3}\sqrt{\langle s\rangle^{2} + \langle p \rangle^{2}} \:. \label{31.3.19}
\end{equation}
引入复引力微子质量有时是方便的, 这个复质量的定义是
\begin{equation}
    \tilde{m}_{g} \equiv \frac{2\kappa}{3}\Bigl(\langle s\rangle +\mi  \langle p \rangle\Bigr)\:, \label{31.3.20}
\end{equation}
它的绝对值是物理的引力微子质量(\ref{31.3.19}).


\section{反常传递的超对称破缺} \label{sec:31.4}


在\,\ref{sec:28.3}\,节提出的如下的可能性: 超场的某类不携带标准模型的$\,SU(3)\times SU(2)\times U(1)\,$量子数的隐藏分区可能破坏了超对称性, 而这部分超场与可观测粒子通过引力交互. 在这一节, 我们将处理最小超对称标准模型中的一类超对称破缺效应, 即$\,\kappa\equiv \sqrt{8\uppi G}\,$的一阶效应. 这包括规范微子质量以及拉格朗日密度(\ref{28.4.1})中的参量$\,A_{ij}\,$和$\,B$. 其他的超对称破缺效应, 例如标量夸克和标量轻子的质量平方, 是$\,\kappa\,$的二阶, 我们将在\,\ref{sec:31.7}\,节利用\,\ref{sec:31.6}\,节描述的一般超引力形式理论考虑引力引导的超对称破缺时进行处理.

可以看到, 引力引导的超对称性破缺效应在$\,\kappa\,$的一阶就是将相互作用(\ref{31.1.34})中引力超多重态的分量场换成它们的期望值. 这些分量场中唯一一个可以从物质场的隐藏分区中的超对称性自发破缺获得非零真空期望值是无自旋辅助场$\,s\,$和$\,p$, 所以, 加上方程(\ref{31.1.33}), 这就给出了一阶超对称破缺相互作用
\begin{equation}
    \mathscr{L}^{(1)} = 2\kappa \Bigl[-A^{X}\langle p\rangle +B^{X}\langle s\rangle \Bigr]
    =3\operatorname{Im} \Bigl[\tilde{m}_{g}^{\ast}\,(A^{X}+\mi B^{X}) \Bigr] \:, \label{31.4.1}
\end{equation}
其中$\,\tilde{m}_{g}\,$是复引力微子质量(\ref{31.3.20}), 而$\,A^{X}\,$和$\,B^{X}\,$是\,\ref{sec:26.7}\,节中讨论的实手征非标度不变超场$\,X\,$的 $A$-分量和$\,B$-分量.


我们在\,\ref{sec:26.7}\,节证明了, 对于左手征超场$\,\Phi_{n}\,$的超势为$\,f(\Phi)\,$的可重整理论, $X\,$超场是
\begin{equation}
    X = \frac{2}{3}\operatorname{Im}\Biggl[\sum_{n}\Phi_{n}\frac{\partial f(\Phi)}{\partial \Phi_{n}} - 3\,f(\Phi)\Biggr] \:.
    \label{31.4.2}
\end{equation}
通过用无量纲参量和量纲为质量的参量$\,\mathscr{M}\,$表示超势的耦合常数, 上式就能变成一个可以立即推广到一般理论的形式. 由于超势的量纲是$(\text{质量})^{3}$, 我们有
\begin{equation}
    \mathscr{M}\frac{\partial f(\Phi)}{\partial \mathscr{M}} +
    \sum_{n}\Phi_{n}\frac{\partial f(\Phi)}{\partial \Phi_{n}} =3\,f(\Phi) \:. \label{31.4.3}
\end{equation}
因此方程(\ref{31.4.2})可以写成
\begin{equation}
    X = \frac{2}{3}\operatorname{Im}\biggl[\mathscr{M}\frac{\partial f(\Phi)}{\partial \mathscr{M}}\biggr]\:.
    \label{31.4.4}
\end{equation}
这个公式可以被推广以给出拉格朗日量中的任何一类$\,\mathscr{F}$-项的标度相关部分对$\,X\,$的贡献. 拉格朗日量中的$\,2\operatorname{Re}[f(\Phi,W)]_{\mathscr{F}}\,$项对$\,X\,$的贡献由方程(\ref{31.4.4})的一个显然推广给出
\begin{equation}
    X = \frac{2}{3}\operatorname{Im}\biggl[\mathscr{M}\frac{\partial f(\Phi,W)}{\partial \mathscr{M}}\biggr]\:.
    \label{31.4.5}
\end{equation}
(拉格朗日量中$\,D$-项的任何质量标度相关部分对$\,X\,$也有一个贡献.) 用方程(\ref{31.4.5})和方程(\ref{26.3.10}) 以及(\ref{26.3.13})做一比较, 我们看到
\begin{equation}
    A^{X}+\mi B^{X}= \frac{2\mathscr{M}}{3\mi}\frac{\partial}{\partial \mathscr{M}}\Bigl[f(\Phi,W)\Bigr]_{\theta=0}
    =\frac{2\mathscr{M}}{3\mi}\frac{\partial f(\phi,\lambda_{L})}{\partial \mathscr{M}} \:. \label{31.4.6}
\end{equation}
这样有效拉格朗日中就有一个$\,\kappa\,$阶的超对称性破缺项, 由方程(\ref{31.4.1})和(\ref{31.4.6})给出
\begin{equation}
    \mathscr{L}_{f}^{(1)} = -2\operatorname{Re}\biggl[\tilde{m}_{g}^{\ast}\,
    \frac{ \mathscr{M}\partial f(\phi,\lambda_{L})}{\partial \mathscr{M}}\biggr] \:. \label{31.4.7}
\end{equation}

方程(\ref{31.4.6})不仅对拉格朗日量中有显式标度相关性的项成立, 对由重整化群方程描述的耦合常数的标度相关性, 这个方程也是成立的.\cite{11a} 作为标度相关性来源的量子力学反常既赋予了能动量张量的迹一个非零值, 也赋予流散度一个非零值, 所以以这种方式产生的可观测超对称破缺效应称为是{\kai{反常引导的}}. 


特别地, 考虑可重整超对称规范理论的动能项$\,\mathscr{L}_{\text{gauge}}$, 它由方程(\ref{27.3.22})和(\ref{27.3.23})给出
\begin{equation}
    \mathscr{L}_{\text{gauge}} = -\frac{1}{2g^{2}}\operatorname{Re}\sum_{A\alpha\beta}
    [\epsilon_{\alpha\beta}W_{A\alpha}W_{A\beta}]_{\mathscr{F}} \:. \label{31.4.8}
\end{equation}
这并不显式地依赖于任何质量标度, 但耦合常数$\,g\,$以我们熟悉的方式依赖于重整化标度$\,\mathscr{M}$, 由重整化群方程给出
\begin{equation}
    \mathscr{M} \frac{\dif g(\mathscr{M})}{\dif\mathscr{M}} = \beta\Bigl(g(\mathscr{M})\Bigr) \:. \label{31.4.9}
\end{equation}
那么方程(\ref{31.4.7})表明规范拉格朗日量(\ref{31.4.8})在拉格朗日量中产生了一个超对称破缺项
\begin{equation}
    \mathscr{L}_{\text{gauge}}^{(1)} = -\frac{\beta(g)}{g^{3}} \operatorname{Re}
    \Biggl[\tilde{m}_{g}^{\ast}\,\sum_{A\alpha\beta}\epsilon_{\alpha\beta}\lambda_{AL\alpha}\lambda_{AL\beta}\Biggr]\:,
    \label{31.4.10}
\end{equation}
其中$\,\lambda_{AL\alpha}\,$是规范微子场的左手部分, 它像$\,W_{A\alpha}\,$那样通过乘以规范耦合$\,g\,$来归一化, 这使得$\,g\,$不出现在结构常数或规范超场与夸克超场的相互作用中. 将这一归一化约定考虑在内, 规范微子质量等于$\,g^{2}\,$乘以$\,\frac{1}{2}\sum_{A\alpha\beta}\epsilon_{\alpha\beta}\lambda_{AL\alpha}\lambda_{AL\beta}\,$的系数的绝对值, 或者
\begin{equation}
    m_{\text{gaugino}} = m_{g}\,\biggl\lvert \frac{\beta(g)}{g}\biggr\rvert \:. \label{31.4.11}
\end{equation}
在这个公式中, $m_{\text{gaugino}}\,$和$\,g\,$是拉格朗日量中的截断相关裸参量, 由威尔逊型重整化方程控制. 在 \ref{sec:27.6}\,节, 我们看到$\,\beta(g)\,$仅来自于单圈图, 使得$\,\beta(g)=bg^{3}$, 其中$\,b\,$是常数, 因此
\begin{equation}
    m_{\text{gaugino}} = m_{g}\,\lvert b\rvert\,g^{2} \:. \label{31.4.12}
\end{equation}
规范微子的物理质量与方程(\ref{31.4.12})的差是$\,g\,$的高阶修正, 但是, 由于我们知道规范微子质量必远重于量子色动力学的特征标度, 
这些修正对胶微子以及$\,W\,$微子和$\,B\,$微子非常小.


超对称不允许胶子和夸克超场的拉格朗日量中有显式的标度相关项, 所以, 当电弱相互作用可以忽视时, 方程(\ref{31.4.12})对胶微子仅有$\,\kappa\,$阶贡献. 对三代夸克, 方程(\ref{28.2.10})给出$\,b=-3g_{s}^{2}/16\uppi^{2}$. 取$\,g_{s}^{2}/4\uppi =0.118$, 方程(\ref{31.4.12})给出了胶微子质量
\begin{equation}
    m_{\text{gluino}} = \frac{3g_{s}^{2}m_{g}}{16\uppi^{2}} = 2.8 \times 10^{-2}\,m_{g} \:. \label{31.4.13}
\end{equation}
另一方面, Higgs\,超场的拉格朗日量中有一个标度相关的相互作用, 由(\ref{28.1.7})中$\,\mu$-项\,$-\mu(H_{2}^{\mathrm{T}}eH_{1})$ 产生, 所以在方程(\ref{31.4.7})产生的拉格朗日量中有一项$\,2\operatorname{Re}[\tilde{m}_{g}^{\ast}\mu(\mathscr{H}_{2}^{\mathrm{T}}e\mathscr{H}_{1})].$ 与方程(\ref{28.4.1})中的$\,B\mu$-项比较, 我们看到这给出
\begin{equation}
B=-\tilde{m}_{g}^{\ast} \:. \label{31.4.14}
\end{equation}


当有三代夸克和轻子超场以及一对\,Higgs\,双重态超场$\,H_{1}\,$和$\,H_{2}$时, (\ref{28.2.8})和(\ref{28.2.9})对\,$U(1)$ 规范耦合$\,g'\,$给出$\,b=11/16\uppi^{2}$, 对$\,SU(2)\,$规范耦合$\,g\,$则给出$\,b=1/16\uppi^{2}$. 取$\,g'^{2}/4\uppi=0.0102\,$和 $g^{2}/4\uppi=0.0338$, 方程对$\,B\,$微子和$\,W\,$微子分别给出质量$\,8.9\times 10^{-3}m_{g}\,$和$\,2.7\times 10^{-3}m_{g}$. 
然而, 当超对称性因在拉格朗日量中引入$\,2\operatorname{Re}[\tilde{m}_{g}^{\ast}\mu(\mathcal{H}_{2}^{\mathrm{T}}e\mathscr{H}_{1})]\,$破缺后, $B\,$微子和$\,W\,$微子分别还有$\,g'^{2}m_{g}/16\uppi^{2}$ 和$\,g^{2}m_{g}/16\uppi^{2}\,$阶的贡献, 它们来自于$\,B\,$微子或$\,W\,$微子与一个希格斯玻色子--希格斯微子圈相连的图. 这给出了$\,B\,$微子或$\,W\,$微子质量\cite{11b}
\begin{align}
    m_{\text{bino}} &= \frac{g^{\prime 2}m_{g}}{16\uppi^{2}}\,\Biggl\lvert 11 -f\Biggl(\frac{\mu^{2}}{m_{A}^{2}}\Biggr)\Biggr\rvert \:, \label{31.4.15} \\
    m_{\text{wino}} &= \frac{g^{2}m_{g}}{16\uppi^{2}}\,\Biggl\lvert 1
    -f\Biggl(\frac{\mu^{2}}{m_{A}^{2}}\Biggr)\Biggr\rvert \:, \label{31.4.16}
\end{align}
其中$\,m_{A}\,$是方程(\ref{28.5.21})定义的赝标量粒子质量, 且
\begin{equation}
    f(x) \equiv \frac{2x\ln x}{x-1} \:. \label{31.4.17}
\end{equation}
在\,\ref{sec:31.7}\,节将进一步考虑这些结果的应用.

最后, 与任何左手征超场的动能拉格朗日密度$\,[\Phi_{r}^{\ast}\me^{-V}\Phi_{r}]_{D}\,$相乘的场重整化因子$\,Z_{r}\,$是标度相关的. 
通过将因子$\,Z_{r}^{1/2}\,$吸收进$\,\Phi_{r}\,$中, 这些因子可以从动能$\,D$-项移到超势$\,\mathscr{F}$-项. 这样, 三线性超势项$\,\sum_{rst}h_{rst}\Phi_{r}\Phi_{s}\Phi_{t}\,$中的\,Yukawa\,耦合(例如方程(\ref{28.1.7})中的$\,h_{ij}^{E}$, $h_{ij}^{D}\,$和$\,h_{ij}^{U}$)就要乘以一个依赖于截断$\,\mathscr{M}\,$的因子$\,Z_{r}^{-1/2}Z_{s}^{-1/2}Z_{t}^{-1/2}$. 根据方程(\ref{31.4.7}), 相互作用$\,\mathscr{L}^{(1)}\,$对拉格朗日密度就会有贡献
\begin{equation}
    \mathscr{L}_{\text{Yukawa}}^{(1)} = -2 \sum_{rst} \gamma_{rst} \operatorname{Re}
    \Bigl[ \tilde{m}_{g}^{\ast} h_{rst}\phi_{r}\phi_{s}\phi_{t}\Bigr] \:, \label{31.4.18}
\end{equation}
其中
\begin{align}
    \gamma_{rst} &\equiv \mathscr{M}\frac{\partial\ln h_{rst}(\mathscr{M})}{\partial\mathscr{M}} \nonumber \\
    &=-\frac{1}{2}\mathscr{M}\frac{\partial\ln Z_{r}(\mathscr{M})}{\partial\mathscr{M}}
    -\frac{1}{2}\mathscr{M}\frac{\partial\ln Z_{s}(\mathscr{M})}{\partial\mathscr{M}}
    -\frac{1}{2}\mathscr{M}\frac{\partial\ln Z_{t}(\mathscr{M})}{\partial\mathscr{M}} \:. \label{31.4.19}
\end{align}
我们看到方程(\ref{28.1.7})中的系数$\,A_{ij}^{E}$, $A_{ij}^{D}\,$和$\,A_{ij}^{N}\,$由
\begin{equation}
    A_{ij}^{N} = \tilde{m}_{g}^{\ast} \gamma_{ij}^{N} = \tilde{m}_{g}^{\ast}
    \mathscr{M}\frac{\partial\ln h_{ij}^{N}}{\partial\mathscr{M}} \label{31.4.20}
\end{equation}
给出, 其中$\,N=E$, $D\,$或$\,U$. 这给出的$\,A_{ij}^{D}\,$和$\,A_{ij}^{U}\,$是$\,g_{s}^{2}m_{g}/8\uppi^{2}\,$阶的, 而它给出的$\,A_{ij}^{E}\,$是$\,g^{2}m_{g}/8\uppi^{2}$ 或$\,g'^{2}m_{g}/8\uppi\,$阶的.


\section{定域超对称变换} \label{sec:31.5}

在考虑超对称变换规则上的$\,G\,$的高阶效应之前, 我们先来完成对引力超场$\,H_{\mu}\,$以及其他超场的物理分量的变换规则在$\,G\,$最低阶的讨论.

我们首先注意到, 当表示成物理场$\,h_{\mu\nu}$, $\psi_{\mu}$, $b_{\mu}$, $s\,$和$\,p\,$时, 这些变换规则取了``Wess--Zumino''规范, 在这个规范下,
\begin{equation}
    C_{\mu}^{H} = \omega_{\mu}^{H} = V_{\mu\nu}^{H}-V_{\nu\mu}^{H} = 0 \:. \label{31.5.1}
\end{equation}
利用物理场与超场$\,H_{\mu}\,$分量之间的关系(\ref{31.1.25})---(\ref{31.1.27}), (\ref{31.1.29})和(\ref{31.1.11}), 
连同一般变换规则(\ref{26.2.11})---(\ref{26.2.17})和``规范''条件(\ref{31.5.1}), 我们发现变换规则
\begin{align}
    \delta h_{\mu\nu} &= \tfrac{1}{2}\Bigl(\bar{\alpha}\,[\gamma_{\mu}\psi_{\nu}+\gamma_{\nu}\psi_{\mu}]\Bigr) \:, \label{31.5.2} \\
    \delta \psi_{\mu} &= \Bigl[ \tfrac{1}{2}[\gamma^{\nu},\gamma^{\lambda}]\partial_{\lambda}h_{\mu\nu}
    +\partial_{\mu}h^{\lambda}{}_{\lambda} +2\mi\,\gamma_{5}\,b_{\mu}
    -\tfrac{2}{3}\mi\gamma_{\mu}\gamma_{\rho}\gamma_{5}\,b^{\rho} \nonumber \\
    &\qquad +\tfrac{2}{3}\gamma_{\mu}\,s -\tfrac{2}{3}\mi\,\gamma_{\mu}\gamma_{5}\,p\Bigr] \alpha \:, \label{31.5.3} \\
    \delta s &= \tfrac{1}{4}\Bigl(\bar{\alpha}\,\gamma_{\lambda}\,L^{\lambda} \Bigr) \:, \label{31.5.4} \\
    \delta p &= -\tfrac{1}{4}\mi\Bigl(\bar{\alpha}\,\gamma_{5}\gamma_{\lambda}\,L^{\lambda}\Bigr) \:, \label{31.5.5}\\
    \delta b_{\mu} &= \tfrac{3}{4}\mi \Bigl(\bar{\alpha}\,\gamma_{5}\,L^{\mu}\Bigr)
    -\tfrac{1}{4}\mi\Bigl(\bar{\alpha}\,\gamma_{5}\gamma_{\mu}\gamma^{\nu} L_{\nu} \Bigr) \:, \label{31.5.6}
\end{align}
其中$\,L_{\mu}\,$由方程(\ref{31.2.8})给出:
\[
L^{\mu} \equiv \mi\,\epsilon^{\mu\nu\kappa\rho}\,\gamma_{5}\gamma_{\nu}\,\partial_{\kappa}\psi_{\rho} \:.
\]
这个变换对分量$\,C_{\mu}^{H}$, $\omega_{\mu}^{H}\,$和$\,V_{\mu\nu}^{H}-V_{\nu\mu}^{H}\,$有偏移
\begin{align}
    &\delta C_{\mu}^{H} = 0\:, \qquad \delta \omega_{\mu}^{H} = V_{\mu\nu}^{H}\gamma^{\nu}\alpha\:,\label{31.5.7} \\
    &\delta[V_{\nu\mu}^{H}-V_{\mu\nu}^{H}] = \Bigl(\bar{\alpha}\,
    [\gamma_{\mu}\lambda_{\nu}^{H}-\gamma_{\nu}\lambda_{\mu}^{H}] \Bigr) \:, \label{31.5.8}
\end{align}
使得它们不再是零. 通过一个合适的规范变换$\,H_{\mu}\to H_{\mu}+\Delta_{\mu}$, 其中$\,\Delta_{\mu}\,$是形如(\ref{31.1.17})和(\ref{31.1.18}) 的超场, 并有分量
\begin{align}
     &C_{\mu}^{\Delta} = 0\:, \qquad  \omega_{\mu}^{\Delta} = - V_{\mu\nu}^{H}\gamma^{\nu}\alpha\:,\label{31.5.9} \\
    &V_{\nu\mu}^{\Delta}-V_{\mu\nu}^{\Delta} = \Bigl(\bar{\alpha}\,
    [\gamma_{\nu}\lambda_{\mu}^{H}-\gamma_{\mu}\lambda_{\nu}^{H}] \Bigr) \:, \label{31.5.10}
\end{align}
我们可以回到满足方程(\ref{31.5.1})的规范. 



迄今为止, 这还只是整体超对称变换, 其中的$\,\alpha\,$是一个无限小的常\,Majorana\,旋量. 至少在最低阶, 这个对称性可以轻松地推广至{\kai{定域}}超对称变换, 其中$\,\alpha(x)\,$对$\,x^{\mu}\,$有任意的依赖关系. 根据方程(\ref{26.7.11}), 在这种变换下, 物质作用量的变化是
\begin{equation}
    \delta \int \dif^{4}x\:\mathscr{L}_{M} = -\int \dif^{4}x\:\Bigl(\bar{S}^{\mu}(x)\,\partial_{\mu}\alpha(x)\Bigr) \:.
    \label{31.5.11}
\end{equation}
对方程(\ref{31.2.17})的观察表明, 如果我们给方程(\ref{31.5.3})右边加上一次非齐次项$\,(2/\kappa)\partial_{\mu}\alpha(x)$, 作用量的这个变化就能被抵消, 这使得引力微子场的变换现在是
\begin{align}
    \delta\psi_{\mu}(x) &= (2/\kappa)\partial_{\mu}\alpha(x) + \Bigl[\tfrac{1}{2}[\gamma^{\nu},\gamma^{\lambda}]\partial_{\lambda}h_{\mu\nu}(x)
    +\partial_{\mu}h^{\lambda}{}_{\lambda}(x) +2\mi\,\gamma_{5}\,b_{\mu}(x) \nonumber \\
    &\qquad  -\tfrac{2}{3}\mi\gamma_{\mu}\gamma_{\rho}\gamma_{5}\,b^{\rho}(x) +\tfrac{2}{3}\gamma_{\mu}\,s(x) -\tfrac{2}{3}\mi\,\gamma_{\mu}\gamma_{5}\,p(x)\Bigr]\,\alpha(x) \:. \label{31.5.12}
\end{align}


重写方程(\ref{31.5.12})使得它能更显然地推广到广义坐标将是有用的. 首先, 我们注意到, 通过将参量$\,\alpha(x)\,$换成
\begin{equation}
    \tilde{\alpha} \equiv (\operatorname{Det}g)^{1/4}\alpha \simeq \alpha
    + \tfrac{1}{2}\kappa h^{\lambda}{}_{\lambda}\alpha \:. \label{31.5.13}
\end{equation}
方程(\ref{31.5.12})右边的$\,\partial_{\mu}h^{\lambda}{}_{\lambda}\alpha\,$项可以被消掉. 扔掉波浪符, 那么到$\,\kappa\,$的零阶, 方程(\ref{31.5.12})变成
\begin{align}
    \delta\psi_{\mu}(x) &= (2/\kappa)\partial_{\mu}\alpha(x) + \Bigl[\tfrac{1}{2}[\gamma^{\nu},\gamma^{\lambda}]\partial_{\lambda}h_{\mu\nu}(x)
     +2\mi\,\gamma_{5}\,b_{\mu}(x) \nonumber \\
    &\qquad  -\tfrac{2}{3}\mi\gamma_{\mu}\gamma_{\rho}\gamma_{5}\,b^{\rho}(x) +\tfrac{2}{3}\gamma_{\mu}\,s(x) -\tfrac{2}{3}\mi\,\gamma_{\mu}\gamma_{5}\,p(x)\Bigr]\,\alpha(x) \:, \label{31.5.14}
\end{align}
我们可以将这表示成$\,\alpha(x)\,$的协变导数, 在广义坐标下, 它采取形式
\begin{equation}
    D_{\mu}\alpha(x) = \partial_{\mu}\alpha(x) + \tfrac{1}{2}\mi\mathscr{J}_{bc}\,\omega_{\mu}^{bc}\,\alpha(x) \:,
    \label{31.5.15}
\end{equation}
其中$\,\mathscr{J}_{bc}\,$是在\,Dirac\,表示下表示\,Lorentz\,变换生成元的矩阵元(\textcolor{foo}{5.4.6})
\begin{equation}
    \mathscr{J}^{bc} \equiv -\frac{\mi}{4} \Bigl[\gamma^{b},\gamma^{c}\Bigr] \:, \label{31.5.16}
\end{equation}
而$\,\omega_{\mu}^{bc}(x)\,$是自旋联络,
\begin{equation}
    \omega_{\mu}^{bc} = e^{b}{}_{\lambda}\,e^{c}{}_{\nu;\mu}\,g^{\lambda\nu}
    =e^{b}{}_{\lambda}\,\frac{\slashed{\partial}e^{c}{}_{\nu}}{\partial x^{\mu}}\,g^{\lambda\nu}
    -\Gamma^{\rho}_{\nu\mu}\,e^{b}{}_{\lambda}\,e^{c}{}_{\rho}\,g^{\lambda\nu} \:. \label{31.5.17}
\end{equation}
利用弱场近似(\ref{31.1.5}), (\ref{31.1.10})和(\ref{31.1.11}), 连同规范条件$\,\phi_{\mu\nu}=\phi_{\nu\mu}$, 以及在这个近似下可以忽视定域\,Lorentz\,指标$\,a,b\,$等与时空指标$\,\mu,\nu\,$等之间的差异, 这给出
\begin{equation}
    D_{\mu}\alpha(x) \simeq \partial_{\mu}\alpha(x) + \tfrac{1}{4}\kappa\,[\gamma^{\nu},\gamma^{\lambda}]\,
    \partial_{\lambda}h_{\mu\nu}(x)\,\alpha(x) \:. \label{31.5.18}
\end{equation}
因此定域超对称变换规则(\ref{31.5.12})可以写成
\begin{align}
    \delta \psi_{\mu}(x) &= (2/\kappa)D_{\mu}\alpha(x) + \Bigl[2\mi\gamma_{5}\,b_{\mu}(x)-\tfrac{2}{3}\mi\gamma_{\mu}\gamma_{\rho}\gamma_{5}\,b^{\rho}(x) \nonumber \\
    &\qquad +\tfrac{2}{3}\gamma_{\mu}\,s(x)-\tfrac{2}{3}\mi\,\gamma_{\mu}\gamma_{5}\,p(x)\Bigr]\alpha(x) \:. \label{31.5.19}
\end{align}
我们看到在\,Wess--Zumino\,规范下, 超对称变换中的导数变成了协变导数. 在这个意义下, 这里概述的方法类似于\,\ref{sec:27.8}\,节讨论的超对称规范理论的\,de Wit--Freedman\,方法.

变换$\,\psi_{\mu}(x)\to\psi_{\mu}(x)+(2/\kappa)\partial_{\mu}\alpha(x)\,$是与方程(\ref{31.1.13})同类项的规范变换, 
所以它保持零阶引力微子作用量$\,-\frac{1}{2}\int\dif^{4}x(\bar{\psi}_{\mu}L^{\mu})\,$保持不变. 因此, 在定域超对称变换(\ref{31.5.2}), (\ref{31.5.4})---(\ref{31.5.6}), (\ref{31.5.12})(或(\ref{31.5.19})), 以及像(\ref{26.7.15})这样的物质超场变换下, 从拉格朗日密度(\ref{31.2.15})中获得的整个作用量直到$\,\kappa\,$的零阶是不变的. 我们由此得出: {\kai{引力与超对称的组合自动给出了定域超对称性}}.



\section{直到所有阶的超引力} \label{sec:31.6}

尽管从拉格朗日密度(\ref{31.2.15})中推出的作用量在\,\ref{sec:26.7}\,和$\,\ref{sec:31.5}\,$节构建的定域超对称变换下直到 $\,\kappa\equiv \sqrt{8\uppi G}\,$的零阶是不变的, 但是, 因为物质与引力超多重态的相互作用会在$\,\partial_{\mu}S^{\mu}\,$中引入$\,\kappa\,$阶项, 这个作用量在$\,\kappa\,$的一阶{\kai{不是}}不变的. 为了使得整个作用量有可能是不变的, 我们不得不在拉格朗日量以及物质超场和引力超场分量的超对称变换规则加入$\,\kappa\,$的高阶项. 实现这点的方法是, 先在变换规则引入$\,\kappa\,$的高阶项, 选择这些高阶项使得这些变换连同定域\,Lorentz\,变换和广义坐标变换构成封闭的代数, 然后在作用量加入$\,\kappa\,$的高阶项使得它在所有这些变换下不变.

这是一个漫长且繁琐的过程. 我们在这里只给出结果,\cite{12} 然后到下一节研究它们最重要的应用是什么. 标架, 引力微子场和辅助场的定域超对称变换采取形式:
\begin{align}
    \delta e^{a}{}_{\mu} &= \kappa\,\Bigl(\bar{\alpha}\,\gamma^{a}\psi_{\mu}\Bigr) \:, \label{31.6.1} \\
    \delta \psi_{\mu} &= (2/\kappa)D_{\mu}\alpha + 2\mi\,\gamma_{5}\,(b_{\mu}-\tfrac{1}{3}\gamma_{\mu}\gamma_{\rho}b^{\rho})\,
    \alpha + \tfrac{2}{3}\gamma_{\mu}(s-\mi\gamma_{5}p)\,\alpha \:, \label{31.6.2} \\
    \delta s &= \frac{1}{4e}\,\Bigl(\bar{\alpha}\,\gamma_{\mu}L^{\mu}\Bigr)+\frac{\kappa}{2} \Bigl(\bar{\alpha}\,
    [\mi\gamma_{5}\,b^{\nu}-s\,\gamma^{\nu}-\mi\,p\,\gamma_{5}\gamma^{\nu}]\psi_{\nu}\Bigr) \:, \label{31.6.3} \\
    \delta p &= -\frac{\mi}{4e}\,\Bigl(\bar{\alpha}\,\gamma_{5}\gamma_{\mu}L^{\mu}\Bigr)+\frac{\kappa}{2} \Bigl(\bar{\alpha}\,
    [b^{\nu}+ \mi\,s\,\gamma_{5}\gamma^{\nu}- p\,\gamma^{\nu}]\psi_{\nu}\Bigr) \:, \label{31.6.4} \\
    \delta b_{\mu} &= \frac{3\mi}{4e}\Bigl(\bar{\alpha}\gamma_{5}\,(L_{\mu}
    -\tfrac{1}{3}\gamma_{\mu}\gamma_{\rho}\,L^{\rho}) \Bigr) + \frac{\kappa}{2}\,b_{\nu}\,\Bigl(\bar{\alpha}\gamma^{\nu}\psi_{\mu}\Bigr) \nonumber \\
    &\quad +\mi\frac{\kappa}{2}\Bigl(\bar{\psi}_{\mu}\gamma_{5}(s-\mi\gamma_{5}p)\,\alpha\Bigr)
    -\mi\frac{\kappa}{4}\epsilon_{\mu\nu\kappa\sigma}b^{\nu}\,\Bigl(\bar{\alpha}\,
    \gamma_{5}\gamma^{\kappa}\psi^{\sigma}\Bigr) \:. \label{31.6.5}
\end{align}
这里的$\,D_{\mu}\,$依旧是(\ref{31.5.15})和(\ref{31.5.16})给出的协变导数:
\begin{equation}
    D_{\mu} \equiv \partial_{\mu} + \tfrac{1}{8}[\gamma_{a},\gamma_{b}]\,\omega_{\mu}^{ab} \:, \label{31.6.6}
\end{equation}
但现在自旋联络中含有引力微子场的双线性项
\begin{align}
    \omega^{ab}_{\mu} &= e^{a}{}_{\lambda}e^{b}{}_{\nu;\mu}\,g^{\lambda\nu} \nonumber \\
    &\quad+\frac{\kappa^{2}}{4}\Bigl[ e^{b}{}_{\nu}\,\Bigl(\bar{\psi}_{\mu}\gamma^{a}\psi^{\nu}\Bigr)
    +e^{a}{}_{\nu}e^{b}{}_{\rho}\,\Bigl(\bar{\psi}^{\nu}\gamma_{\mu}\psi^{\rho}\Bigr)
    -e^{a}{}_{\nu}\,\Bigl(\bar{\psi}_{\mu}\gamma^{b}\psi^{\nu}\Bigr)\Bigr] \:. \label{31.6.7}
\end{align}
另外, $L_{\mu}\,$是\,Rarita--Schwinger\,算符(\ref{31.2.8})的协变版本
\begin{equation}
    L^{\mu} = \mi\,\gamma_{5}\gamma_{\nu}D_{\rho}\psi_{\sigma}\epsilon^{\mu\nu\rho\sigma} \:, \label{31.6.8}
\end{equation}
$\gamma_{\mu}\,$用通常的\,Dirac\,矩阵$\,\gamma_{a}\,$定义成
\begin{equation}
    \gamma_{\mu} = e^{a}{}_{\mu}\gamma_{a} \:, \label{31.6.9}
\end{equation}
而这里的$\,e\,$是标架的行列式
\begin{equation}
    e\equiv \sqrt{\operatorname{Det} g} \:. \label{31.6.10}
\end{equation}
很容易看到这些变换在弱场极限下退化至方程(\ref{31.5.2}), (\ref{31.5.19})和(\ref{31.5.4})---(\ref{31.5.6}).

在这些变换下不变且在弱场近似下退化至(\ref{31.2.9})的纯超引力作用量是
\begin{equation}
    I_{\text{SUGRA}} = \int \dif^{4}x\:\biggl[ -\frac{e}{2\kappa^{2}}R -\frac{1}{2}\Bigl(\bar{\psi}_{\mu}L^{\mu}\Bigr)
    -\frac{4e}{3}\,\Bigl(s^{2}+p^{2}-b_{\mu}b^{\mu} \Bigr)\biggr] \:, \label{31.6.11}
\end{equation}
其中$\,R\,$是利用自旋联络(\ref{31.6.7})计算的曲率标量. 当没有物质时, 作用量在$\,s=p=b_{\mu}=0\,$时是稳定的, 这给了我们更加简单的作用量
\[
 I_{\text{SUGRA}} = \int \dif^{4}x\:\biggl[ -\frac{e}{2\kappa^{2}}R-\frac{1}{2}\Bigl(\bar{\psi}_{\mu}L^{\mu}\Bigr)\biggr] \:,
\]

物质场的变换现在也更加复杂了. 对一般标量超多重态的分量, 它们是
\begin{align}
    \delta C &= \mi\,\Bigl(\bar{\alpha}\,\gamma_{5}\,\omega\Bigr) \:, \label{31.6.12} \\
    \delta \omega &= [-\mi\gamma_{5}\,\slashed{\mathscr{D}}C-M+\mi\gamma_{5}N+\slashed{V}]\,\alpha \:, \label{31.6.13} \\
    \delta M &= -\Bigl(\bar{\alpha}\,[\lambda+\slashed{\mathscr{D}}\omega] \Bigr)+\frac{2\kappa}{3}\,
    \Bigl(\bar{\alpha}\,[s-\mi\,\gamma_{5}p+\mi\,\gamma_{5}\,\slashed{b}]\,\omega\Bigr) \:, \label{31.6.14} \\
    \delta N &= \mi\Bigl(\bar{\alpha}\,\gamma_{5}\,[\lambda+\slashed{\mathscr{D}}\omega] \Bigr)+\frac{2\mi\kappa}{3}\,
    \Bigl(\bar{\alpha}\,[s-\mi\,\gamma_{5}p+\mi\,\gamma_{5}\,\slashed{b}]\,\gamma_{5}\,\omega\Bigr) \:, \label{31.6.15}\\
    \delta V_{a} &= \Bigl(\bar{\alpha}\gamma_{a}\lambda\Bigr)+\Bigl(\bar{\alpha}\mathscr{D}_{a}\omega\Bigr)
    +\frac{\kappa}{3}\, \Bigl(\bar{\alpha}\,[s-\mi\,\gamma_{5}p+\mi\,\gamma_{a}\,\slashed{b}]\,\omega\Bigr) \:, \label{31.6.16}\\
    \delta \lambda &= -\frac{1}{4}[\gamma^{a},\gamma^{b}]\,\alpha\,F_{ab}+\mi\,D\,\gamma_{5}\,\alpha\:, \label{31.6.17} \\
    \delta D &= \mi\,\Bigl(\bar{\alpha}\,\gamma_{5}\,\slashed{\mathscr{D}}\lambda\Bigr)\:, \label{31.6.18}
\end{align}
这里的协变导数是
\begin{align}
    \mathscr{D}_{a} C &= e_{a}{}^{\mu}\,\biggl[\partial_{\mu}C
    -\frac{\mi\kappa}{2}\Bigl(\bar{\psi}_{\mu}\,\gamma_{5}\,\omega\Bigr)\biggr] \:, \label{31.6.19} \\
    \mathscr{D}_{a} \omega &= e_{a}{}^{\mu}\,\biggl[\partial_{\mu}\omega +\frac{1}{8}\omega_{\mu}^{cb}\,[\gamma_{c},\gamma_{b}]\,\omega -\mi\kappa\,b_{\mu}\gamma_{5}\,\omega \nonumber \\
    &\quad \qquad -\frac{\kappa}{2}\Bigl(\slashed{V}-\mi\,\gamma_{5}\,\slashed{\partial}C -M
    +\mi\gamma_{5}N\Bigr)\psi_{\mu}\biggr] \:, \label{31.6.20}  \\
    \mathscr{D}_{a} \lambda &= e_{a}{}^{\mu}\,\biggl[\partial_{\mu}\lambda  +\frac{1}{8}\omega_{\mu}^{cb}\,[\gamma_{c},\gamma_{b}]\,\lambda \nonumber \\
    &\quad \qquad +\mi\kappa\,b_{\mu}\gamma_{5}\,\lambda+\frac{\kappa}{8}[\gamma_{b},\gamma_{c}]\,\psi_{\mu}F_{bc}
    -\frac{\mi\kappa}{2}\gamma_{5}D\,\psi_{\mu}\biggr] \:, \label{31.6.21} \\
    F_{ab} &= e_{a}{}^{\mu}e_{b}{}^{\nu}\,\biggl[\partial_{\mu}V_{\nu}
    +\frac{\kappa}{2}\partial_{\mu}\Bigl(\bar{\psi}_{\nu}\omega\Bigr)
    -\frac{\kappa}{2}\Bigl(\bar{\psi}_{\mu}\lambda_{\nu}\lambda\Bigr)\biggr] - a\leftrightarrow b\:, \label{31.6.22}
\end{align}
其中$\,V_{\mu}\equiv e^{a}{}_{\mu}V_{a}$. 一般标量多重态相乘的规则与方程(\ref{26.2.19})---(\ref{26.2.25})几乎相同, 除了要把其中的$\,\partial_{\mu}\,$换成$\,\mathscr{D}_{a}$, 这使得超多重态$\,S=S_{1}S_{2}\,$的分量是
\begin{align}
    &C = C_{1}C_{2} \:, \label{31.6.23} \\
    &\omega = C_{1}\omega_{2}+C_{2}\omega_{1} \:, \label{31.6.24} \\
    &M = C_{1}M_{2}+C_{2}M_{1} +\tfrac{1}{2}\mi\,\Bigl(\overline{\omega_{1}}\,\gamma_{5}\,\omega_{2}\Bigr)\:, \label{31.6.25}\\
    &N= C_{1}N_{2}+C_{2}N_{1} -\tfrac{1}{2}\,\Bigl(\overline{\omega_{1}}\,\omega_{2}\Bigr)\:, \label{31.6.26}\\
    &V^{a}= C_{1}V_{2}^{a}+C_{2}V_{1}^{a}-\tfrac{1}{2}\mi
    \Bigl(\overline{\omega_{1}}\,\gamma_{5}\gamma^{a}\,\omega_{2}\Bigr) \:, \label{31.6.27} \\
    &\lambda = C_{1}\lambda_{2}+C_{2}\lambda_{1}-\tfrac{1}{2}\gamma^{a}\omega_{1}\mathscr{D}_{a}C_{2}
    -\tfrac{1}{2}\gamma^{a}\omega_{2}\mathscr{D}_{a}C_{1}+ \tfrac{1}{2}\mi\,\slashed{V}_{1}\gamma_{5}\,\omega_{2}\nonumber\\
    &\qquad + \tfrac{1}{2}\mi\,\slashed{V}_{2}\gamma_{5}\,\omega_{1}+\tfrac{1}{2}(N_{1}-\mi\gamma_{5}M_{1})\omega_{2}
    +\tfrac{1}{2}(N_{2}-\mi\gamma_{5}M_{2})\omega_{1} \:, \label{31.6.28} \\
    &D= -\mathscr{D}_{a}C_{1}\,\mathscr{D}^{a}C_{2}+C_{1}D_{2}+C_{2}D_{1}+M_{1}M_{2}+N_{1}N_{2} \nonumber \\
    &\qquad -\Bigl(\overline{\omega_{1}}\,[\lambda_{2}+\tfrac{1}{2}\,\slashed{\mathscr{D}}\omega_{2}]\Bigr)
    -\Bigl(\overline{\omega_{2}}\,[\lambda_{1}+\tfrac{1}{2}\,\slashed{\mathscr{D}}\omega_{1}]\Bigr)
    -V_{1a}V_{2}^{a} \:. \label{31.6.29}
\end{align}


就和平坦空间中一样, 这些一般超多重态上还有超对称约束. 其中一个约束就是实性. 对方程(\ref{31.6.12})---(\ref{31.6.22})和方程(\ref{26.A.20})---(\ref{26.A.21})的观察表明, 如果$\,C$, $M$, $N$, $V_{a}$, $D$, $\omega\,$和$\,\lambda\,$构成超多重态$\,S$, 那么$\,C^{\ast}$, $M^{\ast}$, $N^{\ast}$, $V_{a}^{\ast}$, $D^{\ast}$, $\beta\epsilon\gamma_{5}\omega^{\ast}\,$和$\,beta\epsilon\gamma_{5}\lambda^{\ast}\,$也是一个超多重态的分量, 我们记其为$\,S^{\ast}$. 特别地, 一个{\kai{实}}超多重态是$\,S=S^{\ast}\,$的超多重态, 这使得$\,C$, $M$, $N$, $V_{a}\,$和$\,D\,$是实的而$\,\omega\,$和$\,\lambda\,$的 Majorana 旋量.

我们也可以附加手征性的一个超对称条件. 假定我们令
\begin{equation}
    \lambda=0\:, \qquad D=0 \:, \qquad V_{\nu}+\tfrac{1}{2}\kappa\,\Bigl(\bar{\psi}_{\nu}\,\omega\Bigr)=\partial_{\nu}Z\:,
    \label{31.6.30}
\end{equation}
其中$\,Z\,$是一场. 那么方程(\ref{31.6.22})给出$\,F_{ab}=0$, 所以方程(\ref{31.6.17})表明$\,\delta\lambda=0$, 而方程(\ref{31.6.21})给出$\,\mathscr{D}_{a}\lambda=0$, 所以方程(\ref{31.6.18})表明$\,\delta D=0$. 因此定域超对称变换保持$\,\lambda=D=0\,$这个条件. 再稍微计算一下, 可以证明
\begin{equation}
    \delta\Bigl[V_{\nu}+\tfrac{1}{2}\kappa\,\Bigl(\bar{\psi}_{\nu}\,\omega\Bigr)\Bigr]
    =\partial_{\nu}\Bigl(\bar{\alpha}\omega\Bigr) \:,     \label{31.6.31}
\end{equation}
所以剩下的条件, 即$\,V_{\nu}+\frac{1}{2}\kappa\bigl(\bar{\psi}_{\nu}\,\omega\bigr)\,$是时空梯度, 也是超对称的. 满足方程(\ref{31.6.30})的分量场构成的超多重态被称为是{\kai{手征的}}. 就像整体超对称性中的情况一样, 手征超多重态的分量一般被记为$\,A$, $B$, $\psi$, $F\,$和$\,G$, 定义成
\begin{equation}
    C=A\:, \qquad \omega=-\mi\gamma_{5}\psi\:, \qquad M=G\:,\qquad N=-F\:, \qquad Z= B\:. \label{31.6.32}
\end{equation}
如果$\,A$, $B$, $F$\,和$\,G\,$是实的且$\,\psi\,$是\,Majorana\,旋量, 那么手征超多重态就是实的. 这种实手征超多重态可以写成一个左手征超多重态$\,\Phi\,$与其复共轭(一个右手征超多重态)的和, $\Phi,$的分量通常定义成
\begin{equation}
    \phi\equiv \frac{A+\mi B}{\sqrt{2}} \:, \qquad \psi_{L}\equiv \biggl(\frac{1+\gamma_{5}}{2}\biggr)\psi \:, \qquad
    \mathscr{F} \equiv \frac{F-\mi G}{\sqrt{2}} \:. \label{31.6.33}
\end{equation}
从这些定义和方程(\ref{31.6.12})---(\ref{31.6.15})以及(\ref{31.6.31}), 我们看到左手征超多重态的分量有变换性质
\begin{align}
    \delta\phi &= \sqrt{2}\,\Bigl(\bar{\alpha}\,\psi_{L}\Bigr) \:, \label{31.6.34} \\
    \delta\psi_{L} &= \sqrt{2}(\slashed{\partial}\phi)\alpha_{R} - \kappa\gamma^{\mu}\,\Bigl(\bar{\psi}_{\mu}\psi_{L}\Bigr)\alpha_{R} + \sqrt{2}\mathscr{F}\,\alpha_{L}\:,\label{31.6.35} \\
    \delta\mathscr{F} &= \sqrt{2}\Bigl(\bar{\alpha}\slashed{\mathscr{D}}\psi_{L}\Bigr) -\frac{2\kappa}{3}\,
    \Bigl(\bar{\alpha}\,[s-\mi p-\mi \slashed{b}]\psi_{L}\Bigr) \:, \label{31.6.36}
\end{align}
其中$\,\mathscr{D}_{a}\psi\,$由方程(\ref{31.6.20})和(\ref{31.6.32})给出. 左手征超场相乘的规则与整体超对称形下的相应规则(\ref{26.3.27})---(\ref{26.3.29})相同: 左手征超多重态$\,\Phi_{1}\,$和$\,\Phi_{2}\,$的乘积是被记做$\,\Phi_{1}\Phi_{2}\,$的左手征超多重态, 其有分量
\begin{align}
    \phi &= \phi_{1}\phi_{2} \:, \label{31.6.37} \\
    \psi_{L} &= \phi_{1}\psi_{2L}+\phi_{2}\psi_{1L} \:, \label{31.6.38} \\
    \mathscr{F} &= \phi_{1}\mathscr{F}_{2} +  \phi_{2}\mathscr{F}_{1}-\Bigl(\psi_{1L}^{\mathrm{T}}\,\epsilon\,\psi_{2L}\Bigr) \:. \label{31.6.39}
\end{align}

现在我们必须考虑如何构造在定域超对称变换, 广义坐标变换以及定域\,Lorentz\,变换下不变的作用量. 方程(\ref{31.6.18})和(\ref{31.6.21})表明一般超多重态$\,S\,$的$\,D$-分量在超对称变换下的变换不再是一个时空导数, 所以这种$\,D$-分量的积分就作为作用量中的一项不再合适. 取而代之, 从任意一个超场$\,S$, 我们可以构建一个密度, 使得它的积分{\kai{是}}超对称的,
\begin{align}
    [S]_{D} &= e\,\Biggl[ D^{S} - \frac{\mi\kappa}{2}\Bigl(\bar{\psi}^{\mu}\gamma_{\mu}\gamma_{5}\lambda^{S}\Bigr)
    +\frac{4\kappa}{3}\,[-sN^{S}+pM^{S}-b^{\mu}V_{\mu}^{S}] \nonumber \\
    &\qquad\quad-\frac{\mi\kappa}{3}\Bigl(\overline{\omega^{S}}\,\gamma_{5}\,\slashed{L}\Bigr)
    -\frac{\kappa^{2}}{4}\epsilon^{\mu\rho\sigma\tau}V_{\sigma}^{S}\Bigl(\bar{\psi}_{\rho}\gamma_{\tau}\psi_{\mu}\Bigr)
    \nonumber \\
    &\qquad\quad -\frac{\kappa^{2}}{8}\epsilon^{\mu\rho\sigma\tau}\Bigl(\overline{\omega^{S}}\psi_{\sigma}\Bigr)
    \Bigl(\bar{\psi}_{\rho}\gamma_{\tau}\psi_{\mu}\Bigr)\Biggr] -\frac{2\kappa^{2}}{3} C^{S}\,
    \mathscr{L}_{\text{SUGRA}} \:, \label{31.6.40}
\end{align}
其中$\,\mathscr{L}_{\text{SUGRA}}\,$是方程(\ref{31.6.11})中定义的超引力拉格朗日量:
\begin{equation}
     \mathscr{L}_{\text{SUGRA}} = -\frac{e}{2\kappa^{2}}R -\frac{1}{2}\Bigl(\bar{\psi}_{\mu}L^{\mu}\Bigr)
     -\frac{4e}{3}\Bigl(s^{2}+p^{2}-b_{\mu}b^{\mu}\Bigr) \:. \label{31.6.41}
\end{equation}
同理, 方程(\ref{31.6.36})和(\ref{31.6.20})表明左手征超场$\,X\,$的$\,\mathscr{F}$-分量在超对称变换下是不变的, 所以必须要给这个$\,\mathscr{F}$-项加项以使得密度的积分是超对称的:
\begin{align}
    [X]_{\mathscr{F}} &= e\,\Biggl[\mathscr{F}^{X} + \frac{\kappa}{\sqrt{2}}\Bigl(\bar{\psi}_{\mu R}\gamma^{\mu}\psi_{L}^{X}\Bigr) + \frac{\kappa^{2}}{4}\Bigl(\bar{\psi}_{\mu R}[\gamma^{\mu},\gamma^{\nu}]\psi_{\nu R}\Bigr) \phi^{X} \nonumber \\
    &\quad \qquad +2\kappa (s-\mi p)\phi^{X}\Biggr] \:. \label{31.6.42}
\end{align}


存在物理上特别感兴趣的一些超多重态. 一个是非手征实超多重态$\,\textsc{\textbf{i}}$, 它唯一的非零分量是 $C=1$. 根据方程(\ref{31.6.40}), 这个超多重态有
\begin{equation}
    [\textsc{\textbf{i}}]_{D} = -\frac{2\kappa^{2}}{3} \mathscr{L}_{\text{SUGRA}} \:, \label{31.6.43}
\end{equation}
所以这并没有给出新东西.


一个更加有趣的例子是唯一非零分量是$\,\phi=1\,$的{\kai{左手征}}超多重态$\,\textbf{I}$. 根据方程(\ref{31.6.42}), 这个超多重态有
\begin{equation}
    \operatorname{Re}[\textbf{I}]_{\mathscr{F}} = e \Biggl[\frac{\kappa^{2}}{4}\Bigl(\bar{\psi}_\mu
    [\gamma^{\mu},\gamma^{\nu}]\psi_{\nu}\Bigr) + 4\,\kappa\,s \Biggr] \:. \label{31.6.44}
\end{equation}
如果拉格朗日密度中有一项$\,c\int\dif^{4}x\:\operatorname{Re}[\textbf{I}]_{\mathscr{F}}\,$且系数$\,c\,$为实数, 那么这个项与超引力作用量(\ref{31.6.11}) 的和相对辅助场在
\begin{equation}
    s=3\,c\,\kappa/2\:, \qquad p =b^{\mu}=0  \label{31.6.45}
\end{equation}
处的变分是稳定的, 所以消掉辅助场后, 作用量将会包含一个宇宙学常数项
\begin{equation}
    3\,\kappa^{2}\,c^{2}\int\dif^{4}x\:e \:, \label{31.6.46}
\end{equation}
这对应真空能密度$\,-3\kappa^{2}c^{2}$. 需要这样的项是为了在超对称自发破缺时保持真空态\,Lorentz\,不变; 为了抵消与超对称破缺相应的正真空能密度$\,F^{2}/2$, 我们必须取
\begin{equation}
    3\,\kappa^{2}\,c^{2} =F^{2}/2 \:. \label{31.6.47}
\end{equation}
回过来看方程(\ref{31.6.44})并与方程(\ref{31.3.11})相比较, 我们看到这给出引力微子质量\cite{13}
\begin{equation}
    m_{g} =c\,\kappa^{2}=\frac{F\,\kappa}{\sqrt{6}}=\sqrt{\frac{4\uppi G\,F^{2}}{3}} \:, \label{31.6.48}
\end{equation}
这与我们之前的结果(\ref{31.3.17})一致.

作为一个例子来说明如何使用这些公式以及提供下一节需要的一些结果, 我们来计算拉格朗日密度的玻色部分\footnote{$\mathscr{L}_{\text{SUGRA}}$\,这一项通常会省略掉, 取而代之, 可以通过在$\,K(\phi,\phi^{\ast})\,$中加入常数项$\,-3/\kappa^{2}\,$来引入超引力拉格朗日量. 我们不遵循这个惯例; 在这里, 在标量场按惯例归一化后, $K(\phi,\phi^{\ast})\,$中$\,\kappa\,$的领头项是$\,\sum_{n}\lvert\phi_{n}\rvert^{2}$.}
\begin{equation}
    \mathscr{L} = \mathscr{L}_{\text{SUGRA}} + \tfrac{1}{2}[K(\Phi,\Phi^{\ast})]_{D} +
    2\operatorname{Re}[f(\Phi)]_{\mathscr{F}} \:, \label{31.6.49}
\end{equation}
其中$\,\mathscr{L}_{\text{SUGRA}}\,$在这里是超引力拉格朗日密度(\ref{31.6.41})的玻色部分, $K(\Phi,\Phi^{\ast})\,$是一组左手征超多重态$\,\Phi_{n}\,$及其复共轭的实函数, 而$\,f(\Phi)\,$仅是$\,\Phi_{n}\,$的函数. 乘积规则(\ref{31.6.23}), (\ref{31.6.25})---(\ref{31.6.27})以及(\ref{31.6.29})中的纯玻色项与整体超对称性中的相同, 所以我们既可以使用这些规则也可以使用第 \ref{cha:26}\,章中的超空间形式体系来计算超多重态$\,K(\Phi,\Phi^{\ast})\,$有玻色分量
\begin{align}
    C^{K} &= K(\phi,\phi^{\ast}) \:, \label{31.6.50} \\
    M^{K} &= -2\operatorname{Im} \sum_{n}\biggl(\frac{\partial K(\phi,\phi^{\ast})}{\partial\phi_{n}}\mathscr{F}_{n}\biggr)
    +\cdots \:, \label{31.6.51} \\
    N^{K} &= -2\operatorname{Re} \sum_{n}\biggl(\frac{\partial K(\phi,\phi^{\ast})}{\partial\phi_{n}}\mathscr{F}_{n}\biggr)
    +\cdots \:, \label{31.6.52} \\
    V_{\mu}^{K} &=  2\operatorname{Im} \sum_{n}\biggl(\frac{\partial K(\phi,\phi^{\ast})}{\partial\phi_{n}}\partial_{\mu}\phi_{n}\biggr)+\cdots \:, \label{31.6.53} \\
    D^{K} &= 2\sum_{nm}\frac{\partial^{2}K(\phi,\phi^{\ast})}{\partial\phi_{n}\partial\phi_{m}}
    \Bigl(-g^{\mu\nu}\,\partial_{\mu}\phi_{n}\partial_{\nu}\phi_{m}^{\ast}+\mathscr{F}_{n}\mathscr{F}_{m}^{\ast}\Bigr)
    +\cdots \:, \label{31.6.54}
\end{align}
其中省略号是指包含费米子场的项. 另外, 左手征超多重态相乘的规则(\ref{31.6.37})---(\ref{31.6.39})与整体超对称性下的那些规则相同, 所以通过这些规则或者第 \ref{cha:26}\,章中的超场形式体系, 我们可以计算出$\,f(\Phi)\,$有玻色分量 
\begin{align}
    \phi^{f} &= f(\phi) \:, \label{31.6.55} \\
    \mathscr{F}^{f} &= \sum_{n}\frac{\partial f(\phi)}{\partial\phi_{n}}\mathscr{F}_{n} +\cdots \:, \label{31.6.56}
\end{align}
其中省略号依旧表示包含费米子的项. 将这些结果代入方程(\ref{31.6.40})和(\ref{31.6.42})就给出了拉格朗日密度(\ref{31.6.49})的玻色项
\begin{align}
    \mathscr{L}_{\text{bosonic}} &= \Biggl[-\frac{e}{2\kappa^{2}}R - \frac{4e}{3}\Bigl(s^{2}+p^{2}-b_{\mu}b^{\mu}\Bigr)\Biggr]\,
    \Biggl[1-\frac{\kappa^{2}}{3}K(\phi,\phi^{\ast})\Biggr] \nonumber\\
    &\quad -e\sum_{nm}\frac{\partial^{2}K(\phi,\phi^{\ast})}{\partial\phi_{n}\partial\phi_{m}^{\ast}}
    \Bigl(g^{\mu\nu}\,\partial_{\mu}\phi_{n}\partial_{\nu}\phi_{m}^{\ast}-\mathscr{F}_{n}\mathscr{F}_{m}^{\ast}\Bigr)
    \nonumber \\
    &\quad+\frac{4\,\kappa\,e}{3}\operatorname{Re}\sum_{n}\frac{\partial K(\phi,\phi^{\ast})}{\partial\phi_{n}}
    \Bigl(\mathscr{F}_{n}(s+\mi p)+\mi b^{\mu}\partial_{\mu}\phi_{n}\Bigr) \nonumber \\
    &\quad+ 2e\operatorname{Re}\Biggl(\sum_{n}\frac{\partial f(\phi)}{\partial\phi_{n}}\mathscr{F}_{n}
    +2\kappa(s-\mi p)\,f(\phi) \Biggr ) \:. \label{31.6.57}
\end{align}
现在我们必须消掉辅助场, 方法是取它们使得拉格朗日密度稳定的值.\footnote{尽管这是拉格朗日量是辅助场二次型时的通用规则, 但它不是严格正确的, 辅助场二阶项的系数不是与场无关的. 结果是, 在对辅助场做路径积分时, 我们会遇到二次项系数的行列式, 这等于在拉格朗日量加入正比于$\,\delta^{4}(0)=(2\uppi)^{-4}\int\dif^{4}k\:1\,$的项. 通过使用使得$\,\int\dif^{4}k\:1=0\,$的维度正规化, 这种项可以被消掉.} 这给出了辅助场
\begin{align}
    \mathscr{F}_{n} &= \frac{\kappa^{2}}{3N}\sum_{m}\Bigl(\mathscr{G}^{-1}\Bigr)_{mn}\frac{\partial K}{\partial \phi_{m}^{\ast}} \Biggl(-\sum_{k\ell} \Bigl(\mathscr{G}^{-1}\Bigr)_{k\ell} \,
    \frac{\partial K}{\partial \phi_{\ell}} \,\biggl(\frac{\partial f}{\partial \phi_{k}}\biggr)^{\ast}
    +3f^{\ast} \Biggr) \nonumber \\
    &\quad -\sum_{m}\Bigl(\mathscr{G}^{-1}\Bigr)_{mn} \biggl(\frac{\partial f}{\partial \phi_{m}}\biggr)^{\ast} \:,\label{31.6.58}  \\
    s-\mi p &= \frac{\kappa}{2N} \Biggl(-\sum_{k\ell} \Bigl(\mathscr{G}^{-1}\Bigr)_{k\ell}  \,
    \frac{\partial K}{\partial \phi_{\ell}} \,\biggl(\frac{\partial f}{\partial \phi_{k}}\biggr)^{\ast}
    +3f^{\ast} \Biggr) \:, \label{31.6.59} \\
    b_{\mu} &= \frac{\kappa}{2(1-\kappa^{2}K/3)}\operatorname{Im}\Biggl(\sum_{n}\frac{\partial K}{\partial\phi_{n}}\partial_{\mu}\phi_{n}\Biggr) \:, \label{31.6.60}
\end{align}
其中
\[
    N\equiv 1- \frac{\kappa^{2}}{3}K + \frac{\kappa^{2}}{3}\sum_{k\ell}\Bigl(\mathscr{G}^{-1}\Bigr)_{k\ell}  \,
    \frac{\partial K}{\partial \phi_{\ell}} \frac{\partial K}{\partial \phi_{k}^{\ast}}
\]
而$\,\mathscr{G}(\phi,\phi^{\ast})\,$是\,Kahler\,度规
\begin{equation}
    \mathscr{G}_{nm}(\phi,\phi^{\ast}) \equiv \frac{\partial^{2}K(\phi,\phi^{\ast})}{\partial\phi_{n}\partial\phi_{m}^{\ast}}\:, \label{31.6.61}
\end{equation}
在方程(\ref{31.6.57})使用上式就给出了玻色拉格朗日量
\begin{align}
    \mathscr{L}_{\text{bosonic}} &= -\frac{e}{2\kappa^{2}}R\Biggl[1-\frac{\kappa^{2}}{3}K\Biggr]
    -e\sum_{nm}\mathscr{G}_{nm}\,g^{\mu\nu}\,\partial_{\mu}\phi_{n}\partial_{\nu}\phi_{m}^{\ast} \nonumber \\
    &\quad +\frac{e\kappa^{2}}{3N} \Biggl\lvert \sum_{mn} \Bigl(\mathscr{G}^{-1}\Bigr)_{nm}\frac{\partial f}{\partial\phi_{m}}
    \frac{\partial K}{\partial\phi_{n}^{\ast}} - 3f\Biggr\rvert^{2} \nonumber \\
    &\quad - \frac{\kappa}{2(1-\kappa^{2}K/3)}\operatorname{Im}\Biggl[\sum_{n}\frac{\partial K}{\partial\phi_{n}}\partial_{\mu}\phi_{n}\Biggr] \operatorname{Im}\Biggl[\sum_{n}\frac{\partial K}{\partial\phi_{n}}\partial_{\nu}\phi_{n}\Biggr]\,g^{\mu\nu} \nonumber \\
    &\quad -e \sum_{mn}\Bigl(\mathscr{G}^{-1}\Bigr)_{nm}\frac{\partial f}{\partial\phi_{m}}
    \biggl(\frac{\partial f}{\partial\phi_{n}}\biggr)^{\ast} \:. \label{31.6.62}
\end{align}

正如在\,\ref{sec:31.2}\,节的弱场近似下已经注意到的, 拉格朗日密度(\ref{31.6.62})有一个让人不满意的特征: Einstein--Hilbert 项$\,-eR/2\kappa^{2}\,$前面有一个因子$\,(1-\kappa^{2}K(\phi,\phi^{\ast})/3)$, 这使得引力常数在时空的不同点是不同的. 为了补救这点, 我们进行一个\,Weyl\,变换, 定义新度规
\begin{equation}
    \tilde{g}_{\mu\nu} =\Bigl(1-\kappa^{2}K/3\Bigr)\,g_{\mu\nu}  \label{31.6.63}
\end{equation}
表示成新度规, Einstein\,拉格朗日密度是
\[
    eg^{\mu\nu}R_{\mu\nu} = \Biggl(1-\frac{\kappa^{2}}{3}K\Biggr)^{-1} \,\tilde{e}\tilde{g}^{\mu\nu}
    \Biggl(\tilde{R}_{\mu\nu}+\frac{3}{2}\partial_{\mu}\ln\biggl(1-\frac{\kappa^{2}}{3}\biggr)
    \partial_{\nu}\ln\biggl(1-\frac{\kappa^{2}}{3}\biggr)\Biggr) \:,
\]
其中$\,\tilde{R}_{\mu\nu}\,$是用度规$\,\tilde{g}_{\mu\nu}\,$而非$\,g_{\mu\nu}\,$计算的曲率张量, 而$\,\tilde{e}\equiv \sqrt{\operatorname{Det}\tilde{g}}$. 这样, 一个直接的计算就给出玻色拉格朗日量(\ref{31.6.62})是
\begin{align}
    \mathscr{L}_{\text{bosonic}} &= -\frac{\tilde{e}}{2\kappa^{2}}\tilde{g}^{\mu\nu}\tilde{R}_{\mu\nu} \nonumber \\
    &\quad -\tilde{e}\tilde{g}^{\mu\nu}\sum_{nm} \partial_{\mu}\phi_{n}\partial_{\nu}\phi_{m}^{\ast}
    \left[ \Biggl(1-\frac{\kappa^{2}}{3}K\Biggr)^{-1}\,\frac{\partial^{2}K}{\partial\phi_{n}\partial\phi_{m}^{\ast}}\right.
    \nonumber \\
    &\qquad \qquad \quad \left.+\frac{\kappa^{2}}{3}\Biggl(1-\frac{\kappa^{2}}{3}K\Biggr)^{-2}\,\frac{\partial K}{\partial\phi_{n}}
    \frac{\partial K}{\partial\phi_{m}^{\ast}} \right] \nonumber \\
    &\quad +\frac{\tilde{e}\kappa^{2}}{3N}\Biggl(1-\frac{\kappa^{2}}{3}K\Biggr)^{-2}\,
    \Biggl\lvert \sum_{mn} \Bigl(\mathscr{G}^{-1}\Bigr)_{nm}\frac{\partial f}{\partial\phi_{m}}
    \frac{\partial K}{\partial\phi_{n}^{\ast}} - 3f\Biggr\rvert^{2} \nonumber \\
    &\quad -\tilde{e}\Biggl(1-\frac{\kappa^{2}}{3}K\Biggr)^{-2}\,
    \sum_{mn}\Bigl(\mathscr{G}^{-1}\Bigr)_{nm}\frac{\partial f}{\partial\phi_{m}}
    \biggl(\frac{\partial f}{\partial\phi_{n}}\biggr)^{\ast} \label{31.6.64}
\end{align}
Weyl\,变换不仅从\,Einstein--Hilbert\,项中移除了因子$\,(1-\kappa^{2}K/3)$; 它还消除了正比于$\,\partial_{\mu}\phi_{n}\partial_{\nu}\phi_{m}\,$和 $\partial_{\mu}\phi_{n}^{\ast}\partial_{\nu}\phi_{m}^{\ast}\,$的项.


通过引入{\kai{修正}}\,\emph{Kahler}\,{\kai{势}}$\,d(\phi,\phi^{\ast})\,$来取代$\,K(\phi,\phi^{\ast})$, 这个结果可以进一步简化, 这个修正\,Kahler 势定义成
\begin{equation}
    1-\frac{\kappa^{2}}{3}K \equiv \exp\Biggl(-\frac{\kappa^{2}d}{3}\Biggr) \:. \label{31.6.65}
\end{equation}
我们也在标量场空间上引入一个新度规
\begin{equation}
    g_{nm} \equiv \frac{\partial^{2}d}{\partial\phi_{n}\partial\phi_{m}^{\ast}} \:. \label{31.6.66}
\end{equation}
新旧度规的逆之间的关系是
\[
\mathscr{G}^{-1}_{\ell k} = \exp (\kappa^{2}d/3)\,\Biggl[g_{\ell k}^{-1}
+\frac{\kappa^{2}}{3}\frac{\sum_{mn}g_{\ell n}^{-1}g_{mk}^{-1}(\partial d/\partial\phi_{n})(\partial d/\partial\phi_{m}^{\ast})}{1-(\kappa^{2}/3)\sum_{mn}g_{mn}^{-1}(\partial d/\partial\phi_{n})(\partial d/\partial\phi_{m}^{\ast})}\Biggr] \:.
\]
玻色拉格朗日密度(\ref{31.6.64})现在采取更简单的形式
\begin{equation}
     \mathscr{L}_{\text{bosonic}} = -\frac{\tilde{e}}{2\kappa^{2}}\tilde{g}^{\mu\nu}\tilde{R}_{\mu\nu}
     -\tilde{e}\tilde{g}^{\mu\nu}\sum_{nm}g_{nm}\partial_{\mu}\phi_{n}\partial_{\nu}\phi_{m}^{\ast} -\tilde{e}V\:, \label{31.6.67}
\end{equation}
其中$\,V(\phi,\phi^{\ast})\,$是势
\begin{equation}
    V=\exp(\kappa^{2}d)\Biggl[\sum_{nm}g_{nm}^{-1}L_{m}L_{n}^{\ast} - 3\kappa^{2}\,\lvert f\rvert^{2}\Biggr] \label{31.6.68}
\end{equation}
而
\begin{equation}
    L_{m} \equiv \frac{\partial f}{\partial \phi_{m}}+\kappa^{2}\,f\frac{\partial d}{\partial\phi_{m}} \:. \label{31.6.69}
\end{equation}


势(\ref{31.6.68})在场强满足条件$\,L_{m}=0\,$时有一个显然的稳定点. 然而, 正如我们在弱场情况下发现的, 在这个点, 真空能一般取负值$\,-3\kappa^{2}\lvert f\rvert^{2}$. 为了使稳定点$\,L_{m}=0\,$给出平坦空间的解, $f(\phi)\,$和$\,\partial f(\phi)/\partial \phi_{n}\,$必须要在这些场值处为零. 对方程(\ref{31.6.58})和(\ref{31.6.59})的观察表明, 标量辅助场$\,\mathscr{F}_{n}$, $s\,$和$\,p\,$在这些场值处为零, 这使得引力微子和手征旋量场在$\,\alpha\,$为常数的整体超对称变换下为零. 因此使得$\,f(\phi)\,$和$\,\partial f(\phi)/\partial\phi_{n}\,$全为零的真空场值是使得整体超对称形在经典极限下不破缺的真空场值. 在下一节, 我们将考虑超对称{\kai{确实}}破缺的真空场构形.

我们不会在这里证明, 但除了一个总的行列式因子$\,\tilde{e}\,$和用来升降指标的度规因子, 为引入规范超场而在玻色拉格朗日量中引入的项不受引力影响. 在消除辅助场以及做一个\,Weyl\,变换后, 一个有规范超场, 手征超场以及引力超场的理论, 它的完整玻色拉格朗日量是
\begin{align}
    \mathscr{L}_{\text{bosonic}}/\tilde{e} &= -\frac{1}{2\kappa^{2}}\,\tilde{R}^{\mu}{}_{\mu}-
    \sum_{nm}g_{nm}D_{\mu}\phi_{n}D^{\mu}\phi_{m}^{\ast} -\tfrac{1}{4}\sum_{AB}\operatorname{Re}f_{AB}\,F_{\mu\nu}^{A}
    F^{B\mu\nu} \nonumber \\
    &\quad -\tfrac{1}{8}\sum_{AB}\operatorname{Im}f_{AB}\,F^{A}_{\mu\nu}F^{B}_{\rho\sigma}\epsilon^{\mu\nu\rho\sigma}
    -V \:. \label{31.6.70}
\end{align}
这里$\,D_{\mu}$, $F_{\mu\nu}^{A}\,$和$\,t_{A}\,$分别指规范协变导数, 场强张量, 规范生成元在手征标量场上的表示, 这里所用的符号约定是\,15.1\,节中描述的; $f_{AB}\,$是$\,\phi_{n}\,$的独立全纯函数; 所有时空指标用$\,\tilde{g}_{\mu\nu}\,$进行升降; 而势$\,V\,$现在采取形式
\begin{align}
    V&=\exp(\kappa^{2}d)\Biggl[\sum_{nm}g_{nm}^{-1}L_{m}L_{n}^{\ast} - 3\kappa^{2}\,\lvert f\rvert^{2}\Biggr] \nonumber\\
    &\quad +\frac{1}{2}\operatorname{Re}\sum_{AB}f_{AB}^{-1}\,\Biggl(\sum_{nm}\frac{\partial d}{\partial\phi_{n}}(t_{A})_{nm}\phi_{m}\Biggr) \Biggl(\sum_{kl}\frac{\partial d}{\partial\phi_{k}}(t_{B})_{kl}\phi_{l}\Biggr)^{\ast} \:. \label{31.6.71}
\end{align}

玻色势的形式(\ref{31.6.71})足够简单以至于$\,d\,$中仅依赖于$\,\phi\,$或$\,\phi_{n}^{\ast}\,$的项可以用来交换超势的修正. 特别地, 如果我们写下
\begin{equation}
    d(\phi,\phi^{\ast})=\tilde{d}(\phi,\phi^{\ast})+a(\phi)+a(\phi)^{\ast} \:, \qquad
    f(\phi) = \tilde{f}(\phi)\exp\Bigl(-\kappa^{2}\,a(\phi)\Bigr)\:, \label{31.6.72}
\end{equation}
其中$\,a(\phi)\,$是满足规范不变条件
\[
\sum_{nm}\frac{\partial a(\phi)}{\partial\phi_{n}}(t_{A})_{nm}\phi_{m} 
\]
的任意全纯函数, 这样势(\ref{31.6.71})表示成$\,\tilde{d}\,$和$\,\tilde{f}\,$的形式就和表示成$\,d\,$和$\,f\,$的形式相同. 在超势合适的重定义下, 我们就可以消掉$\,d(\phi,\phi^{\ast})\,$的幂级数展开中任何只依赖于$\,\phi_{n}\,$或任何只依赖于$\,\phi_{n}^{\ast}\,$的项. 有了这个理解, $d(\phi,\phi^{\ast})\,$(现在扔掉波浪符)的幂级数展开中的领头项形如$\,\sum_{nm}d_{nm}\phi_{n}\phi_{m}^{\ast}$. 通过对超场做一个合适的线性变换, 我们就能把矩阵$\,d_{nm}\,$变成$\,\delta_{nm}$, 这使得$\,d(\phi,\phi^{\ast})\,$的幂级数展开始于
\begin{equation}
    d(\phi,\phi^{\ast}) = \sum_{n}\lvert \phi_{n}\rvert^{2} +\cdots \:, \label{31.6.73}
\end{equation}
而度规(\ref{31.6.66})的幂级数展开始于
\begin{equation}
    g_{nm} = \delta_{nm} + \cdots \:. \label{31.6.74}
\end{equation}
对方程(\ref{31.6.67})右边第二项的观察表明以这种方式定义的标量场是正则归一化的.


费米子项要复杂得多. 这里我们只引用规范微子场的二次项
\begin{align}
    \mathscr{L}^{(2)}_{\text{gaugino}}/\tilde{e} &= -\frac{1}{2}\operatorname{Re}\sum_{AB}f_{AB}\,
    \Bigl(\bar{\lambda}_{A}\slashed{D}\lambda_{B}\Bigr) \nonumber \\
    &\quad +\frac{1}{2}\exp(\kappa^{2}d/2)\operatorname{Re}\sum_{mn}\sum_{AB}g_{nm}^{-1}L_{m}
    \biggl(\frac{\partial f_{AB}}{\partial\phi_{n}}\biggr)^{\ast} \,\Bigl(\bar{\lambda}_{A}\lambda_{B}\Bigr)\:,
    \label{31.6.75}
\end{align}
其中$\,L_{m}\,$由方程(\ref{31.6.69})给出. 我们看到, 如果规范场是正则归一化的, 那么$\,f_{AB}\,$在对标量场的幂级数展开中的常数项就是$\,\delta_{AB}$, 这样规范微子场$\,\lambda_{A}\,$也是正则归一化的.



\subsection*{* * *}
取代将$\,d(\phi,\phi^{\ast})\,$及其复共轭中的所有全纯项移到超势中, 我们可以使用变换(\ref{31.6.72})把新超势取成常数, 通过取$\,a(\phi)=-\kappa^{-2}\ln f(\phi)$, 这个常数可以被选为\,1. 这样, 势就只依赖于函数
\begin{equation}
    \mathscr{D}(\phi,\phi^{\ast}) \equiv d(\phi,\phi^{\ast}) +2\,\kappa^{-2}\operatorname{Re}\ln f(\phi) \:, \label{31.6.76}
\end{equation}
并取如下的形式
\begin{align}
    V &= \exp(\kappa^{2}\mathscr{D})\Biggl[\kappa^{4}\sum_{nm}g_{nm}^{-1}\,
    \biggl(\frac{\partial\mathscr{D}}{\partial\phi_{m}}\biggr)\,
     \biggl(\frac{\partial\mathscr{D}}{\partial\phi_{n}}\biggr)^{\ast}-3\kappa^{2} \Biggr] \nonumber \\
     &\quad +\frac{1}{2}\operatorname{Re}\sum_{AB}f_{AB}^{-1}\Biggl(\sum_{nm}\frac{\partial \mathscr{D}}{\partial\phi_{n}}(t_{A})_{nm}\phi_{m}\Biggr) \Biggl(\sum_{kl}\frac{\partial \mathscr{D}}{\partial\phi_{k}}(t_{B})_{kl}\phi_{l}\Biggr)^{\ast}  \label{31.6.77}
\end{align}
另外, 标量场的度规(\ref{31.6.66})可以写成
\begin{equation}
    g_{nm} = \frac{\partial^{2} \mathscr{D}}{\partial\phi_{n}\partial\phi_{m}^{\ast}} \:. \label{31.6.78}
\end{equation}
尽管我们没有在这里证明, 但对称性允许我们将\,Kahler\,势和超势换成单个函数$\,\mathscr{D}(\phi,\phi^{\ast})$, 也允许我们在包含费米子和规范场的整个拉格朗日量中做这个替换.


有一类有趣的``无标度''理论,\cite{13a}, 在这类理论中, 势$\,V\,$对$\,\phi_{m}\,$的所有值都为零. 例如, 对单个规范中性手征标量超场, 当其有
\begin{equation}
    \mathscr{D} = -3\kappa^{-2}\ln\Bigl(h(\phi)+h(\phi)^{\ast}\Bigr) \:,\label{31.6.79}
\end{equation}
其中$\,h(\phi)\,$是$\,\phi\,$的任意函数, 就是这样的情况. 但没有已知的原理要求$\,\mathscr{D}\,$取这样的形式.



\section{引力传递的超对称破缺} \label{sec:31.7}

我们现在再次处理超对称破缺的问题. 正如第\,\ref{cha:28}\,章开头所讨论的, 如果超对称被用来解决等级问题------即, 理解\,Planck\,质量$\,m_{\text{Pl}}\equiv 1/\sqrt{8\uppi G}\,$与观测的粒子的质量标度的大比值------那么, 超对称在\,Planck\,标度处必须不破缺, 而是在某个低得多的能量标度处自发破缺. 唯一已知看似合理的能够自然产生质量标度之间大比值的机制是渐进自由规范相互作用的非微扰效应. 如果这些相互作用在\,Planck\,标度不是特别弱, 那么它们会随着能量降低而缓慢增长, 这将使得它们在低得多的标度$\,\Lambda\ll m_{\text{Pl}}\,$处变强. 已知的基本粒子并没有受到这么强的力, 所以无论超对称破缺是由这些强规范相互作用直接还是间接产生的, 它必须通过可观测粒子参与的某个相互作用与其进行交互.


在\,\ref{sec:28.3}\,节, 我们注意到超对称破缺与可观测粒子由两种可能的交互机制. 一个机制是在\,\ref{sec:28.6} 节细致讨论过的规范传递的超对称破缺. 我们现在就可以讨论另一个机制, 引导超对称破缺的是引力强度的效应.

在\,20\,世纪\,80\,年代早期, 当引力首次被考虑为超对称破缺的媒介时,\cite{14} 通常假定超势由两项构成: {\kai{可观测分区}}中的各种左手征超场$\,\Phi_{r}\,$的函数$\,f(\Phi)$, 这包含了所有可观测粒子的超场, 再加上{\kai{隐藏分区}}中的左手征超场$\,Z_{k}\,$的函数$\,\tilde{f}(Z)$,\cite{15} 其中的$\,Z\,$在标准模型的$\,SU(3)\times SU(2)\times U(1)\,$规范群下均是中性的. 更进一步, 隐藏分区的超势被假定采取形式
\begin{equation}
    \tilde{f}(Z) = \epsilon^{3}\,F(\kappa Z) \:, \label{31.7.1}
\end{equation}
其中$\,\epsilon\,$是某个远小于\,Planck\,质量的质量, 而$\,F(\kappa Z)\,$是$\,\kappa Z\,$的系数量级为\,1\,的幂级数. 总超势应该是$\,f(\Phi)+\tilde{f}(Z)\,$这个假定看起来有些任意, 但是我们将会看到, 不难找到为什么它至少是近似正确的原因. 对这个方法一个更加严格的批判是: 它没有为解决等级问题提供任何希望; 已经先行假定了能量$\,\epsilon\,$远小于\,Planck\,质量.


在发展了首批引力传递的超对称破缺模型之后, 以在能量$\,\Lambda\ll m_{\text{Pl}}\,$处很强的规范耦合随着能量减小缓慢增长的这种方法, 出现了其他能够自然解释能量标度等级的模型. 取决于对超对称破缺来源做出的假定, 这些模型被分成了两个版本. 我们将会看到, 这两个版本中的标量夸克和标量轻子均获得了与引力微子质量$\,m_{g}\,$量级相同的超对称破缺质量, 但是$\,m_{g}\,$的公式在这两个版本中是不同的; 在第一版中$\,m_{g}\approx \kappa \Lambda^{2}$, 而在第二版中$\,m_{g}\approx\kappa^{2}\Lambda^{3}$, 这分别给出$\,\Lambda\approx 10^{11}\,\mathrm{GeV}\,$和$\,\Lambda\approx 10^{13}\,\mathrm{GeV}$. 其他软超对称破缺参量, 包括$\,B\mu$, $A$-参量以及规范微子质量, 在这两个版本中的公式也有所不同.

\subsection{第一版\cite{16}}

在引力超导超对称破缺的这个版本中, 理论的超场被假定落入两个分区:

\noindent {\hei{可观测分区:}} 它们是最小超对称标准模型中的超场: $SU(3)\times SU(2)\times U(1)\,$规范超场以及夸克、反夸克、轻子、反轻子和\,Higgs\,左手征超场, 我们一般地把它们记做$\,\Phi_{r}$.

\noindent {\hei{隐藏分区:}} 它们是在$m_{W}\ll \Lambda\ll m_{\text{Pl}}\,$的中间能量标度$\,\Lambda\,$处变强的渐进自由规范相互作用的规范超场, 以及能够感受到这个规范相互作用的左手征超场$\,Z_{k}$.

$Z_{k}\,$必须假定成在$\,SU(3)\times SU(2)\times U(1)\,$规范群下是中性的, 若非如此, 我们将回到规范引导超对称破缺的情况. 另外我们对可观测分区的了解程度足以确定它的左手征超场不会感受到隐藏分区的规范相互作用.

为了使得总超势的可重整部分自然地有$\,f(\Phi)+\tilde{f}(Z)\,$的形式, 我们可以假定在\,Planck\,标度以下幸存的对称性包含群$\,G_{H}\,$(它可以是隐藏分区规范群的一部分)和群$\,G_{O}\,$(它可以是可观测分区的$\,SU(3)\times SU(2)\times U(1)\,$规范群的一部分), 在群$\,G_{H}\,$下, 可观测分区的所有场都是不变的而隐藏分区中没有一个场是不变的, 在群$\,G_{O}\,$下, 隐藏分区的场都是不变的而可观测分区中没有一个场是不变的. 在这个情况下, 如果超势中的一项含有可观测分区的任何场, 那么这一项中至少还有两个这样的场, 同理, 如果超势中的一项含有隐藏分区的任何场, 那么这一项中至少还有两个这样的场, 这使得三次多项式超势中的项不可能同时有隐藏分区和可观测分区中的场. 这个讨论保留了超势中的不可重整项还有两个或多个隐藏分区和可观测分区超场的可能性, 我们稍后会回到这个可能性. 当然, 我们假定隐藏分区的强相互作用在总的隐藏分区超势$\,\tilde{f}(Z)\,$中产生了不可重整项, 但它们也只依赖于隐藏分区的超场.

然后假定超势取$\,f(\Phi)+\tilde{f}(Z)\,$的形式, 方程(\ref{31.6.71})给出了这些超场的标量分量的势
\begin{align}
    V &= \me^{\kappa^{2}d}\,\Biggl[ \sum_{rs}g_{rs}^{-1}\,\biggl(\frac{\partial f}{\partial \phi_{r}}+
    \kappa^{2}\,(f+\tilde{f})\,\frac{\partial d}{\partial \phi_{r}}\biggr)\biggl(\frac{\partial f}{\partial \phi_{s}}+
    \kappa^{2}\,(f+\tilde{f})\,\frac{\partial d}{\partial \phi_{s}}\biggr)^{\ast} \nonumber \\
    &\quad +2\operatorname{Re}\sum_{rk}g_{rk}^{-1}\biggl(\frac{\partial f}{\partial \phi_{r}}+
    \kappa^{2}\,(f+\tilde{f})\,\frac{\partial d}{\partial \phi_{r}}\biggr)\biggl(\frac{\partial \tilde{f}}{\partial z_{k}}+
    \kappa^{2}\,(f+\tilde{f})\,\frac{\partial d}{\partial z_{k}}\biggr)^{\ast} \nonumber \\
    &\quad +\sum_{kl}g_{kl}^{-1}\,\biggl(\frac{\partial \tilde{f}}{\partial z_{k}}+
    \kappa^{2}\,(f+\tilde{f})\,\frac{\partial d}{\partial z_{k}}\biggr)\biggl(\frac{\partial \tilde{f}}{\partial z_{l}}+
    \kappa^{2}\,(f+\tilde{f})\,\frac{\partial d}{\partial z_{l}}\biggr)^{\ast} \nonumber \\
    &\quad -3\kappa^{2}\,\bigl\lvert f +\tilde{f}\bigr\rvert^{2}\Biggr] \nonumber \\
    &\quad +\frac{1}{2}\operatorname{Re}\sum_{AB}f_{AB}^{-1}\Biggl(\sum_{kl}\frac{\partial d}{\partial z_{k}}
    (t_{A})_{kl}z_{l}\Biggr) \Biggl(\sum_{mn}\frac{\partial d}{\partial z_{m}}(t_{B})_{mn}z_{n}\Biggr)^{\ast} \nonumber \\
    &\quad +\frac{1}{2}\operatorname{Re}\sum_{AB}f_{AB}^{-1}\Biggl(\sum_{rs}\frac{\partial d}{\partial \phi_{r}}
    (t_{A})_{rs}\phi_{s}\Biggr) \Biggl(\sum_{tu}\frac{\partial d}{\partial \phi_{t}}(t_{B})_{tu}\phi_{u}\Biggr)^{\ast} \:.
    \label{31.7.2}
\end{align}
在写下来自规范相互作用的势时, 我们在这里假定了隐藏分区和可观测分区的规范玻色子之间没有混合------即, 对任意一对规范生成元$\,t_{A}\,$和$\,t_{B}$, 当$\,t_{A}\,$在$\,\phi_{r}\,$上的作用不平庸且$\,t_{B}\,$在$\,z_{k}\,$上的作用不平庸时, $f_{AB}^{-1}\,$为零, 反之亦然.



在我们感兴趣探索的场空间区域中, 隐藏分区的标量场是$\,\Lambda\,$阶的, 隐藏分区超势的可变部分是$\,\Lambda^{3}\,$阶的, 基于量纲分析, $\partial \tilde{f}/\partial z_{k}\,$就是$\,\Lambda^{2}\,$阶的. 我们暂且不对$\,\tilde{f}\,$的常数部分做任何假定; 正如我们将看到的, 为了抵消宇宙学常数, 我们必须在$\,\tilde{f}\,$中引入一个远大于$\,\Lambda^{3}\,$的项.

我们进一步假定在要探索的场空间区域中可观测分区的场$\,\phi_{r}\,$是$\,\kappa\Lambda^{2}\,$阶的, 这是因为, 我们将会看到这是隐藏分区中的超对称破缺的引力效应在可观测分区中产生的特征质量标度. 对\,$\kappa\Lambda^{2}\,$阶的场, 可观测分区超势$\,f(\phi)\,$假定是$\,\kappa^{3}\Lambda^{6}\,$阶的, 而它的导数$\,\partial f(\phi)/\partial\phi_{r}\,$则被取成$\,\kappa^{2}\Lambda^{4}\,$阶的.

由于上节末尾讨论的超势和标量场定义, 以及上面假定了对称性是$\,G_{H}\times G_{O}$, 修正\,Kahler\,势取如下形式\footnote{$O(\kappa^{2}z^{\ast 2}z^{2})\,$是指形如$\,\kappa^{2}\sum_{klmn}C_{klmn}z_{k}^{\ast}z_{l}^{\ast}z_{m}z_{n}\,$的项, 其中系数$\,C_{klmn}\,$量级为\,1, 方程(\ref{31.7.3})---(\ref{31.7.6})中其他的项也同样如此.}
\begin{align}
    d(\phi,\phi^{\ast},z,z^{\ast}) &= \sum_{r}\lvert\phi_{r}\rvert^{2} +\sum_{k}\lvert z_{k}\rvert^{2}
    +O(\kappa^{2}z^{\ast2}z^{2}) \nonumber \\
    &\quad +O(\kappa^{2}z^{\ast}z^{3}) +O(\kappa^{2}z^{\ast3}z) +O(\kappa^{2}\phi^{\ast2}z^{2})
    +O(\kappa^{2}\phi^{\ast2}z^{\ast}z) \nonumber \\
    &\quad +O(\kappa^{2}z^{\ast2}\phi^{2})+O(\kappa^{2}z^{\ast}z\phi^{2}) +O(\kappa^{2}\phi^{\ast}\phi z^{2})\nonumber\\
    &\quad +O(\kappa^{2}z^{\ast2}\phi^{\ast}\phi)+O(\kappa^{2}\phi^{\ast}z^{\ast}\phi z)
    +O(\kappa^{2}\phi^{\ast2}\phi^{2}) \nonumber \\
    &\quad +O(\kappa^{2}\phi^{\ast}\phi^{3})+O(\kappa^{2}\phi^{\ast 3}\phi) +\cdots \:, \label{31.7.3}
\end{align}
其中省略号代表高阶项. 这样, 度规(\ref{31.6.66})就有分量
\begin{align}
    g_{rs} &= \delta_{rs} + O(\kappa^{2}z^{2}) + O(\kappa^{2}z^{\ast2})+ O(\kappa^{2}z^{\ast}z) \nonumber \\
    &\quad + O(\kappa^{2}\phi^{2}) + O(\kappa^{2}\phi^{\ast2})+ O(\kappa^{2}\phi^{\ast}\phi) +\cdots \label{31.7.4} \\
    g_{kl} &= \delta_{kl} + O(\kappa^{2}z^{2}) + O(\kappa^{2}z^{\ast2})+ O(\kappa^{2}z^{\ast}z) \nonumber \\
    &\quad + O(\kappa^{2}\phi^{2}) + O(\kappa^{2}\phi^{\ast2})+ O(\kappa^{2}\phi^{\ast}\phi) +\cdots \label{31.7.5} \\
    g_{rk} &= g_{kr}^{\ast} =O(\kappa^{2}\phi z^{\ast})+O(\kappa^{2}\phi z)+O(\kappa^{2}\phi^{\ast}z^{\ast})
    +O(\kappa^{2}\phi^{\ast}z) +\cdots \:. \label{31.7.6}
\end{align}
(因为$\,d\,$是\,Planck\,标度处的未知动力学效应产生的修正\,Kahler\,势, $d\,$中的特征能量标度被假定成 $1/\kappa$, 这与$\,\tilde{f}\,$不同, 它从标度$\,\Lambda\,$处的强规范耦合的动力学效应获得了结构.) $g_{rs}\,$和$\,g_{kl}\,$一般是\,1\,阶的, 而混合分量$\,g_{rk}\,$和$\,g_{kr}\,$是$\,\kappa^{2}(\kappa\Lambda^{2})\Lambda=\kappa^{3}\Lambda^{3}\ll 1\,$阶的. 由此可以得出这对$\,g^{-1}\,$同样成立: $(g^{-1})_{rs}\,$和 $(g^{-1})_{kl}\,$一般是\,1\,阶的, 而$\,(g^{-1})_{rk}\,$和$\,(g^{-1})_{kr}\,$是$\,\kappa^{3}\Lambda^{3}\ll 1\,$阶的.

从这些估计可以得出, 除非抵消产生了干涉, 势(\ref{31.7.2})中的主导项将至少是$\,\Lambda^{4}\,$阶的, 并取如下的形式
\begin{equation}
    [V]_{\Lambda^{4}} = \sum_{k}\biggl\lvert\frac{\partial \tilde{f}}{\partial z_{k}}\biggr\rvert^{2}
    -3\kappa^{2}\,\bigl\lvert \tilde{f}^{0}\bigr\rvert^{2} \:, \label{31.7.7}
\end{equation}
其中$\,\tilde{f}^{0}\,$是$\,\tilde{f}\,$的常数项, 需要这样的项是为了抵消真空能. 我们假定超对称在隐藏分区自发破缺, 这要求存在一个点$\,z_{k}^{0}\,$使得$\,\sum_{k}\lvert\partial \tilde{f}/\partial z_{k}\rvert^{2}\,$至少是一个定域极小值但不是零. 这样, 为了抵消到这阶的真空能, 我们必须取
\begin{equation}
    3\kappa^{2}\,\bigl\lvert \tilde{f}^{0}\bigr\rvert^{2} =
     \sum_{k}\Biggl\lvert\biggl(\frac{\partial \tilde{f}}{\partial z_{k}}\biggr)^{0}\Biggr\rvert^{2} \:, \label{31.7.8}
\end{equation}
右边的上标\,0\,是指这个量在$\,z=z^{0}\,$处计算的. 因此$\,\tilde{f}^{0}\,$必须被赋予一个异常大的值, 量级为$\,\Lambda^{2}/\kappa$. 这个精细调节相比引力传递超对称破缺的第二个版本中必要发生的精细调节要极端得多, 但当对宇宙常数还没有一个真正的理解时, 在超对称性破缺的任何理论中, 某个精细调节将总是必须的.

通过令真空能密度$\,F^{2}/2\,$等于平坦空间的值$\,\sum_{k}\lvert (\partial \tilde{f}/\partial z_{k})^{0}\rvert^{2}\,$, 我们就能计算出引力微子质量公式(\ref{31.3.17})中的超对称破缺参量$\,F$. 这样, 方程(\ref{31.3.17})和(\ref{31.7.8})就给出了引力微子质量
\begin{equation}
    m_{g} = \kappa\sqrt{\frac{1}{3}
     \sum_{k}\Biggl\lvert\biggl(\frac{\partial \tilde{f}(z)}{\partial z_{k}}\biggr)^{0}\Biggr\rvert^{2}}
     =\kappa^{2}\,\bigl\lvert \tilde{f}^{0}\bigr\rvert \:. \label{31.7.9}
\end{equation}
这与可观测分区的标量场在量级上相同, 均$\,\approx \kappa \Lambda^{2}$.

现在我们转向方程(\ref{31.7.2})中{\kai{确实}}依赖于可观测分区标量$\,\phi_{r}\,$的项. 我们考虑的场值使得通常的超对称项$\,\sum_{r}\lvert \partial f/\partial \phi_{r}\rvert^{2}\,$是$\,m_{g}^{4}\approx \kappa^{4}\Lambda^{8}\,$阶的, 所以我们必须要把方程(\ref{31.7.2})中这一阶或更大的$\,\phi$-相关项汇总在一起. 我们来逐个观察方程(\ref{31.7.2})右边的六行.

$\kappa^{2}(f+\tilde{f})\partial d/\partial\phi_{r}\,$中的领头项是$\,\kappa^{2}\tilde{f}^{0}\phi_{r}^{\ast}$, 同$\,\partial f/\partial \phi_{r}\,$一样是$\,\kappa^{2}(\Lambda^{2}/\kappa)(\kappa\Lambda^{2})=\kappa^{2}\Lambda^{4}\,$阶的, 
而 $\kappa^{2}(f+\tilde{f})\partial d/\partial\phi_{r}\,$中的其他项要小得多. 到领头阶, 我们可以将$\,\exp(\kappa^{2}d)\,$近似为\,1, 把$\,g_{rs}^{-1}\,$近似为$\,\delta_{rs}$, 所以方程(\ref{31.7.2})中的第一行在这一阶给出
\[
\sum_{r}\,\biggl\lvert \frac{\partial f}{\partial\phi_{r}} + \kappa^{2}\tilde{f}^{0}\phi_{r}^{\ast} \biggr\rvert^{2}\:.
\]
它在量级上正是希望的$\,\kappa^{4}\Lambda^{8}$, 所以没有必要考虑高阶修正.

$g_{rk}^{-1}\,$中的领头项是$\,\kappa^{3}\Lambda^{3}\,$阶; $\partial \tilde{f}/\partial z_{k}\,$中的领头项是$\,\Lambda^{2}\,$阶的, 而$\,\kappa^{2}(f+\tilde{f})\partial d/\partial z_{k}\,$要小一些, 是 $\kappa\Lambda^{3}\,$阶; 我们已经看到$\,\partial f/\partial\phi_{r}+\kappa^{2}(f+\tilde{f})\partial d/\partial\phi_{r}\,$是$\,\kappa^{2}\Lambda^{4}\,$阶的, 
所以方程(\ref{31.7.2})第二行中的项是 $(\kappa^{3}\Lambda^{3})(\Lambda^{2})(\kappa^{2}\Lambda^{4})=\kappa^{5}\Lambda^{9}\,$阶, 与方程(\ref{31.7.2})中的$\,\kappa^{4}\Lambda^{8}\,$阶项相比可以忽略.

方程(\ref{31.7.2})第三行和第四行中的领头项是$\,\Lambda^{4}\,$阶, 但它们独立于$\,\phi_{r}$. 第三行中包含$\,f\,$的项会给出与$\,\phi\,$相关项, 这类的领头项是$\,2\kappa^{2}\operatorname{Re}[f\sum_{k}z_{k}^{\ast}(\partial \tilde{f}/\partial z_{k})^{\ast}]$, 它的量级是$\,\kappa^{5}\Lambda^{9}\ll\kappa^{4}\Lambda^{8}$, 因此可以被忽略. 另外, 第四行中包含$\,f\,$的项会给出与$\,\phi\,$相关项. 这类的领头项是
\[
-6\kappa^{2}\operatorname{Re}[f\tilde{f}^{0\ast}]
\]
它是$\,\kappa^{2}(\kappa\Lambda^{2})^{3}\Lambda^{2}/\kappa=\kappa^{4}\Lambda^{8}\,$阶的. 因子$\,\exp(\kappa^{2}d)\,$中包含$\,\kappa^{2}(\kappa\Lambda^{2})^{2}\,$阶的$\,\phi$-相关项, 但它们要乘以一个势, 这个势的领头项被调整成直到$\,\Lambda^{4}\,$阶都是抵消的, 所以它给出的$\,\phi$-相关项要远小于$\,\kappa^{4}\Lambda^{8}\,$阶. 还存在一类$\,\phi$-相关项, 来自于$\,g^{-1}_{kl}\,$中的$\,\phi$-相关项. 根据方程(\ref{31.7.4})---(\ref{31.7.6}), 这些项可以写成 $\kappa^{2}u_{kl}(\phi,\phi^{\ast})$, 其中$\,u_{kl}\,$是$\,\phi_{r}\,$和$\,\phi_{r}^{ast}\,$系数量级为\,1\,的齐次二次多项式. 它们在方程(\ref{31.7.2})第三行给出了一个$\,\phi$-相关项, 形如
\[
\kappa^{2}\sum_{kl}\,u_{kl}(\phi,\phi^{\ast})\,\biggl(\frac{\partial \tilde{f}}{\partial z_{k}}\biggr)^{0}
\,\biggl(\frac{\partial \tilde{f}}{\partial z_{l}}\biggr)^{0\ast} \:,
\]
它是$\,\kappa^{2}(\kappa\Lambda)^{2}\Lambda^{4}=\kappa^{4}\Lambda^{8}\,$阶的.

方程(\ref{31.7.2})第五行中的$\,\phi$-相关项要么来自于$\,f_{AB}^{-1}\,$中的$\,\kappa^{2}\phi^{2}\,$项, 这时两个$\,d\,$由$\,z^{\ast}z\,$节的领头项给出, 要么来自于其中一个$\,d\,$的有两个$\,\kappa\,$因子, 两个$\,\phi\,$和(或)$\,\phi^{\ast}\,$因子, 以及两个$\,z\,$和(或)$\,z^{\ast}\,$因子的项, 这时$\,f_{AB}\,$和另外一个$\,d\,$由它们的领头项给出, 这两个领头项分别是$\,1\,$阶和$\,z^{\ast}z\,$阶的. 两种$\,\phi$-相关项给出的贡献是$\,\kappa^{2}(\kappa\Lambda^{2})^{2}\Lambda^{4}=\kappa^{4}\Lambda^{8}\,$阶的, 所以高阶项可以忽略.

方程(\ref{31.7.2})第六行中的领头项来自$\,f_{AB}^{-1}\,$中的领头项, 这是一阶的, 以及$\,d\,$中的领头$\,\phi$-相关项, 这是$\,\phi^{\ast}\phi\,$阶的. 它对势的贡献是$\,(\kappa\Lambda^{2})^{4}\,$阶的, 所以这里的高阶项也可以被忽略掉.

势(\ref{31.7.2})也包含$\,\kappa^{2}\Lambda^{6}\,$阶、$\kappa^{4}\Lambda^{8}\,$阶等的$\,\phi$-无关项. $\kappa^{2}\Lambda^{6}\,$阶项可以被抵消掉, 而通过对$\,\tilde{f}(z)\,$中的常数项做一个偏移使得它远离方程(\ref{31.7.8})给出的值, $\kappa^{4}\Lambda^{8}\,$阶项可以由一个任意值$\,\mathscr{C}\,$给出.

汇总所有结果, 到$\,\kappa^{4}\Lambda^{8}\approx m_{g}^{4}\,$阶, 可观测分区的势现在是
\begin{align}
    V_{O}(\phi,\phi^{\ast}) &= \sum_{r}
    \,\biggl\lvert \frac{\partial f}{\partial\phi_{r}} + \kappa^{2}\tilde{f}^{0}\phi_{r}^{\ast} \biggr\rvert^{2}
    -6\kappa^{2}\operatorname{Re}\Bigl[f(\phi)\tilde{f}^{0\ast}\Bigr] \nonumber \\
    &\quad +\frac{1}{2}\sum_{A}\Biggl\lvert\sum_{rs}\phi_{r}^{\ast}(t_{A})_{rs}\phi_{s}\Biggr\rvert^{2} \nonumber \\
    &\quad +Q(\phi,\phi^{\ast}) +\mathscr{C} \:, \label{31.7.10}
\end{align}
其中$\,Q(\phi,\phi^{\ast})\,$是$\,\phi\,$和(或)$\,\phi^{\ast}\,$的二次多项式, 其系数是$\,\kappa^{2}\Lambda^{4}\approx m_{g}^{2}\,$阶, 这一项来自于方程(\ref{31.7.2})第三行$\,g_{kl}^{-1}\,$中的$\,\phi$-相关项以及第五行$\,f_{AB}^{-1}\,$中的$\,\phi$-相关项. 我们已经归一化了规范超场使得在所有标量场为零时$\,f_{AB}=\delta_{AB}$. (对隐藏分区$\,z_{k}\,$的平衡值做一个$\,\phi$-相关的偏移也会产生与$\,\phi\,$相关的项, 但这个偏移至多是$(\kappa\Lambda^{2})^{4}/\Lambda^{3}\,$阶的, 又因为(\ref{31.7.7})在$\,z=z^{0}\,$处是稳定的, 这个偏移会以二次的形式进入到可观测分区场的有效势中, 因此可以被忽略.) 常数$\,\mathscr{C}\,$也可以选择成使得势在它的最小值处为零.

最后, 我们回到超势中的不可重整项. 正如已经提及的, 可以预期这类的领头$\,\Phi$-相关项是 $\kappa\Phi^{2}Z^{2}\,$阶. 
当隐藏分区超场$\,Z_{k}\,$被设成它们的平衡值$\,z_{k}^{0}\,$时, 这些项变成$\,\Phi_{r}\,$的二阶多项式, 其系数是$\,\kappa\Lambda^{2}\,$阶. 因此, 通过在超势中对$\,\Phi_{r}\,$引入系数为$\,\kappa\Lambda^{2}\approx m_{g}\,$阶的二次多项式函数, 我们就把这些不可重整项直到领头阶的效应考虑在内了.

以这种方式, 引力传递超对称破缺的理论就避免了\,\ref{sec:28.1}\,和\,\ref{sec:28.5}\,节中讨论的$\,\mu$-项问题. 
回忆, 标准模型的$\,SU(3)\times SU(2)\times U(1)\,$对称性允许最小超对称标准模型的超势中有一个超可重整项, 即$\,\mu(H_{1}^{\mathrm{T}}eH_{2}).$ 为了自然地解释为什么系数$\,\mu\,$与\,Planck\,质量不是同阶的, 必须要附加某种对称性, 例如\,23.6\,节中讨论的与强$\,\mathsf{CP}\,$破坏相关的``Peccei--Quinn''对称性, 在这个对称性, 乘积$\,(H_{1}^{\mathrm{T}}eH_{2})\,$不是中性的. 但是, 在\,\ref{sec:28.5}\,节看到, $\mu\,$与其他超对称破缺质量均为$\,m_{g}\,$阶的$\,\mu$-项在唯象上是必须的. 如果超势包含一个含有$\,(H_{1}^{\mathrm{T}}eH_{2})\,$的不可重整项, 且这项与隐藏分区场$\,z_{k}\,$的二次以$\,\kappa\,$阶的系数乘在一起, 隐藏分区场的真空期望值造成的\,Peccei-Quinn\,对称性破缺可以自然地产生这样的项.\footnote{这被称为\,Giudice--Masiero\,机制\cite{17}. 通常是用修正\,Kahler\,势$\,d\,$中的不可重整全纯项和反全纯项进行描述的, 但正如上节末尾所讨论的, 任何这样的项可以被超势中的全纯因子取代. 这里我们已经把$\,d\,${\kai{定义}}成不含全纯和反全纯项, 在这个定义下, $\mu$-项只可能来自于超势中的不可重整项.}

这样, 我们就假定组成有效超势$\,f(\phi)\,$的是: 场$\,\phi_{r}\,$的三阶齐次多项式$\,f^{(3)}(\phi)$, 其系数量级粗略为\,1, 加上一个$\,\mu$-项, 形如场$\,\phi_{r}\,$的二阶齐次多项式, 系数量级为$\,m_{g}\approx\kappa\Lambda^{2}$. 这样势(\ref{31.7.10})就变成
\begin{align}
    V_{O}(\phi,\phi^{\ast}) &= \sum_{r}\,\biggl\lvert \frac{\partial f}{\partial\phi_{r}} \biggr\rvert^{2}
    +\frac{1}{2}\sum_{A}\Biggl\lvert\sum_{rs}\phi_{r}^{\ast}(t_{A})_{rs}\phi_{s}\Biggr\rvert^{2}  \nonumber \\
    &\quad-2\kappa^{2}\operatorname{Re}\Bigl[f^{(2)}(\phi)\tilde{f}^{0\ast}\Bigr]+
    \kappa^{4}\lvert \tilde{f}^{0}\rvert^{2}\sum_{r}\lvert\phi_{r}\rvert^{2} \nonumber \\
    &\quad +Q(\phi,\phi^{\ast}) +\mathscr{C} \:. \label{31.7.11}
\end{align}
右边第一行中的项给出了超势, 而第二行和第三行中的项代表超对称的软破缺. 当$\,\tilde{f}^{0}\approx \Lambda^{2}/\kappa\,$且 $(\partial \tilde{f}/\partial z)^{0}\approx \Lambda^{2}\,$时, 方程(\ref{31.7.10})中软超对称破缺项中的量纲系数均是$\,\kappa\Lambda^{2}\approx m_{g}\,$的幂次, 这正是我们期待发现可观测分区标量场的真空期望值的地方, 证明了我们将其选为要探索的场空间区域的合理性.

令$\,\kappa\Lambda^{2}\,$等于超对称标准模型有效拉格朗日量中的特征质量$\,\approx 1\,\mathrm{TeV}$, 我们发现$\,\Lambda\approx 10^{11}\,\mathrm{GeV}$. 这让人稍感振奋, 正如\,23.6\,节中讨论过的, Peccei--Quinn\,对称性在标度$\,\Lambda\approx 10^{11}\,\mathrm{GeV}\,$处自发破缺正是在天文学观测允许的$\,10^{10}\,\mathrm{GeV}\,$到$\,10^{12}\,\mathrm{GeV}\,$对称性破缺标度窗口解决强$\,\mathsf{CP}\,$问题所需要的.

用这个势与最小超对称标准模型拉格朗日密度(\ref{28.4.1})中的标量场项做一比较, 我们看到引力传递超对称破缺的这个版本预测了破缺超对称形的仅是系数为\,$m_{g}^{2}\,$阶的(包含$\,B\mu\,$)软标量质量项. 到$\,\kappa\Lambda\,$的领头阶, 三线性超对称破缺项的系数$\,A\,$和$\,C\,$都为零.

这些结果的一个严重问题是, 方程(\ref{31.7.11})中的二次多项式$\,Q(\phi,\phi^{\ast})\,$没有什么理由要遵循\ref{sec:28.4} 节中所讨论的标量夸克质量之间以及标量轻子之间的简并性, 这将避开未观测到的味改变过程. 然而, 势的方程(\ref{31.7.11})中的第四项对标量质量平方给出了对所有标量都相等的额外贡献$\,\kappa^{4}\lvert \tilde{f}\rvert^{2}\approx \kappa^{2}\Lambda^{4}$, 所以, 如果$\,Q(\phi,\phi^{\ast})\,$中的系数(来自于方程(\ref{31.7.2})第三行和第四行中的项)碰巧小于$\,\kappa^{2}\Lambda^{4}$, 那么味改变过程的实验上界所附加的约束就可以被满足.

如果真实世界中$\,Q(\phi,\phi^{\ast})\,$可以被忽略的话, 对最小超对称标准模型的参量, 我们将会得到它们之间的一个有趣关系. 将超势的二次部分$\,f^{(2)}\,$取为$\,\mu(\phi_{1}^{\mathrm{T}}e\phi_{2})$, 方程(\ref{28.4.1})中的系数$\,B\mu\,$将由方程(\ref{31.7.11})的第二项给定为$\,B\mu=-\kappa^{2}\mu\tilde{f}^{0\,\ast}$, 所以
\[
\lvert B \vert =\kappa^{2}\lvert \tilde{f}^{0}\rvert = m_{g} \:,
\]
与方程(\ref{31.4.13})一致. 另外, 所有标量夸克和标量轻子的质量$\,M_{s}\,$将由方程(\ref{31.7.11})中的第三项给定为$\,\kappa^{2}\lvert \tilde{f}^{0}\rvert$, 因而我们就有了新关系
\begin{equation}
    \lvert B \rvert = M_{s} \:. \label{31.7.12}
\end{equation}
当标量夸克和标量轻子质量相等时, 味改变过程上的各种限制之间就没有不相容性. 更进一步, 当$\,Q\,$被忽略时, 超势(\ref{31.7.11})的超对称破缺部分将只有一个复参量$\,\tilde{f}^{0}$, 通过对超势总相位的从定义可将其选成实的, 所以现在势的超对称破缺部分不会对\,$\mathsf{CP}\,$不变性产生新破坏. 但是没有已知的原因表明$\,Q\,$应该是小量. 

这个版本的引力传递超对称破缺的另一严重问题是它不会产生充分大的规范微子质量.\cite{18} 根据方程(\ref{31.6.75}), 在树级近似下, 
$SU(3)\times SU(2)\times U(1)\,$规范微子质量矩阵是
\begin{equation}
    m_{AB} = \exp(\kappa^{2}d/2)\sum_{NM}[g^{-1}]_{NM} L_{N} \biggl(\frac{\partial f_{AB}}{\partial \varphi_{M}}\biggr)^{\ast}\:, \label{31.7.13}
\end{equation}
其中$\,\varphi_{N}\,$在这里取遍所有与$\,f_{AB}\,$相关的标量场$\,\phi_{r}\,$和$\,z_{k}$, 而$\,g_{NM}\,$和$\,L_{M}\,$由方程(\ref{31.6.66})和(\ref{31.6.69}) 给出. 根据我们这里所做的估计, $\kappa^{2}d=O(\kappa^{2}\Lambda^{2})\ll 1$; $L_{k}=O(\Lambda^{2})$, 而$\,L_{r}\,$小得多; $g_{kl}^{-1}\,$的量级为\,1. 另外, 我们假定$\,f_{AB}\,$是一个一阶项加上一个$\,\kappa^{2}\,$阶项乘以标量场及其复共轭的双线性型, 所以$\,\partial f_{AB}/\partial z_{k}\,$是$\,\kappa^{2}\Lambda\,$阶的. 这给出了$\,\Lambda^{2}\times \kappa^{2}\Lambda\,$阶的规范微子质量. 这比引力微子质量$\,m_{g}\approx \kappa\Lambda^{2}$(这设定了可观测分区标量场的势(\ref{31.7.11})中的超对称破缺项标度)要小一个$\,\kappa\Lambda\approx 10^{-7}\,$因子, 所以如果超对称破缺产生的标量质量在量级上是$\,1\,\mathrm{TeV}$, 那么规范微子的质量在量级上将是$\,100\,\mathrm{keV}$, 这个质量太小, 与现在还没有观测到规范微子的事实相矛盾.

有数个方法可以避免这个问题. 一个是在隐藏分区的$\,z_{k}\,$之间引入规范单态标量场, 它们可以以线性的方式出现在$\,f_{AB}\,$中.\cite{19} 在这个情况下, $\partial f_{AB}/\partial z_{k}\,$将是$\,\kappa\,$阶而不是$\,\kappa^{2}\Lambda\,$阶, 产生的规范微子质量是$\,\kappa\Lambda^{2}\,$阶, 这就与标量夸克和标量轻子的质量相差无几了. 这个方法的一个问题是, 引入对所有规范群都是中性的标量将使得超势的可重整部分不会自然地采取$\,f(\phi)+\tilde{f}(z)\,$的形式.

即使没有规范单态, 对胶微子、$W\,$微子和$\,B\,$微子质量(以及$\,A\,$参量)还有\,\ref{31.4}\,节中计算的单圈修正. 例如, 如果我们把$\,m_{g}\,$取在\,\ref{28.1}\,节中讨论过的``自然性''约束所允许的最大值$\,\approx 10\,\mathrm{TeV}$, 那么伴随$\,g_{s}^{2}/4\uppi =0.118$, 方程(\ref{31.4.13})将给出胶微子质量$\,3g_{s}^{2}m_{g}/16\uppi^{2}=280\,\mathrm{GeV}$, 这个能量足够高以至于胶微子能逃过探测. $B\,$微子和$\,W\,$微子质量依赖于$\,\mu\,$参量与赝实\,Higgs\,质量$\,m_{A}\,$的未知比值. 取这个比值为\,1\,以及$\,m_{g}<10\,\mathrm{TeV}$, 方程(\ref{31.4.15})和(\ref{31.4.16})将给出$\,m_{\text{bino}}=9g'^{2}m_{g}/16\uppi^{2}<73\,\mathrm{GeV}\,$和$\,m_{\text{wino}}=g^{2}m_{g}/16\uppi^{2}<27\,\mathrm{GeV}$.\cite{19a} 在\,LEP\,上, $e^{+}$--$e^{-}\,$在能量足以产生$\,W\,$玻色子对的碰撞实验中还没有看到\,$W$\,微子对, 使得$\,m_{\text{wino}}>m_{W}$, 所以与\,$W$\,微子的这个上界相矛盾. 为了避免这个矛盾, 要么取$\,m_{g}>30\,\mathrm{TeV}$, 这从自然性上看是很惊人的, 要么$\,\mu^{2}/m_{A}^{2}>8$.\cite{11b} 在任何情况下, 这个模型会给出规范微子远轻于标量夸克和标量轻子的一般结果.

如果多项式$\,Q(\phi,\phi^{ast})\,$可以被忽略, 那么, 就像\,\ref{31.4}\,节所展示的那样, $A\,$参量也将由单圈修正给出. 它们对于标量夸克是$\,g_{s}^{2}m_{g}/16\uppi^{2}\,$阶, 对于标量轻子则是$\,g^{2}m_{g}/16\uppi^{2}\,$或$\,g'^{2}m_{g}/16\uppi^{2}\,$阶.


\subsection{第二版\cite{20}}

同第一版一样, 这个版本的引力传递超对称破缺有手征超场$\,\Phi_{r}\,$组成的可观测分区和手征超场$\,Z{k}\,$构成的隐藏分区. 差别是超对称性现在假定成在隐藏分区{\kai{不}}破缺. 取而代之, 在能量$\,\Lambda\ll m_{\text{Pl}}\,$处变强的隐藏分区规范耦合对第三分区超场, {\kai{模超场}}(modular superfields), 中的标量场产生了一个非微扰的超势. 在各种理论, 例如现代超弦理论中, 存在额外维, 它们因``蜷缩''进非常小的紧流形而无法观测, 其中流形的大小通常是$\,\kappa\,$阶. 一般来说, 用来描述这个紧流形的一些参量直到微扰论的任意阶都是不固定的. 这些参量的值会在四维时空中逐点变化, 并在能量远低于\,Planck\,标度$\,\kappa^{-1}\,$时作为规范不变的标量场$\,y_{a}\,$出现, 称为{\kai{模场}}. (这里的指标$\,a,b\,$等显然与\,\ref{sec:31.6}\,节中使用的定域\,Lorentz\,参照系指标无关.) 假定超对称性在额外维的紧致化中不破缺, 则必须要给这些场配上费米场和辅助场, 它们合起来构成规范不变的左手征模超场$\,Y_{a}\,$及其共轭.\footnote{显然, 即使在第一版引力传递超对称破缺的假定下也可能存在模场, 当由于$\,\Lambda\,$在那个情况下取更小的值, 模场的耦合太弱以至于在哪里没有什么影响.}

就在紧致化标度稍稍往下, 我们有一个超势依赖所有超场的超对称理论, 但在这个理论中有量纲的参量只有$\,\kappa$. 通常会发生\,$y_{a}\,$在微扰论中无法固定下来, 这是因为紧致化并不会给出任何只依赖模超场$\,Y_{a}\,$的超势. 正如我们在\,\ref{sec:28.1}\,节看到的, 除了可能会有\,Higgs\,超场$\,H_{1}\,$和$\,H_{2}\,$的一个双线性项, $SU(3)\times SU(2)\times U(1)\,$规范对称性排除了超势中任何仅有一个或两个可观测分区超场的项. 我们将再次假定裸超势中的这个双线性项要么没有(在这个情况下, 它不出现在微扰论中的任意阶), 要么被某个对称性排除了, 例如\,23.6\,节中讨论过的与强\,$\mathsf{CP}$\,破坏相关的``Peccei--Quinn''对称性. 隐藏分区的规范对称性排除了超势中任何仅有一个$\,Z_{k}\,$的项, 我们也将假定含有两个$\,Z_{k}\,$的项要么碰巧没有要么被某个对称性(或许是同一个\,Peccei--Quinn\,对称性)排除了.

因此, 裸超势采取如下的形式:
\begin{equation}
    f_{\text{bare}}(\Phi,Y,Z) = \sum_{rst} f_{rst}(\kappa Y)\Phi_{r}\Phi_{s}\Phi_{t}
    +\sum_{klm}f_{klm}(\kappa Y)Z_{k}Z_{l}Z_{m} + \cdots \:, \label{31.7.14}
\end{equation}
其中$\,f_{rst}\,$和$\,f_{klm}\,$是其变量的幂级数, 系数量级约为\,1, 而省略号代表包含$\,n>3\,$个$\,\Phi\,$和$\,Z\,$以及任意多个$\,\kappa Y_{a}\,$的项, 它们会被正比于$\,\kappa^{n-3}\,$的因子压低.

我们假定隐藏分区中的非微扰效应, 例如``规范微子凝聚''(规范微子场的双线性函数有真空期望值), 会产生模超场的超势且它们自身不破缺超对称形. 由于$\,\Lambda\,$是问题中(除了被因子$\,\kappa\Lambda\,$压低的引力效应)唯一的标度, 这个超势将必须是如下的形式:
\begin{equation}
    \hat{f}(Y) = \Lambda^{3}F(\kappa Y) \:. \label{31.7.15}
\end{equation}
在方程(\ref{31.7.14})中, $Z^{3}\,$项中$\,Z\,$超场的标量分量的真空期望值也会产生这样的项. 与此同时, 对方程(\ref{31.7.14})中省略号所表示的不可重整项, 将其中的$\,Z_{k}\,$换成它们的真空期望值将在超势中产生额外的$\,\Phi$-相关项, 我们将在后面进一步讨论. 超势(\ref{31.7.15})取(\ref{31.7.1})的形式最初是在引力传递超对称破缺的理论中假定的, 只不过现在$\,\epsilon\,$现在被视为隐藏分区规范相互作用变强所处的中间标度$\,\Lambda$.

超势(\ref{31.7.15})通过出现了模手征超场的$\,\mathscr{F}$-项而导致超对称破缺貌似是可信的. 我们暂且忽略其他超场, 后面将说明合理性. 方程(\ref{31.6.68})和(\ref{31.6.69})给出了模标量的势
\begin{equation}
    \hat{V}(y,y^{\ast}) = \exp\Bigl(\kappa^{2}\hat{d}(y,y^{\ast})\Bigr)
    \Biggl[\sum_{ab}[\hat{g}^{-1}(y,y^{\ast})]_{ab}\hat{L}_{a}(y)\hat{L}_{b}(y)^{\ast}
    -3\kappa^{2}\,\lvert \hat{f}(y)\rvert^{2}\Biggr]\:, \label{31.7.16}
\end{equation}
其中$\,\hat{d}(y,y^{\ast})\,$是忽略了$\,Z_{k}\,$和$\,\Phi_{r}\,$的标量分量的\,Kahler $d\,$函数, 而
\begin{gather}
    \hat{g}_{ab} = \frac{\partial^{2}\hat{d}}{\partial y_{a}\partial y_{b}^{\ast}} \:, \label{31.7.17} \\
    \hat{L}_{a} =\frac{\partial \hat{f}}{\partial y_{a}} + \kappa^{2}\hat{f} \frac{\partial \hat{d}}{\partial y_{a}} \:.
    \label{31.7.18}
\end{gather}
我们在这里假定$\,\hat{V}\,$有一个稳定点, 用上标\,0\,标记, 在整个稳定点上$\,\hat{L}_{a}^{0}\neq 0$, 使得超对称是破缺的, 但$\,\hat{V}^{0}\,$非常小, 使得它可以被来自于可观测分区中的项抵消, 留给我们平坦时空. 由于$\,\hat{f}\,$形如(\ref{31.7.15})且$\,\hat{d}\,$等于$\,\kappa^{-2}\,$乘以$\,\kappa y_{a}\,$和$\,\kappa y_{a}^{\ast}\,$系数量级为\,1\,的幂级数, 整个超势所采取的形式就是$\,\kappa^{2}\Lambda^{6}$ 乘以$\,\kappa y_{a}\,$和$\,\kappa y_{a}^{\ast}\,$的幂级数, 这里幂级数的系数量级依旧为\,1. 因此, 超势中各个元素的大小量级就是
\begin{equation}
    \begin{split}
        y_{a}^{0}&=O(\kappa^{-1}) \:, \qquad \hat{f}^{0} = O(\Lambda^{3})\:, \qquad \hat{d}^{0}=O(\kappa^{-2})\:,\\
        \hat{L}_{a}^{0} &= O(\kappa\Lambda^{3})\:, \qquad \hat{g}_{ab}^{0}=O(1) \:.
    \end{split} \label{31.7.19}
\end{equation}


当模场和隐藏分区场固定为它们的期望值后, 可观测分区的超势现在形如
\begin{equation}
    f(\Phi)=\sum_{rs}\mu_{rs}\Phi_{r}\Phi_{s}+\sum_{rst}g_{rst}\Phi_{r}\Phi_{s}\Phi_{t}+\cdots \:, \label{31.7.20}
\end{equation}
其中$\,g_{rst}\,$是量级被假定成\,1\,的$\,f_{rst}(\kappa y_{0})\,$加上被$\,\kappa\Lambda\,$的幂次压低的项. 这里的省略号代表所含$\,\Phi\,$因子多余\,3\,个的项, 它们被额外的$\,\kappa\Phi\,$因子压低. 系数$\,\mu_{rs}\,$来自于方程(\ref{31.7.14})省略号所代表的不可重整项; 如果它来自于有$\,n>1\,$个$\,Z\,$因子和两个$\,\Phi\,$因子的项, 那么它的量级就是
\begin{equation}
    \mu_{rs} = O(\kappa^{n-1}\Lambda^{n}) \:. \label{31.7.21}
\end{equation}
我们将看到$\,\mu\,$希望的量级是$\,m_{g}=O(\kappa^{2}\Lambda^{3}),$ 这来自于有$\,n=3\,$个$\,Z\,$因子的项.


模分区中的超对称破缺将通过引力及其超对称伴的效应传递到可观测分区. 方程(\ref{31.6.68})给出了可观测分区的势
\begin{align}
    V_{O} &= \me^{\kappa^{2}d^{0}}\Biggl[\sum_{rs}[g^{0\:{-}1}]_{rs}\biggl(\frac{\partial f}{\partial \phi_{r}}+
    \kappa^{2}\,(f+\hat{f}^{0})\,\frac{\partial d^{0}}{\partial \phi_{r}}\biggr)\biggl(\frac{\partial f}{\partial \phi_{s}}+
    \kappa^{2}\,(f+\hat{f}^{0})\,\frac{\partial d^{0}}{\partial \phi_{s}}\biggr)^{\ast} \nonumber \\
    &\quad +2\operatorname{Re}\sum_{ra}[g^{0\:{-}1}]_{ra}\biggl(\frac{\partial f}{\partial \phi_{r}}+
    \kappa^{2}\,(f+\hat{f}^{0})\,\frac{\partial d^{0}}{\partial \phi_{r}}\biggr)\,\hat{L}_{a}^{0\ast} \nonumber \\
    &\quad +\sum_{ab}[g^{0\:{-}1}]_{ab}\hat{L}_{a}^{0}\hat{L}_{b}^{0\ast} - 3\kappa^{2}\Bigl\lvert f+\hat{f}^{0}\Bigr\rvert^{2}\Biggr] \nonumber \\
    &\quad +\frac{1}{2}\operatorname{Re}\sum_{AB}[f^{-1}_{AB}]^{0}\Biggl(\sum_{rs}\frac{\partial d^{0}}{\partial \phi_{r}}
    (t_{A})_{rs}\phi_{s}\Biggr) \Biggl(\sum_{tu}\frac{\partial d^{0}}{\partial \phi_{t}}(t_{B})_{tu}\phi_{u}\Biggr)^{\ast} \:,
    \label{31.7.22}
\end{align}
其中上标\,0\,再次是指模场和隐藏分区场被固定成它们的平衡值. (这会在后面重新考虑.) 注意到, 尽管方程(\ref{31.7.22})包含像$\,\hat{L}_{a}^{0}\,$这样来自模分区的项, 但这里没有显然与隐藏分区相关的项. 这是因为在引力传递超对称破缺的这个版本中, 超对称在这个分区被假定成不破缺, 使得$\,L_{k}^{0}=0$, 以及任何与隐藏分区场相互作用的规范场有$\,D_{A}^{0}=0$.

在我们感兴趣探索的场空间区域中, 可观测分区标量场是$\,\kappa^{2}\Lambda^{3}\,$阶的, 这是因为, 正如我们将看到的, 这是隐藏分区中超对称破缺的引力效应在可观测分区产生的特征能量. 根据方程(\ref{31.6.48}) 和我们的估计$\,\hat{f}^{0}\approx \Lambda^{3}$, 这也是引力微子质量$\,m_{g}\,$的量级:
\[
    m_{g} \approx \kappa^{2}\Lambda^{3} \:.
\]

为了计算这一阶场的势, 我们注意到可观测标量场和模标量场的\,Kahler $d\,$函数采取如下的形式:
\begin{align}
    &d(\phi,\phi^{\ast},y,y^{\ast}) = \kappa^{-2}\hat{d}(\kappa y,\kappa y^{\ast})
    +\sum_{rs}\phi_{r}\phi_{s}^{\ast}A_{rs}(\kappa y,\kappa y^{\ast})  \nonumber \\
    &\quad  +\sum_{rs}\phi_{r}\phi_{s}B_{rs}(\kappa y,\kappa y^{\ast})
     +\sum_{rs}\phi_{r}^{\ast}\phi_{s}^{\ast}B_{rs}^{\ast}(\kappa y,\kappa y^{\ast})+\cdots \:,  \label{31.7.23}
\end{align}
其中$\,\hat{d}$, $A_{rs}\,$和$\,B_{rs}\,$是它们自变量系数量级为\,1\,的幂级数, 而省略号代表的是含有$\,n>2\,$个$\,\phi\,$和(或) $\phi^{\ast}\,$因子的项, 这些项会被因子$\,\kappa^{n-2}\,$压低. 根据方程(\ref{31.6.72}), 通过给总超势乘上一个合适的全纯因子, 我们可以移除$\,d\,$中任何全纯项及其复共轭. 特别地, 通过给总超势乘上因子$\,\exp[\kappa^{2}\hat{d}^{0}+\kappa^{2}\sum_{rs}B_{rs}^{0}\phi_{r}\phi_{s}]$, 我们可以整理使得变换后的$\,d\,$函数有
\begin{equation}
    \hat{d}^{0}=0\:, \qquad B_{rs}^{0} = 0\:. \label{31.7.24}
\end{equation}
我们将假定这已经被实现了. 注意到, 由于总超势包含一个$\,\Lambda^{3}\,$阶的常数项$\,\hat{f}^{0}$, 而$\,B_{rs}^{0}\,$是\,1\,阶的, 这个变换将在超势中生成$\,\phi_{r}\,$的二次项, 它对系数$\,\mu_{rs}\,$的贡献是$\,\kappa^{2}\Lambda^{3}\,$阶, 与$\,m_{g}\,$和$\,\phi\,$的量级相同. 

对这个系数还有另一个同样量级的贡献, 它来自于超势中有$\,n>1\,$个$\,Y\,$因子以及两个$\,\Phi\,$因子的项. 方程(\ref{31.7.21})告诉我们, 为了使得它对系数$\,\mu_{rs}\,$的贡献也是$\,\kappa^{2}\Lambda^{3}\,$阶的, 我们必须有$\,n=3$. 通过给$\,H_{1}\,$和$\,H_{2}\,$赋予\,Peccei--Quinn\,量子数$\,+1$, 给$\,Y\,$赋予量子数$\,-2/3$, 我们可以整理使得这一项是允许的, 而有两个$\,\Phi\,$因子和$\,n=2\,$个$\,Y\,$因子的项是被禁止的. Peccei--Quinn\,对称性因$\,Y_{k}\,$的期望值而破缺就产生了一个轴子.


当$\,\mu_{rs}\,$是$\,\kappa^{2}\Lambda^{3}\,$阶, 可观测分区超势中的双线性和三线性项就是$\,\kappa^{6}\Lambda^{9}\,$阶. 通常的超对称势能项$\,\sum_{r}\lvert \partial f/\partial \phi_{r}\rvert^{2}\,$就是$\,\kappa^{8}\Lambda^{12}\,$阶, 所以我们必须收集方程(\ref{31.7.22})中所有这一阶以及更大的项.

当$\,\phi_{r}\,$是$\,\kappa^{2}\Lambda^{3}\,$阶且$\,y_{a}\,$被固定到它的平衡值$\,y_{a}^{0}\approx \kappa^{-1}$, 方程(\ref{31.7.23})给出
\begin{align}
    g_{rs}^{0} &= A_{rs}^{0} + O(\kappa^{3}\Lambda^{3}) \:, \label{31.7.25} \\
    g_{ab}^{0} &= \Biggl(\frac{\partial^{2}\hat{d}}{\partial y_{a}\partial y_{b}^{\ast}}\Biggr)^{0}
    +\sum_{rs}\phi_{r}\phi_{s}^{\ast} \Biggl(\frac{\partial^{2}A_{rs}}{\partial y_{a}\partial y_{b}^{\ast}}\Biggr)^{0}
    +\sum_{rs}\phi_{r}\phi_{s} \Biggl(\frac{\partial^{2}B_{rs}}{\partial y_{a}\partial y_{b}^{\ast}}\Biggr)^{0} \nonumber \\
    &\quad +\sum_{rs}\phi_{r}^{\ast}\phi_{s}^{\ast} \Biggl(\frac{\partial^{2}B_{rs}^{\ast}}{\partial y_{a}\partial y_{b}^{\ast}}\Biggr)^{0}+O(\kappa^{6}\Lambda^{6}) \:, \label{31.7.26} \\
    g_{ra}^{0} &= \sum_{s}\phi_{s}\biggl(\frac{\partial A_{rs}}{\partial y_{a}^{\ast}}\biggr)^{0}
    +O(\kappa^{6}\Lambda^{6}) = {g_{ar}^{0}}^{\ast} \:, \label{31.7.27}
\end{align}
其中上标\,0\,依旧是指$\,y_{a}\,$被固定到它们的平衡值$\,y_{a}^{0}$. 我们将可观测分区超场$\,\Phi_{r}\,$和模超场$\,Z_{a}\,$分别做一线性变换, 使得
\begin{equation}
    A_{rs}^{0} = \delta_{rs} \:, \qquad \Biggl(\frac{\partial^{2}\hat{d}}{\partial y_{a}\partial y_{b}^{\ast}}\Biggr)^{0}
    =\delta_{ab} \:. \label{31.7.28}
\end{equation}
当度规(\ref{31.7.25})---(\ref{31.7.27})由单位矩阵加上远小于\,1\,的项给出时, 不难计算出它们的逆:
\begin{align}
    g_{rs}^{0\:{-1}} &= \delta_{rs} + O(\kappa^{3}\Lambda^{3}) \:, \label{31.7.29} \\
    g_{ab}^{0\:{-1}} &= \delta_{ab}-
    \sum_{rs}\phi_{r}\phi_{s}^{\ast} \Biggl(\frac{\partial^{2}A_{rs}}{\partial y_{a}\partial y_{b}^{\ast}}\Biggr)^{0}
    -\sum_{rs}\phi_{r}\phi_{s} \Biggl(\frac{\partial^{2}B_{rs}}{\partial y_{a}\partial y_{b}^{\ast}}\Biggr)^{0} \nonumber \\
    &\quad -\sum_{rs}\phi_{r}^{\ast}\phi_{s}^{\ast} \Biggl(\frac{\partial^{2}B_{rs}^{\ast}}{\partial y_{a}\partial y_{b}^{\ast}}\Biggr)^{0}+O(\kappa^{6}\Lambda^{6}) \:, \label{31.7.30} \\
    g_{ra}^{0\:{-1}} &= -\sum_{s}\phi_{s}\biggl(\frac{\partial A_{rs}}{\partial y_{a}^{\ast}}\biggr)^{0}
    +O(\kappa^{6}\Lambda^{6}) = {g_{ar}^{0\:{-1}}}^{\ast} \:, \label{31.7.31}
\end{align}
特别地, 我们对函数$\,A_{rs}\,$和$\,B_{rs}\,$的形式作出的假定给出了量级估计
\begin{equation}
    \Biggl(\frac{\partial^{2}A_{rs}}{\partial y_{a}\partial y_{b}^{\ast}}\Biggr)^{0}= O(\kappa^{2}) \:, \qquad
    \Biggl(\frac{\partial^{2}B_{rs}}{\partial y_{a}\partial y_{b}^{\ast}}\Biggr)^{0}= O(\kappa^{2}) \:, \qquad
    \biggl(\frac{\partial A_{rs}}{\partial y_{a}^{\ast}}\biggr)^{0}=O(\kappa)  \:, \label{31.7.32}
\end{equation}
使得, 对$\,\phi_{r}=O(\kappa^{2}\Lambda^{3})$,
\begin{equation}
    [g^{0\:{-1}}]_{rs} =O(1) \:, \qquad [g^{0\:{-1}}]_{ab} = O(1)\:, \qquad
    [g^{0\:{-1}}]_{ra} =O(\kappa^{3}\Lambda^{3}) \:. \label{31.7.33}
\end{equation}
更进一步,
\begin{equation}
    f= O(\phi^{3}) = O(\kappa^{6}\Lambda^{9}) \:, \qquad
    \frac{\partial f}{\partial \phi_{r}} = O(\phi^{2}) = O(\kappa^{4}\Lambda^{6}) \:, \label{31.7.34}
\end{equation}
且
\begin{equation}
    \frac{\partial d^{0}}{\partial\phi_{r}} = O(\phi) =O(\kappa^{2}\Lambda^{3}) \:. \label{31.7.35}
\end{equation}
当$\,\hat{f}^{0}\,$是$\,\Lambda^{3}\,$阶, 方程(\ref{31.7.22})中的$\,\kappa^{2}(f+\hat{f}^{0})\partial d/\partial \phi_{r}\,$就由$\,\kappa^{2}\hat{f}^{0}\phi_{r}^{\ast}\,$主导, 它是$\,\kappa^{2}\times\Lambda^{3}\times\kappa^{2}\Lambda^{3}=\kappa^{4}\Lambda^{6}\,$阶. 这与$\,\partial f/\partial \phi_{r}\,$量级相同, 所以在领头阶, 这两项都要予以保留:
\begin{equation}
    \frac{\partial f}{\partial\phi_{r}} + \kappa^{2}(f+\hat{f}^{0})\frac{\partial d}{\partial \phi_{r}}
    \simeq \frac{\partial f}{\partial\phi_{r}} + \kappa^{2}\hat{f}^{0}\,\phi_{r}^{\ast}
    =O(\kappa^{4}\Lambda^{6}) \:. \label{31.7.36}
\end{equation}
在这个近似下, 且$\,g_{rs}^{0\,-1}\,$被换成了它的主导项$\,\delta_{rs}\,$后, 方程(\ref{31.7.22})方括号中的第一项已经是希望的$\,\kappa^{8}\Lambda^{12}\,$阶, 所以我们可以用这些近似写下
\begin{align}
    &\sum_{rs}[g^{0\:{-}1}]_{rs}\biggl(\frac{\partial f}{\partial \phi_{r}}+
    \kappa^{2}\,(f+\hat{f}^{0})\,\frac{\partial d^{0}}{\partial \phi_{r}}\biggr)\biggl(\frac{\partial f}{\partial \phi_{s}}+
    \kappa^{2}\,(f+\hat{f}^{0})\,\frac{\partial d^{0}}{\partial \phi_{s}}\biggr)^{\ast} \nonumber \\
    &\qquad \qquad \simeq \sum_{r}\,\biggl\lvert\frac{\partial f}{\partial \phi_{r}}+\kappa^{2}\hat{f}^{0}\,\phi_{r}^{\ast}\biggr\rvert^{2} \:. \label{31.7.37}
\end{align}
方程(\ref{31.7.22})方括号中的第二项也是$\,\kappa^{3}\Lambda^{3}\times \kappa^{4}\Lambda^{6}\times\kappa\Lambda^{3}=\kappa^{8}\Lambda^{12}\,$阶的, 所以我们可以只用领头项计算它并得到
\begin{align}
    &2\operatorname{Re}\sum_{ra}[g^{0\:{-}1}]_{ra}\biggl(\frac{\partial f}{\partial \phi_{r}}+
    \kappa^{2}\,(f+\hat{f}^{0})\,\frac{\partial d^{0}}{\partial \phi_{r}}\biggr)\,\hat{L}_{a}^{0\ast} \nonumber \\
    &\qquad\qquad \simeq -2\operatorname{Re}\sum_{ras}\phi_{s}\,\biggl(\frac{\partial A_{rs}}{\partial y_{a}^{\ast}}\biggr)^{0}\,
    \biggl[\frac{\partial f}{\partial \phi_{r}}+\kappa^{2}\hat{f}^{0}\,\phi_{r}^{\ast}\biggr]\,\hat{L}_{a}^{0\ast} \:.
    \label{31.7.38}
\end{align}
方程(\ref{31.7.22})方括号中的第三项和第四项各是$\,\kappa^{2}\Lambda^{6}\,$阶的, 但假定它们近乎抵消, 所以在计算它们对方括号中的量的贡献是必须要计入非领头项. 一个贡献来自于$\,g_{ab}^{0\,-1}\,$的方程(\ref{31.7.30})中$\,\phi$\,和(或) $\phi^{\ast}\,$的二阶项, 这个贡献在量级上是$\,\kappa^{2}\phi^{2}\times (\kappa\Lambda^{3})^{2}$, 即是$\,\kappa^{8}\Lambda^{12}\,$阶的. 方程(\ref{31.7.22})中的最后一项对势也有一个没有抵消的贡献, 它来自于$\,f\,$和$\,\tilde{f}^{0}\,$之间的干涉, 它给出了量级为$\,\kappa^{2}\phi^{3}\Lambda^{3}\,$的贡献, 也是$\,\kappa^{8}\Lambda^{12}\,$阶的. 还可能有一个常数贡献$\,\mathscr{C}$, 它来自于方程(\ref{31.7.22})方括号中的最后两项在$\,\phi_{r}=0$\,时可能没有成功相消; 为了避免宇宙学常数太大, 我们将不得不假定$\,\mathscr{C}\,$也是$\,\kappa^{8}\Lambda^{12}\,$阶的. (这是一个不自然的精细调节, 迄今为止, 在任何理论中, 为了避免巨大的宇宙学常数, 这都是必须的.) 将这些估计放在一起, 到$\,\kappa^{8}\Lambda^{12}\,$阶, 方程(\ref{31.7.22})方括号中的最后两项是
\begin{align}
    &\sum_{ab}[g^{0\:{-}1}]_{ab}\hat{L}_{a}^{0}\hat{L}_{b}^{0\ast} - 3\kappa^{2}\Bigl\lvert f+\hat{f}^{0}\Bigr\rvert^{2}
    \simeq - \sum_{abrs}\Biggl[\phi_{r}\phi_{s}^{\ast}\biggl(\frac{\partial^{2}A_{rs}}{\partial y_{a}\partial y_{b}^{\ast}}\biggr)^{0}\nonumber \\
   &\quad +\sum_{rs}\phi_{r}\phi_{s} \biggl(\frac{\partial^{2}B_{rs}}{\partial y_{a}\partial y_{b}^{\ast}}\biggr)^{0}
   +\sum_{rs}\phi_{r}^{\ast}\phi_{s}^{\ast} \biggl(\frac{\partial^{2}B_{rs}^{\ast}}{\partial y_{a}\partial y_{b}^{\ast}}\biggr)^{0}\Biggr] \hat{L}_{a}^{0}\hat{L}_{b}^{0\ast} - 6\kappa^{2}\operatorname{Re}(f\hat{f}^{0\ast})
   +\mathscr{C} \:. \label{31.7.39}
\end{align}
所有这些项都是希望的$\,\kappa^{8}\Lambda^{12}\,$阶, 而方程(\ref{31.7.24})给出$\,\kappa^{2}d^{0}=O(\kappa^{2}\phi^{2})=O(\kappa^{6}\Lambda^{6})$, 所以我们可以忽视掉方程(\ref{31.7.22})中的因子$\,\exp(\kappa^{2}d^{0})$. 最后, 方程(\ref{31.7.22})最后一行中的规范项是$\,\phi^{4}=O(\kappa^{8}\Lambda^{12})\,$阶, 所以它们可以用$\,\partial d^{0}/\partial \phi_{r}\,$中的领头项$\,\phi_{r}^{\ast}\,$计算.

汇总以上结果, 到$\,\kappa^{8}\Lambda^{12}\,$阶, 可观测分区的完整标量势是
\begin{align}
    V_{O} &\simeq \sum_{r}\,\biggl\lvert\frac{\partial f}{\partial \phi_{r}}+\kappa^{2}\hat{f}^{0}\,\phi_{r}^{\ast}\biggr\rvert^{2} \nonumber \\
    &\quad-2\operatorname{Re}\sum_{ras}\phi_{s}\,\biggl(\frac{\partial A_{rs}}{\partial y_{a}^{\ast}}\biggr)^{0}\,
    \biggl[\frac{\partial f}{\partial \phi_{r}}+\kappa^{2}\hat{f}^{0}\,\phi_{r}^{\ast}\biggr]\,\hat{L}_{a}^{0\ast} \nonumber \\
    &\quad-  \sum_{abrs}\Biggl[\phi_{r}\phi_{s}^{\ast}\biggl(\frac{\partial^{2}A_{rs}}{\partial y_{a}\partial y_{b}^{\ast}}\biggr)^{0}+\sum_{rs}\phi_{r}\phi_{s} \biggl(\frac{\partial^{2}B_{rs}}{\partial y_{a}\partial y_{b}^{\ast}}\biggr)^{0} \nonumber \\
    &\quad\qquad +\sum_{rs}\phi_{r}^{\ast}\phi_{s}^{\ast} \biggl(\frac{\partial^{2}B_{rs}^{\ast}}{\partial y_{a}\partial y_{b}^{\ast}}\biggr)^{0}\Biggr] \hat{L}_{a}^{0}\hat{L}_{b}^{0\ast} - 6\kappa^{2}\operatorname{Re}(f\hat{f}^{0\ast}) \nonumber \\
    &\quad +\frac{1}{2}\operatorname{Re}\sum_{AB}[f^{-1}_{AB}]^{0}\Biggl(\sum_{rs}\phi_{r}^{\ast}
    (t_{A})_{rs}\phi_{s}\Biggr) \Biggl(\sum_{tu}\phi_{t}^{\ast}(t_{B})_{tu}\phi_{u}\Biggr)^{\ast} +\mathscr{C} \:. \label{31.7.40}
\end{align}


可观则分区标量场的势(\ref{31.7.40})的势正是\,\ref{sec:28.4}\,节中所讨论的最小超对称标准模型所假定的形式: 它是超对称项$\,V_{\text{susy}}\,$和软超对称破缺项$\,V_{\text{soft}}\,$的和
\begin{equation}
    V_{O} = V_{\text{susy}}+V_{\text{soft}} \:. \label{31.7.41}
\end{equation}
同往常一样, 超对称项是
\begin{equation}
     V_{\text{susy}}=\sum_{r}\,\biggl\lvert\frac{\partial f}{\partial \phi_{r}}\biggr\rvert^{2}
     +\frac{1}{2}\operatorname{Re}\sum_{AB}[f^{-1}_{AB}]^{0}\Biggl(\sum_{rs}\phi_{r}^{\ast}
    (t_{A})_{rs}\phi_{s}\Biggr) \Biggl(\sum_{tu}\phi_{t}^{\ast}(t_{B})_{tu}\phi_{u}\Biggr)^{\ast} \label{31.7.42}
\end{equation}
而软超对称破缺项在这里是
\begin{align}
    V_{\text{soft}} &\simeq 2\kappa^{2}\operatorname{Re}\sum_{r}\Biggl(\phi_{r}\frac{\partial f}{\partial \phi_{r}}\hat{f}^{0\ast} \Biggr) + \kappa^{4}\,\Bigl\lvert\hat{f}^{0}\Bigr\rvert^{2} \sum_{r}\lvert\phi_{r}\rvert^{2} \nonumber \\
    &\quad -2\operatorname{Re}\sum_{ras}\phi_{s}\,\biggl(\frac{\partial A_{rs}}{\partial y_{a}^{\ast}}\biggr)^{0}\,
    \biggl[\frac{\partial f}{\partial \phi_{r}}+\kappa^{2}\hat{f}^{0}\,\phi_{r}^{\ast}\biggr]\,\hat{L}_{a}^{0\ast} \nonumber \\
    &\quad-  \sum_{abrs}\Biggl[\phi_{r}\phi_{s}^{\ast}\biggl(\frac{\partial^{2}A_{rs}}{\partial y_{a}\partial y_{b}^{\ast}}\biggr)^{0}+\sum_{rs}\phi_{r}\phi_{s} \biggl(\frac{\partial^{2}B_{rs}}{\partial y_{a}\partial y_{b}^{\ast}}\biggr)^{0} \nonumber \\
    &\quad\qquad +\sum_{rs}\phi_{r}^{\ast}\phi_{s}^{\ast} \biggl(\frac{\partial^{2}B_{rs}^{\ast}}{\partial y_{a}\partial y_{b}^{\ast}}\biggr)^{0}\Biggr] \hat{L}_{a}^{0}\hat{L}_{b}^{0\ast} - 6\kappa^{2}\operatorname{Re}(f\hat{f}^{0\ast})
    +\mathscr{C} \:. \label{31.7.43}
\end{align}
当$\,f(\phi)\,$由方程(\ref{31.7.20})给出时, 它采取形式
\begin{equation}
    V_{\text{soft}}= \sum_{rs}M_{rs}^{2}\phi_{r}\phi_{s}^{\ast} + 2\operatorname{Re}\sum_{rs}N_{rs}^{2}\phi_{r}\phi_{s}
    +2\operatorname{Re}\sum_{rst}A_{rst}\phi_{r}\phi_{s}\phi_{t}+\mathscr{C} \:, \label{31.7.44}
\end{equation}
其中
\begin{equation}
    M_{rs}^{2}=\kappa^{4}\lvert \hat{f}^{0}\rvert^{2}\delta_{rs} -2\kappa^{2}\operatorname{Re}
    \Biggl[\hat{f}^{0}\sum_{a}\biggl(\frac{\partial A_{rs}}{\partial y_{a}^{\ast}}\biggr)^{0} \,\hat{L}_{a}^{0\ast}\Biggr]
    -\sum_{ab}\Biggl(\frac{\partial^{2}A_{rs}}{\partial y_{a}\partial y_{b}^{\ast}}\Biggr)^{0}\, \hat{L}_{a}^{0}\hat{L}_{b}^{0\ast} \:, \label{31.7.45}
\end{equation}
\begin{align}
    N_{rs}^{2} &= 2\kappa^{2}\mu_{rs}\hat{f}^{0\,\ast} -\frac{1}{2}\sum_{at}\mu_{tr}
    \biggl(\frac{\partial A_{ts}}{\partial y_{a}^{\ast}}\biggr)^{0} \,\hat{L}_{a}^{0\ast} -\frac{1}{2}\sum_{at}\mu_{tr}
    \biggl(\frac{\partial A_{tr}}{\partial y_{a}^{\ast}}\biggr)^{0} \,\hat{L}_{a}^{0\ast} \nonumber \\
    &\quad -\sum_{ab}\Biggl(\frac{\partial^{2}B_{rs}^{\ast}}{\partial y_{a}\partial y_{b}^{\ast}}\Biggr)^{0} \hat{L}_{a}^{0}\hat{L}_{b}^{0\ast} \:, \label{31.7.46}
\end{align}
且
\begin{equation}
    A_{rst} = -\sum_{au}\Biggl[\biggl(\frac{\partial A_{ur}}{\partial y_{a}^{\ast}}\biggr)^{0} g_{ust}
    +\biggl(\frac{\partial A_{us}}{\partial y_{a}^{\ast}}\biggr)^{0} g_{urt}
    +\biggl(\frac{\partial A_{ut}}{\partial y_{a}^{\ast}}\biggr)^{0} g_{urs}\Biggr] \hat{L}_{a}^{0\ast} \:. \label{31.7.47}
\end{equation}
我们前面的量级估计给出
\begin{equation}
    M_{rs}^{2}=O(\kappa^{4}\Lambda^{6}) \:, \qquad N_{rs}^{2}=O(\kappa^{4}\Lambda^{6}) \:, \qquad
    A_{rst}= O(\kappa^{2}\Lambda^{3}) \:. \label{31.7.48}
\end{equation}
连同我们对$\,V_{\text{susy}}\,$中的常数做出的估计$\,g_{rst}=O(1)\,$和$\,\mu_{rs}=O(\kappa^{2}\Lambda^{3})$, 这表明, 如果势在某个$\,\phi_{a}\neq 0\,$处有稳定点, 那么$\,\phi_{a}^{0}=O(\kappa^{2}\Lambda^{3})$, 证实了我们把要探测的场空间定在这一阶的合理性. 对于$\,\phi_{r}\,$这一阶的平衡值, 势中各项的平衡值是$\,O(\phi^{4})=O(\kappa^{8}\Lambda^{12})$, 所以这是需要抵消真空能的常数$\,\mathscr{C}\,$的量级.

为了使最小超对称标准模型中的特征质量$\,\kappa^{2}\Lambda^{3}\,$在量级上是$\,1\,\mathrm{TeV}$, 我们需要$\,\Lambda\approx 10^{13}\,\mathrm{GeV}$. 如果如上面所建议的那样, Peccei--Quinn\,对称性禁止了裸超势中有两个可观测分区场和两个隐藏分区场的项, 那么隐藏分区标量的真空期望值将会破坏这个对称性, 而对称性破缺标度(在\,23.6\,节记做\,$M$)在量级上是$\,10^{13}\,\mathrm{GeV}$, 并且方程(\ref{23.6.26})会给出一个量级为$\,10^{-6}\,\mathrm{eV}\,$的轴子. 对称性破缺标度的值要比\,23.6\,节引用的上界$\,10^{12}\,\mathrm{GeV}\,$稍微高一些, 但由于宇宙学讨论的不确定性, 这个矛盾不是决定性的.


在考虑这些结果进一步的物理应用之前, 我们重新考虑一些在推导这些结果时所走的捷径. 在计算可观测分区标量$\,\phi_{r}\,$的势时, 我们将模场固定到了它们在没有可观测分区标量场$\,\phi_{r}\,$时所处的平衡值$\,y_{a}^{0}$. 诚然, 我们应该把模场设为它们对$\,\phi_{r}\,$的真实值所处的平衡值$\,y_{a}(\phi)$, 方法是找到势
\begin{equation}
    V_{\text{total}}(\phi,\phi^{\ast},y,y^{\ast})= \hat{V}(y,y^{\ast})+V_{O}(\phi,\phi^{\ast},y,y^{\ast})\label{31.7.49}
\end{equation}
的稳定点, 在这之后再寻找$\,\phi_{r}\,$的平衡值. 由于对感兴趣的场, $V_{O}\,$远小于$\,\hat{V}$, $y_{a}\,$的平衡值可以写为
\begin{equation}
    y_{a}(\phi,\phi^{\ast}) = y_{a}^{0}+\delta y_{a}(\phi,\phi^{\ast}) \:, \label{31.7.50}
\end{equation}
其中$\,y_{a}^{0}\,$处在$\,\hat{V}(y,y^{\ast})\,$的最小值且
\begin{equation}
    \sum_{b}\frac{\partial^{2} \hat{V}}{\partial y_{a}\partial y_{b}}\delta y_{b}
    +\sum_{b}\frac{\partial^{2} \hat{V}}{\partial y_{a}\partial y_{b}^{\ast}}\delta y_{b}^{\ast}
    =-\frac{\partial V_{O}}{\partial y_{a}} \:. \label{31.7.51}
\end{equation}
$\hat{V}\,$的二阶导数是$\,\kappa^{2}\times(\kappa\Lambda^{3})^{2}=\kappa^{4}\Lambda^{6}\,$阶的, 而$\,V_{O}\,$的一阶导数是$\,\kappa\times \kappa^{8}\Lambda^{12}=\kappa^{9}\Lambda^{12}\,$阶的, 所以 $\delta y_{a}\,$是$\,\kappa^{5}\Lambda^{6}\,$阶. 因$\,y_{a}\,$的平衡值附近的这个$\,\phi_{a}$-相关偏移对势产生的变化是$\,\delta y_{a}\,$和$\,\delta y_{a}^{\ast}\,$的二次型, 其系数由$\,\hat{V}\,$相对$\,y_{a}\,$和(或)$\,y_{a}^{\ast}\,$的二阶导数给出, 因此在量级上是
\[
(\kappa^{5}\Lambda^{6})^{2}\times \kappa^{2}\times(\kappa\Lambda^{3})^{2}=\kappa^{14}\Lambda^{18} \:,
\]
这比我们计算出的势要小一个因子$\,(\kappa\Lambda)^{6}\ll 1$.

当不考虑对函数$\,A_{rs}(y,y^{\ast})\,$和$\,B(y,y^{\ast})\,$做进一步假定时, 对软超对称破缺势(\ref{31.7.44})中的系数 $M_{rs}^{2}$, $N_{rs}^{2}$\,和$\,A_{rst}$的精确值, 我们无法从方程(\ref{31.7.45})---(\ref{31.7.47})中得到任何信息. 唯一从这些结果中得到的确定预测是软超对称破缺拉格朗日量(\ref{28.4.1})中的系数$\,C_{ij}\,$均可忽略, 正如通常所假定的.

结果(\ref{31.7.44})---(\ref{31.7.48})中所呈现处的最大问题是: 没有进一步的假定, 它们不能确保标量夸克质量和标量轻子质量的简并性, 也就不能避免\,\ref{28.4}\,节中讨论的味改变过程. $SU(3)\times SU(2)\times U(1)\,$只允许软超对称破缺势(\ref{31.7.44})以及超势(\ref{31.7.20})中的$\,\phi_{r}\phi_{s}$-项依赖\,Higgs\,标量, 所以它们不能给出味改变过程. 因此这个问题的源头就是: 方程(\ref{31.7.44})中的系数$\,M_{rs}^{2}\,$和$\,A_{rst}\,$在与\,Yukawa 耦合$\,g_{rst}\,$相同的基中没有保持味守恒. 避免这个问题的方法是: 基于某个原因, 函数$\,A_{rs}(y,y^{\ast})\,$恰巧非常弱地依赖于$\,y_{a}\,$和$\,y_{a}^{\ast}$, 使得方程(\ref{31.7.45})给出$\,M_{rs}^{2}\propto\delta_{rs}\,$以及方程(\ref{31.7.47})使得$\,A_{rst}\,$异常的小(方程(\ref{28.4.1})中的系数$\,A_{ij}\,$也随之很小). 另一种可能性是, 尽管不是缓慢变化, 但由于某个原因, 整个函数$\,A_{rs}(y,y^{\ast})$(或者至少是它在$\,y_{a}=y_{a}^{0}\,$处的一阶导数和二阶导数)正比于$\,\delta_{rs}$. 在这个情况下, 方程(\ref{31.7.45})再次给出$\,M_{rs}^{2}\propto \delta_{rs}$, 而方程(\ref{31.7.47})现在给出三线性耦合$\,A_{rst}\propto g_{rst}$, 这使得方程(\ref{28.4.1})中的系数$\,A_{ij}\,$都是相等的.\cite{21}

我们也必须检验这个版本的引力传递超对称破缺产生的规范微子质量. 根据方程(\ref{31.6.75}), $SU(3)\times SU(2)\times U(1)\,$规范微子的质量矩阵一般是
\begin{equation}
    m_{AB}=\exp(\kappa^{2}d/2)\sum_{NM}[g^{-1}]_{NM}L_{N}\,\biggl(\frac{\partial f_{AB}}{\partial \varphi_{M}}\biggr)^{\ast}\:.
    \label{31.7.52}
\end{equation}
其中这里的$\,\varphi_{N}\,$取遍$\,f_{AB}\,$可能依赖的所有场, 而$\,g_{NM}\,$和$\,L_{M}\,$由方程(\ref{31.6.66})和(\ref{31.6.69})给出. 根据我们这里做出的估计, $\kappa^{2}d\ll 1$, 对模场$\,y_{a}$, $L_{a}=O(\kappa\Lambda^{3})\,$且远小于其他场; 以及$\,g_{ab}^{-1}\simeq \delta_{ab}$. 另外, 我们假定$\,f_{AB}\,$是$\,\kappa y\,$系数量级为\,1\,的幂级数, 所以$\,\partial f_{AB}/\partial y_{a}\,$是$\,\kappa\,$阶的 规范微子质量(\ref{31.7.52})因此是$\,\kappa^{2}\Lambda^{3}\,$阶的, 这与标量质量和期望值在量级上相同, 因此足够重以至于很可能避免了与观测的矛盾. \ref{31.4}\,节考虑的单圈修正在这里要小的多, 不需要考虑在内.

总结一下, 第一版的引力传递超对称破缺的优势在于它所给出的轴子质量在宇宙学限制之内, 而第二版的优势在于它所给出的规范微子质量在量级上与标量夸克和标量轻子的质量相称. 相较于规范传递的超对称破缺, 引力传递超对称破缺的两个版本有一个共同的优势: 它们自然地会给出实验所需要量级的$\,\mu$-项. 另一方面, 规范传递超对称破缺的理论拥有的优势在于它们会自然的给出与代无关的标量夸克和标量轻子质量.

无论是那个版本的引力传递超对称破缺理论, 它们会自然需要存在一个缓慢衰变的超重粒子, 这或许会有有趣的天文学效应.\cite{22} 
在能量$\,\Lambda\,$处变强的隐藏分区规范耦合可能组合出质量量级为$\,\Lambda\,$的复合粒子. 如果超重粒子的衰变被隐藏分区拉格朗日量的可重整部分的偶然对称性禁止了, 且仅能通过拉格朗日量的不可重整项发生, 而这些项被$\,\kappa\Lambda\,$因子压低了, 那么它们的寿命将会很长. 


%+++++++++++++++++++++++附录++++++
\titleformat{\chapter}{\centering\CJKfamily{zhhei}\huge}{\chaptertitlename}{1em}{}
\titlespacing{\chapter}{0pt}{3.5ex plus .1ex minus .2ex}{10\wordsep}
\titleformat{\section}{\centering\CJKfamily{zhhei}\Large}{附 录}{1em}{}
\titlespacing{\section}{2em}{3.5ex plus .1ex minus .2ex}{1.5\wordsep}
\titleformat{\subsection}{\centering\CJKfamily{zhhei}\large}{}{0em}{}
\titlespacing{\subsection}{2em}{1.5ex plus .1ex minus .2ex}{\wordsep}
\renewcommand{\captionfont}{\small} \newcounter{app31}[chapter]
\setcounter{app31}{1}
\renewcommand\thesection{\Alph{app31}}
\renewcommand\theequation{\arabic{chapter}.\Alph{app31}.\arabic{equation}}
\fancyhf{} \fancyhead[CE]{\leftmark} \fancyhead[CO]{\rightmark}
\fancyhead[RO,LE]{$\cdot$\ \thepage\ $\cdot$}
\renewcommand{\headrulewidth}{0.8pt} \pagestyle{fancy}
\renewcommand{\chaptermark}[1]{\markboth{第\,\thechapter\,章\ #1}{}} \renewcommand{\sectionmark}[1]{\markright{附录 \quad\ #1}{}}


\section{标架形式体系}

当物质场被限制成标量、矢量、和张量, 我们熟知的用度规建立的引力形式理论是足够的, 但对于超引力, 由于旋量是其不可或缺的元素, 这个形式理论是不够的. 不像矢量和张量, 旋量的\,Lorentz\,变换规则没有到任意坐标系的自然推广. 取而代之, 为了处理旋量, 我们将不得不引入坐标系$\,\xi^{a}_{X}(x)$, 其中$\,a=0,1,2,3$, 它在任意坐标系的任意给定点$\,X\,$是定域惯性的. 惯性原理告诉我们引力对这些定域惯性坐标没有影响, 所以作用量可以表示成在这些定域惯性系下定义的物质场, 例如旋量、矢量等, 以及定域惯性坐标和广义坐标之间的变换产生的标架
\begin{equation}
e^{a}{}_{\mu}(X) \equiv \frac{\partial \xi_{X}^{a}(x)}{\partial x^{\mu}}\biggr\rvert_{x=X} \:. \label{31.A.1}
\end{equation}
整个作用量将在广义坐标变换$\,x^{\mu}\to x'^{\mu}\,${\kai{和}}满足$\Lambda^{a}{}_{c}(x)\Lambda^{b}{}_{d}(x)\eta_{ab}=\eta_{cd}\,$定域\,Lorentz\,变换$\,\xi^{a}\to\xi'^{a}=\Lambda^{a}{}_{b}(x)\xi^{b}\,$下不变. 标架的定义(\ref{31.A.1})表明, 在广义坐标变换$\,x\to x'\,$下, 它的变换是
\begin{equation}
    e^{a}{}_{\mu}(x) \to e^{\prime a}{}_{\mu}(x^{\prime})
    = \frac{\partial x^{\nu}}{\partial x^{\prime \mu}}e^{a}{}_{\nu}(x)\:, \label{31.A.2}
\end{equation}
而在定域\,Lorentz\,变换$\,\xi^{a}(x)\to\Lambda^{a}{}_{b}(x)\xi^{b}(x)\,$下, 它的变换是
\begin{equation}
     e^{a}{}_{\mu}(x) \to \Lambda^{a}{}_{b}(x) e^{b}{}_{\mu}(x) \:. \label{31.A.3}
\end{equation}

例如, 纯引力的理论可以表示成在定域\,Lorentz\,变换下不变而在广义坐标变换下按张量变换的场. 即度规
\begin{equation}
    g_{\mu\nu} \equiv e^{a}{}_{\mu}e^{b}{}_{\nu}\,\eta_{ab} \:. \label{31.A.4}
\end{equation}
矢量可以视为在定域\,Lorentz\,变换下
\begin{equation}
    V^{a}(x) \to \Lambda^{a}{}_{b}(x)V^{b}(x)  \label{31.A.5}
\end{equation}
按照矢量变换的量$\,V^{a}$, 但它在广义坐标变换下则是标量, 或者视为在定域\,Lorentz\,变换下按照标量变换而在广义坐标变换下按照矢量变换的量$\,v^{\mu}$, 二者的关系是
\[
V^{a} = e^{a}{}_{\mu}v^{\mu} \:.
\]
但超引力作用量也包含旋量场, 它必然在广义坐标变换下是个标量但在定域\,Lorentz\,变换下按照旋量变换:
\begin{equation}
    \psi_{\alpha}(x) \to D_{\alpha\beta}(\Lambda(x))\psi_{\beta}(x) \:, \label{31.A.6}
\end{equation}
其中$\,D_{\alpha\beta}(\Lambda)\,$是齐次\,Lorentz\,群的旋量表示.

因为方程(\ref{31.A.5})和(\ref{31.A.6})中的\,Lorentz\,变换依赖于坐标$\,x^{\mu}$, 像$\,V^{a}(x)\,$或者$\,\psi_{\alpha}(x)\,$这种量的时空导师不仅仅是另外一个在定域\,Lorentz\,变换以相同方式变换且在广义坐标变换下按协变矢量变换的量. 例如, 方程(\ref{31.A.6})的导数给出定域\,Lorentz\,变换规则
\[
\partial_{\mu}\psi_{\alpha} \to D_{\alpha\beta}(\Lambda)\biggl\{\partial_{\mu}\psi_{\beta}
+\Bigl[D^{-1}(\Lambda)\partial_{\mu}D(\Lambda)\Bigr]_{\beta\gamma}\psi_{\gamma}\biggr\} \:.
\]
为了抵消右边括号中的第二项, 我们需要引入联络矩阵$\,\Omega_{\mu}$, 其有定域\,Lorentz\,变换性质
\begin{equation}
    \Omega_{\mu} \to D(\Lambda)\Omega_{\mu}D^{-1}(\Lambda) - \Bigl(\partial_{\mu}D(\Lambda)\Bigr)D^{-1}(\Lambda) \label{31.A.7}
\end{equation}
并定义协变导数
\begin{equation}
    \mathscr{D}_{\mu}\psi \equiv \partial_{\mu}\psi + \Omega_{\mu}\psi \:, \label{31.A.8}
\end{equation}
它在定域\,Lorentz\,变换下像$\,\psi\,$自身那样变换:
\begin{equation}
    \mathscr{D}_{\mu}\psi \to D(\Lambda)\mathscr{D}_{\mu}\psi \:. \label{31.A.9}
\end{equation}
另外, $\Omega_{\mu}\,$在广义坐标变换下必须项协变矢量那样变换, 这使得当$\,\mathscr{D}_{\mu}\,$作用在坐标标量上时将给出一个协变矢量. 为了使$\,\mathscr{D}_{\mu}\,$作用在张量上给出有一个额外下指标的张量, 必须要给它补上通常的仿射联络项. 例如, 当作用在引力微子场$\,\psi_{\mu}\,$上时, 协变导数定义成
\begin{equation}
     \mathscr{D}_{\mu}\psi \equiv \psi_{\nu;\mu} + \Omega_{\mu}\psi_{\nu} \equiv \partial_{\mu}\psi_{\nu}
     -\Gamma_{\mu\nu}^{\lambda}\psi_{\lambda} + \Omega_{\mu}\psi_{\nu} \:. \label{31.A.10}
\end{equation}


方程(\ref{31.A.8})---(\ref{31.A.10})不仅适用于旋量, 对于那些在定域\,Lorentz\,变换下按照\,Lorentz\,群的任意表示$\,D(\Lambda)\,$变换的场, 这些同样是适用的. 矩阵$\,\Omega_{\mu}\,$依赖于这个表示, 但在任何表示下, 它可以写成
\begin{equation}
    [\Omega_{\mu}]_{\alpha\beta}(x) =\frac{1}{2}\mi[\mathscr{J}_{ab}]_{\alpha\beta}\omega_{\mu}^{ab}(x) \:, \label{31.A.11}
\end{equation}
其中$\,\mathscr{J}_{ab}\,$是问题中的场所构成的齐次\,Lorentz\,群表示生成元的矩阵:
\begin{equation}
    \mi\,[\mathscr{J}_{ab},\mathscr{J}_{cd}]= \eta_{bc}\mathscr{J}_{ad} -\eta_{ac}\mathscr{J}_{bd}
    +\eta_{bd}\mathscr{J}_{ca}- \eta_{ad}\mathscr{J}_{cb} \:, \label{31.A.12}
\end{equation}
而$\,\omega_{\mu}^{ab}\,$是表示无关的场, 称为{\kai{自旋联络}}, 它在广义坐标变换下按照协变矢量变换. 为了满足非齐次定域\,Lorentz\,变换规则(\ref{31.A.7}), 我们可以取
\begin{equation}
    \omega_{\mu}^{ab} = g^{\nu\lambda} e^{a}{}_{\nu}e^{b}{}_{\lambda;\mu} \:, \label{31.A.13}
\end{equation}
其中的分号依旧表示使用仿射联络$\,\Gamma_{\mu\nu}^{\lambda}\,$构造的普通协变导数. (因为方程(\ref{31.A.4})给出$\,g^{\nu\lambda}e^{a}{}_{\nu}e^{b}{}_{\lambda}=\eta^{ab}$, 一个协变导数为为零的项, 所以自旋联络关于指标$\,a\,$和$\,b\,$是反对称的.) 这不是唯一满足方程(\ref{31.A.7})的自旋联络; 我们可以给它加上任何在广义坐标变换下是个协变矢量而在定域\,Lorentz\,变换下是个张量的场, 这是超引力理论中一个重要的自由度.

对于自旋联络的任何选择, 存在一个相应的曲率张量. 从方程(\ref{31.A.7})可以直接证明$\,\partial_{\nu}\Omega_{\mu}-\partial_{\mu}\Omega_{\nu}+[\Omega_{\nu},\Omega_{\mu}]\,$在定域\,Lorentz\,变换下进行齐次变换
\begin{equation}
    \partial_{\nu}\Omega_{\mu} - \partial_{\mu}\Omega_{\nu} + [\Omega_{\nu},\Omega_{\mu}]
    \to D(\Lambda)\,(\partial_{\nu}\Omega_{\mu} - \partial_{\mu}\Omega_{\nu} + [\Omega_{\nu},\Omega_{\mu}])\,D^{-1}(\Lambda) \:.\label{31.A.14}
\end{equation}
利用方程(\ref{31.A.11})和(\ref{31.A.12}), 这个矩阵可以表示成
\begin{equation}
    \partial_{\nu}\Omega_{\mu} - \partial_{\mu}\Omega_{\nu} + [\Omega_{\nu},\Omega_{\mu}]
    =\tfrac{1}{2}\mi\,\mathscr{J}_{ab}R_{\mu\nu}{}^{ab} \:, \label{31.A.15}
\end{equation}
其中
\begin{equation}
    R_{\mu\nu}{}^{ab} \equiv \partial_{\nu}\omega_{\mu}^{ab} -\partial_{\mu}\omega_{\nu}^{ab}
    +\omega_{\nu}^{ac}\omega_{\mu\,c}{}^{b} -\omega_{\mu}^{ac}\omega_{\nu\,c}{}^{b} \:. \label{31.A.16}
\end{equation}
从方程(\ref{31.A.14})可以得出$\,R_{\mu\nu}{}^{ab}\,$在定域\,Lorentz\,变换按照张量变换
\begin{equation}
    R_{\mu\nu}{}^{ab} \to D(\Lambda)^{a}{}_{c}D(\Lambda)^{b}{}_{d}R_{\mu\nu}{}^{cd} \:. \label{31.A.17}
\end{equation}
它在广义坐标变换下显然按照张量变换
\begin{equation}
    R_{\mu\nu}{}^{ab} \to \frac{\partial x^{\rho}}{\partial x^{\prime\mu}}
    \frac{\partial x^{\sigma}}{\partial x^{\prime\nu}} R_{\rho\sigma}{}^{ab} \:. \label{31.A.18}
\end{equation}
通过写下
\begin{equation}
    R_{\mu\nu}{}^{ab} = e^{a}{}_{\kappa}\,e^{b}{}_{\lambda}\,R_{\mu\nu}{}^{\kappa\lambda} \:. \label{31.A.19}
\end{equation}
我们可以构造出一个四秩的坐标张量. 以这种方式构造出的张量$\,R_{\mu\nu}{}^{\kappa\lambda}\,$是与特定自旋联络$\,\omega_{\nu}^{ab}\,$相对应的\,Riemann--Christoffel\,曲率张量.

\section*{习题}
\noindent 1. 推导出\,Einstein\,超场分量的公式(\ref{31.2.3})---(\ref{31.2.6}). \\

\noindent 2. 假定超对称不破缺. 展示如何计算在一个一般过程中发射一个能量很低的引力微子的振幅, 并写成这个过程不带引力微子的振幅. \\

\noindent 3. 验证超引力作用量(\ref{31.6.11})在定域超对称变换(\ref{31.6.1})---(\ref{31.6.6})下直到$\,G\,$的所有阶都是不变的. \\

\noindent 4. 计算推广的\,$D$-分量在一般定域超对称变换下的变化. \\

\noindent 5. 计算拉格朗日密度(\ref{31.6.49})的费米部分. \\

\noindent 6. 考虑单个手征标量超场$\,\Phi\,$与超引力相互作用的理论, 其有修正\,Kahler\,势$\,d(\Phi,\Phi^{\ast})=\Phi^{\ast}\Phi$\,和超势$\,f(\Phi)=M^{2}(\Phi+\beta)$, 其中$\,M\,$和$\,\beta\,$是常数. 找到$\,\beta\,$的值使得经典场方程拥有平坦时空的解. 对这个解, $\phi\,$取何值?






%++++++++++++++++++参考文献+++++++++
\renewcommand{\sectionmark}[1]{\markright{ #1}{}}
\renewcommand{\bibname}{参考文献}

\begin{thebibliography}{99}
    \bibitem{1} P. Nath and R. Arnowitt, {\textit{Phys. Lett.}} {\bf{56B}}, 177 (1975); B. Zumino, 收录于\,{\textit{Proceedings of the Conference on Gauge Theories and Modern Field Theories at Northeastern University, 1975}}, R. Arnowitt and P. Nath\,编辑\,(MIT Press, Cambridge, MA, 1976). J. Wess\,和\,J. Bagger\,细致地描述了这个方法, {\textit{Supersymmetry and Supergravity,}} 2nd edition (Princeton University Press, Princeton, NJ, 1992).
    \bibitem{2} D. Z. Freedman, P. van Nieuwenhuizen, and S. Ferrara, {\textit{Phys. Rev.}} {\bf{D13}}, 3214 (1976); S. Deser and B. Zumino, {\textit{Phys. Lett.}} {\bf{62B}}, 335 (1976); S. Ferrara, J. Scherk, and P. van Nieuwenhuizen, {\textit{Phys. Rev. Lett.}} {\bf{37}}, 1035 (1976); S. Ferrara, F. Gliozzi, J. Scherk, and P. van Nieuwenhuizen, {\textit{Nucl. Phys.}} {\bf{B117}}, 333 (1976). 这些文章重印于\,{\textit{Supersymmetry}}, S. Ferrara\,编辑\,(North Holland/World Scientific, Amsterdam/Singapore, 1987). 关于这个方法的清楚描述, 参看\,P. West, {\textit{Introduction to Supersymmetry and Supergravity}}, 2nd edition (World Scientific, Singapore, 1990).
    \bibitem{3} S. Ferrara and B. Zumino, {\textit{Nucl. Phys.}} {\bf{B134}}, 301 (1978).
    \bibitem[3a]{3a} M. T. Grisaru and H. N. Pendleton, {\textit{Phys. Lett.}} {\bf{67B}}, 323 (1977).
    \bibitem{4} K. Stelle and P. C. West, {\textit{Phys. Lett.}} {\bf{74B}}, 330 (1978); S. Ferrara and P. van Nieuwenhuizen, {\textit{Phys. Lett.}} {\bf{74B}}, 333 (1978). 这些文章重印于\,{\textit{Supersymmetry}}, 参考文献[2].
    \bibitem{5} 例如, 可参看, S. Weinberg, {\textit{Gravitation and Cosmology}} (Wiley, New York, 1972), Sec. 12.5.
    \bibitem{6} 例如, 可参看, {\textit{Gravitation and Cosmology}}, 参考文献[5]: 方程(12.4.3).
    \bibitem{7} 例如, 可参看, {\textit{Gravitation and Cosmology}}, 参考文献[5], Section 10.1
    \bibitem[7a]{7a} W. Nahm, {\textit{Nucl. Phys.}} {\bf{B135}}, 149 (1978).
    \bibitem{8} S. Coleman and F. de Luccia, {\textit{Phys. Rev.}} {\bf{D21}}, 3305 (1980).
    \bibitem{9} S. Weinberg, {\textit{Phys. Rev. Lett.}} {\bf{48}}, 1176 (1982).
    \bibitem{10} S. Deser and C. Teitelboim, {\textit{Phys. Rev. Lett.}} {\bf{39}}, 249 (1977); M. Grisaru, {\textit{Phys. Lett.}} {\bf{73B}}, 207 (1978); E. Witten, \textit{Commum. Math. Phys.} {\bf{80}}, 381 (1981).
    \bibitem{11} W. Rarita and J. Schwinger, {\textit{Phys. Rev.}} {\bf{60}}, 61 (1941).
    \bibitem[11a]{11a} L. Randall and R. Sundrum, hep-th/9810155, 待发表; G. F. Giudice, M. Luty, R. Rattazzi, and H. Murayama, {\textit{JHEP}}, {\bf{12}}, 027 (1998); A. Pomerol and R. Rattazzi, hep-ph/9903448, 待发表; E. Katz, Y. Shadmi, and Y. Shirman, hep-ph/9906296, 待发表.
    \bibitem[11b]{11b} G. F. Giudice, M. Luty, R. Rattazzi, and H. Murayama, 参考文献[11a].
    \bibitem{12} K. Stelle and P. C. West, 参考文献[4]; S. Ferrara and P. van Nieuwenhuizen, 参考文献[4]; E. Cremmer, B. Julia, J. Scherk, S. Ferrara, L. Girardello, and P. van Nieuwenhuizen, {\textit{Phys. Lett.}} {\bf{79B}}, 231 (1978); {\textit{Nucl. Phys.}} {\bf{B147}}, 105 (1979) (重印于\,\textit{Supersymmetry}, 参考文献[2]); D. G. Boulware, S. Deser, and J. H. Kay, {\textit{Physica}} {\bf{280}}, 141 (1979); E. Cremmer, S. Ferrara, L. Girardello, and A. Van Proeyen, {\textit{Phys. Lett.}} {\bf{116B}}, 231 (1982); \textit{Nucl. Phys.} {\bf{B212}}, 413 (1983) (重印于\,\textit{Supersymmetry}, 参考文献[2]). 关于两维中的这种构造方法, 参看\,S. Deser and B. Zumino, {\textit{Phys. Lett.}} {\bf{65B}}, 369 (1976).
    \bibitem{13} S. Deser and B. Zumino, {\textit{Phys. Rev. Lett.}} {\bf{38}}, 1433 (1977). 这篇文章重印于\,\textit{Supersymmetry}, 参考文献[2]. 另见\,D. Z. Freedman and A. Das, \textit{Nucl. Phys.} {\bf{B120}}, 221 (1977); P. K. Townsend, {\textit{Phys. Rept.}} {\bf{145}}, 1 (1987).
    \bibitem[13a]{13a} E. Cremmer, S. Ferrara, C. Kounnas, and D. V. Nanopoulos, {\textit{Phys. Lett.}} {\bf{133B}}, 61 (1983). 对基于这个概念的模型综述, 参看\,A. B. Lahanas and D. V. Nanopoulos, {\textit{Phys. Rept.}} {\bf{145}}, 1 (1987).
    \bibitem{14} H. P. Nilles, {\textit{Phys. Lett.}} {\bf{115B}}, 193 (1982); A. Chamseddine, R. Arnowitt, and P. Nath, {\textit{Phys. Rev. Lett.}} {\bf{49}}, 970 (1982); R. Barbieri, S. Ferrara, and C. A. Savoy, {\textit{Phys. Lett.}} {\bf{119B}}, 343 (1982); E. Cremmer, P. Fayet, and L. Girardello, {\textit{Phys. Lett.}} {\bf{122B}}, 41 (1983); L. Iba\~{n}ez, {\textit{Phys. Lett.}} {\bf{118B}}, 73 (1982); H. P. Nilles, M. Srednicki, and D. Wyler, {\textit{Phys. Lett.}} {\bf{120B}}, 346 (1983); L. Hall, J. Lykken, and S. Weinberg, {\textit{Phys. Rev.}} {\bf{D27}}, 2359 (1983); L. Alvarez-Gaum\'{e}, J. Polchinski, and M. B. Wise, {\textit{Nucl. Phys.}} {\bf{B221}}, 495 (1983). 上述文章重印于\,\textit{Supersymmetry}, 参考文献[2]. 另见\,S. Ferrara, D. V. Nanopoulos, and C. A. Savoy, {\textit{Phys. Lett.}} {\bf{12B}}, 214 (1983); J. M. Leon, M. Quiros, and M. Ramon Medrano, {\textit{Phys. Lett.}} {\bf{127B}}, 85 (1983); {\textit{Phys. Lett.}} {\bf{129B}}, 61 (1983); N. Ohta, {\textit{Prog. Theor. Phys.}} {\bf{70}}, 542 (1983); P. Nath, R. Arnowitt, and A. Chamseddine, {\textit{Phys. Lett.}} {\bf{121B}}, 33 (1983); J. Ellis, D. V. Nanopoulos, and K. Tamvakis, {\textit{Phys. Lett.}} {\bf{121B}}, 123 (1983). 关于综述, 参看\,H. P. Nilles, {\textit{Phys. Rept.}} {\bf{110}}, 1 (1984).
    \bibitem{15} 超对称在有手征标量的不可观测分区的超引力理论中破缺似乎是由\,J. Polonyi\,首先在布达佩斯大学的预言本中出首次提出的, 未发表(1977). 
    \bibitem{16} I. Affleck, M. Dine, and N. Seiberg, {\textit{Nucl. Phys.}} {\bf{B256}}, 557 (1985).
    \bibitem{17} G. F. Giudice and A. Masiero, {\textit{Phys. Lett.}} {\bf{B206}}, 480 (1988). 另见\,J. A. Casas and C. Mu\~{n}oz, {\textit{Phys. Lett.}} {\bf{B306}}, 288 (1993); J. E. Kim, hep-ph/9901204, 待发表.
    \bibitem{18} M. Dine and D. A. MacIntire, {\textit{Phys. Rev.}} {\bf{D46}}, 2594 (1992).
    \bibitem{19} T. Banks, D. B. Kaplan, and A. Nelson, {\textit{Phys. Rev.}} {\bf{D49}}, 779 (1994); K. I. Izawa and T. Yanagida, {\textit{Prog. Theor. Phys.}} {\bf{94}}, 1105 (1995); A. Nelson, {\textit{Phys. Lett.}} {\bf{B369}}, 277 (1996).
    \bibitem[19a]{19a} T. Moroi\,和\,L. Randall\,讨论了\,$W\,$微子轻于$\,B\,$微子的宇宙学含义, hep-ph/9906527, 待发表.
    \bibitem{20} V. Kaplunovsky and J. Louis, {\textit{Phys. Lett.}} {\bf{B306}}, 269 (1993); {\textit{Nucl. Phys.}} {\bf{B422}}, 57 (1994). P. Bin\'{e}truy, M. K. Gaillard\,和\,Y-Y. Wu\,提出了这类具体模型, {\textit{Nucl. Phys.}} {\bf{B493}}, 27 (1997); {\textit{Phys. Lett.}} {\bf{B412}}, 288 (1997).
    \bibitem{21} L. Randall\,和\,R. Sundrum\,提出了味改变效应被压低的额外维理论, 参考文献[11a]. Z. Chacko, M. A. Luty, I. Maksymyk\,和\,E. Pont\'{o}n\,指出了这类理论的问题, hep-ph/9905390, 待发表.
    \bibitem{22} K. Hamaguchi, K.-I. Izawa, Y. Nomura, and T. Yanagida, hep-ph/9903207, 待发表.
\end{thebibliography}

\renewcommand\thesection{\thechapter.\arabic{section}}
%===============================================================================%
\fancyhf{}
\fancyhead[CE]{\leftmark}
\fancyhead[CO]{\rightmark}
\fancyhead[RO,LE]{$\cdot$\ \thepage\ $\cdot$}
\renewcommand{\headrulewidth}{0.8pt}
\renewcommand\theequation{\arabic{chapter}.\arabic{section}.\arabic{equation}}
\pagestyle{fancy}
\renewcommand{\chaptermark}[1]{\markboth{第\,\thechapter\,章\ #1}{}}
\renewcommand{\sectionmark}[1]{\markright{\thesection\ #1}{}}
%===========================章节的标题格式====================================%
\titleformat{\chapter}{\centering\CJKfamily{zhhei}\huge}{\chaptertitlename}{1em}{}
\titlespacing{\chapter}{0pt}{3.5ex plus .1ex minus .2ex}{10\wordsep}
\titleformat{\section}{\centering\CJKfamily{zhhei}\Large}{\thesection}{1em}{}
\titlespacing{\section}{2em}{3.5ex plus .1ex minus .2ex}{1.5\wordsep}
\titleformat{\subsection}{\centering\CJKfamily{zhkai}\large}{}{0em}{}
\titlespacing{\subsection}{2em}{1.5ex plus .1ex minus .2ex}{\wordsep}
\renewcommand{\captionfont}{\small}

%第三十二章 高维时空中的超对称代数

\chapter{高维中的超对称代数} \label{cha:32}

自\,Kaluza\cite{1}和\,Klein\cite{2}的开创性工作起, 理论家们时不时地尝试在高于四维的时空中建立更加基本的物理理论. 这个方法在在\,10\,维时空中采取最简单形式的弦理论\cite{3}中复兴了. 稍近一些, 有人提出弦论的各个版本可能被统一进一个理论中, 即$\,M$\,{\kai{理论}}, 这个理论的一个极限可以用\,11\,维时空中的超引力近似描述.\cite{4} 在本章, 我们将对高维中不同类型的超对称代数进行编录, 并用它们来分类粒子的超多重态. 

\section{一般超对称代数}  \label{sec:32.1}

我们对高维中的一般超对称代数的分析将沿用\,\ref{25.2}\,节中所讨论的\,Haag, Lopuszanski\,和\,Sohnius 对四维时空超对称代数采用的逻辑. 第\,\ref{cha:24}\,章末尾\,Coleman--Mandula\,定义的证明表明, 当时空维数$\,d>2\,$时, 可能的玻色超对称生成元与四维中的相同: 在粒子的一个$\,S$-矩阵理论中, 只有动量\,$d$-矢$\,P^{\mu}$, Lorentz\,生成元$\,J^{\mu\nu}=-J^{\nu\mu}\,$(其中$\,\mu\,$和$\,\nu\,$这里的取值范围是$\,1,2,\cdots,d{-}1,0$), 以及各种\,Lorentz\,标量``荷''. (在一些理论中存在点粒子以外的在拓扑上稳定的扩展物体, 例如闭弦、膜等, 我们会在\,\ref{sec:32.3}节回到这点.) 费米对称性生成元彼此之间的反对易子是玻色对称性生成元, 因而必须是$\,P^{\mu}$, $J^{\mu\nu}\,$和各种守恒标量的线性组合. 这给费米生成元的\,Lorentz\,变换性质以及它们所属的超代数附加了几个限制.

我们首先将证明一般费米对称性生成元将按照\,Lorentz\,群的基础旋量表示进行变换, 而不是更高的旋量表示, 例如那些通过给旋量加上矢量指标获得的, 我们会在本章附录简单回顾基础旋量表示. 正如我们在\,\ref{sec:25.2}\,节看到的, Haag, Lopuszanski\,和\,Sohnius\,对$\,d=4\,$给出的证明使用了$\,SO(4)$ 与$\,SU(2)\times SU(2)\,$同构, 而这在高维中没有类似物. 这里我们将使用\,Nahm\,的一个讨论,\cite{6} 这个讨论实际上更加简单并且适用于任何维度.

由于任何费米对称性生成元的\,Lorentz\,变换是另一个费米对称性生成元, 费米对称性生成元构成了齐次\,Lorentz\,群$\,O(d{-}1,1)\,$的一个表示(或者, 更严格地说, 是其覆盖群$\,Spin(d{-}1,1)\,$的一个表示). 假定费米对称对称性生成元的个数至多是有限多个, 它们必须按照齐次\,Lorentz\,群的有限维表示进行变换. 通过设$\,x^{d}=\mi x^{0}$, 所有这些表示可以从相应的正交群$\,O(d)\,$(实际上是$\,Spin(d)$)的有限维{\kai{幺正}}表示获得. 所以我们先来考虑费米生成元在$\,O(d)\,$下的变换. 当$\,d\,$为偶或奇时, 我们可以找到$\,d/2\,$或$\,(d{-}1)/2\,$个\,Lorentz\,生成元$\,J_{d1},J_{23},J_{45},\cdots$, 它们必须对易, 并且根据湮灭费米生成元$\,Q\,$的值$\,\sigma_{d1},\sigma_{23},\cdots\,$对$\,Q\,$做了分类:
\begin{equation}
    [J_{d1},Q]=-\sigma_{d1}Q \:, \qquad [J_{23},Q]= -\sigma_{23}Q \:, \qquad [J_{45},Q] = -\sigma_{45}Q\:,\cdots\:. \label{32.1.1}
\end{equation}
由于$\,O(d)\,$的有限维表示都是幺正的, 这些$\,\sigma\,$都是实的.

我们先集中在其中一个量子数上, $\sigma_{d1}\equiv w$, 对任何费米或玻色算符$\,O$, 如果
\begin{equation}
    [J_{d1},O]=-w\,O \:, \label{32.1.2}
\end{equation}
或者表示成\,Minkowski\,分量$\,J_{01}=\mi J_{d1}$, 
\begin{equation}
    [J_{01},O] = -\mi w\, O \:, \label{32.1.3}
\end{equation}
我们称它的{\kai{权}}为$\,w$. 集中于这个特定的量子数是因为它对一个算符及其厄米共轭有相同的值这个特殊性质. 这是因为, $J_{01}\,$在\,Hilbert\,空间(尽管不是场变量或对称性生成元的空间)上必须被厄米算符表示, 这使得(回忆起$\,w\,$是实的)方程(\ref{32.1.3})的厄米共轭是
\begin{equation}
    {-}[J_{01},O^{\ast}]=+\mi wO^{\ast} \:, \label{32.1.4}
\end{equation}
所以$\,O^{\ast}\,$的权和$\,O\,$相同.

现在考虑任意费米对称性生成元$\,Q\,$与其厄米共轭的反对易子$\,\{Q,Q^{\ast}\}$. 根据\,Coleman--Mandula 定理, 它至多是$\,P_{\mu}$, $J_{\mu\nu}\,$和标量的线性组合. 为了计算$\,P_{\mu}\,$各分量的权, 我们回忆起对易关系(\textcolor{foo}{2.4.13})
\[
\mi[P_{\mu},J_{\rho\sigma}] = \eta_{\eta\rho}P_{\sigma} - \eta_{\mu\sigma}P_{\rho} \:,
\]
这表明$\,P_{0}\pm P_{1}\,$的权$\,w=\pm 1$, 而其他分类$\,P_{2},P_{3},\cdots,P_{d-1}\,$的权均为零. 以同样的方法, $J_{\mu\nu}\,$彼此之间的对易关系(\textcolor{foo}{2.4.12})表明$\,i=2,3,\cdots,d{-}1\,$的$,J_{0i}\pm J_{1i}\,$的权$\,w=\pm 1$, 而$\,i\,$和$\,j\,$均在$\,2\,$和$\,d{-}1\,$之间的$\,J_{ij}\,$的权为零, $J_{10}\,$的权为零, 所有标量的权显然为零. 我们由此得出所有玻色对称性生成元的权是$\,\pm 1\,$或\,0\,以及反对易子$\,\{Q,Q^{\ast}\}\,$必须是权取这些值的算符的线性组合. 如果$\,Q\,$的权为$\,w$, 那么$\,\{Q,Q^{\ast}\}\,$的权就是$\,2w$, 并且对任何非零的$\,Q$, 它显然不为零, 所以每个费米生成元的权只能是 $\pm 1/2$. (权为零被自旋统计关系排除了------费米算符只能用奇数个权为半整数的算符构造.) 回到欧几里得体系, 由于在$\,O(d)\,$的表示下所有生成元$\,Q\,$与特定$\,O(d)\,$生成元$\,J_{01}\,$的对易子由$\,w=\pm 1/2\,$的方程(\ref{32.1.2})给出, 又因为\,01\,平面并没有什么特殊之处, $O(d)\,$对称性要求这对所有$\,O(d)\,$生成元$\,J_{ij}\,$都成立, 这使得方程(\ref{32.1.1})中的所有$\,\sigma\,$都是$\,\pm 1/2$. 在齐次\,Lorentz\,群的不可约表示中, 所有$\,\sigma\,$等于$\,\pm 1/2\,$的不可约表示只能是基础旋量表示, 所以$\,Q\,$必属于这些表示的直和.

我们也可以使用这个方法来证明所有费米生成元$\,Q\,$与$\,d$-动量$\,P_{\mu}\,$都对易. 对于这个目的, 注意到, 如果$\,Q\,$的权是$\,\pm 1/2$, 那么权为$\,\pm 1\,$的动量算符$\,P_{0}\pm P_{1}\,$与$\,Q\,$的双重对易子的权是$\,\pm 5/2$, 如果$\,Q\,$的权是$\,\mp 1/2$, 那么双重对易子的权则是$\,\pm 3/2$, 又因为我们已经知道不出只能在权为\,$\pm 3/2\,$或 $\pm 5/2\,$的费米对称性生成元, 这些双重对易子必须全部为零:
\[
[P_{0} \pm P_{1}, [P_{0},\pm P_{1},Q]] =0 \:.
\]
由此可以得出
\[
[P_{0} \pm P_{1}, [P_{0},\pm P_{1},{Q,Q^{\ast}}]] = -2\{Q_{\pm},Q_{\pm}^{\ast}\} \:.
\]
其中
\[
Q_{\pm} \equiv [P_{0}\pm P_{1},Q] \:.
\]
现在, $\{Q,Q^{\ast}\}\,$至多是$\,J$, $P\,$和标量对称性生成元的线性组合. $P_{0}\pm P_{1}\,$与$\,P\,$和标量对称性生成元的对易子为零, 而$\,P_{0}\pm P_{1}\,$与$\,J\,$的对易子是$\,P\,$的线性组合, 这与其他$\,P_{0}\pm P_{1}\,$对易, 所以双重对易子$\,[P_{0} \pm P_{1}, [P_{0},\pm P_{1},{Q,Q^{\ast}}]]\,$必须为零, 因此$\,\{Q_{\pm},Q_{pm}^{\ast}\}=0$, 这给出$\,Q_{\pm}=0$. 这样$\,Q\,$所给出的\,Lorentz\,群表示的{\kai{所有}}成员与$\,P_{0}\,$和$\,P_{1}\,$对易, Lorentz\,不变性则表明所有$\,Q\,$与所有$\,P\,$对易, 这正是所要证明的.

由于\,Lorentz\,生成元$J_{\mu\nu}\,$与动量算符不对易, 有一个重要的推论: 它们不能出现在反对易关系的右边, 我们暂且将$\,Q\,$记做$\,Q_{n}$, 其中$\,n\,$标记$\,Q\,$之中不同(不一定不等价)的不可约表示, 现在{\kai{引入}}它们的共轭$\,Q^{\ast}$, 以及用来标记表示成员的指标. 这样, 一般反对易关系形如
\begin{equation}
    \{Q_{n},Q_{m}\} = \Gamma_{nm}^{\mu}P_{\mu} +Z_{nm} \:, \label{32.1.5}
\end{equation}
其中$\,\Gamma_{nm}^{\mu}\,$是\,c-数系数, $Z_{nm}\,$是守恒的标量对称性生成元, 它们与$\,P_{\mu}\,$和$\,J_{\mu\nu}\,$对易. 我们现在希望证明$\,Z_{nm}\,$是超对称代数的{\kai{中心荷}}------即, 与$\,Q\,$对易, 与自身对易, 与$\,P_{\mu}$、$J_{\mu\nu}\,$以及所有其他对称性生成元对易.

为了对$\,d\geq 4\,$时证明这点, 注意到, 对于一个给定的非零$\,Z_{nm}$, 由于它是一个标量, 方程(\ref{32.1.1}) 中的所有$\,\sigma\,$对$\,Q_{n}\,$和$\,Q_{m}\,$必须取相反值. 考虑另一个费米子对称性生成元$\,Q_{\ell}$, 使得对这个生成元, 方程(\ref{32.1.2})的$\,\sigma\,$与$\,Q_{n}\,$或$\,Q_{m}\,$的$\,\sigma\,$不全相同. (当$\,d\geq 4$\,时, 在构成$\,O(d)\,$不可约旋量表示的每组$\,Q$ 中总存在这样一个$\,Q$.) 我们使用超\,Jacobi\,恒等式
\begin{equation}
    [Q_{\ell},\{Q_{m},Q_{n}\}] + [Q_{m},\{Q_{n},Q_{\ell}\}] + [Q_{n},\{Q_{\ell},Q_{m}] =0 \:. \label{32.1.6}
\end{equation}
反对易子$\,\{Q_{n},Q_{\ell}\}\,$和$\,\{Q_{\ell},Q_{m}\}\,$是一些$\,\sigma\,$非零的算符, 所以它们只能是$\,P\,$而非$\,Z\,$的线性组合, 因而必须与所有$\,Q\,$对易. 这样就只剩下
\begin{equation}
     0=[Q_{\ell},\{Q_{m},Q_{n}\}] = [Q_{\ell},Z_{mn}] \:. \label{32.1.7}
\end{equation}
因此, 在构成\,$\,O(d)\,$不可约旋量表示的每组$\,Q\,$中, 至少存在一个$\,Q\,$与给定的$\,Z_{nm}\,$对易. 但$\,Z_{nm}\,$是 Lorentz\,标量, 所以它必须与所有$\,Q\,$对易. 由此立刻可以从方程(\ref{32.1.5})得出它们彼此对易.

费米生成元必须要构成由{\kai{所有}}标量玻色对称性生成元构成的代数$\,\mathscr{A}\,$的一个表示(可能是平庸的). 借此, 通过使用与\,\ref{sec:25.2}\,节中精确相同的讨论, 我们可以得出中心荷$\,Z_{nm}\,$构成$\,\mathscr{A}\,$的一个不变阿贝尔子代数. Coleman--Mandula\,定理告诉我们$\,\mathscr{A}\,$必须是半单紧\,Lie\,代数与$\,U(1)\,$生成元的直和, 根据定义, 前者不包含不变的阿贝尔子代数, 所以$\,Z_{nm}\,$必须是$\,U(1)\,$生成元, 这样不仅彼此对易, 它们与所有其他玻色对称性生成元都对易. 

为了获得反对易关系(\ref{32.1.5})更细致的信息, 我们必须进一步明确\,Lorentz\,变换以及费米对称性生成元的实性质. 这些对偶数维和奇数维时空差别很大.


\subsection{奇数维}

本章附录证明了, 当时空维数$\,d\,$是奇数时, Lorentz\,代数的基础旋量表示只有一个, 由方程(\ref{32.A.2})用\,Dirac\,矩阵表示的矩阵$\,\mathscr{J}_{\mu\nu}\,$给出, 所以我们必须将费米生成元记为$\,Q_{\alpha r}$, 其中$\,\alpha\,$是$\,2^{(d-1)/2}$-值的\,Dirac\,指标, 而$\,r=1,2,\cdots,N\,$则用来标记$\,N$-扩充超对称性中的不同旋量. 在这个符号约定下, $Q\,$的\,Lorentz\,变换性质给出
\begin{equation}
    [J_{\mu\nu},Q_{\alpha r}] = -\sum_{\beta}(\mathscr{J}_{\mu\nu})_{\alpha\beta}\,Q_{\beta r} \:, \label{32.1.8}
\end{equation}
使得这些生成元的反对易子有变换规则
\[
[J_{\mu\nu},\{Q_{\alpha r},Q_{\beta s}\}] = -\sum_{\bar{\alpha}}(\mathscr{J}_{\mu\nu})_{\alpha\bar{\alpha}}
\{Q_{\bar{\alpha}r},Q_{\beta s}\} -\sum_{\bar{\beta}}(\mathscr{J}_{\mu\nu})_{\beta\bar{\beta}}
\{Q_{\alpha r},Q_{\bar{\beta} s}\}
\]
回忆动量算符$\,P_{\lambda}\,$的\,Lorentz\,变换规则(\textcolor{foo}{2.4.13}), 我们看到方程(\ref{32.1.5})中的矩阵$\,\Gamma_{rs}^{\lambda}\,$和算符$\,Z_{rs}\,$(其中隐去了\,Dirac\,指标)必须满足条件
\begin{equation}
    \mathscr{J}_{\mu\nu} (\Gamma_{\lambda})_{rs} + (\Gamma_{\lambda})_{rs} \mathscr{J}_{\mu\nu}^{\mathrm{T}}
    =-\mi\,(\Gamma_{\mu})_{rs}\eta_{\nu\lambda} +\mi\,(\Gamma_{\nu})_{rs}\eta_{\mu\lambda} \:, \label{32.1.9}
\end{equation}
\begin{equation}
    \mathscr{J}_{\mu\nu}Z_{rs} +Z_{rs}\mathscr{J}_{\mu\nu}^{\mathrm{T}} = 0\:. \label{32.1.10}
\end{equation}
但方程(\ref{32.A.8})给出$\,\mathscr{J}_{\mu\nu}^{\mathrm{T}}=-\mathscr{C}^{-1}\mathscr{J}_{\mu\nu}\mathscr{C}$, 所以方程(\ref{32.1.9})和(\ref{32.1.10})可以表示成要求$\,(\Gamma_{\mu})_{rs}\mathscr{C}^{-1}$ 和$\,\mathscr{J}_{\mu\nu}\,$的对易关系与$\,\gamma_{\mu}\,$和$\,\mathscr{J}_{\mu\nu}\,$的对易关系相同, 以及$\,Z_{rs}\mathscr{C}^{-1}\,$与$\,\mathscr{J}_{\mu\nu}\,$对易. 对奇数的$\,d$, 这些矩阵在相差一个常数因子的意义下是唯一的, 所以我们可以得出
\begin{equation}
    \Gamma_{\alpha r\:\beta s}^{\lambda}=\mi\,g_{rs}\,(\gamma^{\lambda}\mathscr{C})_{\alpha\beta} \label{32.1.11}
\end{equation}
和
\begin{equation}
    Z_{\alpha r\:\beta s} =\mathscr{C}_{\alpha \beta}\,z_{rs} \:, \label{32.1.12}
\end{equation}
其中在方程(\ref{32.1.11})中插入因子$\,\mi\,$是为了后面的方便. 隐去\,Dirac\,指标后, 反对易关系(\ref{32.1.5})现在写成
\begin{equation}
    \{Q_{r},Q_{s}^{\mathrm{T}}\} = \mi\,g_{rs}\gamma^{\lambda}\,\mathscr{C}\,P_{\lambda}+z_{rs}\mathscr{C} \:. \label{32.1.13}
\end{equation}

$\Gamma_{\alpha r\:\beta s}$和$\,Z_{\alpha r\:\beta s}\,$在$\,\alpha,r\,$和$\,\beta,s\,$的交换下不变, 而方程(\ref{32.A.30})和(\ref{32.A.31})(在$\,d=2n{+}1\,$时)表明: 当$\,d=1\:(\operatorname{mod}8)$\,时, $\gamma^{\lambda}\mathscr{C}\,$和$\,\mathscr{C}\,$都是对称的; 当$\,d=3\:(\operatorname{mod}8)$\,时, $\gamma^{\lambda}\mathscr{C}\,$是对称的, 而$\,\mathscr{C}\,$是反对称的; 当$\,d=5\:(\operatorname{mod}8)$\,时, $\gamma^{\lambda}\mathscr{C}\,$和$\,\mathscr{C}\,$都是反对称的; $\,d=7\:(\operatorname{mod}8)$\,时, $\gamma^{\lambda}\mathscr{C}\,$是反对称的, 而$\,\mathscr{C}\,$是对称的. 由此可以得出: 当$\,d=1\:(\operatorname{mod}8)$\,时, $g_{rs}\,$和$\,z_{rs}\,$是对称的; 当$\,d=3\:(\operatorname{mod}8)$\,时, $g_{rs}\,$是对称的, 而$\,z_{rs}\,$是反对称的; 当$\,d=5\:(\operatorname{mod}8)$\,时, $g_{rs}\,$和$\,z_{rs}\,$都是反对称的; 当$\,d=7\:(\operatorname{mod}8)$\,时, $g_{rs}\,$是反对称的, 而$\,z_{rs}\,$是对称的.

(\ref{32.A.37})给出了矩阵$\,\mathscr{J}_{\mu\nu}\,$的复共轭. 通过取(\ref{32.1.8})的厄米共轭, 我们看到$\,\sum_{\beta}(\mathscr{C}\beta)_{\alpha\beta}Q_{\beta r}^{\ast}$ 的\,Lorentz\,变换性质与$\,Q_{\alpha s}\,$相同, 因此它必须是它们的线性组合
\begin{equation}
    \sum_{\beta}(\mathscr{C}\beta)_{\alpha\beta}Q_{\beta r}^{\ast} = \sum_{s}\mathscr{S}_{rs}Q_{\alpha s} \:.\label{32.1.14}
\end{equation}
取这个方程的厄米伴随并使用$\,d=2n+1\,$的方程(\ref{32.A.28})和(\ref{32.A.29}), 这给出
\begin{equation}
    \mathscr{S}^{\ast}\mathscr{S}= (-1)^{a}\cdot 1\:,\qquad a=(d-1)(d-3)/8 \:. \label{32.1.15}
\end{equation}
当$\,d=1\:(\operatorname{mod}8)\,$以及$\,d=3\:(\operatorname{mod}8)\,$时, Lorentz\,代数的旋量表示是实的, 这样我们就可以选到费米生成元的一组基使得$\,\mathscr{S}=1$. 与之相反, 当$\,d=5\:(\operatorname{mod}8)\,$以及$\,d=7\:(\operatorname{mod}8)\,$时, Lorentz\,代数的旋量表示是赝实的, 那么显然不可能找到$\,\mathscr{S}\propto 1\,$的一组基. 通过取方程(\ref{32.1.15})的行列式, 我们看到在这个情况下$\,\operatorname{Det}(-1)>0$, 所以在$\,d=5\:(\operatorname{mod}8)\,$以及$\,d=7\:(\operatorname{mod}8)\,$时, 费米生成元的个数必然是偶数. 在这个情况下, 我们可以选择一组基使得$\,\mathscr{S}=\Omega$, 其中$\,\Omega\,$是反对称分块实矩阵
\begin{equation}
    \Omega = \begin{pmatrix}
     e & 0 & 0 & \cdots \\
     0 & e & 0 & \cdots \\
     0 & 0 & e & \cdots \\
     \vdots & \vdots & \vdots & \ddots
    \end{pmatrix} \:, \qquad
     e= \begin{pmatrix}
     0 & 1 \\ -1 & 0
     \end{pmatrix} \:. \label{32.1.16}
\end{equation}
通过使用方程(\ref{32.1.14})将反对易关系(\ref{32.1.13})重写成
\[
\{Q_{r},Q_{s}^{\dag}\} = \mi (g\mathscr{S}^{\mathrm{T}})_{rs}\gamma_{\lambda} \mathscr{C}[(\mathscr{C}\beta)^{\mathrm{T}}]^{-1} P^{\lambda} + (z\mathscr{S}^{\mathrm{T}})_{rs}
\mathscr{C}[(\mathscr{C}\beta)^{\mathrm{T}}]^{-1} \:,
\]
我们可以推断出$\,g_{rs}\,$和$\,z_{rs}\,$的实性和正性. 方程(\ref{32.A.12}), (\ref{32.A.16})\,和\,(\ref{32.A.30})在$\,d=2n{+1}$\,时表明$\,\beta^{\mathrm{T}}=-\beta\,$以及$\,\mathscr{C}\mathscr{C}^{\mathrm{T}-1}=(-1)^{(d-1)(d+1)/8}\cdot 1$, 所以
\begin{equation}
    \{Q_{r},Q_{s}^{\dag}\} = -(-1)^{(d-1)(d+1)/8}\,\Bigl[\mi (g\mathscr{S}^{\mathrm{T}})_{rs}\gamma_{\lambda}\beta P^{\lambda} + (z\mathscr{S}^{\mathrm{T}})_{rs} \beta )\Bigr] \:. \label{32.1.17}
\end{equation}
回忆起$\,\gamma_{0}=\mi\beta$, 我们注意到算符矩阵$\,-\mi\gamma_{\lambda}\beta\,P^{\lambda}\,$是正的, 且除了真空态以外是正定的. 
通过考虑一个动量足够大以至于方程(\ref{32.1.17})中的中心荷可以被忽略的态, 我们得出矩阵$\,(-1)^{(d-1)(d+1)/8}g\mathscr{S}^{\mathrm{T}}$ 是正的且是厄米的. 然后考虑任意动量, 我们发现算符$\,(z\mathscr{S}^{\mathrm{T}})_{rs}\,$的阵列也是厄米的. (当中心荷不为零时, 质量上有类似于方程(\ref{25.5.22})的一个下界, 在这里我们不展示这个下界.) 从$\,g\mathscr{S}^{\mathrm{T}}\,$的厄米性, 我们有
\begin{equation}
    g^{\dag} = [\mathscr{S}^{\mathrm{T}\dag}]^{-1} g\mathscr{S}^{\mathrm{T}} =(-1)^{a}\mathscr{S}g\mathscr{S}^{\mathrm{T}} \:. \label{32.1.18}
\end{equation}
我们现在能够通过对基的一个合适选择将反对易关系变成传统的正则形式.

当$\,d=1\:(\operatorname{mod} 8)\,$时, 我们有对称的$\,g\,$和$\,z\,$以及$\,(-1)^{a}=+1$, 所以, 如果我们选择的基中$\,\mathscr{S}=1$, 那么$\,g\,$是实的且各个$\,z_{rs}\,$是厄米算符. 通过给旧的$\,\mathscr{Q}\,$乘上任何实矩阵$\,\mathscr{A}$, 我们可以在不改变$\,\mathscr{S}=1\,$的情况下引入新的$\,Q$, 而$\,g\,$变成了$\,\mathscr{A}g\mathscr{A}^{\mathrm{T}}$. 由于$\,g\,$在$\,d=1\:(\operatorname{mod} 8)\,$时是正矩阵, 通过一个著名定理,\cite{6a}, 我们可以选择$\,\mathscr{A}\,$使得$\,g=1$.

当$\,d=3\:(\operatorname{mod} 8)\,$时, 我们有对称的$\,g\,$和反对称的$\,z\,$以及$\,(-1)^{a}=+1$, 所以, 如果我们选择的基中$\,\mathscr{S}=1$, 那么$\,g\,$是实的且各个$\,z_{rs}\,$是反厄米算符. 和$\,d=1\:(\operatorname{mod} 8)\,$的情况相同, 我们可以调整这个基使得$\,g=1$.

当$\,d=5\:(\operatorname{mod} 8)\,$时, 我们有反对称的$\,g$, 所以, 在$\,\mathscr{S}=\Omega\,$的选择下, 方程(\ref{32.1.18})变成$\,g^{\ast}=-\Omega g\Omega$, 其中$\,\Omega\,$是标准反对称矩阵(\ref{32.1.16}). 这里, 通过给旧的$\,Q\,$乘上任何满足$\,\mathscr{B}^{\ast}=-\Omega\mathscr{B}\Omega\,$的矩阵$\,\mathscr{B}$, 我们可以引入新的$\,Q\,$而保持$\,\mathscr{S}=\Omega$, 这个操作对$\,g\,$的效应是将其变成$\,\mathscr{B}g\mathscr{B}^{\mathrm{T}}$. 由于$\,(-1)^{a}=1$, $g\Omega\,$是正的, 因此以这样的方式我们令$\,g=-\Omega$. 另外, $z\,$是反对称的且$\,z\Omega\,$是厄米的, 所以$\,z^{\ast}=-\Omega\,z\,\Omega$.

当$\,d=7\:(\operatorname{mod} 8)\,$时, 我们依旧有反对称的$\,g$, 但现在$\,(-1)^{a}=+1$, 所以通过使用$\,d=5\:(\operatorname{mod} 8)$ 时的方法, 我们可以选到一组$\,g=+\Omega\,$的基. 另外, $z\,$现在是反对称的而$\,z\Omega\,$依旧是厄米的, 所以现在$\,z^{\ast}=+\Omega\,z\,\Omega$.

\subsection{偶数维}

这章的附录表明, 当时空维数$\,d\,$是偶数时, Lorentz\,代数有两个不定价的基础旋量表示, 由方程(\ref{32.A.22}), (\ref{32.A.2})和(\ref{32.A.17})用\,Dirac\,矩阵表示的矩阵$\,\mathscr{J}_{\mu\nu}^{\pm}\,$给出. 因此我们必须将费米生成元记为$\,Q_{\alpha r}^{\pm},$ 其中$\,\alpha\,$是$\,2^{d/2}\,$值\,Dirac\,指标, $r\,$则是在扩充超对称性的情况下标记属于\,Lorentz\,代数等价表示中不同的$\,Q$, 并有
\begin{equation}
    \sum_{\beta}(\gamma_{d+1})_{\alpha\beta}\,Q_{\beta r}^{\pm} = \pm Q_{\alpha r}^{\pm} \:, \label{32.1.19}
\end{equation}
其中$\,\gamma_{d+1}\equiv \mi^{d/2-1}\gamma_{1}\cdots \gamma_{d-1}\gamma_{0}$. 在这个符号约定下, $Q\,$的\,Lorentz\,变换性质给出
\begin{equation}
    [J_{\mu\nu},Q_{\alpha r}^{\pm}] = -\sum_{\beta}(\mathscr{J_{\mu\nu}}^{\pm})_{\alpha \beta}\,Q_{\beta r}^{\pm} \:,
    \label{32.1.20}
\end{equation}
其中$\,\mathscr{J}_{\mu\nu}^{+}\,$是矩阵(\ref{32.A.22}). 将方程(\ref{32.1.19})和关系$\,\mathscr{C}^{-1}\gamma_{d+1}\mathscr{C}=(-1)^{d/2}\gamma_{d+1}$, 给出奇数$\,d\,$的反对易关系(\ref{32.1.13})的那个讨论现在给出
\begin{align}
    \{Q_{r}^{\pm}, Q_{s}^{\mp(-1)^{d/2}\,\mathrm{T}} \} &= \mi\,g_{rs}^{\pm} \biggl(\frac{1\pm\gamma_{d+1}}{2}\biggr)\,\gamma_{\lambda} \,\mathscr{C}\,P_{\lambda} \:, \label{32.1.21} \\
    \{Q_{r}^{\pm}, Q_{s}^{\pm(-1)^{d/2}\,\mathrm{T}} \} &= z_{rs}^{\pm} \biggl(\frac{1\pm\gamma_{d+1}}{2}\biggr)\,\gamma_{\lambda} \,\mathscr{C} \:, \label{32.1.22}
\end{align}
方程(\ref{32.A.30})和(\ref{32.A.31})表明$\,\mathscr{C}\gamma^{\lambda}\,$在$\,d=0\:(\operatorname{mod}8)$和$\,d=2\:(\operatorname{mod}8)\,$时是对称的, 在$\,d=4\:(\operatorname{mod}8)$ 和$\,d=6\:(\operatorname{mod}8)\,$时是反对称的, 
而$\,\mathscr{C}\,$在$\,d=0\:(\operatorname{mod}8)$和$\,d=6\:(\operatorname{mod}8)\,$时是对称的, 在$\,d=2\:(\operatorname{mod}8)$和$\,d=4\:(\operatorname{mod}8)\,$时是反对称的. 因此方程(\ref{32.1.21})要求有如下的对称性
\begin{equation}
    g_{rs}^{\pm} = \begin{cases}
    g_{sr}^{\mp(-1)^{d/2}}  &\qquad d=0,2\:(\operatorname{mod}8) \\
    -g_{sr}^{\mp(-1)^{d/2}}  &\qquad d=4,6\:(\operatorname{mod}8)
    \end{cases} \:, \label{32.1.23}
\end{equation}
而方程(\ref{32.1.22})则要求
\begin{equation}
    z_{rs}^{\pm} = \begin{cases}
    z_{sr}^{\pm(-1)^{d/2}}  &\qquad d=0,6\:(\operatorname{mod}8) \\
    -z_{sr}^{\pm(-1)^{d/2}}  &\qquad d=2,4\:(\operatorname{mod}8)
    \end{cases} \:. \label{32.1.24}
\end{equation}
特别地, $z^{\pm}\,$在$\,d=0\:(\operatorname{mod}8)$时是对称的, $g^{\pm}\,$在$\,d=2\:(\operatorname{mod}8)\,$是对称的, $z^{\pm}\,$在$\,d=4\:(\operatorname{mod}8)$时是反对称的, $g^{\pm}\,$在$\,d=6\:(\operatorname{mod}8)\,$是对称的.

取方程(\ref{32.1.20})的厄米共轭并使用方程(\ref{32.A.25})表明$\,\mathscr{C}\beta Q_{r}^{\pm\ast}\,$与算符$\,Q_{s}^{\mp(-1)^{d/2}}\,$的\,Lorentz\,变换性质相同, 因此前者是后者的线性组合:
\begin{equation}
    \mathscr{C}\beta\,Q_{r}^{\pm\ast} = \sum_{s}\mathscr{S}_{rs}^{\pm} \,Q_{s}^{\mp(-1)^{d/2}} \:. \label{32.1.25}
\end{equation}
取这个方程的厄米共轭并使用$\,d=2n\,$的方程(\ref{32.A.28})和(\ref{32.A.29}), 这给出
\begin{equation}
    \mathscr{S}^{\pm\ast}\mathscr{S}^{\mp(-)^{d/2}}=(-1)^{a}\cdot 1 \:,\qquad a=d(d-2)/8 \:. \label{32.1.26}
\end{equation}
当$\,d=0\:(\operatorname{mod}8)\,$和$\,d=4\:(\operatorname{mod}8)\,$时, 方程(\ref{32.1.25})将一个不可约表示与另一个不可约表示关联起来, 并且我们可以选择基使得对$\,d=0\:(\operatorname{mod}8)\,$有$\,\mathscr{S}^{\pm}=1$, 对$\,d=4\:(\operatorname{mod}8)\,$有$\,\mathscr{S}^{\pm}=\pm 1$. 当$\,d=2\:(\operatorname{mod}8)\,$时, 方程(\ref{32.1.25})给出的是实表示自身之间的关系, 我们可以选择一组基使得$\,\mathscr{S}^{\pm}=1$. 当$\,d=6\:(\operatorname{mod}8)\,$时, 方程(\ref{32.1.25})给出的是赝实表示自身之间的关系; 方程(\ref{32.1.26})表明$\,Q^{+}\,$和$\,Q^{-}$ 的个数必然是偶数个(二者不必相等!), 我们可以选择$\,\mathscr{S}^{\pm}=\Omega^{\pm}\,$的基, 其中$\,\Omega^{\pm}\,$是形如(\ref{32.1.16})的标准反对称实矩阵.

通过使用方程(\ref{32.1.25})将反对易关系(\ref{32.1.21})和(\ref{32.1.22})写成
\begin{equation}
    \{Q_{r}^{\pm},Q_{s}^{\pm\dag}\} = \biggl(\frac{1\pm\gamma_{d+1}}{2}\biggr)\,\gamma_{\lambda}\beta\,P^{\lambda}
    \mathscr{C}[(\mathscr{C}\beta)^{\mathrm{T}}]^{-1} \Bigl(g^{\pm}\mathscr{S}^{\mp\mathrm{T}}\Bigr)_{rs} \:, \label{32.1.27}
\end{equation}
\begin{equation}
    \{Q_{r}^{\pm},Q_{s}^{\mp\dag}\} = \biggl(\frac{1\pm\gamma_{d+1}}{2}\biggr)\,
    \mathscr{C}[(\mathscr{C}\beta)^{\mathrm{T}}]^{-1} \Bigl(z^{\pm}\mathscr{S}^{\mp\mathrm{T}}\Bigr)_{rs} \:, \label{32.1.28}
\end{equation}
我们可以推断出$\,g_{rs}^{\pm}\,$的实性和正性以及$\,z_{rs}^{\pm}\,$的实性. 再次使用关系$\,\mathscr{C}\mathscr{C}^{\mathrm{T}-1}=(-1)^{d(d+2)/8}$, $\beta^{\mathrm{T}}=-\beta\,$和$\,\gamma_{0}=\mi\beta$, 我们得出$\,(-1)^{d(d+2)/8}g^{\pm}\mathscr{S}^{\pm\mathrm{T}}\,$是厄米的且是实的, 而$\,(z^{+}\mathscr{S}^{-\mathrm{T}})^{\dag}=z^{-}\mathscr{S}^{+\mathrm{T}}$. 当$\,d=0\:(\operatorname{mod}8)\,$时, 我们可以采用$\,\mathscr{S}^{\pm}=1$, $g^{\pm}=1\,$和$\,z^{+\dag}=z^{-}\,$的一组基; 当$\,d=2\:(\operatorname{mod}8)\,$时, 我们可以采用$\,\mathscr{S}^{\pm}=1$, $g^{\pm}=-1\,$和$\,z^{+\dag}=z^{-}\,$的一组基; 当$\,d=4\:(\operatorname{mod}8)\,$时, 我们可以采用$\,\mathscr{S}^{\pm}=\pm 1$, $g^{\pm}=\mp 1\,$和$\,z^{+\dag}=-z^{-}\,$的一组基; 当$\,d=6\:(\operatorname{mod}8)\,$时, 我们可以采用$\,\mathscr{S}^{\pm}=\Omega^{\pm}$, $g^{\pm}=\Omega^{\pm}\,$和$\,(z^{+}\Omega^{-})^{\dag}=z^{-}\Omega^{+}\,$的一组基.

总结一下, 在合适的基下, 反对易关系, 实性以及对称性条件如下:\cite{6b} 

\noindent $d=0\:(\operatorname{mod}8)$
\begin{equation}
    \{Q_{r}^{\pm},Q_{s}^{\mp\mathrm{T}}\} = \mi\,\delta_{rs} \biggl(\frac{1\pm\gamma_{d+1}}{2}\biggr)\gamma^{\lambda}\,
    \mathscr{C}\,P_{\lambda}\:, \label{32.1.29}
\end{equation}
\begin{equation}
    \{Q_{r}^{\pm},Q_{s}^{\pm\mathrm{T}}\} = z_{rs}^{\pm} \biggl(\frac{1\pm\gamma_{d+1}}{2}\biggr)\,
    \mathscr{C}\:, \label{32.1.30}
\end{equation}
\begin{equation}
    \mathscr{C}\beta\,Q_{r}^{\pm\ast} = Q_{r}^{\mp} \:, \qquad z_{rs}^{\pm}=z_{sr}^{\pm}=(z_{rs}^{\mp})^{\ast} \:. \label{32.1.31}
\end{equation}
$d=1\:(\operatorname{mod}8)$
\begin{equation}
     \{Q_{r},Q_{s}^{\mathrm{T}}\} = \mi\,\delta_{rs}\gamma^{\lambda}\, \mathscr{C}\,P_{\lambda}+z_{rs}\mathscr{C}\:, \label{32.1.32}
\end{equation}
\begin{equation}
    \mathscr{C}\beta Q_{r}^{\ast} = Q_{r} \:, \qquad z_{rs}=z_{sr}=z_{rs}^{\ast} \:. \label{32.1.33}
\end{equation}
$d=2\:(\operatorname{mod}8)$
\begin{equation}
    \{Q_{r}^{\pm},Q_{s}^{\pm\mathrm{T}}\} = -\mi\,\delta_{rs} \biggl(\frac{1\pm\gamma_{d+1}}{2}\biggr)\gamma^{\lambda}\,
    \mathscr{C}\,P_{\lambda}\:, \label{32.1.34}
\end{equation}
\begin{equation}
    \{Q_{r}^{\pm},Q_{s}^{\mp\mathrm{T}}\} = z_{rs}^{\pm} \biggl(\frac{1\pm\gamma_{d+1}}{2}\biggr)\,
    \mathscr{C}\:, \label{32.1.35}
\end{equation}
\begin{equation}
    \mathscr{C}\beta\,Q_{r}^{\pm\ast} = Q_{r}^{\pm} \:, \qquad z_{rs}^{\pm}=-z_{sr}^{\pm\ast}=-z_{rs}^{\mp} \:. \label{32.1.36}
\end{equation}
$d=3\:(\operatorname{mod}8)$
\begin{equation}
     \{Q_{r},Q_{s}^{\mathrm{T}}\} = \mi\,\delta_{rs}\gamma^{\lambda}\, \mathscr{C}\,P_{\lambda}+z_{rs}\mathscr{C}\:, \label{32.1.37}
\end{equation}
\begin{equation}
    \mathscr{C}\beta Q_{r}^{\ast} = Q_{r} \:, \qquad z_{rs}=-z_{sr}=-z_{rs}^{\ast} \:. \label{32.1.38}
\end{equation}
$d=4\:(\operatorname{mod}8)$
\begin{equation}
    \{Q_{r}^{\pm},Q_{s}^{\mp\mathrm{T}}\} = \mp\mi\,\delta_{rs} \biggl(\frac{1\pm\gamma_{d+1}}{2}\biggr)\gamma^{\lambda}\,
    \mathscr{C}\,P_{\lambda}\:, \label{32.1.39}
\end{equation}
\begin{equation}
    \{Q_{r}^{\pm},Q_{s}^{\pm\mathrm{T}}\} = z_{rs}^{\pm} \biggl(\frac{1\pm\gamma_{d+1}}{2}\biggr)\,
    \mathscr{C}\:, \label{32.1.40}
\end{equation}
\begin{equation}
    \mathscr{C}\beta\,Q_{r}^{\pm\ast} =\pm Q_{r}^{\mp} \:, \qquad z_{rs}^{\pm}=z_{rs}^{\mp\ast}=-z_{sr}^{\pm} \:. \label{32.1.41}
\end{equation}
$d=5\:(\operatorname{mod}8)$
\begin{equation}
     \{Q_{r},Q_{s}^{\mathrm{T}}\} = -\mi\,\Omega_{rs}\gamma^{\lambda}\, \mathscr{C}\,P_{\lambda}+z_{rs}\mathscr{C}\:, \label{32.1.42}
\end{equation}
\begin{equation}
    \mathscr{C}\beta Q_{r}^{\ast} = \sum_{s}\Omega_{rs}Q_{s} \:, \qquad z_{rs}=-z_{sr}\:,\qquad
    z^{\ast} = -\Omega z\Omega  \label{32.1.43}
\end{equation}
$d=6\:(\operatorname{mod}8)$
\begin{equation}
    \{Q_{r}^{\pm},Q_{s}^{\pm\mathrm{T}}\} = \mi\,\Omega_{rs}^{\pm} \biggl(\frac{1\pm\gamma_{d+1}}{2}\biggr)\gamma^{\lambda}\,
    \mathscr{C}\,P_{\lambda}\:, \label{32.1.44}
\end{equation}
\begin{equation}
    \{Q_{r}^{\pm},Q_{s}^{\mp\mathrm{T}}\} = z_{rs}^{\pm} \biggl(\frac{1\pm\gamma_{d+1}}{2}\biggr)\,
    \mathscr{C}\:, \label{32.1.45}
\end{equation}
\begin{equation}
    \mathscr{C}\beta\,Q_{r}^{\pm\ast} =\sum_{s}\Omega_{rs}^{\pm} Q_{s}^{\pm} \:, \qquad z_{rs}^{\pm\ast}=\Omega^{\pm}z_{rs}^{\pm}\Omega^{\mp}=z_{sr}^{\mp} \:. \label{32.1.46}
\end{equation}
$d=7\:(\operatorname{mod}8)$
\begin{equation}
     \{Q_{r},Q_{s}^{\mathrm{T}}\} = \mi\,\Omega_{rs}\gamma^{\lambda}\, \mathscr{C}\,P_{\lambda}+z_{rs}\mathscr{C}\:, \label{32.1.47}
\end{equation}
\begin{equation}
    \mathscr{C}\beta Q_{r}^{\ast} = \sum_{s}\Omega_{rs}Q_{s} \:, \qquad z_{rs}=z_{sr}\:,\qquad
    z^{\ast} = +\Omega z\Omega  \label{32.1.48}
\end{equation}

对这些反对易关系的观察指出, 当没有中心荷时, 它们在费米生成元的线性变换下不变, 对于奇数的$\,d$, 线性变换形如$\,Q_{r}\to\sum_{s}V_{rs}Q_{s}$, 对于偶数的$\,d\,$则是$\,Q_{r}^{\pm}\to\sum_{s}V_{rs}^{\pm}Q_{s}^{\pm}$. 为了保留关系(\ref{32.1.29})---(\ref{32.1.48}), $V\,$必须要满足条件:

\noindent $d=0\,$且$\,d=4\:(\operatorname{mod}8)$
\begin{equation}
    V^{\pm}V^{\mp\mathrm{T}} = 1 \:, \qquad V^{\pm\ast} = V^{\mp} \:. \label{32.1.49}
\end{equation}
$d=1\,$且$\,d=3\:(\operatorname{mod}8)$
\begin{equation}
    VV^{\mathrm{T}}=1 \:, \qquad V^{\ast} = V \:. \label{32.1.50}
\end{equation}
$d=2\:(\operatorname{mod}8)$
\begin{equation}
    V^{\pm}V^{\pm\mathrm{T}} = 1 \:, \qquad V^{\pm} = V^{\pm\ast} \:. \label{32.1.51}
\end{equation}
$d=5\:(\operatorname{mod}8)$
\begin{equation}
    V\Omega V^{\mathrm{T}}=\Omega \:, \qquad V^{\ast} =-\Omega V \Omega \:. \label{32.1.52}
\end{equation}
$d=6\:(\operatorname{mod}8)$
\begin{equation}
    V^{\pm}V^{\pm\mathrm{T}} = 1 \:, \qquad V^{\pm\ast} =-\Omega^{\pm} V^{\pm}\Omega^{\pm} \:. \label{32.1.53}
\end{equation}
$d=7\:(\operatorname{mod}8)$
\begin{equation}
    V\Omega V^{\mathrm{T}}=\Omega \:, \qquad V^{\ast} =-\Omega V \Omega \:. \label{32.1.54}
\end{equation}
这些矩阵构成群:\\
$d=0\,\text{且}\,d=4\:(\operatorname{mod}8)\qquad \qquad \qquad U(N)\:.$ \\
$d=1\,\text{且}\,d=3\:(\operatorname{mod}8)\qquad \qquad \qquad O(N)\:.$ \\
$d=2\:(\operatorname{mod}8)\qquad \qquad \qquad O(N_{+})\times O(N_{-})\:.$ \\
$d=5\,\text{且}\,d=7\:(\operatorname{mod}8)\qquad \qquad \qquad USp(N)\quad {N\,\text{为偶}}\:.$ \\
$d=6\:(\operatorname{mod}8)\qquad \qquad \qquad USp(N_{+})\times USp(N_{-})\quad {N_{\pm}\,\text{为偶}}\:.$ \\
这里的$\,N\,$对于奇数的$\,d\,$是$\,Q\,$之中基本旋量表示的个数, 对于$\,d=0\:(\operatorname{mod}8)\,$和$\,d=4\:(\operatorname{mod}8)\,$则是$\,Q\,$之中每个手性的基础旋量表示的个数. 当$\,d=2\:(\operatorname{mod}8)\,$以及$\,d=6\:(\operatorname{mod}8)\,$时, $Q\,$之中每个手性的基础旋量表示个数无需相等, 所以这里记做$\,N_{+}\,$和$\,N_{-}$

\section{无质量多重态} \label{sec:32.2}

我们现在将考虑如何用上节构造的超对称代数在维数$\,d\geq 4\,$的时空中构造无质量粒子态的超多重态. 动量算符$\,P^{\mu}\,$与所有费米对称性生成元相对易, 所以我们可以在\,Hilbert\,空间中的单粒子子空间中进行处理, 在这个子空间中$\,P^{\mu}\,$有确定的类光本征矢$\,p^{\mu}$, 可以将其取在$\,p^{1}=p^{0}\,$的方向上, 而$\,p^{\mu}\,$的其它所有空间分量均为零. 就像在\,2.5\,节中对四维时空的讨论, 可以根据这些所提供的有限维小群表示对其分类, 其中小群是齐次\,Lorentz\,群中保持$\,p^{\mu}\,$不变的子群. 小群包含沿着$\,\mathbf{p}\,$垂直方向上的增速(boost)和在$\,\mathbf{p}\,$所处平面的旋转以及二者的组合, 例如四维时空中的变换(\textcolor{foo}{2.5.6}), 但它们构成一个不变的阿贝尔子群, 因此在有限维表示中被表示成单位算符. 忽略这样的子群, $d\,$维中的约化子群是$\,O(d{-}2)$, 由与$\,\mathbf{p}\,$垂直的平面中的旋转构成. 因此我们将根据粒子提供的$\,O(d{-2})\,$表示以及上一节末尾所描述的自同构群对它们进行分类.

这些表示要比我们在四维时空中处理过的那些复杂得多, 在四维中, 约化小群是$\,O(2)$, 而表示是一维的, 由螺旋度这个数字表征. 然而, 用``自旋''标记约化小群$\,O(d{-}2)\,$的表示将是有用的, 它被定义成任何生成元$\,J_{ij}\,$在这个表示中的本征值能取到的{\kai{最大}}绝对值.

大家普遍详细自洽的量子场论中不会包含自旋大于\,2\,的无质量粒子. 已知\cite{7}自旋$\,j>1/2\,$的{\kai{软}}无质量粒子智只能与携带自旋$\,j\,$的守恒流相互作用. 当$\,j=1\,$时, 它们是普通守恒标量的流, 例如电荷; 当$\,j=3/2\,$时, 它们是一个或数个与超对称性相对应的流; 当$\,j=2\,$时, 这个流只能是能动量张量; 但当$\,j>5/2\,$时, 不存在可以与软无质量粒子相互作用的流. 通过采用无质量粒子的类型只有一种, 或者任何无质量粒子自旋不能大于\,2\,这个限制, 我们可以对超对称能够发生的维数有一个很强的限制.

我们暂且回到\,\ref{sec:32.1}\,节中所用到的算符分类, 即根据$\,O(d)\,$生成元$\,J_{d1}\,$湮灭的值定义的权. (回忆起$\,J_{01}=\mi J_{d1}$.) 费米超对称性生成元的权是$\,1/2\,$或$\,-1/2$, 所以这些生成元与其厄米共轭的反对易子的权分别只能是$\,+1\,$或$\,-1$, 因此必须分别正比于算符$\,P^{0}+P^{1}\,$或$\,P^{0}-P^{1}$. 但在我们进行处理的\,Hilbert\,子空间中算符$\,P^{0}-P^{1}\,$为零, 所以在这个子空间中, 所有权为$\,-1/2\,$的费米超对称性生成元为零. 因此在对单粒子态分类时, 可用的超对称性生成元只有一半, 即$\,2^{n-1}\,$个权$\,\sigma_{d1}=+1/2\,$的生成元.

我们可以进一步将剩下的生成元分成两类, 即除了$\,\sigma_{d1}=+1/2$, $\sigma_{23}=1/2\,$或$\,\sigma_{23}=-1/2$. 由于算符$\,P^{0}+P^{1}\,$有$\,\sigma_{23}=0$, 每类费米超对称性生成元彼此对易, 但每类生成元不一定与它们的共轭或者另一类生成元对易.

现在考虑小群$\,O(d{-}2)\,$的一个自旋$\,j\,$表示, 并考虑$\,J_{23}\,$的本征值$\,\lambda>0\,$且被所有$\,\sigma_{23}=-1/2\,$的超对称性生成元湮灭的本征态$\,\lvert\lambda\rangle$. (任何对$\,J_{23}\,$的本征值取最大值$\,j\,$的态都将是这个类似的, 但一般会有其它这样的态.) 我们可以用$\,k\,$个$\,\sigma_{23}=+1/2\,$且$\,\sigma_{d1}=+1/2\,$的费米生成元作用在$\,\lvert\lambda\rangle\,$上, 这样就会构成$\,J_{23}=\lambda-k/2\,$的态. (因为用$\,k\,$个这种费米生成元的共轭作用这个态会回到$\,\lvert\lambda\rangle$, 这就证明了这些态中没有一个为零.) 如果所有类型的费米超对称生成元共有$\,\mathscr{N}\,$个, 那么$\,\sigma_{23}=+1/2\,$且$\,\sigma_{d1}=+1/2\,$的只有$\,\mathscr{N}/4\,$个, 又因为这些算符都对易, 以这种方式构成的$\,J_{23}=\lambda-k/2\,$的态, 其总数将是二项式系数
\begin{equation}
\binom{\mathscr{N}/4}{k} \:, \label{32.2.1}
\end{equation}
当从$\,k=0\,$求和到最大值$\,k=\mathscr{N}/4\,$, 这给出$\,2^{\mathscr{N}/4}\,$个分量. 以这种方法能够得到的$\,J_{23}\,$的最小本征值是$\,\lambda-\mathscr{N}/8$, 这样的态是通过用$\,k=\mathscr{N}/4\,$个超对称性生成元作用在$\,\lvert\lambda\rangle\,$上得到的. 取$\,\lambda=j$, 我们看到为了避免$\,J_{23}\,$的本征值大于$\,+2\,$或小于$\,-2$, 我们必须有$\,j\leq 2\,$和$\,j-\mathscr{N}/8\geq -2$, 这要求费米生成元的总数$\,\mathscr{N}\,$不超过\,32.

更进一步, 对$\,\mathscr{N}=32\,$个超对称性生成元, 通过用$\,\sigma_{23}=+1/2\,$和$\,\sigma_{d1}=+1/2\,$的超对称性生成元的乘积作用态$\,\lvert 2\rangle\,$至多只能构成一个超多重态. 对小群中任何生成元的本征值, 这些态以步长$\,1/2\,$在$\,-2\,$到$\,+2\,$之间取值. 仅在$\,\mathscr{N}<32\,$时, 才能有``物质''超多重态, 即不包含引力子的超多重态.

$2n\,$或$\,2n+1\,$中的单个基础旋量表示有$\,2^{n}\,$个分量, 所以为了使费米生成元的个数不超过$\,32$, 我们必须有$\,n\leq 5$. 因此时空维数不能超过$\,d=11$, 且在这个情况下必有$\,N=1$. 因为\,11\,维中的超引力理论可能是所谓$\,M\,${\kai{理论}}这样一个基础理论的``低能''极限,\cite{4} 而$\,M\,$理论又被认为在其它极限下会给出各种弦理论, 所以它是我们特别感兴趣的一个理论. 我们现在细致地解出\,$d=11\,$维中$\,N=1\,$超对称性的自旋分量, 这将作为如何用启发式计数实现这点的一个例子.

在$\,d=11\,$时, 通过给$\,J_{23}\,$的本征值为$\,2\,$的本征态$\,\lvert 2\rangle\,$作用$\,k=0,1,\cdots,8\,$个$\,\sigma_{23}=+1/2\,$且 $\sigma_{2n-1\,2n}=+1/2\,$的超对称性生成元, 我们可以构造出无质量多重态中的所有态. 根据方程(\ref{32.2.1}), 我们会得到$\,J_{23}=\pm 2\,$的态各一个, $J_{23}=\pm 3/2\,$的态各\,8\,个, $J_{23}=\pm 1\,$的态各\,28\,个, $J_{23}=\pm 1/2\,$的态各\,56\,个, 以及\,70\,个$\,J_{23}=0\,$的态.

在$\,d=11\,$时, 小群$\,O(9)\,$的自旋\,2\,引力表示是有$\,9\times 10/2-1=44\,$个独立分量的对称无迹张量: 有\,1\,个$\,J_{23}=\pm 2\,$的$\,2{\pm} \mi 3,2{\pm} \mi 3\,$分量; 7\,个$\,J_{23}=\pm 1\,$的$\,2{\pm} \mi 3, k\,$分量; 28\,个$\,J_{23}=0\,$的$\,k,\ell\,$分量. (这里的$\,k\,$和$\,\ell\,$在$\,4,5,\cdots,10\,$这\,7\,个值中取值. 因为在这个表示下$\,2{+}\mi 3,2{-}\mi 3$\,这个分量会通过无迹条件与$\,k,\ell\,$分量相关, 所以我们在这里不计入这个分量.) 

还有一个自旋$\,3/2\,$的引力微子表示. 它由带有一个额外\,9-矢指标$\,i\,$的旋量$\,\psi_{i}\,$构成, 服从将自旋\,1/2\,分量排除在外的不可约条件$\,\sum_{i}\gamma_{i}\psi_{i}=0$, 因此有$\,9\times 16-16=128\,$个独立分量.

通过从用超对称性生成元作用在$\,\lvert 2\rangle\,$上给出的态中减去引力子和引力微子中$\,J_{23}\,$取相应值的分量, 我们看到我们还需要一个或多个态, 使得$\,J_{23}=\pm 1\,$的分量总数是$\,28-7=21\,$而$\,J_{23}=0\,$的分量总数是$\,70-28=42$. 在正交群的表示中, 只有反对称张量对$\,J_{ij}\,$没有$\,\pm 1\,$和$\,0\,$以外的本征值. 在\,9\,维, 一个秩为$\,p\,$的反对称张量$\,T_{i_{1}\cdots i_{p}}\,$有
\begin{align*}
& \binom{7}{p} \,{\text{个}}\,J_{23}=0\,{\text{的分量}}\,T_{k_{1}\cdots k_{p}}\:, \\
& \binom{7}{p-1} \,{\text{个}}\,J_{23}=\pm 1\,{\text{的分量}}\,T_{2{\pm}\mi 3\,k_{2}\cdots k_{p}} \:,\\
& \binom{7}{p-2}\,{\text{个}}\,J_{23}=0\,{\text{的分量}}\,T_{2{+}\mi 3\,2{-}\mi 3\,k_{3}\cdots k_{p}} \:,\\
\end{align*}
其中$\,k_{1},\cdots ,k_{p}\,$在$\,4,5,\cdots,10\,$这\,7\,个值中取值. 对于$\,O(9)$, 独立的反对称张量只有秩$\,p=0,1,2,3$ 和$\,4\,$的那些. 秩为\,4\,的反对称张量有\,35\,个$\,J_{23}=\pm 1\,$的分量, 这超出需要的个数, 所以它必须被排除. 对于$\,p=0$, $p=1\,$和$\,p=2\,$的张量的任何组合, 如果它们给出所需的\,21\,个$\,J_{23}=\pm 1\,$的分量, 那么它们同时将给出过多$\,J_{23}=0\,$的分量. (21\,个\,1-形式和\,0\,个\,2-形式将给出\,147\,个$\,J_{23}=0\,$的分量, 14\,个\,1-形式和\,1\,个\,2-形式将给出\,120\,个, 7\,个\,1-形式和\,2\,个\,2-形式将给出\,53\,个, 0\,个\,1-形式和\,3\,个\,2-形式将给出\,66\,个), 所以我们至少要引入一个\,3-形式. 秩$\,p=3\,$的反对称张量恰好有\,21\,$\,J_{23}=\pm 1\,$的分量以及\,42\,个$\,J_{23}=\pm 0\,$的分量, 这正是所需要的. 我们由此得出: {\kai{在$\,d=11\,$时, 唯一的$\,N=1\,$无质量超多重态包含一个引力子, 一个引力微子, 以及一个在小群下按照秩为\,3\,的反对称张量变换的粒子.}}

$d=10\,$的可能性则要丰富的多. 这时有两种方法可以给出$\,\mathscr{N}=32\,$个生成元: 费米生成元可以两个手征性相同的\,16-分量\,Weyl\,旋量构成, 这时有自同构群$\,O(2)$, 或者两个手征相反的\,Weyl\,旋量构成, 这时没有自同构群. 在$\,d=10\,$时, 还有可能只有一个拥有\,$\mathscr{N}=16\,$个分量的\,Weyl\,费米生成元. 这三种可能性在现代超弦理论中扮演了一个重要角色------它们代表了三类超弦的无质量频谱: 两个手征性各\,16\,个生成元的\,IIA\,型; 手征性相同共\,32\,个生成元的\,IIB\,型; 以及只有一种手征性共\,16\,个生成元的杂化超弦.

$d=10\,$的\,IIA\,型及其相反的手征性就像$\,d=11\,$的情况, 除了\,$d=11\,$时的小群$\,O(9)\,$的不可约表示要破缺成$\,d=10\,$时的小群$\,O(8)\,$的各个不可约表示. 因此$\,O(9)\,$引力子超多重态要分解成一个有$\,8\times 9/2-1=35\,$个分量的$\,O(8)\,$引力子, 一个有\,8\,个分量的$\,O(8)\,$矢量, 以及一个只有一个分量的标量; $O(9)\,$引力微子多重态要分成两个手征性各\,$(16\times 8-16)/2=56$\,个分量的$\,O(8)\,$引力微子和两个手征性各\,$8$\,个分量的$\,O(8)\,$旋量; $O(9)$ 3-形式要分解成一个有\,56\,个分量的$\,O(8)$ 3-形式和一个有\,28\,个分量的$\,O(8)$ 2-形式.

在\,$d=10\,$的\,IIB\,型中, $N_{+}=2\,$且$\,N_{-}=0$, 我们必须根据小群$\,O(8)\,$的表示以及标记自同构群$\,O(2)\,$表示的量子数$\,q\,$来对态分类, 超对称性生成元在这个自同构群下按照一个\,2-矢量变换. 由于引力子只有一个, 它必有$\,q=0$. 用一个超对称性生成元作用这些态给出{\kai{两}}个$\,q=\pm 1\,$的引力微子, 各有\,56\,个分量; 再用另一个超对称性生成元作用就给出了\,2\,个$\,q=\pm 2\,$的\,2-形式, 各\,28\,个分量; 再用另一个超对称性生成元作用就给出了\,2\,个$\,q=\pm 3\,$的\,Weyl\,旋量, 各\,8\,个分量; 再用另一个超对称性生成元作用就给出了\,2\,个$\,q=\pm 4\,$的标量, 以及$\,q=0\,$且有\,35\,个分量的自对偶\,4-形式.

在$\,d=10\,$且只有一个\,Weyl\,费米生成元的杂化情况下, 独立分量的个数$\,\mathscr{N}=16$. 在这个情况下, 有一个引力子超多重态, 它有一个在$\,O(8)\,$在按照对称无迹张量变换的引力子, 其有\,35\,个独立分量; 一个有\,56\,个独立分量的引力微子; 一个有\,28\,个独立分量的$\,O(8)$ 2-形式; 一个有\,8\,个独立分量的\,Weyl\,旋量; 以及一个标量. (引力子超多重态是通过用超对称性生成元作用在一个$\,\lvert 2\rangle\,$态, {\kai{六}}个$\,\lvert 1\rangle\,$态以及一个$\,\lvert  0\rangle\,$态上构造出来的, 这总共给出$\,8\times 2^{4}=128=35+56+28+8+1\,$个分量.) 这里我们还可能有规范超多重态, 即多重态中的粒子对任何$\,J_{ij}\,$的本征值都处在$\,-1\,$和$\,1\,$之间. 这些超多重态是通过用超对称性生成元作用在$\,\lvert 1\rangle\,$态上形成的, 它包含一个属于$\,O(8)\,$矢量表示的规范粒子, 有\,8\,个分量, 以及一个按照$\,O(8)\,$的基础\,Weyl\,旋量那样变换的粒子, 也有\,8\,个分量.

\section{$p\,$-膜} \label{sec:32.3}


在一些理论中, 除了粒子还存在稳定的扩张物体, 要么无限扩展, 要么通过``缠绕''在拓扑不平庸的时空上稳定下来. 在高维中对这类超对称和超引力的研究打开了如下显著的可能性: 在低维时空中构造弦论和超对称场论以及证明这些理论的等价性,\cite{4,8} 这超出了本书的范畴. 对于这些扩张物体, 我们在这里关心的特征是它们可以携带\,Coleman--Mandula\,定理不允许的玻色守恒量. 这些新的守恒量可能伴随通常的动量算符和普通守恒量出现在超对称反对易关系的右边.\cite{9}

在迄今为止研究的情况中, 新的玻色守恒量都是{\kai{形式}}------反对称张量. 例如, 在$\,d\,$维时空, 一个空间维度为$\,p\,$的物体(称为$\,p$-膜)就可以通过将$\,d\,$个时空坐标$\,x^{\mu}(\sigma,t)\,$(一般处在覆盖这个物体的重叠补片(patches))指定为时刻$\,t\,$以及用来描述这个物体各个位置的$\,p\,$个坐标$\,\sigma^{r}\,$的函数. 如果在给定时刻的流形$\,x^{\mu}=x^{\mu}(\sigma,t)\,$是拓扑不平庸的, 也就是说它不能连续形变到一个点, 那么它会有不为零的拓扑不变积分\footnote{为了看到这个积分是拓扑不变的, 注意到函数$\,x^{\mu}(\sigma,t)\,$的一个无限小变化$\,\delta x^{\mu}(\sigma,t)\,$会导致$\,I^{\mu_{1}\mu_{2}\cdots \mu_{p}}\,$产生变化
\begin{align*}
    \delta I^{\mu_{1}\mu_{2}\cdots \mu_{p}}&=\sum_{n=1}^{p}\sum_{r_{1}=1}^{p}\sum_{r_{2}=1}^{p}\cdots\sum_{r_{p}=1}^{p}
    \int \dif \sigma_{1}\,\dif\sigma_{2}\cdots \dif\sigma_{p}\:\frac{\partial}{\partial\sigma^{r_{n}}}\Biggl[
    \epsilon^{r_{1}r_{2}\cdots r_{p}}  \\    
    &\quad \times \frac{\partial x^{\mu_{1}}}{\partial\sigma^{r_{1}}} \frac{\partial x^{\mu_{2}}}{\partial\sigma^{r_{2}}}
    \cdots \frac{\partial x^{\mu_{n-1}}}{\partial\sigma^{r_{n-1}}}\delta x^{\mu_{n}} 
    \frac{\partial x^{\mu_{n+1}}}{\partial\sigma^{r_{n+1}}}\cdots \frac{\partial x^{\mu_{p}}}{\partial\sigma^{r_{p}}}
    \Biggr] \:,
\end{align*}
当这个积分取在紧流形上时, 它为零. 假定$\,\delta x^{\mu}(\sigma,t)\,$被限制成在$\,\sigma^{r}\to \infty\,$时迅速归零, 那么如果对所有$\,\sigma\,$积分, 这个积分也为零.
}
\begin{align}
    I^{\mu_{1}\mu_{2}\cdots\mu_{p}} &= \int \dif\sigma^{1}\,\dif\sigma^{2}\cdots \dif\sigma^{p}\:
    \sum_{r_{1}=1}^{p} \sum_{r_{2}=1}^{p} \cdots \sum_{r_{p}=1}^{p} \epsilon^{r_{1}r_{2}\cdots r_{p}} \nonumber \\
    &\quad \times \frac{\partial x^{\mu_{1}}(\sigma,t)}{\partial\sigma^{r_{1}}}
    \frac{\partial x^{\mu_{2}}(\sigma,t)}{\partial\sigma^{r_{2}}} \cdots
    \frac{\partial x^{\mu_{p}}(\sigma,t)}{\partial\sigma^{r_{p}}} \:. \label{32.3.1}
\end{align}
特别地, 这种积分在函数$\,x^{\mu}(\sigma,t)\,$的微小变化下不变表明它们在时空变化下不变, 因此可能伴随$\,P^{\mu}$ 以及中心荷出现在超对称性生成元反对易关系的右边.\cite{10} 计算这种张量在超对称性生成元反对易关系右边的系数类似于我们在\,\ref{sec:27.9}\,节讨论过的在四维$\,N=2\,$超对称理论中对标量中心荷$\,Z_{rs}\,$的 Olive-Witten\,计算. 在本节, 我们并不打算计算这些系数或是调查其它可能出现在反对易关系中的非拓扑$\,p$-形式, 而只是考虑引入这些与动量算符对易的守恒反对称张量对超对称代数的影响.


十分重要的一点是: 这个可能性并不影响超对称代数总属于\,Lorentz\,群的基础旋量表示这个关键结果. 这是因为, 欧几里得坐标中的一个全反对称张量至多有一个等于\,1\,的时空指标以及一个等于$\,d\,$的时空指标, 因此方程(\ref{32.1.2})定义的权对它只能是$\,\pm 1\,$或零. 同之前一样, 这意味着超对称性生成元的权只能是$\,\pm 1/2$; 而这只有当超对称性生成元属于$\,O(d{-}1,1)\,$的基础旋量表示时才是可能的. 另外, 因为超对称性生成元反对易子中的新项与动量对易, \ref{sec:32.1}\,节中的讨论再次表明超对称性生成元与动量对易.

Lorentz\,不变性告诉我们对$\,p$-形式``荷''的非零值, 反对易关系(\ref{32.1.13})和(\ref{32.1.21})---(\ref{32.1.22})只能取如下的形式(与\,\ref{sec:32.1}\,节所用的符号约定相同): \\
$d\,${\hei{为奇}}
\begin{equation}
    \{Q_{r},Q_{s}^{\mathrm{T}}\} = g_{rs}\gamma^{\lambda}\mathscr{C}P_{\lambda}
    +\sum_{p}z_{rs}^{\mu_{1}\mu_{2}\cdots\mu_{p}}\,\gamma_{\mu_{1}}\gamma_{\mu_{2}}\cdots \gamma_{\mu_{p}}\,\mathscr{C}\:. \label{32.3.2}
\end{equation}
$d\,${\hei{为偶}}
\begin{align}
    &\{Q_{r}^{\pm}, Q_{s}^{\mp(-1)^{d/2}\,\mathrm{T}}\} = \biggl(\frac{1\pm\gamma_{d+1}}{2}\biggr) \nonumber \\
    &\quad \times \Biggl[g_{rs}^{\pm}\gamma^{\lambda}\mathscr{C}P_{\lambda}+\sum_{\text{odd}\,p}
    z_{rs}^{\mu_{1}\mu_{2}\cdots\mu_{p}\,\pm}\,\gamma_{\mu_{1}}\gamma_{\mu_{2}}\cdots \gamma_{\mu_{p}}\,\mathscr{C}
    \Biggr]\:, \label{32.3.3} \\
    &\{Q_{r}^{\pm}, Q_{s}^{\pm(-1)^{d/2}\,\mathrm{T}}\} = \biggl(\frac{1\pm\gamma_{d+1}}{2}\biggr) \nonumber \\
    &\quad \times \sum_{\text{even}\,p}
    z_{rs}^{\mu_{1}\mu_{2}\cdots\mu_{p}\,\pm}\,\gamma_{\mu_{1}}\gamma_{\mu_{2}}\cdots \gamma_{\mu_{p}}\,\mathscr{C}\:. \label{32.3.4}
\end{align}
(回忆, $\mathscr{C}\,$因为$\,\mathscr{J}_{\mu\nu}^{\mathrm{T}}=-\mathscr{C}^{-1}\mathscr{J}_{\mu\nu}\mathscr{C}\,$出现在反对易关系中; 在$\,d\,$为偶时, $Q_{r}^{\pm}\,$是满足$\,\gamma_{d+1}Q_{r}^{\pm}=\pm Q_{r}^{\pm}\,$的超对称性生成元; 且$\,\mathscr{C}^{-1}\gamma_{d+1}\mathscr{C}=(-1)^{d/2}\gamma_{d+1}$.) 对偶数的$\,d$, 我们有
\[
\epsilon^{\mu_{1}\mu_{2}\cdots\mu_{d}}\,\gamma_{\mu_{1}}\gamma_{\mu_{2}}\cdots \gamma_{\mu_{p}}
\propto \gamma_{d+1}\gamma_{\mu_{p+1}}\gamma_{\mu_{p+2}}\cdots \gamma_{\mu_{d}} \:,
\]
而对奇数的$\,d$
\[
\epsilon^{\mu_{1}\mu_{2}\cdots\mu_{d}}\,\gamma_{\mu_{1}}\gamma_{\mu_{2}}\cdots \gamma_{\mu_{p}}
\propto \gamma_{\mu_{p+1}}\gamma_{\mu_{p+2}}\cdots \gamma_{\mu_{d}} \:.
\]
因此对任何$\,d$, $p\,$膜和$\,d{-}p\,$膜在方程(\ref{32.3.1})---(\ref{32.3.3})中的贡献相同, 这使得对偶数的$\,d$, 我们可以限制$\,p\,$只在$\,0\,$到$\,d/2\,$之间取值, 而对奇数的$\,d$, 则是$\,0\,$到$\,(d{-}1)/2$.

方程(\ref{32.3.2})---(\ref{32.3.4})中$\,p$-膜中心荷$\,z_{rs}^{p}\,$上的对称条件反应了反对易子的对称性. 方程(\ref{32.A.15}) 和(\ref{32.A.30})对$\,d=2n$\,和$\,d=2n+1\,$均给出
\begin{equation}
    \gamma_{\mu}^{\mathrm{T}} = (-1)^{n}\mathscr{C}^{-1}\gamma_{\mu}\mathscr{C} \:, \qquad
    \mathscr{C}^{\mathrm{T}} = (-1)^{n(n+1)/2}\mathscr{C} \:. \label{32.3.5}
\end{equation}
它们赋予了反对易乘积$\,\gamma_{[\mu_{1}}\gamma_{\mu_{2}}\cdots\gamma_{\mu_{p}]}\,$以对称性
\begin{align}
    \gamma_{[\mu_{1}}\gamma_{\mu_{2}}\cdots \gamma_{\mu_{p}]}\mathscr{C} &= (-1)^{pn}(-1)^{n(n+1)/2}
    \Bigl[\gamma_{[\mu_{p}}\gamma_{\mu_{p-1}}\cdots \gamma_{\mu_{1}]}\mathscr{C}\Bigr] ^{\mathrm{T}} \nonumber \\
    &= (-1)^{pn}(-1)^{n(n+1)/2} (-1)^{p(p-1)/2}
    \Bigl[\gamma_{[\mu_{1}}\gamma_{\mu_{2}}\cdots \gamma_{\mu_{p}]}\mathscr{C}\Bigr] ^{\mathrm{T}}\:. \label{32.3.6}
\end{align}
由此可以立即得出, 对奇数的\,$d$,
\begin{equation}
    z_{rs}^{\mu_{1}\mu_{2}\cdots\mu_{p}} = (-1)^{pn}(-1)^{n(n+1)/2} (-1)^{p(p-1)/2}
    z_{sr}^{\mu_{1}\mu_{2}\cdots\mu_{p}} \:, \label{32.3.7}
\end{equation}
而对偶数的$\,d$,
\begin{equation}
    z_{rs}^{\mu_{1}\mu_{2}\cdots\mu_{p}\,\pm} = (-1)^{pn}(-1)^{n(n+1)/2} (-1)^{p(p-1)/2}
    z_{sr}^{\mu_{1}\mu_{2}\cdots\mu_{p}\,(-1)^{n}(-1)^{p}\mp} \:. \label{32.3.8}
\end{equation}

例如, 考虑\,$d=11\,$维时空中$\,N=1\,$超对称性这个重要情况, 它是弦论的$\,M\,$理论推广的一个版本. 方程(\ref{32.3.8})表明, 除非
\begin{equation}
    -(-1)^{p}(-1)^{p(p-1)/2} = +1 \:, \label{32.3.9}
\end{equation}
否则单个$\,p$-形式中心荷$\,z_{\mu_{1}\mu_{2}\cdots\mu_{p}}\,$为零, 上式仅在$\,p\,$等于$\,1,2,5\,$时成立. $p=1\,$的值就由动量算符本身实现, 它来自于粒子以及扩展物体. 其它的可能性, $p=2\,$和$\,p=5$, 分别来自于有$\,2$-膜和$\,5$-膜的理论. 注意到, 不可能存在其它独立的张量中心荷, 例如来自于\,1-膜的\,1-形式, 这是因为$\,P^{\mu}$、$2$-形式以及$\,5$-形式中独立分量的总数是
\[
11 + \binom{11}{2} + \binom{11}{5} =528 \:,
\]
而两个\,32-分量的基础旋量的反对易子中的独立分量总数是$\,32\times 33/2=528$.

正如\,0-形式电荷是\,1-形式规范场$\,A_{\mu}(x)\,$的源, 一个$\,p$-形式守恒量$\,z_{\mu_{1}\mu_{2}\cdots\mu_{p}}\,$可以充当\,8.8\,节中所讨论的那类$\,p{+}1$-形式规范场$\,A_{\mu_{1}\mu_{2}\cdots \mu_{p+1}}\,$的源. 事实上, 这样的规范场确实出现在超引力理论中. 例如, 正如上一节所强调的, $d=11\,$维时空中的$\,N=1\,$超引力理论包含一个在小群$\,O(9)\,$下按照秩为\,3\,的反对称张量变换的无质量粒子, 因此描述它的必然是一个\,3-形式规范场$\,A_{\mu\nu\rho}(x)$. 对超引力理论解的研究表明确实存在为$\,A_{\mu\nu\rho}(x)\,$提供源的\,2-膜\cite{12}. 另外, 正如在\,8.8\,节中注意到的, 这个规范理论等价于一个$\,(d-p-2=6)$-形式规范场的理论, 而它的源可以由\,5-形式$\,z^{\mu_{1}\cdots\mu_{5}}\,$提供, 并且确实存在一个为这个\,6-形式规范场提供源的\,5-膜解.\cite{13} 11维中的$\,N=1\,$超对称代数确实有来自于这些\,2-膜和\,5-膜的贡献.\cite{11}


%+++++++++++++++++++++++附录++++++
\titleformat{\chapter}{\centering\CJKfamily{zhhei}\huge}{\chaptertitlename}{1em}{}
\titlespacing{\chapter}{0pt}{3.5ex plus .1ex minus .2ex}{10\wordsep}
\titleformat{\section}{\centering\CJKfamily{zhhei}\Large}{附 录}{1em}{}
\titlespacing{\section}{2em}{3.5ex plus .1ex minus .2ex}{1.5\wordsep}
\titleformat{\subsection}{\centering\CJKfamily{zhhei}\large}{}{0em}{}
\titlespacing{\subsection}{2em}{1.5ex plus .1ex minus .2ex}{\wordsep}
\renewcommand{\captionfont}{\small} \newcounter{app32}[chapter]
\setcounter{app32}{1}
\renewcommand\thesection{\Alph{app32}}
\renewcommand\theequation{\arabic{chapter}.\Alph{app32}.\arabic{equation}}
\fancyhf{} \fancyhead[CE]{\leftmark} \fancyhead[CO]{\rightmark}
\fancyhead[RO,LE]{$\cdot$\ \thepage\ $\cdot$}
\renewcommand{\headrulewidth}{0.8pt} \pagestyle{fancy}
\renewcommand{\chaptermark}[1]{\markboth{第\,\thechapter\,章\ #1}{}} \renewcommand{\sectionmark}[1]{\markright{附录 \quad\ #1}{}}


\section{高维中的旋量}

这个附录将描述任意$\,d\,$维时空中的\,Lorentz\,群$\,O(d{-}1,1)\,$的\,Lie\,代数的基础旋量表示. 它们可以从相应的\,Clifford\,代数获得, 这些代数由有限多个矩阵$\,\gamma_{\mu}\,$的不可约集构成, 并满足反对易关系
\begin{equation}
    \{\gamma_{\mu},\gamma_{\nu}\} = 2\eta_{\mu\nu} \:, \label{32.A.1}
\end{equation}
其中$\,\eta_{\mu\nu}\,$是对角的, 除了$\,\eta^{00}=-1\,$以外的对角元都是\,$+1$, 其中$\,x^{0}\,$是时间分量. 从这些我们可以构造矩阵
\begin{equation}
    \mathscr{J}_{\mu\nu}\equiv \frac{1}{4\mi}[\gamma_{\mu},\gamma_{\nu}] = -\mathscr{J}_{\nu\mu} \:, \label{32.A.2}
\end{equation}
它们满足\,Lorentz\,群生成元的对易关系(\textcolor{foo}{2.4.12})
\begin{equation}
    \mi\,[\mathscr{J}_{\mu\nu},\mathscr{J}_{\rho\sigma}] = \eta_{\nu\rho}\mathscr{J}_{\mu\sigma}
    -\eta_{\mu\rho}\mathscr{J}_{\nu\sigma} -\eta_{\sigma\mu}\mathscr{J}_{\rho\nu}
    +\eta_{\sigma\nu}\mathscr{J}_{\rho\mu} \:. \label{32.A.3}
\end{equation}
我们将会看到, 尽管方程(\ref{32.A.2})总给出了\,Lorentz\,代数的一个表示, 但这个表示不总是不可约表示.

 我们现在必须区分偶数维和奇数维的情况.

\subsection{偶数维:\,$d=2n$}

为了给$\,d=2n\,$维中的$\,\gamma\,$矩阵构成一个方便的具体表示, 我们引入$\,n\,$个矩阵 
\begin{equation}
    a_{u} \equiv \frac{1}{2}\Bigl(\gamma_{2u-1}+\mi\gamma_{2u}\Bigr) \qquad u=1,2,\cdots,n, \label{32.A.4}
\end{equation}
并将$\,\gamma_{1},\cdots,\gamma_{2n}\,$取为厄米的, 像往常一样有
\begin{equation}
    \gamma_{2n} \equiv -\mi\gamma_{0} \:. \label{32.A.5}
\end{equation}
它们有反对易关系
\begin{equation}
    \{a_{u},a_{v}^{\dag}\} =\delta_{uv} \:, \qquad \{a_{u},a_{v}\}=\{a_{u}^{\dag},a_{v}^{\dag}\} =0 \:. \label{32.A.6}
\end{equation}
我们在$\,\gamma\,$的表示空间中引入态矢$\,\lvert 0\rangle$, 它被定义成满足
\begin{equation}
    a_{u}^{\dag}\lvert0\rangle =0 \:, \label{32.A.7}
\end{equation}
并定义基矢
\begin{equation}
    \lvert s_{1}\,s_{2}\,\cdots s_{n}\rangle = a_{1}^{s_{1}}\,a_{2}^{s_{2}}\cdots a_{n}^{s_{n}}\lvert 0\rangle \:. \label{32.A.8}
\end{equation}
因为$\,a_{u}^{2}=0$, 如果$\,s_{u}=0$, 那么算符$\,a_{u}\,$将提高$\,s_{u}\,$的值至$\,+1$, 如果$\,s_{u}=+1$, 那么它将湮灭这个态矢(并产生一个符号因子$\,(-1)^{S}$, 其中$\,S\equiv \sum_{v<u}s_{v}$), 所以所有的$\,s_{u}\,$只能取$\,0\,$和$\,+1$, 态矢张开的空间因此是$\,2^{n}\,$维的. 在这个基下, 矩阵$\,a_{u}\,$取如下的形式
\begin{equation}
    a_{u} = \begin{pmatrix}
    -1 & 0 \\ 0 & 1
    \end{pmatrix} \otimes \cdots \otimes
     \begin{pmatrix}
    -1 & 0 \\ 0 & 1
    \end{pmatrix} \otimes
     \begin{pmatrix}
    0 & 1 \\ 0 & 0
    \end{pmatrix} \otimes 1 \cdots \otimes 1 \:, \label{32.A.9}
\end{equation}
其中最后一个$\,2\times 2\,$矩阵处在第$\,u\,$个位置上. 取厄米共轭和反厄米共轭就给出了$\,\gamma\,$矩阵
\begin{align}
     \gamma_{2u-1} &= \begin{pmatrix}
    -1 & 0 \\ 0 & 1
    \end{pmatrix} \otimes \cdots \otimes
     \begin{pmatrix}
    -1 & 0 \\ 0 & 1
    \end{pmatrix} \otimes
     \begin{pmatrix}
    0 & 1 \\ 1 & 0
    \end{pmatrix} \otimes 1 \cdots \otimes 1 \:, \label{32.A.10} \\
     \gamma_{2u} &= \begin{pmatrix}
    -1 & 0 \\ 0 & 1
    \end{pmatrix} \otimes \cdots \otimes
     \begin{pmatrix}
    -1 & 0 \\ 0 & 1
    \end{pmatrix} \otimes
     \begin{pmatrix}
    0 & -\mi \\ \mi & 0
    \end{pmatrix} \otimes 1 \cdots \otimes 1 \:. \label{32.A.11}
\end{align}
(注意, 这给出的表示与我们在\,5.4\,节中为四维中的$\,\gamma\,$矩阵引入的表示以及本书中所使用的四维 $\gamma\,$矩阵表示并不形同.)


方程(\ref{32.A.10})---(\ref{32.A.11})赋予了欧几里得$\,\gamma\,$简单的实性质和对称性
\begin{equation}
    \gamma_{i}^{\ast} = \gamma_{i}^{\mathrm{T}} = \begin{cases}
    \phantom{-}\gamma_{i} &  \qquad \text{当}\,i\,\text{是奇数时} \\
    -\gamma_{i} & \qquad \text{当}\,i\,\text{是偶数时}
    \end{cases}  \:, \label{32.A.12}
\end{equation}
其中$\,i=1,2,\cdots,2n$. 这可以表示成一个类似的关系
\begin{equation}
    \mathscr{C}^{-1}\gamma_{i}\mathscr{C} = (-1)^{n}\,\gamma_{i}^{\mathrm{T}}
    =(-1)^{n}\,\gamma_{i}^{\ast} \:, \label{32.A.13}
\end{equation}
其中$\,\mathscr{C}\,$是矩阵
\begin{equation}
    \mathscr{C} \equiv \gamma_{2}\gamma_{4}\cdots\gamma_{2n} \:. \label{32.A.14}
\end{equation}
将方程(\ref{32.A.5})中的因子$\,-\mi\,$考虑在内, 我们可以将其写成闵可夫斯基分量的形式
\begin{equation}
    \gamma_{\mu}^{\ast} = -\beta\gamma_{\mu}^{\mathrm{T}}\beta = -(-1)^{n}(\mathscr{C}\beta)^{-1}\gamma_{\mu}(\mathscr{C\beta})\:, \label{32.A.15}
\end{equation}
其中
\begin{equation}
    \beta \equiv \gamma_{2n} = -\mi\gamma_{0} \:. \label{32.A.16}
\end{equation}

在任意偶数维中, 类比四维中的$\,\gamma_{5}$, 我们可以定义矩阵$\,\gamma_{2n+1}$. 我们取
\begin{equation}
    \gamma_{2n+1}\equiv \mi^{n}\gamma_{1}\gamma_{2}\cdots \gamma_{2n} \:. \label{32.A.17}
\end{equation}
这里对相位的选取使得
\begin{equation}
    \gamma_{2n+1}^{2}=1 \:. \label{32.A.18}
\end{equation}
从反对易关系(\ref{32.A.1}), 可以理解得出$\,\gamma_{2n+1}\,$与其它\,$\gamma\,$矩阵反对易
\begin{equation}
    \{\gamma_{2n+1},\gamma_{\mu}\}=0 \qquad \text{对于}\:\mu=1,\,2,\,\cdots,\,2n-1,\,0 \:. \label{32.A.19}
\end{equation}
可以直接验证$\,\gamma_{2n+1}\,$是实的以及对称的
\begin{equation}
    \gamma_{2n+1}^{\dag} = \gamma_{2n+1}^{\ast} = \gamma_{2n+1}^{\mathrm{T}} =\gamma_{2n+1} \:. \label{32.A.20}
\end{equation}
我们从方程(\ref{32.A.19})中看到$\,\gamma_{2n+1}\,$与$\,O(2n{-}1,1)\,$代数的生成元(\ref{32.A.2})对易:
\begin{equation}
    [\gamma_{2n+1},\mathscr{J}_{\mu\nu}] = 0 \:, \label{32.A.21}
\end{equation}
这使得$\,\mathscr{J}_{\mu\nu}\,$不可能构成$\,O(2n{-}1,1)\,$代数的{\kai{不可约}}表示. 取而代之, 通过投影到$\,\gamma_{2n+1}=\pm 1\,$的子空间:
\begin{equation}
    \mathscr{J}_{\mu\nu}^{\pm} \equiv \mathscr{J}_{\mu\nu}\,\biggl(\frac{1\pm \gamma_{2n+1}}{2}\biggr)\:, \label{32.A.22}
\end{equation}
我们可以定义一对``Weyl''不可约表示. 从方程(\ref{32.A.15})和关系$\,(\mathscr{C}\beta)^{-1}\gamma_{2n+1}\mathscr{C}\beta=-(-1)^{n}\gamma_{2n+1}$, 我们看到\,Weyl Lorentz\,生成元的复共轭和转置是
\begin{equation}
    (\mathscr{J}_{\mu\nu}^{\pm})^{\ast} = -(\mathscr{C}\beta)^{-1}\,\mathscr{J}_{\mu\nu}^{\mp(-1)^{n}}\,(\mathscr{C}\beta) \:,\label{32.A.23}
\end{equation}
\begin{equation}
    (\mathscr{J}_{\mu\nu}^{\pm})^{\mathrm{T}} = -\mathscr{C}^{-1}\,\mathscr{J}_{\mu\nu}^{\pm(-1)^{n}}\,\mathscr{C}\:. \label{32.A.24}
\end{equation}
因此, 对于偶数的$\,n$, Weyl\,不可约表示等价于彼此的复共轭, 而对于奇数的$\,n$, 每个则等价于它自身的复共轭.\footnote{如果\,Lorentz\,代数的一个表示由矩阵$\,\mathscr{L}_{\mu\nu}\,$(例如$\,\mathscr{J}_{\mu\nu}$, $\mathscr{J}_{\mu\nu}^{+}\,$或$\,\mathscr{J}_{\mu\nu}^{-}$)给出, 而另一个表示由矩阵$\,\mathscr{L}'_{\mu\nu}\,$给出, 且$\,\mathscr{L}'_{\mu\nu}=-\mathscr{L}_{\mu\nu}^{\ast}$, 那么我们则称前者是后者的复共轭. 这里引入负号是因为在表示单位元附近\,Lorentz\,群元的矩阵形如$\,1+\frac{1}{2}\mi\,\omega^{\mu\nu}\mathscr{L}_{\mu\nu}$, 其中$\,\omega^{\mu\nu}\,$是实的无限小量.} 对于奇数的\,$n$, 我们仍然需要推导\,Weyl\,表示是{\kai{实的}}还是{\kai{赝实的}}, 如果是实的, 这将意味着存在矩阵$\,\mathscr{S}\,$使得
\begin{equation}
    -(\mathscr{S}\mathscr{J}_{\mu\nu}^{\pm}\mathscr{S}^{-1})^{\ast} =
    \mathscr{S}\mathscr{J}_{\mu\nu}^{\pm}\mathscr{S}^{-1} \:, \label{32.A.25}
\end{equation}
如果是赝实的, 则不存在这样的$\,\mathscr{S}$. 利用方程(\ref{32.A.23}), 条件(\ref{32.A.25})可以写成要求$\,\mathscr{S}^{-1}\mathscr{S}^{\ast}(\mathscr{C}\beta)^{-1}$ 与所有$\,\mathscr{J}_{\mu\nu}^{\pm}\,$都对易. 由于矩阵$\,\mathscr{J}_{\mu\nu}^{\pm}\,$构成了一个不可约集, 这将要求$\,\mathscr{S}^{-1}\mathscr{S}^{\ast}(\mathscr{C}\beta)^{-1}\,$正比于单位矩阵
\begin{equation}
    \mathscr{C} \beta = \alpha \mathscr{S}^{-1}\mathscr{S}^{\ast} \:, \label{32.A.26}
\end{equation}
其中$\,\alpha\,$是某个常数. 为了使之是可能的, 我们必须有
\begin{equation}
    \mathscr{C} \beta \,(\mathscr{C} \beta )^{\ast} = \lvert\alpha\rvert^{2}\cdot 1 \:. \label{32.A.27}
\end{equation}
但$\,\mathscr{C}\beta=\gamma_{2}\gamma_{4}\cdots\gamma_{2n-2}$, 又因为所有$\,i\,$为偶数的$\,\gamma_{i}\,$都是虚的, 我们有
\begin{equation}
    \mathscr{C} \beta \,(\mathscr{C} \beta )^{\ast} = (-1)^{n-1}(\gamma_{2}\gamma_{4}\cdots\gamma_{2n-2})^{2}
    (-1)^{a}\cdot 1 \:, \label{32.A.28}
\end{equation}
其中
\begin{equation}
    a= (n-1) + (n-2) + \cdots + 1  = n(n-1)/2 \:. \label{32.A.29}
\end{equation}
因此\,Weyl\,表示仅在$\,n=1\:(\operatorname{mod}4)\,$是实的且在$\,n=3\:(\operatorname{mod}4)\,$必须是赝实的.

为了在\,\ref{sec:32.1}\,节使用, 我们又注意到
\begin{equation}
    \mathscr{C}^{\ast} = (-1)^{n}\mathscr{C} \:, \qquad \mathscr{C}^{\mathrm{T}}=(-1)^{n(n+1)/2}\mathscr{C} \:,\qquad
    \mathscr{C}^{-1} = (-1)^{n(n-1)/2}\mathscr{C} \:, \label{32.A.30}
\end{equation}
因此方程(\ref{32.A.13})给出
\begin{equation}
    (\mathscr{C}\gamma_{\mu})^{\mathrm{T}} = (-1)^{n(n-1)/2}\mathscr{C}\gamma_{\mu} \:. \label{32.A.31}
\end{equation}

$\gamma_{\mu}$构成一个矢量, 也就是说
\begin{equation}
    [\mathscr{J}_{\mu\nu},\gamma_{\rho}] = -\mi\gamma_{\mu}\eta_{\nu\rho} + \mi \gamma_{\nu}\eta_{\mu\rho} \:, \label{32.A.32}
\end{equation}
并且它们有正常的宇称, 即
\begin{equation}
    \beta\,\gamma_{0}\,\beta = +\gamma_{0} \:, \qquad
    \beta\,\gamma_{i}\,\beta =-\gamma_{i} \quad \text{对于}\: i =1,\,\cdots,\,2n-1 \:. \label{32.A.33}
\end{equation}
反对易关系(\ref{32.A.1})阻止我们用$\,\gamma\,$的对称积来构造新的张量, 但它允许我们构造秩至多为$\,2n\,$的反对称张量
\begin{equation}
    \gamma_{[\mu_{1}}\gamma_{\mu_{2}}\cdots \gamma_{\mu_{p}]}\:, \label{32.A.34}
\end{equation}
其中方括号代表反对称化, 且$\,p\geq 2n$. 每个秩的独立时空分量个数是二项式系数$\,\binom{2n}{p}$, 所以这类矩阵的总数是
\begin{equation}
    \sum_{p=0}^{2n} \binom{2n}{p} = 2^{2n} \:. \label{32.A.35}
\end{equation}
这些矩阵没有一个为零(可以通过计算它们的平方看到)并且它们的\,Lorentz\,和(或)宇称变换性质均不同, 因此是线性独立的, 所以任何$\,2^{n}\times 2^{n}\,$的矩阵可以表示成$\,2^{2n}\,$个反对称张量(\ref{32.A.34})的线性组合.


\subsection{奇数维:\,$d=2n+1$}

现在我们来考虑时空维数$\,d=2n+1\,$是奇数的情况. 我们可以轻松地找到$\,2n{+}1\,$个满足反对易关系(\ref{32.A.1})$\,n\times n$ Dirac\,矩阵: 对于$\,\mu=1,2,\cdots,2n{-}1,0$的$\,\gamma_{\mu}$, 我们就可以使用$\,d=2n\,$时的那些$\,\gamma_{\mu}$, 然后加入方程(\ref{32.A.17})定义的矩阵$\,\gamma_{2n+1}$. 根据方程(\ref{32.A.18})和(\ref{32.A.19}), 这些$\,\gamma\,$矩阵满足反对易关系(\ref{32.A.1}), 其中$\,\mu\,$和$\,\nu\,$在$\,1,2,\cdots,2n{-}1,0,2n{+}1\,$中取值, 并且依旧有$\,\gamma_{0}=\mi\gamma_{2n}$.

不像偶数维的情况, 这里我们找不到任何与所有\,Lorentz\,生成元都对易的非平庸矩阵, 这是因为方程(\ref{32.A.17})和(\ref{32.A.18})表明$\,2n{+1}\,$个$\,\gamma\,$矩阵的乘积是平庸的:
\begin{equation}
    \gamma_{1}\gamma_{2}\cdots\gamma_{2n}\gamma_{2n+1} = \mi^{-n}\cdot 1 \:. \label{32.A.36}
\end{equation}
因此, $\mu\,$和$\,\nu\,$在$\,1,2,\cdots,2n{-}1,0,2n{+}1\,$中取值的\,Lorentz\,生成元(\ref{32.A.2})自身就构成了\,Lorentz\,群的一个不可约表示. 为了检验它们的实性质, 注意到$\,\gamma_{2n+1}\,$是满足\,$(\mathscr{C}\beta)^{-1}\gamma_{2n+1}\mathscr{C}\beta=-(-1)^{n}\gamma_{2n+1}$ 的实对称矩阵, 所以方程(\ref{32.A.15})在$\,\mu=2n+1\,$以及$\,\mu=1,2,\cdots,2n-1,0\,$时都成立. Lorentz\,生成元因此满足
\begin{equation}
    \mathscr{J}_{\mu\nu}^{\ast} = -(\mathscr{C}\beta)^{-1}\,\mathscr{J}_{\mu\nu}\,\mathscr{C}\beta \:,\label{32.A.37}
\end{equation}
\begin{equation}
    \mathscr{J}_{\mu\nu}^{\mathrm{T}} = -\mathscr{C}^{-1}\,\mathscr{J}_{\mu\nu}\,\mathscr{C}\:, \label{32.A.38}
\end{equation}
所以奇数维中的基础旋量表示不是实的就是赝实的. 与$\,d=2n\,$时精确相同的讨论告诉我们$\,d=2n+1\,$时的旋量表示再次根据方程(\ref{32.A.28})中的符号$\,(-1)^{a}\,$是正还是负而分别是实的还是赝实的, 因此根据方程(\ref{32.A.29}), 它们在$\,n=0\:(\operatorname{mod}4)\,$以及$\,n=1\:(\operatorname{mod}4)\,$是实的而在$\,n=2\:(\operatorname{mod}4)\,$以及$\,n=3\:(\operatorname{mod}4)\,$是赝实的.

我们现在可以再次构造出反对称张量(\ref{32.A.34}), 现在的秩$\,p\,$至多是$\,2n+1$, 但是因为它们满足关系
\begin{equation}
    \epsilon^{\mu_{1}\mu_{2}\cdots\mu_{2n+1}} \gamma_{[\mu_{1}}\gamma_{\mu_{2}}\cdots \gamma_{\mu_{p}]}
    \propto \gamma^{[\mu_{p+1}}\gamma^{\mu_{p+2}}\cdots \gamma^{\mu_{2n+1}]} \:, \label{32.A.39}
\end{equation}
其中$\, \epsilon^{\mu_{1}\mu_{2}\cdots\mu_{2n+1}} \,$像往常一样全反对称, 所以只有一半是独立的. (当$\,d=2n\,$时, 因为方程(\ref{32.A.39}) 两边宇称相反, 所以这样的关系不可能存在, 而对于$\,d=2n+1$, 因为其中的$\, \epsilon^{\mu_{1}\mu_{2}\cdots\mu_{2n+1}} \,$宇称为偶, 所以这样的讨论不再成立.) 形如(\ref{32.A.34})的独立矩阵的总数是
\begin{equation}
    \sum_{p=0}^{n} \binom{2n+1}{p} = 2^{2n} \:, \label{32.A.40}
\end{equation}
所以任何$\,2n\times 2n\,$矩阵可以写成$\,n{+1}\,$个$\,0\leq p\leq n+1\,$的独立反对称张量(\ref{32.A.34})的线性组合.

最后, 我们注意到无论$\,d=2n\,$还是$\,d=2n+1$, $O(d{-}1,1)$的Dirac\,和\,Lorentz\,代数可以通过令
\begin{equation}
    \gamma_{2n} \equiv -\mi\gamma_{0} \:, \qquad \mathscr{J}_{i\,2n} \equiv -\mi\mathscr{J}_{i0}  \label{32.A.41}
\end{equation}
和相应的$\,O(d)\,$代数相关联, 这使得
\begin{equation}
    \{\delta_{i},\delta_{j}\} =2\delta_{ij} \:, \label{32.A.42}
\end{equation}
和
\begin{equation}
    \mathscr{J}_{ij} = \frac{1}{4\mi}[\gamma_{i},\gamma_{j}] = -\mathscr{J}_{ji} \:, \label{32.A.43}
\end{equation}
其中$\,i\,$和$\,j\,$在$\,1\,$到$\,d\,$之间取值. 从方程(\ref{32.A.42})可以得出, 当$\,i\neq j\,$时, $\mathscr{J}_{ij}^{2}=1/4$, 所以每个$\,\mathscr{J}_{ij}\,$的本征值被限制到$\,\pm 1/2$. 更确切一些, 在基础旋量表示中, Cartan\,子代数的生成元表示成
\begin{equation}
    \mathscr{J}_{2u{-}1\,2u} =\frac{1}{2}[a_{u},a_{u}^{\ast}] = aa_{u}^{\ast} -\frac{1}{2} \:, \label{32.A.44}
\end{equation}
使得基矢(\ref{32.A.8})是本征矢, 且
\begin{equation}
     \mathscr{J}_{2u{-}1\,2u}\,\lvert s_{1}\,s_{2}\,\cdots s_{n} \rangle =
     \biggl(s_{u}-\frac{1}{2}\biggr)\,\lvert s_{1}\,s_{2}\,\cdots s_{n} \rangle \:. \label{32.A.45}
\end{equation}
维数$\,d=2n$\, 与$\,d=2n+1\,$之间的差异是: 当$\,d=2n\,$时, 我们有两个基础旋量表示, 而$\,\gamma_{2n+1}\,$的本征值$(-2\sigma_{1})(-2\sigma_{2})\cdots(-2\sigma_{n})\,$被限制成$\,+1\,$或$\,-1$, 而$\,d=2n{+}1\,$是, 基础旋量表示只有一个, $\sigma_{\mu}\,$上没有这样的限制.

在\,\ref{sec:32.1}\,节中, 正是$\,\mathscr{J}_{ij}\,$被限制成了$\,\pm 1/2\,$指出了基础旋量表示是费米对称性生成元唯一能够构成的\,Lorentz\,代数表示. 诚然, 从这个条件我们可以推断出$\,O(d)\,$生成元可以被形如(\ref{32.A.8})的一组基表示, 其中$\,s_{u}=\sigma_{u}+1/2$, 然后反向进行这个附录的推导, 利用方程(\ref{32.A.4})---(\ref{32.A.7})(以及$\,d\,$为奇时的(\ref{32.A.17}))将\,Lorentz\,生成元表示成一组满足反对易关系(\ref{32.A.1})的$\,\gamma_{\mu}$.




\section*{习题}
\noindent 1. 假定所有中心荷为零, 对六维时空中所有允许超对称性中的无质量粒子多重态进行分类. \\

\noindent 2. 假定粒子的自旋最多可达$\,j=3$, 但不能更高. 将自旋\,2\,无质量粒子确实存在这个事实考虑在内, 能够允许超对称性存在的最大时空维数是多少? 在每个允许的时空维数中, 超对称生成元的最大数目是多少? \\

\noindent 3. 考虑十维时空中的\,IIA\,型和\,IIB\,型超对称性, 假定在扩充超对称线性反对易关系中只有标量出现了. 找到粒子质量的一个下界, 并表示成中心荷. 描述粒子质量处在这个下界上的``BPS''有质量粒子多重态. \\

\noindent 4. 对九维时空中的$\,N=1\,$超对称性, 列出所有可能的独立标量和(或)张量中心荷.



%++++++++++++++++++参考文献+++++++++
\renewcommand{\sectionmark}[1]{\markright{ #1}{}}
\renewcommand{\bibname}{参考文献}

\begin{thebibliography}{99}
    \bibitem{1} T. Kaluza, {\textit{Sitz. Preuss. Akad. Wiss}}. {\bf{K1}}, 966 (1921).
    \bibitem{2} O. Klein, {\textit{Z. Phys.}} {\bf{37}}, 895 (1926); {\textit{Nature}} {\bf{118}}, 516 (1926).
    \bibitem{3} J. H. Schwarz, {\textit{Nucl. Phys.}} {\bf{B46}}, 61 (1972); R. C. Brower and K. A. Friedman, {\textit{Phys. Rev.}} {\bf{D7}}, 535 (1972).
    \bibitem{4} E. Cremmer, B. Julia\,和\,J. Scherk\,建立了\,11\,维时空中的超引力理论, {\textit{Phys. Lett.}} {\bf{76B}}, 409 (1978). 十维时空中的弱耦合\,IIA\,型弦论有一个\,11\,维起源的想法来自于\,M. J. Duff, P. S. Howe, T. Inami, and K. Stelle, {\textit{Phys. Lett.}} {\bf{B191}}, 70 (1987). P. K. Townsend\,证明了强耦合十维\,IIA\,型弦论的情况, {\textit{Phys. Lett.}} {\bf{B350}}, 184 (1995). E. Witten指出这些理论以及所有其它十维弦论之间的联系, {\textit{Nucl. Phys.}} {\bf{B445}}, 85 (1995).
    \bibitem{5} R. Haag, J. Lopuszanski, and M. Sohnius, {\textit{Nucl. Phys.}} {\bf{B88}}, 257 (1975).
    \bibitem{6} W. Nahm, {\textit{Nucl. Phys.}} {\bf{B135}}, 149 (1978).
    \bibitem[6a]{6a} 例如, 可参看, H. W. Turnbull and A. C. Aitkens, {\textit{An Introduction to the Theory of Canonical Matrices}} (Dover Publications, New York, 1961).
    \bibitem[6b]{6b} J. Strathdee\,给出了一个有用的总结, {\textit{Int. J. Mod. Phys.}} {\bf{A2}}, 273 (1987).
    \bibitem{7} 关于四维时空中的这些讨论, 参看\,S. Weinberg, {\textit{Phys. Rev.}} {\bf{135}}, B1049 (1964); {\textit{Phys. Rev.}} {\bf{138}}, B988 (1965); R. P. Feynman, 未发表; M. T. Grisaru and H. N. Pendleton, {\textit{Phys. Lett.}} {\bf{67B}}, 323 (1977). 更高维时空中的讨论是类似的.
    \bibitem{8} 例如, 可参看, J. Hughes, J. Liu, and J. Polchinski, {\textit{Phys. Lett.}} {\bf{B180}}, 370 (1986); E. Bergshoeff, E. Sezgin, P. K. Townsend, {\textit{Phys. Lett.}} {\bf{B189}}, 75 (1987); A. Ach\'{u}carro, J. M. Evans, P. K. Townsend, and D. L. Wiltshire, {\textit{Phys. Lett.}} {\bf{B198}}, 441 (1987); P. K. Townsend, {\textit{Phys. Lett.}} {\bf{B202}}, 53 (1988); P. K. Townsend, 收录于\,\textit{Particles, Strings, and Cosmology: Proceedings of Workshop on Current Problems in Particle Theory 19 at Johns Hopkins University, March 1995} (World Scientific, Singapore, 1996); J. Maldacena, {\textit{Adv. Theor. Math. Phys.}} {\bf{2}}, 231 (1998). 关于综述, 参看\,M. J. Duff, R. R. Khuri, and J.-X. Lu, {\textit{Phys. Rep.}} {\bf{259}}, 213 (1995); A. Giveon and D. Kutasov, 1998 preprint hep-th/9802067, 将发表于\,{\textit{Rev. Mod. Phys}}.
    \bibitem{9} J. W. van Holten and A. Van Proeyen, {\textit{J. Phys. A: Math. Gen.}} {\bf{15}}, 3763 (1982).
    \bibitem{10} J. A. de Azc\'{a}rraga, J. P. Gauntlett, J. M. Izquierdo, and P. K. Townsend, {\textit{Phys. Rev. Lett.}} {\bf{63}}, 2443 (1989).
    \bibitem{11} D. Sorokin and P. K. Townsend, {\textit{Phys. Lett.}} {\bf{B412}}, 265 (1997); J. P. Gauntlett, J. Gomis, and P. K. Townsend, {\textit{J. High Energy Phys.}} {\bf{9801}}, 003 (1998).
    \bibitem{12} 这些\,2-膜的协变场方程是\,E. Bergshoeff, E. Sezgin, P. K. Townsend\,给出的, 参考文献[8]和\,{\textit{Ann. Phys. (NY)}} {\bf{185}}, 330 (1988). M. J. Duff\,和\,K. Stelle\,证明了这些\,2-膜是超引力场方程的解并为\,3-形式规范场提供了源, {\textit{Phys. Lett.}} {\bf{B253}}, 113 (1991).
    \bibitem{13} R. Gueven\,证明了这些\,5-膜是超引力场方程的解并为\,6-形式规范场提供了源, {\textit{Phys. Lett.}} {\bf{B276}}, 49 (1992). M. Aganagic, J. Park, C. Popescu, and J. Schwarz\,给出了这些\,5-膜的协变场方程, {\textit{Nucl. Phys.}} {\bf{B496}}, 191 (1997); P. S. Howe and E. Sezgin, {\textit{Phys. Lett.}} {\bf{B394}}, 62 (1997); P. Pasti, D. Sorokin, and M. Tonin, {\textit{Phys. Lett.}} {\bf{B398}}, 41 (1997); P. S. Howe, E. Sezgin, and P. C. West, {\textit{Phys. Lett.}} {\bf{B399}}, 49 (1997); I. Bandos, K. Lechner, A. Nurmagambetov, P. Pasti, D. Sorokin, and M. Tonin, {\textit{Phys. Rev. Lett.}} {\bf{78}}, 4332 (1997).
\end{thebibliography}


%\end{CJK*}%结束CJK
\end{document} %文档结束
