\chapter{超对称代数} \label{cha:25}

这一章将会从第一原理出发, 沿用\,Haag, Lopuszanski\,和\,Sohnius\,的处理,\cite{1} 来发展超对称代数的形式. 我们将会看到, 在\,Coleman-Mandula\,定理成立的条件下, 这一结构几乎被\,Lorentz\,不变的要求完全确定了. 在这之后, 超多重态的结构就可以直接从超对称代数中推出.


\section{阶化\,Lie\,代数和阶化参量} \label{sec:25.1}

我们在\,2.2\,节看到了如何将任意连续对称变换写成\,Lie\,代数的形式, 这个\,Lie\,代数由线性独立的对称性生成元$\,t_{a}\,$生成, 并且生成元满足对易关系$\,[t_{a},t_{b}]=\mi\sum_{c}C_{ab}^{c}t_{c}$. 以非常类似的方式, 用来表示超对称的生成元$\,t_{a}\,$构成了{\kai{阶化}}Lie代数,\cite{2} 这样的\,Lie\,代数表现为如下形式的对易关系{\kai{和}}反对易关系
\begin{equation}
t_{a}t_{b}-(-1)^{\eta_{a}\eta_{b}}t_{b}t_{a}=\mi\sum_{c}C_{ab}^{c}t_{c}\:.\label{25.1.1}
\end{equation}
(本节不使用求和决定.) 对于每个$\,a$, $\eta_{a}\,$是$\,+1\,$或$\,0$, 它是生成元$\,t_{a}\,$的{\kai{阶数}}, 而$\,C_{ab}^{c}\,$是一组数值的结构常数. $\eta_{a}=1\,$的生成元$\,t_{a}\,$被称为{\kai{费米的}}; 其它那些$\,\eta_{a}=0\,$的生成元则被称为{\kai{玻色的}}. 对于玻色算符和玻色算符以及玻色算符和费米算符, 方程(\ref{25.1.1})提供了对易关系, 而对于费米算符和费米算符, 它则提供了反对易关系. 我们暂且先来看一下它对结构常数产生的影响, 在此之后再回到提出方程(\ref{25.1.1})的动机.

根据方程(\ref{25.1.1}), 结构常数必须满足条件
\begin{equation}
C_{ab}^{c}=-(-1)^{\eta_{a}\eta_{b}}C_{ab}^{c}\:.\label{25.1.2}
\end{equation}
对于任何由场算符的泛函构成的算符, 两个玻色算符的乘积或者两个费米算符的乘积是玻色的, 而一个费米算符与一个玻色算符的乘积是费米的, 这使得
\begin{equation}
C_{ab}^{c}=0\quad\text{除非}\quad \eta^{c}=\eta^{a}+\eta^{b}\:(\operatorname{mod} 2)\:. \label{25.1.3}
\end{equation}
另外, 对于任何以这种方式构建的算符, 玻色算符和费米算符的厄米伴分别是玻色的和费米的. 如果$\,t_{a}\,$是厄米算符, 那么结构常数满足实条件
\begin{equation}
{C_{ab}^{c}}^{\ast}=-C_{ba}^{c}\:.\label{25.1.4}
\end{equation}

结构常数同时还满足一个非线性约束, 这个约束来自于超\,Jacobi\,恒等式
\begin{equation}
(-1)^{\eta_{c}\eta_{a}}[[t_{a},t_{b}\},t_{c}\}+(-1)^{\eta_{a}\eta_{b}}[[t_{b},t_{c}\},t_{a}\}
+(-1)^{\eta_{b}\eta_{c}}[[t_{c},t_{a}\},t_{b}\}=0\:.   \label{25.1.5}
\end{equation}
这里的``$[\cdots\}$''类似于方程(\ref{25.1.1})左边出现的对易子/反对易子, 但在这里推广至任意的阶化算符 $O,O^{\prime}\,\cdots$
\begin{equation}
[O,O^{\prime}\}\equiv OO^{\prime}-(-1)^{\eta(O)\eta(O^{\prime})}O^{\prime}O=-(-1)^{\eta(O)\eta(O^{\prime})}[O^{\prime},O\}\:, \label{25.1.6}
\end{equation}
现在它被理解成, 生成元的任意乘积$\,O=t_{a}t_{b}t_{c}\cdots\,$被赋予阶数$\,\eta(O)\equiv \eta_{a}+\eta_{b}+\eta_{c}+\cdots(\operatorname{mod}2)$. (为证明方程(\ref{25.1.5}), 只需证明$\,t_{a}t_{b}t_{c}\,$和$\,t_{a}t_{c}t_{b}\,$的系数为零即可, 至于生成元的其它乘积, 方程(\ref{25.1.5}) 左边在轮换$\,abc\to bca\to cab\,$下的对称性会确保它们的系数为零. $t_{a}t_{b}t_{c}\,$在方程(\ref{25.1.5})中的系数是
\[
(-1)^{\eta_{c}\eta_{a}}-(-1)^{\eta_{a}\eta_{b}}(-1)^{\eta_{a}(\eta_{b}+\eta_{c})}=0\:,
\]
而$\,t_{a}t_{c}t_{b}\,$的系数是
\[
(-1)^{\eta_{a}\eta_{b}}(-1)^{\eta_{b}\eta_{c}}(-1)^{\eta_{a}(\eta_{b}+\eta_{c})}
-(-1)^{\eta_{b}\eta_{c}}(-1)^{\eta_{c}\eta_{a}}=0\:,
\]
证毕.) 将方程(\ref{25.1.1})代入方程(\ref{25.1.5}), 我们发现约束
\begin{equation}
\sum_{d}(-1)^{\eta_{c}\eta_{a}}C_{ab}^{d}C_{dc}^{e}+\sum_{d}(-1)^{\eta_{a}\eta_{b}}C_{bc}^{d}C_{da}^{e}
+\sum_{d}(-1)^{\eta_{b}\eta_{c}}C_{ca}^{d}C_{db}^{e}=0 \:.\label{25.1.7}
\end{equation}
当然, 在所有生成元都是玻色生成元的情况下, 方程(\ref{25.1.5})就是通常的\,Jacobi\,恒等式, 而方程 (\ref{25.1.7}) 就是结构常数之间通常的非线性约束(\textcolor{foo}{2.2.22}).

方程(\ref{25.1.1})可以取作我们的出发点, 但是就像在\,2.2\,节我们对普通\,Lie\,代数所做的那样, 我们可以赋予它一个动机, 这样它就不是出发点而是有限连续对称变换的一个必要特征. 与\,2.2\,节不同的是, 现在这些变换依赖于连续的{\kai{阶}}参量. 一组阶化\,c\,-数参量可以视为``数'', 这些数既包含格拉斯曼参量(参看\,9.5\,节)也包含普通数, 它们满是算术的结合律和分配率, 但是不再满足简单的交换律, 而是满足关系
\begin{equation}
\alpha^{a}\beta^{b}=(-1)^{\eta_{a}\eta_{b}}\beta^{b}\alpha^{a}\:, \label{25.1.8}
\end{equation}
其中$\,\alpha^{a},\,\beta^{a},\cdots\,$用来区分第$\,a\,$个参量的不同值, 以矢量代数中的方法, 我们可以用$\,v^{a}\,$和$\,u^{a}\,$来标记两个不同实矢量的$\,a\,$-分量. 和以前一样, 第$\,a\,$个阶化参量被赋予阶数$\,\eta_{a}\,$, 当$\,\alpha^{a}\,$分别是费米参量和玻色参量时, $\eta_{a}\,$分别等于$\,+1\,$和$\,0$. 即, 如果这些参量中有一个是玻色的, 那么它们就是对易的, 如果两个参量都是费米的, 那么它们就是反对易的. 阶化参量的乘积$\,\alpha^{a}\beta^{b}\gamma^{c}\cdots\,$被赋予阶数$\,\eta_{a}+\eta_{b}+\eta_{c}+\cdots\:%
(\operatorname{mod}2)$; 即, 如果这个乘积中包含奇数个费米参量, 那么它就是费米的, 否则就是玻色的. 有了这个阶数, 很容易看到阶化参量的乘积满足的对易规则或反对易规则类似于方程(\ref{25.1.8}).

考察这样的连续变换$\,T_{\alpha}$, 在形式上它由阶化参量$\,\alpha^{a}\,$的幂级数给出:
\begin{equation}
T(\alpha)=1+\sum_{a}\alpha^{a}t_{a}+\sum_{ab}\alpha^{a}\alpha^{b}t_{ab}+\cdots \:, \label{25.1.9}
\end{equation}
其中$\,t_{a}\,$, $\,t_{ab}\,$等是一组与$\,\alpha\,$无关的算符系数, 这时我们还没有假定它们要满足任何像方程(\ref{25.1.1})这样的代数关系. 由于参量$\,\alpha^{a}\,$满足方程(\ref{25.1.8}), 系数$\,t_{ab\cdots}\,$必须要满足对称/反对称条件, 例如
\begin{equation}
t_{ab}=(-1)^{\eta_{a}\eta_{b}}t_{ba}\:.\label{25.1.10}
\end{equation}
同时假定变换$\,T(\beta)\,$与任意阶化参量的任意值$\,\alpha^{a}\,$对易, 这将会方便我们的讨论, 在这一情况下, (\ref{25.1.9})中的算符系数满足条件
\begin{equation}
\alpha^{a}t_{b}=(-1)^{\eta_{a}\eta_{b}}t_{b}\alpha^{a}\:,\label{25.1.11}
\end{equation}
\begin{equation}
\alpha^{a}t_{bc}=(-1)^{\eta_{a}(\eta_{b}+\eta_{c})}t_{bc}\alpha^{a}\:.\label{25.1.12}
\end{equation}
即, 在$\,t_{a}\,$和$\,t_{bc}\,$与阶化参量满足的对易关系和反对易关系中, 它们自身分别就像是阶数分别为$\,\eta_{b}\,$和 $\eta_{b}+\eta_{c}\,(\operatorname{mod}2)\,$的阶化参量.

算符上的其它约束来自于$\,T(\alpha)\,$构成半群的要求; 即, 对于阶化参量取不同值$\,\alpha\,$和$\,\beta\,$时的$\,T\,$算符, 它们的乘积也是一个$\,T\,$算符
\begin{equation}
T(\alpha)T(\beta)=T(f(\alpha,\beta))\:,\label{25.1.13}
\end{equation}
其中$\,f^{c}(\alpha,\beta)\,$本身是阶化参量的形式幂级数. 由于$\,T(0)T(\beta)=T(\beta)\,$以及$\,T(\alpha)T(0)=T(\alpha)$, 我们必须有
\begin{equation}
f^{c}(0,\beta)=\beta^{c}\:,\qquad\qquad f^{c}(\alpha,0)=\alpha^{c}\:,\label{25.1.14}
\end{equation}
因此$\,f(\alpha,\beta)\,$的幂级数展开必须采取如下的形式
\begin{equation}
f^{c}(\alpha,\beta)=\alpha^{c}+\beta^{c}+\sum_{ab}f_{ab}^{c}\,\alpha^{a}\,\beta^{b}+\cdots\:,\label{25.1.15}
\end{equation}
其中$\,f_{ab}^{c}\,$是一组普通常数(即, 玻色常数), 而``$\cdots$''代表阶化参量的三阶项或者更高阶项. 为了使 $f^{c}(\alpha,\beta)\,$是阶化参量, 方程(\ref{25.1.15})的每一项必须要有相同的阶数, 这意味着
\begin{equation}
f_{ab}^{c}=0\quad{\text{除非}}\quad \eta^{c}=\eta^{a}+\eta^{b}\:(\operatorname{mod}2)\:.\label{25.1.16}
\end{equation}
将幂级数(\ref{25.1.9})和(\ref{25.1.15})代入乘积规则(\ref{25.1.13}), 这给出
\begin{align*}
&\Bigl[1+\sum_{a}\alpha^{a}t_{a}+\sum_{ab}\alpha^{a}\alpha^{b}t_{ab}+\cdots\Bigr]
\Bigl[1+\sum_{a}\beta^{a}t_{a}+\sum_{ab}\beta^{a}\beta^{b}t_{ab}+\cdots\Bigr]  \\
&\quad=1+\sum_{c}\Bigl(\alpha^{c}+\beta^{c}+\sum_{ab}f_{ab}^{c}\alpha^{a}\beta^{b}+\cdots \Bigr)t_{c}\\
&\qquad +\sum_{cd}\Bigl(\alpha^{c}+\beta^{c}+\cdots\Bigr)\Bigl(\alpha^{d}+\beta^{d}+\cdots\Bigr)t_{cd}+\cdots\:.
\end{align*}
$1,\,\alpha^{a},\,\beta^{a},\,\alpha^{a}\alpha^{b}\,$和$\,\beta^{a}\beta^{b}\,$的系数在方程两边自动匹配, 而要求$\,\alpha^{a}\beta^{b}\,$的系数相等这个条件给出了不平庸的关系
\begin{equation}
(-1)^{\eta_{a}\eta_{b}}t_{a}t_{b}=\sum_{c}f_{ab}^{c}t_{c}+t_{ab}+(-1)^{\eta_{a}\eta_{b}}t_{ba}
=\sum_{c}f_{ab}^{c}t_{c}+2t_{ab}\:.\label{25.1.17}
\end{equation}
(左边的符号因子来自于$\,t_{a}\,$和$\,\beta^{b}\,$的交换.) 加上同一类的高阶关系, 如果我们知道生成元$\,t_{a}\,$和群组合函数$\,f^{a}(\alpha,\beta)$, 这将使得我们
可以计算出整个函数(\ref{25.1.9}). 但为了使这个计算是可能的, $t_{a}\,$必须要满足一个约束. 利用方程(\ref{25.1.10}), 方程(\ref{25.1.17})与交换$\,a,b\,$后的同一方程的差或和给出\,Lie\,超代数关系(\ref{25.1.1}), 而结构常数给定为
\begin{equation}
\mi\,C_{ab}^{c}=(-1)^{\eta_{a}\eta_{b}}f_{ab}^{c}-f_{ba}^{c}\:.\label{25.1.18}
\end{equation}
另外, 从方程(\ref{25.1.16})和(\ref{25.1.18})就立即得出了方程(\ref{25.1.3}).

对于反对易\,c\,-数$\,\alpha\,$的复共轭$\,\alpha^{\ast}$, 它的定义要使得$\,\alpha\,$与任意算符$\,\mathcal{O}\,$的乘积的厄米共轭是
\begin{equation}
(\alpha\mathcal{O})^{\ast}=\mathcal{O}^{\ast}\alpha^{\ast}\:.\label{25.1.19}
\end{equation}
由此得出\,c\,-数在复共轭下的行为与算符在厄米共轭下的行为相同:
\begin{equation}
(\alpha\beta)^{\ast}=\beta^{\ast}\alpha^{\ast}\:, \label{25.1.20}
\end{equation}
并且$\,\alpha^{\ast}\,$与$\,\alpha\,$的阶数相同.

阶化\,Lie\,代数对物理的意义被时空对称性严格限制了. 我们现在转向对这些约束的考察.

\section{超对称代数} \label{sec:25.2}

考察一个对称性生成元与$S$-矩阵对易的一般阶化\,Lie\,代数. 如果$\,Q\,$是其中的一个费米对称性生成元, 那么$\,U^{-1}(\Lambda)\,Q\,U(\Lambda)\,$也是, 其中$\,U(\Lambda)\,$是任意齐次\,Lorentz\,变换$\,\Lambda\indices{^\mu_{\nu}}\,$对应的量子力学算符. 因此$\,U^{-1}(\Lambda)\,Q\,U(\Lambda)\,$是费米对称性生成元完备集的线性组合, 由此得出, 这组生成元必须构成齐次\,Lorentz\,群的一个表示. 这样, 根据各个生成元所属的齐次\,Lorentz\,群不可约表示就可以对它们进行分类.

正如\,5.6\,节所描述的, 对于任何一组构成齐次\,Lorentz\,群表示的算符, 我们可以通过给出这组算符与生成元$\,\mathbf{A}\,$和$\,\mathbf{B}\,$的对易关系来指定这个表示, 其中$\,\mathbf{A}\,$和$\,\mathbf{B}\,$定义成
\begin{equation}
\mathbf{A}\equiv\tfrac{1}{2}\Bigl(\mathbf{J}+\mi\mathbf{K}\Bigr)\:,\qquad\qquad
\mathbf{B}\equiv\tfrac{1}{2}\Bigl(\mathbf{J}-\mi\mathbf{K}\Bigr)\:, \label{25.2.1}
\end{equation}
其中$\,\mathbf{J}\,$和$\,\mathbf{K}\,$分别是旋转和增速(boost)的厄米生成元. $\mathbf{A}\,$和$\,\mathbf{B}\,$满足对易关系
\begin{equation}
[A_{i},A_{j}]=\sum_{k}\epsilon_{ijk}A_{k}\:,\qquad  [B_{i},B_{j}]=\sum_{k}\epsilon_{ijk}B_{k}\:,\qquad
[A_{i},B_{j}]=0\:, \label{25.2.2}
\end{equation}
其中$\,i,j,k\,$取遍值$\,1,2,3$, $\epsilon_{ijk}$全反对称, $\epsilon_{123}=+1$. 因此, 就像带有两个独立自旋的态一样 齐次\,Lorentz\,群的表示由一对整数或半整数$\,A\,$和$\,B\,$标记, 而表示中的元素由一对指标$\,a,b\,$标记, 它们以\,1\,为步长分别从$\,-A\,$取到$\,A\,$以及从$\,-B\,$取到$\,B$. 更确切一些, 一组总数为$\,(2A+1)(2B+1)\,$的算符$\,Q_{ab}^{AB}\,$构成了齐次\,Lorentz\,群的一个$\,(A,B)\,$表示, 它们满足对易关系
\begin{equation}
[\mathbf{A},Q_{ab}^{AB}]=-\sum_{a^{\prime}}\mathbf{J}_{aa'}^{(A)}Q_{a'b}^{AB}\:,\qquad \qquad
[\mathbf{B},Q_{ab}^{AB}]=-\sum_{b^{\prime}}\mathbf{J}_{bb'}^{(B)}Q_{ab'}^{AB}\:, \label{25.2.3}
\end{equation}
其中$\,\mathbf{J}^{(j)}\,$是角动量为$\,j\,$的自旋\,3\,-矢矩阵:
\begin{equation}
\Bigl(J_{1}^{(j)}\pm\mi J_{2}^{(j)}\Bigr)_{\sigma'\sigma}=\delta_{\sigma',\sigma\pm1}\sqrt{(j\mp\sigma)(j\pm\sigma+1)}\:, \qquad
\Bigl(J_{3}^{(j)}\Bigr)_{\sigma'\sigma}=\delta_{\sigma'\sigma}\sigma\:.\label{25.2.4}
\end{equation}
从方程(\ref{25.2.4})得出\footnote{我们用星号来表示算符的厄米共轭或者数的复共轭. 对于那些由算符的厄米共轭或者数的复共轭构成的矩阵, 我们用剑号$\,\dag\,$来标记它的转置.}
\begin{equation}
-\Bigl(\mathbf{J}^{(j)}\Bigr)^{\ast}_{\sigma',\sigma}
=(-1)^{\sigma'-\sigma}\Bigl(\mathbf{J}^{(j)}\Bigr)_{-\sigma',-\sigma}\:.\label{25.2.5}
\end{equation}
因此, 如果$\,Q_{\sigma}^{j}\,$是一组按照旋转群的自旋$\,j\,$表示进行变换的算符, 那么$\,(-1)^{j-\sigma}Q_{-\sigma}^{j\ast}\,$也是. 另外, 方程(\ref{25.2.1})表明$\,\mathbf{A}^{\ast}=\mathbf{B}$. 通过对方程(\ref{25.2.3})取厄米共轭, 我们看到, 按照齐次Lorentz群的$\,(A,B)\,$表示进行变换的算符的厄米共轭$\,Q_{ab}^{AB\ast}\,$与按照$\,(B,A)\,$表示进行变换的算符%
$\,\bar{Q}_{ba}^{BA}$, 它们通过一个相似变换彼此关联:
\begin{equation}
Q_{ab}^{AB\ast}=(-1)^{A-a}(-1)^{B-b}\bar{Q}^{BA}_{-b,-a}\:.\label{25.2.6}
\end{equation}

Haag-Lopuszanski-Sohnius\,定理\cite{1}的部分表述是, 费米对称性生成元只能属于$\,(0,1/2)\,$表示和 $(1/2,0)\,$表示. 我们已经看到\,$(0,1/2)\,$算符或$\,(1/2,0)\,$算符的厄米共轭分别是$\,(1/2,0)\,$算符或$\,(0,1/2)$ 算符的线性组合, 因此, 费米对称性算符的完备集可以分成$\,(0,1/2)\,$生成元$\,\mathcal{Q}_{ar}\,$(省略了下标$\,0\tfrac{1}{2}\,$)%
和它们的$\,(1/2,0)\,$厄米共轭$\,\mathcal{Q}^{\ast}_{ar}$, 其中$\,a\,$是取值$\,\pm1/2\,$的旋量指标, $r\,$用来区分\,Lorentz\,变换性质相同的不同\,2\,分量生成元.\footnote{为了与本节后面将要引入的%
\,4\,分量\,Dirac\,旋量区分, 取代斜体字母, 我们用手写体$\,\mathcal{Q}_{ar}\,$(原书用的是罗马体字母, 因为与斜体字母比较像, 译本中改用手写体.------译者注)来标记\,Weyl\,旋量. 还有一种\,van der Waerden\,提倡的符号约定, 根据这个符号约定, 在写$\,\mathcal{Q}\,$这样的$\,(0,1/2)\,$算符时指标要加点, 例如$\,\mathcal{Q}_{\dot{a}}$, 而$\,(1/2,0)\,$算符的指标不加点. 在这里我们不使用这个符号约定, 相反我们将清楚地指明那些\,2\,分量旋量按照齐次\,Lorentz\,群的$\,(0,1/2)\,$表示变换, 那些按照$\,(1/2,0)\,$表示变换.} 这个定理进一步表述了, 可以定义费米生成元使它们满足反对易关系
\begin{align}
\{\mathcal{Q}_{ar},\mathcal{Q}_{bs}^{\ast}\}&=2\delta_{rs}\,\sigma_{ab}^{\mu}\,P_{\mu}\:,\label{25.2.7} \\
\{\mathcal{Q}_{ar},\mathcal{Q}_{bs}\}&=e_{ab}\,Z_{rs}\:,\label{25.2.8}
\end{align}
其中$\,P_{\mu}\,$是4-动量算符, $Z_{rs}=-Z_{sr}\,$是玻色对称性生成元, $\sigma_{\mu}\,$和$\,e\,$是$\,2\times2\,$矩阵(行与列用$\,+1/2,-1/2$ 标记):
\begin{equation}
\begin{split}
\sigma_{1}&= \begin{pmatrix}
0 & 1 \\ 1 & 0
\end{pmatrix} \:, \qquad
\sigma_{2}= \begin{pmatrix}
0 & -\mi \\ \mi & 0
\end{pmatrix}\:, \qquad
\sigma_{3}= \begin{pmatrix}
1 & 0 \\ 0 & -1
\end{pmatrix} \:, \\
\sigma_{0}&=\begin{pmatrix}
1 & 0 \\ 0 & 1
\end{pmatrix}\:, \qquad
e=\begin{pmatrix}
0 & 1 \\ -1 & 0
\end{pmatrix}\:.
\end{split} \label{25.2.9}
\end{equation}
最后, 费米对称性生成元与能量和动量对易:
\begin{equation}
[P_{\mu},\mathcal{Q}_{ar}]=[P_{\mu},\mathcal{Q}_{ar}^{\ast}]=0\:,\label{25.2.10}
\end{equation}
而$\,Z_{rs}\,$和$\,Z_{rs}^{\ast}\,$是这个代数的一组中心荷, 也就是说
\begin{align}
0&=[Z_{rs},\mathcal{Q}_{at}]=[Z_{rs},\mathcal{Q}^{\ast}_{at}]=[Z_{rs},Z_{tu}]=[Z_{rs},Z_{tu}^{\ast}]\nonumber\\
&=[Z_{rs}^{\ast},\mathcal{Q}_{at}]=[Z^{\ast}_{rs},\mathcal{Q}^{\ast}_{at}]=[Z^{\ast}_{rs},Z^{\ast}_{tu}]\:.
\label{25.2.11}
\end{align}


为了证明这些结果, 我们先来考察非零费米对称性生成元, 并要求它属于齐次\,Lorentz\,群的某个$\,(A,B)\,$不可约表示的, 这样它就可以记做$\,Q_{ab}^{AB}$, 其中$\,a\,$和$\,b\,$以\,1\,为步长分别从$\,-A\,$取到$\,+A\,$以及从$\,-B\,$取到$\,B$. 正如前面提到的, 厄米共轭通过方程(\ref{25.2.6})与$\,(B,A)\,$表示下的算符相关联, 所以这些算符的反对易子必须采取如下的形式
\begin{align}
\{Q_{ab}^{AB},Q_{a'b'}^{AB\ast}\}&=(-1)^{A-a'}(-1)^{B-b'}\sum_{C=\lvert A-B\rvert}^{A+B}
\sum_{D=\lvert A-B\rvert}^{A+B}\sum_{c=-C}^{C}\sum_{d=-D}^{D}\nonumber\\
&\quad\times C_{AB}(Cc;a,-b')\,C_{AB}(Dd;-a'b)\,X_{cd}^{CD}\:, \label{25.2.12}
\end{align}
其中$\,C_{AB}(j\sigma;ab)\,$是通常的\,Clebsch-Gordan\,系数, 它耦合了自旋$\,A\,$和自旋$\,B\,$以形成自旋$\,j$, $X_{cd}^{CD}\,$ 是按照齐次\,Lorentz\,群的$\,(C,D)\,$表示变换的算符的\,$(c,d)\,$-分量. 利用\,Clebsch-Gordan\,系数从所周知的幺正性, 我们可以将算符$\,X_{cd}^{CD}\,$表示成这些反对易子:
\begin{align}
X_{cd}^{CD}&=\sum_{a=-A}^{A}\sum_{b=-B}^{B}\sum_{a'=-A}^{A}\sum_{b'=-B}^{B}(-1)^{A-a'}(-1)^{B-b'}\nonumber\\
&\quad\times C_{AB}(Cc;a,-b')\,C_{AB}(Dd;-a'b)\,\{Q_{ab}^{AB},Q_{a'b'}^{AB\ast}\}\:. \label{25.2.13}
\end{align}
这些算符不一定都非零. 但是, 当$\,j=\sigma=A+B\,$和$\,j=-\sigma=A+B\,$时, 不为零的\,Clebsch-Gordan\,系数$\,C_{AB}(j\sigma,ab)\,$分别只有那些$\,a=A$, $b=B\,$的和$\,a=-A$, $b=-B\,$的, 而这些系数的值均为\,1, 所以通过在方程中取$\,C=D=c=-d=A+B$, 我们发现
\begin{equation}
X_{A+B,-A-B}^{A+B,A+B}=(-1)^{2B}\,\{Q_{A,-B}^{AB},Q_{A,-B}^{AB\ast}\}\:. \label{25.2.14}
\end{equation}
除非$\,Q_{A,-B}^{AB}=0$, 否则这不可能为零, 而$\,Q_{A,-B}^{AB}=0\,$(通过取$\,Q_{A,-B}^{AB}\,$与``下降''算符$\,A_{1}-\mi A_{2}$%
和``上升''算符$\,B_{1}+\mi B_{2}\,$的对易子)又暗示了所有$\,Q_{ab}^{AB}\,$为零. 因此, 如果存在任何不为零的$\,(A,B)\,$费米生成元, 那么它们与共轭的反对易子至少必须要包含属于表示$\,(A+B,A+B)\,$的非零玻色对称性生成元.

现在, Coleman-Mandula\,定理告诉我们, 组成玻色对称性生成元的是平移的$\,(1/2,1/2)\,$生成元$\,P_{\mu}$, 固有\,Lorentz\,变换的$\,(1,0)+(0,1)\,$生成元$\,J_{\mu\nu}$, 以及可有可无的内部对称性的$\,(0,0)\,$生成元$\,T_{A}$. (回顾一下, $N\,$阶对称无迹张量按照表示$\,(N/2,N/2)\,$变换, 2\,阶反对称张量按照表示$\,(1,0)+(0,1)\,$变换, 而\,Dirac\,场按照表示$\,(1/2,0)+(0,1/2)\,$变换.) 因此费米对称性生成元只能属于$\,A+B\leq 1/2\,$的 $(A,B)\,$表示. 这些算符将玻色子变成费米子并将费米子变成玻色子, 所以它们不能是标量, 这样只剩下了$\,(1/2,0)\,$表示和$\,(0,1/2)\,$表示, 正是所要证明的. 用$\,\mathcal{Q}_{ar}\,$标记线性独立的$\,(0,1/2)\,$费米生成元, 反对易子$\,\{\mathcal{Q}_{ar},\mathcal{Q}_{bs}^{\ast}\}\,$属于表示$\,(0,1/2)\times(1/2,0)=(1/2,1/2)$, 因此它只能正比于$\,(1/2,1/2)\,$玻色对称性生成元, 即动量\,4\,-矢$\,P_{\mu}$. Lorentz\,不变性表明这个关系的形式必须是
\begin{equation}
\{\mathcal{Q}_{ar},\mathcal{Q}^{\ast}_{bs}\} = 2N_{rs}\,\sigma_{ab}^{\mu}\,P_{\mu} \:, \label{25.2.15}
\end{equation}
其中$\,N_{rs}\,$是数值矩阵.

为了看到这点, 我们使用\,2.7\,节讨论的\,Lorentz\,群(或者更准确些, 它的覆盖群)与二维幺模复矩阵群$\,SL(2,C)\,$之间的同构. Lorentz\,变换$\,\Lambda\indices{^\mu_\nu}\,$在$\,(0,1/2)\,$费米生成元上的作用效果是
\begin{equation}
U^{-1}(\Lambda)\,\mathcal{Q}_{ar}\,U(\Lambda) = \sum_{b}\lambda_{ab}\,\mathcal{Q}_{br} \:, \label{25.2.16}
\end{equation}
其中$\,\Lambda\,$是
\begin{equation}
\lambda\, \sigma_{\mu} \,\lambda^{\dag} =\Lambda\indices{_\nu^\mu}\sigma_{\nu}  \label{25.2.17}
\end{equation}
定义的\,Lorentz\,变换. 我们可以验证方程(\ref{25.2.16})对$\,(0,1/2)\,$算符是成立的, 方法是, 注意到对于无限小\,Lorentz\,变换$\,\Lambda\indices{^\mu_\nu}=\delta\indices{^\mu_\nu}+\omega\indices{^\mu_\nu}$, 其中$\,\omega_{\mu\nu}=\omega_{\nu\mu}$, 方程(\ref{25.2.17})对
\[
\lambda=1+\tfrac{1}{2}\Bigl[\tfrac{1}{2}\,\mi\,\epsilon_{ijk}\omega_{ij}+\omega_{k0}\Bigr]\sigma_{k}
\]
是满足的, 而此时\footnote{这里的$\,K_{i}\,$定义成$\,J_{i0}$. 在卷\,I\,的前两次印刷中有一个错误: $K_{i}\,$在\,2.4\,节, 3.3\, 节和\,3.5\,节定义成了$\,J^{i0}$, 而在\,5.6\,节和\,5.9\,节定义成了$\,J_{i0}$, 而$\,\mathbf{A}\,$和$\,\mathbf{B}\,$自始至终由方程(\ref{25.2.1})给定.}
\[
U(\Lambda)=1+\tfrac{1}{2}\,\mi\,\omega_{\mu\nu}J^{\mu\nu}=1 + \tfrac{1}{2}\,\mi\,\epsilon_{ijk}\omega_{ij}J_{k}-\mi\,\omega_{i0}K_{i} \:.
\]
(重复拉丁指标$\,i,j,k\,$对值$\,1,2,3\,$求和.) 在这一情况下, 通过$\,\omega_{ij}\,$和$\,\omega_{i0}\,$在方程(\ref{25.2.16})两边的系数相等, 我们发现
\[
[\mathbf{J},\mathcal{Q}_{a}]=-\tfrac{1}{2}\sum_{b}\bm{\sigma}_{ab}\,\mathcal{Q}_{b} \:, \qquad
[\mathbf{K},\mathcal{Q}_{a}]=-\tfrac{1}{2}\,\mi\sum_{b} \bm{\sigma}_{ab}\,\mathcal{Q}_{b} \:,
\]
或者等价的
\[
[\mathbf{B},\mathcal{Q}_{a}]=-\tfrac{1}{2}\sum_{b}\bm{\sigma}_{ab}\,\mathcal{Q}_{b} \:, \qquad
[\mathbf{A},\mathcal{Q}_{a}]=0 \:,
\]
这表明满足方程(\ref{25.2.16})的算符属于$\,(0,1/2)\,$表示. 现在, $\sigma_{\mu}\,$构成了$\,2\times2\,$矩阵的一个完备集, 所以我们可以将反对易子$\,\{\mathcal{Q}_{ar},\mathcal{Q}_{bs}^{\ast}\}\,$%
写成$\,N_{rs}^{\mu}\,(\sigma_{\mu})_{ab}\,$的形式, 其中$\,N^{\mu}\,$是算符的某个矩阵. 方程(\ref{25.2.16}) 和(\ref{25.2.17})表明这些算符是\,4\,-矢, 也就是说$\,U^{-1}(\Lambda)N^{\mu}U(\Lambda)=\Lambda\indices{^\mu_\nu}\,N^{\nu}$, 那么根据 Coleman-Mandula 定理它们只能正比于玻色对称性算符中的唯一\,4\,-矢, $P^{\mu}$. 令$\,N^{\mu}_{rs}=2P^{\mu}N_{rs}\,$就给出了方程(\ref{25.2.15}).

现在我们要对$\,\mathcal{Q}_{ar}\,$做一个线性变换使得它们的反对易子是(\ref{25.2.7})的形式. 为了这个目的, 我们需要构建厄米且正定的矩阵$\,N_{rs}$. 通过取方程(\ref{25.2.15})的厄米共轭我们可以立刻得出\,$N_{rs}\,$是厄米的. 为了看到它是正定的, 回忆\,$\mathcal{Q}_{ar}\,$是被取成线性独立的, 所以对于任何非零的线性组合$\,\mathcal{Q}\equiv\sum_{r}d_{a}\,c_{r}\,\mathcal{Q}_{ar}$, 必存在某个$\,\mathcal{Q}\,$湮灭不了的态$\,\lvert\Psi\rangle$. 取方程(\ref{25.2.15})在这个态上的期望值, 这给出
\[
2\langle\Psi\vert\sum_{ab}\sigma_{ab}^{\mu}P_{\mu}d_{a}d_{b}^{\ast}\vert\Psi\rangle\sum_{rs}c_{r}c_{s}^{\ast}N_{rs}
=\langle\Psi\{\mathcal{Q},\mathcal{Q}^{\ast}\}\vert\Phi\rangle >0 \:.
\]
由此可以立刻得出, 对于任何不全为零的$\,c_{r}$, $\sum_{rs}\,c_{r}c_{s}^{\ast}N_{rs}\,$必不为零, 所以$\,N_{rs}\,$不是正定的就是负定的. 在$\,-P^{\mu}P_{\mu}\geq0\,$且$\,P^{0}>0\,$的物理态的空间上, 算符$\,\sum_{ab}(\sigma_{\mu})_{ab}P^{\mu}d_{a}d_{b}^{\ast}\,$是正定的, 所以矩阵$\,N^{rs}\,$也必须是正定的.\footnote{这个论证也可以反过来. 假定一个$\,N_{rs}\,$正定的超对称性, 就像方程(\ref{25.2.7})中那样, 我们可以推出所有态的$\,P^{0}>0$.\cite{2} 然而, 当引力也考虑在内时, 这个结论是不成立的, 一个例外情况是, 引力被当成了所有态中能量的一个偏移, 并且这个偏移都一样以至于没有任何物理效应.}

现在我们可以定义新的费米生成元
\[
\mathcal{Q}_{ar}^{\prime}\equiv \sum_{s} N_{rs}^{-1/2}\mathcal{Q}_{as} \:,
\]
使得反对易子取如下的形式
\[
\{\mathcal{Q}_{ar}^{\prime},\mathcal{Q}_{bs}^{\prime\ast}\}=2\delta_{rs}\,\sigma_{ab}^{\mu}\,P_{\mu}\:.
\]
从现在起, 我们将假定所有费米生成元都以这种方式定义并去掉撇号, 使得方程(\ref{25.2.7})成立.


接下来我们必须要证明\,$\mathcal{Q}_{ar}\,$与动量\,4\,-矢$\,P_{\mu}\,$对易. $P_{\mu}\,$这样的$\,(1/2,1/2)$ 算符与$\,\mathcal{Q}\,$这样的$\,(0,1/2)\,$算符, 它们的对易子只能是$\,(1/2,0)\,$算符或$\,(1/2,1)\,$算符, 但是我们看到不存在$\,(1/2,1)\,$对称性生成元, 所以$\,P_{\mu}\,$与$\,\mathcal{Q}\,$的对易子只能正比于$\,(1/2,0)\,$对称性生成元$\,\mathcal{Q}^{\ast}$. Lorentz\,不变性要求这个关系取如下的形式
\begin{equation}
    [\mathscr{M}_{ab},\mathcal{Q}_{cr}]=\sum_{s}e_{ac}\,K_{rs}\,\mathcal{Q}_{bs}^{\ast}\:,\label{25.2.18}
\end{equation}
其中$\,K\,$是一数值矩阵, $\mathscr{M}\,$是算符矩阵
\begin{equation}
    \mathscr{M}\equiv\sigma_{\mu}P^{\mu}\:. \label{25.2.19}
\end{equation}
(矩阵$\,e_{ac}\,$是将两个自旋$\,1/2\,$耦合成零自旋的\,Clebsch-Gordan\,系数.) 由此可以直接得出
\begin{equation}
    [\mathscr{M}_{{-}\frac{1}{2}{-}\frac{1}{2}},[\mathscr{M}_{{-}\frac{1}{2}{-}\frac{1}{2}},
    \{\mathcal{Q}_{\frac{1}{2}r},\mathcal{Q}_{\frac{1}{2}s}^{\ast}\}]]
    =-4(\mathscr{M})_{{-}\frac{1}{2}{-}\frac{1}{2}}(KK^{\dag})_{rs}\:.\label{25.2.20}
\end{equation}
利用方程(\ref{25.2.7}), 左边是多重对易子$\,[P_{\mu},[P_{\nu},P_{\lambda}]]\,$的线性组合, 所有这样的对易子都为零, 而 $\mathscr{M}_{{-}1/2{-}1/2}$ 对于一般的动量不为零, 所以$\,K^{\dag}K=0$, 因此$\,K=0$, 加上方程(\ref{25.2.18}), 这表明\,$[P_{\mu},\mathcal{Q}_{ar}]=0$. 复共轭给出$\,[P_{\mu},\mathcal{Q}_{ar}^{\ast}]=0$.

现在我们可以着手处理两个$\,\mathcal{Q}\,$的反对易子. 两个$\,(0,1/2)\,$对称性算符的反对易子必须是$\,(0,1)$ 对称性生成元和$\,(0,0)\,$对称性生成元的线性组合. Coleman-Mandula\,定理告诉我们唯一的$\,(0,1)\,$对称性生成元是固有齐次\,Lorentz\,变换的%
生成元$\,J_{\nu\lambda}\,$的线性组合, 但由于$\,\mathcal{Q}\,$与$\,P_{\mu}$对易, 继而它们的反对易子也与$\,P_{\mu}\,$对易, 而方程(\textcolor{foo}{2.4.13})告诉我们$\,J_{\nu\lambda}\,$的线性组合与$\,P_{\mu}\,$不对易. 这样就只剩下了$\,(0,0)$ 算符, 它既与$\,P_{\mu}\,$对易又与$\,J_{\nu\lambda}\,$对易. 这样, Lorentz\,不变性就会要求$\,\mathcal{Q}\,$之间的反对易子必须采取方程(\ref{25.2.8})的形式. 内部对称性生成元$\,Z_{rs}\,$关于$\,r\,$和$\,s\,$是反对称的, 这是因为整个表达式必须在$\,r\,$与 $s\,$和$\,a\,$与$\,b\,$的交换下是对称的, 而矩阵$\,e_{ab}\,$关于$\,a\,$和$\,b\,$是反对称的.

剩下来要证明的是$\,Z\,$是中心荷. 从方程(\ref{25.2.8})和(\ref{25.2.10})立即可以得出
\begin{equation}
    [P_{\mu},Z_{rs}]=0 \:. \label{25.2.21}
\end{equation}
接下来考察包含两个$\,\mathcal{Q}\,$和一个$\,\mathcal{Q}^{\ast}\,$的推广\,Jacobi\,恒等式(\ref{25.1.5}):
\[
0=[\{\mathcal{Q}_{ar},\mathcal{Q}_{bs}\},\mathcal{Q}_{ct}^{\ast}]
+[\{\mathcal{Q}_{bs},\mathcal{Q}_{ct}^{\ast}\},\mathcal{Q}_{ar}]
+[\{\mathcal{Q}_{ct}^{\ast},\mathcal{Q}_{ar}\},\mathcal{Q}_{bs}] \:.
\]
方程(\ref{25.2.7})和(\ref{25.2.10})表明第二项和第三项为零, 所以
\begin{equation}
    [Z_{rs},\mathcal{Q}_{ct}^{\ast}]=0\:. \label{25.2.22}
\end{equation}
最后, 考察一个$\,Z$, 一个$\,\mathcal{Q}\,$和一个$\,\mathcal{Q}^{\ast}\,$的推广\,Jacobi\,恒等式:
\[
0=-[Z_{rs},\{\mathcal{Q}_{at},\mathcal{Q}_{bu}^{\ast}\}]
+\{\mathcal{Q}_{bu}^{\ast},[Z_{rs},\mathcal{Q}_{at}]\}
-\{\mathcal{Q}_{at},[\mathcal{Q}_{bu}^{\ast},Z_{rs}]\} \:.
\]
第一项和第三项分别因为方程(\ref{25.2.21})和(\ref{25.2.22})为零, 所以我们只剩下了第二项
\begin{equation}
    \{\mathcal{Q}_{bu}^{\ast},[Z_{rs},\mathcal{Q}_{at}]\} =0 \:. \label{25.2.23}
\end{equation}
现在, $[Z_{rs},\mathcal{Q}_{at}]\,$是$\,(0,1/2)\,$对称性生成元, 所以它必须是$\,\mathcal{Q}\,$的线性组合:
\begin{equation}
    [Z_{rs},\mathcal{Q}_{at}]=\sum_{u}M_{rstu}\,\mathcal{Q}_{au} \:. \label{25.2.24}
\end{equation}
那么对于所有$a$, $b$, $r$, $s$, $t\,$和$\,u$, 方程(\ref{25.2.23})就变成
\[
\sigma_{ab}^{\mu}P_{\mu}M_{rstu}=0\:.
\]
由于算符$\,\sigma_{ab}^{\mu}P_{\mu}\,$不为零, 我们得出$\,M_{rstu}=0$, 这使得
\begin{equation}
    [Z_{rs},\mathcal{Q}_{at}]=0 \:. \label{25.2.25}
\end{equation}
利用反对易关系(\ref{25.2.8})和它的共轭, 再加上对易关系(\ref{25.2.22})和(\ref{25.2.25})与它们的的共轭, 这给出
\begin{equation}
    [Z_{rs},Z_{tu}]=[Z_{rs},Z_{tu}^{\ast}] = [Z_{rs}^{\ast},Z_{tu}^{\ast}] =0\:, \label{25.2.26}
\end{equation}
这完成了方程(\ref{25.2.11})的证明, 有了这个也就证明了\,Hagg-Lopuszanski-Sohnius\,定理.

当然, $Z_{rs}\,$是超对称代数的中心荷这一点并不会排除还存在{\kai{其它}}阿贝尔或非阿贝尔内部对称性的可能性. 设$\,T_{A}\,$张开了玻色内部对称性的整个\,Lie\,代数. 那么$\,[T_{A},\mathcal{Q}_{ar}]\,$就是$\,(0,1/2)\,$算符, 所以它必须是$\,\mathcal{Q}\,$的线性组合:
\begin{equation}
    [T_{A},\mathcal{Q}_{ar}]=-\sum_{s}(t_{A})_{rs}\mathcal{Q}_{as} \label{25.2.27}
\end{equation}
从两个$\,T\,$和一个$\,\mathcal{Q}\,$的\,Jacobi\,恒等式, 我们可以得知$\,t_{A}\,$矩阵构成了内部对对称性代数的一个表示
\begin{equation}
    [t_{A},t_{B}] = \mi \sum_{C}C_{AB}^{C}\,t_{C} \:, \label{25.2.28}
\end{equation}
其中系数$\,C_{AB}^{C}\,$是内部对称性代数的结构常数
\begin{equation}
    [T_{A},T_{B}] = \mi \sum_{C}C_{AB}^{C}\,T_{C} \:. \label{25.2.29}
\end{equation}
这样, $Z_{rs}\,$不仅是$\,\mathcal{Q}$, $\mathcal{Q}^{\ast}$, $P_{\mu}$, $Z\,$和$\,Z^{\ast}\,$构成的超代数的中心荷, 同时还是包含所有$\,T_{A}\,$的更大的超代数的中心荷. 为了看到这点, 从方程(\ref{25.2.27})和(\ref{25.2.8})中注意到
\[
    [T_{A},Z_{rs}] = -\sum_{r^{\prime}}(t_{A})_{rr^{\prime}} Z_{r^{\prime}s}
    -\sum_{s^{\prime}}(t_{A})_{ss^{\prime}}Z_{rs^{\prime}} \:,
\]
所以$\,Z_{rs}\,$构成了整个玻色对称性代数的一个{\kai{不变}}阿贝尔子代数. 但是回顾\,Coleman-Mandula\,定理的证明, 我们发现内部玻色对称性的整个\,Lie\,代数, 在这里就是$\,T_{A}\,$张开的\,Lie\,代数, 它同构于一个紧致半单\,Lie\,代数和几个\,$U(1)$\,代数的直和. 这种\,Lie\,代数的不变阿贝尔子代数只有那些$\,U(1)\,$生成元张开的, 所以$\,Z_{rs}\,$必须是$\,U(1)\,$生成元, 因而与所有$\,T_{A}\,$对易.

即使$\,Z\,$与所有对称性算符都对易, 它们也不只是个数; 它们是量子算符, 它们的值可能随着态的变化而变化. 事实上, 对于超对称真空态, 由于它被所有超对称性生成元湮灭, $Z\,$显然必须在这个态上取零值, 但是一般而言它们不需要为零. 在\,27.9\,节, 我们将看到如何在有扩充超对称性的规范理论中计算$\,Z$.

在没有中心荷的情况下, 超对称代数(\ref{25.2.7}), (\ref{25.2.8})在内部对称群$\,U(N)\,$下不变
\begin{equation}
    \mathcal{Q}_{ar}\to \sum_{s}V_{rs}\mathcal{Q}_{as} \:, \label{25.2.30}
\end{equation}
其中$\,V_{rs}\,$是$\,N\times N\,$幺正(不一定幺模)矩阵. 这被称为$\,R\,$-{\kai{对称性}}. 这个对称性可能是也可能不是一个好对称性, 如果它是, 那么它可能被反常破坏也可能自发破缺, 或者它就是自然的一个好对称性.

$r,s\,$等指标的取值$\,N>1\,$的超对称代数被称为$\,N\,$-{\kai{扩充超对称性}}. 当只有一个$\,\mathcal{Q}\,$时, $Z_{rs}=-Z_{sr}\,$的条件告诉我们$\,Z\,$为零, 这给出了反对易关系的一个更加简单的形式
\begin{align}
    \{\mathcal{Q}_{a},\mathcal{Q}_{b}^{\ast}\} &= 2\sigma_{ab}^{\mu}P_{\mu}\:,\label{25.2.31} \\
    \{\mathcal{Q}_{a},\mathcal{Q}_{b}\} &= 0 \:. \label{25.2.32}
\end{align}
这种情况被称为{\kai{简单超对称}}, 或者$\,N=1\,$超对称. 在这一情况下, $R\,$-对称变换是\,$U(1)\,$相位变换
\begin{equation}
    \mathcal{Q}_{a}\to \exp(\mi\varphi)\, \mathcal{Q}_{a} \:, \label{25.2.33}
\end{equation}
其中$\,\varphi\,$是一个实相位.

为了多个目的, 将$\,(0,1/2)\,$算符$\,\mathcal{Q}_{ar}\,$与$\,(1,2)\,$算符相结合写成\,4\,分量\,Majorana\,旋量%
生成元$\,Q_{ar}$ 将是方便的, 其中$\,(1,2)\,$算符根据方程(\ref{25.2.6})可以取成$\,e_{ab}\mathcal{Q}_{br}^{\ast}$, 那么$\,Q\,$定义成
\begin{equation}
    Q_{r}\equiv
    \begin{pmatrix}
    e\mathcal{Q}_{r}^{\ast} \\ \mathcal{Q}_{r}
    \end{pmatrix} \:, \label{25.2.34}
\end{equation}
或者更明显些
\[
Q_{1r}=\mathcal{Q}^{\ast}_{-\frac{1}{2}\,r} \:, \quad
Q_{2r}=-\mathcal{Q}^{\ast}_{\frac{1}{2}\,r} \:, \quad
Q_{3r}=\mathcal{Q}_{\frac{1}{2}\,r} \:, \quad
Q_{4r}=\mathcal{Q}_{-\frac{1}{2}\,r} \:. \quad
\]
这是一个\,Majorana\,旋量, 也就是说
\[
Q_{r}=\beta \epsilon \gamma_{5} Q_{r}^{\ast} \:,
\]
其中$\,\beta$, $\epsilon\,$和$\,\gamma_{5}\,$是$\,4\times4\,$矩阵, 它们可以写成\,$2\times2$分块矩阵:
\[
\beta = \begin{pmatrix}
0 & 1 \\ 1 & 0
\end{pmatrix} \qquad
\epsilon = \begin{pmatrix}
e & 0 \\ 0 & e
\end{pmatrix} \qquad
\gamma_{5} = \begin{pmatrix}
1 & 0 \\ 0 & -1
\end{pmatrix} \:.
\]
(第\,26\,章的附录将会回顾\,Majorana\,旋量的性质.) 选择(\ref{25.2.34})的形式是为了与齐次\,Lorentz\,群通常的\,4\,分量\,Dirac\,表示的记法一致, 根据方程(\textcolor{foo}{5.4.4}), 在这个表示中旋转和增速生成元按照方程(\textcolor{foo}{5.4.19})和(\textcolor{foo}{5.4.20})被表示成
\begin{equation}
    \mathscr{J}_{i} = \frac{1}{2}
    \begin{bmatrix}
    \sigma_{i} & 0 \\ 0 & \sigma_{i}
    \end{bmatrix} \:,  \qquad
     \mathscr{K}_{i} = -\frac{\mi}{2}
    \begin{bmatrix}
    \sigma_{i} & 0 \\ 0 & -\sigma_{i}
    \end{bmatrix} \:.
\end{equation}
再加上方程(\ref{25.2.1}), 这表明$\,\mathbf{A}\,$和$\,\mathbf{B}\,$分别只作用在\,Dirac\,旋量的前两个分量和后两个分量上, 这就是为什么我们将$\,(0,1/2)\,$算符$\,\mathcal{Q}_{ar}\,$用作方程(\ref{25.2.34})的下分量而不是上分量

在这个\,4\,分量记法中, 简单超对称的基础反对易关系(\ref{25.2.31})和(\ref{25.2.32})写成
\begin{equation}
    \{Q,\overline{Q}\} = 2
    \begin{pmatrix}
    0 & -e\,(\sigma_{\mu}P^{\mu})^{\mathrm{T}}\,e \\
    \sigma_{\mu} P^{\mu} & 0
    \end{pmatrix}
    =-2\mi\,P_{\mu}\gamma^{\mu} \:. \label{25.2.36}
\end{equation}
本卷的前言部分回顾了我们使用了\,Dirac\,矩阵的约定; 这里我们仅需要记起
\begin{equation}
    \gamma^{0} = -\mi\beta = -\mi
    \begin{pmatrix}
    0 & \sigma_{0} \\ \sigma_{0} & 0
    \end{pmatrix} \:, \qquad
    \bm{\gamma} = -\mi
    \begin{pmatrix}
    0 & \bm{\sigma} \\ -\bm{\sigma} & 0
    \end{pmatrix}  \:, \label{25.2.37}
\end{equation}
以及$\,e\,\bm{\sigma}^{\mathrm{T}}\,e=\bm{\sigma}$, $e\sigma_{0}e=-\sigma_{0}$, 和通常的$\,\overline{Q}\equiv Q^{\dag}\beta$. 在扩充超对称的情况下, 中心荷的出现会改变这个公式; 取代方程(\ref{25.2.36}), 我们有
\begin{equation}
    \{Q_{r},\overline{Q}_{s}\} = -2\mi\,P_{\mu}\gamma^{\mu}\delta_{rs}
    +\biggl(\frac{1+\gamma_{5}}{2}\biggr) Z_{sr}^{\ast} + \biggl(\frac{1-\gamma_{5}}{2}\biggr) Z_{rs} \:.
    \label{25.2.38}
\end{equation}

这里给出的分析针对的是时空维数为\,4\,的情况, 在第\,32\,章, 我们将对一般的时空维数以一种不太显式的形式重复这个分析. 在那里我们将会看到, 在高维时空中, 即使理论中有扩充的量使得可以构造出\,Coleman-Mandula\,定理允许范围以外的玻色对称性生成元, 超对称性生成元也总是属于高维\,Lorentz\,群的基础旋量表示.


\subsection{* * *}

在无质量的理论中, 对于那些在共形对称性代数(\ref{24.B.34})---(\ref{24.B.35})下不变的理论, 存在两个额外的能够出现在超对称反对易关系右边的对称性生成元, $D\,$和$\,K_{\mu}$. 这些新生成元分别拥有标量和矢量的\,Lorentz\,变换性质, 就像$\,Z_{rs}\,$和$\,P_{\mu}$, 所以和前面一样, 费米生成元必须属于\,Lorentz\,代数的基础$\,(1/2,0)\,$旋量表示, 而它的共轭必须属于$\,(0,1/2)\,$表示. 同时, 根据所有这些元与伸缩生成元$\,D\,$的对易关系对它们进行分类是方便的; 如果一个算符$\,X\,$有
\begin{equation}
    [X,D] = \mi a X\:,\label{25.2.39}
\end{equation}
那就称它有量纲$\,a$. 对方程(\ref{24.B.34})的观察表明玻色对称性生成元$\,J^{\mu\nu}$, $P^{\mu}$, $K^{\mu}\,$和$\,D\,$分别拥有量纲$\,0$, $+1$, $-1\,$和$\,0$. 另外, 对于任何内部对称性的\,Lie\,群, 它的生成元的量纲为零. 量纲为$\,a\,$的费米生成元与它的共轭的反对易子是量纲为$\,2a\,$的正定玻色算符, 又因为正定玻色对称性生成元只能是$\,P_{\mu}\,$分量和$\,K_{\mu}\,$分量的线性组合, 费米对称性算符只能有量纲$\,+1/2\,$和$\,-1/2$. 量纲$\,1/2\,$的$\,(0,1/2)$ 费米对称性算符和它们的共轭可以再次被装配成\,Majorana\,旋量$\,Q_{r\alpha}$, 并满足
\begin{align}
    &\{Q_{r\alpha},\overline{Q}_{s\beta}\} = -2\mi P_{\mu}(\gamma^{\mu})_{\alpha\beta}\delta_{rs} \:, \label{25.2.40} \\
    &[P_{\mu},Q_{r\alpha}] = 0 \:, \label{25.2.41} \\
    &[D,Q_{r\alpha}] = -\tfrac{1}{2}\mi Q_{r\alpha} \:. \label{25.2.42}
\end{align}
(注意, 因为中心荷的量纲是$\,0\,$而不是$\,+1$, 这里是不允许存在中心荷的.) $K_{\mu}\,$与$\,Q_{r\alpha}\,$的对易子是 Majorana\,费米对称性生成元$\,Q_{r\alpha}^{\sharp}\,$的线性组合, Lorentz\,不变性使得我们可以将它写成如下形式
\begin{equation}
    [K^{\mu},Q_{r\alpha}] = \mi\,(\gamma^{\mu})_{\alpha\beta}Q_{r\beta}^{\sharp} \:. \label{25.2.43}
\end{equation}
(右边的一个任意因子已经被吸收进$\,Q_{r\beta}^{\sharp}\,$的定义中. 对右边的相位已经进行了选择, 使得$\,Q_{r\beta}^{\sharp}\,$满足\,Majorana\,旋量的标准实条件(\ref{26.A.2}).) $\,Q_{r\beta}^{\sharp}\,$的量纲是$\,+1/2-1=-1/2$, 所以
\begin{equation}
    [D,Q_{r\alpha}^{\sharp}] = + \tfrac{1}{2}\mi\,Q_{r\alpha}^{\sharp} \:. \label{25.2.44}
\end{equation}
取方程(\ref{25.2.43})与$\,P^{\nu}\,$的对易子并使用方程(\ref{24.B.34})给出的$\,K^{\mu}\,$和$\,P^{\nu}\,$的对易关系, 这给出
\begin{equation}
    [P^{\nu},Q_{r\alpha}^{\sharp}] = -\mi(\gamma^{\nu})_{\alpha\beta}Q_{r\beta} \:. \label{25.2.45}
\end{equation}
我们看到$\,Q\,$与$\,Q^{\sharp}\,$是相联系的. 通过取反对易关系(\ref{25.2.40})与$\,K_{\mu}\,$的对易子, 我们发现了$\,Q^{\sharp}\,$与$\,Q\,$的反对易子:
\begin{equation}
    \{Q_{r\alpha}^{\sharp},\overline{Q}_{s\beta}\} = 2\mi D\delta_{rs}\delta_{\alpha\beta}
    + 2J_{\mu\nu}\delta_{rs}\mathscr{J}_{\alpha\beta}^{\mu\nu}
    + O_{rs}\delta_{\alpha\beta} + O_{rs}^{\prime}(\gamma_{5})_{\alpha\beta} \:, \label{25.2.46}
\end{equation}
其中$\,\mathscr{J}^{\mu\nu}=-\mi[\gamma^{\mu},\gamma^{\nu}]/4$, $O_{rs}\,$和$\,O_{rs}^{\prime}\,$是量纲为零的\,Lorentz\,不变算符, 并满足
\begin{equation}
    O_{rs}=-O_{sr}\:, \qquad O_{rs}^{\prime} = + Q_{sr}^{\prime} \:. \label{25.2.47}
\end{equation}
取方程(\ref{25.2.43})与$\,K_{\nu}\,$的对易子并使用$\,[K_{\nu},K_{\mu}]=0$, 我们发现$\,(\gamma^{\mu})_{\alpha\beta}[K^{\nu},Q_{r\beta}]\,$关于$\,\mu\,$和$\,\nu\,$是对称的, 在经过一些计算, 这告诉我们
\begin{equation}
    [K^{\nu},Q_{r\beta}] = 0 \:. \label{25.2.48}
\end{equation}
另外, 取方程(\ref{25.2.46})与$\,K_{\nu}\,$的对易子给出
\begin{equation}
    \{Q_{r\alpha}^{\sharp},\overline{Q^{\sharp}}_{s\beta}\}
    = +2\mi K_{\mu}(\gamma^{\mu})_{\alpha\beta}\delta_{rs} \:. \label{25.2.49}
\end{equation}
最后, 取方程(\ref{25.2.46})与$\,Q_{t\gamma}\,$的对易子, 这会表明$\,O_{rs}\,$和$\,O_{rs}^{\prime}\,$的作用就像$\,R\,$-对称群$\,U(N)\,$的生成元, 而$\,Q_{r\alpha}\,$的左边和右边分别按照表示$\,\mathbf{N}\,$和$\,\bar{\mathbf{N}}\,$变换, $P_{\mu}$, $K_{\mu}\,$和$\,D\,$都是$\,U(N)\,$-不变量. 这些生成元彼此之间以及它们与其它生成元的$\,U(N)\,$对易关系, 再加上$\,J_{\mu\nu}\,$和$\,D\,$与各种生成元的对易子, 这些合起来构成了{\kai{超共形代数}}. 这个代数与普通的简单超对称或者$\,N\,$-扩充超对称之间的一个重要差异是, $U(N)\,$对称性不再只是超对称代数的一个外自同构, 那时它可以是也可以不是作用量的一个对称性------现在它是超共形代数的一部分, 因此它必须是任何共形不变的超对称理论的一个对称性.

\section{超对称性生成元的空间反演性质} \label{sec:25.3}

在遵循宇称守恒的理论中, 用宇称算符$\,\mathsf{P}\,$作用费米对称性算符$\,\mathcal{Q}_{ar}\,$得到的%
$\,\mathsf{P}^{-1}\mathcal{Q}_{ar}\mathsf{P}\,$也必须是个%
费米对称性算符. 由于$\,J_{i}\,$和$\,K_{i}\,$在空间反演分别为偶和奇, 方程(\ref{25.2.1})表明用宇称算符作用$\,A_{i}$ 的结果是
\begin{equation}
    \mathsf{P}^{-1}A_{i}\mathsf{P}=B_{i} \:. \label{25.3.1}
\end{equation}
根据方程(\ref{25.2.3}), 将$\mathcal{Q}_{ar}\,$定义为$\,(0,1/2)\,$算符意味着
\begin{equation}
    [B_{i},\mathcal{\mathcal{Q}}_{ar}]=-\tfrac{1}{2}\sum_{b}\bigl(\sigma_{i}\bigr)_{ab}\mathcal{Q}_{br}\:, \qquad
    [A_{i},\mathcal{Q}_{ar}]=0 \:. \label{25.3.2}
\end{equation}
使用宇称算符给出
\begin{equation}
    [A_{i},\mathsf{P}^{-1}\mathcal{Q}_{ar}\mathsf{P}]
    =-\tfrac{1}{2}\sum_{b}\bigl(\sigma_{i}\bigr)_{ab}\mathsf{P}^{-1}\mathcal{Q}_{br}\mathsf{P} \:, \qquad
    [B_{i},\mathsf{P}^{-1}\mathcal{Q}_{ar}\mathsf{P}] =0 \:, \label{25.3.3}
\end{equation}
所以$\,\mathsf{P}^{-1}\mathcal{Q}_{ar}\mathsf{P}\,$是$\,(1/2,0)\,$对称性算符, 因此必须是$\,\mathcal{Q}_{ar}^{\ast}\,$的线性组合. 根据方程(\ref{25.2.6}), Lorentz\,不变性指明这个关系的形式是
\begin{equation}
    \mathsf{P}^{-1}\mathcal{Q}_{ar}\mathsf{P}=\sum_{bs}\mathscr{P}_{rs}\,e_{ab}\mathcal{Q}_{bs}^{\ast} \:, \label{25.3.4}
\end{equation}
其中$\,\mathscr{P}\,$是数值矩阵, 而矩阵$\,e\,$由方程(\ref{25.2.9})给出.

通过要求方程(\ref{25.3.4})与基础反对易关系(\ref{25.2.7})自洽, 我们可以知道矩阵$\,\mathscr{P}\,$的一些性质. 方程(\ref{25.3.4})和它的共轭给出
\[
\mathsf{P}^{-1}\{\mathcal{Q}_{ar},\mathcal{Q}_{bs}^{\ast}\} \mathsf{P}
=\sum_{cdtu}\mathscr{P}_{rt}\,e_{ac}\,\mathscr{P}_{su}^{\ast}\,e_{bd}\,\{\mathcal{Q}_{ct}^{\ast},\mathcal{Q}_{du}\} \:.
\]
代入方程(\ref{25.2.7}), 这变成
\[
\delta_{rs}\sigma_{ab}^{\mu}\,\mathsf{P}^{-1}P_{\mu}\mathsf{P}
=\sum_{cdtu}\mathscr{P}_{rt}\,e_{ac}\,\mathscr{P}_{su}^{\ast}\,e_{bd}\,\delta_{tu}\,\sigma_{dc}^{\mu}\,P_{\mu}\:.
\]
但是$\,e\sigma_{i}^{\mathrm{T}}e^{-1}=-\sigma_{i}\,$且$\,e\sigma_{0}^{\mathrm{T}}e^{-1}=+\sigma_{0}$, 而$\,\mathsf{P}^{-1}P_{i}\mathsf{P}=-P_{i}\,$以及$\,\mathsf{P}^{-1}P_{0}\mathsf{P}=-P_{0}$, 所以这退化成表述$\,\mathscr{P}\,$是幺正的
\begin{equation}
    \mathscr{P}\,\mathscr{P}^{\dag}=1\:. \label{25.3.5}
\end{equation}

矩阵$\,\mathscr{P}\,$在某种意义上是任意的, 这是因为, 对于任何一组满足方程(\ref{25.3.2})和(\ref{25.2.7})的费米生成元$\,\mathcal{Q}_{ar}$, 通过幺正变换
\begin{equation}
    \mathcal{Q}_{ar}^{\prime}=\sum_{s}\mathscr{U}_{rs}\,\mathcal{Q}_{as} \:, \qquad
    \mathscr{U}^{\dag}=\mathscr{U}^{-1}\:, \label{25.3.6}
\end{equation}
我们可以构建另一组也满足方程(\ref{25.3.2})和(\ref{25.2.7})的$\,\mathcal{Q}_{ar}^{\prime}$, 使得宇称变换规则(\ref{25.3.4})变成
\begin{equation}
    \mathsf{P}^{-1}\mathcal{Q}_{ar}^{\prime}\mathsf{P}
    =\sum_{bs}\mathscr{P}_{rs}^{\prime}e_{ab}\mathcal{Q}_{bs}^{\prime\ast}\:,\label{25.3.7}
\end{equation}
其中
\begin{equation}
    \mathscr{P}^{\prime} =\mathscr{U}\mathscr{P}\mathscr{U}^{-1\ast}
    =\mathscr{U}\mathscr{P}\mathscr{U}^{\mathrm{T}} \:. \label{25.3.8}
\end{equation}

对于简单超对称, $\mathscr{P}\,$就是$\,1\times1\,$相因子, 而方程(\ref{25.3.4})变成
\begin{equation}
    \mathsf{P}^{-1}\mathcal{Q}_{a}\mathsf{P}=\mathscr{P}\sum_{b}e_{ab}\mathcal{Q}_{b}^{\ast} \:. \label{25.3.9}
\end{equation}
结合它的共轭给出
\begin{equation}
    \mathsf{P}^{-2}\mathcal{Q}_{a}\mathsf{P}^{2} = -\mathcal{Q}_{a}\:, \label{25.3.10}
\end{equation}
这独立于我们对相位因子$\,\mathscr{P}\,$选的值. 这有一个显著的结果, 如果处在粒子超重态的一个玻色子拥有使得内禀宇称, 那么通过用$\,\mathcal{Q}_{a}\,$作用这个玻色态获得的费米子就会有{\kai{纯虚}}的内禀宇称.

因为对于简单超对称$\,\mathscr{U}\,$和$\,\mathscr{P}\,$就是相因子, 从方程(\ref{25.3.8})显然可以得出, 通过合适地选择$\,\mathscr{U}$, 相因子$\,\mathscr{P}^{\prime}\,$可以变成任何我们想要的东西. 选择$\,\mathscr{P}^{\prime}=+\mi\,$将是方便的, 这使得方程(\ref{25.3.7})取如下的简单形式(现在扔掉撇号)
\begin{equation}
    \mathsf{P}^{-1}\mathcal{Q}_{a}\mathsf{P} = \mi\sum_{b}e_{ab}\mathcal{Q}_{b}^{\ast} \:. \label{25.3.11}
\end{equation}
就像我们对旋量场算符做的那样, 如果我们将$\,(0,1/2)\,$算符$\,\mathcal{Q}_{a}\,$和$\,(1/2,0)\,$算符$\,\sum_{b}e_{ab}\mathcal{Q}_{b}^{\ast}\,$%
并入方程 (\ref{25.2.34})定义的\,Dirac\,旋量生成元$\,Q_{\alpha}$, 空间反演的表示会更加简单. 写成这些项, 方程(\ref{25.3.11})和它的共轭变成
\begin{equation}
    \mathsf{P}^{-1}Q\mathsf{P}=\mi\beta\,Q\:. \label{25.3.12}
\end{equation}
(我们对\,Dirac\,矩阵使用的约定是\,5.4\,节和本卷前言给出的那个约定, 这个约定中
\[
\beta = \begin{pmatrix}
0 & 1 \\ 1 & 0
\end{pmatrix} \:,
\]
其中$\,1\,$和$\,0\,$被理解成$\,2\times2\,$子矩阵.)

对于扩充超对称, 选择$\,\mathscr{U}\,$使得$\,\mathscr{P}^{\prime}\,$对角不总是可行的. 然而, 矩阵代数中的(第\,2\,章附录\,C\,证明的)一个定理表明, 选择$\,\mathscr{U}\,$使得$\,\mathscr{P}^{\prime}\,$分块对角是可行的, 其中, 一般而言, 主对角上$\,1\times1\,$子矩阵可以选成$\,\mi\,$(或者任何其它我们想要的相因子), 而其它子矩阵是$\,2\times2\,$矩阵, 可以对其进行选择使其拥有如下形式
\[
\begin{pmatrix}
0 & \exp(\mi\phi) \\ \exp(-\mi\phi) & 0
\end{pmatrix} \:,
\]
其中$\,\phi\,$是各种相位. 相应地, 在$\,\mathscr{U}\,$的这个选择下, (现在扔掉撇号) 有两种\,2\,分量$\,\mathcal{Q}$. 第一种$\,\mathcal{Q}\,$也满足方程(\ref{25.3.11}):
\begin{equation}
    \mathsf{P}^{-1}\mathcal{Q}_{ar}\mathsf{P}=\mi\sum_{b}e_{ab}\mathcal{Q}_{br}^{\ast} \:. \label{25.3.13}
\end{equation}
第二种的\,2\,分量$\,\mathcal{Q}\,$成对出现, 我们记做$\,\mathcal{Q}_{a\,s1}\,$和$\,\mathcal{Q}_{a\,s2}$, 其中第\,$s$\,对的宇称变换规则是
\begin{equation}
    \mathsf{P}^{-1}\mathcal{Q}_{a\,s1}\mathsf{P} = \me^{\mi \phi_{s}}\sum_{b}e_{ab}\mathcal{Q}_{b\,s2}^{\ast} \:,
    \qquad
    \mathsf{P}^{-1}\mathcal{Q}_{a\,s2}\mathsf{P} = \me^{\mi \phi_{s}}\sum_{b}e_{ab}\mathcal{Q}_{b\,s1}^{\ast} \:,
    \label{25.3.14}
\end{equation}
特别地, 我们现在有
\begin{equation}
    \mathsf{P}^{-2}\mathcal{Q}_{a\,s1}\mathsf{P}^{2} = -\me^{2\mi\phi_{s}}\mathcal{Q}_{a\,s1}\:, \qquad
    \mathsf{P}^{-2}\mathcal{Q}_{a\,s2}\mathsf{P}^{2} = -\me^{-2\mi\phi_{s}}\mathcal{Q}_{a\,s2}\:.\label{25.3.15}
\end{equation}
这表明, 除非$\,\phi_{s}=0\:(\operatorname{mod}\uppi)$, 否则不可能用第二种扩充超对称性生成元的线性组合来构建第一种超对称性生成元.

写成\,4\,分量旋量(\ref{25.2.34})的形式, 宇称算符在第一种扩充超对称性生成元上的效果是
\begin{equation}
    \mathsf{P}^{-1}Q_{r}\mathsf{P}=\mi\beta\,Q_{r} \:, \label{25.3.16}
\end{equation}
对于第二种生成元则是
\begin{equation}
    \mathsf{P}^{-1} Q_{s1}\mathsf{P}=\beta\,\gamma_{5}\,\exp(\mi\gamma_{5}\phi_{s})Q_{s2}\:, \qquad
    \mathsf{P}^{-1} Q_{s2}\mathsf{P}=\beta\,\gamma_{5}\,\exp(-\mi\gamma_{5}\phi_{s})Q_{s1}\:. \label{25.3.17}
\end{equation}

\section{无质量粒子的超多重态} \label{sec:25.4}

超对称要求已知的粒子在超对称代数的不可约表示下要伴随``超粒子''(sparticles): 伴随夸克和轻子的是玻色``标量夸克''(squarks)和``标量轻子''(sleptons), 伴随规范玻色子的是费米``规范微子''(gauginos). 所有这些粒子还都没有观测到, 所以超对称性肯定破缺, 而超粒子的质量几乎肯定要比电弱$\,SU(2)\times U(1)\,$规范群自发破缺产生的夸克质量, 轻子质量和规范玻色子质量大得多, 因此和超对称多重态内部分裂的大小处于同一量级. 因此, 在能标足够高时很有可能我们可以忽略掉超对称破缺和这些质量分裂, 此时我们也可以将已知的夸克, 轻子, 规范玻色子和它们的超对称伴视作是无质量的. 因此, 我们对无质量粒子的超对称多重态很感兴趣.

考察这样一个态, 它只包含一个无质量粒子且这个粒子属于某个超对重态. 通过用算符$\,\mathcal{Q}_{ar}$ 和(或)$\,\mathcal{Q}_{ar}^{\ast}\,$作用这个态, 我们可以获得同一超多重态中的其它态. 由于$\,\mathcal{Q}_{ar}\,$和$\,\mathcal{Q}_{ar}^{\ast}\,$与$\,P_{\mu}\,$对易, 所有这些态有相同的$\,4\,$-动量值. 我们将在这些态的\,4\,-动量是$\,p^{1}=p^{2}=0\,$且$\,p^{3}=p^{0}=E\,$的\,Lorentz\,参考系下进行处理. 在\,4\,-动量的这一选择下, 我们有
\begin{equation}
    \sigma_{\mu}p^{\mu} = E(\sigma_{0}+\sigma_{3})= 2E
    \begin{pmatrix}
    1 & 0 \\ 0 & 0
    \end{pmatrix} \:, \label{25.4.1}
\end{equation}
除去因子$\,2E\,$来看, 这是到螺旋度为$\,+1/2\,$的子空间上的投射矩阵. 因此, 反对易关系(\ref{25.2.7})表明, 对于有这种\,4\,-动量的超多重态, $\{\mathcal{Q}_{(-1/2)\,r}, \mathcal{Q}^{\ast}_{(-1/2)\,r} \}\,$作用在这个超多重态中的任何态上都给出零, 同理, $\mathcal{Q}_{(-1/2)\,r}\,$和$\,\mathcal{Q}^{\ast}_{(-1/2)\,r}\,$也是如此. 因此我们只能通过用$\,\mathcal{Q}_{(1/2)\,r}\,$和$\,\mathcal{Q}^{\ast}_{(1/2)\,r}\,$进行作用来构建%
这个超多重态中的态. 更进一步, 我们可以用$\,\mathcal{Q}\,$的$\,J_{3}\,$值来标记它们, 也就是说
\begin{equation}
    [J_{3},\mathcal{Q}_{ar}] =-a\,\mathcal{Q}_{ar}\:, \label{25.4.2}
\end{equation}
所以$\,\mathcal{Q}_{(1/2)\,r}\,$和$\,\mathcal{Q}^{\ast}_{(1/2)\,r}\,$分别将螺旋度减小和提高$\,1/2$.

我们首先来考察简单超对称的情况. 考察一个最大螺旋度为$\,\lambda_{\mathrm{max}}\,$的超多重态, 用$\,\lvert \lambda_{\mathrm{max}}\rangle\,$标记任何螺旋度为$\,\lambda_{\mathrm{max}}\,$的单粒子态, 并设它的\,4\,-动量是$\,p^{\mu}$. 那么
\begin{equation}
    \mathcal{Q}_{\frac{1}{2}}^{\ast}\lvert \lambda_{\mathrm{max}}\rangle = 0 \:, \label{25.4.3}
\end{equation}
用$\,\mathcal{Q}_{1/2}\,$作用这个态则给出螺旋度为$\,\lambda_{\mathrm{max}}-1/2\,$的态%
$\,\lvert\lambda_{\mathrm{max}}-1/2\rangle$. 我们会将这个态定义成
\begin{equation}
    \lvert \lambda_{\mathrm{max}} - 1/2\rangle \equiv (4E)^{-1/2}\,\mathcal{Q}_{\frac{1}{2}}\,\lvert \lambda_{\mathrm{max}}\rangle \:. \label{25.4.4}
\end{equation}
加上方程(\ref{25.4.1})和(\ref{25.4.3}), 基础反对易关系(\ref{25.2.7})表明这个态的归一化方式与$\,\lvert\lambda_{\mathrm{max}}\rangle\,$相同
\begin{equation}
    \langle \lambda_{\mathrm{max}}-1/2\vert \lambda_{\mathrm{max}}-1/2\rangle
    = \langle \lambda_{\mathrm{max}}\vert \lambda_{\mathrm{max}} \rangle \:,\label{25.4.5}
\end{equation}
特别地, 这个态不能为零. 方程(\ref{25.2.32})表明$\,\mathcal{Q}_{1/2}^{2}=0$, 所以用$\,\mathcal{Q}_{1/2}\,$作用$\,\lvert\lambda_{\mathrm{max}}-1/2\rangle\,$给出零:
\begin{equation}
    \mathcal{Q}_{\frac{1}{2}}\lvert \lambda_{\mathrm{max}}-1/2\rangle =
    (4E)^{-1/2}\mathcal{Q}_{\frac{1}{2}}^{2}\lvert \lambda_{\mathrm{max}}\rangle = 0 \:.\label{25.4.6}
\end{equation}
另一方面, 用$\,\mathcal{Q}_{1/2}^{\ast}\,$作用这个态给出的是我们作为出发点的态. 即,
\[
 \mathcal{Q}_{\frac{1}{2}}^{\ast}\lvert \lambda_{\mathrm{max}}-1/2\rangle =
    (4E)^{-1/2}\mathcal{Q}_{\frac{1}{2}}^{\ast}\mathcal{Q}_{\frac{1}{2}}\lvert \lambda_{\mathrm{max}}\rangle
    = (4E)^{-1/2}\{\mathcal{Q}_{\frac{1}{2}}^{\ast},\mathcal{Q}_{\frac{1}{2}}\}\lvert \lambda_{\mathrm{max}}\rangle \:,
\]
这使得方程(\ref{25.4.1})和反对易关系(\ref{25.2.31})产生
\begin{equation}
    \mathcal{Q}_{\frac{1}{2}}^{\ast}\lvert \lambda_{\mathrm{max}}-1/2\rangle =
    (4E)^{1/2} \lvert \lambda_{\mathrm{max}}\rangle \:. \label{25.4.7}
\end{equation}
因此超多重态仅由两个态构成, 螺旋度分别是$\,\lambda_{\mathrm{max}}\,$和$\,\lambda_{\mathrm{max}}-1/2$. 在这两个态提供的基下, 算符$\,\mathcal{Q}_{1/2}\,$和$\,\mathcal{Q}_{1/2}^{\ast}\,$被表示成矩阵
\begin{equation}
    q_{\frac{1}{2}} = \sqrt{4E} \begin{pmatrix}
    0 & 0 \\ 1 & 0
    \end{pmatrix} \:, \qquad
    q_{\frac{1}{2}}^{\dag} = \sqrt{4E} \begin{pmatrix}
    0 & 1 \\ 0 & 0
    \end{pmatrix} \:, \label{25.4.8}
\end{equation}
而算符算符$\,\mathcal{Q}_{-1/2}\,$和$\,\mathcal{Q}_{-1/2}^{\ast}\,$被表示成零.

值得强调的是, 在有简单超对称性的理论中, 这是{\kai{唯一}}一种无质量超多重态. 不存在没有超对称伴的无质量粒子, 也不存在有多个超对称伴的无质量粒子. 当然, $\mathsf{CPT}\,$不变性暗示了, 对于每个螺旋度为$\,\lambda\,$和$\,\lambda-1/2\,$的无质量粒子超多重态, 必存在一个螺旋度为$\,-\lambda+1/2\,$和$\,-\lambda\,$的反多重态. 特别地, 螺旋度为$\,+1/2\,$和$\,-1/2\,$的无质量粒子和反粒子不是伴随着螺旋度为$\,+1\,$和$\,-1\,$的无质量粒子和反粒子, 就是伴随着螺旋度均为零的无质量粒子和反粒子.

那么已知的夸克, 轻子和规范玻色子该如何填进这个图景中呢? 我们将假定超对称性生成元与$\,SU(3)\times SU(2)\times U(1)\,$规范群的生成元对易.\footnote{在简单超对称性中, 由于$\,SU(3)\times SU(2)\,$这样的半单\,Lie\,代数没有不平庸的一维表示, 所以生成元$\,Q_{\alpha}\,$在任何情况下都必须与$\,SU(3)\times SU(2)\,$生成元对易.} 夸克和轻子所属的规范群表示与规范玻色子所属的规范群表示不同, 所以它们不可能处在同一个超多重态中. 由此我们不得不得到如下的结论: 在$\,SU(2)\times U(1)\,$对称性破缺可以被忽略的高能极限下, 每种色和味的无质量夸克和轻子都伴随着色和味相同但螺旋度为零的无质量标量夸克和标量轻子{\kai{成对}}出现在超多重态中, 而与无质量规范玻色子伴随的螺旋度为$\,\pm1/2\,$的无质量规范微子构成了$\,SU(3)\times SU(2)\times U(1)\,$的一个伴随表示.

由于存在引力, 我们知道除了标准模型的粒子以外还必须存在螺旋度为$\,\pm2\,$的粒子, {\kai{引力子}}. 对于螺旋度为$\,\lambda\,$的无质量粒子, 如果$\,\lvert\lambda\rvert >1/2$, 那么在低动量时它必须与守恒量耦合.\footnote{我们在\,13.1\,节讨论过螺旋度是整数的情况. 关于半整数螺旋度的讨论是由\,Grisaur\,和\,Pendleton\,给出的.\cite{3}} 螺旋度为$\,\pm1\,$的软无质量粒子可以与各种内部对称性的生成元耦合, 螺旋度为$\,\pm3/2\,$的软无质量粒子可以与超对称生成元$\,\mathcal{Q}_{a}\,$耦合, 而螺旋度为$\,\pm2\,$的软粒子可以与一个守恒量耦合, 动量\,4\,-矢$\,P_{\mu}$, 但是没有什么守恒量可以与$\,\lvert\lambda\rvert>2\,$的软粒子耦合. 由此我们得出, 引力子所处的超多重态不能含有螺旋度为$\,\pm5/2\,$的粒子, 所以它所处的超多重态必须含有螺旋度为$\,\pm3/2\,$的粒子, 这个粒子被称为{\kai{引力微子}}(gravitino), 它与超对称性生成元自身耦合. 这个超多重态的场论被称为{\kai{超引力}}, 我们将在第\,31\,章讨论.

现在我们来考察有$\,N\,$个超对称生成元的扩充超对称性情况. 我们首先注意到, $\mathcal{Q}_{(-1/2)\,r}\,$作用在超多重态中的态(包含那些通过$\,\mathcal{Q}_{(1/2)\,s}$\,作用这个多重态的任何其它态获得的态)上时都给出零, 所以$\,Z_{rs}\,$也湮灭这个多重态的任何态.
在中心荷不在这个图景的前提下, 当超对称生成元$\,\mathcal{Q}_{(1/2)\,r}$ 作用在无质量粒子超多重态上时, 它们反对易, 所以用$\,n\,$个这样的生成元作用拥有最大螺旋度 $\lambda_{\mathrm{max}}$ 且动量为$\,p^{\mu}\,$的单粒子态时, 我们会得到$\,N!/n!(N-n)!\,$个螺旋度为$\,\lambda_{\mathrm{max}}-n/2\,$且动量相同的单粒子态, 这些单粒子态构成$\,SU(N)\,R$-对称性\footnote{$U(N)\,R$-对称性的$\,U(1)\,$部分通常会被量子力学反常破坏掉.}%
(\ref{25.2.30})的一个$\,n\,$阶反对称表示. 能够给出一个非零态的最大$\,n\,$值是$\,n=N$, 所以一个超多重态中的最小螺旋度是
\begin{equation}
    \lambda_{\mathrm{min}} =\lambda_{\mathrm{max}}-N/2 \:. \label{25.4.9}
\end{equation}
如果我们希望排除掉那些螺旋度$\,\lambda\,$满足$\,\lvert\lambda\rvert>2\,$的无质量粒子, 那么$\,\lambda_{\mathrm{max}}-\lambda_{\mathrm{min}}\leq 4$, 所以只有那些$\,N\leq 8\,$的扩充超对称性才是被允许的.

当$\,N=8\,$且$\,\lvert\lambda\rvert>2\,$的螺旋度被排除时, 只有一种可能的超多重态, 组成它的是: 螺旋度为$\,\pm2\,$的引力子各\,1\,个; 螺旋度为$\,\pm3/2\,$的引力微子各$\,8\,$个; 螺旋度为$\,\pm1\,$的规范玻色子各$\,28\,$个; 螺旋度为$\,\pm1/2\,$的费米子各$\,56\,$个; 以及$\,70\,$个螺旋度为零的玻色子.

对比$\,N=7\,$的情况, 依旧排除掉$\,\lvert\lambda\rvert>2\,$的螺旋度. 这里存在两个超多重态. 一个超多重态包含: 1\,个螺旋度为$\,+2\,$的引力子; 7\,个螺旋度为$\,+3/2\,$的引力微子; 21\,个螺旋度为$\,+1\,$的规范玻色子; 35\,个螺旋度为$\,+1/2\,$的费米子; 35\,个螺旋度为零的玻色子; 21\,个螺旋度为$\,-1/2\,$的费米子; 7\,个螺旋度为$\,-1\,$的规范玻色子; 以及\,1\,个螺旋度为$\,-3/2\,$的引力微子. 另一个是\,$\mathsf{CPT}\,$-共轭的超多重态, 它的所有螺旋度反号. 将这两个超多重态中的粒子数加一下, 我们有螺旋度为$\,\pm2\,$的引力子各\,1\,个; 螺旋度为$\,\pm3/2\,$的引力微子各$\,7+1=8\,$个; 螺旋度为$\,\pm1\,$的规范玻色子各$\,21+7=28\,$个; 螺旋度为$\,\pm1/2\,$的费米子各$\,35+21=56\,$个; 以及$\,35+35=70\,$个螺旋度为零的玻色子. 因此$\,N=8\,$和$\,N=7\,$的扩充超引力理论有精确相同的粒子内容, 它们实质上是等价的.

另一方面, 在$\,N\leq6\,$的扩充超引力理论中, 螺旋度为$\,\pm3/2\,$的引力微子分别只有$\,N\,$个, 因此它们都是不同的.

当$\,N\leq 4\,$时, 还存在{\kai{整体}}超对称理论的可能性, 在这样的理论中, 超多重态没有引力子和引力微子. 对于整体$\,N=4\,$超对称性, 仅存在一个超多重态, 组成它的是: 螺旋度为$\,\pm1\,$的规范玻色子各\,1\,个; 螺旋度为$\,\pm 1/2\,$的费米子各\,4\,个; 以及\,6\,个螺旋度为零的玻色子. 这等价于$\,N=3\,$的整体超对称理论, 它有两个超多重态: 一个超多重态有\,1\,个螺旋度为$\,+1\,$的规范玻色子; 3\,个螺旋度为$\,+1/2\,$的费米子; 3\,个螺旋度为零的玻色子; 以及\,1\,个螺旋度为$\,-1/2\,$的费米子; 而另一个\,$\mathsf{CPT}$\,共轭的超多重态有相反的螺旋度. 将这两个$\,N=3\,$超多重态中螺旋度相同的粒子数加起来就给出了同$\,N=4\,$整体超对称性相同的粒子内容. 拥有$\,N=4\,$超对称性的规范场论拥有显著的性质, 这将在\,27.9\,节进行讨论.

对于$\,N=2\,$的扩充超对称, 除了那些通过$\,\mathsf{CPT}\,$关联的超多重态外, 存在两种不同类型的超多重态. 一种是{\kai{规范}}超多重态, 它们包含一个螺旋度为$\,+1\,$的规范玻色子, 两个螺旋度为$\,+1/2\,$的费米子, 并且这两个费米子构成了$\,SU(2)\,R$-对称性下的一个双重态, 以及一个螺旋度为零的玻色子, 再加上螺旋度都反号的$\,\mathsf{CPT}\,$-共轭超多重态. 每个规范超多重态和和它的反多重态合起来包含了: 螺旋度为$\,\pm1\,$的规范玻色子各\,1\,个, 螺旋度为$\,\pm1/2\,$的费米子双重态各一个, 以及两个螺旋度为零的$\,SU(2)\,$单态玻色子. 另一种是{\kai{极多重态}}(hypermultiplet), 它们包含螺旋度$\,\pm1/2\,$的费米子各一个, 以及螺旋度为零的玻色子构成的$\,SU(2)\,$双重态, 再加上这种超多重态的$\,\mathsf{CPT}$-共轭. (在量子场论中, 极多重态不能是自身的反多重态, 若非如此, 螺旋度为零的粒子将会被两个{\kai{实}}标量场描述, 而它们是无法形成一个$\,SU(2)\,$多重态的.) 当然, 在真实世界中还必须存在引力子超多重态, 它包含一个螺旋度为$\,+2\,$的引力子, 一个螺旋度为$\,+3/2\,$的引力微子的$\,SU(2)\,$双重态, 以及一个螺旋度为$\,+1\,$的规范玻色子, 再加上它们螺旋度相反的$\,\mathsf{CPT}\,$-共轭. 我们将在\,27.9\,节构造$\,N=2\,$超对称规范理论, 并在\,29.5\,节以非微扰的方式探索它的性质.

在可达到的能标试图将扩充超对称性融入进粒子的真实理论时, 这些超多重态的粒子内容反映出了一个困难. 除了一种情况外, 在任何其它情况中, 螺旋度$\,+1/2\,$的费米子和螺旋度$\,+1\,$的规范玻色子同属一个超多重态. 规范玻色子属于规范群的伴随表示, 所以如果超对称生成在规范群下不变, 那么螺旋度$\,+1/2\,$的费米子也必须属于伴随表示, 而这是一个实表示. 而已知的夸克和轻子所属的$\,SU(3)\times SU(2)\times U(1)\,$表示是{\kai{手征的}}------即, 对于这样的表示, 螺旋度$\,+1/2\,$的费米子属于一个复表示, 那么它们的$\,\mathsf{CPT}\,$共轭, 螺旋度$\,-1/2\,$的费米子构建的表示肯定与这个表示不同, 二者是矛盾的. 唯一的例外是上面讨论的$\,N=2\,$极多重态, 在这种情况中, 螺旋度$\,+1/2\,$的费米子不在规范玻色子所处的超多重态中. 但在这一情况中, 螺旋度为$\,+1/2\,$和$\,-1/2\,$的粒子处在同一超多重态中, 因此在任何使得超对称生成元不变的规范变换下, 它们的变换必须相同. 它们可能属于这个规范群的一个复表示, 那么这个极多重态的$\,\mathsf{CPT}\,$-共轭就属于复共轭表示, 这样一来, 各个螺旋度的费米子属于两个表示的和, 这是实的, 依旧与已知夸克和轻子的手征性相矛盾.

与之相反, 对于简单超对称性存在只包含螺旋度$\,+1/2\,$和零的超多重态, 它们可能处在规范群的一个复表示中并与$\,\mathsf{CPT}\,$-共轭超多重态构建的表示不同. 这里不存在与手征性相悖的矛盾. 由于这个原因, 将超对称在可实现的能标视为没有破缺的对称性的讨论大多集中在简单超对称性而非扩充超对称性上.


\section{有质量粒子的超多重态} \label{sec:25.5}

尽管已知的夸克, 轻子和规范玻色子和它们的超对称伴在超对称破缺可以被忽视的能标处可以被视为是无质量的, 但对于其它粒子, 包括统一强相互作用和电弱相互作用的理论要求的质量很大的额外规范玻色子, 这不一定是成立的. 另外, 自\,Wess-Zumino\,模型起, 对于研究超对称理论, 有质量粒子的理论就已经是个很有用的测试情况. 因此对我们来说, 简要地考察未破缺超对称对有质量粒子的意义将是值得的.

就像在上一节, 通过用算符$\,\mathcal{Q}_{ar}\,$和$\,\mathcal{Q}^{\ast}_{ar}$作用超多重态中的任何一个单粒子态, 我们获得了超多重态中的各种单粒子态, 并且所有这些态有相同的\,4\,-动量. 不同于零质量的情况, 当质量$\,M>0\,$时, 我们现在可以取静止粒子的\,4\,-动量, 其中$\,i=1,2,3\,$的$\,p^{i}=0\,$且$\,p^{0}=M$. 在这个参考系下, 我们有
\begin{equation}
    \sigma_{\mu}p^{\mu} = M\sigma_{0} = M
    \begin{pmatrix}
    1 & 0 \\ 0 & 1
    \end{pmatrix} \:. \label{25.5.1}
\end{equation}
因此, 作用在有这一\,4\,-动量的超多重态中的任何态$\,\lvert\:\rangle\,$上, 反对易关系(\ref{25.2.7})给出
\begin{equation}
    \{\mathcal{Q}_{ar},\mathcal{Q}_{bs}^{\ast}\}\lvert\:\rangle = 2M\,\delta_{ab}\,\delta_{rs}\,\lvert\:\rangle \:.\label{25.5.2}
\end{equation}
与零质量的情况相反, 这里没有哪个$\,\mathcal{Q}_{ar}\,$或$\,\mathcal{Q}_{ar}^{\ast}\,$的分量可以在整个多重态上为零, 所以我们有两组上升和下降算符: $\mathcal{Q}_{(1/2)\,r}\,$和$\,\mathcal{Q}^{\ast}_{(-1/2)\,r}\,$均将自旋\,3\,-分量降低\,1/2, 而$\,\mathcal{Q}_{(-1/2)\,r}\,$和$\,\mathcal{Q}^{\ast}_{(1/2)\,r}\,$均将自旋\,3\,-分量提高\,1/2. 然而, 我们将会看到, 对于扩充超对称性, $Q\,$和$\,Q^{\ast}\,$的特定线性组合有可能为零.

我们将首先考察简单超对称的情况. 通过使用超对称代数(\ref{25.2.31})和(\ref{25.2.32}), 我们将证明一般的有质量超多重态由一个自旋\,$j+1/2\,$的粒子, 一{\kai{对}}自旋$\,j\,$的粒子和一个自旋$\,j-1/2\,$的粒子构成. 当宇称守恒时, 自旋为$\,j\pm1/2\,$的两个粒子拥有相同的内禀宇称, 由某个相位$\,\eta\,$给定, 而两个自旋$\,j\,$的粒子分别有宇称$\,+\mi\eta\,$和$\,-\mi\eta$. 这里的$\,j\,$是大于零的整数或半整数. 同时还存在坍缩超多重态, 它由两个自旋零的粒子一个自旋$\,1/2\,$的粒子构成. 当宇称守恒时, 自旋零的粒子有宇称$\,\mi\eta\,$和$\,-\mi\eta$, 其中$\,\eta\,$是自旋$\,1/2\,$粒子的宇称.

下面是证明. 我们首先证明任何超多重态将包含至少一个自旋多重态$\,\lvert j,\sigma\rangle$, 其中自旋\,3\,-分量$\,\sigma\,$以一为步长从$\,-j\,$取到$\,+j$, 它有特殊性质, 对于所有这样的$\,\sigma\,$和$\,a=\pm1/2$,
\begin{equation}
    \mathcal{Q}_{a}\,\lvert j,\sigma \rangle =0 \:. \label{25.5.3}
\end{equation}
从这个超多重态中的任何非零态$\,\lvert\psi\rangle\,$出发, 我们可以定义非零态
\[
\lvert \psi^{\prime}\rangle\equiv
\begin{cases}
(2M)^{-1/2}\mathcal{Q}_{1/2}\lvert\psi \rangle & \qquad \mathcal{Q}_{1/2}\lvert\psi\rangle \neq 0 \\
\lvert \psi \rangle & \qquad \mathcal{Q}_{1/2}\lvert\psi\rangle =0
\end{cases} \:,
\]
和
\[
\lvert \psi^{\prime\prime}\rangle\equiv
\begin{cases}
(2M)^{-1/2}\mathcal{Q}_{-1/2}\lvert\psi^{\prime} \rangle & \qquad \mathcal{Q}_{-1/2}\lvert\psi^{\prime}\rangle \neq 0 \\
\lvert \psi^{\prime} \rangle & \qquad \mathcal{Q}_{-1/2}\lvert\psi^{\prime}\rangle =0
\end{cases} \:.
\]
由于$\,\mathcal{Q}_{a}\,$反对易, $\mathcal{Q}_{1/2}\lvert\psi^{\prime}\rangle=0$, 因此对于$\,a=\pm1/2\,$有$\,\mathcal{Q}_{a}\lvert\psi^{\prime\prime}\rangle=0$. 如果任何态$\,\lvert\psi^{\prime\prime}\rangle\,$满足条件$\,\mathcal{Q}_{a}\lvert\psi^{\prime\prime}=0$, 那么对于表示任意空间旋转的幺正表示$\,U(R)$, $U(R)\lvert\psi^{\prime\prime}\rangle\,$也满足这个条件. 由此得出满足这个条件态可以被分解进完整的自旋多重态$\,\lvert j,\sigma\rangle$, 它满足条件(\ref{25.5.3})

现在集中在任何一个满足方程(\ref{25.5.3})的自旋多重态上, 对它进行归一化使得
\begin{equation}
    \langle j,\sigma^{\prime}\vert j,\sigma \rangle = \delta_{\sigma^{\prime}\sigma} \:. \label{25.5.4}
\end{equation}
当$\,j>0\,$时, 通过用自旋$\,1/2\,$算符\footnote{既然$\,\mathcal{Q}_{a}\,$在旋转下的变换就像湮灭一个自旋为$\,1/2\,$且\,3\,-分量%
为$\,\sigma\,$的粒子的场, 那么$\,\mathcal{Q}_{a}^{\ast}\,$的变换就像产生这种粒子的场, 因此它的变换就像粒子本身. 形式上, 因为$[J_{i},\mathcal{Q}_{a}]=-\sum_{b}\tfrac{1}{2}(\sigma_{i})_{ab}\mathcal{Q}_{b}$, 所以$[J_{i},\mathcal{Q}^{\ast}_{a}]=-\sum_{b}\tfrac{1}{2}(\sigma_{i})_{ba}\mathcal{Q}_{b}^{\ast}$, 它可以与自旋$\,1/2\,$粒子的变换性质, $J_{i}\lvert a\rangle=\sum_{b}\tfrac{1}{2}(\sigma_{i})_{ba}\lvert b\rangle$, 相比较.}$\,\mathcal{Q}_{a}^{\ast}\,$作用这些态, 我们可以构造出自旋$\,j\pm1/2\,$的态:
\begin{equation}
    \lvert j\pm 1/2,\sigma \rangle = \frac{1}{\sqrt{2M}}
    \sum_{a}C_{\frac{1}{2}\,j}\Bigl(j\pm 1/2 ,\sigma\, ;a, \sigma-a\Bigr)\, \mathcal{Q}_{a}^{\ast} \lvert j,\sigma-a\rangle\:,
    \label{25.5.5}
\end{equation}
其中$\,C_{jj^{\prime}}(j^{\prime\prime},\sigma^{\prime\prime};\sigma,\sigma^{\prime})\,$是%
传统的\,Clebsch-Gordan\,系数, 它将\,3\,-分量为$\,\sigma\,$和$\,\sigma^{\prime}\,$的自旋$\,j\,$和$\,j^{\prime\prime}\,$耦合成 3\,-分量为%
$\,\sigma^{\prime\prime}\,$的自旋$\,j^{\prime\prime}$. 利用方程(\ref{25.5.2})---(\ref{25.5.5})和%
\,Clebsch-Gordan\,系数的正交性, 我们可以证明这些态的归一化是正确的:
\begin{equation}
    \langle j\pm 1/2,\sigma \vert j\pm 1/2,\sigma^{\prime}\rangle =\delta_{\sigma\sigma^{\prime}}\:,\qquad
    \langle j\pm 1/2,\sigma \vert j\mp 1/2,\sigma^{\prime}\rangle =0 \:, \label{25.5.6}
\end{equation}
所以态$\,\lvert j\pm 1/2,\sigma\rangle\,$中的任何一个都不能为零. 唯一的例外是$\,j=0$, 这时显然是由于不存在态$\,\lvert j-1/2,\sigma\rangle$. 我们以可以通过作用{\kai{两}}个$\,\mathcal{Q}^{\ast}\,$%
在$\,\lvert j,\sigma\rangle\,$上获得其它态. 由于每个$\,\mathcal{Q}_{a}^{\ast}\,$与它自身反对易, 唯一这样的非零态是通过作用算符$\,\mathcal{Q}_{1/2}^{\ast}\mathcal{Q}_{-1/2}^{\ast}
=-\mathcal{Q}_{-1/2}^{\ast}\mathcal{Q}_{1/2}^{\ast}\,$形成的. 这个算符可以写成$\,\frac{1}{2}e_{ab}\mathcal{Q}_{a}^{\ast}\mathcal{Q}_{b}^{\ast}$, 这表明它是一个旋转不变量, 所以这给出了第二个自旋为$\,j\,$的自旋多重态:
\begin{equation}
    \lvert j,\sigma\rangle^{\flat} = \frac{1}{2M} \, \mathcal{Q}_{1/2}^{\ast}\, \mathcal{Q}_{-1/2}^{\ast} \lvert j,\sigma \rangle\:,
    \label{25.5.7}
\end{equation}
它与$\,\lvert j,\sigma\rangle\,$不同是因为, 取代方程(\ref{25.5.3}), 我们有
\begin{equation}
    \mathcal{Q}_{a}^{\ast} \lvert j,\sigma\rangle^{\flat} = 0  \:. \label{25.5.8}
\end{equation}
再次使用方程(\ref{25.5.2})---(\ref{25.5.4}), 我们发现它们也是归一化态:
\begin{equation}
    {}^{\flat}\langle j,\sigma^{\prime}\vert j,\sigma\rangle^{\flat}=\delta_{\sigma^{\prime}\sigma}\:, \qquad
    \langle j,\sigma^{\prime}\vert j,\sigma\rangle ^{\flat} = 0\:. \label{25.5.9}
\end{equation}
那么很容易证明迄今为止构造的态构成了超对称代数的一个完整表示. Clebsch-Gordan\,系数的正交性使得我们可以将方程(\ref{25.5.5})重写成
\begin{equation}
    \mathcal{Q}_{a}^{\ast}\lvert j,\sigma\rangle = \sqrt{2M}\sum_{\pm}
    C_{\frac{1}{2}\,j}\Bigl(j\pm1/2,\sigma+a\,;a,\sigma\Bigr)\,\lvert j\pm1/2,\sigma+a\rangle \:. \label{25.5.10}
\end{equation}
另外, 方程(\ref{25.5.2})表明, 对于超对重态中的任何态$\,\lvert \:\rangle$,
\begin{equation}
    \Bigl[\mathcal{Q}_{a}, \mathcal{Q}_{\frac{1}{2}}^{\ast}\mathcal{Q}_{-\frac{1}{2}}^{\ast}\Bigr]\,\lvert \:\rangle
    =2M \sum_{b}e_{ab}\,\mathcal{Q}_{b}^{\ast}\,\lvert\:\rangle \:, \label{25.5.11}
\end{equation}
所以方程(\ref{25.5.7})和(\ref{25.5.3})给出
\begin{align}
    & \mathcal{Q}_{a}\,\lvert j,\sigma\rangle^{\flat} = \sum_{b} e_{ab}\,\mathcal{Q}_{b}^{\ast}\,\lvert j,\sigma\rangle \nonumber \\
    & \quad = \sqrt{2M}\sum_{b}e_{ab}\sum_{\pm}C_{\frac{1}{2}\,j}\Bigl(j\pm 1/2,\sigma+b\,;b,\sigma\Bigr)\,
    \lvert j\pm 1/2,\sigma+b\rangle \:. \label{25.5.12}
\end{align}
从方程(\ref{25.5.2}), (\ref{25.5.3})和(\ref{25.5.5})中我们得出
\begin{equation}
    \mathcal{Q}_{a}\,\lvert j\pm1/2,\sigma\rangle = \sqrt{2M}C_{\frac{1}{2}\,j}\Bigl(j\pm1/2,\sigma\,;a,\sigma-a\Bigr)\,
    \lvert j,\sigma-a\rangle \:, \label{25.5.13}
\end{equation}
而方程(\ref{25.5.5}), (\ref{25.2.31})和(\ref{25.5.7})给出
\begin{equation}
    \mathcal{Q}_{a}^{\ast}\,\lvert j\pm 1/2,\sigma\rangle = \sqrt{2M}\sum_{b}e_{ab}\,
    C_{\frac{1}{2}\,j}\Bigl(j\pm 1/2,\sigma\,;b,\sigma-b\Bigr)\,\lvert j,\sigma-b\rangle^{\flat}\:. \label{25.5.14}
\end{equation}
方程(\ref{25.5.3}), (\ref{25.5.8}), (\ref{25.5.10})和(\ref{25.5.12})---(\ref{25.5.14})给出了$\,\mathcal{Q}\,$和$\,\mathcal{Q}^{\ast}\,$在这个
超多重态中的所有态上的作用.

对于$\,j=0\,$我们有坍缩超多重态: 方程(\ref{25.5.3}), (\ref{25.5.8}), (\ref{25.5.10})和(\ref{25.5.12})---(\ref{25.5.14})变成
\begin{align}
    &\mathcal{Q}_{a}\,\lvert0,0\rangle =0 \:, && \mathcal{Q}_{a}^{\ast}\,\lvert0,0\rangle^{\flat} =0\:, \nonumber \\
    &\mathcal{Q}_{a}^{\ast}\,\lvert0,0\rangle =\sqrt{2M}\,\lvert 1/2\rangle\:,
    &&\mathcal{Q}_{a}\,\lvert0,0\rangle^{\flat}=\sqrt{2M}\,{\textstyle\sum}_{b}\,e_{ab}\,\lvert 1/2,b\rangle\:, \nonumber\\
    &\mathcal{Q}_{a}\,\lvert 1/2,b\rangle =\sqrt{2M}\,\delta_{ab}\,\lvert0,0\rangle \:,
    && \mathcal{Q}_{a}^{\ast}\,\lvert1/2,b\rangle=\sqrt{2M}\,e_{ab}\lvert 0,0\rangle^{\flat}\:. \label{25.5.15}
\end{align}


现在假定宇称是守恒的. 回忆, 我们可以选择超对称算符的相位使得宇称算符在这些生成元山的作用由方程(\ref{25.3.13})给定. 那么$\,\mathcal{Q}_{a}^{\ast}\,$作用在$\,\mathsf{P}\lvert j,\sigma\rangle\,$是态$\,\mathsf{P}\mathcal{Q}_{a}\lvert j,\sigma\rangle\,$的线性组合, 它们为零, 又由于$\,\mathsf{P}\lvert j,\sigma\rangle\,$和$\,\lvert j,\sigma\rangle\,$有着相同的选择性质, 它必须正比于它
\begin{equation}
    \mathsf{P}\lvert j,\sigma\rangle =-\eta \lvert j,\sigma \rangle^{\flat} \:. \label{25.5.16}
\end{equation}
由于$\,\mathsf{P}\,$是幺正的, $\eta\,$是满足$\,\lvert\eta\rvert =1\,$的相位因子. 对应的讨论表明$\,\mathsf{P}\lvert j,\sigma\rangle^{\flat}\,$正比于$\,\lvert j,\sigma$. 为了找到比例系数, 我们注意到
\begin{align*}
    \mathsf{P}\lvert j,\sigma\rangle &= (2M)^{-1}\mathsf{P}\,\mathcal{Q}_{\frac{1}{2}}^{\ast}\,\mathcal{Q}_{-\frac{1}{2}}^{\ast}\,
    \lvert j,\sigma \rangle = -\eta (2M)^{-1}\,\mathcal{Q}_{-\frac{1}{2}}\,\mathcal{Q}_{\frac{1}{2}}\,\lvert j,\sigma \rangle^{\flat}\\
    &=-\eta (2M)^{-2}\,\mathcal{Q}_{-\frac{1}{2}}\,\mathcal{Q}_{\frac{1}{2}}\,\mathcal{Q}_{\frac{1}{2}}^{\ast}\,\mathcal{Q}_{-\frac{1}{2}}^{\ast}\,
    \lvert j,\sigma\rangle = -\eta \lvert j,\sigma\rangle\:.
\end{align*}
这样我们就可以定义自旋$\,j\,$的态
\begin{equation}
    \lvert j,\sigma \rangle^{\pm} \equiv \frac{1}{\sqrt{2}}\Bigl(\lvert j,\sigma \rangle \pm
    \mi\,\lvert j,\sigma \rangle^{\flat}\Bigr) \:, \label{25.5.17}
\end{equation}
它们有确定的宇称
\begin{equation}
    \mathsf{P}\lvert j,\sigma \rangle^{\pm} =\pm \mi\eta \,\lvert j,\sigma \rangle^{\pm}\:.\label{25.5.18}
\end{equation}
最后, 用宇称算符作用方程(\ref{25.5.5})并使用方程(\ref{25.3.13})和(\ref{25.5.16}), 给出
\[
    \mathsf{P}\,\lvert j\pm 1/2,\sigma \rangle = -\frac{\eta}{\sqrt{2M}}
    \sum_{a}C_{\frac{1}{2}\,j}\Bigl(j\pm 1/2,\sigma\,; a,\sigma-a \Bigr)\,
    \sum_{b}e_{ab}\mathcal{Q}_{b}\,\lvert j,\sigma-a\rangle^{\flat} \:.
\]
那么方程(\ref{25.5.12})和\,Clebsch-Gordan\,系数的正交性给出
\begin{equation}
    \mathsf{P}\,\lvert j\pm 1/2,\sigma \rangle =\eta \,\lvert j\pm 1/2 ,\sigma \rangle\:, \label{25.5.19}
\end{equation}
这正是所要证明的.

我们现在简单地提一下有$\,N\,$个超对称生成元的扩充超对称性的情况. 正如上一节提及的, 对于任何中心荷, 不可能存在有非零本征值的无质量粒子. 我们可以更进一步并证明中心荷算符的的本征值为任何超多重态的质量提供了一个下界. 由于中心荷$\,Z_{rs}\,$和$\,Z_{rs}^{\ast}\,$彼此对易且与$\,P_{\mu}\,$对易, 单粒子态可以选择成所有中心荷和$\,P_{\mu}\,$的共同本征态, 又因为中心荷与$\,\mathcal{Q}_{ar}\,$和$\,\mathcal{Q}_{ar}^{\ast}\,$对易, 超多重态中的所有态拥有相同的本征值.

为了推导出将超多重态的质量$\,M\,$与中心荷在这个多重态上的本征值关联起来的不等式, 我们使用反对易关系(\ref{25.2.7})和(\ref{25.2.8})写下
\begin{align}
    &\sum_{ar} \Bigl\{\Bigl(\mathcal{Q}_{ar}-\sum_{bs}e_{ab}U_{rs}\mathcal{Q}_{bs}^{\ast}\Bigr)\:,
    \Bigl(\mathcal{Q}_{ar}^{\ast} - \sum_{ct} e_{ac} U_{rt}^{\ast}\mathcal{Q}_{ct}\Bigr) \Bigr\} \nonumber \\
    &\phantom{\sum_{ar} \Bigl\{\Bigl(\mathcal{Q}_{ar}}= 8NP^{0}
    - 2\operatorname{Tr}\Bigl(ZU^{\dag}+UZ^{\dag}\Bigr)\:, \label{25.5.20}
\end{align}
其中$\,U_{rs}\,$是一任意的$\,N\times N\,$幺正矩阵. 左边是正定算符, 通过让它作用在静止的超多重态上, 我们发现
\begin{equation}
    M\geq \frac{1}{4N}\operatorname{Tr}\Bigl(ZU^{\dag}+UZ^{\dag}\Bigr)\:, \label{25.5.21}
\end{equation}
其中$\,Z_{rs}\,$现在是指质量为$\,M\,$的超多重态的中心荷值. 极分解定理告诉我们任何方阵$\,Z\,$可以写成 $H\,V$, 其中$\,H\,$是正定厄米矩阵而$\,V\,$是幺正的. 通过令$\,U=V$, 我们可以获得一个有用的不等式(事实上是最理想的), 在这一情况下, 方程(\ref{25.5.21})变成
\begin{equation}
    M\geq \frac{1}{2N}\operatorname{Tr}H=\frac{1}{2N}\operatorname{Tr}\sqrt{Z^{\dag}Z}\:.\label{25.5.22}
\end{equation}
类比\,23.3\,节中讨论的\,Bogomol'nyi-Prasad-Sommerfeld\,磁单极构形, 在那里质量等于一般单极子质量下界的态被称为\,\emph{BPS}\,态, $M\,$等于这个不等式所允许的最小值的态被称为\,\textit{BPS}\,态. 事实上, 这不只是个类比; 我们将会在\,27.9\,节看到, 在有扩充超对称性的理论中, 单极子质量的下界是下界(\ref{25.5.22})的一个特殊情况.

从方程(\ref{25.5.22})的推导中可以看到, 对于\,BPS\,超多重态, 当算符$\,\mathcal{Q}_{ar}-\sum_{bs}e_{ab}U_{rs}\mathcal{Q}_{bs}^{\ast}\,$作用在这个超多重态中的任何态上时, 它给出零, 所以只有$\,N\,$个独立的螺旋度下降算符$\,\mathcal{Q}_{(1/2)\,r}\,$和$\,N\,$个独立的螺旋度上升%
算符$\,\mathcal{Q}_{(-1/2)\,r}$, 和无质量超多重态的情况相同. 这给出的超多重态要比一般情况下超多重态要小.

例如, 对于$\,N=2\,$超对称性, 中心荷由一个复数给定\footnote{在一些关于$\,N=2\,$超对称性的文章中, 那里的中心荷$\,Z\,$是我们的$\,Z/2\sqrt{2}$.}
\begin{equation}
    Z= \begin{pmatrix}
    0 & Z_{12} \\ -Z_{12} & 0
    \end{pmatrix} \:. \label{25.5.23}
\end{equation}
不等式(\ref{25.5.22})在这里是
\begin{equation}
    M\geq \lvert Z_{12} \rvert /2 \:. \label{25.5.24}
\end{equation}
当$\,M=\lvert Z_{12}\rvert /2\,$时, 有质量粒子超多重态的螺旋度列表与无质量的相同: 有规范超多重态, 它们由一个自旋$\,1\,$的粒子, 一个自旋$\,1/2\,$的$\,SU(2)\,R\,$-对称性双重态以及一个自旋$\,0\,$的粒子构成%
(另一个螺旋度为零的态属于自旋\,1\,粒子), 还有极多重态, 它们由一个自旋$\,1/2\,$的粒子和一个自旋\,0\,的$\,SU(2)$ $R\,$-对称性二重态构成. 为了与在$\,M>\lvert Z_{12}\rvert /2\,$时遇到的较长的超多重态相区分, 它们有时被称为``短''超多重态.


\section*{习题}
\noindent 1. 找到一组$\,2\times2\,$矩阵, 使得它们构成既包含费米生成元又包含玻色生成元的阶化\,Lie\,代数. \\

\noindent 2. 沿用\,Haag, Lopuszanski\,和\,Sohnius\,的方法, 推导出$\,2+1\,$维时空中最一般对称超代数的形式. (提示: 在将\,$2+1\,$维时空中的\,Lorentz\,群生成元标记成$\,A_{1}=-\mi J_{10}$, $A_{2}=-\mi J_{20}\,$和$\,A_{3}=J_{12}\,$后, Poincar\'{e}\, 代数的对易关系是$\,[A_{i},A_{j}]=\mi\sum_{k}\epsilon_{ijk}A_{k}$, 所以$\,2+1\,$维时空中的齐次\,Lorentz\,群的表示只用{\kai{一}}个整数或半整数指标$\,A\,$标记.) 在这里假定\,Coleman-Mandula\,定理成立的条件是满足的. \\

\noindent 3. 假定没有螺旋度大于$\,+3/2\,$或小于$\,-3/2\,$的无质量粒子. 找到$\,N=6\,$扩充超对称性和(使用$\,\mathsf{CPT}$ 对称性)$\,N=5\,$扩充超对称性的最一般的无质量粒子超多重态. 对这两个扩充超对称性, 你发现的这两个超多重态之间的差异说明了什么? \\

\noindent 4. 对于扩充$\,N=2\,$超对称性短超多重态中的粒子, 它们可能的宇称是什么?



%++++++++++++++++++参考文献+++++++++
\renewcommand{\sectionmark}[1]{\markright{ #1}{}}
\renewcommand{\bibname}{参考文献}

\begin{thebibliography}{99}
    \bibitem{1} R. Haag, J. T. Lopuszanski, and M. Sohnius, {\textit{Nucl. Phys.}} {\bf{B88}}, 257 (1975). 这篇文章重印于{\textit{Supersymmetry}}, S. Ferrar编辑(North Holland/World Scientific, Amsterdam/Singapore, 1987).
    \bibitem{2} B. Zumino, {\textit{Nucl. Phys.}} {\bf{B89}}, 535 (1975). 这篇文章重印于{\textit{Supersymmetry}}, 参考文献[1].
    \bibitem{3} M. T. Grisaru and H. N. Pendleton, {\textit{Phys. Lett.}} {\bf{67B}}, 323 (1977).
\end{thebibliography}
