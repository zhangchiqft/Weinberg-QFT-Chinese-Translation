\chapter{历史介绍} \label{cha:24}


如同科学史中的任何事物一样, 超对称自身的历史是非常独特的. 自\,20\,世纪\,70\,年代早期一提出, 超对称就被整合进一个优美的数学体系, 这个数学体系将不同自旋的粒子统一成对称性多重态并对基础物理有深远的影响. 迄今为止, 只有少数几个间接证据表明超对称与现实世界无关, 而直接证据则是一个都没有. 如果超对称最后确实与自然相关(也是我所期望的), 那么它将代表纯理论洞察的一个惊人的成功.

第\,25\,章将开始从第一原理出发来构建超对称理论. 在本章, 我们将沿着历史而非逻辑的顺序来介绍超对称.

\section{非传统的对称性与``止步''定理} \label{sec:24.1}

在\,20\,世纪\,60\,年代早期, (在\,19.7\,节讨论过的)Gell-Mann\,和\,Ne'eman\,的$\,SU(3)\,$对称性成果解释了各种强相互作用粒子之间的关系, 这些粒子有着不同的荷和奇异数, 但是它们的自旋相同. 这一想法然后发展为$\,SU(3)\,$也许是某个更大对称性的一部分, 这个更大对称性有非传统的效应, 它将不同自旋的$\,SU(3)\,$多重态统一在一起.\cite{1} 在非相对论性的夸克模型中有这样的近似对称性, 类比\,Wigner\,于\,1937\,年在核物理中引入的早期$\,SU(4)\,$对称性,\cite{2} 在夸克自旋和味道的$\,SU(6)\,$变换下有一近似对称性. 这一对称性将在本章附录\,A\,进行详细讨论, 它将赝标量介子八重态$\,\pi,K,\bar{K},\eta$, 矢量介子八重态$\,\rho,K^{\ast},\bar{K}^{\ast},\omega$, 以及矢量介子单态$\,\phi\,$统一进一个$\,\mathbf{35}\,$多重态, 并将自旋$\,1/2\,$重子八重态$\,N,\Sigma,\Lambda,\Theta\,$和自旋$\,3/2\,$的重子十重态%
$\,\Delta,\Sigma(1385),\Theta(1530),\Omega\,$统一进一个$\,\mathbf{56}\,$多重态. $SU(6)\,$对称性取得了数个成功, 但是它实际上只是夸克模型中的力与自旋和味道无关的结果; $SU(6)\,$对称性比自旋与味道无关性的假定还要稍微弱一些, 但是, 正如附录\,A\,所表明的, 与假定完全的自旋和味道无关性所给出的预测相比, 没有什么证据表明$\,SU(6)\,$对称性的预测要更精确.

然而, 曾有数次努力试图将非相对论夸克模型的$\,SU(6)\,$对称性推广到完全相对论性的量子理论.\cite{3} 这些尝试都失败了, 但是数个学者证明了, 在各种限制性假设下, 这实际上是不可能的.\cite{4} 这一类定理中最深远的那一个是\,Coleman\,和\,Mandula\,在\,1967\,年证明的一个定理.\cite{5} 他们采取了数个合理的假定, 包括在任意给定质量下只有有限多个粒子种类, 在几乎所有能标上都能发生散射, 以及$\,S\,$-矩阵的解析性, 然后使用这些假定证明了, 对于那些与$\,S\,$-矩阵对易, 那些将单粒子态变到单粒子态, 以及在多粒子态上的作用等价于在单粒子态上作用的直和, 对于这些对称性算符, 它们的最一般\,Lie\,代数的构成元素只有: Poincar\'{e}\,群的生成元$\,P_{\mu}\,$和$\,J_{\mu\nu}$, 再加上作用在单粒子态的内部对称性的生成元, 其中内部对称性生成元的矩阵与自旋和动量无关且关于它们是对角的. 在第\,25\,章分析四维时空中所有可能的超对称代数, 以及在第\,32\,章分析高维时空中所有可能的超对称代数时, 这个定理将是我们要使用的一个重要元素. 在\,32.3\,节, 我们将会在包含扩展物体的理论中考察超对称代数, 那里\,Coleman-Mandula\,定理是不成立的.

Coleman\,和\,Mandula\,的证明非常巧妙且比较复杂. 本章的附录\,B\,会给出这个证明的一个版本. 在本节, 我们仅对这个定理的一部分给出一个非常简单的纯运动学证明, 尽管要证明的只是\,Coleman-Mandula\,定理的一部分, 但是这部分已经可以非常清楚地说明为什么$\,SU(6)\,$这种非传统的对称性可以出现在非相对论性的理论%
却不能出现在相对论性的理论中. 我们将使用\,Lorentz\,不变性证明: 对于所有与动量生成元$\,P_{\mu}\,$的对称性算符$\,B_{\alpha}$, 如果它们的\,Lie\,代数的生成元只有$\,P_{\mu}\,$本身, 再加上某个只有有限个参量的半单紧致\footnote{关于半单紧致\,Lie\,代数的定义, 参看\,15.2\,节中的脚注.}%
\,Lie\,子代数$\,\mathscr{A}\,$的厄米生成元$\,B_{A}$, 那么$\,B_{A}\,$必须只能是某个内部对称性的生成元, 也就是说, 它们在单粒子态上的作用是一个关于动量和自旋对角且与动量和自旋无关的矩阵. 在这个定理的证明中没有使用$\,S\,$-矩阵的性质和粒子频谱的有限性, 对对称性算符是如何作用在物理态上也没有做假定. $SU(6)\,$的\,Lie\,代数当然是半单且紧致的, 所以这个定理表明, 在相对论性的理论中不能使用这样的对称性去推导不同自旋的粒子之间的关系.

下面是证明. 设所有与\,4\,-动量$\,P_{\mu}\,$对易的对称性生成元构成了一个\,Lie\,代数, 且这个\,Lie\,代数由生成元$\,B_{\alpha}\,$张开. 考察固有\,Lorentz\,变换$\,x^{\mu}\to \Lambda\indices{^\mu_\nu}\,$在这些生成元上的效应, 它在\,Hilbert\,空间中由幺正算符$\,U(\Lambda)\,$表示. 很容易看到算符$\,U(\Lambda)B_{\alpha}U^{-1}(\Lambda)\,$是与$\,\Lambda\indices{_\mu^\nu}P_{\nu}\,$%
对易的厄米对称性算符, 所以, 由于$\,\Lambda\indices{_\mu^\nu}\,$是非奇异的, 这个算符也必须与$\,P_{\mu}\,$对易, 因此必须是$\,B_{\alpha}\,$的线性组合:
\begin{equation}
U(\Lambda)B_{\alpha}U^{-1}(\Lambda)
=\sum_{\beta}D\indices{^{\beta}_{\alpha}}(\Lambda)B_{\beta} \:,  \label{24.1.1}
\end{equation}%
$D\indices{^{\beta}_{\alpha}}\,$是一组实系数, 它们构成了齐次\,Lorentz\,群的一个表示
\begin{equation}
D(\Lambda_{1})D(\Lambda_{2})=D(\Lambda_{1}\Lambda_{2}) \:. \label{24.1.2}
\end{equation}%
更进一步, $U(\Lambda)B_{\alpha}U^{-1}\Lambda\,$满足的对易关系与$\,B_{\alpha}\,$相同, 所以这个\,Lie\,代数的结构常数$\,C_{\alpha\beta}^{\gamma}\,$是不变张量, 也就是说
\begin{equation}
C_{\alpha \beta }^{\gamma} = \sum_{\alpha^{\prime}\beta^{\prime}\gamma^{\prime}}
D\indices{^{\alpha^{\prime}}_{\alpha}}(\Lambda)\, D\indices{^{\beta^{\prime}}_{\beta}}(\Lambda)\,
D\indices{^{\gamma}_{\gamma^{\prime}}}(\Lambda^{-1})\,
C_{\alpha^{\prime}\beta^{\prime}}^{\gamma^{\prime}}\:.  \label{24.1.3}
\end{equation}%
用$\,C_{\gamma\delta}^{\alpha}\,$的相应方程进行收缩, 我们发现
\begin{equation}
g_{\beta\delta} = \sum_{\beta^{\prime}\delta^{\prime}}
D\indices{^{\beta^{\prime}}_{\beta}}(\Lambda)\,
D\indices{^{\delta^{\prime}}_{\delta}}(\Lambda)\,g_{\beta^{\prime}\delta^{\prime}} \:,  \label{24.1.4}
\end{equation}%
其中$\,g_{\beta\delta}\,$是\,Lie\,代数度规
\begin{equation}
g_{\beta\delta} \equiv \sum_{\alpha\gamma}C_{\alpha\beta}^{\gamma}\,C_{\gamma\delta}^{\alpha}\:. \label{24.1.5}
\end{equation}%
由于所有这些生成元与$\,P_{\mu}\,$对应, 我们有$\,C_{\mu\beta}^{\alpha}=-C^{\alpha}_{\beta\mu}=0$, 所以$\,g_{\mu\alpha}=g_{\alpha\mu}=0$.

取代$\,\alpha,\beta,\cdots$, 我们用下标$\,A,B,\cdots\,$来标记$\,P_{\mu}\,$以外的对称性生成元. 在方程(\ref{24.1.5})中使用$\,C^{A}_{\mu B}=-C_{B\mu}^{A}\,$为零就给出了$\,g_{AB}=\sum_{CD}C_{AC}^{D}\,C_{BD}^{C}$. 我们假定了生成元$\,B_{A}\,$张成了紧致半单\,Lie\,代数, 所以矩阵$\,g_{AB}\,$是正定的. 方程(\ref{24.1.4})和(\ref{24.1.2})就表明矩阵$\,g^{1/2}D(\Lambda)g^{-1/2}\,$构成了齐次\,Lorentz\,群的有%
限维实正交表示, 也即有限维幺正表示. 但是, 因为\,Lorentz\,群是不紧的, {\kai{这样的表示只能是平庸表示}}, 对这样的表示$\,D(\Lambda)=1$. (这就是为什么相对论性产生差异的所有原因; 伽利略群的半单部分是{\kai{紧}}群$\,SU(2)$, 它显然有无限多个有限维幺正表示.) 有了$\,D(\Lambda)=1$, 对于所有\,Lorentz\,变换$\,\Lambda\indices{^{\mu}_{\nu}}$, 生成元$\,B_{A}\,$与$\,U(\Lambda)\,$对易.

对于动量为$\,p^{\mu}\,$的单粒子稳定态$\,\lvert p,n \rangle$, 其中离散指标$\,n\,$标记粒子的自旋和种类, 当$\,B_{A}\,$这种与$\,P_{\mu}\,$对易的算符作用在它上面时, 能够产生的只能是这种态的线性组合
\begin{equation}
B_{A}\lvert p,n\rangle =\sum_{m}\Bigl( b_{A}(p)\Bigr) _{mn}\lvert p,m\rangle \:. \label{24.1.6}
\end{equation}
$B_{A}\,$与我们在\,2.5\,节中所谓的``增速''对易, 这一性质暗示了$\,b_{A}(p)\,$与动量无关, 而$\,B_{A}\,$与旋转对易这一性质暗示了$\,b_{A}(p)\,$以单位矩阵的方式作用在自旋指标上, 正如我们所要证明的, 所以$\,B_{A}\,$是普通内部对称性的生成元.

\section{超对称的诞生} \label{sec:24.2}

如果理论物理的发展是遵循逻辑的, 那么在\,Coleman-Mandula\,定理证明之后, 我们应该试图去寻找这个定理的例外, 然后注意到这个定理只涉及玻色子到玻色子和费米子到费米子的变换, 因而它只由满足对易关系而非反对易关系的算符生成. 这就会给出一个问题: 一个相对论性的理论是否能够有一个作用在粒子自旋上的非平庸对称性, 这个对称性将玻色子和费米子变换到彼此. 沿着下一章描述的路线探索这种超代数的可能结构, 超对称性将作为唯一的可能性浮现出来.

历史上却并非如此. 相反, 超对称独立出现在弦论的一系列文章和两篇少为人知的文章中, 关于后者我们会在后面进一步讨论, 但是它们中的任何一篇都没有迹象表明作者关心过\,Coleman-Mandula\,定理.

自\,20\,世纪\,60\,年代后期, 对强相互作用过程就开始试图构建满足各种理论要求的$\,S\,$-矩阵元, 这种努力导致了各种强子的一个新图景, 即各种强子是弦振动的不同模式.\cite{6} 对于弦上由参量$\,\sigma\,$标记的那一点, 它在某个固定时钟的时刻$\,\tau\,$会有时空坐标$\,X^{\mu}(\sigma,\tau)$, 所以弦在$\,d\,$维时空中运动的理论可以认为是有$\,d\,$个玻色场的二维场论, 它的作用量是
\begin{align}
I[X] &= \frac{T}{2} \int \dif\sigma \int \dif\tau \:\eta_{\mu\nu}
\left[ \frac{\partial X^{\mu}}{\partial \tau}\,\frac{\partial X^{\nu}}{\partial \tau}
-\frac{\partial X^{\mu}}{\partial \sigma}\,\frac{\partial X^{\nu}}{\partial\sigma}\right]   \nonumber \\
&=T\int \dif\sigma^{+} \int \dif\sigma^{-}\:\eta_{\mu\nu}\,
\frac{\partial X^{\mu}}{\partial \sigma^{+}}\,\frac{\partial X^{\nu}}{\partial \sigma^{-}}\:, \label{24.2.1}
\end{align}%
其中$\,T\,$是被称为弦张力的常数; $\mu=0,1,\cdots,d-1$; $\sigma^{\pm}\,$是二维``光锥''坐标$\,\sigma^{\pm}\equiv \tau\pm\sigma$. 这个作用量可以从下面这个更一般的版本中推导出来,
\begin{equation}
I[X]= -\frac{T}{2} \int \dif^{2}\sigma \: \eta_{\mu \nu}\sqrt{\operatorname{Det}g}\,g^{kl}\,
\frac{\partial X^{\mu}}{\partial \sigma^{k}}\frac{\partial X^{\nu}}{\partial \sigma^{l}} \:, \label{24.2.2}
\end{equation}%
它在两个``世界面坐标''$\,\sigma_{k}\,$的变换下拥有完全的不变性,\footnote{对于玻色弦, 除非处在$\,d=26\,$维时空中, 或者在引入费米子后处在$\,d=10\,$维时空中, 同第\,22\,章中讨论的那些量子反常一样, 这个对称性会被量子反常破坏.} 通过回到一个特殊的坐标系, 使得世界面度规$\,g_{kl}\,$满足
\begin{equation}
\sqrt{\operatorname{Det}g}\,g^{kl}=
\begin{pmatrix}
1 & 0 \\
0 & -1%
\end{pmatrix} \:,   \label{24.2.3}
\end{equation}%
我们就能回到原来的作用量. 在电动力学中, 作用量对于类时光子是负号, 它带来的问题被理论的规范不变性消除了, 这里也是一样的, 方程(\ref{24.2.1})和(\ref{24.2.2})中的$\,\eta_{\mu\nu}\,$在$\,\mu=\nu=0\,$也是负号, 它带来的问题被作用量(\ref{24.2.2})(对于合适的边界条件)在广义世界面坐标变换下的不变性消除了. 在作用量取(\ref{24.2.1})的特殊坐标系下, 在广义世界面坐标变换下还残存着一个重要的不变性: 在定域{\kai{共形变换}}\footnoteB{原书误植为整体共形变换, 在二维情况下, 整体共形变换是分式线性变换$z\to (a+bz){\big/}(c+dz)$.\qquad ------译者注}下的不变性:
\begin{equation}
\sigma^{\pm} \to f^{\pm}(\sigma^{\pm}) \:,  \label{24.2.4}
\end{equation}%
其中$\,f^{\pm}\,$是两个独立的任意函数.

弦论描述的粒子并没有对上现实世界中的粒子. 在\,1971\,年, Ramond\cite{7}以及\,Neveu\,和\,Schwarz\cite{8}, 分别以引入半整数自旋的粒子和带有$\,\pi\,$子量子数的粒子为目的, 他们提出再加上$\,d\,$个费米场二元组$\,\psi_{1}^{\mu}(\sigma,\tau)\,$和$\,\psi_{2}^{\mu}(\sigma,\tau)$. 不久之后, Gervais\,和\,Sakita\cite{9}就为这个理论引入了一个作用量:
\begin{equation}
I[X,\psi] = \int \dif\sigma^{+} \int \dif\sigma^{-}
\left[ T\frac{\partial X^{\mu}}{\partial \sigma^{+}}\,\frac{\partial X_{\mu}}{\partial \sigma^{-}}%
+\mi\psi_{2}^{\mu}\frac{\partial}{\partial \sigma^{+}}\psi_{2\mu}
+\mi\psi_{1}^{\mu}\frac{\partial}{\partial \sigma^{-}}\psi_{1\mu}\right] \:,  \label{24.2.5}
\end{equation}%
注意到, 通过扩展共形变换(\ref{24.2.4}), 使其以如下的方式也作用在费米场上, 共性不变依旧是存在的,
\begin{equation}
\psi_{1}^{\mu} \to \left( \frac{\dif f^{+}}{\dif\sigma^{+}}\right)^{-1/2}\psi_{1}^{\mu} \:,\qquad\qquad
\psi_{2}^{\mu} \to \left( \frac{\dif f^{-}}{\dif\sigma^{-}}\right)^{-1/2}\psi_{2}^{\mu} \:. \label{24.2.6}
\end{equation}%
Gervais\,和\,Sakita\,指出, 除了二维共性不变性和$\,d\,$-维\,Lorentz\,不变性, 对于合适的边界条件, 这个理论在交换玻色场$\,X^{\mu}\,$和费米场$\,\psi_{r}^{\mu}\,$的无限小变换下有一个对称性
\begin{align}
\delta \psi_{2}^{\mu}(\sigma^{+},\sigma^{-}) &=\mi T\,\alpha_{2}(\sigma^{-})\,
\frac{\partial}{\partial \sigma^{-}}X^{\mu}(\sigma^{+},\sigma^{-}) \:,  \nonumber \\
\delta \psi_{1}^{\mu}(\sigma^{+},\sigma^{-}) &=\mi T\,\alpha_{1}(\sigma^{+})\,
\frac{\partial}{\partial \sigma^{+}}X^{\mu}(\sigma^{+},\sigma^{-}) \:,  \label{24.2.7} \\
\delta X^{\mu}(\sigma^{+},\sigma^{-}) &=\alpha_{2}(\sigma^{-})\,\psi_{2}^{\mu}(\sigma^{+},\sigma^{-})
+\alpha_{1}(\sigma^{+})\,\psi_{1}^{\mu}(\sigma^{+},\sigma^{-}) \:,  \nonumber
\end{align}%
其中$\,\alpha_{1}\,$和$\,\alpha_{2}\,$分别是$\,\sigma^{+}\,$和$\,\sigma^{-}\,$的无限小费米函数, 类似于\,9.5\,节引入的格拉斯曼变量. 这个对称性是后面被称为超对称性的一个例子, 这个对称性将费米子和玻色子联系在了一起, 但现在只是二维场论的一个对称性, 还不是四维时空中的物理理论的对称性.

 几年之后, Wess\,和\,Zumino\,回顾了参考文献[7-9]中提供的超对称性的例子, 并做了评述: 很自然地, 应该尝试将超对称的概念推广到{\kai{四}}维时空中的量子场论. 他们构建了几个超对称模型. 最简单的那一个包含一个\,Majorana\,(自荷共轭\,Dirac\,)场$\,\psi$, 实的标量和赝标量玻色场$\,A\,$和$\,B$, 实的标量和赝标量辅助玻色场$\,F\,$和$\,G$, 它在如下的无限小变换下是不变的\footnote{这里对\,Dirac\,矩阵使用的符号是前言和\,5.4\,节解释过的那些符号. 这里的$\,\gamma_{5}\,$(满足$\,\gamma_{5}^{2}=1$\,)是\,Wess\,和\,Zumino\,使用的$\,\gamma_{5}\,$乘以$\,\mi$, 对于任何旋量$\,\psi\,$的协变共轭$\,\bar{\psi}$, 这里的定义是\,Wess\,和\,Zumino\,的定义乘以$\,\mi$. 由于这个原因, 方程(\ref{24.2.8})---(\ref{24.2.10})中的一些相位与参考文献[10]中的那些相位不同.}
\begin{align}
\delta A &=\Bigl(\bar{\alpha}\,\psi\Bigr) \:, \qquad
\delta B=-\mi\Bigl(\bar{\alpha}\,\gamma_{5}\,\psi \Bigr) \:,  \nonumber \\
\delta \psi  &= \partial_{\mu}(A+\mi\gamma_{5}B)\gamma^{\mu}\alpha
+(F-\mi\gamma_{5}G)\alpha \:,  \label{24.2.8} \\
\delta F &=\Bigl(\bar{\alpha}\,\gamma^{\mu}\,\partial_{\mu}\psi \Bigr) \:,\qquad
\delta G=-\mi\Bigl( \bar{\alpha}\,\gamma_{5}\gamma^{\mu}\,\partial_{\mu}\psi \Bigr) \:,  \nonumber
\end{align}
其中$\,\alpha\,$是任意的无限小\,Majorana\,费米\,c\,-数常参量. 如果我么要求作用量在这些变换下不变, 那么用这些元素构造出来的最一般的Lorentz\,不变且宇称守恒的可重整实拉格朗日密度是
\begin{align}
\mathscr{L} &= -\tfrac{1}{2}\partial_{\mu} A\partial^{\mu }A
-\tfrac{1}{2}\partial_{\mu}B\partial^{\mu}B-\frac{1}{2}\bar{\psi}\gamma^{\mu}\partial_{\mu}\psi \nonumber \\
&\quad +\tfrac{1}{2}(F^{2}+G^{2}) + m[FA+GB-\tfrac{1}{2}\bar{\psi}\psi] \nonumber \\
&\quad +g\Bigl[ F(A^{2}+B^{2})+2GAB-\bar{\psi}(A+\mi\gamma_{5}B)\psi \Bigr] \:.  \label{24.2.9}
\end{align}%
由于辅助场$\,F\,$和$\,G\,$是以二次型的方式进入到理论中, 通过令它们等于场方程给出的值
\begin{equation}
F=-mA-g(A^{2}+B^{2})\:, \qquad G=-mB-2gAB  \: .  \label{24.2.10}
\end{equation}%
我们就能导出等价的拉格朗日量. 这样拉格朗日量就变成
\begin{align}
\mathscr{L} &= -\tfrac{1}{2}\partial_{\mu}A\partial^{\mu}A
-\tfrac{1}{2}\partial_{\mu}B\partial^{\mu}B-\tfrac{1}{2}\bar{\psi}\gamma^{\mu}\partial_{\mu}\psi \nonumber \\
&\quad-\tfrac{1}{2}m^{2}[A^{2}+B^{2}]-\tfrac{1}{2}m\bar{\psi}\psi   \nonumber \\
&\quad -gmA(A^{2}+B^{2}) - \tfrac{1}{2}g^{2}(A^{2}+B^{2})^{2}-g\bar{\psi}(A+\mi\gamma_{5}B)\psi \:. \label{24.2.11}
\end{align}
这个拉格朗日密度不仅呈现出标量质量和费米子质量相关, 并且呈现出\,Yukawa\,相互作用和标量自耦合相关, 这些正是超对称理论的特征. Wess\,和\,Zumino\,还描述了包含一个矢量场的超多重态的超对称变换, 并给出了拉格朗日量. (我们会在第\,26\,章更细致的讨论这些.) 最后, 在第二篇文章中, Wess\,和\,Zumino\cite{11}回顾了\,Coleman-Mandula\,定理, 并追溯出了对这一定理的明显破坏是由于这里的对称性生成元满足反对易关系而不是对易关系. 在几年之后, Gliozzi, Scherk\,和\,Olive\cite{11a}证明了, 通过给\,Ramond-Neveu-Schwarz\,模型的场附加合适的边界条件, 可以构造出一个既有时空超对称性又有世界面超对称性的超弦理论.

Wess\,和\,Zumino\,不知道的是, 在他们第一篇关于四维时空超对称的文章之前, 这个对称性就已经出现在发表在苏联的两篇文章中了. 在\,1971\,年, Gol'fand\,和\,Likhtman\,就已经将\,2.4\,节讨论的\,Poincar\'{e}\,群拓展至超代数, 并使用在这个超代数下不变的要求构建了四维时空中的超对称场论. 他们的文章虽然很有预见性, 但只给了少量的细节, 并在之后很长的一段时间内被普遍忽视了. Volkov\,和\,Akulov\cite{13}在\,1973\,年独立地发现了现在所谓的自发破缺超对称性, 但是他们用他们的体系将与超对称破缺相联系的\,Goldstone\,费米子同中微子等同起来, 这是一个注定失败的想法. 对于大多数理论物理学家, 尤其是苏联以外的理论物理学家, 超对称性能够作为四维时空中真实世界的一种可能的对称性, 是从\,Wess\,和\,Zumino\,他们\,1974\,年的文章开始的.

%+++++++++++++++++++++++附录A++++++
\titleformat{\chapter}{\centering\CJKfamily{zhhei}\huge}{\chaptertitlename}{1em}{}
\titlespacing{\chapter}{0pt}{3.5ex plus .1ex minus .2ex}{10\wordsep}
\titleformat{\section}{\centering\CJKfamily{zhhei}\Large}{附 录\thesection}{1em}{}
\titlespacing{\section}{2em}{3.5ex plus .1ex minus .2ex}{1.5\wordsep}
\titleformat{\subsection}{\centering\CJKfamily{zhhei}\large}{\thesubsection}{0em}{}
\titlespacing{\subsection}{2em}{1.5ex plus .1ex minus .2ex}{\wordsep}
\renewcommand{\captionfont}{\small}
\newcounter{app}[chapter]
\setcounter{app}{1}
\renewcommand\thesection{\Alph{app}}
\renewcommand\theequation{\arabic{chapter}.\Alph{app}.\arabic{equation}}
\fancyhf{}
\fancyhead[CE]{\leftmark}
\fancyhead[CO]{\rightmark}
\fancyhead[RO,LE]{$\cdot$\ \thepage\ $\cdot$}
\renewcommand{\headrulewidth}{0.8pt}
\pagestyle{fancy}
\renewcommand{\chaptermark}[1]{\markboth{第\,\thechapter\,章\ #1}{}}
\renewcommand{\sectionmark}[1]{\markright{附录 \thesection\quad\ #1}{}}

\section{非相对论夸克模型的$SU(6)$对称性}

在这个附录中, 我们将描述$\,SU(6)\,$对称性是如何将非相对论夸克模型中各种自旋的粒子关联起来的. 这与超对称性没有直接关系, 但是它为\,Coleman-Mandula\,定理提供了一个历史背景, 而这对于\,25.1\,节和\,31.1\,节中一般超对称代数的构造是一个重要的输入信息.

一般而言, 非相对论夸克模型的哈密顿量不仅依赖动量和位置, 而且依赖自旋算符和味算符 $\,\sigma_{i}^{(n)}\,$和$\,\lambda_{A}^{(n)}$, 其中$\,\sigma_{i}^{(n)}\,$(\,$i=1,2,3$\,)就作为方程(\textcolor{foo}{5.4.18})中定义%
的\,Pauli\,矩阵$\,\sigma_{i}\,$作用在第$\,n\,$个夸克的自旋指标上, 而$\,\lambda_{A}^{(n)}\,$(\,$A=1,2,\cdots,8$\,)就作为方程(\textcolor{foo}{19.7.2})中定义的\,Gell-Manna\,%
$SU(3)\,$矩阵$\,\lambda_{A}\,$作用在第$\,n\,$个夸克的味指标上. (当$\,n\,$指代反夸克时, $\sigma_{i}^{(n)}\,$和$\,\lambda_{A}^{(n)}\,$就作为逆步表示的矩阵$\,-\sigma_{i}^{\mathrm{T}}\,$%
和$\,-\lambda_{A}^{\mathrm{T}}$ 进行作用.) 如果我们假定没有自旋-轨道耦合, 使得总轨道角动量$\,L_{i}\,$分别守恒, 那么我们可以得出: 哈密顿量与总自旋和总幺旋(\,unitary spin\,)
\begin{equation}
S_{i}\equiv \tfrac{1}{2}\sum_{n}\sigma_{i}^{(n)}\:,\qquad\qquad
T_{A}\equiv \tfrac{1}{2}\sum_{n}\lambda_{A}^{(n)} \:,  \label{24.A.1}
\end{equation}%
以及$\,L_{i}\,$对易. 另一方面, 如果我们假定哈密顿量只依赖于夸克的动量和位置, 但是与自旋和夸克味完全无关, 那么这样的哈密顿量不仅与总轨道角动量$\,\mathbf{L}\,$对易, 还与算符$\,\sigma_{i}^{(n)}\,$和$\,\lambda_{A}^{(n)}\,$中的{\kai{每一}}个都分别对易. 在这两种极端情况之间有一种有趣的可能性,  除了与$\,L_{i}$, $S_{i}\,$和$\,T_{A}\,$对易外, 哈密顿量还与如下算符对易
\begin{equation}
R_{iA}\equiv \tfrac{1}{2}\sum_{n}\pm \sigma_{i}^{(n)}\lambda_{A}^{(n)} \:,  \label{24.A.2}
\end{equation}%
其中符号对于夸克和反夸克分别是\,$+$\,号和$\,-\,$号.\footnote{对于反夸克产生负号是因为: 对于反夸克, $R_{iA}\,$中的项必须作为矩阵$\,-(\sigma_{i}\lambda_{A})^{\mathrm{T}}=
-(-\sigma_{i}^{\mathrm{T}})(-\lambda_{A}^{\mathrm{T}})\,$作用在自旋指标和味指标上} $S_{i}$, $T_{A}\,$和$\,R_{iA}\,$构成了$\,SU(6)\,$群的\,Lie\,代数, 对易关系是
\begin{align}
&[S_{i},S_{j}] =\mi\sum_{k}\epsilon_{ijk}S_{k} \: ,\qquad%
[T_{A},T_{B}]=\mi\sum_{C}f_{ABC}T_{C}\:,\qquad [S_{i},T_{A}]=0\:, \nonumber \\
&[S_{i},R_{jA}] =\mi\sum_{k}\epsilon _{ijk}R_{kA} \:,\qquad%
[T_{A},R_{iB}]=\mi\sum_{C}f_{ABC}R_{iC} \:, \label{24.A.3} \\
&[R_{Ai},R_{Bj}] =\mi\delta _{ij}\sum_{C}f_{ABC}T_{C}+\tfrac{2}{3}%
\mi\delta _{AB}\sum_{k}\epsilon _{ijk}S_{k}+\mi\sum_{kC}\epsilon_{ijk}d_{ABC}R_{kC} \:,  \nonumber
\end{align}%
这里的$\,f_{ABC}\,$和$\,d_{ABC}\,$分别是全反对称和完全对称的数值系数,\cite{14} 其中独立且不为零的值是
\begin{equation}
\begin{split}
f_{123} &=1\:, \qquad f_{458}=f_{678}=\sqrt{3}/2 \:,  \\
f_{147} &=f_{165}=f_{246}=f_{257}=f_{345}=f_{376}=1/2  \end{split}  \label{24.A.4}
\end{equation}
和
\begin{align}
d_{146} &=d_{157}=-d_{247}=d_{256}=d_{344}=d_{355}=-d_{366}=-d_{377}=1/2 \:,  \nonumber \\
d_{118} &=d_{228}=d_{338}=-d_{888}=1/\sqrt{3} \:,  \label{24.A.5} \\
d_{448} &=d_{558}=d_{668}=d_{778}=-1/(2\sqrt{3}) \:.  \nonumber
\end{align}%
如果我们在哈密顿量中引入的是与$\,R_{iA}$, $S_{i}\,$和$\,T_{A}\,$相对易的两体相互作用, 这一对称性会保留下来. 这样的相互作用是存在的, 由如下形式的两体算符的线性组合给出
\begin{equation}
H^{(nm)}\propto \left[ 1\pm \sum_{i}\sigma_{i}^{(n)}\sigma_{i}^{(m)}\right]
\left[ \frac{2}{3}\pm \sum_{A}\lambda_{A}^{(n)}\lambda_{A}^{(m)}\right] \: ,  \label{24.A.6}
\end{equation}%
其中, 如果粒子$\,n,m\,$中一个是夸克而另一个是反夸克, 符号$\,\pm\,$取负号, 如果都是夸克或都是反夸克则取正号.

当然, 哪怕是在非相对论夸克模型中, $SU(6)\,$对称性最多也只是个近似对称性. 它被自旋-轨道耦合, 自旋-自旋力以及$\,s\,$夸克的质量破缺掉了, 而后者将$\,SU(3)\,$味对称性降成同位旋守恒和超荷(hypercharge)\footnoteB{在本卷的翻译过程中, 有这样一个问题, 粒子物理中的守恒荷\,hypercharge\,和超对称守恒荷\,supercharge\,译成中文均是超荷, 为了区分, 译者在这两种荷后面附上英文以便区分.\qquad ------译者注}守恒的$\,SU(2)\,$和$\,U(1)$. 如果我们限制在那些只由$\,u\,$夸克, $d\,$夸克和它们的反夸克这种轻夸克构成的强子, 这样就能夸克的质量差带来的影响, 这时不为零的$\,\lambda_{A}\,$矩阵是$\,a=1,2,3\,$的$\,\lambda_{a}\,$%
(这些矩阵由\,Pauli\,矩阵(\textcolor{foo}{5.4.18})给出, 在这一情形下通常记做$\,\tau_{a}$)和$\,\lambda_{8}$(对于$\,u\,$夸克和$\,d\,$夸克只是数字$\,1/\sqrt{3}$, 对于反夸克$\,\bar{u}\,$和$\,\bar{d}\,$是$\,-1/\sqrt{3}$). 相互作用(\ref{24.A.6})因此变成
\begin{equation}
H^{(nm)}\propto \left[ 1\pm \sum_{i}\sigma_{i}^{(n)}\sigma_{i}^{(m)}\right]
\left[ 1\pm \sum_{A}\tau _{A}^{(n)}\tau _{A}^{(m)}\right] \:.
\label{24.A.7}
\end{equation}%
除了夸克味守恒外, 剩下的对称性是$\,SU(4)$, 生成元是与(\ref{24.A.7})对易的$\,S_{i}$, $T_{a}\,$和$\,R_{ia}$. 在\,1937\,年, Wigner\,提出它可能是核力的一种对称性, 当然, 那时认为这是质子和中子的对称性, 而非$\,u\,$夸克和$\,d\,$夸克. 在核力理论中, 不依赖于自旋或同位旋的相互作用被称为\,{\textit{Wigner}\,\kai{势}}, 正比于(\ref{24.A.7})自旋部分的相互作用被称为\,{\textit{Bartlett}\,\kai{势}}, 只正比于(\ref{24.A.7})同位旋部分的相互作用被称为\,{\textit{Heisenberg}\,\kai{势}}, 与这三者相区分, 相互作用(\ref{24.A.7})被称为\,\textit{Majorana}{\kai{势}}.

有趣的是, 尽管像$\,SU(6)\,$这种既作用在自旋上又作用在粒子种类上的对称性在非相对论理论中没有任何理论障碍, {\kai{但是, 没有任何证据表明, 相较于完全独立于自旋和味的假定, 非相对论夸克模型更好地满足这种$\,SU(6)\,$对称性.}} 这些假定是不同的; 如果\,$N$\,个非相对论夸克和(或)反夸克组成的系统的哈密顿量完全独立于自旋和味, 那么它的对称性不是$\,SU(6)\,$而是$\,SU(6)^{N}$. 例如像(\ref{24.A.6})的两体相互作用以及各种其它多粒子相互作用会把$\,SU(6)^{N}\,$破缺至$\,SU(6)$. 当然, 所有这些对称性都只是个近似. 问题是$\,SU(6)\,$被破坏的程度是不是要弱于$\,SU(6)^{N}$?

要想回答这个问题不能去研究包含重子八重态的多重态, 即由核子和重核子$\,\Lambda$, $\Sigma\,$和$\,\Xi\,$构成的多重态. 在非相对论夸克模型中, 这些粒子被解释成三个夸克的轨道角动量为零的束缚态. 由于这些态是色中性的, 波函数关于没写出的色指标是全反对称的, 所以它在自旋和味的组合交换下是完全对称的. 因此, 重子八重态必须被放进$\,SU(6)\,$的对称三阶张量表示$\,\mathbf{56}\,$中, 而这个表示除了重子八重态外还包含一个自旋\,3/2\,十重态, 这个十重态可以被认为是由著名的``3-3''共振$\,\Delta$以及粒子$\,\Sigma(1385)$, $\Xi(1530)\,$和$\,\Omega\,$构成的那个十重态. (括号中的数字是以$\,\mathrm{Mev}\,$为单位的粒子质量, 写出它们是为了与那些同位旋和奇异数相同但质量更低的粒子进行区分.) $SU(6)\,$对称性对重子磁矩给出了一个很好的预测: 夸克荷算符是$\,q=e(\lambda_{3}/2+\lambda_{8}/2\sqrt{3})$, 而有这样的荷且质量为$\,m_{N}/3\,$的\,Dirac\,粒子的磁矩是$\,3q/2m_{N}$, 如果夸克拥有磁矩$\,3q/2m_{N}$, 那么磁矩算符是
\[
\mu _{i}=3\mu_{N}\left[ \frac{1}{2}R_{i3}+\frac{1}{2\sqrt{3}}R_{i8}\right] \: ,
\]
其中$\,\mu_{N}\equiv e/2m_{N}\,$是核磁矩, $R_{iA}\,$由方程(\ref{24.A.2})定义. 计算这个对称性生成元在$\,\mathbf{56}\,$多重态的成员之间的矩阵元是直接的, 以$\,\mu_{N}\,$为单位, 对于$\,p$, $n$, $\Lambda$, $\Sigma^{+}$, $\Sigma^{-}$, $\Xi^{-}\,$和$\,\Xi^{0}$, 磁矩的计算结果分别是$\,+3$, $-2$, $-1$, $+3$, $-1$, $-1\,$和$\,-2$, 可以与相应的实验值$\,+2.79$, $-1.91$, $-0.61$, $+2.46$, $-1.16$, $-0.65\,$和$\,-1.25\,$进行比较. 与实验的误差是可接受的, 如果我们取夸克的磁矩比$\,3\mu_{N}\,$稍小一些, 结果(除了$\,\Sigma^{-}\,$)还要更好些. 由于\,3\,-夸克波函数的对称性, 如果假定哈密顿量完全独立于自旋和味, 这里并没有新东西; 角动量为零的态仍然会落进由$\,6\times 7\times 8/6!=56\,$个成员构成的多重态中. 特别地, 对于两个夸克的任意态, 只要它在自旋和味的同时交换下是对称的------无论在自旋的交换和味的交换下均为对称还是反对称, 算符(\ref{24.A.6})都有相同的值$\,4$.

为了判定$\,SU(6)^{N}\,$是否比$\,SU(6)\,$要好些, 研究介子要更有用些, 在非相对论夸克模型中, 它们被解释成夸克和反夸克的束缚态. 如果这些态的哈密顿量完全独立于自旋和味, 那么它的对称性是$\,SU(6)^{2}$, 介子态就落入它的\,36\,维表示$\,(6,\bar{6})\,$中, 然而对于$\,SU(6)\,$对称性, 我们只能说介子属于$\,\mathbf{6}\times \bar{\mathbf{6}}\,$中包含的两个$\,SU(6)\,$表示中的一个: 伴随表示$\,\mathbf{35}\,$或单态表示. 更具体些, 组成$\,\mathbf{35}\,$的是一个自旋$\,S=1\,$的$\,SU(3)\,$单态, 一个自旋$\,S=0\,$的$\,SU(3)\,$八重态, 以及一个自旋$\,S=1\,$的$\,SU(3)\,$八重态, 对应于$\,SU(6)\,$生成元$\,S_{i}$, $T_{A}\,$和$\,R_{iA}$, 它通过相互作用(\ref{24.A.6})与$\,S=0\,$的$\,SU(3)\,$单态分离开来. 既然所有这些假定都只是个近似, 判定$\,SU(6)\,$对称性是否要比完全独立于自旋和味更加精确这个问题就变成: $S=0\,$的$\,SU(3)\,$单态与其它轨道角动量相同的$\,35\,$个态的分离程度是否比%
$\,\mathbf{35}\,$超多重态内部的分离程度大.

当轨道角动量$\,L=0\,$时, 夸克-反夸克态的宇称$\,\mathsf{P}\,$为负, 再根据总自旋$\,S\,$是零还是一, 它的荷共轭量子数$\,\mathsf{C}\,$(对于自荷共轭态)分别为正和负. (关于这一点的解释, 参看\,5.5\,节.) 因此组成$\,\mathbf{35}\,$的是一个$\,J^{\mathsf{PC}}=1^{--}\,$的单态, 一个$\,0^{-+}\,$八重态, 以及一个$\,1^{--}\,$八重态, 它们可以被认为是: $\phi(1020)$; 赝标量八重态$\,\pi$, $\eta$, $K\,$和$\,\bar{K}$; 以及矢量八重态$\,\rho$, $\omega$, $K^{\ast}\,$和$\,\bar{K}^{\ast}$. 在$\,958\,\mathrm{MeV}\,$处还存在一个$\,0^{-+}\,$的$\,SU(3)\,$单态$\,\eta^{\prime}$, 可以认为它是$\,SU(6)\,$单态. 这个单态与$\,\mathbf{35}\,$多重态中的粒子的分离程度并没有比$\,\mathbf{35}\,$多重态内部的分离大多少.

有人可能会说$\,L=0\,$的介子并不能给非相对论模型的对称性提供一个很好的检验, 它们包含\,Goldstone\,玻色子$\,\pi$, $\eta$, $K\,$ 和$\,\bar{K}$, 而这些粒子在$\,u\,$夸克和$\,d\,$夸克质量为零的情况下变成无质量粒子, 因而不能被这一模型很好的描述. 那么我们来考虑$\,L=1\,$的夸克-反夸克态. 这些态的$\,\mathsf{P}\,$为正, $\mathsf{C}\,$根据$\,S=1\,$还是$\,S=0\,$分别为正或负, 所以组成\,p\,-波$\,\mathbf{35}\,$的是: $S=1\,$且$\,J^{\mathsf{PC}}=0^{++},\,1^{++},\,2^{++}$的\\$\,SU(3)\,$单态, 它们可以被认为是$\,f_{0}(1370)$, $f_{1}(1285)\,$和$\,f_{2}(1270)$; $S=0\,$的$\,1^{+-}\,$八重态, 它们可以被认为是$\,h_{1}(1170)$, $b_{1}(1235)$, $K_{1}(1400)\,$和$\,\bar{K}_{1}(1400)$; 以及$\,S=1\,$的八重态: 由$\,f_{0}(980)$, $a_{0}(980)$, $K_{0}^{+}(1950)\,$和$\,\bar{K}_{0}^{+}(1950)\,$构成的$\,0^{++}\,$八重态; $f_{1}(1420)$, $a_{1}(1260)$, $K_{1}^{+}(1650)\,$ 和$\,\bar{K}_{1}^{+}(1650)\,$构成的$\,1^{++}\,$\\八重态; 以及$f_{2}(1430)$, $a_{2}(1320)$, $K_{2}^{+}(1980)\,$和$\,\bar{K}_{2}^{+}(1980)\,$构成的$\,2^{++}\,$八重态. 除了这$\,35\times 3\,$个态外, 还有另外一个粒子拥有成为\,p\,-波$\,SU(6)\,$单态的正确量子数: $1^{+-}\,$的同位旋标量$\,h_{1}(1380)$. 当然, 我们可以交换$\,h_{1}(1170)\,$和$\,h_{1}(1380)\,$的身份, 或者认为$\,SU(3)\,$单态和八重态同位旋标量$\,1^{+-}\,$态是%
$\,h_{1}(1170)\,$和$\,h_{1}(1380)\,$的两个相互正交的线性组合. 问题的重点在于: 有{\kai{两个}}$\,1^{+-}\,$同位旋标量, 我们无法指出其中那一个属于$\,SU(6)\,$单态, 就更谈不上说$\,SU(6)\,$单态中的粒子与$\,\mathbf{35}\,$中的粒子的分离程度要强于$\,\mathbf{35}\,$内部的分离程度. 那么, 这里依旧没有什么证据表明$\,SU(6)\,$对称性要比完全独立于自旋和味这个假定更加精确.

\setcounter{app}{2}

\section{Coleman-Mandula定理}

这个附录将为著名的\,Coleman-Mandula\,定理\cite{5}给出一个证明, 即, 对称性生成元的唯一可能\\ Lie\,代数(与超代数相反)的组成部分是: 平移和齐次\,Lorentz\,变换的生成元$\,P_{\mu}\,$和$\,J_{\mu\nu}$, 以及可能的内部对称性的生成元, 其中内部对称性的生成元与$\,P_{\mu}\,$和$\,J_{\mu\nu}\,$对易, 并且它们作用在物理态上的方式是用一个与自旋和动量无关的厄米矩阵乘这些态.\footnote{我们将会看到, 在只有无质量粒子的理论中, 除了生成元$\,P_{\mu}\,$和$\,J_{\mu\nu}\,$外还可能存在额外的生成元$\,D\,$和$\,K_{\mu}$, 它们填充了共形群的\,Lie\,代数.\cite{15}} 这里的``对称性生成元''是指任何满足如下条件的厄米算符: 与$\,S\,$-矩阵对易; 对易子也是对称性生成元; 将单粒子态变到另一个单粒子态; 在多粒子态上的作用是它们在单粒子态上的作用的直和(就像方程(\ref{24.B.1})中那样). 一个更加技巧性的要求会在后面需要的时候加上来. 除了第\,2\,章和第\,3\,章所描述的相对论量子力学的一般原理外, 这个证明还需要的其它假设只有:
\begin{assumption}
对于任意$\,M$, 只有有限种粒子的质量小于$\,M$.
\end{assumption}
\begin{assumption}
任何二粒子态在几乎所有能量(即, 除了一个离散集外的所有能量)处都能发生某个反应. \label{assum:2}
\end{assumption}
\begin{assumption}
弹性两体散射的振幅在几乎所有能量和角度处都是散射角的解析函数.\footnote{\song{严格来讲, 在有红外发散的理论中, 例如量子电动力学, 这个假定是不被满足, 而在电动力学中, 我们在\,13.3\,节看到, 除了向前弹性散射外, 对于任何一个包含带电粒子的散射过程, 它的$\,S\,$-矩阵元实际上为零. 在电动力学这样的阿贝尔规范理论中, 通过对带有虚拟规范玻色子质量的理论使用\,Coleman-Mandula\,定理, 并且只处理那些``红外安全''的量, 例如质量以及一些合适的积分截面------在规范玻色子质量为零的极限下保持有限的截面, 这个问题可以被规避掉. 在量子色动力学这种所有无质量粒子都是陷俘的非阿贝尔规范理论中是没有问题的------%
对称性如果没有破缺的话将只能决定规范中性束缚态的$\,S\,$-矩阵元, 例如量子色动力学中的介子和重子. 就我所知, 在有非陷俘无质量粒子的非阿贝尔规范理论中, 例如带有多个夸克味的量子色动力学, Coleman-Mandula\,定理还没有被证明.}}
\end{assumption}
\noindent 不需要假定$\,S\,$-矩阵由一个定域量子场论决定. 这里给出的证明在一定程度上重新安排并进行了梳理, 并且补充了一些\,Coleman\,和\,Mandula\,留给读者的步骤.

证明这个定理的一个比较方便的出发点是:
从那些与\,4\,-动量算符$\,P_{\mu}\,$对易的对称性生成元$\,B_{\alpha}$\\ 组成的子代数着手处理. (定理的这个部分本身是很有意思的; 它在相对论理论中排除了作用方式类似于非相对论夸克模型的$\,SU(6)\,$对称性这样的对称性.)  这种对称性生成元在多粒子态上的作用是
\begin{align}
&B_{\alpha }\,\lvert p\,m,q\,n,\cdots \rangle  =\sum_{m^{\prime }}\Bigl( b_{\alpha
}(p)\Bigr) _{m^{\prime }m}\lvert p\,m^{\prime },q\,n,\cdots \rangle   \nonumber \\
&\qquad\qquad+\sum_{n^{\prime }}\Bigl( b_{\alpha }(q)\Bigr) _{n^{\prime
}n}\lvert p\,m,q\,n^{\prime },\cdots \rangle +\cdots \:,  \label{24.B.1}
\end{align}%
其中$\,m,n\,$等是标记拥有确定质量$\,\sqrt{-p_{\mu}p^{\mu}}\,$的粒子的自旋$\,z\,$-分量和粒子种类的离散指标, 而$\,b_{\alpha}(p)$ 是有限厄米矩阵, 它定义了$\,B_{\alpha}\,$在单粒子态上的作用.

现在, 从方程(\ref{24.B.1})我们可以看到, 对于固定的$\,p$, 将$\,B_{\alpha}\,$映到$\,b_{\alpha}(p)\,$的映射是个同态, 也就是说, $B_{\alpha}\,$满足的对易关系
\begin{equation}
[B_{\alpha},B_{\beta }]=\mi\sum_{\gamma}C_{\alpha\beta}^{\gamma}B_{\gamma}  \label{24.B.2}
\end{equation}%
也是厄米矩阵$\,b_{\alpha}(p)\,$满足的对易关系:
\begin{equation}
[b_{\alpha}(p),b_{\beta}(q)]=\mi\sum_{\gamma}C_{\alpha \beta}^{\gamma}\,b_{\gamma }(p) \:.  \label{24.B.3}
\end{equation}%
15.2\,节证明的一个著名定理告诉我们, 像$\,b_{\alpha}(p)\,$这样的有限厄米矩阵, 它的任何\,Lie\,代数必须是紧致半单\,Lie\,代数和$\,U(1)\,$代数的直和. 然而, 由于算符$\,B_{\alpha}\,$和矩阵$\,b_{\alpha}(p)\,$之间的同态不一定是个同构, 我们并不能直接应用这个定理. 为了使它成为一个同构, 我们还得要求, 只要对某些系数$\,c^{\alpha}\,$和某个动量$\,p\,$有$\,\sum_{\alpha}c^{\alpha}b_{\alpha}(p)=0$, 那么对所有动量$\,k\,$都有$\,\sum_{\alpha}c^{\alpha}b_{\alpha}(k)=0$, 这等价于条件$\,\,\sum_{\alpha}c^{\alpha}B_{\alpha}$ $=0$.

取代考察将$\,B_{\alpha}\,$映射到单粒子矩阵$\,b_{\alpha}(p)\,$的同态, Coleman\,和\,Mandula\,考察的同态是将$\,B_{\alpha}\,$映射到定义了$\,B_{\alpha}\,$在{\kai{二粒子态}}上的作用的矩阵, 若这个二粒子态有固定的\,4\,-动量$\,p\,$和$\,q$, 那么这个矩阵是:
\begin{equation}
\Bigl(b_{\alpha}(p,q)\Bigr)_{m^{\prime}n^{\prime},mn}
=\Bigl( b_{\alpha}(p)\Bigr)_{m^{\prime}m}\delta_{n^{\prime}n}
+\Bigl( b_{\alpha}(q)\Bigr)_{n^{\prime}n}\delta_{m^{\prime}m} \:.  \label{24.B.4}
\end{equation}%
对于两个\,4\,-动量为$\,p\,$和$\,q\,$的粒子到两个\,4\,动量为$\,p^{\prime}\,$和$\,q^{\prime}\,$%
的粒子的弹性或准弹性散射, 其中这两个粒子有质量$\,\sqrt{-p_{\mu}^{\prime}p^{\prime\mu}}=\sqrt{-p_{\mu}p^{\mu}}\,$%
和$\,\sqrt{-q_{\mu}^{\prime}q^{\prime\mu}}=\sqrt{-q_{\mu}q^{\mu}}$, $S\,$-矩阵的不变性给出条件
\begin{equation}
b_{\alpha}(p^{\prime},q^{\prime})\,S(p^{\prime},q^{\prime};p,q)
=S(p^{\prime},q^{\prime};p,q)\,b_{\alpha}(p,q) \: . \label{24.B.5}
\end{equation}%
这里的$\,S(p^{\prime},q^{\prime};p,q)\,$是一个与$\,b(p,q)\,$和$\,b(p^{\prime},q^{\prime})\,$量纲相同的矩阵, 以连通\,$S$\,-矩阵元$\,S(p\,m,q\,n\to p^{\prime}\,m^{\prime},q^{\prime}\,n^{\prime})\,$的形式, 它被定义成
\begin{equation}
S(p\,m,q\,n\rightarrow p^{\prime }\,m^{\prime },q^{\prime }n^{\prime })\equiv
\delta ^{4}(p^{\prime }+q^{\prime }-p-q)\,\Bigl( S(p^{\prime },q^{\prime
};p,q)\Bigr)_{m^{\prime }n^{\prime },mn} \:.  \label{24.B.6}
\end{equation}%
根据假设\,2\,和光学定理(见\,3.6\,节), 对于几乎任何$\,p\,$和$\,q\,$的选择, 弹性散射振幅在向前方向上是不为零的, 那么根据假设\,3, 对于几乎所有满足守恒条件以及在壳的$\,p^{\prime}\,$和$\,q^{\prime}$, 矩阵$\,S(p^{\prime},q^{\prime};p,q)\,$是非奇异的, 所以对于几乎所有这样的\,4\,-动量, 方程(\ref{24.B.5})是个{\kai{相似变换}}.

由此可以得出, 如果对于几乎任意的固定\,4\,-动量$\,p\,$和$\,q\,$有$\,\sum_{\alpha}c^{\alpha}b_{\alpha}(p,q)=0$, 那么对于几乎任何处在同一质壳且满足$\,p^{\prime}+q^{\prime}=p+q\,$的$\,p^{\prime}\,$和$\,q^{\prime}\,$有%
$\,\sum_{\alpha}c^{\alpha}b_{\alpha}(p^{\prime},q^{\prime})=0$. 不幸的是, 这并不能告诉我们$\,\sum_{\alpha}c^{\alpha}b_{\alpha}(p^{\prime})\,$和%
$\,\sum_{\alpha}c^{\alpha}b_{\alpha}(q^{\prime})\,$为零, 只是告诉我们这两个矩阵正比于单位矩阵(比例系数相反). 为了做的更好些, 我们要考察的既不是$\,b_{\alpha}(p)\,$也不是$\,b_{\alpha}(p,q)$, 而是它们的无迹部分.

方程(\ref{24.B.5})的一个直接结果是
\begin{equation}
\operatorname{Tr}b_{\alpha }(p^{\prime},q^{\prime})=\operatorname{Tr}b_{\alpha}(p,q)\:.\label{24.B.7}
\end{equation}%
再加上方程(\ref{24.B.4}), 这告诉我们
\begin{align}
&N(\sqrt{-q_{\mu}q^{\mu}})\operatorname{tr}b_{\alpha}(p^{\prime})
+N(\sqrt{-p_{\mu}p^{\mu}})\operatorname{tr}b_{\alpha}(q^{\prime})  \nonumber \\
&\quad=N(\sqrt{-q_{\mu}q^{\mu}})\operatorname{tr}b_{\alpha }(p)
+N(\sqrt{-p_{\mu}p^{\mu}})\operatorname{tr}b_{\alpha}(q) \:,  \label{24.B.8}
\end{align}%
其中$\,N(m)\,$是质量为$\,m\,$的粒子种类的多重数,\footnote{Coleman\,和\,Mandula\,没有显式地写出这些多重数因子. 他们不加说明地定义了核无迹的对称性生成元$\,B_{\alpha}^{\sharp}$, 而要说明这一步是合理的就需要这些因子.} 而``$\operatorname{tr}$''中小写的$\,\mathrm{t}\,$是指对单粒子指标而不是二粒子指标求和. 为了使上式对几乎所有满足$\,p^{\prime}+q^{\prime}=p+q\,$的在壳\,4\,-动量都满足, 函数$\operatorname{tr}b_{\alpha}(p)/N(\sqrt{-p_{\mu}p^{\mu}})$ 对$\,p\,$必是线性%
\footnote{很容易看到常数项从方程(\ref{24.B.9})中被排除了, 这是因为存在粒子数不守恒的过程, 而这样的过程在满足集团分解原理的相对论量子理论中是不可避免的. 即使我们仅考虑二粒子过程并且不使用这个讨论, 方程(\ref{24.B.9})中的常数项最后也只相当于内部对称性在粒子态上的作用中有一个变化.}的:
\begin{equation}
\frac{\operatorname{tr}b_{\alpha}(p)}{N(\sqrt{-p_{\mu }p^{\mu}})}
=a_{\alpha}^{\mu}\,p_{\mu} \:,  \label{24.B.9}
\end{equation}
其中$\,a_{\alpha}^{\mu}\,$独立于$\,p\,$(以及已写出指标外的任何东西.) 通过减除掉动量算符的一个线性项, 我们可以定义新的对称性算符:
\begin{equation}
B_{\alpha}^{\sharp}\equiv B_{\alpha}-a_{\alpha}^{\mu}P_{\mu}\:,\label{24.B.10}
\end{equation}
根据方程(\ref{24.B.9}), 它在单粒子态上被表示成无迹矩阵
\begin{equation}
    \Bigl(b_{\alpha}^{\sharp}(p)\Bigr)_{n^{\prime}n}=\Bigl(b_{\alpha}(p)\Bigr)_{n^{\prime}n}
    -\frac{\operatorname{tr}b_{\alpha}(p)}{N(\sqrt{-p_{\mu}p^{\mu}})}\delta_{n^{\prime}n}\:.\label{24.B.11}
\end{equation}
因为$\,P_{\mu}\,$与$\,B_{\alpha}\,$对易, 单位矩阵与所有一切都对易, $B_{\alpha}^{\sharp}\,$的对易子与$\,B_{\alpha}\,$的对易子相同, 并且$\,b_{\alpha}^{\sharp}(p)\,$的对易子与$\,b_{\alpha}(p)\,$的对易子相同:
\begin{align}
    &[B_{\alpha}^{\sharp},B_{\beta}^{\sharp}]=\mi\sum_{\gamma}C_{\alpha\beta}^{\gamma}B_{\gamma}
    =\mi\sum_{\gamma}C_{\alpha\beta}^{\gamma}\,\Bigl[B_{\gamma}^{\sharp}+a_{\gamma}^{\mu}P_{\mu}\Bigr]\:,
    \label{24.B.12} \\
    &[b_{\alpha}^{\sharp}(p),b_{\beta}^{\sharp}(p)]=\mi\sum_{\gamma}C_{\alpha\beta}^{\gamma}b_{\gamma}(p)
    =\mi\sum_{\gamma}C_{\alpha\beta}^{\gamma}\,\Bigl[b_{\gamma}^{\sharp}(p)+a_{\gamma}^{\mu}p_{\mu}\Bigr]\:.
    \label{24.B.13}
\end{align}
另外, 根据有限矩阵$\,b_{\alpha}^{\sharp}\,$的对易子的迹为零\footnote{这是我们用到假设\,1\,的一个地方, 没有这个假设, 对易子的迹不一定为零. 另外, 这里关键的地方是我们处理的是对易关系而不是反对易关系, 对于反对易关系, 首先单位矩阵并不与其它矩阵反对易, 其次有限矩阵的反对易子的迹不一定为零.}这个性质, 方程(\ref{24.B.13})给出了$\,\sum_{\gamma}C^{\gamma}_{\alpha\beta}\,a_{\gamma}^{\mu}=0$, 在方程(\ref{24.B.12})中使用这个结果就表明$\,B_{\alpha}^{\sharp}\,$满足的对易关系与$\,B_{\alpha}\,$相同:
\begin{equation}
[B_{\alpha}^{\sharp},B_{\beta}^{\sharp}] =\mi\sum_{\gamma}C_{\alpha\beta}^{\gamma}B_{\gamma}^{\sharp}\:.  \label{24.B.14}
\end{equation}%
因为$\,B_{\alpha}^{\sharp}\,$是对称性算符, 散射振幅满足
\begin{equation}
b_{\alpha}^{\sharp}(p^{\prime},q^{\prime})\,S(p^{\prime},q^{\prime};p,q)
=S(p^{\prime},q^{\prime};p,q)\,b_{\alpha}^{\sharp}(p,q) \label{24.B.15}
\end{equation}%
其中$\,b_{\alpha}^{\sharp}\,$是$\,B_{\alpha}^{\sharp}\,$在二粒子态上的表示矩阵
\begin{equation}
\Bigl(b_{\alpha}^{\sharp}(p,q)\Bigr)_{m^{\prime}n^{\prime},mn}
=\Bigl(b_{\alpha}^{\sharp}(p)\Bigr)_{m^{\prime}m}\delta_{n^{\prime}n}
+\Bigl(b_{\alpha}^{\sharp}(q)\Bigr)_{n^{\prime}n}\delta_{m^{\prime}m} \label{24.B.16}
\end{equation}%
并且它满足的对易关系与$\,B_{\alpha}^{\sharp}\,$相同:
\begin{equation}
[b_{\alpha}^{\sharp}(p,q),b_{\beta}^{\sharp}(p,q)]=\mi\sum_{\gamma}C_{\alpha\beta}^{\gamma}
\,b_{\gamma}^{\sharp}(p,q)\:. \label{24.B.17}
\end{equation}%
用这些二粒子矩阵进行处理的优点是, 由于$\,S(p^{\prime},q^{\prime};p,q)\,$是非奇异矩阵, 由此可以得出, 如果对于某两个固定的$\,4\,$-动量$\,p\,$和$\,q\,$有$\,\sum_{\alpha}c^{\alpha}b_{\alpha}^{\sharp}(p,q)=0$, 那么对于几乎所有处在同一质壳上且满足$\,p^{\prime}+q^{\prime}=p+q\,$的$\,p^{\prime}\,$和$\,q^{\prime}\,$都有%
$\,\sum_{\alpha}c^{\alpha}b_{\alpha}^{\sharp}(p^{\prime},q^{\prime})=0$. 由于我们现在处理的是无迹矩阵, 这告诉我们
\begin{equation}
 \sum_{\alpha}c^{\alpha}\,b_{\alpha}^{\sharp}(p^{\prime})
=\sum_{\alpha}c^{\alpha}\,b_{\alpha}^{\sharp}(q^{\prime})=0 \:.  \label{24.B.18}
\end{equation}%
我们本希望从此得出, 对于{\kai{所有}}在壳\,4\,-动量$\,k\,$都有$\sum_{\alpha}c^{\alpha}b_{\alpha}^{\sharp}(k)=0$, 但到现在为止我们仅证明了, 对于几乎所有那些使得$\,q^{\prime}=p+q-p^{\prime}\,$和$\,p^{\prime}\,$在壳的$\,p^{\prime}\,$有%
$\sum_{\alpha}c^{\alpha}b_{\alpha}^{\sharp}(p^{\prime})=0$, (对$\,q^{\prime}\,$类似.) 为了跳出这个限制, 我们可以使用\,Coleman\,和\,Mandula\,的一个技巧, 注意到, 如果$\,\sum_{\alpha}c^{\alpha}b_{\alpha}^{\sharp}(p,q)=0$, 那么方程(\ref{24.B.18})加上方程(\ref{24.B.16})给出 \[
\sum_{\alpha}c^{\alpha}\,b_{\alpha}^{\sharp}(p,q^{\prime})=0 \:,
\]%
根据方程(\ref{24.B.15}), 这使得
\[
\sum_{\alpha}c^{\alpha}\,b_{\alpha}^{\sharp}(k,p+q^{\prime}-k)=0 \:,
\]%
因此, 对于几乎所有使得$\,p+q^{\prime}-k\,$和$\,k\,$都在壳的\,4\,-动量$\,k$, 我们有
\begin{equation}
\sum_{\alpha}c^{\alpha}\,b_{\alpha}^{\sharp}(k)=0  \:.  \label{24.B.19}
\end{equation}%
现在, $q^{\prime}\,$和$\,p+q-q^{\prime}\,$都在质壳上这个条件使得$\,q^{\prime}\,$中自由参量的个数还有两个, 这使得我们在$\,q^{\prime}\,$的选择上还有足够的自由度以至于$\,p+q^{\prime}-k\,$在壳这个条件仍然可以让%
我们自由地选择$\,\mathbf{k}$, 或者至少可以在动量空间的某个有限体积内可以随便选择$\,\mathbf{k}$. 通过将$\,\mathbf{p}\,$和$\,\mathbf{q}\,$取得充分大, 这个空间可以被调成我们希望的大小, 所以, 如果对某两个固定的在壳\,4\,-动量$\,p\,$和$\,q\,$有$\,\sum_{\alpha}c^{\alpha}b_{\alpha}^{\sharp}(k)=0$, 那么对于几乎所有在壳\,4\,-动量$\,k\,$都有$\,\sum_{\alpha}c^{\alpha}b_{\alpha}^{\sharp}(k)=0$. 这样一来, 如果对某个特定的\,4\,-动量$\,k_{0}\,$有$\,\sum_{\alpha}c^{\alpha}b_{\alpha}^{\sharp}(k_{0})\neq 0$, 那么对于\,4\,-动量为$\,k_{0}\,$和$\,k\,$的两个粒子散射散射成\,4\,-动量为$\,k^{\prime}\,$和%
$\,k^{\prime\prime}\,$这样的散射过程, 对于几乎所有$\,k$, $k^{\prime}\,$和$\,k^{\prime\prime}$, 它都会被$\,\sum_{\alpha}c^{\alpha}B_{\alpha}^{\sharp}\,$生成的对称性禁止, 而这与我们对散射振幅的解析性所做的假设矛盾. 我们由此得出, 如果对某两个固定的在壳\,4\,-动量$\,p\,$和$\,q\,$有$\,\sum_{\alpha}c^{\alpha}b_{\alpha}^{\sharp}(k)=0$, 那么对于{\kai{所有}}$\,k\,$都有$\,\sum_{\alpha}c^{\alpha}b_{\alpha}^{\sharp}(k)=0$, 这使得将$\,B_{\alpha}\,$变成$\,b_{\alpha}^{\sharp}\,$的映射是个同构.

由此马上可以得到一个推论, 由于独立矩阵$\,b_{\alpha}^{\sharp}(p,q)\,$的个数不能超过%
$\,N(\sqrt{-p_{\mu}p^{\mu}})N(\sqrt{-q_{\mu}q^{\mu}})$, 所以独立的对称性生成元$\,B_{\alpha}\,$最多只有有限个. 这也是\,Coleman\,和\,Mandula\,强调的, 在证明他们的定理时, 没有必要就对称性代数是有限维这一点再单独做一假设.

15.2\,节的定理进一步告诉我们, 对于有限厄米矩阵, 例如$\,p,q\,$固定的$\,b_{\alpha}^{\sharp}(p,q)$, 它的\,Lie\,代数至多是一个半单紧致\,Lie\,代数与数个$\,U(1)\,$Lie\,代数的直和. 我们已经看到这个\,Lie\,代数同构于对称性生成元为$\,B_{\alpha}^{\sharp}\,$的那个\,Lie\,代数, 所以$\,B_{\alpha}^{\sharp}\,$张开的至多也只是一个半单紧致\,Lie\,代数与数个$\,U(1)\,$Lie\,代数的直和.

我们先来分解\,$U(1)$\,Lie\,代数. 对于任意一对在壳动量$\,p\,$和$\,q$, 我们能找到一个\,Lorentz\,生成元$\,J$ 保持它们不变. (如果$\,p\,$和$\,q\,$类光且平行, 那么选择$\,J\,$使其生成围绕$\,\mathbf{p}\,$和$\,\mathbf{q}\,$的共同方向的旋转即可. 若不是这种情况, $p+q\,$将是类时的, 那么在$\,\mathbf{p}=-\mathbf{q}\,$的质心系中围绕$\,\mathbf{p}\,$和$\,\mathbf{q}\,$%
的共同方向的旋转保持$\,p,q\,$不变, 选择$\,J\,$使其生成这个旋转即可.) 我们可以选择二粒子态的基去对角化\,$J$, 使得
\begin{equation}
J\lvert pm,qn\rangle =\sigma (m,n)\lvert pm,qn\rangle \: .  \label{24.B.20}
\end{equation}%
现在, $P_{\mu}\,$与所有$\,B_{\alpha}^{\sharp}\,$对易, 而$\,[J,P_{\mu}]\,$又是$\,P_{\mu}\,$的线性组合, 所以$\,P_{\mu}\,$与所有$\,[J,B_{\alpha}^{\sharp}]\,$对易, 因而对称性生成元$\,[J,B_{\alpha}^{\sharp}]\,$必是$\,B_{\beta}\,$的线性组合, 根据定义, 这构成了与$\,P_{\mu}\,$对易的对称性生成元的一组完备集. 更确切些, 由于表示对易子的矩阵必然是无迹的, $[J,B_{\alpha}^{\sharp}]\,$必然是$\,B_{\beta}^{\sharp}\,$的线性组合. 然而, 在$\,B_{\beta}^{\sharp}\,$的代数中, 任何$\,U(1)\,$生成元$\,B_{i}^{\sharp}\,$(取成厄米的)必与所有$\,B_{\beta}^{\sharp}\,$对易, 因此也必须与$\,[J,B_{i}^{\sharp}]\,$对易:
\[
[B_{i}^{\sharp},[J,B_{i}^{\sharp}]]=0 \:.
\]%
在$\,J\,$是对角化的二粒子基上取这个二重对易子的期望值, 那么对于任意的$\,m\,$和$\,n$, 我们有
\begin{equation}
0=\sum_{m^{\prime },n^{\prime}}\Bigl(\sigma(m^{\prime},n^{\prime})-\sigma(m,n)\Bigr)
\biggl\lvert\Bigl(b_{\alpha}^{\sharp}(p,q)\Bigr)_{m^{\prime}n^{\prime},mn}\biggr\rvert^{2} \:.\label{24.B.21}
\end{equation}%
指标的取值范围是有限的, 所以如果存在任何$\,\sigma\,$使得存在$\,\sigma(m,n)=\sigma\,$的 $m\,$和$\,n$, $\sigma(m^{\prime},n^{\prime})\neq\sigma\,$的$\,m^{\prime}\,$和$\,n^{\prime}$, 并有$(b_{i}^{\sharp}(p,q))_{m^{\prime}n^{\prime},mn}\neq0$, 那么这样的$\,\sigma\,$必须会有最小的一个, 取这个最小的$\,\sigma$, 那么方程(\ref{24.B.21})的右边对于这个$\,m\,$和$\,n\,$必然是正定的, 与方程(\ref{24.B.21})矛盾. 我们由此得出, 对于所有使得$\,\sigma(m^{\prime},n^{\prime})\neq\sigma(m,n)\,$的$\,m,n,m^{\prime},n^{\prime}$, $(b_{i}^{\sharp}(p,q))_{m^{\prime}n^{\prime},mn}\,$必须为零. 由于$\,b_{i}^{\sharp}(p,q)\,$的代数同构于$\,B_{i}^{\sharp}\,$的代数, 这意味每个\,$U(1)\,$生成元$\,B_{i}^{\sharp}\,$与$\,J\,$对易. 由于我们可以将$\,p+q\,$选在任何类时方向上, 由此得出每个\,$U(1)$\,生成元$\,B_{i}^{\sharp}\,$与齐次\,Lorentz\,群的所有生成元$\,J^{\mu\nu}\,$对易.
它们与我们在\,2.5\,节称作``增速''的变换对易这一性质暗示了$\,(b_{i}^{\sharp}(p))_{n^{\prime}n}\,$与%
\,3\,-动量独立, 而它们与旋转对易这一性质暗示了$\,(b_{i}^{\sharp}(p))_{n^{\prime}n}\,$作为%
单位矩阵作用在自旋指标上, 所以这些生成元是普通内部对称性的生成元.

剩下的$\,B_{\alpha}^{\sharp}\,$生成了半单紧致\,Lie\,代数. 24.1\,节的讨论(比\,Coleman\,和\,Mandula\,给出的推导稍微明显一些)告诉我们, Lie\,代数的半单紧致部分的生成元与\,Lorentz\,变换对易, 而正如对\,$U(1)\,$生成元所证明的, 这意味着它们也是内部对称性的生成元. 因此我们证明了, 与$\,P_{\mu}\,$对易的生成元$\,B_{\alpha}\,$不是内部对称性的生成元就是$\,P_{\mu}\,$自身的线性组合.

接下来, 我们必须要考虑对称性生成元与动量算符不对易的可能性. 一般对称性的生成元$\,A_{\alpha}$ 在$\,4\,$-动量为$\,p\,$的单粒子态$\,\lvert p\,n\rangle\,$上的作用是
\begin{equation}
A_{\alpha }\,\lvert p,n\rangle = \sum_{n^{\prime}}\int \dif^{4}p^{\prime }\:
\Bigl(\mathscr{A}_{\alpha}(p^{\prime},p)\Bigr)_{n^{\prime}n}\,\lvert p^{\prime},n^{\prime}\rangle \:,  \label{24.B.22}
\end{equation}%
其中$\,n\,$和$\,n^{\prime}\,$依旧是标记自旋\,$z$\,-分量和粒子种类的离散指标. 当然, 除非$\,p\,$和$\,p^{\prime}\,$都在质壳上, 否则核$\,\mathscr{A}_{\alpha}(p^{\prime},p)\,$必须为零. 我们会先证明任何$\,p^{\prime}\neq p\,$的$\,\mathscr{A}_{\alpha}(p^{\prime},p)\,$为零.

对于这个目的, 注意到, 如果$\,A_{\alpha}\,$是对称性生成元, 那么
\begin{equation}
A_{\alpha}^{f}\equiv \int \dif^{4}x\,\exp(\mi \,P\cdot x)\,A_{\alpha }\,\exp(-\mi\,P\cdot x)\,f(x)  \label{24.B.23}
\end{equation}%
也是对称性生成元, 其中$\,P_{\mu}(x)\,$是\,4\,-动量算符, $f(x)\,$是一个我们可以任意选择的函数. 这个生成元作用在单粒子态上给出
\begin{equation}
A_{\alpha }^{f}\,\lvert p,n\rangle =\sum_{n^{\prime }}\int \dif^{4}p^{\prime }\, \tilde{f%
}(p^{\prime }-p)\Bigl( \mathscr{A}_{\alpha }(p^{\prime },p)\Bigr) _{n^{\prime
}n}\lvert p^{\prime },n^{\prime }\rangle \:,  \label{24.B.24}
\end{equation}%
其中$\,\tilde{f}\,$是\,Fourier\,变换
\begin{equation}
\tilde{f}(k)\equiv \int \dif^{4}x\,\exp (\mi x\cdot k)\,f(x) \:.  \label{24.B.25}
\end{equation}%
假定存在一对在壳\,4\,-动量$\,p\,$和$\,p+\Delta$, 其中$\,\Delta\neq 0$, 这对\,4\,-动量使得$\,\mathscr{A}(p+\Delta,p)\neq0$. 对于满足$\,p^{\prime}+q^{\prime}=p+q\,$的在壳\,4\,-动量$\,q$, $p^{\prime}\,$和$\,q^{\prime}$, $q+\Delta\,$或者$\,p^{\prime}+\Delta\,$或者$\,q^{\prime}+\Delta\,$一般不会在壳. 如果我们取$\,\tilde{f}(k)\,$仅在$\,\Delta\,$附近一块充分小的区域内不为零, 那么$\,A_{\alpha}^{f}\,$将会湮灭所有\,4\,-动量为$\,q,p^{\prime},q^{\prime}\,$的单粒子态, 但是它不会湮灭\,4\,-动量为$\,p\,$的单粒子态, 所以, 对于任何散射过程, 只要其中有粒子的动量从$\,p\,$和$\,q\,$变成$\,p^{\prime}\,$和$\,q^{\prime}$, 这样的散射过程就会被这个对称性禁止, 与假设\,2\,和\,3\,的``在几乎所有能标和所有角度处都存在散射''的结果相矛盾.

因为核$\,\mathscr{A}_{\alpha}(p^{\prime},p)\,$可能会包含正比于$\,\delta^{4}(p^{\prime}-p)\,$的导数的项%
以及正比于$\,\delta^{4}(p^{\prime}-p)\,$自身的项, 这一结果并不意味着所有对称性生成元$\,A_{\alpha}\,$都要与$\,P_{\mu}\,$对易. 为了处理这种情况, Coleman\,和\,Mandula 做了``丑陋的技术性假定'': 核$\,\mathscr{A}_{\alpha}(p^{\prime},p)\,$是{\kai{分布}}, 这意味着核中包含的$\,\delta^{4}(p^{\prime}-p)\,$的导数至多是有限的$\,D_{\alpha}\,$阶导数. 换句话说, 就是假定每个对称性生成元$\,A_{\alpha}\,$在单粒子态上的作用是求导运算$\,\partial/\partial p_{\mu}$ 的$\,D_{\alpha}\,$次多项式, 其中在这一点的矩阵系数允许依赖于动量和自旋. 为了使用上面对称性生成元与动量算符对易的结果, Coleman\,和\,Mandula\,考虑了动量算符对$\,A_{\alpha}\,$的$\,D_{\alpha}\,$重对易子
\begin{equation}
B_{\alpha }^{\mu _{1}\cdots \mu _{D_{\alpha }}}\equiv \lbrack P^{\mu_{1}},[P^{\mu _{2}},\cdots [P^{\mu_{D_{\alpha}}},A_{\alpha}]]\cdots] \: .  \label{24.B.26}
\end{equation}%
在\,4\,动量为$\,p^{\prime}\,$和$\,p\,$的态上, $B_{\alpha}^{\mu_{1}\cdots\mu_{D_{\alpha}}}\,$与$\,P^{\mu}\,$的对易子的矩阵元正比于\,$p^{\prime}-p\,$的%
$\,D_{\alpha}+1\,$次方, 再乘以动量导数的$\,D_{\alpha}\,$次多项式作用在$\,\delta^{4}(p^{\prime}-p)\,$上的结果, 因此为零. 由于生成元$\,B_{\alpha}^{\mu_{1}\cdots\mu_{D_{\alpha}}}\,$与动量算符对易, 根据上面获得的结果, 它们在单粒子态上的作用是如下形式的矩阵
\begin{equation}
b_{\alpha }^{\mu _{1}\cdots \mu _{D_{\alpha }}}(p)=b_{\alpha}^{\sharp\mu_{1}\cdots\mu_{D_{\alpha}}}
+a_{\alpha}^{\mu\mu_{1}\cdots\mu_{D_{\alpha}}}p_{\mu}\,1\:,\label{24.B.27}
\end{equation}%
其中$\,b_{\alpha}^{\sharp\mu_{1}\cdots\mu_{D_{\alpha}}}\,$是与动量无关的无迹厄米矩阵, 它们生成了通常的内部对称性代数, $a_{\alpha}^{\mu\mu_{1}\cdots\mu_{D_{\alpha}}}\,$是与动量无关的数值常数, $b_{\alpha}^{\sharp\mu_{1}\cdots\mu_{D_{\alpha}}}\,$和$\,a_{\alpha}^{\mu\mu_{1}\cdots\mu_{D_{\alpha}}}\,$%
关于指标$\,\mu_{1},\cdots\mu_{D_{\alpha}}\,$均是对称的. 另外, 尽管$\,A_{\alpha}\,$不一定与$\,P_{\mu}\,$对易, 但是它们不能使单粒子态离壳, 而假设\,1\,又要求指标质量平方算符$\,-P_{\mu}P^{\mu}\,$仅有离散本征值, 所以$\,A_{\alpha}\,$必须与$\,-P_{\mu}P^{\mu}\,$对易. 特别地, 对于$\,D\geq1\,$可以得出
\[
0=[P^{\mu _{1}}P_{\mu _{1}},[P^{\mu _{2}},\cdots \lbrack P^{\mu _{D_{\alpha
}}},A_{\alpha }]]\cdots ]=2P_{\mu _{1}}B_{\alpha }^{\mu _{1}\cdots\mu_{D_{\alpha}}} \:,
\]%
这使得
\begin{equation}
0=p_{\mu _{1}}b_{\alpha }^{\mu _{1}\cdots \mu _{D_{\alpha }}}(p)\:. \label{24.B.28}
\end{equation}%
只要理论包含有质量粒子, 对于处在任何类时方向的$\,p$, 这都必须满足, 所以对于$\,D_{\alpha}\geq1$
\begin{equation}
b_{\alpha }^{\sharp \mu _{1}\cdots \mu _{D_{\alpha }}}=0 \:,  \label{24.B.29}
\end{equation}%
且
\begin{equation}
a_{\alpha}^{\mu\mu_{1}\cdots\mu_{D_{\alpha}}}=-a_{\alpha}^{\mu_{1}\mu\cdots\mu_{D_{\alpha}}} \:.  \label{24.B.30}
\end{equation}%
但是当$\,D_{\alpha}\geq2\,$时, 加上$\,a_{\alpha}^{\mu\mu_{1}\cdots\mu_{D_{\alpha}}}\,$关于指标$\,\mu_{1},\cdots,\mu_{D_{\alpha}}\,$的对称性, 方程(\ref{24.B.30})将要求$\,a_{\alpha}^{\mu\mu_{1}\cdots\mu_{D_{\alpha}}}\\=0$. (这时$\,a_{\alpha}^{\mu\mu_{1}\mu_{2}\cdots\mu_{D_{\alpha}}}=
a_{\alpha}^{\mu\mu_{2}\mu_{1}\cdots\mu_{D_{\alpha}}}=-a_{\alpha}^{\mu_{2}\mu\mu_{1}\cdots\mu_{D_{\alpha}}}
=-a_{\alpha}^{\mu_{2}\mu_{1}\mu\cdots\mu_{D_{\alpha}}}=a_{\alpha}^{\mu_{1}\mu_{2}\mu\cdots\mu_{D_{\alpha}}}
=a_{\alpha}^{\mu_{1}\mu\mu_{2}\cdots\mu_{D_{\alpha}}}\\=-a_{\alpha}^{\mu\mu_{1}\mu_{2}\cdots\mu_{D_{\alpha}}}$.)
因此我们最多还有两种非零的对称性生成元: $\,D_{\alpha}=0\,$的, 这时生成元$\,A_{\alpha}\,$与 $P_{\mu}\,$对易, 因此它不是内部对称性的生成元就是$\,P_{\mu}\,$的某个线性组合; 以及$\,D_{\alpha}=1\,$的, 在这种情况下,
\begin{equation}
[P^{\nu },A_{\alpha}]=a_{\alpha}^{\mu\nu}P_{\mu}\:, \label{24.B.31}
\end{equation}%
其中$\,a_{\alpha}^{\mu\nu}\,$是一些关于$\,\mu\,$和$\,\nu\,$反对称的数值常数. 方程(\ref{24.B.31})要求
\begin{equation}
A_{\alpha }=-\tfrac{1}{2}\mi a_{\alpha}^{\mu\nu}J_{\mu\nu}+B_{\alpha}\:,  \label{24.B.32}
\end{equation}%
其中$\,J_{\mu\nu}\,$是固有\,Lorentz\,变换的生成元, 根据方程(\textcolor{foo}{2.4.13})它满足$\,[P^{\nu},J^{\rho\sigma}]
=-\mi\eta^{\nu\rho}P^{\sigma}+\mi\eta^{\nu\sigma}P^{\rho}$, $B_{\alpha}\,$与$\,P_{\mu}\,$对易. 由于$\,A_{\alpha}\,$和$\,J_{\mu\nu}\,$是对称性生成元, 所以$\,B_{\alpha}\,$也是, 因此它必须是内部对称性生成元和(或)$\,P_{\mu}\,$分量的线性组合. 方程(\ref{24.B.32})因此补完了\,Coleman-Mandula\,定理的证明.

\subsection*{* * *}

在只有无质量粒子的理论中, 方程(\ref{24.B.28})不一定只能给出方程(\ref{24.B.30}); 由于$\,p_{\mu}p^{\mu}=0$, 我们还可以有
\begin{equation}
a_{\alpha}^{\mu \mu_{1}\cdots \mu_{D_{\alpha}}}+a_{\alpha}^{\mu_{1}\mu
\cdots \mu_{D_{\alpha}}}\propto\eta^{\mu\mu_{1}} \:. \label{24.B.33}
\end{equation}%
在这一情况下, 对称性代数由内部对称性和共形群的代数构成, 其中共形群由生成元$\,K_{\mu}\,$和$\,D\,$以及\,Poincar\'{e}\,群的生成元$\,J^{\mu\nu}\,$和$\,P^{\mu}\,$张开. 对易关系是
\begin{align}
&[P^{\mu},D] =\mi P^{\mu} \:, \qquad [K^{\mu},D]=-\mi K^{\mu} \:,  \nonumber \\
&[P^{\mu},K^{\nu}] =2\mi\eta^{\mu\nu}D + 2\mi J^{\mu\nu} \:,
 \qquad [K^{\mu},K^{\nu}]=0 \:, \label{24.B.34} \\
&[J^{\rho\sigma},K^{\mu}] =\mi\eta^{\mu\rho}K^{\sigma}-\mi\eta^{\mu\sigma}K^{\rho} \:, \qquad
[J^{\rho\sigma},D]=0 \:,  \nonumber
\end{align}%
和\,Poincar\'{e}\,代数的对易关系(\textcolor{foo}{2.4.12})---(\textcolor{foo}{2.4.14})
\begin{align}
\mi[J^{\mu\nu},J^{\rho\sigma}] &=\eta^{\nu \rho }J^{\mu \sigma }-\eta
^{\mu \rho }J^{\nu \sigma }-\eta ^{\sigma \mu }J^{\rho \nu }+\eta ^{\sigma
\nu }J^{\rho \mu } \:,  \nonumber \\
\mi[P^{\mu },J^{\rho \sigma }] &=\eta ^{\mu \rho }P^{\sigma }-\eta ^{\mu
\sigma }P^{\rho }\:,  \label{24.B.35} \\
[P^{\mu },P^{\rho }]& =0 \:.  \nonumber
\end{align}%
无穷小群元
\begin{equation}
U(1+\omega ,\epsilon ,\lambda ,\rho )=1+(\mi/2)J_{\mu \nu }\omega ^{\mu \nu
}+\mi P_{\mu }\epsilon ^{\mu }+\mi\lambda D+\mi K_{\mu }\rho ^{\mu }
\label{24.B.36}
\end{equation}%
诱导出了无穷小时空变换
\begin{equation}
x^{\mu }\to x^{\mu }+\omega ^{\mu \nu }x_{\nu }+\epsilon ^{\mu
}+\lambda x^{\mu }+\rho ^{\mu }x^{\nu }x_{\nu }-2x^{\mu }\rho ^{\nu }x_{\nu }\:.  \label{24.B.37}
\end{equation}
这些是最一般的保持光锥不变的无穷小时空变换.



\section*{习题}
\noindent 1. 在所有粒子都是无质量的情况下,对于\,Coleman-Mandula\,定理的假设所允许的最一般对称性代数, 证明组成这个代数的不是内部对称性生成元加上\,Poincar\'{e}\,代数%
就是内部对称性生成元加上共形代数(\ref{24.B.34}), (\ref{24.B.35}).\\

\noindent 2. 证明\,Gervais-Sakita\,作用量(\ref{24.2.5})在世界面超对称变换(\ref{24.2.7})下不变.\\

\noindent 3. 计算\,Wess-Zumino\,拉格朗日密度(\ref{24.2.9})在时空超对称变换(\ref{24.2.8})下的变化.

%++++++++++++++++++参考文献+++++++++
\renewcommand{\sectionmark}[1]{\markright{ #1}{}}
\renewcommand{\bibname}{参考文献}

\begin{thebibliography}{99}
\bibitem{1} B. Sakita, \textit{Phys. Rev.} {\bf{136}}, B 1756 (1964); F. Gursey and L. A. Radicati, \textit{Phys. Rev. Lett.} {\bf{13}}, 173 (1964); A. Pais, \textit{Phys. Rev. Lett.} {\bf{13}}, 175 (1964); F. Gursey, A. Pais and L. A. Radicati, {\textit{Phys. Rev. Lett.}} {\bf{13}}, 299 (1964). 这些文章翻印于\,\textit{Symmetry Groups in Nuclear and Particle Physics}, F. J. Dyson, ed. (W. A. Benjamin, New York, 1966), 附带还有\,Dyson\,关于这一主题的一套课堂笔记.
\bibitem{2} E. P. Wigner, {\textit{Phys. Rev.}} {\bf{51}}, 106 (1937). 翻印于\,\textit{Symmetry Groups in Nuclear and Particle Physics}, 参考文献[1].
\bibitem{3} A. Salam, R. Delbourgo, and J. Strathdee, {\textit{Proc. Roy. Soc. (London)}} {\bf{A 284}}, 146 (1965); M. A. Beg and A. Pais, {\textit{Phys. Rev. Lett.}} {\bf{14}}, 267 (1965); B. Sakita and K. C. Wali, {\textit{Phys. Rev.}} {\bf{139}}, B 1355 (1965). 这些文章翻印于\,\textit{Symmetry Groups in Nuclear and Particle Physics}, 参考文献[1].
\bibitem{4} W. D. McGlinn, {\textit{Phys. Rev. Lett.}} {\bf{12}}, 467 (1964); O. W. Greenberg, {\textit{Phys. Rev.}} {\bf{135}}, B 1447 (1964); L. Michel, {\textit{Phys. Rev.}} {\bf{137}}, B 405 (1964); L. Michel and B. Sakita, {\textit{Ann. Inst. Henri-Poincar\'{e}}} {\bf{2}}, 167 (1965); M. A. B. Beg and A. Pais, {\textit{Phys. Rev. Lett.}} {\bf{14}}, 509, 577 (1965); S. Coleman, {\textit{Phys. Rev.}} {\bf{138}}, B 1262 (1965); S. Weinberg, {\textit{Phys. Rev.}} {\bf{139}}, B 597 (1965); L. O'Raifeartaigh, {\textit{Phys. Rev.}} {\bf{139}}, B 1052 (1065). 这些文章翻印于\,\textit{Symmetry Groups in Nuclear and Particle Physics}, 参考文献[1].
\bibitem{5} S. Coleman and J. Mandula, {\textit{Phys. Rev.}} {\bf{159}}, 1251 (1967).
\bibitem{6} 关于原始文献的简介和引用, 参看\,M. B. Green, J. H. Schwarz, and E. Witten, {\textit{Superstring Theory}} (Cambridge University Press, Cambridge, 1987); J. Polchinski, {\textit{String Theory}} (Cambridge University Press, Cambridge, 1998).
\bibitem{7} P. Ramond, {\textit{Phys. Rev.}} {\bf{D3}}, 2415 (1971). 这篇文章翻印于\,{\textit{Superstrings --- The First 15 Years of Superstring Theory,}} J. H. Schwarz, ed. (World Scientific, Singapore, 1985).
\bibitem{8} A. Neveu and J. H. Schwarz, {\textit{Nucl. Phys.}} {\bf{B31}}, 86 (1971); {\textit{Phys. Rev.}} {\bf{D4}}, 1109 (1971). 这些文章翻印于\,{\textit{Superstrings --- The First 15 Years of Superstring Theory,}} 参考文献[7]. 另见\,Y. Aharonov, A. Casher, and L. Susskind, {\textit{Phys. Rev.}} {\bf{D5}}, 988 (1972).
\bibitem{9} J.-L. Gervais and B. Sakita, {\textit{Nucl. Phys.}} {\bf{B34}}, 632 (1971). 这篇文章翻印于\,{\textit{Superstrings --- The First 15 Years of Superstring Theory,}} 参考文献[7].
\bibitem{10} J. Wess and B. Zumino, {\textit{Nucl. Phys.}} {\bf{B70}}, 39 (1974). 这篇文章翻印于\,{\textit{Supersymmetry}}, S. Ferrara, ed. (North Holland/World Scientific, Amsterdam/Singapore, 1987).
\bibitem{11} J. Wess and B. Zumino, {\textit{Phys. Lett.}} {\bf{49B}}, 52 (1974). 这篇文章翻印于\,{\textit{Supersymmetry}}, 参考文献[10].
\bibitem[11a]{11a} F. Gliozzi, J. Scherk, and D. Olive, {\textit{Nucl. Phys.}} {\bf{B122}}, 253 (1977).
\bibitem{12} Yu. A. Gol'fand and E. P. Likhtman, {\textit{JETP Letters}} {\bf{13}}, 323 (1971). 这篇文章翻印于\,{\textit{Supersymmetry}}, 参考文献[10].
\bibitem{13} D. V. Volkov and V. P. Akulov, {\textit{Phys. Lett.}} {\bf{46B}}, 109 (1973). 这篇文章翻印于\,{\textit{Supersymmetry}}, 参考文献[10].
\bibitem{14} M. Gell-Mann, Cal. Tech. Synchotron Laboratory Report CTSL-20 (1961), 未发表. \,Gell-Mann\,和\,Y. Ne'eman\,在一些关于$\,SU(3)\,$对称性的文章中加入了这篇文章, {\textit{The Eightfold Way}} (Benjamin, New York, 1964).
\bibitem{15} R. Haag, J. T. Lopuszanski, and M. Sohnius, {\textit{Nucl. Phys.}} {\bf{B88}}, 257 (1975). 这篇文章翻印于\,{\textit{Supersymmetry}}, 参考文献[10].
\end{thebibliography}
