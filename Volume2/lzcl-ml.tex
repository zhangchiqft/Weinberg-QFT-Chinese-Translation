
\chapter*{目\qquad 录}
 \markboth{目\qquad 录}{目\qquad 录}      %%前双后单书眉
\thispagestyle{empty}
\setcounter{page}{1}

 \contentsline{chapter}{第一卷序言}{i}
 \contentsline{chapter}{符号约定}{i}
 \contentsline{chapter}{第1章\quad 历史介绍}{1}
      \contentsline{section}{\numberline {1.1}相对论波动力学}{2}
      \contentsline{section}{\numberline {1.2}量子场论的诞生}{13}
      \contentsline{section}{\numberline {1.3}无限大的问题}{27}
      \contentsline{section}{参考书目}{33}
\contentsline {section}{参考文献}{35}
 \contentsline{chapter}{第2章\quad 相对论量子力学}{45}
      \contentsline{section}{\numberline {2.1}量子力学}{45}
      \contentsline{section}{\numberline {2.2}对称性}{46}
      \contentsline{section}{\numberline {2.3}量子\,Lorentz\,变换}{51}
      \contentsline{section}{\numberline {2.4}Poincar\'{e}\,代数}{54}
      \contentsline{section}{\numberline {2.5}单粒子态}{58}
      \contentsline{section}{\numberline {2.6}空间反演和时间反演}{69}
      \contentsline{section}{\numberline {2.7}投影表示}{76}
      \contentsline{section}{附录A\quad 对称表示定理}{84}
      \contentsline{section}{附录B\quad 群算符和同伦类}{89}
      \contentsline{section}{附录C\quad 反演和简并多重态}{93}
      \contentsline{section}{习题}{97}
\contentsline {section}{参考文献}{98}
 \contentsline{chapter}{第3章\quad 散射理论}{101}
      \contentsline{section}{\numberline {3.1}``入''态和``出''态}{101}
      \contentsline{section}{\numberline {3.2}$S$-矩阵}{106}
      \contentsline{section}{\numberline {3.3}$S$-矩阵的对称性}{109}
      \contentsline{section}{\numberline {3.4}速率与截面}{125}
      \contentsline{section}{\numberline {3.5}微扰论}{132}
      \contentsline{section}{\numberline {3.6}幺正性的影响}{137}
      \contentsline{section}{\numberline {3.7}分波展开}{141}
      \contentsline{section}{\numberline {3.8}共振}{148}
      \contentsline{section}{习题}{153}
\contentsline {section}{参考文献}{154}
 \contentsline{chapter}{第4章\quad 集团分解原理}{157}
      \contentsline{section}{\numberline {4.1}玻色子与费米子}{158}
      \contentsline{section}{\numberline {4.2}产生和湮没算符}{160}
      \contentsline{section}{\numberline {4.3}集团分解和连通振幅}{165}
      \contentsline{section}{\numberline {4.4}相互作用的结构}{169}
      \contentsline{section}{习题}{174}
\contentsline {section}{参考文献}{175}
 \contentsline{chapter}{第5章\quad 量子场与反粒子}{177}
      \contentsline{section}{\numberline {5.1}自由场}{177}
      \contentsline{section}{\numberline {5.2}因果标量场}{186}
      \contentsline{section}{\numberline {5.3}因果矢量场}{191}
      \contentsline{section}{\numberline {5.4}Dirac\,形式体系}{197}
      \contentsline{section}{\numberline {5.5}因果\,Dirac\,场}{203}
      \contentsline{section}{\numberline {5.6}齐次Lorentz群的一般不可约表示}{212}
      \contentsline{section}{\numberline {5.7}一般因果场}{215}
      \contentsline{section}{\numberline {5.8}\textsf {CPT}\,定理}{227}
      \contentsline{section}{\numberline {5.9}无质量粒子场}{228}
      \contentsline{section}{习题}{236}
\contentsline {section}{参考文献}{237}
 \contentsline{chapter}{第6章\quad Feynman\,规则}{241}
      \contentsline{section}{\numberline {6.1}规则的推导}{241}
      \contentsline{section}{\numberline {6.2}传播子的计算}{253}
      \contentsline{section}{\numberline {6.3}动量空间规则}{259}
      \contentsline{section}{\numberline {6.4}离质量壳}{265}
      \contentsline{section}{习题}{268}
\contentsline {section}{参考文献}{269}
 \contentsline{chapter}{第7章\quad 正则体系}{271}
      \contentsline{section}{\numberline {7.1}正则变量}{272}
      \contentsline{section}{\numberline {7.2}拉格朗日体系}{277}
      \contentsline{section}{\numberline {7.3}整体对称性}{284}
      \contentsline{section}{\numberline {7.4}Lorentz\,不变性}{291}
      \contentsline{section}{\numberline {7.5}过渡到相互作用绘景\,: 例子}{295}
      \contentsline{section}{\numberline {7.6}约束与\,Dirac\,括号}{302}
      \contentsline{section}{\numberline {7.7}场重定义与冗余耦合}{307}
      \contentsline{section}{附录\quad 从正则对易子到Dirac 括号}{309}
      \contentsline{section}{习题}{313}
\contentsline {section}{参考文献}{313}
 \contentsline{chapter}{第8章\quad 电动力学}{315}
      \contentsline{section}{\numberline {8.1}规范不变性}{315}
      \contentsline{section}{\numberline {8.2}约束与规范条件}{319}
      \contentsline{section}{\numberline {8.3}Coulomb\,规范下的量子化}{322}
      \contentsline{section}{\numberline {8.4}相互作用绘景中的电动力学}{325}
      \contentsline{section}{\numberline {8.5}光子传播子}{328}
      \contentsline{section}{\numberline {8.6}旋量电动力学的\,Feynman\,规则}{330}
      \contentsline{section}{\numberline {8.7}Compton\,散射}{336}
      \contentsline{section}{\numberline {8.8}推广: \textit {p}\,-形式规范场}{343}
      \contentsline{section}{附录\quad 迹}{346}
      \contentsline{section}{习题}{348}
\contentsline {section}{参考文献}{348}
 \contentsline{chapter}{第9章\quad 路径积分方法}{351}
      \contentsline{section}{\numberline {9.1}普遍的路径积分公式}{352}
      \contentsline{section}{\numberline {9.2}过渡到$S$-矩阵}{359}
      \contentsline{section}{\numberline {9.3}路径积分公式的拉格朗日版本}{363}
      \contentsline{section}{\numberline {9.4}Feynman\,规则的路径积分推导}{368}
      \contentsline{section}{\numberline {9.5}费米子的路径积分}{372}
      \contentsline{section}{\numberline {9.6}量子电动力学的路径积分表述}{386}
      \contentsline{section}{\numberline {9.7}各种统计}{390}
      \contentsline{section}{附录A\quad 高斯多重积分}{393}
      \contentsline{section}{习题}{395}
\contentsline {section}{参考文献}{396}
 \contentsline{chapter}{第10章\quad 非微扰方法}{399}
      \contentsline{section}{\numberline {10.1}对称性}{399}
      \contentsline{section}{\numberline {10.2}极点学}{402}
      \contentsline{section}{\numberline {10.3}场重正化和质量重正化}{409}
      \contentsline{section}{\numberline {10.4}重正化荷与\,Ward\,恒等式}{415}
      \contentsline{section}{\numberline {10.5}规范不变性}{419}
      \contentsline{section}{\numberline {10.6}电磁形状因子与磁矩}{423}
      \contentsline{section}{\numberline {10.7}K\"{a}llen-Lehmann表示}{428}
      \contentsline{section}{\numberline {10.8}色散关系}{433}
      \contentsline{section}{习题}{439}
\contentsline {section}{参考文献}{440}
 \contentsline{chapter}{第11章\quad 量子电动力学中的单圈辐射修正}{443}
      \contentsline{section}{\numberline {11.1}抵消项}{443}
      \contentsline{section}{\numberline {11.2}真空极化}{444}
 \contentsline{subsection}{\numberline {1.}* * *}{454}
      \contentsline{section}{\numberline {11.3}反常磁矩与电荷半径}{454}
      \contentsline{section}{\numberline {11.4}电子自能}{462}
      \contentsline{section}{附录A\quad 各种积分}{465}
      \contentsline{section}{习题}{466}
\contentsline {section}{参考文献}{466}
 \contentsline{chapter}{第12章\quad 重正化的一般理论}{469}
      \contentsline{section}{\numberline {12.1}发散度}{470}
      \contentsline{section}{\numberline {12.2}发散的消除}{474}
      \contentsline{section}{\numberline {12.3}可重整性是必要的吗?}{484}
      \contentsline{section}{\numberline {12.4}浮动截断}{491}
      \contentsline{section}{\numberline {12.5}偶然对称性}{494}
      \contentsline{section}{习题}{496}
\contentsline {section}{参考文献}{497}
 \contentsline{chapter}{第13章\quad 红外效应}{499}
      \contentsline{section}{\numberline {13.1}软光子振幅}{499}
      \contentsline{section}{\numberline {13.2}虚软光子}{504}
      \contentsline{section}{\numberline {13.3}实软光子; 发散的抵消}{508}
      \contentsline{section}{\numberline {13.4}一般的红外发散}{511}
      \contentsline{section}{\numberline {13.5}软光子散射}{515}
      \contentsline{section}{\numberline {13.6}外场近似}{518}
      \contentsline{section}{习题}{523}
\contentsline {section}{参考文献}{524}
 \contentsline{chapter}{第14章\quad 外场中的束缚态}{525}
      \contentsline{section}{\numberline {14.1}Dirac\,方程}{526}
      \contentsline{section}{\numberline {14.2}外场中的辐射修正}{533}
      \contentsline{section}{\numberline {14.3}轻原子中的\,Lamb\,移动}{538}
      \contentsline{section}{习题}{552}
\contentsline {section}{参考文献}{553}
 \contentsline{part}{人名索引}{555}
 \contentsline{part}{主题索引}{561}
